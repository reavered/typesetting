It has often been said that, whatever the inadequacy of Bolshevik organization in other fields, in art and in education at least they have made great progress.

To take first of all art: it is true that they began by recognizing, as perhaps no other revolutionary government would, the importance and spontaneity of the artistic impulse, and therefore while they controlled or destroyed the counter-revolutionary in all other social activities, they allowed the artist, whatever his political creed, complete freedom to continue his work. Moreover, as regards clothing and rations they treated him especially well. This, and the care devoted to the upkeep of churches, public monuments, and museums, are well-known facts, to which there has already been ample testimony.

The preservation of the old artistic community practically intact was the more remarkable in view of the pronounced sympathy of most of them with the old régime. The theory, however, was that art and politics belonged to two separate realms; but great honour would of course be the portion of those artists who would be inspired by the revolution.

Three years' experience, however, have proved the falsity of this doctrine and led to a divorce between art and popular feeling which a sensitive observer cannot fail to remark. It is glaringly apparent in the hitherto most vital of all Russian arts, the theatre. The artists have continued to perform the old classics in tragedy or comedy, and the old-style operette. The theatre programmes have remained the same for the last two years, and, but for the higher standard of artistic performance, might belong to the theatres of Paris or London. As one sits in the theatre, one is so acutely conscious of the discrepancy between the daily life of the audience and that depicted in the play that the latter seems utterly dead and meaningless. To some of the more fiery Communists it appears that a mistake has been made. They complain that \emph{bourgeois} art is being preserved long after its time, they accuse the artists of showing contempt for their public, of being as untouched by the revolutionary mood as an elderly \emph{bourgeoise} bewailing the loss of her personal comfort; they would like to see only the revolutionary mood embodied in art, and to achieve this would make a clean sweep, enforcing the writing and performance of nothing but revolutionary plays and the painting of revolutionary pictures. Nor can it be argued that they are wrong as to the facts: it is plain that the preservation of the old artistic tradition has served very little purpose; but on the other hand it is equally plain that an artist cannot be drilled like a military recruit. There is, fortunately, no sign that these tactics will be directly adopted, but in an indirect fashion they are already being applied. An artist is not to blame if his temperament leads him to draw cartoons of leading Bolsheviks, or satirize the various comical aspects---and they are many---of the Soviet régime. To force such a man, however, to turn his talent only against Denikin, Yudenitch and Kolchak, or the leaders of the Entente, is momentarily good for Communism, but it is discouraging to the artist, and may prove in the long run bad for art, and possibly for Communism also. It is plain from the religious nature of Communism in Russia, that such controlling of the impulse to artistic creation is inevitable, and that propaganda art alone can flourish in such an atmosphere. For example, no poetry or literature that is not orthodox will reach the printing press. It is so easy to make the excuse of lack of paper and the urgent need for manifestoes. Thus there may well come to be a repetition of the attitude of the mediæval Church to the sagas and legends of the people, except that, in this case, it is the folk tales which will be preserved, and the more sensitive and civilized products banned. The only poet who seems to be much spoken of at present in Russia is one who writes rough popular songs. There are revolutionary odes, but one may hazard a guess that they resemble our patriotic war poetry.

I said that this state of affairs may in the long run be bad for art, but the contrary may equally well prove to be the truth. It is of course discouraging and paralysing to the old-style artist, and it is death to the old individual art which depended on subtlety and oddity of temperament, and arose very largely from the complicated psychology of the idle. There it stands, this old art, the purest monument to the nullity of the art-for-art's-sake doctrine, like a rich exotic plant of exquisite beauty, still apparently in its glory, till one perceives that the roots are cut, and that leaf by leaf it is gradually fading away.

But, unlike the Puritans in this respect, the Bolsheviks have not sought to dig up the roots, and there are signs that the paralysis is merely temporary. Moreover, individual art is not the only form, and in particular the plastic arts have shown that they can live by mass action, and flourish under an intolerant faith. Communist artists of the future may erect public buildings surpassing in beauty the mediæval churches, they may paint frescoes, organize pageants, make Homeric songs about their heroes. Communist art will begin, and is beginning now, in the propaganda pictures, and stories such as those designed for peasants and children. There is, for instance, a kind of Rake's Progress or "How she became a Communist," in which the Entente leaders make a sorry and grotesque appearance. Lenin and Trotsky already figure in woodcuts as Moses and Aaron, deliverers of their people, while the mother and child who illustrate the statistics of the maternity exhibition have the grace and beauty of mediæval madonnas. Russia is only now emerging from the middle ages, and the Church tradition in painting is passing with incredible smoothness into the service of Communist doctrine. These pictures have, too, an oriental flavour: there are brown Madonnas in the Russian churches, and such an one illustrates the statistics of infant mortality in India, while the Russian mother, broad-footed, in gay petticoat and kerchief, sits in a starry meadow suckling her baby from a very ample white breast. I think that this movement towards the Church tradition may be unconscious and instinctive, and would perhaps be deplored by many Communists, for whom grandiose bad Rodin statuary and the crudity of cubism better express what they mean by revolution. But this revolution is Russian and not French, and its art, if all goes well, should inevitably bear the popular Russian stamp. It is would-be primitive and popular art that is vulgar. Such at least is the reflection engendered by an inspection of Russian peasant work as compared with the spirit of \emph{Children's Tales}.

The Russian peasant's artistic impulse is no legend. Besides the carving and embroidery which speak eloquently to peasant skill, one observes many instances in daily life. He will climb down, when his slowly-moving train stops by the wayside, to gather branches and flowers with which he will decorate the railway carriage both inside and out, he will work willingly at any task which has beauty for its object, and was all too prone under the old régime to waste his time and his employer's material in fashioning small metal or wooden objects with his hands.

If the \emph{bourgeois} tradition then will not serve, there is a popular tradition which is still live and passionate and which may perhaps persist. Unhappily it has a formidable enemy in the organization and development of industry, which is far more dangerous to art than Communist doctrine. Indeed, industry in its early stages seems everywhere doomed to be the enemy of beauty and instinctive life. One might hope that this would not prove to be so in Russia, the first Socialist State, as yet unindustrial, able to draw on the industrial experience of the whole world, were it not that one discovers with a certain misgiving in the Bolshevik leaders the rasping arid temperament of those to whom the industrial machine is an end in itself, and, in addition, reflects that these industrially minded men have as yet no practical experience, nor do there exist men of goodwill to help them. It does not seem reasonable to hope that Russia can pass through the period of industrialization without a good deal of mismanagement, involving waste resulting in too long hours, child labour and other evils with which the West is all too familiar. What the Bolsheviks would not therefore willingly do to art, the Juggernaut which they are bent on setting in motion may accomplish for them.

The next generation in Russia will have to consist of practical hard-working men, the old-style artists will die off and successors will not readily arise. A State which is struggling with economic difficulties is bound to be slow to admit an artistic vocation, since this involves exemption from practical work. Moreover the majority of minds always turn instinctively to the real need of the moment. A man therefore who is adapted by talent and temperament to becoming an opera singer, will under the pressure of Communist enthusiasm and Government encouragement turn his attention to economics. (I am here quoting an actual instance.) The whole Russian people at this stage in their development strike one as being forced by the logic of their situation to make a similar choice.

It may be all to the good that there should be fewer professional artists, since some of the finest work has been done by men and groups of men to whom artistic expression was only a pastime. They were not hampered by the solemnity and reverence for art which too often destroy the spontaneity of the professional. Indeed a revival of this attitude to art is one of the good results which may be hoped for from a Communist revolution in a more advanced industrial community. There the problem of education will be to stimulate the creative impulses towards art and science so that men may know how to employ their leisure hours. Work in the factory can never be made to provide an adequate outlet. The only hope, if men are to remain human beings under industrialism, is to reduce hours to the minimum. But this is only possible when production and organization are highly efficient, which will not be the case for a long time in Russia. Hence not only does it appear that the number of artists will grow less, but that the number of people undamaged in their artistic impulses and on that account able to create or appreciate as amateurs is likely to be deplorably small. It is in this damaging effect of industry on human instinct that the immediate danger to art in Russia lies.

The effect of industry on the crafts is quite obvious. A craftsman who is accustomed to work with his hands, following the tradition developed by his ancestors, is useless when brought face to face with a machine. And the man who can handle the machine will only be concerned with quantity and utility in the first instance. Only gradually do the claims of beauty come to be recognized. Compare the modern motor car with the first of its species, or even, since the same law seems to operate in nature, the prehistoric animal with its modern descendant. The same relation exists between them as between man and the ape, or the horse and the hipparion. The movement of life seems to be towards ever greater delicacy and complexity, and man carries it forward in the articles that he makes and the society that he develops. Industry is a new tool, difficult to handle, but it will produce just as beautiful objects as did the mediæval builder and craftsman, though not until it has been in being for a long time and belongs to tradition.

One may expect, therefore, that while the crafts in Russia will lose in artistic value, the drama, sculpture and painting and all those arts which have nothing to do with the machine and depend entirely upon mental and spiritual inspiration will receive an impetus from the Communist faith. Whether the flowering period will be long or short depends partly on the political situation, but chiefly on the rapidity of industrial development. It may be that the machine will ultimately conquer the Communist faith and grind out the human impulses, and Russia become during this transition period as inartistic and soulless as was America until quite recent years. One would like to hope that mechanical progress will be swift and social idealism sufficiently strong to retain control. But the practical difficulties are almost insuperable.

Such signs of the progress of art as it is possible to notice at this early stage would seem to bear out the above argument. For instance, an attempt is being made to foster the continuation of peasant embroidery, carving, etc., in the towns. It is done by people who have evidently lost the tradition already. They are taught to copy the models which are placed in the Peasant Museum, but there is no comparison between the live little wooden lady who smiles beneath the glass case, and the soulless staring-eyed creature who is offered for sale, nor between the quite ordinary carved fowl one may buy and the amusing life-like figure one may merely gaze at.

But when one comes to art directly inspired by Communism it is a different story. Apart from the propaganda pictures already referred to, there are propaganda plays performed by the Red Army in its spare moments, and there are the mass pageant plays performed on State occasions. I had the good fortune to witness one of each kind.

The play was called \emph{Zarevo} (The Dawn), and was performed on a Saturday night on a small stage in a small hall in an entirely amateur fashion. It represented Russian life just before the revolution. It was intense and tragic and passionately acted. Dramatic talent is not rare in Russia. Almost the only comic relief was provided by the Tsarist police, who made one appearance towards the end, got up like comic military characters in a musical comedy---just as, in mediæval miracle plays, the comic character was Satan. The play's intention was to show a typical Russian working-class family. There were the old father, constantly drunk on vodka, alternately maudlin and scolding; the old mother; two sons, the one a Communist and the other an Anarchist; the wife of the Communist, who did dressmaking; her sister, a prostitute; and a young girl of \emph{bourgeois} family, also a Communist, involved in a plot with the Communist son, who was of course the hero of the play.

The first act revealed the stern and heroic Communist maintaining his views despite the reproaches of father and mother and the nagging of his wife. It showed also the Anarchist brother (as might be expected from the Bolshevik hostility to Anarchism) as an unruly, lazy, ne'er-do-well, with a passionate love for Sonia, the young \emph{bourgeoise}, which was likely to become dangerous if not returned. She, on the other hand, obviously preferred the Communist. It was clear that he returned her love, but it was not quite clear that he would wish the relation to be anything more than platonic comradeship in the service of their common ideal. An unsuccessful strike, bringing want and danger from the police, together with increasing jealousy on the part of the Anarchist, led up to the tragic dénouement. I was not quite definite as to how this was brought about. All violent action was performed off the stage, and this made the plot at times difficult to follow. But it seemed that the Anarchist in a jealous rage forged a letter from his brother to bring Sonia to a rendezvous, and there murdered her, at the same time betraying his brother to the police. When the latter came to effect his arrest, and accuse him also, as the most likely person, of the murder, the Anarchist was seized with remorse and confessed. Both were therefore led away together. Once the plot is sketched, the play calls for no comment. It had not great merit, though it is unwise to hazard a judgment on a play whose dialogue was not fully interpreted, but it was certainly real, and the link between audience and performers was established as it never seemed to be in the professional theatre. After the performance, the floor was cleared for dancing, and the audience were in a mood of thorough enjoyment.

The pageant of the "World Commune," which was performed at the opening of the Third International Congress in Petrograd, was a still more important and significant phenomenon. I do not suppose that anything of the kind has been staged since the days of the mediæval mystery plays. It was, in fact, a mystery play designed by the High Priests of the Communist faith to instruct the people. It was played on the steps of an immense white building that was once the Stock Exchange, a building with a classical colonnade on three sides of it, with a vast flight of steps in front, that did not extend the whole width of the building but left at each side a platform that was level with the floor of the colonnade. In front of this building a wide road ran from a bridge over one arm of the river to a bridge over the other, so that the stretches of water and sky on either side seemed to the eye of imagination like the painted wings of a gigantic stage. Two battered red columns of fantastic design, that were once light towers to guide ships, stood on either side midway between the extremities of the building and the water, but on the opposite side of the road. These two towers were beflagged and illuminated and carried the limelight, and between and behind them was gathered a densely packed audience of forty or fifty thousand people. The play began at sundown, while the sky was still red away to the right and the palaces on the far bank to the left still aglow with the setting sun, and it continued under the magic of the darkening sky. At first the beauty and grandeur of the setting drew the attention away from the performers, but gradually one became aware that on the platform before the columns kings and queens and courtiers in sumptuous conventional robes, and attended by soldiers, were conversing in dumb show with one another. A few climbed the steps of a small wooden platform that was set up in the middle, and one indicated by a lifted hand that here should be built a monument to the power of capitalism over the earth. All gave signs of delight. Sentimental music was heard, and the gay company fell to waltzing away the hours. Meanwhile, from below on the road level, there streamed out of the darkness on either side of the building and up the half-lit steps, their fetters ringing in harmony with the music, the enslaved and toiling masses coming in response to command to build the monument for their masters. It is impossible to describe the exquisite beauty of the slow movement of those dark figures aslant the broad flight of steps; individual expressions were of course indistinguishable, and yet the movement and attitude of the groups conveyed pathos and patient endurance as well as any individual speech or gesture in the ordinary theatre. Some groups carried hammer and anvil, and others staggered under enormous blocks of stone. Love for the ballet has perhaps made the Russians understand the art of moving groups of actors in unison. As I watched these processions climbing the steps in apparently careless and spontaneous fashion, and yet producing so graceful a result, I remembered the mad leap of the archers down the stage in \emph{Prince Igor}, which is also apparently careless and spontaneous and full of wild and irregular beauty, yet never varies a hair-breadth from one performance to the next.

For a time the workers toiled in the shadow in their earthly world, and dancing continued in the lighted paradise of the rulers above, until presently, in sign that the monument was complete, a large yellow disc was hoisted amid acclamation above the highest platform between the columns. But at the same moment a banner was uplifted amongst the people, and a small figure was seen gesticulating. Angry fists were shaken and the banner and speaker disappeared, only to reappear almost immediately in another part of the dense crowd. Again hostility, until finally among the French workers away up on the right, the first Communist manifesto found favour. Rallying around their banner the \emph{communards} ran shouting down the steps, gathering supporters as they came. Above, all is confusion, kings and queens scuttling in unroyal fashion with flying velvet robes to safe citadels right and left, while the army prepares to defend the main citadel of capitalism with its golden disc of power. The \emph{communards} scale the steps to the fortress which they finally capture, haul down the disc and set their banner in its place. The merry music of the \emph{Carmagnole} is heard, and the victors are seen expressing their delight by dancing first on one foot and then on the other, like marionettes. Below, the masses dance with them in a frenzy of joy. But a pompous procession of Prussian legions is seen approaching, and, amid shrieks and wails of despair, the people are driven back, and their leaders set in a row and shot. Thereafter came one of the most moving scenes in the drama. Several dark-clad women appeared carrying a black pall supported on sticks, which they set in front of the bodies of the leaders so that it stood out, an irregular pointed black shape against the white columns behind. But for this melancholy monument the stage was now empty. Thick clouds of black smoke arose from braziers on either side and obscured the steps and the platform. Through the smoke came the distant sound of Chopin's \emph{Marche Funèbre}, and as the air became clearer white figures could be dimly seen moving around the black pall in a solemn dance of mourning. Behind them the columns shone ghostly and unreal against the glimmering mauve rays of an uncertain and watery dawn.

The second part of the pageant opened in July 1914. Once again the rulers were feasting and the workers at toil, but the scene was enlivened by the presence of the leaders of the Second International, a group of decrepit professorial old men, who waddled in in solemn procession carrying tomes full of international learning. They sat in a row between the rulers and the people, deep in study, spectacles on nose. The call to war was the signal for a dramatic appeal from the workers to these leaders, who refused to accept the Red Flag, but weakly received patriotic flags from their respective governments. Jaurès, elevated to be the symbol of protest, towered above the people, crying in a loud voice, but fell back immediately as the assassin's shot rang out. Then the people divided into their national groups and the war began. It was at this point that "God Save the King" was played as the English soldiers marched out, in a comic manner which made one think of it as "\emph{Gawd} save the King." Other national anthems were burlesqued in a similar fashion, but none quite so successfully. A ridiculous effigy of the Tsar with a knout in his hand now occupied the symbolic position and dominated the scene. The incidents of the war which affected Russia were then played. Spectacular cavalry charges on the road, marching soldiers, batteries of artillery, a pathetic procession of cripples and nurses, and other scenes too numerous to describe, made up that part of the pageant devoted to the war.

Then came the Russian Revolution in all its stages. Cars dashed by full of armed men, red flags appeared everywhere, the people stormed the citadel and hauled down the effigy of the Tsar. The Kerensky Government assumed control and drove them forth to war again, but soon they returned to the charge, destroyed the Provisional Government, and hoisted all the emblems of the Russian Soviet Republic. The Entente leaders, however, were seen preparing their troops for battle, and the pageant went on to show the formation of the Red Army under its emblem the Red Star. White figures with golden trumpets appeared foretelling victory for the proletariat. The last scene, the World Commune, is described in the words of the abstract, taken from a Russian newspaper, as follows:---
\begin{displayquote}
Cannon shots announce the breaking of the blockade against Soviet Russia, and the victory of the World Proletariat. The Red Army returns from the front, and passes in triumphant review before the leaders of the Revolution. At their feet lie the crowns of kings and the gold of the bankers. Ships draped with flags are seen carrying workers from the west. The workers of the whole world, with the emblems of labour, gather for the celebration of the World Commune. In the heavens luminous inscriptions in different languages appear, greeting the Congress: "Long live the Third International! Workers of the world, unite! Triumph to the sounds of the hymn of the World Commune, the International."
\end{displayquote}
Even so glowing an account, however, hardly does it justice. It had the pomp and majesty of the Day of Judgment itself. Rockets climbed the skies and peppered them with a thousand stars, fireworks blazed on all sides, garlanded and beflagged ships moved up and down the river, chariots bearing the emblems of prosperity, grapes and corn, travelled slowly along the road. The Eastern peoples came carrying gifts and emblems. The actors, massed upon the steps, waved triumphant hands, trumpets sounded, and the song of the International from ten thousand throats rose like a mighty wave engulfing the whole.

Though the end of this drama may have erred on the side of the grandiose, this may perhaps be forgiven the organizers in view of the occasion for which they prepared it. Nothing, however, could detract from the beauty and dramatic power of the opening and of many of the scenes. Moreover, the effects obtained by movement in the mass were almost intoxicating. The first entrance of the masses gave a sense of dumb and patient force that was moving in the extreme, and the frenzied delight of the dancing crowd at the victory of the French \emph{communards} stirred one to ecstasy. The pageant lasted for five hours or more, and was as exhausting emotionally as the Passion Play is said to be. I had the vision of a great period of Communist art, more especially of such open-air spectacles, which should have the grandeur and scope and eternal meaning of the plays of ancient Greece, the mediæval mysteries, or the Shakespearean theatre. In building, writing, acting, even in painting, work would be done, as it once was, by groups, not by one hand or mind, and evolution would proceed slowly until once again the individual emerged from the mass.

In considering Education under the Bolshevik régime, the same two factors which I have already dealt with in discussing art, namely industrial development and the communist doctrine, must be taken into account. Industrial development is in reality one of the tenets of Communism, but as it is one which in Russia is likely to endanger the doctrine as a whole I have thought it better to consider it as a separate item.

As in the matter of art, so in education, those who have given unqualified praise seem to have taken the short and superficial view. It is hardly necessary to launch into descriptions of the crèches, country homes or palaces for children, where Montessori methods prevail, where the pupils cultivate their little gardens, model in plasticine, draw and sing and act, and dance their Eurythmic dances barefoot on floors once sacred to the tread of the nobility. I saw a reception and distributing house in Petrograd with which no fault could be found from the point of view of scientific organization. The children were bright-eyed and merry, and the rooms airy and clean. I saw, too, a performance by school children in Moscow which included some quite wonderful Eurythmic dancing, in particular an interpretation of Grieg's \emph{Tanz in der Halle des Bergkönigs} by the Dalcroze method, but with a colour and warmth which were Russian, and in odd contrast to the mathematical precision associated with most Dalcroze performances.

But in spite of the obvious merit of such institutions as exist, misgivings would arise. To begin with, it must be remembered that it is necessary first to admit that children should be delivered up almost entirely to the State. Nominally, the mother still comes to see her child in these schools, but in actual fact, the drafting of children to the country must intervene, and the whole temper of the authorities seemed to be directed towards breaking the link between mother and child. To some this will seem an advantage, and it is a point which admits of lengthy discussion, but as it belongs rather to the question of women and the family under Communism, I can do no more than mention it here.

Then, again, it must be remembered that the tactics of the Bolsheviks towards such schools as existed under the old régime in provincial towns and villages, have not been the same as their tactics towards the theatres. The greater number of these schools are closed, in part, it would seem, from lack of personnel, and in part from fear of counter-revolutionary propaganda. The result is that, though those schools which they have created are good and organized on modern lines, on the whole there would seem to be less diffusion of child education than before. In this, as in most other departments, the Bolsheviks show themselves loath to attempt anything which cannot be done on a large scale and impregnated with Communist doctrine. It goes without saying that Communist doctrine is taught in schools, as Christianity has been taught hitherto, moreover the Communist teachers show bitter hostility to other teachers who do not accept the doctrine. At the children's entertainment alluded to above, the dances and poems performed had nearly all some close relation to Communism, and a teacher addressed the children for something like an hour and a half on the duties of Communists and the errors of Anarchism.

This teaching of Communism, however necessary it may appear for the building of the Communist state of the future, does seem to me to be an evil in that it is done emotionally and fanatically, with an appeal to hate and militant ardour rather than to constructive reason. It binds the free intellect and destroys initiative. An industrial state needs not only obedient and patient workers and artists, it needs also men and women with initiative in scientific research. It is idle to provide channels for scientific research later if it is to be choked at the source. That source is an enquiring and free intellect unhampered by iron dogma. Beneficial to artistic and emotional development therefore, the teaching of Communism as a faith may well be most pernicious to the scientific and intellectual side of education, and will lead direct to the pragmatist view of knowledge and scientific research which the Church and the capitalist already find it so convenient to adopt.

But to come to the chief and most practical question, the relation of education to industry. Sooner or later education in Russia must become subordinate to the needs of industrial development. That the Bolsheviks already realize this is proved by the articles of Lunacharsky which recently appeared in \emph{Le Phare} (Geneva). It was the spectre of industry that haunted me throughout the consideration of education as in the consideration of art, and what I have said above of its dangers to the latter seems to me also to apply here. Montessori schools belong, in my view, to that stage in industrial development when education is directed as much towards leisure occupations as towards preparation for professional life. Possibly the fine flower of useless scientific enquiry belongs to this stage also. Nobody in Russia is likely to have much leisure for a good many years to come, if the Bolshevik programme of industrial development is efficiently carried out. And there seemed to me to be something pathetic and almost cruel in this varied and agreeable education of the child, when one reflected on the long hours of grinding toil to which he was soon to be subject in workshop or factory. For I repeat that I do not believe industrial work in the early days of industry can be made tolerable to the worker. Once again I experienced the dread of seeing the ideals of the Russian revolutionaries go down before the logic of necessity. They are beginning to pride themselves on being hard, practical men, and it seems quite reasonable to fear that they should come to regard this full and humane development of the child as a mere luxury and ultimately neglect it. Worse still, the few of these schools which already exist may perhaps become exclusive to the Communists and their children, or that company of Samurai which is to leaven and govern the mass of the people. If so, they will soon come to resemble our public schools, in that they will prepare, in an artificial play atmosphere, men who will pass straight to the position of leaders, while the portion of the proletariat who serve under them will be reading and writing, just so much technical training as is necessary, and Communist doctrine.

This is a nightmare hypothesis, but the difficulties of the practical problem seem to warrant its entertainment. The number of people in Russia who can even read and write is extremely small, the need to get them employed industrially as rapidly as possible is very great, hence the system of education which develops out of this situation cannot be very ambitious or enlightened. Further it will have to continue over a sufficiently long period of time to allow of the risk of its becoming stable and traditional. In adult education already the pupil comes for a short period, learns Communism, reading and writing---there is hardly time to give him much more---and returns to leaven the army or his native village. In achieving this the Bolsheviks are already doing a very important and valuable work, but they cannot hope for a long while to become the model of public instruction which they have hitherto been represented to be. And the conditions of their becoming so ultimately are adherence to their ideals through a very long period of stress, and a lessening of fanaticism in their Communist teaching, conditions which, unhappily, seem to be mutually incompatible.

The whole of the argument set out in this chapter may be summed up in the statement of one fact which the mere idealist is prone to overlook, namely that Russia is a country at a stage in economic development not much more advanced than America in the pioneer days. The old civilization was aristocratic and exotic; it could not survive in the modern world. It is true that it produced great men, but its foundations were rotten. The new civilization may, for the moment, be less productive of individual works of genius, but it has a new solidity and gives promise of a new unity. It may be that I have taken too hopeful a view and that the future evolution of Russia will have as little connection with the life and tradition of its present population as modern America with the life of the Red Indian tribes. The fact that there exists in Russia a population at a far higher stage of culture, which will be industrially educated, not exterminated, militates against this hypothesis, but the need for education may make progress slower than it was in the United States.

One would not have looked for the millennium of Communism, nor even for valuable art and educational experiment in the America of early railroading and farming days. Nor must one look for such things from Russia yet. It may be that during the next hundred years there, economic evolution will obscure Communist ideals, until finally, in a country that has reached the stage of present-day America, the battle will be fought out again to a victorious and stable issue. Unless, indeed, the Marxian scripture prove to be not infallible, and faith and heroic devotion show themselves capable of triumphing over economic necessity.