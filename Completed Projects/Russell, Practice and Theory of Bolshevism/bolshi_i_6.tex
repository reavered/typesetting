At first sight it is surprising that Russian industry should have collapsed as badly as it has done, and still more surprising that the efforts of the Communists have not been more successful in reviving it. As I believe that the continued efficiency of industry is the main condition for success in the transition to a Communist State, I shall endeavour to analyse the causes of the collapse, with a view to the discovery of ways by which it can be avoided elsewhere.

Of the fact of the collapse there can be no doubt. The Ninth Congress of the Communist Party (March-April, 1920) speaks of "the incredible catastrophes of public economy," and in connection with transport, which is one of the vital elements of the problem, it acknowledges "the terrible collapse of the transport and the railway system," and urges the introduction of "measures which cannot be delayed and which are to obviate the complete paralysis of the railway system and, together with this, the ruin of the Soviet Republic." Almost all those who have visited Russia would confirm this view of the gravity of the situation. In the factories, in great works like those of Putilov and Sornovo, very little except war work is being done; machinery stands idle and plant is becoming unusable. One sees hardly any new manufactured articles in Russia, beyond a certain very inadequate quantity of clothes and boots---always excepting what is needed for the army. And the difficulty of obtaining food is conclusive evidence of the absence of goods such as are needed by the peasants.

How has this state of affairs arisen? And why does it continue?

A great deal of disorganization occurred before the first revolution and under Kerensky. Russian industry was partly dependent on Poland; the war was conducted by methods of reckless extravagance, especially as regards rolling-stock; under Kerensky there was a tendency to universal holiday, under the impression that freedom had removed the necessity for work. But when all this is admitted to the full, it remains true that the state of industry under the Bolsheviks is much worse than even under Kerensky.

The first and most obvious reason for this is that Russia was quite unusually dependent upon foreign assistance. Not only did the machinery in the factories and the locomotives on the railways come from abroad, but the organizing and technical brains in industry were mainly foreign. When the Entente became hostile to Russia, the foreigners in Russian industry either left the country or assisted counter-revolution. Even those who were in fact loyal naturally became suspect, and could not well be employed in responsible posts, any more than Germans could in England during the war. The native Russians who had technical or business skill were little better; they almost all practised sabotage in the first period of the Bolshevik régime. One hears amusing stories of common sailors frantically struggling with complicated accounts, because no competent accountant would work for the Bolsheviks.

But those days passed. When the Government was seen to be stable, a great many of those who had formerly sabotaged it became willing to accept posts under it, and are now in fact so employed, often at quite exceptional salaries. Their importance is thoroughly realized. One resolution at the above-mentioned Congress says (I quote verbally the unedited document which was given to us in Moscow):

\begin{displayquote}
Being of opinion that without a scientific organization of industry, even the widest application of compulsory labour service, as the great labour heroism of the working class, will not only fail to secure the establishment of a powerful socialist production, but will also fail to assist the country to free itself from the clutches of poverty---the Congress considers it imperative to register all able specialists of the various departments of public economy and widely to utilize them for the purpose of industrial organization.

The Congress considers the elucidation for the wide masses of the workers of the tremendous character of the economic problems of the country to be one of the chief problems of industrial and general political agitation and propaganda; and of equal importance to this, technical education, and administrative and scientific technical experience. The Congress makes it obligatory on all the members of the party mercilessly to fight that particular obnoxious form, the ignorant conceit which deems the working class capable of solving all problems without the assistance \emph{in the most responsible cases} of specialists of the bourgeois school, the management. Demagogic elements who speculate on this kind of prejudice in the more backward section of our working classes, can have no place in the ranks of the party of Scientific Socialism.
\end{displayquote}

But Russia alone is unable to supply the amount of skill required, and is very deficient in technical instructors, as well as in skilled workmen. One was told, over and over again, that the first step in improvement would be the obtaining of spare parts for locomotives. It seems strange that these could not be manufactured in Russia. To some extent they can be, and we were shown locomotives which had been repaired on Communist Saturdays. But in the main the machinery for making spare parts is lacking and the skill required for its manufacture does not exist. Thus dependence on the outside world persists, and the blockade continues to do its deadly work of spreading hunger, demoralization and despair.

The food question is intimately bound up with the question of industry. There is a vicious circle, for not only does the absence of manufactured goods cause a food shortage in the towns, but the food shortage, in turn, diminishes the strength of the workers and makes them less able to produce goods. I cannot but think that there has been some mismanagement as regards the food question. For example, in Petrograd many workers have allotments and often work in them for eight hours after an eight hours' day in their regular employment. But the food produced in the allotments is taken for general consumption, not left to each individual producer. This is in accordance with Communist theory, but of course greatly diminishes the incentive to work, and increases the red tape and administrative machinery.

Lack of fuel has been another very grave source of trouble. Before the war coal came mostly from Poland and the Donetz Basin. Poland is lost to Russia, and the Donetz Basin was in the hands of Denikin, who so destroyed the mines before retreating that they are still not in working order. The result is a practically complete absence of coal. Oil, which is equally important in Russia, was also lacking until the recent recovery of Baku. All that I saw on the Volga made me believe that real efficiency has been shown in reorganizing the transport of oil, and doubtless this will do something to revive industry. But the oil used to be worked very largely by Englishmen, and English machinery is much needed for refining it. In the meantime, Russia has had to depend upon wood, which involves immense labour. Most of the houses are not warmed in winter, so that people live in a temperature below freezing-point. Another consequence of lack of fuel was the bursting of water-pipes, so that people in Petrograd, for the most part, have to go down to the Neva to fetch their water---a considerable addition to the labour of an already overworked day.

I find it difficult to believe that, if greater efficiency had existed in the Government, the food and fuel difficulties could not have been considerably alleviated. In spite of the needs of the army, there are still many horses in Russia; I saw troops of thousands of horses on the Volga, which apparently belonged to Kalmuk tribes. By the help of carts and sledges, it ought to be possible, without more labour than is warranted by the importance of the problem, to bring food and timber into Moscow and Petrograd. It must be remembered that both cities are surrounded by forests, and Moscow at least is surrounded by good agricultural land. The Government has devoted all its best energies hitherto to the two tasks of war and propaganda, while industry and the food problem have been left to a lesser degree of energy and intelligence. It is no doubt probable that, if peace is secured, the economic problems will receive more attention than hitherto. But the Russian character seems less adapted to steady work of an unexciting nature than to heroic efforts on great occasions; it has immense passive endurance, but not much active tenacity. Whether, with the menace of foreign invasion removed, enough day-by-day detailed energy would exist for the reorganization of industry, is a doubtful question, as to which only time can decide.

This leads to the conclusion---which I think is adopted by most of the leading men in Russia---that it will be very difficult indeed to save the revolution without outside economic assistance. Outside assistance from capitalist countries is dangerous to the principles of Communism, as well as precarious from the likelihood of fresh causes of quarrel. But the need of help is urgent, and if the policy of promoting revolution elsewhere were to succeed, it would probably render the nations concerned temporarily incapable of supplying Russian needs. It is, therefore, necessary for Russia to accept the risks and uncertainties involved in attempting to make peace with the Entente and to trade with America. By continuing war, Russia can do infinite damage to us, especially in Asia, but cannot hope, for many years, to achieve any degree of internal prosperity. The situation, therefore, is one in which, even from the narrowest point of view, peace is to the interest of both parties.

It is difficult for an outsider with only superficial knowledge to judge of the efforts which have been made to reorganize industry without outside help. These efforts have chiefly taken the form of industrial conscription. Workers in towns seek to escape to the country, in order to have enough to eat; but this is illegal and severely punished. The same Communist Report from which I have already quoted speaks on this subject as follows:
\begin{displayquote}
\emph{Labour Desertion.}---Owing to the fact that a considerable part of the workers either in search of better food conditions or often for the purposes of speculation, voluntarily leave their places of employment or change from place to place, which inevitably harms production and deteriorates the general position of the working class, the Congress considers one of the most urgent problems of Soviet Government and of the Trade Union organization to be established as the firm, systematic and insistent struggle with labour desertion, The way to fight this is to publish a list of desertion fines, the creation of a labour Detachment of Deserters under fine, and, finally, internment in concentration camps.
\end{displayquote}
It is hoped to extend the system to the peasantry:
\begin{displayquote}
The defeat of the White Armies and the problems of peaceful construction in connection with the incredible catastrophes of public economy demand an extraordinary effort of all the powers of the proletariat and the drafting into the process of public labour of the wide masses of the peasantry.
\end{displayquote}
On the vital subject of transport, in a passage of which I have already quoted a fragment, the Communist Party declares:
\begin{displayquote}
For the most immediate future transport remains the centre of the attention and the efforts of the Soviet Government. The improvement of transport is the indispensable basis upon which even the most moderate success in all other spheres of production and first of all in the provision question can be gained.

The chief difficulty with regard to the improvement of transport is the weakness of the Transport Trade Union, which is due in the first case to the heterogeneity of the personnel of the railways, amongst whom there are still a number of those who belong to the period of disorganization, and, secondly, to the fact that the most class-conscious and best elements of the railway proletariat were at the various fronts of the civil war.

Considering wide Trade Union assistance to the railway workers to be one of the principal tasks of the Party, and as the only condition under which transport can be raised to its height, the Congress at the same time recognizes the inflexible necessity of employing exclusive and extraordinary measures (martial law, and so forth). Such necessity is the result of the terrible collapse of the transport and the railroad system and is to introduce measures which cannot be delayed and which are to obviate the complete paralysis of the railway system and, together with this, the ruin of the Soviet Republic.
\end{displayquote}
The general attitude to the militarization of labour is stated in the Resolution with which this section of the Proceedings begins:
\begin{displayquote}
The ninth Congress approves of the decision of the Central Committee of the Russian Communist Party on the mobilization of the industrial proletariat, compulsory labour service, militarization of production and the application of military detachments to economic needs.

In connection with the above, the Congress decrees that the Party organization should in every way assist the Trade Unions and the Labour Sections in registering all skilled workers with a view of employing them in the various branches of production with the same consistency and strictness as was done, and is being carried out at the present time, in relation to the commanding staff for army needs.

Every skilled worker is to return to his particular trade Exceptions, i.e. the retention of the skilled worker in any other branch of Soviet service, is allowed only with the sanction of the corresponding central and local authorities.
\end{displayquote}
It is, of course, evident that in these measures the Bolsheviks have been compelled to travel a long way from the ideals which originally inspired the revolution. But the situation is so desperate that they could not be blamed if their measures were successful. In a shipwreck all hands must turn to, and it would be ridiculous to prate of individual liberty. The most distressing feature of the situation is that these stern laws seem to have produced so little effect. Perhaps in the course of years Russia might become self-supporting without help from the outside world, but the suffering meantime would be terrible. The early hopes of the revolution would fade more and more. Every failure of industry, every tyrannous regulation brought about by the desperate situation, is used by the Entente as a justification of its policy. If a man is deprived of food and drink, he will grow weak, lose his reason, and finally die. This is not usually considered a good reason for inflicting death by starvation. But where nations are concerned, the weakness and struggles are regarded as morally culpable, and are held to justify further punishment. So at least it has been in the case of Russia. Nothing produced a doubt in our governing minds as to the rightness of our policy except the strength of the Red Army and the fear of revolution in Asia. Is it surprising that professions of humanitarian feeling on the part of English people are somewhat coldly received in Soviet Russia?