The Bolshevik argument against Parliamentary democracy as a method of achieving Socialism is a powerful one. My answer to it lies rather in pointing out what I believe to be fallacies in the Bolshevik method, from which I conclude that no swift method exists of establishing any desirable form of Socialism. But let us first see what the Bolshevik argument is.

In the first place, it assumes that those to whom it is addressed are absolutely certain that Communism is desirable, so certain that they are willing, if necessary, to force it upon an unwilling population at the point of the bayonet. It then proceeds to argue that, while capitalism retains its hold over propaganda and its means of corruption, Parliamentary methods are very unlikely to give a majority for Communism in the House of Commons, or to lead to effective action by such a majority even if it existed. Communists point out how the people are deceived, and how their chosen leaders have again and again betrayed them. From this they argue that the destruction of capitalism must be sudden and catastrophic; that it must be the work of a minority; and that it cannot be effected constitutionally or without violence. It is therefore, in their view, the duty of the Communist party in a capitalist country to prepare for armed conflict, and to take all possible measure for disarming the bourgeoisie and arming that part of the proletariat which is willing to support the Communists.

There is an air of realism and disillusionment about this position, which makes it attractive to those idealists who wish to think themselves cynics. But I think there are various points in which it fails to be as realistic as it pretends.

In the first place, it makes much of the treachery of Labour leaders in constitutional movements, but does not consider the possibility of the treachery of Communist leaders in a revolution. To this the Marxian would reply that in constitutional movements men are bought, directly or indirectly, by the money of the capitalists, but that revolutionary Communism would leave the capitalists no money with which to attempt corruption. This has been achieved in Russia, and could be achieved elsewhere. But selling oneself to the capitalists is not the only possible form of treachery. It is also possible, having acquired power, to use it for one's own ends instead of for the people. This is what I believe to be likely to happen in Russia: the establishment of a bureaucratic aristocracy, concentrating authority in its own hands, and creating a régime just as oppressive and cruel as that of capitalism. Marxians never sufficiently recognize that love of power is quite as strong a motive, and quite as great a source of injustice, as love of money; yet this must be obvious to any unbiased student of politics. It is also obvious that the method of violent revolution leading to a minority dictatorship is one peculiarly calculated to create habits of despotism which would survive the crisis by which they were generated. Communist politicians are likely to become just like the politicians of other parties: a few will be honest, but the great majority will merely cultivate the art of telling a plausible tale with a view to tricking the people into entrusting them with power. The only possible way by which politicians as a class can be improved is the political and psychological education of the people, so that they may learn to detect a humbug. In England men have reached the point of suspecting a good speaker, but if a man speaks badly they think he must be honest. Unfortunately, virtue is not so widely diffused as this theory would imply.

In the second place, it is assumed by the Communist argument that, although capitalist propaganda can prevent the majority from becoming Communists, yet capitalist laws and police forces cannot prevent the Communists, while still a minority, from acquiring a supremacy of military power. It is thought that secret propaganda can undermine the army and navy, although it is admittedly impossible to get the majority to vote at elections for the programme of the Bolsheviks. This view is based upon Russian experience, where the army and navy had suffered defeat and had been brutally ill used by incompetent Tsarist authorities. The argument has no application to more efficient and successful States. Among the Germans, even in defeat, it was the civilian population that began the revolution.

There is a further assumption in the Bolshevik argument which seems to me quite unwarrantable. It is assumed that the capitalist governments will have learned nothing from the experience of Russia. Before the Russian Revolution, governments had not studied Bolshevik theory. And defeat in war created a revolutionary mood throughout Central and Eastern Europe. But now the holders of power are on their guard. There seems no reason whatever to suppose that they will supinely permit a preponderance of armed force to pass into the hands of those who wish to overthrow them, while, according to the Bolshevik theory, they are still sufficiently popular to be supported by a majority at the polls. Is it not as clear as noonday that in a democratic country it is more difficult for the proletariat to destroy the Government by arms than to defeat it in a general election? Seeing the immense advantages of a Government in dealing with rebels, it seems clear that rebellion could have little hope of success unless a very large majority supported it. Of course, if the army and navy were specially revolutionary, they might effect an unpopular revolution; but this situation, though something like it occurred in Russia, is hardly to be expected in the Western nations. This whole Bolshevik theory of revolution by a minority is one which might just conceivably have succeeded as a secret plot, but becomes impossible as soon as it is openly avowed and advocated.

But perhaps it will be said that I am caricaturing the Bolshevik doctrine of revolution. It is urged by advocates of this doctrine, quite truly, that all political events are brought about by minorities, since the majority are indifferent to politics. But there is a difference between a minority in which the indifferent acquiesce, and a minority so hated as to startle the indifferent into belated action. To make the Bolshevik doctrine reasonable, it is necessary to suppose that they believe the majority can be induced to acquiesce, at least temporarily, in the revolution made by the class-conscious minority. This, again, is based upon Russian experience: desire for peace and land led to a widespread support of the Bolsheviks in November 1917 on the part of people who have subsequently shown no love for Communism.

I think we come here to an essential part of Bolshevik philosophy. In the moment of revolution, Communists are to have some popular cry by which they win more support than mere Communism could win. Having thus acquired the State machine, they are to use it for their own ends. But this, again, is a method which can only be practised successfully so long as it is not avowed. It is to some extent habitual in politics. The Unionists in 1900 won a majority on the Boer War, and used it to endow brewers and Church schools. The Liberals in 1906 won a majority on Chinese labour, and used it to cement the secret alliance with France and to make an alliance with Tsarist Russia. President Wilson, in 1916, won his majority on neutrality, and used it to come into the war. This method is part of the stock-in-trade of democracy. But its success depends upon repudiating it until the moment comes to practise it. Those who, like the Bolsheviks, have the honesty to proclaim in advance their intention of using power for other ends than those for which it was given them, are not likely to have a chance of carrying out their designs.

What seems to me to emerge from these considerations is this: That in a democratic and politically educated country, armed revolution in favour of Communism would have no chance of succeeding unless it were supported by a larger majority than would be required for the election of a Communist Government by constitutional methods. It is possible that, if such a Government came into existence, and proceeded to carry out its programme, it would be met by armed resistance on the part of capital, including a large proportion of the officers in the army and navy. But in subduing this resistance it would have the support of that great body of opinion which believes in legality and upholds the constitution. Moreover, having, by hypothesis, converted a majority of the nation, a Communist Government could be sure of loyal help from immense numbers of workers, and would not be forced, as the Bolsheviks are in Russia, to suspect treachery everywhere. Under these circumstances, I believe that the resistance of the capitalists could be quelled without much difficulty, and would receive little support from moderate people. Whereas, in a minority revolt of Communists against a capitalist Government, all moderate opinion would be on the side of capitalism.

The contention that capitalist propaganda is what prevents the adoption of Communism by wage-earners is only very partially true. Capitalist propaganda has never been able to prevent the Irish from voting against the English, though it has been applied to this object with great vigour. It has proved itself powerless, over and over again, in opposing nationalist movements which had almost no moneyed support. It has been unable to cope with religious feeling. And those industrial populations which would most obviously benefit by Socialism have, in the main, adopted it, in spite of the opposition of employers. The plain truth is that Socialism does not arouse the same passionate interest in the average citizen as is roused by nationality and used to be roused by religion. It is not unlikely that things may change in this respect: we may be approaching a period of economic civil wars comparable to that of the religious civil wars that followed the Reformation. In such a period, nationalism is submerged by party: British and German Socialists, or British and German capitalists, will feel more kinship with each other than with compatriots of the opposite political camp. But when that day comes, there will be no difficulty, in highly industrial countries, in securing Socialist majorities; if Socialism is not then carried without bloodshed, it will be due to the unconstitutional action of the rich, not to the need of revolutionary violence on the part of the advocates of the proletariat. Whether such a state of opinion grows up or not depends mainly upon the stubbornness or conciliatoriness of the possessing classes, and, conversely, upon the moderation or violence of those who desire fundamental economic change. The majority which Bolsheviks regard as unattainable is chiefly prevented by the ruthlessness of their own tactics.

Apart from all arguments of detail, there are two broad objections to violent revolution in a democratic community. The first is that, when once the principle of respecting majorities as expressed at the ballot-box is abandoned, there is no reason to suppose that victory will be secured by the particular minority to which one happens to belong. There are many minorities besides Communists: religious minorities, teetotal minorities, militarist minorities, capitalist minorities. Any one of these could adopt the method of obtaining power advocated by the Bolsheviks, and any one would be just as likely to succeed as they are. What restrains these minorities, more or less, at present, is respect for the law and the constitution. Bolsheviks tacitly assume that every other party will preserve this respect while they themselves, unhindered, prepare the revolution. But if their philosophy of violence becomes popular, there is not the slightest reason to suppose that they will be its beneficiaries. They believe that Communism is for the good of the majority; they ought to believe that they can persuade the majority on this question, and to have the patience to set about the task of winning by propaganda.

The second argument of principle against the method of minority violence is that abandonment of law, when it becomes widespread, lets loose the wild beast, and gives a free rein to the primitive lusts and egoisms which civilization in some degree curbs. Every student of mediæval thought must have been struck by the extraordinarily high value placed upon law in that period. The reason was that, in countries infested by robber barons, law was the first requisite of progress. We, in the modern world, take it for granted that most people will be law-abiding, and we hardly realize what centuries of effort have gone to making such an assumption possible. We forget how many of the good things that we unquestionably expect would disappear out of life if murder, rape, and robbery with violence became common. And we forget even more how very easily this might happen. The universal class-war foreshadowed by the Third International, following upon the loosening of restraints produced by the late war, and combined with a deliberate inculcation of disrespect for law and constitutional government, might, and I believe would, produce a state of affairs in which it would be habitual to murder men for a crust of bread, and in which women would only be safe while armed men protected them. The civilized nations have accepted democratic government as a method of settling internal disputes without violence. Democratic government may have all the faults attributed to it, but it has the one great merit that people are, on the whole, willing to accept it as a substitute for civil war in political disputes. Whoever sets to work to weaken this acceptance, whether in Ulster or in Moscow, is taking a fearful responsibility. Civilization is not so stable that it cannot be broken up; and a condition of lawless violence is not one out of which any good thing is likely to emerge. For this reason, if for no other, revolutionary violence in a democracy is infinitely dangerous.