The larger events in the political life of the world are determined by the interaction of material conditions and human passions. The operation of the passions on the material conditions is modified by intelligence. The passions themselves may be modified by alien intelligence guided by alien passions. So far, such modification has been wholly unscientific, but it may in time become as precise as engineering.

The classification of the passions which is most convenient in political theory is somewhat different from that which would be adopted in psychology.

We may begin with desires for the necessaries of life: food, drink, sex, and (in cold climates) clothing and housing. When these are threatened, there is no limit to the activity and violence that men will display.

Planted upon these primitive desires are a number of secondary desires. Love of property, of which the fundamental political importance is obvious, may be derived historically and psychologically from the hoarding instinct. Love of the good opinion of others (which we may call vanity) is a desire which man shares with many animals; it is perhaps derivable from courtship, but has great survival value, among gregarious animals, in regard to others besides possible mates. Rivalry and love of power are perhaps developments of jealousy; they are akin, but not identical.

These four passions---acquisitiveness, vanity, rivalry, and love of power---are, after the basic instincts, the prime movers of almost all that happens in politics. Their operation is intensified and regularized by herd instinct. But herd instinct, by its very nature, cannot be a prime mover, since it merely causes the herd to act in unison, without determining what the united action is to be. Among men, as among other gregarious animals, the united action, in any given circumstances, is determined partly by the common passions of the herd, partly by imitation of leaders. The art of politics consists in causing the latter to prevail over the former.

Of the four passions we have enumerated, only one, namely acquisitiveness, is concerned at all directly with men's relations to their material conditions. The other three---vanity, rivalry, and love of power---are concerned with social relations. I think this is the source of what is erroneous in the Marxian interpretation of history, which tacitly assumes that acquisitiveness is the source of all political actions. It is clear that many men willingly forego wealth for the sake of power and glory, and that nations habitually sacrifice riches to rivalry with other nations. The desire for some form of superiority is common to almost all energetic men. No social system which attempts to thwart it can be stable, since the lazy majority will never be a match for the energetic minority.

What is called "virtue" is an offshoot of vanity: it is the habit of acting in a manner which others praise.

The operation of material conditions may be illustrated by the statement (Myers's \emph{Dawn of History}) that four of the greatest movements of conquest have been due to drought in Arabia, causing the nomads of that country to migrate into regions already inhabited. The last of these four movements was the rise of Islam. In these four cases, the primal need of food and drink was enough to set events in motion; but as this need could only be satisfied by conquest, the four secondary passions must have very soon come into play. In the conquests of modern industrialism, the secondary passions have been almost wholly dominant, since those who directed them had no need to fear hunger or thirst. It is the potency of vanity and love of power that gives hope for the industrial future of Soviet Russia, since it enables the Communist State to enlist in its service men whose abilities might give them vast wealth in a capitalistic society.

Intelligence modifies profoundly the operation of material conditions. When America was first discovered, men only desired gold and silver; consequently the portions first settled were not those that are now most profitable. The Bessemer process created the German iron and steel industry; inventions requiring oil have created a demand for that commodity which is one of the chief influences in international politics.

The intelligence which has this profound effect on politics is not political, but scientific and technical: it is the kind of intelligence which discovers how to make nature minister to human passions. Tungsten had no value until it was found to be useful in the manufacture of shells and electric light, but now people will, if necessary, kill each other in order to acquire tungsten. Scientific intelligence is the cause of this change.

The progress or retrogression of the world depends, broadly speaking, upon the balance between acquisitiveness and rivalry. The former makes for progress, the latter for retrogression. When intelligence provides improved methods of production, these may be employed to increase the general share of goods, or to set apart more of the labour power of the community for the business of killing its rivals. Until 1914, acquisitiveness had prevailed, on the whole, since the fall of Napoleon; the past six years have seen a prevalence of the instinct of rivalry. Scientific intelligence makes it possible to indulge this instinct more fully than is possible for primitive peoples, since it sets free more men from the labour of producing necessaries. It is possible that scientific intelligence may, in time, reach the point when it will enable rivalry to exterminate the human race. This is the most hopeful method of bringing about an end of war.

For those who do not like this method, there is another: the study of scientific psychology and physiology. The physiological causes of emotions have begun to be known, through the studies of such men as Cannon (\emph{Bodily Changes in Pain, Hunger, Fear and Rage}). In time, it may become possible, by physiological means, to alter the whole emotional nature of a population. It will then depend upon the passions of the rulers how this power is used. Success will come to the State which discovers how to promote pugnacity to the extent required for external war, but not to the extent which would lead to domestic dissensions. There is no method by which it can be insured that rulers shall desire the good of mankind, and therefore there is no reason to suppose that the power to modify men's emotional nature would cause progress.

If men desired to diminish rivalry, there is an obvious method. Habits of power intensify the passion of rivalry; therefore a State in which power is concentrated will, other things being equal, be more bellicose than one in which power is diffused. For those who dislike wars, this is an additional argument against all forms of dictatorship. But dislike of war is far less common than we used to suppose; and those who like war can use the same argument to support dictatorship.