I entered Soviet Russia on May 11th and recrossed the frontier on June 16th. The Russian authorities only admitted me on the express condition that I should travel with the British Labour Delegation, a condition with which I was naturally very willing to comply, and which that Delegation kindly allowed me to fulfil. We were conveyed from the frontier to Petrograd, as well as on subsequent journeys, in a special \emph{train de luxe}; covered with mottoes about the Social Revolution and the Proletariat of all countries; we were received everywhere by regiments of soldiers, with the Internationale being played on the regimental band while civilians stood bare-headed and soldiers at the salute; congratulatory orations were made by local leaders and answered by prominent Communists who accompanied us; the entrances to the carriages were guarded by magnificent Bashkir cavalry-men in resplendent uniforms; in short, everything was done to make us feel like the Prince of Wales. Innumerable functions were arranged for us: banquets, public meetings, military reviews, etc.

The assumption was that we had come to testify to the solidarity of British Labour with Russian Communism, and on that assumption the utmost possible use was made of us for Bolshevik propaganda. We, on the other hand, desired to ascertain what we could of Russian conditions and Russian methods of government, which was impossible in the atmosphere of a royal progress. Hence arose an amicable contest, degenerating at times into a game of hide and seek: while they assured us how splendid the banquet or parade was going to be, we tried to explain how much we should prefer a quiet walk in the streets. I, not being a member of the Delegation, felt less obligation than my companions did to attend at propaganda meetings where one knew the speeches by heart beforehand. In this way, I was able, by the help of neutral interpreters, mostly English or American, to have many conversations with casual people whom I met in the streets or on village greens, and to find out how the whole system appears to the ordinary non-political man and woman. The first five days we spent in Petrograd, the next eleven in Moscow. During this time we were living in daily contact with important men in the Government, so that we learned the official point of view without difficulty. I saw also what I could of the intellectuals in both places. We were all allowed complete freedom to see politicians of opposition parties, and we naturally made full use of this freedom. We saw Mensheviks, Social Revolutionaries of different groups, and Anarchists; we saw them without the presence of any Bolsheviks, and they spoke freely after they had overcome their initial fears. I had an hour's talk with Lenin, virtually \emph{tête-à-tête}; I met Trotsky, though only in company; I spent a night in the country with Kamenev; and I saw a great deal of other men who, though less known outside Russia, are of considerable importance in the Government.

At the end of our time in Moscow we all felt a desire to see something of the country, and to get in touch with the peasants, since they form about 85 per cent, of the population. The Government showed the greatest kindness in meeting our wishes, and it was decided that we should travel down the Volga from Nijni Novgorod to Saratov, stopping at many places, large and small, and talking freely with the inhabitants. I found this part of the time extraordinarily instructive. I learned to know more than I should have thought possible of the life and outlook of peasants, village schoolmasters, small Jew traders, and all kinds of people. Unfortunately, my friend, Clifford Allen, fell ill, and my time was much taken up with him. This had, however, one good result, namely, that I was able to go on with the boat to Astrakhan, as he was too ill to be moved off it. This not only gave me further knowledge of the country, but made me acquainted with Sverdlov, Acting Minister of Transport, who was travelling on the boat to organize the movement of oil from Baku up the Volga, and who was one of the ablest as well as kindest people whom I met in Russia.

One of the first things that I discovered after passing the Red Flag which marks the frontier of Soviet Russia, amid a desolate region of marsh, pine wood, and barbed wire entanglements, was the profound difference between the theories of actual Bolsheviks and the version of those theories current among advanced Socialists in this country. Friends of Russia here think of the dictatorship of the proletariat as merely a new form of representative government, in which only working men and women have votes, and the constituencies are partly occupational, not geographical. They think that "proletariat" means "proletariat," but "dictatorship" does not quite mean "dictatorship." This is the opposite of the truth. When a Russian Communist speaks of dictatorship, he means the word literally, but when he speaks of the proletariat, he means the word in a Pickwickian sense. He means the "class-conscious" part of the proletariat, \emph{i.e.}, the Communist Party.\footnote{See the article "On the rôle of the Communist Party in the Proletarian Revolution," in \emph{Theses presented to the Second Congress of the Communist International, Petrograd-Moscow, 18 July, 1920}---a valuable work which I possess only in French.} He includes people by no means proletarian (such as Lenin and Tchicherin) who have the right opinions, and he excludes such wage-earners as have not the right opinions, whom he classifies as lackeys of the \emph{bourgeoisie}. The Communist who sincerely believes the party creed is convinced that private property is the root of all evil; he is so certain of this that he shrinks from no measures, however harsh, which seem necessary for constructing and preserving the Communist State. He spares himself as little as he spares others. He works sixteen hours a day, and foregoes his Saturday half-holiday. He volunteers for any difficult or dangerous work which needs to be done, such as clearing away piles of infected corpses left by Kolchak or Denikin. In spite of his position of power and his control of supplies, he lives an austere life. He is not pursuing personal ends, but aiming at the creation of a new social order. The same motives, however, which make him austere make him also ruthless. Marx has taught that Communism is fatally predestined to come about; this fits in with the Oriental traits in the Russian character, and produces a state of mind not unlike that of the early successors of Mahomet. Opposition is crushed without mercy, and without shrinking from the methods of the Tsarist police, many of whom are still employed at their old work. Since all evils are due to private property, the evils of the Bolshevik régime while it has to fight private property will automatically cease as soon as it has succeeded.

These views are the familiar consequences of fanatical belief. To an English mind they reinforce the conviction upon which English life has been based ever since 1688, that kindliness and tolerance are worth all the creeds in the world----a view which, it is true, we do not apply to other nations or to subject races.

In a very novel society it is natural to seek for historical parallels. The baser side of the present Russian Government is most nearly paralleled by the Directoire in France, but on its better side it is closely analogous to the rule of Cromwell. The sincere Communists (and all the older members of the party have proved their sincerity by years of persecution) are not unlike the Puritan soldiers in their stern politico-moral purpose. Cromwell's dealings with Parliament are not unlike Lenin's with the Constituent Assembly. Both, starting from a combination of democracy and religious faith, were driven to sacrifice democracy to religion enforced by military dictatorship. Both tried to compel their countries to live at a higher level of morality and effort than the population found tolerable. Life in modern Russia, as in Puritan England, is in many ways contrary to instinct. And if the Bolsheviks ultimately fall, it will be for the reason for which the Puritans fell: because there comes a point at which men feel that amusement and ease are worth more than all other goods put together.

Far closer than any actual historical parallel is the parallel of Plato's Republic. The Communist Party corresponds to the guardians; the soldiers have about the same status in both; there is in Russia an attempt to deal with family life more or less as Plato suggested. I suppose it may be assumed that every teacher of Plato throughout the world abhors Bolshevism, and that every Bolshevik regards Plato as an antiquated \emph{bourgeois}. Nevertheless, the parallel is extraordinarily exact between Plato's Republic and the régime which the better Bolsheviks are endeavouring to create.

Bolshevism is internally aristocratic and externally militant. The Communists in many ways resemble the British public-school type: they have all the good and bad traits of an aristocracy which is young and vital. They are courageous, energetic, capable of command, always ready to serve the State; on the other hand, they are dictatorial, lacking in ordinary consideration for the plebs. They are practically the sole possessors of power, and they enjoy innumerable advantages in consequence. Most of them, though far from luxurious, have better food than other people. Only people of some political importance can obtain motor-cars or telephones. Permits for railway journeys, for making purchases at the Soviet stores (where prices are about one-fiftieth of what they are in the market), for going to the theatre, and so on, are, of course, easier to obtain for the friends of those in power than for ordinary mortals. In a thousand ways, the Communists have a life which is happier than that of the rest of the community. Above all, they are less exposed to the unwelcome attentions of the police and the extraordinary commission.

The Communist theory of international affairs is exceedingly simple. The revolution foretold by Marx, which is to abolish capitalism throughout the world, happened to begin in Russia, though Marxian theory would seem to demand that it should begin in America. In countries where the revolution has not yet broken out, the sole duty of a Communist is to hasten its advent. Agreements with capitalist States can only be make-shifts, and can never amount on either side to a sincere peace. No real good can come to any country without a bloody revolution: English Labour men may fancy that a peaceful evolution is possible, but they will find their mistake. Lenin told me that he hopes to see a Labour Government in England, and would wish his supporters to work for it, but solely in order that the futility of Parliamentarism may be conclusively demonstrated to the British working man. Nothing will do any real good except the arming of the proletariat and the disarming of the \emph{bourgeoisie}. Those who preach anything else are social traitors or deluded fools.

For my part, after weighing this theory carefully, and after admitting the whole of its indictment of \emph{bourgeois} capitalism, I find myself definitely and strongly opposed to it. The Third International is an organization which exists to promote the class-war and to hasten the advent of revolution everywhere. My objection is not that capitalism is less bad than the Bolsheviks believe, but that Socialism is less good, not in its best form, but in the only form which is likely to be brought about by war. The evils of war, especially of civil war, are certain and very great; the gains to be achieved by victory are problematical. In the course of a desperate struggle, the heritage of civilization is likely to be lost, while hatred, suspicion, and cruelty become normal in the relations of human beings. In order to succeed in war, a concentration of power is necessary, and from concentration of power the very same evils flow as from the capitalist concentration of wealth. For these reasons chiefly, I cannot support any movement which aims at world revolution. The damage to civilization done by revolution in one country may be repaired by the influence of another in which there has been no revolution; but in a universal cataclysm civilization might go under for a thousand years. But while I cannot advocate world revolution, I cannot escape from the conclusion that the Governments of the leading capitalist countries are doing everything to bring it about. Abuse of our power against Germany, Russia, and India (to say nothing of any other countries) may well bring about our downfall, and produce those very evils which the enemies of Bolshevism most dread.

The true Communist is thoroughly international. Lenin, for example, so far as I could judge, is not more concerned with the interests of Russia than with those of other countries; Russia is, at the moment, the protagonist of the social revolution, and, as such, valuable to the world, but Lenin would sacrifice Russia rather than the revolution, if the alternative should ever arise. This is the orthodox attitude, and is no doubt genuine in many of the leaders. But nationalism is natural and instinctive; through pride in the revolution, it grows again even in the breasts of Communists. Through the Polish war, the Bolsheviks have acquired the support of national feeling, and their position in the country has been immensely strengthened.

The only time I saw Trotsky was at the Opera in Moscow. The British Labour Delegation were occupying what had been the Tsar's box. After speaking with us in the ante-chamber, he stepped to the front of the box and stood with folded arms while the house cheered itself hoarse. Then he spoke a few sentences, short and sharp, with military precision, winding up by calling for "three cheers for our brave fellows at the front," to which the audience responded as a London audience would have responded in the autumn of 1914. Trotsky and the Red Army undoubtedly now have behind them a great body of nationalist sentiment. The reconquest of Asiatic Russia has even revived what is essentially an imperialist way of feeling, though this would be indignantly repudiated by many of those in whom I seemed to detect it. Experience of power is inevitably altering Communist theories, and men who control a vast governmental machine can hardly have quite the same outlook on life as they had when they were hunted fugitives. If the Bolsheviks remain in power, it is much to be feared that their Communism will fade, and that they will increasingly resemble any other Asiatic Government---for example, our own Government in India.