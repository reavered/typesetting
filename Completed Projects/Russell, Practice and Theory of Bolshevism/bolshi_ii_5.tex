Is it possible to effect a fundamental reform of the existing economic system by any other method than that of Bolshevism? The difficulty of answering this question is what chiefly attracts idealists to the dictatorship of the proletariat. If, as I have argued, the method of violent revolution and Communist rule is not likely to have the results which idealists desire, we are reduced to despair unless we call see hope in other methods. The Bolshevik arguments against all other methods are powerful. I confess that, when the spectacle of present-day Russia forced me to disbelieve in Bolshevik methods, I was at first unable to see any way of curing the essential evils of capitalism. My first impulse was to abandon political thinking as a bad job, and to conclude that the strong and ruthless must always exploit the weaker and kindlier sections of the population. But this is not an attitude that can be long maintained by any vigorous and temperamentally hopeful person. Of course, if it were the truth, one would have to acquiesce. Some people believe that by living on sour milk one can achieve immortality. Such optimists are answered by a mere refutation; it is not necessary to go on and point out some other way of escaping death. Similarly an argument that Bolshevism will not lead to the millennium would remain valid even if it could be shown that the millennium cannot be reached by any other road. But the truth in social questions is not quite like truth in physiology or physics, since it depends upon men's beliefs. Optimism tends to verify itself by making people impatient of avoidable evils; while despair, on the other hand, makes the world as bad as it believes it to be. It is therefore imperative for those who do not believe in Bolshevism to put some other hope in its place.

I think there are two things that must be admitted: first, that many of the worst evils of capitalism might survive under Communism; secondly, that the cure for these evils cannot be sudden, since it requires changes in the average mentality.

What are the chief evils of the present system? I do not think that mere inequality of wealth, in itself, is a very grave evil. If everybody had enough, the fact that some have more than enough would be unimportant. With a very moderate improvement in methods of production, it would be easy to ensure that everybody should have enough, even under capitalism, if wars and preparations for wars were abolished. The problem of poverty is by no means insoluble within the existing system, except when account is taken of psychological factors and the uneven distribution of power.

The graver evils of the capitalist system all arise from its uneven distribution of power. The possessors of capital wield an influence quite out of proportion to their numbers or their services to the community. They control almost the whole of education and the press; they decide what the average man shall know or not know; the cinema has given them a new method of propaganda, by which they enlist the support of those who are too frivolous even for illustrated papers. Very little of the intelligence of the world is really free: most of it is, directly or indirectly, in the pay of business enterprises or wealthy philanthropists. To satisfy capitalist interests, men are compelled to work much harder and more monotonously than they ought to work, and their education is scamped. Wherever, as in barbarous or semi-civilized countries, labour is too weak or too disorganized to protect itself, appalling cruelties are practised for private profit. Economic and political organizations become more and more vast, leaving less and less room for individual development and initiative. It is this sacrifice of the individual to the machine that is the fundamental evil of the modern world.

To cure this evil is not easy, because efficiency is promoted, at any given moment, though not in the long run, by sacrificing the individual to the smooth working of a vast organization, whether military or industrial. In war and in commercial competition, it is necessary to control individual impulses, to treat men as so many "bayonets" or "sabres" or "hands," not as a society of separate people with separate tastes and capacities. Some sacrifice of individual impulses is, of course, essential to the existence of an ordered community, and this degree of sacrifice is, as a rule, not regretable even from the individual's point of view. But what is demanded in a highly militarized or industrialized nation goes far beyond this very moderate degree. A society which is to allow much freedom to the individual must be strong enough to be not anxious about home defence, moderate enough to refrain from difficult external conquests, and rich enough to value leisure and a civilized existence more than an increase of consumable commodities.

But where the material conditions for such a state of affairs exist, the psychological conditions are not likely to exist unless power is very widely diffused throughout the community. Where power is concentrated in a few, it will happen, unless those few are very exceptional people, that they will value tangible achievements in the way of increase in trade or empire more than the slow and less obvious improvements that would result from better education combined with more leisure. The joys of victory are especially great to the holders of power, while the evils of a mechanical organization fall almost exclusively upon the less influential. For these reasons, I do not believe that any community in which power is much concentrated will long refrain from conflicts of the kind involving a sacrifice of what is most valuable in the individual. In Russia at this moment, the sacrifice of the individual is largely inevitable, because of the severity of the economic and military struggle. But I did not feel, in the Bolsheviks, any consciousness of the magnitude of this misfortune, or any realization of the importance of the individual as against the State. Nor do I believe that men who do realize this are likely to succeed, or to come to the top, in times when everything has to be done against personal liberty. The Bolshevik theory requires that every country, sooner or later, should go through what Russia is going through now. And in every country in such a condition we may expect to find the government falling into the hands of ruthless men, who have not by nature any love for freedom, and who will see little importance in hastening the transition from dictatorship to freedom. It is far more likely that such men will be tempted to embark upon new enterprises, requiring further concentration of forces, and postponing indefinitely the liberation of the populations which they use as their material.

For these reasons, equalization of wealth without equalization of power seems to me a rather small and unstable achievement. But equalization of power is not a thing that can be achieved in a day. It requires a considerable level of moral, intellectual, and technical education. It requires a long period without extreme crises, in order that habits of tolerance and good nature may become common. It requires vigour on the part of those who are acquiring power, without a too desperate resistance on the part of those whose share is diminishing. This is only possible if those who are acquiring power are not very fierce, and do not terrify their opponents by threats of ruin and death. It cannot be done quickly, because quick methods require that very mechanism and subordination of the individual which we should struggle to prevent.

But even equalization of power is not the whole of what is needed politically. The right grouping of men for different purposes is also essential. Self-government in industry, for example, is an indispensable condition of a good society. Those acts of an individual or a group which have no very great importance for outsiders ought to be freely decided by that individual or group. This is recognized as regards religion, but ought to be recognized over a much wider field.

Bolshevik theory seems to me to err by concentrating its attention upon one evil, namely inequality of wealth, which it believes to be at the bottom of all others. I do not believe any one evil can be thus isolated, but if I had to select one as the greatest of political evils, I should select inequality of power. And I should deny that this is likely to be cured by the class-war and the dictatorship of the Communist party. Only peace and a long period of gradual improvement can bring it about.

Good relations between individuals, freedom from hatred and violence and oppression, genera diffusion of education, leisure rationally employed, the progress of art and science---these seem to me among the most important ends that a political theory ought to have in view. I do not believe that they can be furthered, except very rarely, by revolution and war; and I am convinced that at the present moment they can only be promoted by a diminution in the spirit of ruthlessness generated by the war. For these reasons, while admitting the necessity and even utility of Bolshevism in Russia, I do not wish to see it spread, or to encourage the adoption of its philosophy by advanced parties in the Western nations.