The problem of inducing the peasants to feed the towns is one which Russia shares with Central Europe, and from what one hears Russia has been less unsuccessful than some other countries in dealing with this problem. For the Soviet Government, the problem is mainly concentrated in Moscow and Petrograd; the other towns are not very large, and are mostly in the centre of rich agricultural districts. It is true that in the North even the rural population normally depends upon food from more southerly districts; but the northern population is small. It is commonly said that the problem of feeding Moscow and Petrograd is a transport problem, but I think this is only partially true. There is, of course, a grave deficiency of rolling-stock, especially of locomotives in good repair. But Moscow is surrounded by very good land. In the course of a day's motoring in the neighbourhood, I saw enough cows to supply milk to the whole child population of Moscow, although what I had come to see was children's sanatoria, not farms. All kinds of food can be bought in the market at high prices. I travelled over a considerable extent of Russian railways, and saw a fair number of goods trains. For all these reasons, I feel convinced that the share of the transport problem in the food difficulties has been exaggerated. Of course transport plays a larger part in the shortage in Petrograd than in Moscow, because food comes mainly from south of Moscow. In Petrograd, most of the people one sees in the streets show obvious signs of under-feeding. In Moscow, the visible signs are much less frequent, but there is no doubt that under-feeding, though not actual starvation, is nearly universal.

The Government supplies rations to every one who works in the towns at a very low fixed price. The official theory is that the Government has a monopoly of the food and that the rations are sufficient to sustain life. The fact is that the rations are not sufficient, and that they are only a portion of the food supply of Moscow. Moreover, people complain, I do not know how truly, that the rations are delivered irregularly; some say, about every other day. Under these circumstances, almost everybody, rich or poor, buys food in the market, where it costs about fifty times the fixed Government price. A pound of butter costs about a month's wages. In order to be able to afford extra food, people adopt various expedients. Some do additional work, at extra rates, after their official day's work is over. For, though there is supposed to be by law an eight-hours day, extended to ten in certain vital industries, the wage paid for it is not a living wage, and there is nothing to prevent a man from undertaking other work in his spare time. But the usual resource is what is called "speculation," i.e., buying and selling. Some person formerly rich sells clothes or furniture or jewellery in return for food; the buyer sells again at an enhanced price, and so on through perhaps twenty hands, until a final purchaser is found in some well-to-do peasant or \emph{nouveau riche} speculator. Again, most people have relations in the country, whom they visit from time to time, bringing back with them great bags of flour. It is illegal for private persons to bring food into Moscow, and the trains are searched; but, by corruption or cunning, experienced people can elude the search. The food market is illegal, and is raided occasionally; but as a rule it is winked at. Thus the attempt to suppress private commerce has resulted in an amount of unprofessional buying and selling which far exceeds what happens in capitalist countries. It takes up a great deal of time that might be more profitably employed; and, being illegal, it places practically the whole population of Moscow at the mercy of the police. Moreover, it depends largely upon the stores of goods belonging to those who were formerly rich, and when these are expended the whole system must collapse, unless industry has meanwhile been re-established on a sound basis.

It is clear that the state of affairs is unsatisfactory, but, from the Government's point of view, it is not easy to see what ought to be done. The urban and industrial population is mainly concerned in carrying on the work of government and supplying munitions to the army. These are very necessary tasks, the cost of which ought to be defrayed out of taxation. A moderate tax in kind on the peasants would easily feed Moscow and Petrograd. But the peasants take no interest in war or government. Russia is so vast that invasion of one part does not touch another part; and the peasants are too ignorant to have any national consciousness, such as one takes for granted in England or France or Germany. The peasants will not willingly part with a portion of their produce merely for purposes of national defence, but only for the goods they need---clothes, agricultural implements, etc.---which the Government, owing to the war and the blockade, is not in a position to supply.

When the food shortage was at its worst, the Government antagonized the peasants by forced requisitions, carried out with great harshness by the Red Army. This method has been modified, but the peasants still part unwillingly with their food, as is natural in view of the uselessness of paper and the enormously higher prices offered by private buyers.

The food problem is the main cause of popular opposition to the Bolsheviks, yet I cannot see how any popular policy could have been adopted. The Bolsheviks are disliked by the peasants because they take so much food; they are disliked in the towns because they take so little. What the peasants want is what is called free trade, i.e., de-control of agricultural produce. If this policy were adopted, the towns would be faced by utter starvation, not merely by hunger and hardship. It is an entire misconception to suppose that the peasants cherish any hostility to the Entente. The \emph{Daily News} of July 13th, in an otherwise excellent leading article, speaks of "the growing hatred of the Russian peasant, who is neither a Communist nor a Bolshevik, for the Allies generally and this country in particular." The typical Russian peasant has never heard of the Allies or of this country; he does not know that there is a blockade; all he knows is that he used to have six cows but the Government reduced him to one for the sake of poorer peasants, and that it takes his corn (except what is needed for his own family) at a very low price. The reasons for these actions do not interest him, since his horizon is bounded by his own village. To a remarkable extent, each village is an independent unit. So long as the Government obtains the food and soldiers that it requires, it does not interfere, and leaves untouched the old village communism, which is extraordinarily unlike Bolshevism and entirely dependent upon a very primitive stage of culture.

The Government represents the interests of the urban and industrial population, and is, as it were, encamped amid a peasant nation, with whom its relations are rather diplomatic and military than governmental in the ordinary sense. The economic situation, as in Central Europe, is favourable to the country and unfavourable to the towns. If Russia were governed democratically, according to the will of the majority, the inhabitants of Moscow and Petrograd would die of starvation. As it is, Moscow and Petrograd just manage to live, by having the whole civil and military power of the State devoted to their needs. Russia affords the curious spectacle of a vast and powerful Empire, prosperous at the periphery, but faced with dire want at the centre. Those who have least prosperity have most power; and it is only through their excess of power that they are enabled to live at all. The situation is due at bottom to two facts: that almost the whole industrial energies of the population have had to be devoted to war, and that the peasants do not appreciate the importance of the war or the fact of the blockade.

It is futile to blame the Bolsheviks for an unpleasant and difficult situation which it has been impossible for them to avoid. Their problem is only soluble in one of two ways: by the cessation of the war and the blockade, which would enable them to supply the peasants with the goods they need in exchange for food; or by the gradual development of an independent Russian industry. This latter method would be slow, and would involve terrible hardships, but some of the ablest men in the Government believe it to be possible if peace cannot be achieved. If we force this method upon Russia by the refusal of peace and trade, we shall forfeit the only inducement we can hold out for friendly relations; we shall render the Soviet State unassailable and completely free to pursue the policy of promoting revolution everywhere. But the industrial problem is a large subject, which has been already discussed in Chapter VI.