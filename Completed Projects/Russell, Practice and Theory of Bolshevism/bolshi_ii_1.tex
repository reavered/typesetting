The materialistic conception of history, as it is called, is due to Marx, and underlies the whole Communist philosophy. I do not mean, of course, that a man could not be a Communist without accepting it, but that in fact it is accepted by the Communist Party, and that it profoundly influences their views as to politics and tactics. The name does not convey at all accurately what is meant by the theory. It means that all the mass-phenomena of history are determined by economic motives. This view has no essential connection with materialism in the philosophic sense. Materialism in the philosophic sense may be defined as the theory that all apparently mental occurrences either are really physical, or at any rate have purely physical causes. Materialism in this sense also was preached by Marx, and is accepted by all orthodox Marxians. The arguments for and against it are long and complicated, and need not concern us, since, in fact, its truth or falsehood has little or no bearing on politics.

In particular, philosophic materialism does not prove that economic causes are fundamental in politics. The view of Buckle, for example, according to which climate is one of the decisive factors, is equally compatible with materialism. So is the Freudian view, which traces everything to sex. There are innumerable ways of viewing history which are materialistic in the philosophic sense without being economic or falling within the Marxian formula. Thus the "materialistic conception of history" may be false even if materialism in the philosophic sense should be true.

On the other hand, economic causes might be at the bottom of all political events even if philosophic materialism were false. Economic causes operate through men's desire for possessions, and would be supreme if this desire were supreme, even if desire could not, from a philosophic point of view, be explained in materialistic terms.

There is, therefore, no logical connection either way between philosophic materialism and what is called the "materialistic conception of history."

It is of some moment to realize such facts as this, because otherwise political theories are both supported and opposed for quite irrelevant reasons, and arguments of theoretical philosophy are employed to determine questions which depend upon concrete facts of human nature. This mixture damages both philosophy and politics, and is therefore important to avoid.

For another reason, also, the attempt to base a political theory upon a philosophical doctrine is undesirable. The philosophical doctrine of materialism, if true at all, is true everywhere and always; we cannot expect exceptions to it, say, in Buddhism or in the Hussite movement. And so it comes about that people whose politics are supposed to be a consequence of their metaphysics grow absolute and sweeping, unable to admit that a general theory of history is likely, at best, to be only true on the whole and in the main. The dogmatic character of Marxian Communism finds support in the supposed philosophic basis of the doctrine; it has the fixed certainty of Catholic theology, not the changing fluidity and sceptical practicality of modern science.

Treated as a practical approximation, not as an exact metaphysical law, the materialistic conception of history has a very large measure of truth. Take, as an instance of its truth, the influence of industrialism upon ideas. It is industrialism, rather than the arguments of Darwinians and Biblical critics, that has led to the decay of religious belief in the urban working class. At the same time, industrialism has revived religious belief among the rich. In the eighteenth century French aristocrats mostly became free-thinkers; now their descendants are mostly Catholics, because it has become necessary for all the forces of reaction to unite against the revolutionary proletariat. Take, again, the emancipation of women. Plato, Mary Wolstonecraft, and John Stuart Mill produced admirable arguments, but influenced only a few impotent idealists. The war came, leading to the employment of women in industry on a large scale, and instantly the arguments in favour of votes for women were seen to be irresistible. More than that, traditional sexual morality collapsed, because its whole basis was the economic dependence of women upon their fathers and husbands. Changes in such a matter as sexual morality bring with them profound alterations in the thoughts and feelings of ordinary men and women; they modify law, literature, art, and all kinds of institutions that seem remote from economics.

Such facts as these justify Marxians in speaking, as they do, of "bourgeois ideology," meaning that kind of morality which has been imposed upon the world by the possessors of capital. Contentment with one's lot may be taken as typical of the virtues preached by the rich to the poor. They honestly believe it is a virtue---at any rate they did formerly. The more religious among the poor also believed it, partly from the influence of authority, partly from an impulse to submission, what MacDougall calls "negative self-feeling," which is commoner than some people think. Similarly men preached the virtue of female chastity, and women usually accepted their teaching; both really believed the doctrine, but its persistence was only possible through the economic power of men. This led erring women to punishment here on earth, which made further punishment hereafter seem probable. When the economic penalty ceased, the conviction of sinfulness gradually decayed. In such changes we see the collapse of "bourgeois ideology."

But in spite of the fundamental importance of economic facts in determining the politics and beliefs of an age or nation, I do not think that non-economic factors can be neglected without risks of errors which may be fatal in practice.

The most obvious non-economic factor, and the one the neglect of which has led Socialists most astray, is nationalism. Of course a nation, once formed, has economic interests which largely determine its politics; but it is not, as a rule, economic motives that decide what group of human beings shall form a nation. Trieste, before the war, considered itself Italian, although its whole prosperity as a port depended upon its belonging to Austria. No economic motive can account for the opposition between Ulster and the rest of Ireland. In Eastern Europe, the Balkanization produced by self-determination has been obviously disastrous from an economic point of view, and was demanded for reasons which were in essence sentimental. Throughout the war wage-earners, with only a few exceptions, allowed themselves to be governed by nationalist feeling, and ignored the traditional Communist exhortation: "Workers of the world, unite." According to Marxian orthodoxy, they were misled by cunning capitalists, who made their profit out of the slaughter. But to any one capable of observing psychological facts, it is obvious that this is largely a myth. Immense numbers of capitalists were ruined by the war; those who were young were just as liable to be killed as the proletarians were. No doubt commercial rivalry between England and Germany had a great deal to do with causing the war; but rivalry is a different thing from profit-seeking. Probably by combination English and German capitalists could have made more than they did out of rivalry, but the rivalry was instinctive, and its economic form was accidental. The capitalists were in the grip of nationalist instinct as much as their proletarian "dupes." In both classes some have gained by the war; but the universal will to war was not produced by the hope of gain. It was produced by a different set of instincts, and one which Marxian psychology fails to recognize adequately.

The Marxian assumes that a man's "herd," from the point of view of herd-instinct, is his class, and that he will combine with those whose economic class-interest is the same as his. This is only very partially true in fact. Religion has been the most decisive factor in determining a man's herd throughout long periods of the world's history. Even now a Catholic working man will vote for a Catholic capitalist rather than for an unbelieving Socialist. In America the divisions in local elections are mainly on religious lines. This is no doubt convenient for the capitalists, and tends to make them religious men; but the capitalists alone could not produce the result. The result is produced by the fact that many working men prefer the advancement of their creed to the improvement of their livelihood. However deplorable such a state of mind may be, it is not necessarily due to capitalist lies.

All politics are governed by human desires. The materialist theory of history, in the last analysis, requires the assumption that every politically conscious person is governed by one single desire---the desire to increase his own share of commodities; and, further, that his method of achieving this desire will usually be to seek to increase the share of his class, not only his own individual share. But this assumption is very far from the truth. Men desire power, they desire satisfactions for their pride and their self-respect. They desire victory over rivals so profoundly that they will invent a rivalry for the unconscious purpose of making a victory possible. All these motives cut across the pure economic motive in ways that are practically important.

There is need of a treatment of political motives by the methods of psycho-analysis. In politics, as in private life, men invent myths to rationalize their conduct. If a man thinks that the only reasonable motive in politics is economic self-advancement, he will persuade himself that the things he wishes to do will make him rich. When he wants to fight the Germans, he tells himself that their competition is ruining his trade. If, on the other hand, he is an "idealist," who holds that his politics should aim at the advancement of the human race, he will tell himself that the crimes of the Germans demand their humiliation. The Marxian sees through this latter camouflage, but not through the former. To desire one's own economic advancement is comparatively reasonable; to Marx, who inherited eighteenth-century rationalist psychology from the British orthodox economists, self-enrichment seemed the natural aim of a man's political actions. But modern psychology has dived much deeper into the ocean of insanity upon which the little barque of human reason insecurely floats. The intellectual optimism of a bygone age is no longer possible to the modern student of human nature. Yet it lingers in Marxism, making Marxians rigid and Procrustean in their treatment of the life of instinct. Of this rigidity the materialistic conception of history is a prominent instance.

In the next chapter I shall attempt to outline a political psychology which seems to me more nearly true than that of Marx.