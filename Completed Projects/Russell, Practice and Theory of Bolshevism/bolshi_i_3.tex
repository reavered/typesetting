Soon after my arrival in Moscow I had an hour's conversation with Lenin in English, which he speaks fairly well. An interpreter was present, but his services were scarcely required. Lenin's room is very bare; it contains a big desk, some maps on the walls, two book-cases, and one comfortable chair for visitors in addition to two or three hard chairs. It is obvious that he has no love of luxury or even comfort. He is very friendly, and apparently simple, entirely without a trace of \emph{hauteur}. If one met him without knowing who he was, one would not guess that he is possessed of great power or even that he is in any way eminent. I have never met a personage so destitute of self-importance. He looks at his visitors very closely, and screws up one eye, which seems to increase alarmingly the penetrating power of the other. He laughs a great deal; at first his laugh seems merely friendly and jolly, but gradually I came to feel it rather grim. He is dictatorial, calm, incapable of fear, extraordinarily devoid of self-seeking, an embodied theory. The materialist conception of history, one feels, is his life-blood. He resembles a professor in his desire to have the theory understood and in his fury with those who misunderstand or disagree, as also in his love of expounding, I got the impression that he despises a great many people and is an intellectual aristocrat.

The first question I asked him was as to how far he recognized the peculiarity of English economic and political conditions? I was anxious to know whether advocacy of violent revolution is an indispensable condition of joining the Third International, although I did not put this question directly because others were asking it officially. His answer was unsatisfactory to me. He admitted that there is little chance of revolution in England now, and that the working man is not yet disgusted with Parliamentary government. But he hopes that this result may be brought about by a Labour Ministry. He thinks that, if Mr. Henderson, for instance, were to become Prime Minister, nothing of importance would be done; organized Labour would then, so he hopes and believes, turn to revolution. On this ground, he wishes his supporters in this country to do everything in their power to secure a Labour majority in Parliament; he does not advocate abstention from Parliamentary contests, but participation with a view to making Parliament obviously contemptible. The reasons which make attempts at violent revolution seem to most of us both improbable and undesirable in this country carry no weight with him, and seem to him mere \emph{bourgeois} prejudices. When I suggested that whatever is possible in England can be achieved without bloodshed, he waved aside the suggestion as fantastic. I got little impression of knowledge or psychological imagination as regards Great Britain. Indeed the whole tendency of Marxianism is against psychological imagination, since it attributes everything in politics to purely material causes.

I asked him next whether he thought it possible to establish Communism firmly and fully in a country containing such a large majority of peasants. He admitted that it was difficult, and laughed over the exchange the peasant is compelled to make, of food for paper; the worthlessness of Russian paper struck him as comic. But he said---what is no doubt true---that things will right themselves when there are goods to offer to the peasant. For this he looks partly to electrification in industry, which, he says, is a technical necessity in Russia, but will take ten years to complete.\footnote{Electrification is desired not merely for reorganizing industry, but in order to industrialize agriculture. In \emph{Theses presented to the Second Congress of the Communist International} (an instructive little book, which I shall quote as \emph{Theses}), it is said in an article on the Agrarian question that Socialism will not be secure till industry is reorganized on a new basis with "general application of electric energy in all branches of agriculture and rural economy," which "alone can give to the towns the possibility of offering to backward rural districts a technical and social aid capable of determining an extraordinary increase of productivity of agricultural and rural labour, and of engaging the small cultivators, in their own interest, to pass progressively to a collectivist mechanical cultivation" (p. 36 of French edition).} He spoke with enthusiasm, as they all do, of the great scheme for generating electrical power by means of peat. Of course he looks to the raising of the blockade as the only radical cure; but he was not very hopeful of this being achieved thoroughly or permanently except through revolutions in other countries. Peace between Bolshevik Russia and capitalist countries, he said, must always be insecure; the Entente might be led by weariness and mutual dissensions to conclude peace, but he felt convinced that the peace would be of brief duration. I found in him, as in almost all leading Communists, much less eagerness than existed in our delegation for peace and the raising of the blockade. He believes that nothing of real value can be achieved except through world revolution and the abolition of capitalism; I felt that he regarded the resumption of trade with capitalist countries as a mere palliative of doubtful value.

He described the division between rich and poor peasants, and the Government propaganda among the latter against the former, leading to acts of violence which he seemed to find amusing. He spoke as though the dictatorship over the peasant would have to continue a long time, because of the peasant's desire for free trade. He said he knew from statistics (what I can well believe) that the peasants have had more to eat these last two years than they ever had before, "and yet they are against us," he added a little wistfully. I asked him what to reply to critics who say that in the country he has merely created peasant proprietorship, not Communism; he replied that that is not quite the truth, but he did not say what the truth is.\footnote{In \emph{Theses} (p. 34) it is said: "It would be an irreparable error\ldots not to admit the gratuitous grant of part of the expropriated lands to poor and even well-to-do peasants."}

The last question I asked him was whether resumption of trade with capitalist countries, if it took place, would not create centres of capitalist influence, and make the preservation of Communism more difficult? It had seemed to me that the more ardent Communists might well dread commercial intercourse with the outer world, as leading to an infiltration of heresy, and making the rigidity of the present system almost impossible. I wished to know whether he had such a feeling. He admitted that trade would create difficulties, but said they would be less than those of the war. He said that two years ago neither he nor his colleagues thought they could survive against the hostility of the world. He attributes their survival to the jealousies and divergent interests of the different capitalist nations; also to the power of Bolshevik propaganda. He said the Germans had laughed when the Bolsheviks proposed to combat guns with leaflets, but that the event had proved the leaflets quite as powerful. I do not think he recognizes that the Labour and Socialist parties have had any part in the matter. He does not seem to know that the attitude of British Labour has done a great deal to make a first-class war against Russia impossible, since it has confined the Government to what could be done in a hole-and-corner way, and denied without a too blatant mendacity.

He thoroughly enjoys the attacks of Lord Northcliffe, to whom he wishes to send a medal for Bolshevik propaganda. Accusations of spoliation, he remarked, may shock the \emph{bourgeois}, but have an opposite effect upon the proletarian.

I think if I had met him without knowing who he was, I should not have guessed that he was a great man; he struck me as too opinionated and narrowly orthodox. His strength comes, I imagine, from his honesty, courage, and unwavering faith---religious faith in the Marxian gospel, which takes the place of the Christian martyr's hopes of Paradise, except that it is less egotistical. He has as little love of liberty as the Christians who suffered under Diocletian, and retaliated when they acquired power. Perhaps love of liberty is incompatible with whole-hearted belief in a panacea for all human ills. If so, I cannot but rejoice in the sceptical temper of the Western world. I went to Russia a Communist; but contact with those who have no doubts has intensified a thousandfold my own doubts, not as to Communism in itself, but as to the wisdom of holding a creed so firmly that for its sake men are willing to inflict widespread misery.

Trotsky, whom the Communists do not by any means regard as Lenin's equal, made more impression upon me from the point of view of intelligence and personality, though not of character. I saw too little of him, however, to have more than a very superficial impression. He has bright eyes, military bearing, lightning intelligence and magnetic personality. He is very good-looking, with admirable wavy hair; one feels he would be irresistible to women. I felt in him a vein of gay good humour, so long as he was not crossed in any way. I thought, perhaps wrongly, that his vanity was even greater than his love of power---the sort of vanity that one associates with an artist or actor. The comparison with Napoleon was forced upon one. But I had no means of estimating the strength of his Communist conviction, which may be very sincere and profound.

An extraordinary contrast to both these men was Gorky, with whom I had a brief interview in Petrograd. He was in bed, apparently very ill and obviously heart-broken. He begged me, in anything I might say about Russia, always to emphasize what Russia has suffered. He supports the Government---as I should do, if I were a Russian---not because he thinks it faultless, but because the possible alternatives are worse. One felt in him a love of the Russian people which makes their present martyrdom almost unbearable, and prevents the fanatical faith by which the pure Marxians are upheld. I felt him the most lovable, and to me the most sympathetic, of all the Russians I saw. I wished for more knowledge of his outlook, but he spoke with difficulty and was constantly interrupted by terrible fits of coughing, so that I could not stay. All the intellectuals whom I met---a class who have suffered terribly---expressed their gratitude to him for what he has done on their behalf. The materialistic conception of history is all very well, but some care for the higher things of civilization is a relief. The Bolsheviks are sometimes said to have done great things for art, but I could not discover that they had done more than preserve something of what existed before. When I questioned one of them on the subject, he grew impatient, and said: "We haven't time for a new art, any more than for a new religion." Unavoidably, although the Government favours art as much as it can, the atmosphere is one in which art cannot flourish, because art is anarchic and resistant to organization. Gorky has done all that one man could to preserve the intellectual and artistic life of Russia. I feared that he was dying, and that, perhaps, it was dying too. But he recovered, and I hope it will recover also.