Daily life in Moscow, so far as I could discover, has neither the horrors depicted by the Northcliffe Press nor the delights imagined by the more ardent of our younger Socialists.

On the one hand, there is no disorder, very little crime, not much insecurity for those who keep clear of politics. Everybody works hard; the educated people have, by this time, mostly found their way into Government offices or teaching or some other administrative profession in which their education is useful. The theatres, the opera and the ballet continue as before, and are quite admirable; some of the seats are paid for, others are given free to members of trade unions. There is, of course, no drunkenness, or at any rate so little that none of us ever saw a sign of it. There is very little prostitution, infinitely less than in any other capital. Women are safer from molestation than anywhere else in the world. The whole impression is one of virtuous, well-ordered activity.

On the other hand, life is very hard for all except men in good posts. It is hard, first of all, owing to the food shortage. This is familiar to all who have interested themselves in Russia, and it is unnecessary to dwell upon it. What is less realized is that most people work much longer hours than in this country. The eight-hour day was introduced with a flourish of trumpets; then, owing to the pressure of the war, it was extended to ten hours in certain trades. But no provision exists against extra work at other jobs, and very many people do extra work, because the official rates do not afford a living wage. This is not the fault of the Government, at any rate as regards the major part; it is due chiefly to war and blockade. When the day's work is over, a great deal of time has to be spent in fetching food and water and other necessaries of life. The sight of the workers going to and fro, shabbily clad, with the inevitable bundle in one hand and tin can in the other, through streets almost entirely empty of traffic, produces the effect of life in some vast village, rather than in an important capital city.

Holidays, such as are common throughout all but the very poorest class in this country, are very difficult in Russia. A train journey requires a permit, which is only granted on good reasons being shown; with the present shortage of transport, this regulation is quite unavoidable. Railway queues are a common feature in Moscow; it often takes several days to get a permit. Then, when it has been obtained, it may take several more days to get a seat in a train. The ordinary trains are inconceivably crowded, far more so, though that seems impossible, than London trains at the busiest hour. On the shorter journeys, passengers are even known to ride on the roof and buffers, or cling like flies to the sides of the waggons. People in Moscow travel to the country whenever they can afford the time and get a permit, because in the country there is enough to eat. They go to stay with relations---most people in Moscow, in all classes, but especially among manual workers, have relations in the country. One cannot, of course, go to an hotel as one would in other countries. Hotels have been taken over by the State, and the rooms in them (when they are still used) are allocated by the police to people whose business is recognized as important by the authorities. Casual travel is therefore impossible even on a holiday.

Journeys have vexations in addition to the slowness and overcrowding of the trains. Police search the travellers for evidences of "speculation," especially for food. The police play, altogether, a much greater part in daily life than they do in other countries---much greater than they did, for example, in Prussia twenty-five years ago, when there was a vigorous campaign against Socialism. Everybody breaks the law almost daily, and no one knows which among his acquaintances is a spy of the Extraordinary Commission. Even in the prisons, among prisoners, there are spies, who are allowed certain privileges but not their liberty.

Newspapers are not taken in, except by very few people, but they are stuck up in public places, where passers-by occasionally glance at them.\footnote{The ninth Communist Congress (March-April, 1920) says on this subject: 
\begin{displayquote}
In view of the fact that the first condition of the success of the Soviet Republic in all departments, including the economic, is chiefly systematic printed agitation, the Congress draws the attention of the Soviet Government to the deplorable state in which our paper and printing industries find themselves. The ever decreasing number of newspapers fail to reach not only the peasants but even the workers, in addition to which our poor technical means render the papers hardly readable. The Congress strongly appeals to the Supreme Council of Public Economy, to the corresponding Trade Unions and other interested institutions, to apply all efforts to raise the quantity, to introduce general system and order in the printing business, and so secure for the worker and peasant in Russia a supply of Socialist printed matter. \end{displayquote}} There is very little to read; owing to paper shortage, books are rare, and money to buy them is still rarer. One does not see people reading, as one does here in the Underground for example. There is practically no social life, partly because of the food shortage, partly because, when anybody is arrested, the police are apt to arrest everybody whom they find in his company, or who comes to visit him. And once arrested, a man or woman, however innocent, may remain for months in prison without trial. While we were in Moscow, forty social revolutionaries and Anarchists were hunger-striking to enforce their demand to be tried and to be allowed visits. I was told that on the eighth day of the strike the Government consented to try them, and that few could be proved guilty of any crime; but I had no means of verifying this.

Industrial conscription is, of course, rigidly enforced. Every man and woman has to work, and slacking is severely punished, by prison or a penal settlement. Strikes are illegal, though they sometimes occur. By proclaiming itself the friend of the proletarian, the Government has been enabled to establish an iron discipline, beyond the wildest dreams of the most autocratic American magnate. And by the same professions the Government has led Socialists from other countries to abstain from reporting unpleasant features in what they have seen.

The Tolstoyans, of whom I saw the leaders, are obliged by their creed to resist every form of conscription, though some have found ways of compromising. The law concerning conscientious objectors to military service is practically the same as ours, and its working depends upon the temper of the tribunal before which a man comes. Some conscientious objectors have been shot; on the other hand, some have obtained absolute exemption.

Life in Moscow, as compared to life in London, is drab, monotonous, and depressed. I am not, of course, comparing life there with that of the rich here, but with that of the average working-class family. When it is realized that the highest wages are about fifteen shillings a month, this is not surprising. I do not think that life could, under any system, be very cheerful in a country so exhausted by war as Russia, so I am not saying this as a criticism of the Bolsheviks. But I do think there might be less police interference, less vexatious regulation, and more freedom for spontaneous impulses towards harmless enjoyments.

Religion is still very strong. I went into many churches, where I saw obviously famished priests in gorgeous vestments, and a congregation enormously devout. Generally more than half the congregation were men, and among the men many were soldiers. This applies to the towns as well as to the country. In Moscow I constantly saw people in the streets crossing themselves.

There is a theory that the Moscow working man feels himself free from capitalist domination, and therefore bears hardships gladly. This is no doubt true of the minority who are active Communists, but I do not think it has any truth for the others. The average working man, to judge by a rather hasty impression, feels himself the slave of the Government, and has no sense whatever of having been liberated from a tyranny.

I recognize to the full the reasons for the bad state of affairs, in the past history of Russia and the recent policy of the Entente. But I have thought it better to record impressions frankly, trusting the readers to remember that the Bolsheviks have only a very limited share of responsibility for the evils from which Russia is suffering.