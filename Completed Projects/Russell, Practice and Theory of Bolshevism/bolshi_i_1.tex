To understand Bolshevism it is not sufficient to know facts; it is necessary also to enter with sympathy or imagination into a new spirit. The chief thing that the Bolsheviks have done is to create a hope, or at any rate to make strong and widespread a hope which was formerly confined to a few. This aspect of the movement is as easy to grasp at a distance as it is in Russia--perhaps even easier, because in Russia present circumstances tend to obscure the view of the distant future. But the actual situation in Russia can only be understood superficially if we forget the hope which is the motive power of the whole. One might as well describe the Thebaid without mentioning that the hermits expected eternal bliss as the reward of their sacrifices here on earth.

I cannot share the hopes of the Bolsheviks any more than those of the Egyptian anchorites; I regard both as tragic delusions, destined to bring upon the world centuries of darkness and futile violence. The principles of the Sermon on the Mount are admirable, but their effect upon average human nature was very different from what was intended. Those who followed Christ did not learn to love their enemies or to turn the other cheek. They learned instead to use the Inquisition and the stake, to subject the human intellect to the yoke of an ignorant and intolerant priesthood, to degrade art and extinguish science for a thousand years. These were the inevitable results, not of the teaching, but of fanatical belief in the teaching. The hopes which inspire Communism are, in the main, as admirable as those instilled by the Sermon on the Mount, but they are held as fanatically, and are likely to do as much harm. Cruelty lurks in our instincts, and fanaticism is a camouflage for cruelty. Fanatics are seldom genuinely humane, and those who sincerely dread cruelty will be slow to adopt a fanatical creed. I do not know whether Bolshevism can be prevented from acquiring universal power. But even if it cannot, I am persuaded that those who stand out against it, not from love of ancient injustice, but in the name of the free spirit of Man, will be the bearers of the seeds of progress, from which, when the world's gestation is accomplished, new life will be born.

The war has left throughout Europe a mood of disillusionment and despair which calls aloud for a new religion, as the only force capable of giving men the energy to live vigorously. Bolshevism has supplied the new religion. It promises glorious things: an end of the injustice of rich and poor, an end of economic slavery, an end of war. It promises an end of the disunion of classes which poisons political life and threatens our industrial system with destruction. It promises an end to commercialism, that subtle falsehood that leads men to appraise everything by its money value, and to determine money value often merely by the caprices of idle plutocrats. It promises a world where all men and women shall be kept sane by work, and where all work shall be of value to the community, not only to a few wealthy vampires. It is to sweep away listlessness and pessimism and weariness and all the complicated miseries of those whose circumstances allow idleness and whose energies are not sufficient to force activity. In place of palaces and hovels, futile vice and useless misery, there is to be wholesome work, enough but not too much, all of it useful, performed by men and women who have no time for pessimism and no occasion for despair.

The existing capitalist system is doomed. Its injustice is so glaring that only ignorance and tradition could lead wage-earners to tolerate it. As ignorance diminishes, tradition becomes weakened, and the war destroyed the hold upon men's minds of everything merely traditional. It may be that, through the influence of America, the capitalist system will linger for another fifty years; but it will grow continually weaker, and can never recover the position of easy dominance which it held in the nineteenth century. To attempt to bolster it up is a useless diversion of energies which might be expended upon building something new. Whether the new thing will be Bolshevism or something else, I do not know; whether it will be better or worse than capitalism, I do not know. But that a radically new order of society will emerge, I feel no doubt. And I also feel no doubt that the new order will be either some form of Socialism or a reversion to barbarism and petty war such as occurred during the barbarian invasion. If Bolshevism remains the only vigorous and effective competitor of capitalism, I believe that no form of Socialism will be realized, but only chaos and destruction. This belief, for which I shall give reasons later, is one of the grounds upon which I oppose Bolshevism. But to oppose it from the point of view of a supporter of capitalism would be, to my mind, utterly futile and against the movement of history in the present age.

The effect of Bolshevism as a revolutionary hope is greater outside Russia than within the Soviet Republic. Grim realities have done much to kill hope among those who are subject to the dictatorship of Moscow. Yet even within Russia, the Communist party, in whose hands all political power is concentrated, still lives by hope, though the pressure of events has made the hope severe and stern and somewhat remote. It is this hope that leads to concentration upon the rising generation. Russian Communists often avow that there is little hope for those who are already adult, and that happiness can only come to the children who have grown up under the new régime and been moulded from the first to the group-mentality that Communism requires. It is only after the lapse of a generation that they hope to create a Russia that shall realize their vision.

In the Western World, the hope inspired by Bolshevism is more immediate, less shot through with tragedy. Western Socialists who have visited Russia have seen fit to suppress the harsher features of the present régime, and have disseminated a belief among their followers that the millennium would be quickly realized there if there were no war and no blockade. Even those Socialists who are not Bolsheviks for their own country have mostly done very little to help men in appraising the merits or demerits of Bolshevik methods. By this lack of courage they have exposed Western Socialism to the danger of becoming Bolshevik through ignorance of the price that has to be paid and of the uncertainty as to whether the desired goal will be reached in the end. I believe that the West is capable of adopting less painful and more certain methods of reaching Socialism than those that have seemed necessary in Russia. And I believe that while some forms of Socialism are immeasurably better than capitalism, others are even worse. Among those that are worse I reckon the form which is being achieved in Russia, not only in itself, but as a more insuperable barrier to further progress.

In judging of Bolshevism from what is to be seen in Russia at present, it is necessary to disentangle various factors which contribute to a single result. To begin with, Russia is one of the nations that were defeated in the war; this has produced a set of circumstances resembling those found in Germany and Austria. The food problem, for example, appears to be essentially similar in all three countries. In order to arrive at what is specifically Bolshevik, we must first eliminate what is merely characteristic of a country which has suffered military disaster. Next we come to factors which are Russian, which Russian Communists share with other Russians, but not with other Communists. There is, for example, a great deal of disorder and chaos and waste, which shocks Westerners (especially Germans) even when they are in close political sympathy with the Bolsheviks. My own belief is that, although, with the exception of a few very able men, the Russian Government is less efficient in organization than the Germans or the Americans would be in similar circumstances, yet it represents what is most efficient in Russia, and does more to prevent chaos than any possible alternative government would do. Again, the intolerance and lack of liberty which has been inherited from the Tsarist régime is probably to be regarded as Russian rather than Communist. If a Communist Party were to acquire power in England, it would probably be met by a less irresponsible opposition, and would be able to show itself far more tolerant than any government can hope to be in Russia if it is to escape assassination. This, however, is a matter of degree. A great part of the despotism which characterizes the Bolsheviks belongs to the essence of their social philosophy, and would have to be reproduced, even if in a milder form, wherever that philosophy became dominant.

It is customary among the apologists of Bolshevism in the West to excuse its harshness on the ground that it has been produced by the necessity of fighting the Entente and its mercenaries. Undoubtedly it is true that this necessity has produced many of the worst elements in the present state of affairs. Undoubtedly, also, the Entente has incurred a heavy load of guilt by its peevish and futile opposition. But the expectation of such opposition was always part of Bolshevik theory. A general hostility to the first Communist State was both foreseen and provoked by the doctrine of the class war. Those who adopt the Bolshevik standpoint must reckon with the embittered hostility of capitalist States; it is not worth while to adopt Bolshevik methods unless they can lead to good in spite of this hostility. To say that capitalists are wicked and we have no responsibility for their acts is unscientific; it is, in particular, contrary to the Marxian doctrine of economic determinism. The evils produced in Russia by the enmity of the Entente are therefore to be reckoned as essential in the Bolshevik method of transition to Communism, not as specially Russian. I am not sure that we cannot even go a step further. The exhaustion and misery caused by unsuccessful war were necessary to the success of the Bolsheviks; a prosperous population will not embark by such methods upon a fundamental economic reconstruction. One can imagine England becoming Bolshevik after an unsuccessful war involving the loss of India--no improbable contingency in the next few years. But at present the average wage-earner in England will not risk what he has for the doubtful gain of a revolution. A condition of widespread misery may, therefore, be taken as indispensable to the inauguration of Communism, unless, indeed, it were possible to establish Communism more or less peacefully, by methods which would not, even temporarily, destroy the economic life of the country. If the hopes which inspired Communism at the start, and which still inspire its Western advocates, are ever to be realized, the problem of minimizing violence in the transition must be faced. Unfortunately, violence is in itself delightful to most really vigorous revolutionaries, and they feel no interest in the problem of avoiding it as far as possible. Hatred of enemies is easier and more intense than love of friends. But from men who are more anxious to injure opponents than to benefit the world at large no great good is to be expected.