In the course of these chapters, I have had occasion to mention disagreeable features of the Bolshevik régime. But it must always be remembered that these are chiefly due to the fact that the industrial life of Russia has been paralysed except as ministering to the wants of the Army, and that the Government has had to wage a bitter and doubtful civil and external war, involving the constant menace of domestic enemies. Harshness, espionage, and a curtailment of liberty result unavoidably from these difficulties. I have no doubt whatever that the sole cure for the evils from which Russia is suffering is peace and trade. Peace and trade would put an end to the hostility of the peasants, and would at once enable the Government to depend upon popularity rather than force. The character of the Government would alter rapidly under such conditions. Industrial conscription, which is now rigidly enforced, would become unnecessary. Those who desire a more liberal spirit would be able to make their voices heard without the feeling that they were assisting reaction and the national enemies. The food difficulties would cease, and with them the need for an autocratic system in the towns.

It must not be assumed, as is common with opponents of Bolshevism, that any other Government could easily be established in Russia. I think every one who has been in Russia recently is convinced that the existing Government is stable. It may undergo internal developments, and might easily, but for Lenin, become a Bonapartist military autocracy. But this would be a change from within---not perhaps a very great change---and would probably do little to alter the economic system. From what I saw of the Russian character and of the opposition parties, I became persuaded that Russia is not ready for any form of democracy, and needs a strong Government. The Bolsheviks represent themselves as the Allies of Western advanced Socialism, and from this point of view they are open to grave criticism. For their international programme there is, to my mind, nothing to be said. But as a national Government, stripped of their camouflage, regarded as the successors of Peter the Great, they are performing a necessary though unamiable task. They are introducing, as far as they can, American efficiency among a lazy and undisciplined population. They are preparing to develop the natural resources of their country by the methods of State Socialism, for which, in Russia, there is much to be said. In the Army they are abolishing illiteracy, and if they had peace they would do great things for education everywhere.

But if we continue to refuse peace and trade, I do not think the Bolsheviks will go under. Russia will endure great hardships, in the years to come as before. But the Russians are inured to misery as no Western nation is; they can live and work under conditions which we should find intolerable. The Government will be driven more and more, from mere self-preservation, into a policy of imperialism. The Entente has been doing everything to expose Germany to a Russian invasion of arms and leaflets, by allowing Poland to engage in war and compelling Germany to disarm. All Asia lies open to Bolshevik ambitions. Almost the whole of the former Russian Empire in Asia is quite firmly in their grasp. Trains are running at a reasonable speed to Turkestan, and I saw cotton from there being loaded on to Volga steamers. In Persia and Turkey, revolts are taking place, with Bolshevik support. It is only a question of a few years before India will be in touch with the Red Army. If we continue to antagonize the Bolsheviks, I do not see what force exists that can prevent them from acquiring the whole of Asia within ten years.

The Russian Government is not yet definitely imperialistic in spirit, and would still prefer peace to conquest. The country is weary of war and denuded of goods. But if the Western Powers insist upon war, another spirit, which is already beginning to show itself, will become dominant. Conquest will be the only alternative to submission. Asiatic conquest will not be difficult. But for us, from the imperialist standpoint, it will mean utter ruin. And for the Continent it will mean revolutions, civil wars, economic cataclysms. The policy of crushing Bolshevism by force was always foolish and criminal; it has now become impossible and fraught with disaster. Our own Government, it would seem, have begun to realize the dangers, but apparently they do not realize them sufficiently to enforce their view against opposition.

In the Theses presented to the Second Congress of the Third International (July 1920), there is a very interesting article by Lenin called "First Sketch of the Theses on National and Colonial Questions" (\emph{Theses} pp. 40-47). The following passages seemed to me particularly illuminating:---
\begin{displayquote}
The present world-situation in politics places on the order of the day the dictatorship of the proletariat; and all the events of world politics are inevitably concentrated round one centre of gravity: the struggle of the international bourgeoisie against the Soviet Republic, which inevitably groups round it, on the one hand the Sovietist movements of the advanced working men of all countries, on the other hand all the national movements of emancipation of colonies and oppressed nations which have been convinced by a bitter experience that there is no salvation for them except in the victory of the Soviet Government over world-imperialism.

We cannot therefore any longer confine ourselves to recognizing and proclaiming the union of the workers of all countries. It is henceforth necessary to pursue the realization of the strictest union of all the national and colonial movements of emancipation with Soviet Russia, by giving to this union forms corresponding to the degree of evolution of the proletarian movement among the proletariat of each country, or of the democratic-bourgeois movement of emancipation among the workers and peasants of backward countries or backward nationalities.

The federal principle appears to us as a transitory form towards the complete unity of the workers of all countries.
\end{displayquote}
This is the formula for co-operation with Sinn Fein or with Egyptian and Indian nationalism. It is further defined later. In regard to backward countries, Lenin says, we must have in view:---
\begin{displayquote}
The necessity of the co-operation of all Communists in the democratic-bourgeois movement of emancipation in those countries.
\end{displayquote}
Again:
\begin{displayquote}
"The Communist International must conclude temporary alliances with the bourgeois democracy of backward countries, but must never fuse with it." The class-conscious proletariat must "show itself particularly circumspect towards the survivals of national sentiment in countries long oppressed," and must "consent to certain useful concessions."
\end{displayquote}
The Asiatic policy of the Russian Government was adopted as a move against the British Empire, and as a method of inducing the British Government to make peace. It plays a larger part in the schemes of the leading Bolsheviks than is realized by the Labour Party in this country. Its method is not, for the present, to preach Communism, since the Persians and Hindoos are considered scarcely ripe for the doctrines of Marx. It is nationalist movements that are supported by money and agitators from Moscow. The method of quasi-independent states under Bolshevik protection is well understood. It is obvious that this policy affords opportunities for imperialism, under the cover of propaganda, and there is no doubt that some among the Bolsheviks are fascinated by its imperialist aspect. The importance officially attached to the Eastern policy is illustrated by the fact that it was the subject of the concluding portion of Lenin's speech to the recent Congress of the Third International (July 1920).

Bolshevism, like everything Russian, is partly Asiatic in character. One may distinguish two distinct trends, developing into two distinct policies. On the one side are the practical men, who wish to develop Russia industrially, to secure the gains of the Revolution nationally, to trade with the West, and gradually settle down into a more or less ordinary State. These men have on their side the fact of the economic exhaustion of Russia, the danger of ultimate revolt against Bolshevism if life continues to be as painful as it is at present, and the natural sentiment of humanity that wishes to relieve the sufferings of the people; also the fact that, if revolutions elsewhere produce a similar collapse of industry, they will make it impossible for Russia to receive the outside help which is urgently needed. In the early days, when the Government was weak, they had unchallenged control of policy, but success has made their position less secure.

On the other side there is a blend of two quite different aims: first, the desire to promote revolution in the Western nations, which is in line with Communist theory, and is also thought to be the only way of obtaining a really secure peace; secondly, the desire for Asiatic dominion, which is probably accompanied in the minds of some with dreams of sapphires and rubies and golden thrones and all the glories of their forefather Solomon. This desire produces an unwillingness to abandon the Eastern policy, although it is realized that, until it is abandoned, peace with capitalist England is impossible. I do not know whether there are some to whom the thought occurs that if England were to embark on revolution we should become willing to abandon India to the Russians. But I am certain that the converse thought occurs, namely that, if India could be taken from us, the blow to imperialist feeling might lead us to revolution. In either case, the two policies, of revolution in the West and conquest (disguised as liberation of oppressed peoples) in the East, work in together, and dovetail into a strongly coherent whole.

Bolshevism as a social phenomenon is to be reckoned as a religion, not as an ordinary political movement. The important and effective mental attitudes to the world may be broadly divided into the religious and the scientific. The scientific attitude is tentative and piecemeal, believing what it finds evidence for, and no more. Since Galileo, the scientific attitude has proved itself increasingly capable of ascertaining important facts and laws, which are acknowledged by all competent people regardless of temperament or self-interest or political pressure. Almost all the progress in the world from the earliest times is attributable to science and the scientific temper; almost all the major ills are attributable to religion.

By a religion I mean a set of beliefs held as dogmas, dominating the conduct of life, going beyond or contrary to evidence, and inculcated by methods which are emotional or authoritarian, not intellectual. By this definition, Bolshevism is a religion: that its dogmas go beyond or contrary to evidence, I shall try to prove in what follows. Those who accept Bolshevism become impervious to scientific evidence, and commit intellectual suicide. Even if all the doctrines of Bolshevism were true, this would still be the case, since no unbiased examination of them is tolerated. One who believes, as I do, that the free intellect is the chief engine of human progress, cannot but be fundamentally opposed to Bolshevism, as much as to the Church of Rome.

Among religions, Bolshevism is to be reckoned with Mohammedanism rather than with Christianity and Buddhism. Christianity and Buddhism are primarily personal religions, with mystical doctrines and a love of contemplation. Mohammedanism and Bolshevism are practical, social, unspiritual, concerned to win the empire of this world. Their founders would not have resisted the third of the temptations in the wilderness. What Mohammedanism did for the Arabs, Bolshevism may do for the Russians. As Ali went down before the politicians who only rallied to the Prophet after his success, so the genuine Communists may go down before those who are now rallying to the ranks of the Bolsheviks. If so, Asiatic empire with all its pomps and splendours may well be the next stage of development, and Communism may seem, in historical retrospect, as small a part of Bolshevism as abstinence from alcohol is of Mohammedanism. It is true that, as a world force, whether for revolution or for empire, Bolshevism must sooner or later be brought by success into a desperate conflict with America; and America is more solid and strong, as yet, than anything that Mohammed's followers had to face. But the doctrines of Communism are almost certain, in the long run, to make progress among American wage-earners, and the opposition of America is therefore not likely to be eternal. Bolshevism may go under in Russia, but even if it does it will spring up again elsewhere, since it is ideally suited to an industrial population in distress. What is evil in it is mainly due to the fact that it has its origin in distress; the problem is to disentangle the good from the evil, and induce the adoption of the good in countries not goaded into ferocity by despair.

Russia is a backward country, not yet ready for the methods of equal co-operation which the West is seeking to substitute for arbitrary power in politics and industry. In Russia, the methods of the Bolsheviks are probably more or less unavoidable; at any rate, I am not prepared to criticize them in their broad lines. But they are not the methods appropriate to more advanced countries, and our Socialists will be unnecessarily retrograde if they allow the prestige of the Bolsheviks to lead them into slavish imitation. It will be a far less excusable error in our reactionaries if, by their unteachableness, they compel the adoption of violent methods. We have a heritage of civilization and mutual tolerance which is important to ourselves and to the world. Life in Russia has always been fierce and cruel, to a far greater degree than with us, and out of the war has come a danger that this fierceness and cruelty may become universal. I have hopes that in England this may be avoided through the moderation of both sides. But it is essential to a happy issue that melodrama should no longer determine our views of the Bolsheviks: they are neither angels to be worshipped nor devils to be exterminated, but merely bold and able men attempting with great skill an almost impossible task.