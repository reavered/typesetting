The Bolsheviks have a very definite programme for achieving Communism---a programme which has been set forth by Lenin repeatedly, and quite recently in the reply of the Third International to the questionnaire submitted by the Independent Labour Party.

Capitalists, we are assured, will stick at nothing in defence of their privileges. It is the nature of man, in so far as he is politically conscious, to fight for the interests of his class so long as classes exist. When the conflict is not pushed to extremes, methods of conciliation and political deception may be preferable to actual physical warfare; but as soon as the proletariat make a really vital attack upon the capitalists, they will be met by guns and bayonets. This being certain and inevitable, it is as well to be prepared for it, and to conduct propaganda accordingly. Those who pretend that pacific methods can lead to the realization of Communism are false friends to the wage-earners; intentionally or unintentionally, they are covert allies of the bourgeoisie.

There must, then, according to Bolshevik theory, be armed conflict sooner or later, if the injustices of the present economic system are ever to be remedied. Not only do they assume armed conflict: they have a fairly definite conception of the way in which it is to be conducted. This conception has been carried out in Russia, and is to be carried out, before very long, in every civilized country. The Communists, who represent the class-conscious wage-earners, wait for some propitious moment when events have caused a mood of revolutionary discontent with the existing Government. They then put themselves at the head of the discontent, carry through a successful revolution, and in so doing acquire the arms, the railways, the State treasure, and all the other resources upon which the power of modern Governments is built. They then confine political power to Communists, however small a minority they may be of the whole nation. They set to work to increase their number by propaganda and the control of education. And meanwhile, they introduce Communism into every department of economic life as quickly as possible.

Ultimately, after a longer or shorter period, according to circumstances, the nation will be converted to Communism, the relics of capitalist institutions will have been obliterated, and it will be possible to restore freedom. But the political conflicts to which we are accustomed will not reappear. All the burning political questions of our time, according to the Communists, are questions of class conflict, and will disappear when the division of classes disappears. Accordingly the State will no longer be required, since the State is essentially an engine of power designed to give the victory to one side in the class conflict. Ordinary States are designed to give the victory to the capitalists; the proletarian State (Soviet Russia) is designed to give the victory to the wage-earners. As soon as the community contains only wage-earners, the State will cease to have any functions. And so, through a period of dictatorship, we shall finally arrive at a condition very similar to that aimed at by Anarchist Communism.

Three questions arise in regard to this method of reaching Utopia. First, would the ultimate state foreshadowed by the Bolsheviks be desirable in itself? Secondly, would the conflict involved in achieving it by the Bolshevik method be so bitter and prolonged that its evils would outweigh the ultimate good? Thirdly, is this method likely to lead, in the end, to the state which the Bolsheviks desire, or will it fail at some point and arrive at a quite different result? If we are to be Bolsheviks, we must answer all these questions in a sense favourable to their programme.

As regards the first question, I have no hesitation in answering it in a manner favourable to Communism. It is clear that the present inequalities of wealth are unjust. In part, they may be defended as affording an incentive to useful industry, but I do not think this defence will carry us very far. However, I have argued this question before in my book on \emph{Roads to Freedom}, and I will not spend time upon it now. On this matter, I concede the Bolshevik case. It is the other two questions that I wish to discuss.

Our second question was: Is the ultimate good aimed at by the Bolsheviks sufficiently great to be worth the price that, according to their own theory, will have to be paid for achieving it?

If anything human were absolutely certain, we might answer this question affirmatively with some confidence. The benefits of Communism, if it were once achieved, might be expected to be lasting; we might legitimately hope that further change would be towards something still better, not towards a revival of ancient evils. But if we admit, as we must do, that the outcome of the Communist revolution is in some degree uncertain, it becomes necessary to count the cost; for a great part of the cost is all but certain.

Since the revolution of October, 1917, the Soviet Government has been at war with almost all the world, and has had at the same time to face civil war at home. This is not to be regarded as accidental, or as a misfortune which could not be foreseen. According to Marxian theory, what has happened was bound to happen. Indeed, Russia has been wonderfully fortunate in not having to face an even more desperate situation. First and foremost, the world was exhausted by the war, and in no mood for military adventures. Next, the Tsarist régime was the worst in Europe, and therefore rallied less support than would be secured by any other capitalist Government. Again, Russia is vast and agricultural, making it capable of resisting both invasion and blockade better than Great Britain or France or Germany. The only other country that could have resisted with equal success is the United States, which is at present very far removed from a proletarian revolution, and likely long to remain the chief bulwark of the capitalist system. It is evident that Great Britain, attempting a similar revolution, would be forced by starvation to yield within a few months, provided America led a policy of blockade. The same is true, though in a less degree, of continental countries. Therefore, unless and until an international Communist revolution becomes possible, we must expect that any other nation following Russia's example will have to pay an even higher price than Russia has had to pay.

Now the price that Russia is having to pay is very great. The almost universal poverty might be thought to be a small evil in comparison with the ultimate gain, but it brings with it other evils of which the magnitude would be acknowledged even by those who have never known poverty and therefore make light of it. Hunger brings an absorption in the question of food, which, to most people, makes life almost purely animal. The general shortage makes people fierce, and reacts upon the political atmosphere. The necessity of inculcating Communism produces a hot-house condition, where every breath of fresh air must be excluded: people are to be taught to think in a certain way, and all free intelligence becomes taboo. The country comes to resemble an immensely magnified Jesuit College. Every kind of liberty is banned as being "\emph{bourgeois}"; but it remains a fact that intelligence languishes where thought is not free.

All this, however, according to the leaders of the Third International, is only a small beginning of the struggle, which must become world-wide before it achieves victory. In their reply to the Independent Labour Party they say:
\begin{displayquote}
It is probable that upon the throwing off of the chains of the capitalist Governments, the revolutionary proletariat of Europe will meet the resistance of Anglo-Saxon capital in the persons of British and American capitalists who will attempt to blockade it. It is then possible that the revolutionary proletariat of Europe will rise in union with the peoples of the East and commence a revolutionary struggle, the scene of which will be the entire world, to deal a final blow to British and American capitalism (\emph{The Times}, July 30, 1920).
\end{displayquote}
The war here prophesied, if it ever takes place, will be one compared to which the late war will come to seem a mere affair of outposts. Those who realize the destructiveness of the late war, the devastation and impoverishment, the lowering of the level of civilization throughout vast areas, the general increase of hatred and savagery, the letting loose of bestial instincts which had been curbed during peace---those who realize all this will hesitate to incur inconceivably greater horrors, even if they believe firmly that Communism in itself is much to be desired. An economic system cannot be considered apart from the population which is to carry it out; and the population resulting from such a world-war as Moscow calmly contemplates would be savage, bloodthirsty and ruthless to an extent that must make any system a mere engine of oppression and cruelty.

This brings us to our third question: Is the system which Communists regard as their goal likely to result from the adoption of their methods? This is really the most vital question of the three.

Advocacy of Communism by those who believe in Bolshevik methods rests upon the assumption that there is no slavery except economic slavery, and that when all goods are held in common there must be perfect liberty. I fear this is a delusion.

There must be administration, there must be officials who control distribution. These men, in a Communist State, are the repositories of power. So long as they control the army, they are able, as in Russia at this moment, to wield despotic power even if they are a small minority. The fact that there is Communism---to a certain extent---does not mean that there is liberty. If the Communism were more complete, it would not necessarily mean more freedom; there would still be certain officials in control of the food supply, and these officials could govern as they pleased so long as they retained the support of the soldiers. This is not mere theory: it is the patent lesson of the present condition of Russia. The Bolshevik theory is that a small minority are to seize power, and are to hold it until Communism is accepted practically universally, which, they admit, may take a long time. But power is sweet, and few men surrender it voluntarily. It is especially sweet to those who have the habit of it, and the habit becomes most ingrained in those who have governed by bayonets, without popular support. Is it not almost inevitable that men placed as the Bolsheviks are placed in Russia, and as they maintain that the Communists must place themselves wherever the social revolution succeeds, will be loath to relinquish their monopoly of power, and will find reasons for remaining until some new revolution ousts them? Would it not be fatally easy for them, without altering economic structure, to decree large salaries for high Government officials, and so reintroduce the old inequalities of wealth? What motive would they have for not doing so? What motive is possible except idealism, love of mankind, non-economic motives of the sort that Bolsheviks decry? The system created by violence and the forcible rule of a minority must necessarily allow of tyranny and exploitation; and if human nature is what Marxians assert it to be, why should the rulers neglect such opportunities of selfish advantage?

It is sheer nonsense to pretend that the rulers of a great empire such as Soviet Russia, when they have become accustomed to power, retain the proletarian psychology, and feel that their class-interest is the same as that of the ordinary working man. This is not the case in fact in Russia now, however the truth may be concealed by fine phrases. The Government has a class-consciousness and a class-interest quite distinct from those of the genuine proletarian, who is not to be confounded with the paper proletarian of the Marxian schema. In a capitalist state, the Government and the capitalists on the whole hang together, and form one class; in Soviet Russia, the Government has absorbed the capitalist mentality together with the governmental, and the fusion has given increased strength to the upper class. But I see no reason whatever to expect equality or freedom to result from such a system, except reasons derived from a false psychology and a mistaken analysis of the sources of political power.

I am compelled to reject Bolshevism for two reasons: First, because the price mankind must pay to achieve Communism by Bolshevik methods is too terrible; and secondly because, even after paying the price, I do not believe the result would be what the Bolsheviks profess to desire.

But if their methods are rejected, how are we ever to arrive at a better economic system? This is not an easy question, and I shall treat it in a separate chapter.