Before I went to Russia I imagined that I was going to see an interesting experiment in a new form of representative government. I did see an interesting experiment, but not in representative government. Every one who is interested in Bolshevism knows the series of elections, from the village meeting to the All-Russian Soviet, by which the people's commissaries are supposed to derive their power. We were told that, by the recall, the occupational constituencies, and so on, a new and far more perfect machinery had been devised for ascertaining and registering the popular will. One of the things we hoped to study was the question whether the Soviet system is really superior to Parliamentarism in this respect.

We were not able to make any such study, because the Soviet system is moribund.\footnote{In \emph{Theses} (p. 6 of French edition) it is said: 
\begin{displayquote}
The ancient classic subdivision of the Labour movement into three forms (parties, trade unions, and co-operatives) has served its time. The proletarian revolution has raised up in Russia the essential form of proletarian dictatorship, the \emph{soviets}. But the work in the Soviets, as in the industrial trade unions which have become revolutionary, must be invariably and systematically directed by the party of the proletariat, i.e. the Communist Party. As the organized advanced guard of the working class, the Communist Party answers equally to the economic, political and spiritual needs of the entire working class. It must be the soul of the trade unions, the soviets, and all other proletarian organizations.

The appearance of the Soviets, the principal historical form of the dictatorship of the proletariat, in no way diminishes the directing rôle of the party in the proletarian revolution. When the German Communists of the `Left'\ldots declare that `the party itself must also adapt itself more and more to the Soviet idea and proletarianize itself,' we see there only an insinuating expression of the idea that the Communist Party must dissolve itself into the Soviets, so that the Soviets can replace it.

This idea is profoundly erroneous and reactionary.

The history of the Russian Revolution shows us, at a certain moment, the Soviets going against the proletarian party and helping the agents of the bourgeoisie\ldots

In order that the Soviets may fulfil their historic mission, the existence of a Communist Party, strong enough not to `adapt' itself to the Soviets but to exercise on them a decisive influence, to force them \emph{not to adapt themselves} to the bourgeoisie and official social democracy,\ldots is on the contrary necessary.\end{displayquote}} No conceivable system of free election would give majorities to the Communists, either in town or country. Various methods are therefore adopted for giving the victory to Government candidates. In the first place, the voting is by show of hands, so that all who vote against the Government are marked men. In the second place, no candidate who is not a Communist can have any printing done, the printing works being all in the hands of the State. In the third place, he cannot address any meetings, because the halls all belong to the State. The whole of the press is, of course, official; no independent daily is permitted. In spite of all these obstacles, the Mensheviks have succeeded in winning about 40 seats out of 1,500 on the Moscow Soviet, by being known in certain large factories where the electoral campaign could be conducted by word of mouth. They won, in fact, every seat that they contested.

But although the Moscow Soviet is nominally sovereign in Moscow, it is really only a body of electors who choose the executive committee of forty, out of which, in turn, is chosen the Presidium, consisting of nine men who have all the power. The Moscow Soviet, as a whole, meets rarely; the Executive Committee is supposed to meet once a week, but did not meet while we were in Moscow. The Presidium, on the contrary, meets daily. Of course, it is easy for the Government to exercise pressure over the election of the executive committee, and again over the election of the Presidium. It must be remembered that effective protest is impossible, owing to the absolutely complete suppression of free speech and free Press. The result is that the Presidium of the Moscow Soviet consists only of orthodox Communists.

Kamenev, the President of the Moscow Soviet, informed us that the recall is very frequently employed; he said that in Moscow there are, on an average, thirty recalls a month. I asked him what were the principal reasons for the recall, and he mentioned four: drinking, going to the front (and being, therefore, incapable of performing the duties), change of politics on the part of the electors, and failure to make a report to the electors once a fortnight, which all members of the Soviet are expected to do. It is evident that the recall affords opportunities for governmental pressure, but I had no chance of finding out whether it is used for this purpose.

In country districts the method employed is somewhat different. It is impossible to secure that the village Soviet shall consist of Communists, because, as a rule, at any rate in the villages I saw, there are no Communists. But when I asked in the villages how they were represented on the Volost (the next larger area) or the Gubernia, I was met always with the reply that they were not represented at all. I could not verify this, and it is probably an overstatement, but all concurred in the assertion that if they elected a non-Communist representative he could not obtain a pass on the railway and, therefore, could not attend the Volost or Gubernia Soviet. I saw a meeting of the Gubernia Soviet of Saratov. The representation is so arranged that the town workers have an enormous preponderance over the surrounding peasants; but even allowing for this, the proportion of peasants seemed astonishingly small for the centre of a very important agricultural area.

The All-Russian Soviet, which is constitutionally the supreme body, to which the People's Commissaries are responsible, meets seldom, and has become increasingly formal. Its sole function at present, so far as I could discover, is to ratify, without discussion, previous decisions of the Communist Party on matters (especially concerning foreign policy) upon which the constitution requires its decision.

All real power is in the hands of the Communist Party, who number about 600,000 in a population of about 120 millions. I never came across a Communist by chance: the people whom I met in the streets or in the villages, when I could get into conversation with them, almost invariably said they were of no party. The only other answer I ever had was from some of the peasants, who openly stated that they were Tsarists. It must be said that the peasants' reasons for disliking the Bolsheviks are very inadequate. It is said--and all I saw confirmed the assertion--that the peasants are better off than they ever were before. I saw no one--man, woman, or child--who looked underfed in the villages. The big landowners are dispossessed, and the peasants have profited. But the towns and the army still need nourishing, and the Government has nothing to give the peasants in return for food except paper, which the peasants resent having to take. It is a singular fact that Tsarist roubles are worth ten times as much as Soviet roubles, and are much commoner in the country. Although they are illegal, pocket-books full of them are openly displayed in the market places. I do not think it should be inferred that the peasants expect a Tsarist restoration: they are merely actuated by custom and dislike of novelty. They have never heard of the blockade; consequently they cannot understand why the Government is unable to give them the clothes and agricultural implements that they need. Having got their land, and being ignorant of affairs outside their own neighbourhood, they wish their own village to be independent, and would resent the demands of any Government whatever.

Within the Communist Party there are, of course, as always in a bureaucracy, different factions, though hitherto the external pressure has prevented disunion. It seemed to me that the personnel of the bureaucracy could be divided into three classes. There are first the old revolutionists, tested by years of persecution. These men have most of the highest posts. Prison and exile have made them tough and fanatical and rather out of touch with their own country. They are honest men, with a profound belief that Communism will regenerate the world. They think themselves utterly free from sentiment, but, in fact, they are sentimental about Communism and about the régime that they are creating; they cannot face the fact that what they are creating is not complete Communism, and that Communism is anathema to the peasant, who wants his own land and nothing else. They are pitiless in punishing corruption or drunkenness when they find either among officials; but they have built up a system in which the temptations to petty corruption are tremendous, and their own materialistic theory should persuade them that under such a system corruption must be rampant.

The second class in the bureaucracy, among whom are to be found most of the men occupying political posts just below the top, consists of \emph{arrivistes}, who are enthusiastic Bolsheviks because of the material success of Bolshevism. With them must be reckoned the army of policemen, spies, and secret agents, largely inherited from the Tsarist times, who make their profit out of the fact that no one can live except by breaking the law. This aspect of Bolshevism is exemplified by the Extraordinary Commission, a body practically independent of the Government, possessing its own regiments, who are better fed than the Red Army. This body has the power of imprisoning any man or woman without trial on such charges as speculation or counter-revolutionary activity. It has shot thousands without proper trial, and though now it has nominally lost the power of inflicting the death penalty, it is by no means certain that it has altogether lost it in fact. It has spies everywhere, and ordinary mortals live in terror of it.

The third class in the bureaucracy consists of men who are not ardent Communists, who have rallied to the Government since it has proved itself stable, and who work for it either out of patriotism or because they enjoy the opportunity of developing their ideas freely without the obstacle of traditional institutions. Among this class are to be found men of the type of the successful business man, men with the same sort of ability as is found in the American self-made Trust magnate, but working for success and power, not for money. There is no doubt that the Bolsheviks are successfully solving the problem of enlisting this kind of ability in the public service, without permitting it to amass wealth as it does in capitalist communities. This is perhaps their greatest success so far, outside the domain of war. It makes it possible to suppose that, if Russia is allowed to have peace, an amazing industrial development may take place, making Russia a rival of the United States. The Bolsheviks are industrialists in all their aims; they love everything in modern industry except the excessive rewards of the capitalists. And the harsh discipline to which they are subjecting the workers is calculated, if anything can, to give them the habits of industry and honesty which have hitherto been lacking, and the lack of which alone prevents Russia from being one of the foremost industrial countries.