The fundamental ideas of Communism are by no means impracticable, and would, if realized, add immeasurably to the well-being of mankind. The difficulties which have to be faced are not in regard to the fundamental ideas, but in regard to the transition from capitalism. It must be assumed that those who profit by the existing system will fight to preserve it, and their fight may be sufficiently severe to destroy all that is best in Communism during the struggle, as well as everything else that has value in modern civilization. The seriousness of this problem of transition is illustrated by Russia, and cannot be met by the methods of the Third International. The Soviet Government, at the present moment, is anxious to obtain manufactured goods from capitalist countries, but the Third International is meanwhile endeavouring to promote revolutions which, if they occurred, would paralyse the industries of the countries concerned, and leave them incapable of supplying Russian needs.

The supreme condition of success in a Communist revolution is that it should not paralyse industry. If industry is paralysed, the evils which exist in modern Russia, or others just as great, seem practically unavoidable. There will be the problem of town and country, there will be hunger, there will be fierceness and revolts and military tyranny. All these things follow in a fatal sequence; and the end of them is almost certain to be something quite different from what genuine Communists desire.

If industry is to survive throughout a Communist revolution, a number of conditions must be fulfilled which are not, at present, fulfilled anywhere. Consider, for the sake of definiteness, what would happen if a Communist revolution were to occur in England to-morrow. Immediately America would place an embargo on all trade with us. The cotton industry would collapse, leaving about five million of the most productive portion of the population idle. The food supply would become inadequate, and would fail disastrously if, as is to be expected, the Navy were hostile or disorganized by the sabotage of the officers. The result would be that, unless there were a counter-revolution, about half the population would die within the first twelve months. On such a basis it would evidently be impossible to erect a successful Communist State.

What applies to England applies, in one form or another, to the remaining countries of Europe. Italian and German Socialists are, many of them, in a revolutionary frame of mind and could, if they chose, raise formidable revolts. They are urged by Moscow to do so, but they realize that, if they did, England and America would starve them. France, for many reasons, dare not offend England and America beyond a point. Thus, in every country except America, a successful Communist revolution is impossible for economico-political reasons. America, being self-contained and strong, would be capable, so far as material conditions go, of achieving a successful revolution; but in America the psychological conditions are as yet adverse. There is no other civilized country where capitalism is so strong and revolutionary Socialism so weak as in America. At the present moment, therefore, though it is by no means impossible that Communist revolutions may occur all over the Continent, it is nearly certain that they cannot be successful in any real sense. They will have to begin by a war against America, and possibly England, by a paralysis of industry, by starvation, militarism and the whole attendant train of evils with which Russia has made us familiar.

That Communism, whenever and wherever it is adopted, will have to begin by fighting the bourgeoisie, is highly probable. The important question is not whether there is to be fighting, but how long and severe it is to be. A short war, in which Communism won a rapid and easy victory, would do little harm. It is long, bitter and doubtful wars that must be avoided if anything of what makes Communism desirable is to survive.

Two practical consequences flow from this conclusion: first, that nothing can succeed until America is either converted to Communism, or at any rate willing to remain neutral; secondly, that it is a mistake to attempt to inaugurate Communism in a country where the majority are hostile, or rather, where the active opponents are as strong as the active supporters, because in such a state of opinion a very severe civil war is likely to result. It is necessary to have a great body of opinion favourable to Communism, and a rather weak opposition, before a really successful Communist state can be introduced either by revolution or by more or less constitutional methods.

It may be assumed that when Communism is first introduced, the higher technical and business staff will side with the capitalists and attempt sabotage unless they have no hopes of a counter-revolution. For this reason it is very necessary that among wage-earners there should be as wide a diffusion as possible of technical and business education, so that they may be able immediately to take control of big complex industries. In this respect Russia was very badly off, whereas England and America would be much more fortunate.

Self-government in industry is, I believe, the road by which England can best approach Communism. I do not doubt that the railways and the mines, after a little practice, could be run more efficiently by the workers, from the point of view of production, than they are at present by the capitalists. The Bolsheviks oppose self-government in industry every where, because it has failed in Russia, and their national self-esteem prevents them from admitting that this is due to the backwardness of Russia. This is one of the respects in which they are misled by the assumption that Russia must be in all ways a model to the rest of the world. I would go so far as to say that the winning of self-government in such industries as railways and mining is an essential preliminary to complete Communism. In England, especially, this is the case. The Unions can command whatever technical skill they may require; they are politically powerful; the demand for self-government is one for which there is widespread sympathy, and could be much more with adequate propaganda; moreover (what is important with the British temperament) self-government can be brought about gradually, by stages in each trade, and by extension from one trade to another. Capitalists value two things, their power and their money; many individuals among them value only the money. It is wiser to concentrate first on the power, as is done by seeking self-government in industry without confiscation of capitalist incomes. By this means the capitalists are gradually turned into obvious drones, their active functions in industry become nil, and they can be ultimately dispossessed without dislocation and without the possibility of any successful struggle on their parts.

Another advantage of proceeding by way of self-government is that it tends to prevent the Communist régime, when it comes, from having that truly terrible degree of centralization which now exists in Russia. The Russians have been forced to centralize, partly by the problems of the war, but more by the shortage of all kinds of skill. This has compelled the few competent men to attempt each to do the work of ten men, which has not proved satisfactory in spite of heroic efforts. The idea of democracy has become discredited as the result first of syndicalism, and then of Bolshevism. But there are two different things that may be meant by democracy: we may mean the system of Parliamentary government, or we may mean the participation of the people in affairs. The discredit of the former is largely deserved, and I have no desire to uphold Parliament as an ideal institution. But it is a great misfortune if, from a confusion of ideas, men come to think that, because Parliaments are imperfect, there is no reason why there should be self-government. The grounds for advocating self-government are very familiar: first, that no benevolent despot can be trusted to know or pursue the interests of his subjects; second, that the practice of self-government is the only effective method of political education; third, that it tends to place the preponderance of force on the side of the constitution, and thus to promote order and stable government. Other reasons could be found, but I think these are the chief. In Russia self-government has disappeared, except within the Communist Party. If it is not to disappear elsewhere during a Communist revolution, it is very desirable that there should exist already important industries competently administered by the workers themselves.

The Bolshevik philosophy is promoted very largely by despair of more gradual methods. But this despair is a mark of impatience, and is not really warranted by the facts. It is by no means impossible, in the near future, to secure self-government in British railways and mines by constitutional means. This is not the sort of measure which would bring into operation an American blockade or a civil war or any of the other catastrophic dangers that are to be feared from a full-fledged Communist revolution in the present international situation. Self-government in industry is feasible, and would be a great step towards Communism. It would both afford many of the advantages of Communism and also make the transition far easier without a technical break-down of production.

There is another defect in the methods advocated by the Third International. The sort of revolution which is recommended is never practically feasible except in a time of national misfortune; in fact, defeat in war seems to be an indispensable condition. Consequently, by this method, Communism will only be inaugurated where the conditions of life are difficult, where demoralization and disorganization make success almost impossible, and where men are in a mood of fierce despair very inimical to industrial construction. If Communism is to have a fair chance, it must be inaugurated in a prosperous country. But a prosperous country will not be readily moved by the arguments of hatred and universal upheaval which are employed by the Third International. It is necessary, in appealing to a prosperous country, to lay stress on hope rather than despair, and to show how the transition can be effected without a calamitous loss of prosperity. All this requires less violence and subversiveness, more patience and constructive propaganda, less appeal to the armed might of a determined minority.

The attitude of uncompromising heroism is attractive, and appeals especially to the dramatic instinct. But the purpose of the serious revolutionary is not personal heroism, nor martyrdom, but the creation of a happier world. Those who have the happiness of the world at heart will shrink from attitudes and the facile hysteria of "no parley with the enemy." They will not embark upon enterprises, however arduous and austere, which are likely to involve the martyrdom of their country and the discrediting of their ideals. It is by slower and less showy methods that the new world must be built: by industrial efforts after self-government, by proletarian training in technique and business administration, by careful study of the international situation, by a prolonged and devoted propaganda of ideas rather than tactics, especially among the wage-earners of the United States. It is not true that no gradual approaches to Communism are possible: self-government in industry is an important instance to the contrary. It is not true that any isolated European country, or even the whole of the Continent in unison, can, after the exhaustion produced by the war, introduce a successful form of Communism at the present moment, owing to the hostility and economic supremacy of America. To find fault with those who urge these considerations, or to accuse them of faint-heartedness, is mere sentimental self-indulgence, sacrificing the good we can do to the satisfaction of our own emotions.

Even under present conditions in Russia, it is possible still to feel the inspiration of the essential spirit of Communism, the spirit of creative hope, seeking to sweep away the incumbrances of injustice and tyranny and rapacity which obstruct the growth of the human spirit, to replace individual competition by collective action, the relation of master and slave by free co-operation. This hope has helped the best of the Communists to bear the harsh years through which Russia has been passing, and has become an inspiration to the world. The hope is not chimerical, but it can only be realized through a more patient labour, a more objective study of facts, and above all a longer propaganda, to make the necessity of the transition obvious to the great majority of wage-earners. Russian Communism may fail and go under, but Communism itself will not die. And if hope rather than hatred inspires its advocates, it can be brought about without the universal cataclysm preached by Moscow. The war and its sequel have proved the destructiveness of capitalism; let us see to it that the next epoch does not prove the still greater destructiveness of Communism, but rather its power to heal the wounds which the old evil system has inflicted upon the human spirit.

