The civilized world seems almost certain, sooner or later, to follow the example of Russia in attempting a Communist organization of society. I believe that the attempt is essential to the progress and happiness of mankind during the next few centuries, but I believe also that the transition has appalling dangers. I believe that, if the Bolshevik theory as to the method of transition is adopted by Communists in Western nations, the result will be a prolonged chaos, leading neither to Communism nor to any other civilized system, but to a relapse into the barbarism of the Dark Ages. In the interests of Communism, no less than in the interests of civilization, I think it imperative that the Russian failure should be admitted and analysed. For this reason, if for no other, I cannot enter into the conspiracy of concealment which many Western Socialists who have visited Russia consider necessary.

I shall try first to recapitulate the facts which make me regard the Russian experiment as a failure, and then to seek out the causes of failure.

The most elementary failure in Russia is in regard to food. In a country which formerly produced a vast exportable surplus of cereals and other agricultural produce, and in which the non-agricultural population is only 15 per cent. of the total, it ought to be possible, without great difficulty, to provide enough food for the towns. Yet the Government has failed badly in this respect. The rations are inadequate and irregular, so that it is impossible to preserve health and vigour without the help of food purchased illicitly in the markets at speculative prices. I have given reasons for thinking that the breakdown of transport, though a contributory cause, is not the main reason for the shortage. The main reason is the hostility of the peasants, which, in turn, is due to the collapse of industry and to the policy of forced requisitions. In regard to corn and flour, the Government requisitions all that the peasant produces above a certain minimum required for himself and his family. If, instead, it exacted a fixed amount as rent, it would not destroy his incentive to production, and would not provide nearly such a strong motive for concealment. But this plan would have enabled the peasants to grow rich, and would have involved a confessed abandonment of Communist theory. It has therefore been thought better to employ forcible methods, which led to disaster, as they were bound to do.

The collapse of industry was the chief cause of the food difficulties, and has in turn been aggravated by them. Owing to the fact that there is abundant food in the country, industrial and urban workers are perpetually attempting to abandon their employment for agriculture. This is illegal, and is severely punished, by imprisonment or convict labour. Nevertheless it continues, and in so vast a country as Russia it is not possible to prevent it. Thus the ranks of industry become still further depleted.

Except as regards munitions of war, the collapse of industry in Russia is extraordinarily complete. The resolutions passed by the Ninth Congress of the Communist Party (April, 1920) speak of "the incredible catastrophes of public economy." This language is not too strong, though the recovery of the Baku oil has done something to produce a revival along the Volga basin.

The failure of the whole industrial side of the national economy, including transport, is at the bottom of the other failures of the Soviet Government. It is, to begin with, the main cause of the unpopularity of the Communists both in town and country: in town, because the people are hungry; in the country, because food is taken with no return except paper. If industry had been prosperous, the peasants could have had clothes and agricultural machinery, for which they would have willingly parted with enough food for the needs of the towns. The town population could then have subsisted in tolerable comfort; disease could have been coped with, and the general lowering of vitality averted. It would not have been necessary, as it has been in many cases, for men of scientific or artistic capacity to abandon the pursuits in which they were skilled for unskilled manual labour. The Communist Republic might have been agreeable to live in---at least for those who had been very poor before.

The unpopularity of the Bolsheviks, which is primarily due to the collapse of industry, has in turn been accentuated by the measures which it has driven the Government to adopt. In view of the fact that it was impossible to give adequate food to the ordinary population of Petrograd and Moscow, the Government decided that at any rate the men employed on important public work should be sufficiently nourished to preserve their efficiency. It is a gross libel to say that the Communists, or even the leading People's Commissaries, live luxurious lives according to our standards; but it is a fact that they are not exposed, like their subjects, to acute hunger and the weakening of energy that accompanies it. No tone can blame them for this, since the work of government must be carried on; but it is one of the ways in which class distinctions have reappeared where it was intended that they should be banished. I talked to an obviously hungry working man in Moscow, who pointed to the Kremlin and remarked: "In there they have enough to eat." He was expressing a widespread feeling which is fatal to the idealistic appeal that Communism attempts to make.

Owing to unpopularity, the Bolsheviks have had to rely upon the army and the Extraordinary Commission, and have been compelled to reduce the Soviet system to an empty form. More and more the pretence of representing the proletariat has grown threadbare. Amid official demonstrations and processions and meetings the genuine proletarian looks on, apathetic and disillusioned, unless he is possessed of unusual energy and fire, in which case he looks to the ideas of syndicalism or the I.W.W. to liberate him from a slavery far more complete than that of capitalism. A sweated wage, long hours, industrial conscription, prohibition of strikes, prison for slackers, diminution of the already insufficient rations in factories where the production falls below what the authorities expect, an army of spies ready to report any tendency to political disaffection and to procure imprisonment for its promoters---this is the reality of a system which still professes to govern in the name of the proletariat.

At the same time the internal and external peril has necessitated the creation of a vast army recruited by conscription, except as regards a Communist nucleus, from among a population utterly weary of war, who put the Bolsheviks in power because they alone promised peace. Militarism has produced its inevitable result in the way of a harsh and dictatorial spirit: the men in power go through their day's work with the consciousness that they command three million armed men, and that civilian opposition to their will can be easily crushed.

Out of all this has grown a system painfully like the old government of the Tsar---a system which is Asiatic in its centralized bureaucracy, its secret service, its atmosphere of governmental mystery and submissive terror. In many ways it resembles our Government of India. Like that Government, it stands for civilization, for education, sanitation, and Western ideas of progress; it is composed in the main of honest and hard-working men, who despise those whom they govern, but believe themselves possessed of something valuable which they must communicate to the population, however little it may be desired. Like our Government in India, they live in terror of popular risings, and are compelled to resort to cruel repressions in order to preserve their power. Like it, they represent an alien philosophy of life, which cannot be forced upon the people without a change of instinct, habit, and tradition so profound as to dry up the vital springs of action, producing listlessness and despair among the ignorant victims of militant enlightenment. It may be that Russia needs sternness and discipline more than anything else; it may be that a revival of Peter the Great's methods is essential to progress. From this point of view, much of what it is natural to criticize in the Bolsheviks becomes defensible; but this point of view has little affinity to Communism. Bolshevism may be defended, possibly, as a dire discipline through which a backward nation is to be rapidly industrialized; but as an experiment in Communism it has failed.

There are two things that a defender of the Bolsheviks may say against the argument that they have failed because the present state of Russia is bad. It may be said that it is too soon to judge, and it may be urged that whatever failure there has been is attributable to the hostility of the outside world.

As to the contention that it is too soon to judge, that is of course undeniable in a sense. But in a sense it is always too soon to judge of any historical movement, because its effects and developments go on for ever. Bolshevism has, no doubt, great changes ahead of it. But the last three years have afforded material for some judgments, though more definitive judgments will be possible later. And, for reasons which I have given in earlier chapters, I find it impossible to believe that later developments will realize more fully the Communist ideal. If trade is opened with the outer world, there will be an almost irresistible tendency to resumption of private enterprise. If trade is not re-opened, the plans of Asiatic conquest will mature, leading to a revival of Yenghis Khan and Timur. In neither case is the purity of the Communist faith likely to survive.

As for the hostility of the Entente, it is of course true that Bolshevism might have developed very differently if it had been treated in a friendly spirit. But in view of its desire to promote world-revolution, no one could expect---and the Bolsheviks certainly did not expect---that capitalist Governments would be friendly. If Germany had won the war, Germany would have shown a hostility more effective than that of the Entente. However we may blame Western Governments for their policy, we must realize that, according to the deterministic economic theory of the Bolsheviks, no other policy was to be expected from them. Other men might have been excused for not foreseeing the attitude of Churchill, Clemenceau and Millerand; but Marxians could not be excused, since this attitude was in exact accord with their own formula.

We have seen the symptoms of Bolshevik failure; I come now to the question of its profounder causes.

Everything that is worst in Russia we found traceable to the collapse of industry. Why has industry collapsed so utterly? And would it collapse equally if a Communist revolution were to occur in a Western country?

Russian industry was never highly developed, and depended always upon outside aid for much of its plant. The hostility of the world, as embodied in the blockade, left Russia powerless to replace the machinery and locomotives worn out during the war. The need of self-defence compelled the Bolsheviks to send their best workmen to the front, because they were the most reliable Communists, and the loss of them rendered their factories even more inefficient than they were under Kerensky. In this respect, and in the laziness and incapacity of the Russian workman, the Bolsheviks have had to face special difficulties which would be less in other countries. On the other hand, they have had special advantages in the fact that Russia is self-supporting in the matter of food; no other country could have endured the collapse of industry so long, and no other Great Power except the United States could have survived years of blockade.

The hostility of the world was in no way a surprise to those who made the October revolution; it was in accordance with their general theory, and its consequences should have been taken into account in making the revolution.

Other hostilities besides those of the outside world have been incurred by the Bolsheviks with open eyes, notably the hostility of the peasants and that of a great part of the industrial population. They have attempted, in accordance with their usual contempt for conciliatory methods, to substitute terror for reward as the incentive to work. Some amiable Socialists have imagined that, when the private capitalist had been eliminated, men would work from a sense of obligation to the community. The Bolsheviks will have none of such sentimentalism. In one of the resolutions of the ninth Communist Congress they say:
\begin{displayquote}
Every social system, whether based on slavery, feudalism, or capitalism, had its ways and means of labour compulsion and labour education in the interests of the exploiters.

The Soviet system is faced with the task of developing its own methods of labour compulsion to attain an increase of the intensity and wholesomeness of labour; this method is to be based on the socialization of public economy in the interests of the whole nation.

In addition to the propaganda by which the people are to be influenced and the repressions which are to be applied to all idlers, parasites and disorganizers who strive to undermine public zeal---the principal method for the increase of production will become the introduction of the system of compulsory labour.

In capitalist society rivalry assumed the character of competition and led to the exploitation of man by man. In a society where the means of production are nationalized, labour rivalry is to increase the products of labour without infringing its solidarity.

Rivalry between factories, regions, guilds, workshops, and individual workers should become the subject of careful organization and of close study on the side of the Trade Unions and the economic organs.

The system of premiums which is to be introduced should become one of the most powerful means of exciting rivalry. The system of rationing of food supply is to get into line with it; so long as Soviet Russia suffers from insufficiency of provisions, it is only just that the industrious and conscientious worker receives more than the careless worker.
\end{displayquote}
It must be remembered that even the "industrious and conscientious worker" receives less food than is required to maintain efficiency.

Over the whole development of Russia and of Bolshevism since the October revolution there broods a tragic fatality. In spite of outward success the inner failure has proceeded by inevitable stages---stages which could, by sufficient acumen, have been foreseen from the first. By provoking the hostility of the outside world the Bolsheviks were forced to provoke the hostility of the peasants, and finally the hostility or utter apathy of the urban and industrial population. These various hostilities brought material disaster, and material disaster brought spiritual collapse. The ultimate source of the whole train of evils lies in the Bolshevik outlook on life: in its dogmatism of hatred and its belief that human nature can be completely transformed by force. To injure capitalists is not the ultimate goal of Communism, though among men dominated by hatred it is the part that gives zest to their activities. To face the hostility of the world may show heroism, but it is a heroism for which the country, not its rulers, has to pay the price. In the principles of Bolshevism there is more desire to destroy ancient evils than to build up new goods; it is for this reason that success in destruction has been so much greater than in construction. The desire to destroy is inspired by hatred, which is not a constructive principle. From this essential characteristic of Bolshevik mentality has sprung the willingness to subject Russia to its present martyrdom. It is only out of a quite different mentality that a happier world can be created.

And from this follows a further conclusion. The Bolshevik outlook is the outcome of the cruelty of the Tsarist régime and the ferocity of the years of the Great War, operating upon a ruined and starving nation maddened into universal hatred. If a different mentality is needed for the establishment of a successful Communism, then a quite different conjuncture must see its inauguration; men must be persuaded to the attempt by hope, not driven to it by despair. To bring this about should be the aim of every Communist who desires the happiness of mankind more than the punishment of capitalists and their governmental satellites.