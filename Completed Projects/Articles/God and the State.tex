\documentclass[12pt]{report}
\usepackage[12pt]{moresize}
\usepackage[utf8]{inputenc}
\usepackage[english]{babel}
\usepackage[top=2.5cm, bottom=2.5cm, left=2.5cm, right=2.5cm]{geometry}
\usepackage{ebgaramond}

%=======SECTION HEADERS=========%
\usepackage{titlesec}
\usepackage{titletoc}

%========QUOTES=========%
\usepackage{epigraph}
\usepackage[autostyle, english = american]{csquotes}

%=======PARAGRAPH FORMATTING=========%
\setlength{\parindent}{0pt} %no paragraph indents
\setlength{\parskip}{1em}   %single space between paragraphs

\renewcommand{\chaptermark}[1]{\markboth{\MakeUppercase{Book \thechapter}}{}} %Book format- heading

%=======CHAPTER FORMATTING=========%
\renewcommand\thesection{{\arabic{section}}}   %section numbering style

\titleformat
{\chapter} 
[display]
{\fontfamily{ppl}\Huge} 
{} 
{\leftmargin}{}[]

\newcommand{\mychapter}[2]{
\setcounter{chapter}{#1}
    \setcounter{section}{0}
    \chapter*{#2}
    \addcontentsline{toc}{chapter}{#2}
}

%=======SECTION HEADER SPACING=========%
\titlespacing{\chapter}{0mm}{-2em}{1em}
\titlespacing{\section}{0mm}{3mm}{2mm}

%=======TITLE PAGE=========%
\title{\HUGE\bfseries{God and the State}}
\author{\Large by Mikhail Bakunin}
\date{\vspace{-4mm}Translated 1883 by Benjamin Tucker}

%=======FOOTNOTES=========%
\renewcommand{\thefootnote}{[\arabic{footnote}]}
\setlength{\skip\footins}{1cm}
\usepackage[]{footmisc}
\renewcommand{\footnotemargin}{3mm} %Setting left margin
\renewcommand{\footnotelayout}{\hspace{2mm}} %spacing between the footnote number and the text of footnote

\usepackage{hyperref}
\hypersetup{bookmarksnumbered} %Bookmarks are numbered in the ToC when converted to PDF or EPUB

\titlecontents{chapter}% formatting-toc-chapters
    [0pt]% <left-indent>
    {\vspace{1em}}% <above-code>
    {\thecontentslabel}% <numbered-entry-format>
    {}% <numberless-entry-format>
    {\titlerule*[1pc]{.}\contentspage}% <filler-page-format>
\titlecontents{section}%formatting-toc-sections
    [3.8em] 
    {\vspace{-3mm}}
    {\contentslabel{2.3em}}
    {}
    {\titlerule*[1pc]{.}\contentspage}
\begin{document}

\begin{titlepage}
    \maketitle
\end{titlepage}

%=======TABLE OF CONTENTS=========%
\renewcommand*\contentsname{\vspace{-1cm}Contents}
\tableofcontents

\titleformat
{\chapter} 
[display]
{\centering\fontfamily{ppl}\Huge} 
{} 
{\leftmargin}{}[]

\mychapter{2}{I}
Who is right, the idealists or the materialists? The question, once stated in this way, hesitation becomes impossible. Undoubtedly the idealists are wrong and the materialists right. Yes, facts are before ideas; yes, the ideal, as Proudhon said, is but a flower, whose root lies in the material conditions of existence. Yes, the whole history of humanity, intellectual and moral, political and social, is but a reflection of its economic history.


All branches of modern science, of true and disinterested science, concur in proclaiming this grand truth, fundamental and decisive: The social world, properly speaking, the human world — in short, humanity — is nothing other than the last and supreme development — at least on our planet and as far as we know — the highest manifestation of animality. But as every development necessarily implies a negation, that of its base or point of departure, humanity is at the same time and essentially the deliberate and gradual negation of the animal element in man; and it is precisely this negation, as rational as it is natural, and rational only because natural — at once historical and logical, as inevitable as the development and realization of all the natural laws in the world — that constitutes and creates the ideal, the world of intellectual and moral convictions, ideas.


Yes, our first ancestors, our Adams and our Eves, were, if not gorillas, very near relatives of gorillas, omnivorous, intelligent and ferocious beasts, endowed in a higher degree than the animals of another species with two precious faculties — \emph{the power to think} and \emph{the desire to rebel.}


These faculties, combining their progressive action in history, represent the essential factor, the negative power in the positive development of human animality, and create consequently all that constitutes humanity in man.


The Bible, which is a very interesting and here and there very profound book when considered as one of the oldest surviving manifestations of human wisdom and fancy, expresses this truth very naively in its myth of original sin. Jehovah, who of all the good gods adored by men was certainly the most jealous, the most vain, the most ferocious, the most unjust, the most bloodthirsty, the most despotic, and the most hostile to human dignity and liberty — Jehovah had just created Adam and Eve, to satisfy we know not what caprice; no doubt to while away his time, which must weigh heavy on his hands in his eternal egoistic solitude, or that he might have some new slaves. He generously placed at their disposal the whole earth, with all its fruits and animals, and set but a single limit to this complete enjoyment. He expressly forbade them from touching the fruit of the tree of knowledge. He wished, therefore, that man, destitute of all understanding of himself, should remain an eternal beast, ever on all-fours before the eternal God, his creator and his master. But here steps in Satan, the eternal rebel, the first freethinker and the emancipator of worlds. He makes man ashamed of his bestial ignorance and obedience; he emancipates him, stamps upon his brow the seal of liberty and humanity, in urging him to disobey and eat of the fruit of knowledge.


We know what followed. The good God, whose foresight, which is one of the divine faculties, should have warned him of what would happen, flew into a terrible and ridiculous rage; he cursed Satan, man, and the world created by himself, striking himself so to speak in his own creation, as children do when they get angry; and, not content with smiting our ancestors themselves, he cursed them in all the generations to come, innocent of the crime committed by their forefathers. Our Catholic and Protestant theologians look upon that as very profound and very just, precisely because it is monstrously iniquitous and absurd. Then, remembering that he was not only a God of vengeance and wrath, but also a God of love, after having tormented the existence of a few milliards of poor human beings and condemned them to an eternal hell, he took pity on the rest, and, to save them and reconcile his eternal and divine love with his eternal and divine anger, always greedy for victims and blood, he sent into the world, as an expiatory victim, his only son, that he might be killed by men. That is called the mystery of the Redemption, the basis of all the Christian religions. Still, if the divine Savior had saved the human world! But no; in the paradise promised by Christ, as we know, such being the formal announcement, the elect will number very few. The rest, the immense majority of the generations present and to come, will burn eternally in hell. In the meantime, to console us, God, ever just, ever good, hands over the earth to the government of the Napoleon Thirds, of the William Firsts, of the Ferdinands of Austria, and of the Alexanders of all the Russias.


Such are the absurd tales that are told and the monstrous doctrines that are taught, in the full light of the nineteenth century, in all the public schools of Europe, at the express command of the government. They call this civilizing the people! Is it not plain that all these governments are systematic poisoners, interested stupefies of the masses?


I have wandered from my subject, because anger gets hold of me whenever I think of the base and criminal means which they employ to keep the nations in perpetual slavery, undoubtedly that they may be the better able to fleece them. Of what consequence are the crimes of all the Tropmanns in the world compared with this crime of treason against humanity committed daily, in broad day, over the whole surface of the civilized world, by those who dare to call themselves the guardians and the fathers of the people? I return to the myth of original sin.


God admitted that Satan was right; he recognized that the devil did not deceive Adam and Eve in promising them knowledge and liberty as a reward for the act of disobedience which he bad induced them to commit; for, immediately they had eaten of the forbidden fruit, God himself said (see Bible): “Behold, man is become as of the Gods, knowing both good and evil; prevent him, therefore, from eating of the fruit of eternal life, lest he become immortal like Ourselves.


Let us disregard now the fabulous portion of this myth and consider its true meaning, which is very clear. Man has emancipated himself; he has separated himself from animality and constituted himself a man; he has begun his distinctively human history and development by an act of disobedience and science — that is, by\emph{ rebellion} and by \emph{thought.}


Three elements or, if you like, three fundamental principles constitute the essential conditions of all human development, collective or individual, in history:
\begin{enumerate}
\item \emph{human animality};
\item \emph{thought;} and
\item \emph{rebellion.}
\end{enumerate}
To the first properly corresponds \emph{social and private economy}; to the second, \emph{science}; to the third,\emph{ liberty.}


Idealists of all schools, aristocrats and \emph{bourgeois}, theologians and metaphysicians, politicians and moralists, religionists, philosophers, or poets, not forgetting the liberal economists — unbounded worshippers of the ideal, as we know — are much offended when told that man, with his magnificent intelligence, his sublime ideas, and his boundless aspirations, is, like all else existing in the world, nothing but matter, only a product of \emph{vile matter}.


We may answer that the matter of which materialists speak, matter spontaneously and eternally mobile, active, productive, matter chemically or organically determined and manifested by the properties or forces, mechanical, physical, animal, and intelligent, which necessarily belong to it — that this matter has nothing in common with the \emph{vile matter} of the idealists. The latter, a product of their false abstraction, is indeed a stupid, inanimate, immobile thing, incapable of giving birth to the smallest product, a \emph{caput mortuum}, an \emph{ugly} fancy in contrast to the \emph{beautiful} fancy which they call \emph{God}; as the opposite of this supreme being, matter, their matter, stripped by that constitutes its real nature, necessarily represents supreme nothingness. They have taken away intelligence, life, all its determining qualities, active relations or forces, motion itself, without which matter would not even have weight, leaving it nothing but impenetrability and absolute immobility in space; they have attributed all these natural forces, properties, and manifestations to the imaginary being created by their abstract fancy; then, interchanging \emph{rôles}, they have called this product of their imagination, this phantom, this God who is nothing, “supreme Being” and, as a necessary consequence, have declared that the real being, matter, the world, is nothing. After which they gravely tell us that this matter is incapable of producing anything, not even of setting itself in motion, and consequently must have been created by their God.


At the end of this book I exposed the fallacies and truly revolting absurdities to which one is inevitably led by this imagination of a God, let him be considered as a personal being, the creator and organizer of worlds; or even as impersonal, a kind of divine soul spread over the whole universe and constituting thus its eternal principle; or let him be an idea, infinite and divine, always present and active in the world, and always manifested by the totality of material and definite beings. Here I shall deal with one point only.


The gradual development of the material world, as well as of organic animal life and of the historically progressive intelligence of man, individually or socially, is perfectly conceivable. It is a wholly natural movement from the simple to the complex, from the lower to the higher, from the inferior to the superior; a movement in conformity with all our daily experiences, and consequently in conformity also with our natural logic, with the distinctive laws of our mind, which being formed and developed only by the aid of these same experiences; is, so to speak, but the mental, cerebral reproduction or reflected summary thereof.


The system of the idealists is quite the contrary of this. It is the reversal of all human experiences and of that universal and common good sense which is the essential condition of all human understanding, and which, in rising from the simple and unanimously recognized truth that twice two are four to the sublimest and most complex scientific considerations — admitting, moreover, nothing that has not stood the severest tests of experience or observation of things and facts — becomes the only serious basis of human knowledge.


Very far from pursuing the natural order from the lower to the higher, from the inferior to the superior, and from the relatively simple to the more complex; instead of wisely and rationally accompanying the progressive and real movement from the world called inorganic to the world organic, vegetables, animal, and then distinctively human — from chemical matter or chemical being to living matter or living being, and from living being to thinking being — the idealists, obsessed, blinded, and pushed on by the divine phantom which they have inherited from theology, take precisely the opposite course. They go from the higher to the lower, from the superior to the inferior, from the complex to the simple. They begin with God, either as a person or as divine substance or idea, and the first step that they take is a terrible fall from the sublime heights of the eternal ideal into the mire of the material world; from absolute perfection into absolute imperfection; from thought to being, or rather, from supreme being to nothing. When, how, and why the divine being, eternal, infinite, absolutely perfect, probably weary of himself, decided upon this desperate\emph{ salto mortale} is something which no idealist, no theologian, no metaphysician, no poet, has ever been able to understand himself or explain to the profane. All religions, past and present, and all the systems of transcendental philosophy hinge on this unique and iniquitous mystery.\footnote{I call it “iniquitous” because, as I believe I have proved In the Appendix alluded to, this mystery has been and still continues to be the consecration of all the horrors which have been and are being committed in the world; I call it unique, because all the other theological and metaphysical absurdities which debase the human mind are but its necessary consequences.}


Holy men, inspired lawgivers, prophets, messiahs, have searched it for life, and found only torment and death. Like the ancient sphinx, it has devoured them, because they could not explain it. Great philosophers from Heraclitus and Plato down to Descartes, Spinoza: Leibnitz, Kant, Fichte, Schelling, and Hegel, not to mention the Indian philosophers, have written heaps of volumes and built systems as ingenious as sublime, in which they have said by the way many beautiful and grand things and discovered immortal truths, but they have left this mystery, the principal object of their transcendental investigations, as unfathomable as before. The gigantic efforts of the most Wonderful geniuses that the world has known, and who, one after another, for at least thirty centuries, have undertaken anew this labor of Sisyphus, have resulted only in rendering this mystery still more incomprehensible. Is it to be hoped that it will be unveiled to us by the routine speculations of some pedantic disciple of an artificially warmed-over metaphysics at a time when all living and serious spirits have abandoned that ambiguous science born of a compromise — historically explicable no doubt — between the unreason of faith and sound scientific reason?


It is evident that this terrible mystery is inexplicable — that is, absurd, because only the absurd admits of no explanation. It is evident that whoever finds it essential to his happiness and life must renounce his reason, and return, if he can, to naive, blind, stupid faith, to repeat with Tertullianus and all sincere believers these words, which sum up the very quintessence of theology: \emph{Credo quia absurdum}. Then all discussion ceases, and nothing remains but the triumphant stupidity of faith. But immediately there arises another question: \emph{How comes an intelligent and well-informed man ever to feel the need of believing in this mystery?}


Nothing is more natural than that the belief in God, the creator, regulator, judge, master, curser, savior, and benefactor of the world, should still prevail among the people, especially in the rural districts, where it is more widespread than among the proletariat of the cities. The people, unfortunately, are still very ignorant, and are kept in ignorance by the systematic efforts of all the governments, who consider this ignorance, not without good reason, as one of the essential conditions of their own power. Weighted down by their daily labor, deprived of leisure, of intellectual intercourse, of reading, in short of all the means and a good portion of the stimulants that develop thought in men, the people generally accept religious traditions without criticism and in a lump. These traditions surround them from infancy in all the situations of life, and artificially sustained in their minds by a multitude of official poisoners of all sorts, priests and laymen, are transformed therein into a sort of mental and moral babit, too often more powerful even than their natural good sense.


There is another reason which explains and in some sort justifies the absurd beliefs of the people — namely, the wretched situation to which they find themselves fatally condemned by the economic organization of society in the most civilized countries of Europe. Reduced, intellectually and morally as well as materially, to the minimum of human existence, confined in their life like a prisoner in his prison, without horizon, without outlet, without even a future if we believe the economists, the people would have the singularly narrow souls and blunted instincts of the bourgeois if they did not feel a desire to escape; but of escape there are but three methods — two chimerical and a third real. The first two are the dram-shop and the church, debauchery of the body or debauchery of the mind; the third is social revolution. Hence I conclude this last will be much more potent than all the theological propagandism of the freethinkers to destroy to their last vestige the religious beliefs and dissolute habits of the people, beliefs and habits much more intimately connected than is generally supposed. In substituting for the at once illusory and brutal enjoyments of bodily and spiritual licentiousness the enjoyments, as refined as they are real, of humanity developed in each and all, the social revolution alone will have the power to close at the same time all the dram-shops and all the churches.


Till then the people. Taken as a whole, will believe; and, if they have no reason to believe, they will have at least a right.


There is a class of people who, if they do not believe, must at least make a semblance of believing. This class comprising all the tormentors, all the oppressors, and all the exploiters of humanity; priests, monarchs, statesmen, soldiers, public and private financiers, officials of all sorts, policemen, gendarmes, jailers and executioners, monopolists, capitalists, tax-leeches, contractors and landlords, lawyers, economists, politicians of all shades, down to the smallest vendor of sweetmeats, all will repeat in unison those words of Voltaire:


“If God did not exist, it would be necessary to invent him.” For, you understand, “the people must have a religion.” That is the safety-valve.


There exists, finally, a somewhat numerous class of honest but timid souls who, too intelligent to take the Christian dogmas seriously, reject them in detail, but have neither the courage nor the strength nor the necessary resolution to summarily renounce them altogether. They abandon to your criticism all the special absurdities of religion, they turn up their noses at all the miracles, but they cling desperately to the principal absurdity; the source of all the others, to the miracle that explains and justifies all the other miracles, the existence of God. Their God is not the vigorous and powerful being, the brutally positive God of theology. It is a nebulous, diaphanous, illusory being that vanishes into nothing at the first attempt to grasp it; it is a mirage, an \emph{ignis fatugs}; that neither warms nor illuminates. And yet they hold fast to it, and believe that, were it to disappear, all would disappear with it. They are uncertain, sickly souls, who have lost their reckoning in the present civilisation, belonging to neither the present nor the future, pale phantoms eternally suspended between heaven and earth, and occupying exactly the same position between the politics of the bourgeois and the Socialism of the proletariat. They have neither the power nor the wish nor the determination to follow out their thought, and they waste their time and pains in constantly endeavouring to reconcile the irreconcilable. In public life these are known as bourgeois Socialists.


With them, or against them, discussion is out of the question. They are too puny.


But there are a few illustrious men of whom no one will dare to speak without respect, and whose vigorous health, strength of mind, and good intention no one will dream of calling in question. I need only cite the names of Mazzini, Michelet, Quinet, John Stuart Mill.\footnote{Mr. Stuart Mill is perhaps the only one whose serious idealism may be fairly doubted, and that for two resons: first, that if not absolutely the disciple, he is a passionate admirer, an adherent of the positive philosphy of Auguste Comte, a philosophy which, in spite of its numerous reservations, is realy Atheistic; second, that Mr. Stuart Mill is English, and in England to proclaim oneself an Atheist is to ostracise oneself, even at this late day.} Generous and strong souls, great hearts, great minds, great writers, and the first the heroic and revolutionary regenerator of a great nation, they are all apostles of idealism and bitter despisers and adversaries of materialism, and consequently of Socialism also, in philosophy as well as in politics.


Against them, then, we must discuss this question.


First, let it be remarked that not one of the illustrious men I have just named nor any other idealistic thinker of any consequence in our day has given any attention to the logical side of this question properly speaking. Not one has tried to settle philosophically the possibility of the divine \emph{salto mortale}; from the pure and eternal regions of spirit into the mire of the material world. Have they feared to approach this irreconcilable contradiction and despaired of solving it after the failures of the greatest geniuses of history, or have they looked upon it as already sufficiently well settled? That is their secret. The fact is that they have neglected the theoretical demonstration of the existence of a God, and have developed only its practical motives and consequences. They have treated it as a fact universally accepted, and, as such, no longer susceptible of any doubt whatever, for sole proof thereof limiting themselves to the establishment of the antiquity and this very universality of the belief in God.


This imposing unanimity, in the eyes of many illustrious men and writers to quote only the most famous of them who eloquently expressed it, Joseph de Maistre and the great Italian patriot, Giuseppe Mazzini — is of more value than all the demonstrations of science; and if the reasoning of a small number of logical and even very powerful, but isolated, thinkers is against it, so much the worse, they say, for these thinkers and their logic, for universal consent, the general and primitive adoption of an idea, has always been considered the most triumphant testimony to its truth. The I sentiment of the whole world, a conviction that is found and maintained always and everywhere, cannot be mistaken; it must have its root in a necessity absolutely inherent in the very nature of man. And since it has been established that all peoples, past and present, have believed and still believe in the existence of God, it is clear that those who have the misfortune to doubt it, whatever the logic that led them to this doubt, are abnormal exceptions, monsters.


Thus, then, the \emph{antiquity}; and \emph{universality}; of a belief should be regarded, contrary to all science and all logic, as sufficient and unimpeachable proof of its truth. Why?


Until the days of Copernicus and Galileo everybody believed that the sun revolved about the earth. Was not everybody mistaken? What is more ancient and more universal than slavery? Cannibalism perhaps. From the origin of historic society down to the present day there has been always and everywhere exploitation of the compulsory labour of the masses — slaves, serfs, or wage workers — by some dominant minority; oppression of the people by the Church and by the State. Must it be concluded that this exploitation and this oppression are necessities absolutely inherent in the very existence of human society? These are examples which show that the argument of the champions of God proves nothing.


Nothing, in fact, is as universal or as ancient as the iniquitous and absurd; truth and justice, on the contrary, are the least universal, the youngest features in the development of human society. In this fact, too, lies the explanation of a constant historical phenomenon — namely, the persecution of which those who first proclaim the truth have been and continue to be the objects at the hands of the official, privileged, and interested representatives of “universal” and “ancient” beliefs, and often also at the hands of the same masses who, after having tortured them, always end by adopting their ideas and rendering them victorious.


To us materialists and Revolutionary Socialists, there is nothing astonishing or terrifying in this historical phenomenon. Strong in our conscience, in our love of truth at all hazards, in that passion for logic which of itself alone constitutes a great power and outside of which there is no thought; strong in our passion for justice and in our unshakeable faith in the triumph of humanity over all theoretical and practical bestialities; strong, finally, in the mutual confidence and support given each other by the few who share our convictions — we resign ourselves to all the consequences of this historical phenomenon, in which we see the manifestation of a social law as natural, as necessary, and as invariable as all the other laws which govern the world.


This law is a logical, inevitable consequence of the \emph{animal origin}; of human society; for in face of all the scientific, physiological, psychological, and historical proofs accumulated at the present day, as well as in face of the exploits of the Germans conquering France, which now furnish so striking a demonstration thereof, it is no longer possible to really doubt this origin. But from the moment that this animal origin of man is accepted, all is explained. History then appears to us as the revolutionary negation, now slow, apathetic, sluggish, now passionate and powerful, of the past. It consists precisely in the progressive negation of the primitive animality of man by the development of his humanity. Man, a wild beast, cousin of the gorilla, has emerged from the profound darkness of animal instinct into the light of the mind, which explains in a wholly natural way all his past mistakes and partially consoles us for his present errors. He has gone out from animal slavery, and passing through divine slavery, a temporary condition between his animality and his humanity, he is now marching on to the conquest and realisation of human liberty. Whence it results that the antiquity of a belief, of an idea, far from proving anything in its favour, ought, on the contrary, to lead us to suspect it. For behind us is our animality and before us our humanity; human light, the only thing that can warm and enlighten us, the only thing that can emancipate us, give us dignity, freedom, and happiness, and realise fraternity among us, is never at the beginning, but, relatively to the epoch in which we live, always at the end of history. Let us, then, never look back, let us look ever forward; for forward is our sunlight, forward our salvation. If it is justifiable, and even useful and necessary, to turn back to study our past, it is only in order to establish what we have been and what we must no longer be, what we have believed and thought and what we must no longer believe or think, what we have done and what we must do nevermore.


So much for \emph{antiquity}. As for the \emph{universality}; of an error, it proves but one thing — the similarity, if not the perfect identity, of human nature in all ages and under all skies. And, since it is established that all peoples, at all periods of their life, have believed and still believe in God, we must simply conclude that the divine idea, an outcome of ourselves, is an error historically necessary in the development of humanity, and ask why and how it was produced in history and why an immense majority of the human race still accept it as a truth.


Until we shall account to ourselves for the manner in which the idea of a supernatural or divine world was developed and had to be developed in the historical evolution of the human conscience, all our scientific conviction of its absurdity will be in vain; until then we shall never succeed in destroying it in the opinion of the majority, because we shall never be able to attack it in the very depths of the hut man being where it had birth. Condemned to a fruitless struggle, without issue and without end, we should for ever have to content ourselves with fighting it solely on the surface, in its innumerable manifestations, whose absurdity will be scarcely beaten down by the blows of common sense before it will reappear in a new form no less nonsensical. While the root of all the absurdities that torment the world, belief in God, remains intact, it will never fail to bring forth new offspring. Thus, at the present time, in certain sections of the highest society, Spiritualism tends to establish itself upon the ruins of Christianity.


It is not only in the interest of the masses, it is in that of the health of our own minds, that we should strive to understand the historic genesis, the succession of causes which developed and produced the idea of God in the consciousness of men. In vain shall we call and believe ourselves Atheists, until we comprehend these causes, for, until then, we shall always suffer ourselves to be more or less governed by the clamours of this universal conscience whose secret we have not discovered; and, considering the natural weakness of even the strongest individual against the all-powerful influence of the social surroundings that trammel him, we are always in danger of relapsing sooner or later, in one way or another, into the abyss of religious absurdity. Examples of these shameful conversions are frequent in society today.

\mychapter{3}{II}


I have stated the chief practical reason of the power still exercised today over the masses by religious beliefs. These mystical tendencies do not signify in man so much an aberration of mind as a deep discontent at Heart. They are the instinctive and passionate protest of the human being against the narrowness, the platitudes, the sorrows, and the shame of a wretched existence. For this malady, I have already said, there is but one remedy — Social Revolution.


In the meantime I have endeavored to show the causes responsible for the birth and historical development of religious hallucinations in the human conscience. Here it is my purpose to treat this question of the existence of a God, or of the divine origin of the world and of man, solely from the standpoint of its moral and social utility, and I shall say only a few words, to better explain my thought, regarding the theoretical grounds of this belief.


All religions, with their gods, their demigods, and their prophets, their messiahs and their saints, were created by the credulous fancy of men who had not attained the full development and full possession of their faculties. Consequently, the religious heaven is nothing but a mirage in which man, exalted by ignorance and faith, discovers his own image, but enlarged and reversed — that is, \emph{divinized.} The history of religion, of the birth, grandeur, and decline of the gods who have succeeded one another in human belief, is nothing, therefore, but the development of the collective intelligence and conscience of mankind. As fast as they discovered, in the course of their historically progressive advance, either in themselves or in external nature, a power, a quality, or even any great defect whatever, they attributed them to their gods, after having exaggerated and enlarged them beyond measure, after the manner of children, by an act of their religious fancy. Thanks to this modesty and pious generosity of believing and credulous men, heaven has grown rich with the spoils of the earth, and, by a necessary consequence, the richer heaven became, the more wretched became humanity and the earth. God once installed, he was naturally proclaimed the cause, reason, arbiter and absolute disposer of all things: the world thenceforth was nothing, God was all; and man, his real creator, after having unknowingly extracted him from the void, bowed down before him, worshipped him, and avowed himself his creature and his slave.


Christianity is precisely the religion \emph{par excellence,} because it exhibits and manifests, to the fullest extent, the very nature and essence of every religious system, which is \emph{the impoverishment, enslavement, and annihilation of humanity for the benefit of divinity.}


God being everything, the real world and man are nothing. God being truth, justice, goodness, beauty, power, and life, man is falsehood, iniquity, evil, ugliness, impotence, and death. God being master, man is the slave. Incapable of finding justice, truth, and eternal life by his own effort, he can attain them only through a divine revelation. But whoever says revelation says revealers, messiahs, prophets, priests, and legislators inspired by God himself; and these, once recognized as the representatives of divinity on earth, as the holy instructors of humanity, chosen by God himself to direct it in the path of salvation, necessarily exercise absolute power. All men owe them passive and unlimited obedience; for against the divine reason there is no human reason, and against the justice of God no terrestrial justice holds. Slaves of God, men must also be slaves of Church and State, \emph{in so far as the State is consecrated by the Church.} This truth Christianity, better than all other religions that exist or have existed, understood, not excepting even the old Oriental religions, which included only distinct and privileged nations, while Christianity aspires to embrace entire humanity; and this truth Roman Catholicism, alone among all the Christian sects, has proclaimed and realized with rigorous logic. That is why Christianity is the absolute religion, the final religion; why the Apostolic and Roman Church is the only consistent, legitimate, and divine church.


With all due respect, then, to the metaphysicians and religious idealists, philosophers, politicians, or poets: \emph{The idea of God implies the abdication of human reason and justice; it is the most decisive negation of human liberty, and necessarily ends in the enslavement of mankind, both in theory and practice.}


Unless, then, we desire the enslavement and degradation of mankind, as the Jesuits desire it, as the \emph{mômiers,} pietists, or Protestant Methodists desire it, we may not, must not make the slightest concession either to the God of theology or to the God of metaphysics. He who, in this mystical alphabet, begins with A will inevitably end with Z; he who desires to worship God must harbor no childish illusions about the matter, but bravely renounce his liberty and humanity.


If God is, man is a slave; now, man can and must be free; then, God does not exist.


I defy anyone whomsoever to avoid this circle; now, therefore, let all choose.


Is it necessary to point out to what extent and in what manner religions debase and corrupt the people? They destroy their reason, the principal instrument of human emancipation, and reduce them to imbecility, the essential condition of their slavery. They dishonor human labor, and make it a sign and source of servitude. They kill the idea and sentiment of human justice, ever tipping the balance to the side of triumphant knaves, privileged objects of divine indulgence. They kill human pride and dignity, protecting only the cringing and humble. They stifle in the heart of nations every feeling of human fraternity, filling it with divine cruelty instead.


All religions are cruel, all founded on blood; for all rest principally on the idea of sacrifice — that is, on the perpetual immolation of humanity to the insatiable vengeance of divinity. In this bloody mystery man is always the victim, and the priest — a man also, but a man privileged by grace — is the divine executioner. That explains why the priests of all religions, the best, the most humane, the gentlest, almost always have at the bottom of their hearts — and, if not in their hearts, in their imaginations, in their minds (and we know the fearful influence of either on the hearts of men) — something cruel and sanguinary.


None know all this better than our illustrious contemporary idealists. They are learned men, who know history by heart; and, as they are at the same time living men, great souls penetrated with a sincere and profound love for the welfare of humanity, they have cursed and branded all these misdeeds, all these crimes of religion with an eloquence unparalleled. They reject with indignation all solidarity with the God of positive religions and with his representatives, past, present, and on earth.


The God whom they adore, or whom they think they adore, is distinguished from the real gods of history precisely in this — that he is not at all a positive god, defined in any way whatever, theologically or even metaphysically. He is neither the supreme being of Robespierre and J. J. Rousseau, nor the pantheistic god of Spinoza, nor even the at once immanent, transcendental, and very equivocal god of Hegel. They take good care not to give him any positive definition whatever, feeling very strongly that any definition would subject him to the dissolving power of criticism. They will not say whether be is a personal or impersonal god, whether he created or did not create the world; they will not even speak of his divine providence. All that might compromise him. They content themselves with saying “God” and nothing more. But, then, what is their God? Not even an idea; it is an aspiration.


It is the generic name of all that seems grand, good, beautiful, noble, human to them. But why, then, do they not say, “Man.” Ah! because King William of Prussia and Napoleon III, and all their compeers are likewise men: which bothers them very much. Real humanity presents a mixture of all I that is most sublime and beautiful with all that is vilest and most monstrous in the world. How do they get over this? Why, they call one \emph{divine} and the other \emph{bestial,} representing divinity and animality as two poles, between which they place humanity. They either will not or cannot understand that these three terms are really but one, and that to separate them is to destroy them.


They are not strong on logic, and one might say that they despise it. That is what distinguishes them from the pantheistical and deistical metaphysicians, and gives their ideas the character of a practical idealism, drawing its inspiration much less from the severe development of a thought than from the experiences, I might almost say the emotions, historical and collective as well as individual, of life. This gives their propaganda an appearance of wealth and vital power, but an appearance only; for life itself becomes sterile when paralyzed by a logical contradiction.


This contradiction lies here: they wish God, and they wish humanity. They persist in connecting two terms which, once separated, can come together again only to destroy each other. They say in a single breath: “God and the liberty of man,” “God and the dignity, justice, equality, fraternity, prosperity of men” — regardless of the fatal logic by virtue of which, if God exists, all these things are condemned to non-existence. For, if God is, he is necessarily the eternal, supreme, absolute master, and, if such a master exists, man is a slave; now, if he is a slave, neither justice, nor equality, nor fraternity, nor prosperity are possible for him. In vain, flying in the face of good sense and all the teachings of history, do they represent their God as animated by the tenderest love of human liberty: a master, whoever he may be and however liberal he may desire to show himself, remains none the less always a master. His existence necessarily implies the slavery of all that is beneath him. Therefore, if God existed, only in one way could he serve human liberty — by ceasing to exist.


A jealous lover of human liberty, and deeming it the absolute condition of all that we admire and respect in humanity, I reverse the phrase of Voltaire, and say that, \emph{if God really existed, it would be necessary to abolish him.}


The severe logic that dictates these words is far too evident to require a development of this argument. And it seems to me impossible that the illustrious men, whose names so celebrated and so justly respected I have cited, should not have been struck by it themselves, and should not have perceived the contradiction in which they involve themselves in speaking of God and human liberty at once. To have disregarded it, they must have considered this inconsistency or logical license \emph{practically} necessary to humanity’s well-being.


Perhaps, too, while speaking of \emph{liberty} as something very respectable and very dear in their eyes, they give the term a meaning quite different from the conception entertained by us, materialists and Revolutionary Socialists. Indeed, they never speak of it without immediately adding another word, \emph{authority} — a word and a thing which we detest with all our heart.


What is authority? Is it the inevitable power of the natural laws which manifest themselves in the necessary concatenation and succession of phenomena in the physical and social worlds? Indeed, against these laws revolt is not only forbidden — it is even impossible. We may misunderstand them or not know them at all, but we cannot disobey them; because they constitute the basis and fundamental conditions of our existence; they envelop us, penetrate us, regulate all our movements, thoughts, and acts; even when we believe that we disobey them, we only show their omnipotence.


Yes, we are absolutely the slaves of these laws. But in such slavery there is no humiliation, or, rather, it is not slavery at all. For slavery supposes an external master, a legislator outside of him whom he commands, while these laws are not outside of us; they are inherent in us; they constitute our being, our whole being, physically — intellectually, and morally: we live, we breathe, we act, we think, we wish only through these laws. Without them we are nothing, \emph{we are not.} Whence, then, could we derive the power and the wish to rebel against them?


In his relation to natural laws but one liberty is possible to man — that of recognizing and applying them on an ever-extending scale in conformity with the object of collective and individual emancipation or humanization which he pursues. These laws, once recognized, exercise an authority which is never disputed by the mass of men. One must, for instance, be at bottom either a fool or a theologian or at least a metaphysician, jurist, or bourgeois economist to rebel against the law by which twice two make four. One must have faith to imagine that fire will not burn nor water drown, except, indeed, recourse be had to some subterfuge founded in its turn on some other natural law. But these revolts, or, rather, these attempts at or foolish fancies of an impossible revolt, are decidedly, the exception; for, in general, it may be said that the mass of men, in their daily lives, acknowledge the government of common sense — that is, of the sum of the natural laws generally recognized — in an almost absolute fashion.


The great misfortune is that a large number of natural laws, already established as such by science, remain unknown to the masses, thanks to the watchfulness of these tutelary governments that exist, as we know, only for the good of the people. There is another difficulty — namely, that the major portion of the natural laws connected with the development of human society, which are quite as necessary, invariable, fatal, as the laws that govern the physical world, have not been duly established and recognized by science itself.


Once they shall have been recognized by science, and then from science, by means of an extensive system of popular education and instruction, shall have passed into the consciousness of all, the question of liberty will be entirely solved. The most stubborn authorities must admit that then there will be no need either of political organization or direction or legislation, three things which, whether they emanate from the will of the sovereign or from the vote of a parliament elected by universal suffrage, and even should they conform to the system of natural laws — which has never been the case and never will be the case — are always equally fatal and hostile to the liberty of the masses from the very fact that they impose upon them a system of external and therefore despotic laws.


The liberty of man consists solely in this: that he obeys natural laws because he has \emph{himself} recognized them as such, and not because they have been externally imposed upon him by any extrinsic will whatever, divine or human, collective or individual.


Suppose a learned academy, composed of the most illustrious representatives of science; suppose this academy charged with legislation for and the organization of society, and that, inspired only by the purest love of truth, it frames none but laws in absolute harmony with the latest discoveries of science. Well, I maintain, for my part, that such legislation and such organization would be a monstrosity, and that for two reasons: first, that human science is always and necessarily imperfect, and that, comparing what it has discovered with what remains to be discovered, we may say that it is still in its cradle. So that were we to try to force the practical life of men, collective as well as individual, into strict and exclusive conformity with the latest data of science, we should condemn society as well as individuals to suffer martyrdom on a bed of Procrustes, which would soon end by dislocating and stifling them, life ever remaining an infinitely greater thing than science.


The second reason is this: a society which should obey legislation emanating from a scientific academy, not because it understood itself the rational character of this legislation (in which case the existence of the academy would become useless), but because this legislation, emanating from the academy, was imposed in the name of a science which it venerated without comprehending — such a society would be a society, not of men, but of brutes. It would be a second edition of those missions in Paraguay which submitted so long to the government of the Jesuits. It would surely and rapidly descend to the lowest stage of idiocy.


But there is still a third reason which would render such a government impossible — namely that a scientific academy invested with a sovereignty, so to speak, absolute, even if it were composed of the most illustrious men, would infallibly and soon end in its own moral and intellectual corruption. Even today, with the few privileges allowed them, such is the history of all academies. The greatest scientific genius, from the moment that he becomes an academician, an officially licensed \emph{savant,} inevitably lapses into sluggishness. He loses his spontaneity, his revolutionary hardihood, and that troublesome and savage energy characteristic of the grandest geniuses, ever called to destroy old tottering worlds and lay the foundations of new. He undoubtedly gains in politeness, in utilitarian and practical wisdom, what he loses in power of thought. In a word, he becomes corrupted.


It is the characteristic of privilege and of every privileged position to kill the mind and heart of men. The privileged man, whether politically or economically, is a man depraved in mind and heart. That is a social law which admits of no exception, and is as applicable to entire nations as to classes, corporations, and individuals. It is the law of equality, the supreme condition of liberty and humanity. The principal object of this treatise is precisely to demonstrate this truth in all the manifestations of human life.


A scientific body to which had been confided the government of society would soon end by devoting itself no longer to science at all, but to quite another affair; and that affair, as in the case of all established powers, would be its own eternal perpetuation by rendering the society confided to its care ever more stupid and consequently more in need of its government and direction.


But that which is true of scientific academies is also true of all constituent and legislative assemblies, even those chosen by universal suffrage. In the latter case they may renew their composition, it is true, but this does not prevent the formation in a few years’ time of a body of politicians, privileged in fact though not in law, who, devoting themselves exclusively to the direction of the public affairs of a country, finally form a sort of political aristocracy or oligarchy. Witness the United States of America and Switzerland.


Consequently, no external legislation and no authority — one, for that matter, being inseparable from the other, and both tending to the servitude of society and the degradation of the legislators themselves.


Does it follow that I reject all authority? Far from me such a thought. In the matter of boots, I refer to the authority of the bootmaker; concerning houses, canals, or railroads, I consult that of the architect or engineer. For such or such special knowledge I apply to such or such a \emph{savant.} But I allow neither the bootmaker nor the architect nor the \emph{savant} to impose his authority upon me. I listen to them freely and with all the respect merited by their intelligence, their character, their knowledge, reserving always my incontestable right of criticism censure. I do not content myself with consulting authority in any special branch; I consult several; I compare their opinions, and choose that which seems to me the soundest. But I recognize no infallible authority, even in special questions; consequently, whatever respect I may have for the honesty and the sincerity of such or such an individual, I have no absolute faith in any person. Such a faith would be fatal to my reason, to my liberty, and even to the success of my undertakings; it would immediately transform me into a stupid slave, an instrument of the will and interests of others.


If I bow before the authority of the specialists and avow my readiness to follow, to a certain extent and as long as may seem to me necessary, their indications and even their directions, it is because their authority is imposed upon me by no one, neither by men nor by God. Otherwise I would repel them with horror, and bid the devil take their counsels, their directions, and their services, certain that they would make me pay, by the loss of my liberty and self-respect, for such scraps of truth, wrapped in a multitude of lies, as they might give me.


I bow before the authority of special men because it is imposed upon me by my own reason. I am conscious of my inability to grasp, in all its details and positive developments, any very large portion of human knowledge. The greatest intelligence would not be equal to a comprehension of the whole. Thence results, for science as well as for industry, the necessity of the division and association of labor. I receive and I give — such is human life. Each directs and is directed in his turn. Therefore there is no fixed and constant authority, but a continual exchange of mutual, temporary, and, above all, voluntary authority and subordination.


This same reason forbids me, then, to recognize a fixed, constant, and universal authority, because there is no universal man, no man capable of grasping in that wealth of detail, without which the application of science to life is impossible, all the sciences, all the branches of social life. And if such universality could ever be realized in a single man, and if be wished to take advantage thereof to impose his authority upon us, it would be necessary to drive this man out of society, because his authority would inevitably reduce all the others to slavery and imbecility. I do not think that society ought to maltreat men of genius as it has done hitherto; but neither do I think it should indulge them too far, still less accord them any privileges or exclusive rights whatsoever; and that for three reasons: first, because it would often mistake a charlatan for a man of genius; second, because, through such a system of privileges, it might transform into a charlatan even a real man of genius, demoralize him, and degrade him; and, finally, because it would establish a master over itself.


To sum up. We recognize, then, the absolute authority of science, because the sole object of science is the mental reproduction, as well-considered and systematic as possible, of the natural laws inherent in the material, intellectual, and moral life of both the physical and the social worlds, these two worlds constituting, in fact, but one and the same natural world. Outside of this only legitimate authority, legitimate because rational and in harmony with human liberty, we declare all other authorities false, arbitrary and fatal.


We recognize the absolute authority of science, but we reject the infallibility and universality of the \emph{savant.} In our church — if I may be permitted to use for a moment an expression which I so detest: Church and State are my two \emph{bÍtes noires} — in our church, as in the Protestant church, we have a chief, an invisible Christ, science; and, like the Protestants, more logical even than the Protestants, we will suffer neither pope, nor council, nor conclaves of infallible cardinals, nor bishops, nor even priests. Our Christ differs from the Protestant and Christian Christ in this — that the latter is a personal being, ours impersonal; the Christian Christ, already completed in an eternal past, presents himself as a perfect being, while the completion and perfection of our Christ, science, are ever in the future: which is equivalent to saying that they will never be realized. Therefore, in recognizing \emph{absolute science} as the only absolute authority, we in no way compromise our liberty.


I mean by the words “absolute science,” which would reproduce ideally, to its fullest extent and in all its infinite detail, the universe, the system or coordination of all the natural laws manifested by the incessant development of the world. It is evident that such a science, the sublime object of all the efforts of the human mind, will never be fully and absolutely realized. Our Christ, then, will remain eternally unfinished, which must considerably take down the pride of his licensed representatives among us. Against that God the Son in whose name they assume to impose upon us their insolent and pedantic authority, we appeal to God the Father, who is the real world, real life, of which he (the Son) is only a too imperfect expression, whilst we real beings, living, working, struggling, loving, aspiring, enjoying, and suffering, are its immediate representatives.


But, while rejecting the absolute, universal, and infallible authority of men of science, we willingly bow before the respectable, although relative, quite temporary, and very restricted authority of the representatives of special sciences, asking nothing better than to consult them by turns, and very grateful for such precious information as they may extend to us, on condition of their willingness to receive from us on occasions when, and concerning matters about which, we are more learned than they. In general, we ask nothing better than to see men endowed with great knowledge, great experience, great minds, and, above all, great hearts, exercise over us a natural and legitimate influence, freely accepted, and never imposed in the name of any official authority whatsoever, celestial or terrestrial. We accept all natural authorities and all influences of fact, but none of right; for every authority or every influence of right, officially imposed as such, becoming directly an oppression and a falsehood, would inevitably impose upon us, as I believe I have sufficiently shown, slavery and absurdity.


In a word, we reject all legislation, all authority, and all privileged, licensed, official, and legal influence, even though arising from universal suffrage, convinced that it can turn only to the advantage of a dominant minority of exploiters against the interests of the immense majority in subjection to them.


This is the sense in which we are really Anarchists.


The modern idealists understand authority in quite a different way. Although free from the traditional superstitions of all the existing positive religions, they nevertheless attach to this idea of authority a divine, an absolute meaning. This authority is not that of a truth miraculously revealed, nor that of a truth rigorously and scientifically demonstrated. They base it to a slight extent upon quasi-philosophical reasoning, and to a large extent also on sentiment, ideally, abstractly poetical. Their religion is, as it were, a last attempt to divinise all that constitutes humanity in men.


This is just the opposite of the work that we are doing. On behalf of human liberty, dignity and prosperity, we believe it our duty to recover from heaven the goods which it has stolen and return them to earth. They, on the contrary, endeavouring to commit a final religiously heroic larceny, would restore to heaven, that divine robber, finally unmasked, the grandest, finest and noblest of humanity’s possessions. It is now the freethinker’s turn to pillage heaven by their audacious piety and scientific analysis.


The idealists undoubtedly believe that human ideas and deeds, in order to exercise greater authority among men, must be invested with a divine sanction. How is this sanction manifested? Not by a miracle, as in the positive religions, but by the very grandeur of sanctity of the ideas and deeds: whatever is grand, whatever is beautiful, whatever is noble, whatever is just, is considered divine. In this new religious cult every man inspired by these ideas, by these deeds, becomes a priest, directly consecrated by God himself. And the proof? He needs none beyond the very grandeur of the ideas which he expresses and the deeds which he performs. These are so holy that they can have been inspired only by God.


Such, in so few words, is their whole philosophy: a philosophy of sentiments, not of real thoughts, a sort of metaphysical pietism. This seems harmless, but it is not so at all, and the very precise, very narrow and very barren doctrine hidden under the intangible vagueness of these poetic forms leads to the same disastrous results that all the positive religions lead to — namely, the most complete negation of human liberty and dignity.


To proclaim as divine all that is grand, just, noble, and beautiful in humanity is to tacitly admit that humanity of itself would have been unable to produce it — that is, that, abandoned to itself, its own nature is miserable, iniquitous, base, and ugly. Thus we come back to the essence of all religion — in other words, to the disparagement of humanity for the greater glory of divinity. And from the moment that the natural inferiority of man and his fundamental incapacity to rise by his own effort, unaided by any divine inspiration, to the comprehension of just and true ideas, are admitted, it becomes necessary to admit also all the theological, political, and social consequences of the positive religions. From the moment that God, the perfect and supreme being, is posited face to face with humanity, divine mediators, the elect, the inspired of God spring from the earth to enlighten, direct, and govern in his name the human race.


May we not suppose that all men are equally inspired by God? Then, surely, there is no further use for mediators. But this supposition is impossible, because it is too clearly contradicted by the facts. It would compel us to attribute to divine inspiration all the absurdities and errors which appear, and all the horrors, follies, base deeds, and cowardly actions which are committed, in the world. But perhaps, then, only a few men are divinely inspired, the great men of history, the \emph{virtuous geniuses}, as the illustrious Italian citizen and prophet, Giuseppe Mazzini, called them. Immediately inspired by God himself and supported upon universal consent expressed by popular suffrage — \emph{Dio e Popolo}; — such as these should be called to the government of human societies.\footnote{In London I once heard M. Louis Blanc express almost the same idea. “The best form of government,” said he to me, “would be that which would invariably call men of virtuous genius to the control of affairs.”}


But here we are again fallen back under the yoke of Church and State. It is true that in this new organization, indebted for its existence, like all the old political organisations, to the \emph{grace of God}, but supported this time — at least so far as form is concerned, as a necessary concession to the spirit of modern times, and just as in the preambles of the imperial decrees of Napoleon III. — on the (pretended) \emph{will of the people}, the Church will no longer call itself Church; it will call itself School. What matters it? On the benches of this School will be seated not children only; there will be found the eternal minor, the pupil confessedly forever incompetent to pass his examinations, rise to the knowledge of his teachers, and dispense with their discipline — the people.\footnote{One day I asked Mazzini what measures would be taken for the emancipation of the people, once his triumphant unitary republic had been definitely established. “The first measure,” he answered “will be the foundation of schools for the people.” “And what will the people be taught in these schools?” “The duties of man — sacrifice and devotion.” But where will you find a sufficient number of professors to teach these things, which no one has the right or power to teach, unless he preaches by example? Is not the number of men who find supreme enjoyment in sacrifice and devotion exceedingly limited? Those who sacrifice themselves in the service of a great idea obey a lofty passion, and, \emph{satisfying this personal passion}, outside of which life itself loses all value in their eyes, they generally think of something else than building their action into doctrine, while those who teach doctrine usually forget to translate it into action, for the simple reason that doctrine kills the life, the living spontaneity, of action. Men like Mazzini, in whom doctrine and action form an admirable unity, are very rare exceptions. In Christianity also there have been great men, holy men, who have really practised, or who, at least, have passionately tried to practice all that they preached, and whose hearts, overflowing with love, were full of contempt for the pleasures and goods of this world. But the immense majority of Catholic and Protestant priests who, by trade, have preached and still preach the doctrines of chastity, abstinence, and renunciation belie their teachings by their example. It is not without reason, but because of several centuries’ experience, that among the people of all countries these phrases have become by-words: \emph{As licentious as a priest; as gluttonous as a priest; as ambitious as a priest; as greedy, selfish, and grasping as a priest}. It is, then, established that the professors of the Christian virtues, consecrated by the Church, the priests, \emph{in the immense majority of cases}, have practised quite the contrary of what they have preached. This very majority, the universality of this fact, show that the fault is not to be attributed to them as individuals, but to the social position, impossible and contradictory in itself, in which these individuals are placed. The position of the Christian priest involves a double contradiction. In the first place, that between the doctrine of abstinence and renunciation and the positive tendencies and needs of human nature — tendencies and needs which, in some individual cases, always very rare, may indeed be continually held back, suppressed, and even entirely annihilated by the constant influence of some potent intellectual and moral passion; which at certain moments of collective exaltation, may be forgotten and neglected for some time by a large mass of men at once; but which are so fundamentally inherent in our nature that sooner or later they always resume their rights: so that, when they are not satisfied in a regular and normal way, they are always replaced at last by unwholesome and monstrous satisfaction. This is a natural and consequently fatal and irresistible law, under the disastrous action of which inevitably fall all Christian priests and especially those of the Roman Catholic Church. It cannot apply to the professors, that is to the priests of the modern Church, unless they are also obliged to preach Christian abstinence and renunciation. But there is another contradiction common to the priests of both sects. This contradiction grows out of the very title and position of the master. A master who commands, oppresses, and exploits is a wholly logical and quite natural personage. But a master who sacrifices himself to those who are subordinated to him by his divine or human privilege is a contradictory and quite impossible being. This is the very constitution of hypocrisy, so well personified by the Pope, who, while calling himself \emph{the lowest servant of the servants of God} — in token whereof, following the example of Christ, he even washes once a year the feet of twelve Roman beggars — proclaims himself at the same time vicar of God, absolute and infallible master of the world. Do I need to recall that the priests of all churches, far from sacrificing themselves to the flocks confided to their care, have always sacrificed them, exploited them, and kept them in the condition of a flock, partly to satisfy their own personal passions and partly to serve the omnipotence of the Church? Like conditions, like causes, always produce like effects. It will, then, be the same with the professors of the modern School divinely inspired and licensed by the State. They will necessarily become, some without knowing it, others with full knowledge of the cause, teachers of the doctrine of popular sacrifice to the power of the State and to the profit of the privileged classes. Must we, then, eliminate from society all instruction and abolish all schools? Far from it! Instruction must be spread among the masses without stint, transforming all the churches, all those temples dedicated to the glory of God and to the slavery of men, into so many schools of human emancipation. But, in the first place, let us understand each other; schools, properly speaking, in a normal society founded on equality and on respect for human liberty, will exist only for children and not for adults: and, in order that they may become schools of emancipation and not of enslavement, it will be necessary to eliminate, first of all, this fiction of God, the eternal and absolute enslaver. The whole education of children and their instruction must be founded on the scientific development of reason, not on that of faith; on the development of personal dignity and independence, not on that of piety and obedience; on the worship of truth and justice at any cost, and above all on respect for humanity, which must replace always and everywhere the worship of divinity. The principle of authority, in the education of children, constitutes the natural point of departure; it is legitimate, necessary, when applied to children of a tender age, whose intelligence has not yet openly developed itself. But as the development of everything, and consequently of education, implies the gradual negation of the point of departure, this principle must diminish as fast as education and instruction advance, giving place to increasing liberty. All rational education is at bottom nothing but this progressive immolation of authority for the benefit of liberty, the final object of education necessarily being the formation of free men full of respect and love for the liberty of others. Therefore the first day of the pupils’ life, if the school takes infants scarcely able as yet to stammer a few words, should be that of the greatest authority and an almost entire absence of liberty; but its last day should be that of the greatest liberty and the absolute abolition of every vestige of the animal or divine principle of authority. The principle of authority, applied to men who have surpassed or attained their majority, becomes a monstrosity, a flagrant denial of humanity, a source of slavery and intellectual and moral depravity. Unfortunately, paternal governments have left the masses to wallow in an ignorance so profound that it will be necessary to establish schools not only for the people’s children, but for the people themselves. From these schools will be absolutely eliminated the smallest applications or manifestations of the principle of authority. They will be schools no longer; they will be popular academies, in which neither pupils nor masters will be known, where the people will come freely to get, if they need it, free instruction, and in which, rich in their own experience, they will teach in their turn many things to the professors who shall bring them knowledge which they lack. This, then, will be a mutual instruction, an act of intellectual fraternity between the educated youth and the people. The real school for the people and for all grown men is life. The only grand and omnipotent authority, at once natural and rational, the only one which we may respect, will be that of the collective and public spirit of a society founded on equality and solidarity and the mutual human respect of all its members. Yes. this is an authority which is not at all divine, wholly human, but before which we shall bow willingly, certain that, far from enslaving them, it will emancipate men. It will be a thousand times more powerful, be sure of it than all your divine, theological metaphysical, political, and judicial authorities, established by the Church and by the State, more powerful than your criminal codes, your jailers, and your executioners. The power of collective sentiment or public spirit is even now a very serious matter. The men most ready to commit crimes rarely dare to defy it, to openly affront it. They will seek to deceive it, but will take care not to be rude with it unless they feel the support of a minority larger or smaller. No man, however powerful he believes himself, will ever have the strength to bear the unanimous contempt of society; no one can live without feeling himself sustained by the approval and esteem of at least some portion of society. A man must be urged on by an immense and very sincere conviction in order to find courage to speak and act against the opinion of all, and never will a selfish, depraved, and cowardly man have such courage. Nothing proves more clearly than this fact the natural and inevitable solidarity — this law of sociability — which binds all men together, as each of us can verify daily, both on himself and on all the men whom he knows But, if this social power exists, why has it not sufficed hitherto to moralise, to humanise men? Simply because hitherto this power has not been humanised itself; it has not been humanised because the social life of which it is ever the faithful expression is based, as we know, on the worship of divinity not on respect for humanity; on authority, not on liberty; on privilege, not on equality; on the exploitation, not on the brotherhood of men; on iniquity and falsehood, not on justice and truth. Consequently its real action, always in contradiction of the humanitarian theories which it professes, has constantly exercised a disastrous and depraving influence. It does not repress vices and crimes; it creates them. Its authority is consequently a divine, anti-human authority; its influence is mischievous and baleful. Do you wish to render its authority and influence beneficent and human? Achieve the social revolution. Make all needs really solidary, and cause the material and social interests of each to conform to the human duties of each. And to this end there is but one means: Destroy all the institutions of Inequality; establish the economic and social equality of all, and on this basis will arise the liberty the morality, the solidary humanity of all. I shall return to this, the most important question of Socialism.}


The State will no longer call itself Monarchy; it will call itself Republic: but it will be none the less the State — that is, a tutelage officially and regularly established by a minority of competent men, \emph{men of virtuous genius or talent}, who will watch and guide the conduct of this great, incorrigible, and terrible child, the people. The professors of the School and the functionaries of the State will call themselves republicans; but they will be none the less tutors, shepherds, and the people will remain what they have been hitherto from all eternity, a flock. Beware of shearers, for where there is a flock there necessarily must be shepherds also to shear and devour it.


The people, in this system, will be the perpetual scholar and pupil. In spite of its sovereignty, wholly fictitious, it will continue to serve as the instrument of thoughts, wills, and consequently interests not its own. Between this situation and what we call liberty, the only real liberty, there is an abyss. It will be the old oppression and old slavery under new forms; and where there is slavery there is misery, brutishness, real social \emph{materialism}, among the privileged classes as well as among the masses.


\emph{In defying human things the idealists always end in the triumph of a brutal materialism}. And this for a very simple reason: the divine evaporates and rises to its own country, heaven, while the brutal alone remains actually on earth.


Yes, the necessary consequence of theoretical idealism is practically the most brutal materialism; not, undoubtedly, among those who sincerely preach it — the usual result as far as they are concerned being that they are constrained to see all their efforts struck with sterility — but among those who try to realise their precepts in life, and in all society so far as it allows itself to be dominated by idealistic doctrines.


To demonstrate this general fact, which may appear strange at first, but which explains itself naturally enough upon further reflection, historical proofs are not lacking.


Compare the last two civilisations of the ancient world — the Greek and the Roman. Which is the most materialistic, the most natural, in its point of departure, and the most humanly ideal in its results? Undoubtedly the Greek civilisation. Which on the contrary, is the most abstractly ideal in its point of departure — sacrificing the material liberty of the man to the ideal liberty of the citizen, represented by the abstraction of judicial law, and the natural development of human society to the abstraction of the State — and which became nevertheless the most brutal in its consequences? The Roman civilisation, certainly. It is true that the Greek civilisation, like all the ancient civilisations, including that of Rome, was exclusively national and based on slavery. But, in spite of these two immense defects, the former none the less conceived and realised the idea of humanity; it ennobled and really idealised the life of men; it transformed human herds into free associations of free men; it created through liberty the sciences, the arts, a poetry, an immortal philosophy, and the primary concepts of human respect. With political and social liberty, it created free thought. At the close of the Middle Ages, during the period of the Renaissance, the fact that some Greek emigrants brought a few of those immortal books into Italy sufficed to resuscitate life, liberty, thought, humanity, buried in the dark dungeon of Catholicism. Human emancipation, that is the name of the Greek civilisation. And the name of the Roman civilisation? Conquest, with all its brutal consequences. And its last word? The omnipotence of the Caesars. Which means the degradation and enslavement of nations and of men.


Today even, what is it that kills, what is it that crushes brutally, materially, in all European countries, liberty and humanity? It is the triumph of the Caesarian or Roman principle.


Compare now two modern civilisations — the Italian and the German. The first undoubtedly represents, in its general character, materialism; the second, on the contrary, represents idealism in its most abstract, most pure, and most transcendental form. Let us see what are the practical fruits of the one and the other.


Italy has already rendered immense services to the cause of human emancipation. She was the first to resuscitate and widely apply the principle of liberty in Europe, and to restore to humanity its titles to nobility: industry, commerce, poetry, the arts, the positive sciences, and free thought. Crushed since by three centuries of imperial and papal despotism, and dragged in the mud by her governing bourgeoisie, she reappears today, it is true, in a very degraded condition in comparison with what she once was. And yet how much she differs from Germany! In Italy, in spite of this decline — temporary let us hope — one may live and breathe humanly, surrounded by a people which seems to be born for liberty. Italy, even bourgeois Italy, can point with pride to men like Mazzini and Garibaldi. .In Germany one breathes the atmosphere of an immense political and social slavery, philosophically explained and accepted by a great people with deliberate resignation and free will. Her heroes — I speak always of present Germany, not of the Germany of the future; of aristocratic, bureaucratic, political and bourgeoisie Germany, not of the Germany of the \emph{prolétaires} — her heroes are quite the opposite of Mazzini and Garibaldi: they are William I., that ferocious and ingenuous representative of the Protestant God, Messrs, Bismarck and Moltke, Generals Manteuffel and Werder. In all her international relations Germany, from the beginning of her existence, has been slowly, systematically invading, conquering, ever ready to extend her own voluntary enslavement into the territory of her neighbours; and, since her definitive establishment as a unitary power, she has become a menace, a danger to the liberty of entire Europe. Today Germany is servility brutal and triumphant.


To show how theoretical idealism incessantly and inevitably changes into practical materialism, one needs only to cite the example of all the Christian Churches, and, naturally, first of all, that of the Apostolic and Roman Church. What is there more sublime, in the ideal sense, more disinterested, more separate from all the interests of this earth, than the doctrine of Christ preached by that Church? And what is there more brutally materialistic than the constant practice of that same Church since the eighth century, from which dates her definitive establishment as a power? What has been and still is the principal object of all her contests with the sovereigns of Europe? Her temporal goods, her revenues first, and then her temporal power, her political privileges. We must do her the justice to acknowledge that she was the first to discover, in modern history, this incontestable but scarcely Christian truth that wealth and power, the economic exploitation and the political oppression of the masses, are the two inseparable terms of the reign of divine ideality on earth: wealth consolidating and augmenting power, power ever discovering and creating new sources of wealth, and both assuring, better than the martyrdom and faith of the apostles, better than divine grace, the success of the Christian propagandism. This is a historical truth, and the Protestant Churches do not fail to recognise it either. I speak, of course, of the independent churches of England, America, and Switzerland, not of the subjected churches of Germany. The latter have no initiative of their own; they do what their masters, their temporal sovereigns, who are at the same time their spiritual chieftains, order them to do, It is well known that the Protestant propagandism, especially in England and America, is very intimately connected with the propagandism of the material, commercial interests of those two great nations; and it is known also that the objects of the latter propagandism is not at all the enrichment and material prosperity of the countries into which it penetrates in company with the Word of God, but rather the exploitation of those countries with a view to the enrichment and material prosperity of certain classes, which in their own country are very covetous and very pious at the same time.


In a word, it is not at all difficult to prove, history in hand, that the Church, that all the Churches, Christian and non-Christian, by the side of their spiritualistic propagandism, and probably to accelerate and consolidate the success thereof, have never neglected to organise themselves into great corporations for the economic exploitation of the masses under the protection and with the direct and special blessing of some divinity or other; that all the States, which originally, as we know, with all their political and judicial institutions and their dominant and privileged classes have been only temporal branches of these various Churches have likewise had principally in view this same exploitation for the benefit of lay minorities indirectly sanctioned by the Church; finally and in general, that the action of the good God and of all the divine idealities on earth has ended at last, always and everywhere, in founding the prosperous materialism of the few over the fanatical and constantly famishing idealism of the masses.


We have a new proof of this in what we see today. With the exception of the great hearts and great minds whom I have before referred to as misled, who are today the most obstinate defenders of idealism? In the first places all the sovereign courts. In France, until lately, Napoleon III. and his wife, Madame Eugénie; all their former ministers, courtiers, and ex-marshals, from Rouher and Bazaine to Fleury and Piétri; the men and women of this imperial world, who have so completely idealised and saved France; their journalists and their \emph{savants} — the Cssagnacs, the Girardins, the Duvernois, the Veuillots, the Leverriers, the Dumas; the black phalanx of Jesuits and Jesuitesses in every garb; the whole upper and middle bourgeoisie of France; the doctrinaire liberals, and the liberals without doctrine — the Guizots, the Thiers, the Jules Favres, the Pelletans, and the Jules Simons, all obstinate defenders of the bourgeoisie exploitation. In Prussia, in Germany, William I., the present royal demonstrator of the good God on earth; all his generals, all his officers, Pomeranian and other; all his army, which, strong in its religious faith, has just conquered France in that ideal way we know so well. In Russia, the Czar and his court; the Mouravieffs and the Bergs, all the butchers and pious proselyters of Poland. Everywhere, in short, religious or philosophical idealism, the one being but the more or less free translation of the other, serves today as the flag of material, bloody, and brutal force, of shameless material exploitation; while, on the contrary, the flag of theoretical materialism, the red flag of economic equality and social justice, is raised by the practical idealism of the oppressed and famishing masses, tending to realise the greatest liberty and the human right of each in the fraternity of all men on the earth.


Who are the real idealists — the idealists not of abstraction, but of life, not of heaven, but of earth — and who are the materialists?


It is evident that the essential condition of theoretical or divine idealism is the sacrifice of logic, of human reason, the renunciation of science. We see, further, that in defending the doctrines of idealism one finds himself enlisted perforce in the ranks of the oppressors and exploiters of the masses. These are two great reasons which, it would seem, should be sufficient to drive every great mind, every great heart, from idealism. How does it happen that our illustrious contemporary idealists, who certainly lack neither mind, nor heart, nor good will, and who have devoted their entire existence to the service of humanity — how does it happen that they persist in remaining among the representatives of a doctrine henceforth condemned and dishonoured?


They must be influenced by a very powerful motive. It cannot be logic or science, since logic and science have pronounced their verdict against the idealistic doctrine. No more can it be personal interests, since these men are infinitely above everything of that sort. It must, then, be a powerful moral motive. Which? There can be but one. These illustrious men think, no doubt, that idealistic theories or beliefs are essentially necessary to the moral dignity and grandeur of man, and that materialistic theories, on the contrary, reduce him to the level of the beasts.


And if the truth were just the opposite!


Every development, I have said, implies the negation of its point of departure. The basis or point of departure, according to the materialistic school, being material, the negation must be necessarily ideal. Starting from the totality of the real world, or from what is abstractly called matter, it logically arrives at the real idealisation — that is, at the humanisation, at the full and complete emancipation of society. \emph{Per contra}; and for the same reason, the basis and point of departure of the idealistic school being ideal, it arrives necessarily at the materialisation of society, at the organization of a brutal despotism and an iniquitous and ignoble exploitation, under the form of Church and State. The historical development of man according to the materialistic school, is a progressive ascension; in the idealistic system it can be nothing but a continuous fall.


Whatever human question we may desire to consider, we always find this same essential contradiction between the two schools. Thus, as I have already observed, materialism starts from animality to establish humanity; idealism starts from divinity to establish slavery and condemn the masses to an endless animality. Materialism denies free will and ends in the establishment of liberty; idealism, in the name of human dignity, proclaims free will, and on the ruins of every liberty founds authority. Materialism rejects the principle of authority, because it rightly considers it as the corollary of animality, and because, on the contrary, the triumph of humanity, the object and chief significance of history, can be realised only through liberty. In a word, you will always find the idealists in the very act of practical materialism, while you will see the materialists pursuing and realising the most grandly ideal aspirations and thoughts.


History, in the system of the idealists, as I have said, can be nothing but a continuous fall. They begin by a terrible fall, from which they never recover — by the \emph{salto mortale}; from the sublime regions of pure and absolute idea into matter. And into what kind of matter ! Not into the matter which is eternally active and mobile, full of properties and forces, of life and intelligence, as we see it in the real world; but into abstract matter, impoverished and reduced to absolute misery by the regular looting of these Prussians of thought, the theologians and metaphysicians, who have stripped it of everything to give everything to their emperor, to their God; into the matter which, deprived of all action and movement of its own, represents, in opposition to the divine idea, nothing but absolute stupidity, impenetrability, inertia and immobility.


The fall is so terrible that divinity, the divine person or idea, is flattened out, loses consciousness of itself, and never more recovers it. And in this desperate situation it is still forced to work miracles ! For from the moment that matter becomes inert, every movement that takes place in the world, even the most material, is a miracle, can result only from a providential intervention, from the action of God upon matter. And there this poor Divinity, degraded and half annihilated by its fall, lies some thousands of centuries in this swoon, then awakens slowly, in vain endeavouring to grasp some vague memory of itself, and every move that it makes in this direction upon matter becomes a creation, a new formation, a new miracle. In this way it passes through all degrees of materiality and bestiality — first, gas, simple or compound chemical substance, mineral, it then spreads over the earth as vegetable and animal organization till it concentrates itself in man. Here it would seem as if it must become itself again, for it lights in every human being an angelic spark, a particle of its own divine being, the immortal soul.


How did it manage to lodge a thing absolutely immaterial in a thing absolutely material; how can the body contain, enclose, limit, paralyse pure spirit? This, again, is one of those questions which faith alone, that passionate and stupid affirmation of the absurd, can solve. It is the greatest of miracles. Here, however, we have only to establish the effects, the practical consequences of this miracle.


After thousands of centuries of vain efforts to come back to itself, Divinity, lost and scattered in the matter which it animates and sets in motion, finds a point of support, a sort of focus for self-concentration. This focus is man his immortal soul singularly imprisoned in a mortal body. But each man considered individually is infinitely too limited, too small, to enclose the divine immensity; it can contain only a very small particle, immortal like the whole, but infinitely smaller than the whole. It follows that the divine being, the absolutely immaterial being, mind, is divisible like matter. Another mystery whose solution must be left to faith.


If God entire could find lodgment in each man, then each man would be God. We should have an immense quantity of Gods, each limited by all the others and yet none the less infinite — a contradiction which would imply a mutual destruction of men, an impossibility of the existence of more than one. As for the particles, that is another matter; nothing more rational, indeed, than that one particle should be limited by another and be smaller than the whole. Only, here another contradiction confronts us. To be limited, to be greater and smaller are attributes of matter, not of mind. According to the materialists, it is true, mind is only the working of the wholly material organism of man, and the greatness or smallness of mind depends absolutely on the greater or less material perfection of the human organism. But these same attributes of relative limitation and grandeur cannot be attributed to mind as the idealists conceive it, absolutely immaterial mind, mind existing independent of matter. There can be neither greater nor smaller nor any limit among minds, for there is only one mind — God. To add that the infinitely small and limited particles which constitute human souls are at the same time immortal is to carry the contradiction to a climax. But this is a question of faith. Let us pass on.


Here then we have Divinity torn up and lodged, in infinitely small particles, in an immense number of beings of all sexes, ages, races, and colours. This is an excessively inconvenient and unhappy situation, for the divine particles are so little acquainted with each other at the outset of their human existence that they begin by devouring each other. Moreover, in the midst of this state of barbarism and wholly animal brutality, these divine particles, human souls, retain as it were a vague remembrance of their primitive divinity, and are irresistibly drawn towards their whole; they seek each other, they seek their whole. It is Divinity itself, scattered and lost in the natural world, which looks for itself in men, and it is so demolished by this multitude of human prisons in which it finds itself strewn, that, in looking for itself, it commits folly after folly.


Beginning with fetishism, it searches for and adores itself, now in a stone, now in a piece of wood, now in a rag. It is quite likely that it would never have succeeded in getting out of the rag, if \emph{the other}; divinity which was not allowed to fall into matter and which is kept in a state of pure spirit in the sublime heights of the absolute ideal, or in the celestial regions, had not had pity on it.


Here is a new mystery — that of Divinity dividing itself into two halves, both equally infinite, of which one — God the Father — stays in the purely immaterial regions, and the other — God the Son — falls into matter. We shall see directly, between these two Divinities separated from each other, continuous relations established, from above to below and from below to above; and these relations, considered as a single eternal and constant act, will constitute the Holy Ghost. Such, in its veritable theological and metaphysical meaning, is the great, the terrible mystery of the Christian Trinity.


But let us lose no time in abandoning these heights to see what is going on upon earth.


God the Father, seeing from the height of his eternal splendour that the poor God the Son, flattened out and astounded by his fall, is so plunged and lost in matter that even having reached human state he has not yet recovered himself, decides to come to his aid. From this immense number of particles at once immortal, divine, and infinitely small, in which God the Son has disseminated himself so thoroughly that he does not know himself, God the Father chooses those most pleasing to him, picks his inspired persons, his prophets, his “men of virtuous genius,” the great benefactors and legislators of humanity: Zoroaster, Buddha, Moses, Confucius, Lycurgus, Solon, Socrates, the divine Plato, and above all Jesus Christ, the complete realisation of God the Son, at last collected and concentrated in a single human person; all the apostles, Saint Peter, Saint Paul, Saint John before all, Constantine the Great, Mahomet, then Charlemagne, Gregory VII Dante, and, according to some, Luther also, Voltaire and Rousseau, Robespierre and Danton, and many other great and holy historical personages, all of whose names it is impossible to recapitulate, but among whom I, as a Russian, beg that Saint Nicholas may not be forgotten.


Then we have reached at last the manifestation of God upon earth. But immediately God appears, man is reduced to nothing. It will be said that he is not reduced to nothing, since he is himself a particle of God. Pardon me! I admit that a particle of a definite, limited whole, however small it be, is a quantity, a positive greatness. But a particle of the infinitely great, compared with it, is necessarily infinitely small, Multiply milliards of milliards by milliards of milliards — their product compared to the infinitely great, will be infinitely small, and the infinitely small is equal to zero. God is everything; therefore man and all the real world with him, the universe, are nothing. You will not escape this conclusion.


God appears, man is reduced to nothing; and the greater Divinity becomes, the more miserable becomes humanity. That is the history of all religions; that is the effect of all the divine inspirations and legislations. In history the name of God is the terrible club with which all divinely inspired men, the great “virtuous geniuses,” have beaten down the liberty, dignity, reason, and prosperity of man.


We had first the fall of God. Now we have a fall which interests us more — that of man, caused solely by the apparition of God manifested on earth.


See in how profound an error our dear and illustrious idealists find themselves. In talking to us of God they purpose, they desire, to elevate us, emancipate us, ennoble us, and, on the contrary, they crush and degrade us. With the name of God they imagine that they can establish fraternity among men, and, on the contrary, they create pride, contempt; they sow discord, hatred, war; they establish slavery. For with God come the different degrees of divine inspiration; humanity is divided into men highly inspired, less inspired, uninspired. All are equally insignificant before God, it is true; but, compared with each other, some are greater than others; not only in fact — which would be of no consequence, because inequality in fact is lost in the collectivity when it cannot cling to some legal fiction or institution — but by the divine right of inspiration, which immediately establishes a fixed, constant, petrifying inequality. The highly inspired must be listened to and obeyed by the less inspired, and the less inspired by the uninspired. Thus we have the principle of authority well established, and with it the two fundamental institutions of slavery: Church and State.


Of all despotisms that of the \emph{doctrinaires}; or inspired religionists is the worst. They are so jealous of the glory of their God and of the triumph of their idea that they have no heart left for the liberty or the dignity or even the sufferings of living men, of real men. Divine zeal, preoccupation with the idea, finally dry up the tenderest souls, the most compassionate hearts, the sources of human love. Considering all that is, all that happens in the world from the point of view of eternity or of the abstract idea, they treat passing matters with disdain; but the whole life of real men, of men of flesh and bone, is composed only of passing matters; they themselves are only passing beings, who, once passed, are replaced by others likewise passing, but never to return in person. Alone permanent or relatively eternal in men is humanity, which steadily developing, grows richer in passing from one generation to another. I say \emph{relatively}; eternal, because, our planet once destroyed — it cannot fail to perish sooner or later, since everything which has begun must necessarily end — our planet once decomposed, to serve undoubtedly as an element of some new formation in the system of the universe, which alone is really eternal, who knows what will become of our whole human development? Nevertheless, the moment of this dissolution being an enormous distance in the future, we may properly consider humanity, relatively to the short duration of human life, as eternal. But this very fact of progressive humanity is real and living only through its manifestations at definite times, in definite places, in really living men, and not through its general idea.


The general idea is always an abstraction and, for that very reason, in some sort a negation of real life. I have stated in the Appendix that human thought and, in consequence of this, science can grasp and name only the general significance of real facts, their relations, their laws — in short, that which is permanent in their continual transformations — but never their material, individual side, palpitating, so to speak, with reality and life, and therefore fugitive and intangible. Science comprehends the thought of the reality, not reality itself; the thought of life, not life. That is its limit, its only really insuperable limit, because it is founded on the very nature of thought, which is the only organ of science.


Upon this nature are based the indisputable rights and grand mission of science, but also its vital impotence and even its mischievous action whenever, through its official licensed representatives, it arrogantly claims the right to govern life. The mission of science is, by observation of the general relations of passing and real facts, to establish the general laws inherent in the development of the phenomena of the physical and social world; it fixes, so to speak, the unchangeable landmarks of humanity’s progressive march by indicating the general conditions which it is necessary to rigorously observe and always fatal to ignore or forget. In a word, science is the compass of life; but it is not life itself. Science is unchangeable, impersonal, general, abstract, insensible, like the laws of which it is but the ideal reproduction, reflected or mental — that is cerebral (using this word to remind us that science itself is but a material product of a material organ, the \emph{brain}). Life is wholly fugitive and temporary, but also wholly palpitating with reality and individuality, sensibility, sufferings, joys, aspirations, needs, and passions. It alone spontaneously creates real things and; beings. Science creates nothing; it establishes and recognises only the creations of life. And every time that scientific men, emerging from their abstract world, mingle with living creation in the real world, all that they propose or create is poor, ridiculously abstract, bloodless and lifeless, still-born, like the homunculus created by Wagner, the pedantic disciple of the immortal Doctor Faust. It follows that the only mission of science is to enlighten life, not to govern it.


The government of science and of men of science, even be they positivists, disciples of Auguste Comte, or, again, disciples of the \emph{doctrinaire}; school of German Communism, cannot fail to be impotent, ridiculous, inhuman, cruel, oppressive, exploiting, maleficent. We may say of men of science, \emph{as such}, what I have said of theologians and metaphysicians: they have neither sense nor heart for individual and living beings. We cannot even blame them for this, for it is the natural consequence of their profession. In so far as they are men of science, they have to deal with and can take interest in nothing except generalities; that do the laws [\dots{}] [\textit{Three pages of the manuscript are missing}]


\dots{} they are not exclusively men of science, but are also more or less men of life.

\mychapter{4}{III}


Nevertheless, we must not rely too much on this. Though we may be well nigh certain that a \emph{savant}; would not dare to treat a man today as he treats a rabbit, it remains always to be feared that the \emph{savants}; as a body, if not interfered with, may submit living men to scientific experiments, undoubtedly less cruel but none the less disagreeable to their victims. If they cannot perform experiments upon the bodies of individuals, they will ask nothing better than to perform them on the social body, and that what must be absolutely prevented.


In their existing organisation, monopolising science and remaining thus outside of social life, the \emph{savants}; form a separate caste, in many respects analogous to the priesthood. Scientific abstractions is their God, living and real individuals are their victims, and they are the consecrated and licensed sacrificers.


Science cannot go outside of the sphere of abstractions. In this respect it is infinitely inferior to art, which, in its turn, is peculiarly concerned also with general types and general situations, but which incarnates them by an artifice of its own in forms which, if they are not living in the sense of real life none the less excite in our imagination the memory and sentiment of life; art in a certain sense individualizes the types and situations which it conceives; by means of the individualities without flesh and bone, and consequently permanent and immortal, which it has the power to create, it recalls to our minds the living, real individualities which appear and disappear under our eyes. Art, then, is as it were the return of abstraction to life; science, on the contrary, is the perpetual immolation of life, fugitive, temporary, but real, on the altar of eternal abstractions.


Science is as incapable of grasping the individuality of a man as that of a rabbit, being equally indifferent to both. Not that it is ignorant of the principle of individuality: it conceives it perfectly as a principle, but not as a fact. It knows very well that all the animal species, including the human species, have no real existence outside of an indefinite number of individuals, born and dying to make room for new individuals equally fugitive. It knows that in rising from the animal species to the superior species the principle of individuality becomes more pronounced; the individuals appear freer and more complete. It knows that man, the last and most perfect animal of earth, presents the most complete and most remarkable individuality, because of his power to conceive, concrete, personify, as it were, in his social and private existence, the universal law. It knows, finally, when it is not vitiated by theological or metaphysical, political or judicial \emph{doctrinairisme,} or even by a narrow scientific pride, when it is not deaf to the instincts and spontaneous aspirations of life — it knows (and this is its last word) that respect for man is the supreme law of Humanity, and that the great, the real object of history, its only legitimate object is the humanization and emancipation, the real liberty, the prosperity and happiness of each individual living in society. For, if we would not fall back into the liberticidal fiction of the public welfare represented by the State, a fiction always founded on the systematic sacrifice of the people, we must clearly recognize that collective liberty and prosperity exist only so far as they represent the sum of individual liberties and prosperities.


Science knows all these things, but it does not and cannot go beyond them. Abstraction being its very nature, it can well enough conceive the principle of real and living individuality, but it can have no dealings with real and living individuals; it concerns itself with individuals in general, but not with Peter or James, not with such or such a one, who, so far as it is concerned, do not, cannot, have any existence. Its individuals, I repeat, are only abstractions.


Now, history is made, not by abstract individuals, but by acting, living and passing individuals. Abstractions advance only when borne forward by real men. For these beings made, not in idea only, but in reality of flesh and blood, science has no heart: it considers them at most as \emph{material for intellectual and social development.} What does it care for the particular conditions and chance fate of Peter or James? It would make itself ridiculous, it would abdicate, it would annihilate itself, if it wished to concern itself with them otherwise than as examples in support of its eternal theories. And it would be ridiculous to wish it to do so, for its mission lies not there. It cannot grasp the concrete; it can move only in abstractions. Its mission is to busy itself with the situation and the \emph{general} conditions of the existence and development, either of the human species in general, or of such a race, such a people, such a class or category of individuals; the \emph{general} causes of their prosperity, their decline, and the best \emph{general} methods of securing, their progress in all ways. Provided it accomplishes this task broadly and rationally, it will do its whole duty, and it would be really unjust to expect more of it.


But it would be equally ridiculous, it would be disastrous to entrust it with a mission which it is incapable of fulfilling. Since its own nature forces it to ignore the existence of Peter and James, it must never be permitted, nor must anybody be permitted in its name, to govern Peter and James. For it were capable of treating them almost as it treats rabbits. Or rather, it would continue to ignore them; but its licensed representatives, men not at all abstract, but on the contrary in very active life and having very substantial interests, yielding to the pernicious influence which privilege inevitably exercises upon men, would finally fleece other men in the name of science, just as they have been fleeced hitherto by priests, politicians of all shades, and lawyers, in the name of God, of the State, of judicial Right.


What I preach then is, to a certain extent, \emph{the revolt of life against science,} or rather against the \emph{government} of science, not to destroy science — that would be high treason to humanity — but to remand it to its place so that it can never leave it again. Until now all human history has been only a perpetual and bloody immolation of millions of poor human beings in honor of some pitiless abstraction — God, country, power of State, national honor, historical rights, judicial rights, political liberty, public welfare. Such has been up to today the natural, spontaneous, and inevitable movement of human societies. We cannot undo it; we must submit to it so far as the past is concerned, as we submit to all natural fatalities. We must believe that that was the only possible way, to educate the human race. For we must not deceive ourselves: even in attributing the larger part to the Machiavellian wiles of the governing classes, we have to recognize that no minority would have been powerful enough to impose all these horrible sacrifices upon the masses if there had not been in the masses themselves a dizzy spontaneous movement which pushed them on to continual self-sacrifice, now to one, now to another of these devouring abstractions the vampires of history ever nourished upon human blood.


We readily understand that this is very gratifying, to the theologians, politicians, and jurists. Priests of these abstractions, they live only by the continual immolation of the people. Nor is it more surprising that metaphysics too, should give its consent. Its only mission is to justify and rationalize as far as possible the iniquitous and absurd. But that positive science itself should have shown the same tendencies is a fact which we must deplore while we establish it. That it has done so is due to two reasons: in the first place, because, constituted outside of life, it is represented by a privileged body; and in the second place, because thus far it has posited itself as an absolute and final object of all human development. By a judicious criticism, which it can and finally will be forced to pass upon itself, it would understand, on the contrary, that it is only a means for the realization of a much higher object — that of the complete humanization of the \emph{real} situation of all the \emph{real} individuals who are born, who live, and who die, on earth.


The immense advantage of positive science over theology, metaphysics, politics, and judicial right consists in this — that, in place of the false and fatal abstractions set up by these doctrines, it posits true abstractions which express the general nature and logic of things, their general relations, and the general laws of their development. This separates it profoundly from all preceding doctrines, and will assure it for ever a great position in society: it will constitute in a certain sense society’s collective consciousness. But there is one aspect in which it resembles all these doctrines: its only possible object being abstractions, it is forced by its very nature to ignore real men, outside of whom the truest abstractions have no existence. To remedy this radical defect positive science will have to proceed by a different method from that followed by the doctrines of the past. The latter have taken advantage of the ignorance of the masses to sacrifice them with delight to their abstractions, which by the way, are always very lucrative to those who represent them in flesh and bone. Positive science, recognizing its absolute inability to conceive real individuals and interest itself in their lot, must definitely and absolutely renounce all claim to the government of societies; for if it should meddle therein, it would only sacrifice continually the living men whom it ignores to the abstractions which constitute the sole object of its legitimate preoccupations.


The true science of history, for instance, does not yet exist; scarcely do we begin today to catch a glimpse of its extremely complicated conditions. But suppose it were definitely developed, what could it give us? It would exhibit a faithful and rational picture of the natural development of the general conditions — material and ideal, economical, political and social, religious, philosophical, aesthetic, and scientific — of the societies which have a history. But this universal picture of human civilization, however detailed it might be, would never show anything beyond general and consequently \emph{abstract} estimates. The milliards of individuals who have furnished the \emph{living and suffering materials} of this history at once triumphant and dismal — triumphant by its general results, dismal by the immense hecatomb of human victims “crushed under its car” — those milliards of obscure individuals without whom none of the great abstract results of history would have been obtained — and who, bear in mind, have never benefited by any of these results — will find no place, not even the slightest in our annals. They have lived and been sacrificed, crushed for the good of abstract humanity, that is all.


Shall we blame the science of history. That would be unjust and ridiculous. Individuals cannot be grasped by thought, by reflection, or even by human speech, which is capable of expressing abstractions only; they cannot be grasped in the present day any more than in the past. Therefore social science itself, the science of the future, will necessarily continue to ignore them. All that, we have a right to demand of it is that it shall point us with faithful and sure hand to the \emph{general causes of individual suffering} — among these causes it will not forget the immolation and subordination (still too frequent, alas!) of living individuals to abstract generalities — at the same time showing us the \emph{general conditions necessary to the real emancipation of the individuals living in society.} That is its mission; those are its limits, beyond which the action of social science can be only impotent and fatal. Beyond those limits being the \emph{doctrinaire} and governmental pretentious of its licensed representatives, its priests. It is time to have done with all popes and priests; we want them no longer, even if they call themselves Social Democrats.


Once more, the sole mission of science is to light the road. Only Life, delivered from all its governmental and \emph{doctrinaire} barriers, and given full liberty of action, can create.


How solve this antinomy?


On the one hand, science is indispensable to the rational organization of society; on the other, being incapable of interesting itself in that which is real and living, it must not interfere with the real or practical organization of society.


This contradiction can be solved only in one way: by the liquidation of science as a moral being existing outside the life of all, and represented by a body of breveted \emph{savants;} it must spread among the masses. Science, being called upon to henceforth represent society’s collective consciousness, must really become the property of everybody. Thereby, without losing anything of its universal character, of which it can never divest itself without ceasing to be science, and while continuing to concern itself exclusively with general causes, the conditions and fixed relations of individuals and things, it will become one in fact with the immediate and real life of all individuals. That will be a movement analogous to that which said to the Protestants at the beginning of the Reformation that there was no further need of priests for man, who would henceforth be his own priest, every man, thanks to the invisible intervention of the Lord Jesus Christ alone, having at last succeeded in swallowing his good God. But here the question is not of Jesus Christ, nor good God, nor of political liberty, nor of judicial right — things all theologically or metaphysically revealed, and all alike indigestible. The world of scientific abstractions is not revealed; it is inherent in the real world, of which it is only the general or abstract expression and representation. As long as it forms a separate region, specially represented by the \emph{savants} as a body, this ideal world threatens to take the place of a good God to the real world, reserving for its licensed representatives the office of priests. That is the reason why it is necessary to dissolve the special social organization of the \emph{savants} by general instruction, equal for all in all things, in order that the masses, ceasing to be flocks led and shorn by privileged priests, may take into their own hands the direction of their destinies.\footnote{Science, in becoming the patrimony of everybody, will wed itself in a certain sense to the immediate and real life of each. It will gain in utility and grace what it loses in pride, ambition, and \emph{doctrinaire} pedantry. This, however, will not prevent men of genius, better organized for scientific speculation than the majority of their fellows, from devoting themselves exclusively to the cultivation of the sciences, and rendering great services to humanity. Only, they will be ambitious for no other social influence than the natural influence exercised upon its surroundings by every superior intelligence, and for no other reward than the high delight which a noble mind always finds in the satisfaction of a noble passion.}


But until the masses shall have reached this degree of instruction, will it be necessary to leave them to the government of scientific men? Certainly not. It would be better for them to dispense with science than allow themselves to be governed by \emph{savants.} The first consequence of the government of these men would be to render science inaccessible to the people, and such a government would necessarily be aristocratic because the existing scientific institutions are essentially aristocratic. An aristocracy of learning! from the practical point of view the most implacable, and from the social point of view the most haughty and insulting — such would be the power established in the name of science. This \emph{régime} would be capable of paralyzing the life and movement of society. The \emph{savants} always presumptuous, ever self-sufficient and ever impotent, would desire to meddle with everything, and the sources of life would dry up under the breath of their abstractions.


Once more, Life, not science, creates life; the spontaneous action of the people themselves alone can create liberty. Undoubtedly it would be a very fortunate thing if science could, from this day forth, illuminate the spontaneous march of the people towards their emancipation. But better an absence of light than a false and feeble light, kindled only to mislead those who follow it. After all, the people will not lack light. Not in vain have they traversed a long historic career, and paid for their errors by centuries of misery. The practical summary of their painful experiences constitutes a sort of traditional science, which in certain respects is worth as much as theoretical science. Last of all, a portion of the youth — those of the bourgeois students who feel hatred enough for the falsehood, hypocrisy, injustice, and cowardice of the bourgeoisie to find courage to turn their backs upon it, and passion enough to unreservedly embrace the just and human cause of the proletariat — those will be, as I have already said, fraternal instructors of the people; thanks to them, there will be no occasion for the government of the \emph{savants.}


If the people should beware of the government of the \emph{savants}, all the more should they provide against that of the inspired idealists. The more sincere these believers and poets of heaven, the more dangerous they become. The scientific abstraction, I have said, is a rational abstraction, true in its essence, necessary to life, of which it is the theoretical representation, or, if one prefers, the conscience. It may, it must be, absorbed and digested by life. The idealistic abstraction, God, is a corrosive poison, which destroys and decomposes life, falsifies and kills it. The pride of the idealists, not being personal but divine, is invincible and inexorable: it may, it must, die, but it will never yield, and while it has a breath left it will try to subject men to its God, just as the lieutenants of Prussia, these practical idealists of Germany, would like to see the people crushed under the spurred boot of their emperor. The faith is the same, the end but little different, and the result, as that of faith, is slavery.


It is at the same time the triumph of the ugliest and most brutal materialism. There is no need to demonstrate this in the case of Germany; one would have to be blind to avoid seeing it at the present hour. But I think it is still necessary to demonstrate it in the case of divine idealism.


Man, like all the rest of nature, is an entirely material being. The mind, the facility of thinking, of receiving and reflecting upon different external and internal sensations, of remembering them when they have passed and reproducing them by the imagination, of comparing and distinguishing them, of abstracting determinations common to them and thus creating general concepts, and finally of forming ideas by grouping and combining concepts according to different methods — intelligence, in a word, sole creator of our whole, ideal world, is a property of the animal body and especially of the quite material organism of the brain.


We know this certainly, by the experience of all, which no fact has ever contradicted and which any man can verify at any moment of his life. In all animals, without excepting the wholly inferior species, we find a certain degree of intelligence, and we see that, in the series of species, animal intelligence develops in proportion as the organization of a species approaches that of man, but that in man alone it attains to that power of abstraction which properly constitutes thought.

\clearpage
Universal experience,\footnote{Universal \emph{experience}, on which all science rests, must be clearly distinguished from universal \emph{faith}, on which the idealists wish to support their beliefs: the first is a real authentication of facts; the second is only a supposition of facts which nobody has seen, and which consequently are at variance with the experience of everybody.} which is the sole origin, the source of all our knowledge, shows us, therefore, that all intelligence is always attached to some animal body, and that the intensity, the power, of this animal function depends on the relative perfection of the organism. The latter of these results of universal experience is not applicable only to the different animal species; we establish it likewise in men, whose intellectual and moral power depends so clearly upon the greater or less perfection of their organism as a race, as a nation, as a class, and as individuals, that it is not necessary to insist upon this point.\footnote{The idealists, all those who believe in the immateriality and immortality of the human soul, must be excessively embarrassed by the difference in intelligence existing between races, peoples, and individuals. Unless we suppose that the various divine particles have been irregularly distributed, how is this difference to be explained? Unfortunately there is a considerable number of men wholly stupid, foolish even to idiocy. Could they have received in the distribution a particle at once divine and stupid? To escape this embarrassment the idealists must necessarily suppose that all human souls are equal. but that the prisons in which they find themselves necessarily confined, human bodies, are unequal, some more capable than others of serving as an organ for the pure intellectuality of soul. According to this. such a one might have very fine organs at his disposition. such another very gross organs. But these are distinctions which idealism has not the power to use without falling into inconsistency and the grossest materialism, for in the presence of absolute immateriality of soul all bodily differences disappear, all that is corporeal, material, necessarily appearing indifferent, equally and absolutely gross. The abyss which separates soul from body, absolute immateriality from absolute materiality, is infinite. Consequently all differences, by the way inexplicable and logically impossible, which may exist on the other side of the abyss, in matter, should be to the soul null and void, and neither can nor should exercise any influence over it. In a word, the absolutely immaterial cannot be constrained, imprisoned, and much less expressed in any degree whatsoever by the absolutely material. Of all the gross and materialistic (using the word in the sense attached to it by the idealists) imaginations which were engendered by the primitive ignorance and stupidity of men, that of an immaterial soul imprisoned in a material body is certainly the grossest, the most stupid. and nothing better proves the omnipotence exercised by ancient prejudices even over the best minds than the deplorable sight of men endowed with lofty intelligence still talking of it in our days.}


On the other hand, it is certain that no man has ever seen or can see pure mind, detached from all material form existing separately from any animal body whatsoever. But if no person has seen it, how is it that men have come to believe in its existence? The fact of this belief is certain and if not universal, as all the idealists pretend, at least very general, and as such it is entirely worthy of our closest attention, for a general belief, however foolish it may be, exercises too potent a sway over the destiny of men to warrant us in ignoring it or putting it aside.


The explanation of this belief, moreover, is rational enough. The example afforded us by children and young people, and even by many men long past the age of majority, shows us that man may use his mental faculties for a long time before accounting to himself for the way in which he uses them, before becoming clearly conscious of it. During this working of the mind unconscious of itself, during this action of innocent or believing intelligence, man, obsessed by the external world, pushed on by that internal goad called life and its manifold necessities, creates a quantity of imaginations, concepts, and ideas necessarily very imperfect at first and conforming but slightly to the reality of the things and facts which they endeavour to express Not having yet the consciousness of his own intelligent action, not knowing yet that he himself has produced and continues to produce these imaginations, these concepts, these ideas, ignoring their wholly \emph{subjective }- that is, human-origin, he must naturally consider them as \emph{objective}; beings, as real beings, wholly independent of him, existing by themselves and in themselves.


It was thus that primitive peoples, emerging slowly from their animal innocence, created their gods. Having created them, not suspecting that they themselves were the real creators, they worshipped them; considering them as real beings infinitely superior to themselves, they attributed omnipotence to them, and recognised themselves as their creatures, their slaves. As fast as human ideas develop, the gods, who, as I have already stated, were never anything more than a fantastic, ideal, poetical reverberation of an inverted image, become idealised also. At first gross fetishes, they gradually become pure spirits, existing outside of the visible world, and at last, in the course of a long historic evolution, are confounded in a single Divine Being, pure, eternal, absolute Spirit, creator and master of the worlds.


In every development, just or false, real or imaginary collective or individual, it is always the first step, the first act that is the most difficult. That step once taken, the rest follows naturally as a necessary consequence. The difficult step in the historical development of this terrible religious insanity which continues to obsess and crush us was to posit a divine world as such, outside the world. This first act of madness, so natural from the physiological point of view and consequently necessary in the history of humanity, was not accomplished at a single stroke. I know not how many centuries were needed to develop this belief and make it a governing influence upon the mental customs of men. But, once established, it became omnipotent, as each insane notion necessarily becomes when it takes possession of man’s brain. Take a madman, whatever the object of his madness — you will find that obscure and fixed idea which obsesses him seems to him the most natural thing in the world, and that, on the contrary, the real things which contradict this idea seem to him ridiculous and odious follies. Well religion is a collective insanity, the more powerful because it is traditional folly, and because its origin is lost in the most remote antiquity. As collective insanity it has penetrated to the very depths of the public and private existence of the peoples; it is incarnate in society; it has become, so to speak, the collective soul and thought. Every man is enveloped in it from his birth; he sucks it in with his mother’s milk, absorbs it with all that he touches, all that he sees. He is so exclusive]y fed upon it, so poisoned and penetrated by it in all his being that later, however powerful his natural mind, he has to make unheard-of efforts to deliver himself from it, and then never completely succeeds. We have one proof of this in our modern idealists, and another in our \emph{doctrinaire}; materialists — the German Communists. They have found no way to shake off the religion of the State.


The supernatural world, the divine world, once well established in the imagination of the peoples, the development of the various religious systems has followed its natural and logical course, conforming, moreover, in all things to the contemporary development of economical and political relations of which it has been in all ages, in the world of religious fancy, the faithful reproduction and divine consecration. Thus has the collective and historical insanity which calls itself religion been developed since fetishism, passing through all the stages from polytheism to Christian monotheism.


The second step in the development of religious beliefs, undoubtedly the most difficult next to the establishment of a separate divine world, was precisely this transition from polytheism to monotheism, from the religious materialism of the pagans to the spiritualistic faith of the Christians. She pagan gods — and this was their principal characteristic — were first of all exclusively national gods. Very numerous, they necessarily retained a more or less material character, or, rather, they were so numerous because they were material, diversity being one of the principal attributes of the real world. The pagan gods were not yet strictly the negation of real things; they were only a fantastic exaggeration of them.


We have seen how much this transition cost the Jewish people, constituting, so to speak, its entire history. In vain did Moses and the prophets preach the one god; the people always relapsed into their primitive idolatry, into the ancient and comparatively much more natural and convenient faith in many good gods, more material, more human, and more palpable. Jehovah himself, their sole God, the God of Moses and the prophets, was still an extremely national God, who, to reward and punish his faithful followers, his chosen people, used material arguments, often stupid, always gross and cruel. It does not even appear that faith in his existence implied a negation of the existence of earlier gods. The Jewish God did not deny the existence of these rivals; he simply did not want his people to worship them side by side with him, because before all Jehovah was a very Jealous God. His first commandment was this:


“I am the Lord thy God, and thou shalt have no other gods before me.”


Jehovah, then, was only a first draft, very material and very rough, of the supreme deity of modern idealism. Moreover, he was only a national God, like the Russian God worshipped by the German generals, subjects of the Czar and patriots of the empire of all the Russias; like the German God, whom the pietists and the German generals, subjects of William I. at Berlin, will no doubt soon proclaim. The supreme being cannot be a national God; he must be the God of entire Humanity. Nor can the supreme being be a material being; he must be the negation of all matter — pure spirit. Two things have proved necessary to the realisation of the worship of the supreme being:


\begin{enumerate}
\item
a realisation, such as it is, of Humanity by the negation of nationalities and national forms of worship;
\item
a development, already far advanced, of metaphysical ideas in order to spiritualise the gross Jehovah of the Jews.
\end{enumerate}

The first condition was fulfilled by the Romans, though in a very negative way no doubt, by the conquest of most of the countries known to the ancients and by the destruction of their national institutions. The gods of all the conquered nations, gathered in the Pantheon, mutually cancelled each other. This was the first draft of humanity, very gross and quite negative.


As for the second condition, the spiritualisation of Jehovah, that was realised by the Greeks long before the conquest of their country by the Romans. They were the creators of metaphysics. Greece, in the cradle of her history, had already found from the Orient a divine world which had been definitely established in the traditional faith of her peoples; this world had been left and handed over to her by the Orient. In her instinctive period, prior to her political history, she had developed and prodigiously humanised this divine world through her poets; and when she actually began her history, she already had a religion readymade, the most sympathetic and noble of all the religions which have existed, so far at least as a religion — that is, a lie — can be noble and sympathetic. Her great thinkers — and no nation has had greater than Greece — found the divine world established, not only outside of themselves in the people, but also in themselves as a habit of feeling and thought, and naturally they took it as a point of departure. That they made no theology — that is, that they did not wait in vain to reconcile dawning reason with the absurdities of such a god, as did the scholastics of the Middle Ages — was already much in their favour. They left the gods out of their speculations and attached themselves directly to the divine idea, one, invisible, omnipotent, eternal, and absolutely spiritualistic but impersonal. As concerns Spiritualism, then, the Greek metaphysicians, much more than the Jews, were the creators of the Christian god. The Jews only added to it the brutal personality of their Jehovah.


That a sublime genius like the divine Plato could have been absolutely convinced of the reality of the divine idea shows us how contagious, how omnipotent, is the tradition of the religious mania even on the greatest minds. Besides, we should not be surprised at it, since, even in our day, the greatest philosophical genius which has existed since Aristotle and Plato, Hegel — in spite even of Kant’s criticism, imperfect and too metaphysical though it be, which had demolished the objectivity or reality of the divine ideas — tried to replace these divine ideas upon their transcendental or celestial throne. It is true that Hegel went about his work of restoration in so impolite a manner that he killed the good God for ever. He took away from these ideas their divine halo, by showing to whoever will read him that they were never anything more than a creation of the human mind running through history in search of itself. To put an end to all religious insanities and the divine \emph{mirage}, he left nothing lacking but the utterance of those grand words which were said after him, almost at the same time, by two great minds who had never heard of each other — Ludwig Feuerbach, the disciple and demolisher of Hegel, in Germany, and Auguste Comte, the founder of positive philosophy, in France. These words were as follows:


“Metaphysics are reduced to psychology.” All the metaphysical systems have been nothing else than human psychology developing itself in history.


To-day it is no longer difficult to understand how the divine ideas were born, how they were created in succession by the abstractive faculty of man. Man made the gods. But in the time of Plato this knowledge was impossible. The collective mind, and consequently the individual mind as well, even that of the greatest genius, was not ripe for that. Scarcely had it said with Socrates: “Know thyself!” This self-knowledge existed only in a state of intuition; in fact, it amounted to nothing. Hence it was impossible for the human mind to suspect that it was itself the sole creator of the divine world. It found the divine world before it; it found it as history, as tradition, as a sentiment, as a habit of thought; and it necessarily made it the object of its loftiest speculations. Thus was born metaphysics, and thus were developed and perfected the divine ideas, the basis of Spiritualism.


It is true that after Plato there was a sort of inverse movement in the development of the mind. Aristotle, the true father of science and positive philosophy, did not deny the divine world, but concerned himself with it as little as possible. He was the first to study, like the analyst and experimenter that he was, logic, the laws of human thought, and at the same time the physical world, not in its ideal, illusory essence, but in its real aspect. After him the Greeks of Alexandria established the first school of the positive scientists. They were atheists. But their atheism left no mark on their contemporaries. Science tended more and more to separate itself from life. After Plato, divine ideas were rejected in metaphysics themselves; this was done by the Epicureans and Sceptics, two sects who contributed much to the degradation of human aristocracy, but they had no effect upon the masses.


Another school, infinitely more influential, was formed at Alexandria. This was the school of neo-Platonists. These, confounding in an impure mixture the monstrous imaginations of the Orient with the ideas of Plato, were the true originators, and later the elaborators, of the Christian dogmas.


Thus the personal and gross egoism of Jehovah, the not less brutal and gross Roman conquest, and the metaphysical ideal speculation of the Greeks, materialised by contact with the Orient, were the three historical elements which made up the spiritualistic religion of the Christians.


Before the altar of a unique and supreme God was raised on the ruins of the numerous altars of the pagan gods, the autonomy of the various nations composing the pagan or ancient world had to be destroyed first. This was very brutally done by the Romans who, by conquering the greatest part of the globe known to the ancients, laid the first foundations, quite gross and negative ones no doubt, of humanity. A God thus raised above the national differences, material and social, of all countries, and in a certain sense the direct negation of them, must necessarily be an immaterial and abstract being. But faith in the existence of such a being, so difficult a matter, could not spring into existence suddenly. Consequently, as I have demonstrated in the Appendix, it went through a long course of preparation and development at the hands of Greek metaphysics, which were the first to establish in a philosophical manner the notion of \emph{the divine idea}, a model eternally creative and always reproduced by the visible world. But the divinity conceived and created by Greek philosophy was an impersonal divinity. No logical and serious metaphysics being able to rise, or, rather, to descend, to the idea of a personal God, it became necessary, therefore, to imagine a God who was one and very personal at once. He was found in the very brutal, selfish, and cruel person of Jehovah, the national God of the Jews. But the Jews, in spite of that exclusive national spirit which distinguishes them even to-day, had become in fact, long before the birth of Christ, the most international people of the world. Some of them carried away as captives, but many more even urged on by that mercantile passion which constitutes one of the principal traits of their character, they had spread through all countries, carrying everywhere the worship of their Jehovah, to whom they remained all the more faithful the more he abandoned them.


In Alexandria this terrible god of the Jews made the personal acquaintance of the metaphysical divinity of Plato, already much corrupted by Oriental contact, and corrupted her still more by his own. In spite of his national, jealous, and ferocious exclusivism, he could not long resist the graces of this ideal and impersonal divinity of the Greeks. He married her, and from this marriage was born the spiritualistic — but not spirited — God of the Christians. The neoplatonists of Alexandria are known to have been the principal creators of the Christian theology.


Nevertheless theology alone does not make a religion, any more than historical elements suffice to create history. By historical elements I mean the general conditions of any real development whatsoever — for example in this case the conquest of the world by the Romans and the meeting of the God of the Jews with the ideal of divinity of the Greeks. To impregnate the historical elements, to cause them to run through a series of new historical transformations, a living, spontaneous fact was needed, without which they might have remained many centuries longer in the state of unproductive elements. This fact was not lacking in Christianity: it was the propagandism, martyrdom, and death of Jesus Christ.


We know almost nothing of this great and saintly personage, all that the gospels tell us being contradictory, and so fabulous that we can scarcely seize upon a few real and vital traits. But it is certain that he was the preacher of the poor, the friend and consoler of the wretched, of the ignorant, of the slaves, and of the women, and that by these last he was much loved. He promised eternal life to all who are oppressed, to all who suffer here below; and the number is immense. He was hanged, as a matter of course, by the representatives of the official morality and public order of that period. His disciples and the disciples of his disciples succeeded in spreading, thanks to the destruction of the national barriers by the Roman conquest, and propagated the Gospel in all the countries known to the ancients. Everywhere they were received with open arms by the slaves and the women, the two most oppressed, most suffering, and naturally also the most ignorant classes of the ancient world. For even such few proselytes as they made in the privileged and learned world they were indebted in great part to the influence of women. Their most extensive propagandism was directed almost exclusively among the people, unfortunate and degraded by slavery. This was the first awakening, the first intellectual revolt of the proletariat.


The great honour of Christianity, its incontestable merit, and the whole secret of its unprecedented and yet thoroughly legitimate triumph, lay in the fact that it appealed to that suffering and immense public to which the ancient world, a strict and cruel intellectual and political aristocracy, denied even the simplest rights of humanity. Otherwise it never could have spread. The doctrine taught by the apostles of Christ, wholly consoling as it may have seemed to the unfortunate, was too revolting, too absurd from the standpoint of human reason, ever to have been accepted by enlightened men According with what joy the apostle Paul speaks of the \emph{scandale de la foi}; and of the triumph of that \emph{divine folie}; rejected by the powerful and wise of the century, but all the more passionately accepted by the simple, the ignorant, and the weak-minded!


Indeed there must have been a very deep-seated dissatisfaction with life, a very intense thirst of heart, and an almost absolute poverty of thought, to secure the acceptance of the Christian absurdity, the most audacious and monstrous of all religious absurdities.


This was not only the negation of all the political, social, and religious institutions of antiquity: it was the absolute overturn of common sense, of all human reason. The living being, the real world, were considered thereafter as nothing; whereas the product of man’s abstractive faculty, the last and supreme abstraction in which this faculty, far beyond existing things, even beyond the most general determinations of the living being, the ideas of space and time. having nothing left to advance beyond, rests in contemplation of his emptiness and absolute immobility.


That abstraction, that \emph{caput mortuum}, absolutely void of all contents the true nothing, God, is proclaimed the only real, eternal, all-powerful being. The real All is declared nothing and the absolute nothing the All. The shadow becomes the substance and the substance vanishes like a shadow.\footnote{I am well aware that in the theological and metaphysical systems of the Orient, and especially in those of India, including Buddhism, we find the principle of the annihilation of the real world in favour of the ideal and of absolute abstraction. But it has not the added character of voluntary and deliberate negation which distinguishes Christianity; when those systems were conceived. The world of human thought of will and of liberty, had not reached that stage of development which was afterwards seen in the Greek and Roman civilisation.}


All this was audacity and absurdity unspeakable, the true \emph{scandale de la foi}, the triumph of credulous stupidity over the mind for the masses; and — for a few — the triumphant irony of a mind wearied, corrupted, disillusioned, and disgusted in honest and serious search for truth; it was that necessity of shaking off thought and becoming brutally stupid so frequently felt by surfeited minds:

\chapter*{IV \\ \LARGE \emph{Credo quod absurdum}}
\addcontentsline{toc}{chapter}{IV \emph{Credo quod absurdum}}

I believe in the absurd; I believe in it, precisely and mainly, because it is absurd. In the same way many distinguished and enlightened minds in our day believe in animal magnetism, spiritualism, tipping tables, and — why go so far? — believe still in Christianity, in idealism, in God.


The belief of the ancient proletariat, like that of the modern, was more robust and simple, less \emph{haut goût}. The Christian propagandism appealed to its heart, not to its mind; to its eternal aspirations, its necessities, its sufferings, its slavery, not to its reason, which still slept and therefore could know nothing about logical contradictions and the evidence of the absurd. It was interested solely in knowing when the hour of promised deliverance would strike, when the kingdom of God would come. As for theological dogmas, it did not trouble itself about them because it understood nothing about them The proletariat converted to Christianity constituted its growing material but not its intellectual strength.


As for the Christian dogmas, it is known that they were elaborated in a series of theological and literary works and in the Councils, principally by the converted neo-Platonists of the Orient. The Greek mind had fallen so low that, in the fourth century of the Christian era, the period of the first Council, the idea of a personal God, pure, eternal, absolute mind, creator and supreme master, existing outside of the world, was unanimously accepted by the Church Fathers; as a logical consequence of this absolute absurdity, it then became natural and necessary to believe in the immateriality and immortality of the human soul, lodged and imprisoned in a body only partially mortal, there being in this body itself a portion which, while material is immortal like the soul, and must be resurrected with it. We see how difficult it was, even for the Church Fathers; to conceive pure minds outside of any material form. It should be added that, in general, it is the character of every metaphysical and theological argument to seek to explain one absurdity by another.


It was very fortunate for Christianity that it met a world of slaves. It had another piece of good luck in the invasion of the Barbarians. The latter were worthy people, full of natural force, and, above all, urged on by a great necessity of life and a great capacity for it; brigands who had stood every test, capable of devastating and gobbling up anything, like their successors, the Germans of today; but they were much less systematic and pedantic than these last, much less moralistic, less learned, and on the other hand much more independent and proud, capable of science and not incapable of liberty, as are the bourgeois of modern Germany. But, in spite of all their great qualities, they were nothing but barbarians — that is, as indifferent to all questions of theology and metaphysics as the ancient slaves, a great number of whom, moreover, belonged to their race. So that, their practical repugnance once overcome, it was not difficult to convert them theoretically to Christianity.


For ten centuries Christianity, armed with the omnipotence of Church and State and opposed by no competition, was able to deprave, debase, and falsify the mind of Europe It had no competitors, because outside of the Church there were neither thinkers nor educated persons. It alone though,, it alone spoke and wrote, it alone taught. Though heresies arose in its bosom, they affected only the theological or practical developments of the fundamental dogma never that dogma itself. The belief in God, pure spirit and creator of the world, and the belief in the immateriality of the soul remained untouched. This double belief became the ideal basis of the whole Occidental and Oriental civilization of Europe; it penetrated and became incarnate in all the institutions, all the details of the public and private life of all classes, and the masses as well.


After that, is it surprising that this belief has lived until the present day, continuing to exercise its disastrous influence even upon select minds, such as those of Mazzini, Michelet, Quinet, and so many others? We have seen that the first attack upon it came from the \emph{renaissance}; of the free mind in the fifteenth century, which produced heroes and martyrs like Vanini, Giordano Bruno, and Galileo. Although drowned in the noise, tumult, and passions of the Reformation, it noiselessly continued its invisible work, bequeathing to the noblest minds of each generation its task of human emancipation by the destruction of the absurd, until at last, in the latter half of the eighteenth century, it again reappeared in broad day, boldly waving the flag of atheism and materialism.


The human mind, then, one might have supposed, was at last about to deliver itself from all the divine obsessions. Not at all. The divine falsehood upon which humanity had been feeding for eighteen centuries (speaking of Christianity only) was once more to show itself more powerful than human truth. No longer able to make use of the black tribe, of the ravens consecrated by the Church, of the Catholic or Protestant priests, all confidence in whom had been lost, it made use of lay priests, short-robed liars and sophists. among whom the principal rôles devolved upon two fatal men, one the falsest mind, the other the most doctrinally despotic will, of the last century — J. J. Rousseau and Robespierre.


The first is the perfect type of narrowness and suspicious meanness, of exaltation without other object than his own person, of cold enthusiasm and hypocrisy at once sentimental and implacable, of the falsehood of modern idealism. He may be considered as the real creator of modern reaction. To all appearance the most democratic writer of the eighteenth century, he bred within himself the pitiless despotism of the statesman. He was the prophet of the doctrinaire State, as Robespierre, his worthy and faithful disciple, tried to become its high priest. Having heard the saying of Voltaire that, if God did not exist, it would be necessary to invent him, J. J. Rousseau invented the Supreme Being, the abstract and sterile God of the deists. And It was in the name of the Supreme Being, and of the hypocritical virtue commanded by this Supreme Being, that Robespierre guillotined first the Hébertists and then the very genius of the Revolution, Danton, in whose person he assassinated the Republic, thus preparing the way for the thenceforth necessary triumph of the dictatorship of Bonaparte I. After this great triumph, the idealistic reaction sought and found servants less fanatical, less terrible nearer to the diminished stature of the actual bourgeoisie. In France, Chateaubriand, Lamartine, and — shall I say it? Why not? All must be said if it is truth — Victor Hugo himself, the democrat, the republican, the quasi-socialist of today! and after them the whole melancholy and sentimental company of poor and pallid minds who, under the leadership of these masters, established the modern romantic school in Germany, the Schlegels, the Tiecks, the Novalis, the Werners, the Schellings, and so many others besides, whose names do not even deserve to be recalled.


The literature created by this school was the very reign of ghosts and phantoms. It could not stand the sunlight; the twilight alone permitted it to live. No more could it stand the brutal contact of the masses. It was the literature of the tender, delicate, distinguished souls, aspiring to heaven, and living on earth as if in spite of themselves. It had a horror and contempt for the politics and questions of the day; but when perchance it referred to them, it showed itself frankly reactionary, took the side of the Church against the insolence of the freethinkers, of the kings against the peoples, and of all the aristocrats against the vile rabble of the streets. For the rest, as I have just said, the dominant feature of the school of romanticism was a quasi-complete indifference to politics. Amid the clouds in which it lived could be distinguished two real points — the rapid development of bourgeois materialism and the ungovernable outburst of individual vanities.


To understand this romantic literature, the reason for its existence must be sought in the transformation which had been effected in the bosom of the bourgeois class since the revolution of 1793.


From the Renaissance and the Reformation down to the Revolution, the bourgeoisie, if not in Germany, at least in Italy, in France, in Switzerland, in England, in Holland, was the hero and representative of the revolutionary genius of history. From its bosom sprang most of the freethinkers of the fifteenth century, the religious reformers of the two following centuries, and the apostles of human emancipation, including this time those of Germany, of the past century. It alone, naturally supported by the powerful arm of the people, who had faith in it, made the revolution of 1789 and ’93. It proclaimed the downfall of royalty and of the Church, the fraternity of the peoples, the rights of man and of the citizen. Those are its titles to glory; they are immortal!


Soon it split. A considerable portion of the purchasers of national property having become rich, and supporting themselves no longer on the proletariat of the cities, but on the major portion of the peasants of France, these also having become landed proprietors, had no aspiration left but for peace, the re-establishment of public order, and the foundation of a strong and regular government. It therefore welcomed with joy the dictatorship of the first Bonaparte, and, although always Voltairean, did not view with displeasure the Concordat with the Pope and the re-establishment of the official Church in France: “\emph{Religion is so necessary to the people!}” Which means that, satiated themselves, this portion of the bourgeoisie then began to see that it was needful to the maintenance of their situation and the preservation of their newly-acquired estates to appease the unsatisfied hunger of the people by promises of heavenly manna. Then it was that Chateaubriand began to preach.\footnote{It seems to me useful to recall at this point an anecdote — one, by the way, well known and thoroughly authentic — which sheds a very clear light on the personal value of this warmed-over of the Catholic beliefs and on the religious sincerity of that period. Chateaubriand submitted to a publisher a work attacking faith. The publisher called his attention to the fact that atheism had gone out of fashion, that the reading public cared no more for it, and that the demand, on the contrary, was for religious works. Chateaubriand withdrew, but a few months later came back with his \emph{Genius of Christianity}.}


Napoleon fell and the Restoration brought back into France the legitimate monarchy, and with it the power of the Church and of the nobles, who regained, if not the whole, at least a considerable portion of their former influence. This reaction threw the bourgeoisie back into the Revolution, and with the revolutionary spirit that of scepticism also was re-awakened in it. It set Chateaubriand aside and began to read Voltaire again; but it did not go so far as Diderot: its debilitated nerves could not stand nourishment so strong. Voltaire, on the contrary, at once a freethinker and a deist, suited it very well. Béranger and P.L. Courier expressed this new tendency perfectly. “The God of the good people” and the ideal of the bourgeois king, at once liberal and democratic, sketched against the majestic and thenceforth inoffensive background of the Empire’s gigantic victories such was at that period the daily intellectual food of the bourgeoisie of France.


Lamartine, to be sure, excited by a vain and ridiculously envious desire to rise to the poetic height of the great Byron, had begun his coldly delirious hymns in honour of the God of the nobles and of the legitimate monarchy. But his songs resounded only in aristocratic salons. The bourgeoisie did not hear them. Béranger was its poet and Courier was its political writer.


The revolution of July resulted in lifting its tastes. We know that every bourgeois in France carries within him the imperishable type of the bourgeois gentleman, a type which never fails to appear immediately the parvenu acquires a little wealth and power. In 1830 the wealthy bourgeoisie had definitely replaced the old nobility in the seats of power. It naturally tended to establish a new aristocracy. An aristocracy of capital first of all, but also an aristocracy of intellect, of good manners and delicate sentiments. It began to feel religious.


This was not on its part simply an aping of aristocratic customs. It was also a necessity of its position. The proletariat had rendered it a final service in once more aiding it to overthrow the nobility. The bourgeoisie now had no further need of its co-operation, for it felt itself firmly seated in the shadow of the throne of July, and the alliance with the people, thenceforth useless, began to become inconvenient. It was necessary to remand it to its place, which naturally could not be done without provoking great indignation among the masses. It became necessary to restrain this indignation. In the name of what? In the name of the bourgeois interest bluntly confessed ? That would have been much too cynical. The more unjust and inhuman an interest is, the greater need it has of sanction. Now, where find it if not in religion, that good protectress of al I the well-fed and the useful consoler of the hungry? And more than ever the triumphant bourgeoisie saw that religion was indispensable to the people.


After having won all its titles to glory in religious, philosophical, and political opposition, in protest and in revolution, it at last became the dominant class and thereby even the defender and preserver of the State, thenceforth the regular institution of the exclusive power of that class. The State is force, and for it, first of all, is the right of force, the triumphant argument of the needle-gun, of the \emph{chassepot}. But man is so singularly constituted that this argument, wholly eloquent as it may appear, is not sufficient in the long run. Some moral sanction or other is absolutely necessary to enforce his respect. Further, this sanction must be at once so simple and so plain that it may convince the masses, who, after having been reduced by the power of the State. must also be induced to morally recognise its right.


There are only two ways of convincing the masses of the goodness of any social institution whatever. The first, the only real one, but also the most difficult to adopt — because it implies the abolition of the State, or, in other words, the abolition of the organised political exploitation of the majority by any minority whatsoever — would be the direct and complete satisfaction of the needs and aspirations of the people, which would be equivalent to the complete liquidation of the political and economical existence of the bourgeois class, or, again, to the abolition of the State. Beneficial means for the masses, but detrimental to bourgeois interests; hence it is useless to talk about them.


The only way, on the contrary, harmful only to the people, precious in its salvation of bourgeois privileges, is no other than religion. That is the eternal \emph{mirage}; which leads away the masses in a search for divine treasures, while much more reserved, the governing class contents itself with dividing among all its members — very unequally, moreover and always giving most to him who possesses most — the miserable goods of earth and the plunder taken from the people, including their political and social liberty.


There is not, there cannot be, a State without religion. Take the freest States in the world — the United States of America or the Swiss Confederation, for instance — and see what an important part is played in all official discourses by divine Providence, that supreme sanction of all States.


But whenever a chief of State speaks of God, be he Wil1iam I., the Knouto-Germanic emperor, or Grant, the president of the great republic, be sure that he is getting ready to shear once more his people-flock.


The French liberal and Voltairean bourgeoisie, driven by temperament to a positivism (not to say a materialism) singularly narrow and brutal, having become the governing class of the State by its triumph of 1830, had to give itself an official religion. It was not an easy thing. The bourgeoisie could not abruptly go back under the yoke of Roman Catholicism. Between it and the Church of Rome was an abyss of blood and hatred, and, however practical and wise one becomes, it is never possible to repress a passion developed by history. Moreover, the French bourgeoisie would have covered itself with ridicule if it had gone back to the Church to take part in the pious ceremonies of its worship, an essential condition of a meritorious and sincere conversion. Several attempted it, it is true, but their heroism was rewarded by no other result than a fruitless scandal. Finally, a return to Catholicism was impossible on account of the insolvable contradiction which separates the invariable politics of Rome from the development of the economical and political interests of the middle class.


In this respect Protestantism is much more advantageous. It is the bourgeois religion \emph{par excellence}. It accords just as much liberty as is necessary to the bourgeois, and finds a way of reconciling celestial aspirations with the respect which terrestrial conditions demand. Consequently it is especially in Protestant countries that commerce and industry have been developed. But it was impossible for the French bourgeoisie to become Protestant. To pass from one religion to another — unless it be done deliberately, as sometimes in the case of the Jews of Russia and Poland, who get baptised three or four times in order to receive each time the remuneration allowed them — to seriously change one’s religion, a little faith is necessary. Now, in the exclusive positive heart of the French bourgeois there is no room for faith. He professes the most profound indifference for all questions which touch neither his pocket first nor his social vanity afterwards. He is as indifferent to Protestantism as to Catholicism. On the other hand, the French bourgeois could not go over to Protestantism without putting himself in conflict with the Catholic routine of the majority of the French people, which would have been great imprudence on the part of a class pretending to govern the nation.


There was still one way left — to return to the humanitarian and revolutionary religion of the eighteenth century. But that would have led too far. So the bourgeoisie was obliged, in order to sanction its new State, to create a new religion which might be boldly proclaimed, without too much ridicule and scandal, by the whole bourgeois class.


Thus was born \emph{doctrinaire} Deism.


Others have told, much better than I could tell it, the story of the birth and development of this school, which had so decisive and — we may well add — so fatal an influence on the political, intellectual, and moral education of the bourgeois youth of France. It dates from Benjamin Constant and Madame de StaÎl; its real founder was Royer-Collard; its apostles, Guizot, Cousin, Villemain, and many others. Its boldly avowed object was the reconciliation of Revolution with Reaction, or, to use the language of the school, of the principle of liberty with that of authority, and naturally to the advantage of the latter.


This reconciliation signified: in politics, the taking away of popular liberty for the benefit of bourgeois rule, represented by the monarchical and constitutional State; in philosophy, the deliberate submission of free reason to the eternal principles of faith. We have only to deal here with the latter.


We know that this philosophy was specially elaborated by M. Cousin, the father of French eclecticism. A superficial and pedantic talker, incapable of any original conception, of any idea peculiar to himself, but very strong on commonplace, which he confounded with common sense, this illustrious philosopher learnedly prepared, for the use of the studious youth of France, a metaphysical dish of his own making the use of which, made compulsory in all schools of the State under the University, condemned several generations one after the other to a cerebral indigestion. Imagine a philosophical vinegar sauce of the most opposed systems, a mixture of Fathers of the Church, scholastic philosophers, Descartes and Pascal, Kant and Scotch psychologists all this a superstructure on the divine and innate ideas of Plato, and covered up with a layer of Hegelian immanence accompanied, of course, by an ignorance, as contemptuous as it is complete, of natural science, and proving just as two times two make \emph{five}; the existence of a personal God\dots{}.

\end{document}


