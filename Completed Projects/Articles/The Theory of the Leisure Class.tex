\documentclass[12pt]{report}
\usepackage[12pt]{moresize}
\usepackage[utf8]{inputenc}
\usepackage[english]{babel}
\usepackage[top=2.5cm, bottom=2.5cm, left=2.5cm, right=2.5cm]{geometry}
\usepackage{ebgaramond}

%=======SECTION HEADERS=========%
\usepackage{titlesec}
\usepackage{titletoc}

%========QUOTES=========%
\usepackage{epigraph}
\usepackage[autostyle, english = american]{csquotes}
\MakeOuterQuote{"}

%=======PARAGRAPH FORMATTING=========%
\setlength{\parindent}{0pt} %no paragraph indents
\setlength{\parskip}{1em}   %single space between paragraphs

\renewcommand{\chaptermark}[1]{\markboth{\MakeUppercase{Book \thechapter}}{}} %Book format- heading

%=======CHAPTER FORMATTING=========%
\renewcommand\thechapter{{\Roman{chapter}}}   %section numbering style

\titleformat
{\chapter} 
[display]
{\centering\fontfamily{ppl}\Huge} 
{\thechapter} 
{\leftmargin}{}[]

%=======SECTION HEADER SPACING=========%
\titlespacing{\chapter}{0mm}{-2em}{1em}
\titlespacing{\section}{0mm}{3mm}{2mm}

%=======TITLE PAGE=========%
\title{\HUGE\bfseries{The Theory of the Leisure Class}
 \\ \vspace{1cm}
\huge\itshape An Economic Study of Institutions}
\author{\Large by Thorstein Veblen}
\date{\vspace{-4mm}1899}

%=======FOOTNOTES=========%
\renewcommand{\thefootnote}{[\arabic{footnote}]}
\setlength{\skip\footins}{1cm}
\usepackage[]{footmisc}
\renewcommand{\footnotemargin}{3mm} %Setting left margin
\renewcommand{\footnotelayout}{\hspace{2mm}} %spacing between the footnote number and the text of footnote

\usepackage{hyperref}
\hypersetup{bookmarksnumbered} %Bookmarks are numbered in the ToC when converted to PDF or EPUB

\titlecontents{chapter}% <section-type>
  [2.5em]% <left>
  {\vspace{1em}}% <above-code>
  {\contentslabel{2.3em}\quad}% <numbered-entry-format>
  {\contentslabel{2.3em}\quad}% <numberless-entry-format>
  {\titlerule*[1pc]{.}\contentspage}% <filler-page-format>

\titlecontents{section}%formatting-toc-sections
    [3.8em] 
    {\vspace{-3mm}}
    {\contentslabel{2.3em}}
    {}
    {\titlerule*[1pc]{.}\contentspage}
\begin{document}

\begin{titlepage}
    \maketitle
\end{titlepage}

%=======TABLE OF CONTENTS=========%
\renewcommand*\contentsname{\vspace{-1cm} Contents}
\tableofcontents

%=======MAIN DOCUMENT=========%
\titlespacing{\chapter}{0mm}{-2em}{1em}
\titleformat
{\chapter} 
[display]
{\centering\fontfamily{ppl}\Huge} 
{\thechapter\vspace{-1em}} 
{\leftmargin}{\normalfont\Large}[]

\chapter{Introductory}
The institution of a leisure class is found in its best development at
the higher stages of the barbarian culture; as, for instance, in feudal
Europe or feudal Japan. In such communities the distinction between
classes is very rigorously observed; and the feature of most striking
economic significance in these class differences is the distinction
maintained between the employments proper to the several classes.
The upper classes are by custom exempt or excluded from industrial
occupations, and are reserved for certain employments to which a degree
of honour attaches. Chief among the honourable employments in any
feudal community is warfare; and priestly service is commonly second to
warfare. If the barbarian community is not notably warlike, the priestly
office may take the precedence, with that of the warrior second. But the
rule holds with but slight exceptions that, whether warriors or priests,
the upper classes are exempt from industrial employments, and this
exemption is the economic expression of their superior rank. Brahmin
India affords a fair illustration of the industrial exemption of both
these classes. In the communities belonging to the higher barbarian
culture there is a considerable differentiation of sub-classes within
what may be comprehensively called the leisure class; and there is a
corresponding differentiation of employments between these sub-classes.
The leisure class as a whole comprises the noble and the priestly
classes, together with much of their retinue. The occupations of the
class are correspondingly diversified; but they have the common economic
characteristic of being non-industrial. These non-industrial upper-class
occupations may be roughly comprised under government, warfare,
religious observances, and sports.

At an earlier, but not the earliest, stage of barbarism, the leisure
class is found in a less differentiated form. Neither the class
distinctions nor the distinctions between leisure-class occupations are
so minute and intricate. The Polynesian islanders generally show this
stage of the development in good form, with the exception that, owing
to the absence of large game, hunting does not hold the usual place of
honour in their scheme of life. The Icelandic community in the time of
the Sagas also affords a fair instance. In such a community there is
a rigorous distinction between classes and between the occupations
peculiar to each class. Manual labour, industry, whatever has to
do directly with the everyday work of getting a livelihood, is the
exclusive occupation of the inferior class. This inferior class includes
slaves and other dependents, and ordinarily also all the women. If there
are several grades of aristocracy, the women of high rank are commonly
exempt from industrial employment, or at least from the more vulgar
kinds of manual labour. The men of the upper classes are not only
exempt, but by prescriptive custom they are debarred, from all
industrial occupations. The range of employments open to them is rigidly
defined. As on the higher plane already spoken of, these employments are
government, warfare, religious observances, and sports. These four lines
of activity govern the scheme of life of the upper classes, and for
the highest rank--the kings or chieftains--these are the only kinds of
activity that custom or the common sense of the community will allow.
Indeed, where the scheme is well developed even sports are accounted
doubtfully legitimate for the members of the highest rank. To the lower
grades of the leisure class certain other employments are open, but they
are employments that are subsidiary to one or another of these typical
leisure-class occupations. Such are, for instance, the manufacture
and care of arms and accoutrements and of war canoes, the dressing
and handling of horses, dogs, and hawks, the preparation of sacred
apparatus, etc. The lower classes are excluded from these secondary
honourable employments, except from such as are plainly of an industrial
character and are only remotely related to the typical leisure-class
occupations.

If we go a step back of this exemplary barbarian culture, into the
lower stages of barbarism, we no longer find the leisure class in fully
developed form. But this lower barbarism shows the usages, motives,
and circumstances out of which the institution of a leisure class has
arisen, and indicates the steps of its early growth. Nomadic hunting
tribes in various parts of the world illustrate these more primitive
phases of the differentiation. Any one of the North American hunting
tribes may be taken as a convenient illustration. These tribes
can scarcely be said to have a defined leisure class. There is a
differentiation of function, and there is a distinction between classes
on the basis of this difference of function, but the exemption of the
superior class from work has not gone far enough to make the designation
"leisure class" altogether applicable. The tribes belonging on this
economic level have carried the economic differentiation to the point
at which a marked distinction is made between the occupations of men and
women, and this distinction is of an invidious character. In nearly
all these tribes the women are, by prescriptive custom, held to those
employments out of which the industrial occupations proper develop at
the next advance. The men are exempt from these vulgar employments and
are reserved for war, hunting, sports, and devout observances. A very
nice discrimination is ordinarily shown in this matter.

This division of labour coincides with the distinction between the
working and the leisure class as it appears in the higher barbarian
culture. As the diversification and specialisation of employments
proceed, the line of demarcation so drawn comes to divide the industrial
from the non-industrial employments. The man's occupation as it stands
at the earlier barbarian stage is not the original out of which any
appreciable portion of later industry has developed. In the later
development it survives only in employments that are not classed as
industrial,--war, politics, sports, learning, and the priestly office.
The only notable exceptions are a portion of the fishery industry
and certain slight employments that are doubtfully to be classed as
industry; such as the manufacture of arms, toys, and sporting goods.
Virtually the whole range of industrial employments is an outgrowth of
what is classed as woman's work in the primitive barbarian community.

The work of the men in the lower barbarian culture is no less
indispensable to the life of the group than the work done by the women.
It may even be that the men's work contributes as much to the food
supply and the other necessary consumption of the group. Indeed, so
obvious is this "productive" character of the men's work that in the
conventional economic writings the hunter's work is taken as the type of
primitive industry. But such is not the barbarian's sense of the matter.
In his own eyes he is not a labourer, and he is not to be classed with
the women in this respect; nor is his effort to be classed with the
women's drudgery, as labour or industry, in such a sense as to admit
of its being confounded with the latter. There is in all barbarian
communities a profound sense of the disparity between man's and woman's
work. His work may conduce to the maintenance of the group, but it is
felt that it does so through an excellence and an efficacy of a kind
that cannot without derogation be compared with the uneventful diligence
of the women.

At a farther step backward in the cultural scale--among savage
groups--the differentiation of employments is still less elaborate
and the invidious distinction between classes and employments is less
consistent and less rigorous. Unequivocal instances of a primitive
savage culture are hard to find. Few of these groups or communities
that are classed as "savage" show no traces of regression from a more
advanced cultural stage. But there are groups--some of them apparently
not the result of retrogression--which show the traits of primitive
savagery with some fidelity. Their culture differs from that of the
barbarian communities in the absence of a leisure class and the absence,
in great measure, of the animus or spiritual attitude on which the
institution of a leisure class rests. These communities of primitive
savages in which there is no hierarchy of economic classes make up but a
small and inconspicuous fraction of the human race. As good an instance
of this phase of culture as may be had is afforded by the tribes of the
Andamans, or by the Todas of the Nilgiri Hills. The scheme of life of
these groups at the time of their earliest contact with Europeans seems
to have been nearly typical, so far as regards the absence of a leisure
class. As a further instance might be cited the Ainu of Yezo, and, more
doubtfully, also some Bushman and Eskimo groups. Some Pueblo communities
are less confidently to be included in the same class. Most, if not all,
of the communities here cited may well be cases of degeneration from a
higher barbarism, rather than bearers of a culture that has never risen
above its present level. If so, they are for the present purpose to be
taken with the allowance, but they may serve none the less as evidence
to the same effect as if they were really "primitive" populations.

These communities that are without a defined leisure class resemble one
another also in certain other features of their social structure
and manner of life. They are small groups and of a simple (archaic)
structure; they are commonly peaceable and sedentary; they are poor; and
individual ownership is not a dominant feature of their economic system.
At the same time it does not follow that these are the smallest of
existing communities, or that their social structure is in all respects
the least differentiated; nor does the class necessarily include
all primitive communities which have no defined system of individual
ownership. But it is to be noted that the class seems to include the
most peaceable--perhaps all the characteristically peaceable--primitive
groups of men. Indeed, the most notable trait common to members of such
communities is a certain amiable inefficiency when confronted with force
or fraud.

The evidence afforded by the usages and cultural traits of communities
at a low stage of development indicates that the institution of a
leisure class has emerged gradually during the transition from primitive
savagery to barbarism; or more precisely, during the transition from
a peaceable to a consistently warlike habit of life. The conditions
apparently necessary to its emergence in a consistent form are: (1) the
community must be of a predatory habit of life (war or the hunting
of large game or both); that is to say, the men, who constitute the
inchoate leisure class in these cases, must be habituated to the
infliction of injury by force and stratagem; (2) subsistence must be
obtainable on sufficiently easy terms to admit of the exemption of
a considerable portion of the community from steady application to a
routine of labour. The institution of leisure class is the outgrowth
of an early discrimination between employments, according to which
some employments are worthy and others unworthy. Under this ancient
distinction the worthy employments are those which may be classed as
exploit; unworthy are those necessary everyday employments into which no
appreciable element of exploit enters.

This distinction has but little obvious significance in a modern
industrial community, and it has, therefore, received but slight
attention at the hands of economic writers. When viewed in the light of
that modern common sense which has guided economic discussion, it seems
formal and insubstantial. But it persists with great tenacity as
a commonplace preconception even in modern life, as is shown, for
instance, by our habitual aversion to menial employments. It is a
distinction of a personal kind--of superiority and inferiority. In the
earlier stages of culture, when the personal force of the individual
counted more immediately and obviously in shaping the course of events,
the element of exploit counted for more in the everyday scheme of life.
Interest centred about this fact to a greater degree. Consequently a
distinction proceeding on this ground seemed more imperative and more
definitive then than is the case to-day. As a fact in the sequence of
development, therefore, the distinction is a substantial one and rests
on sufficiently valid and cogent grounds.

The ground on which a discrimination between facts is habitually made
changes as the interest from which the facts are habitually viewed
changes. Those features of the facts at hand are salient and substantial
upon which the dominant interest of the time throws its light. Any given
ground of distinction will seem insubstantial to any one who habitually
apprehends the facts in question from a different point of view and
values them for a different purpose. The habit of distinguishing and
classifying the various purposes and directions of activity prevails of
necessity always and everywhere; for it is indispensable in reaching a
working theory or scheme of life. The particular point of view, or the
particular characteristic that is pitched upon as definitive in the
classification of the facts of life depends upon the interest from which
a discrimination of the facts is sought. The grounds of discrimination,
and the norm of procedure in classifying the facts, therefore,
progressively change as the growth of culture proceeds; for the end for
which the facts of life are apprehended changes, and the point of view
consequently changes also. So that what are recognised as the salient
and decisive features of a class of activities or of a social class at
one stage of culture will not retain the same relative importance for
the purposes of classification at any subsequent stage.

But the change of standards and points of view is gradual only, and it
seldom results in the subversion or entire suppression of a standpoint
once accepted. A distinction is still habitually made between industrial
and non-industrial occupations; and this modern distinction is a
transmuted form of the barbarian distinction between exploit and
drudgery. Such employments as warfare, politics, public worship, and
public merrymaking, are felt, in the popular apprehension, to differ
intrinsically from the labour that has to do with elaborating the
material means of life. The precise line of demarcation is not the same
as it was in the early barbarian scheme, but the broad distinction has
not fallen into disuse.

The tacit, common-sense distinction to-day is, in effect, that any
effort is to be accounted industrial only so far as its ultimate purpose
is the utilisation of non-human things. The coercive utilisation of man
by man is not felt to be an industrial function; but all effort directed
to enhance human life by taking advantage of the non-human environment
is classed together as industrial activity. By the economists who have
best retained and adapted the classical tradition, man's "power over
nature" is currently postulated as the characteristic fact of industrial
productivity. This industrial power over nature is taken to include
man's power over the life of the beasts and over all the elemental
forces. A line is in this way drawn between mankind and brute creation.

In other times and among men imbued with a different body of
preconceptions this line is not drawn precisely as we draw it to-day.
In the savage or the barbarian scheme of life it is drawn in a different
place and in another way. In all communities under the barbarian
culture there is an alert and pervading sense of antithesis between
two comprehensive groups of phenomena, in one of which barbarian
man includes himself, and in the other, his victual. There is a felt
antithesis between economic and non-economic phenomena, but it is not
conceived in the modern fashion; it lies not between man and brute
creation, but between animate and inert things.
\clearpage
It may be an excess of caution at this day to explain that the barbarian
notion which it is here intended to convey by the term "animate" is not
the same as would be conveyed by the word "living". The term does not
cover all living things, and it does cover a great many others. Such
a striking natural phenomenon as a storm, a disease, a waterfall, are
recognised as "animate"; while fruits and herbs, and even inconspicuous
animals, such as house-flies, maggots, lemmings, sheep, are not
ordinarily apprehended as "animate" except when taken collectively.
As here used the term does not necessarily imply an indwelling soul or
spirit. The concept includes such things as in the apprehension of the
animistic savage or barbarian are formidable by virtue of a real or
imputed habit of initiating action. This category comprises a large
number and range of natural objects and phenomena. Such a distinction
between the inert and the active is still present in the habits of
thought of unreflecting persons, and it still profoundly affects the
prevalent theory of human life and of natural processes; but it does not
pervade our daily life to the extent or with the far-reaching practical
consequences that are apparent at earlier stages of culture and belief.

To the mind of the barbarian, the elaboration and utilisation of what is
afforded by inert nature is activity on quite a different plane from his
dealings with "animate" things and forces. The line of demarcation may
be vague and shifting, but the broad distinction is sufficiently real
and cogent to influence the barbarian scheme of life. To the class of
things apprehended as animate, the barbarian fancy imputes an unfolding
of activity directed to some end. It is this teleological unfolding of
activity that constitutes any object or phenomenon an "animate" fact.
Wherever the unsophisticated savage or barbarian meets with activity
that is at all obtrusive, he construes it in the only terms that are
ready to hand--the terms immediately given in his consciousness of his
own actions. Activity is, therefore, assimilated to human action, and
active objects are in so far assimilated to the human agent. Phenomena
of this character--especially those whose behaviour is notably
formidable or baffling--have to be met in a different spirit and with
proficiency of a different kind from what is required in dealing with
inert things. To deal successfully with such phenomena is a work of
exploit rather than of industry. It is an assertion of prowess, not of
diligence.

Under the guidance of this naive discrimination between the inert and
the animate, the activities of the primitive social group tend to fall
into two classes, which would in modern phrase be called exploit and
industry. Industry is effort that goes to create a new thing, with a
new purpose given it by the fashioning hand of its maker out of passive
("brute") material; while exploit, so far as it results in an outcome
useful to the agent, is the conversion to his own ends of energies
previously directed to some other end by an other agent. We still speak
of "brute matter" with something of the barbarian's realisation of a
profound significance in the term.

The distinction between exploit and drudgery coincides with a difference
between the sexes. The sexes differ, not only in stature and muscular
force, but perhaps even more decisively in temperament, and this must
early have given rise to a corresponding division of labour. The general
range of activities that come under the head of exploit falls to the
males as being the stouter, more massive, better capable of a sudden
and violent strain, and more readily inclined to self assertion, active
emulation, and aggression. The difference in mass, in physiological
character, and in temperament may be slight among the members of the
primitive group; it appears, in fact, to be relatively slight and
inconsequential in some of the more archaic communities with which we
are acquainted--as for instance the tribes of the Andamans. But so soon
as a differentiation of function has well begun on the lines marked
out by this difference in physique and animus, the original difference
between the sexes will itself widen. A cumulative process of selective
adaptation to the new distribution of employments will set in,
especially if the habitat or the fauna with which the group is in
contact is such as to call for a considerable exercise of the sturdier
virtues. The habitual pursuit of large game requires more of the manly
qualities of massiveness, agility, and ferocity, and it can therefore
scarcely fail to hasten and widen the differentiation of functions
between the sexes. And so soon as the group comes into hostile contact
with other groups, the divergence of function will take on the developed
form of a distinction between exploit and industry.

In such a predatory group of hunters it comes to be the able-bodied
men's office to fight and hunt. The women do what other work there is
to do--other members who are unfit for man's work being for this purpose
classed with women. But the men's hunting and fighting are both of the
same general character. Both are of a predatory nature; the warrior
and the hunter alike reap where they have not strewn. Their aggressive
assertion of force and sagacity differs obviously from the women's
assiduous and uneventful shaping of materials; it is not to be accounted
productive labour but rather an acquisition of substance by seizure.
Such being the barbarian man's work, in its best development and widest
divergence from women's work, any effort that does not involve an
assertion of prowess comes to be unworthy of the man. As the tradition
gains consistency, the common sense of the community erects it into a
canon of conduct; so that no employment and no acquisition is morally
possible to the self respecting man at this cultural stage, except such
as proceeds on the basis of prowess--force or fraud. When the predatory
habit of life has been settled upon the group by long habituation, it
becomes the able-bodied man's accredited office in the social economy
to kill, to destroy such competitors in the struggle for existence as
attempt to resist or elude him, to overcome and reduce to subservience
those alien forces that assert themselves refractorily in the
environment. So tenaciously and with such nicety is this theoretical
distinction between exploit and drudgery adhered to that in many hunting
tribes the man must not bring home the game which he has killed, but
must send his woman to perform that baser office.

As has already been indicated, the distinction between exploit and
drudgery is an invidious distinction between employments. Those
employments which are to be classed as exploit are worthy, honourable,
noble; other employments, which do not contain this element of exploit,
and especially those which imply subservience or submission, are
unworthy, debasing, ignoble. The concept of dignity, worth, or honour,
as applied either to persons or conduct, is of first-rate consequence
in the development of classes and of class distinctions, and it is
therefore necessary to say something of its derivation and meaning. Its
psychological ground may be indicated in outline as follows.

As a matter of selective necessity, man is an agent. He is, in his own
apprehension, a centre of unfolding impulsive activity--"teleological"
activity. He is an agent seeking in every act the accomplishment of some
concrete, objective, impersonal end. By force of his being such an agent
he is possessed of a taste for effective work, and a distaste for futile
effort. He has a sense of the merit of serviceability or efficiency
and of the demerit of futility, waste, or incapacity. This aptitude
or propensity may be called the instinct of workmanship. Wherever the
circumstances or traditions of life lead to an habitual comparison
of one person with another in point of efficiency, the instinct of
workmanship works out in an emulative or invidious comparison of
persons. The extent to which this result follows depends in some
considerable degree on the temperament of the population. In any
community where such an invidious comparison of persons is habitually
made, visible success becomes an end sought for its own utility as a
basis of esteem. Esteem is gained and dispraise is avoided by putting
one's efficiency in evidence. The result is that the instinct of
workmanship works out in an emulative demonstration of force.

During that primitive phase of social development, when the community is
still habitually peaceable, perhaps sedentary, and without a developed
system of individual ownership, the efficiency of the individual can
be shown chiefly and most consistently in some employment that goes to
further the life of the group. What emulation of an economic kind there
is between the members of such a group will be chiefly emulation in
industrial serviceability. At the same time the incentive to emulation
is not strong, nor is the scope for emulation large.

When the community passes from peaceable savagery to a predatory phase
of life, the conditions of emulation change. The opportunity and the
incentive to emulate increase greatly in scope and urgency. The activity
of the men more and more takes on the character of exploit; and an
invidious comparison of one hunter or warrior with another grows
continually easier and more habitual. Tangible evidences of
prowess--trophies--find a place in men's habits of thought as an
essential feature of the paraphernalia of life. Booty, trophies of
the chase or of the raid, come to be prized as evidence of pre-eminent
force. Aggression becomes the accredited form of action, and booty
serves as \emph{prima facie} evidence of successful aggression. As accepted at
this cultural stage, the accredited, worthy form of self-assertion
is contest; and useful articles or services obtained by seizure or
compulsion, serve as a conventional evidence of successful contest.
Therefore, by contrast, the obtaining of goods by other methods than
seizure comes to be accounted unworthy of man in his best estate. The
performance of productive work, or employment in personal service, falls
under the same odium for the same reason. An invidious distinction
in this way arises between exploit and acquisition on the other hand.
Labour acquires a character of irksomeness by virtue of the indignity
imputed to it.

With the primitive barbarian, before the simple content of the notion
has been obscured by its own ramifications and by a secondary growth of
cognate ideas, "honourable" seems to connote nothing else than
assertion of superior force. "Honourable" is "formidable"; "worthy" is
"prepotent". A honorific act is in the last analysis little if
anything else than a recognised successful act of aggression; and where
aggression means conflict with men and beasts, the activity which comes
to be especially and primarily honourable is the assertion of the strong
hand. The naive, archaic habit of construing all manifestations of
force in terms of personality or "will power" greatly fortifies this
conventional exaltation of the strong hand. Honorific epithets, in
vogue among barbarian tribes as well as among peoples of a more advance
culture, commonly bear the stamp of this unsophisticated sense of
honour. Epithets and titles used in addressing chieftains, and in the
propitiation of kings and gods, very commonly impute a propensity for
overbearing violence and an irresistible devastating force to the person
who is to be propitiated. This holds true to an extent also in the more
civilised communities of the present day. The predilection shown in
heraldic devices for the more rapacious beasts and birds of prey goes to
enforce the same view.

Under this common-sense barbarian appreciation of worth or honour, the
taking of life--the killing of formidable competitors, whether brute
or human--is honourable in the highest degree. And this high office of
slaughter, as an expression of the slayer's prepotence, casts a
glamour of worth over every act of slaughter and over all the tools and
accessories of the act. Arms are honourable, and the use of them, even
in seeking the life of the meanest creatures of the fields, becomes a
honorific employment. At the same time, employment in industry becomes
correspondingly odious, and, in the common-sense apprehension, the
handling of the tools and implements of industry falls beneath the
dignity of able-bodied men. Labour becomes irksome.

It is here assumed that in the sequence of cultural evolution primitive
groups of men have passed from an initial peaceable stage to a
subsequent stage at which fighting is the avowed and characteristic
employment of the group. But it is not implied that there has been an
abrupt transition from unbroken peace and good-will to a later or higher
phase of life in which the fact of combat occurs for the first time.
Neither is it implied that all peaceful industry disappears on the
transition to the predatory phase of culture. Some fighting, it is safe
to say, would be met with at any early stage of social development.
Fights would occur with more or less frequency through sexual
competition. The known habits of primitive groups, as well as the habits
of the anthropoid apes, argue to that effect, and the evidence from the
well-known promptings of human nature enforces the same view.

It may therefore be objected that there can have been no such initial
stage of peaceable life as is here assumed. There is no point in
cultural evolution prior to which fighting does not occur. But the
point in question is not as to the occurrence of combat, occasional or
sporadic, or even more or less frequent and habitual; it is a question
as to the occurrence of an habitual; it is a question as to the
occurrence of an habitual bellicose frame of mind--a prevalent habit
of judging facts and events from the point of view of the fight. The
predatory phase of culture is attained only when the predatory attitude
has become the habitual and accredited spiritual attitude for the
members of the group; when the fight has become the dominant note in the
current theory of life; when the common-sense appreciation of men and
things has come to be an appreciation with a view to combat.

The substantial difference between the peaceable and the predatory phase
of culture, therefore, is a spiritual difference, not a mechanical one.
The change in spiritual attitude is the outgrowth of a change in the
material facts of the life of the group, and it comes on gradually as
the material circumstances favourable to a predatory attitude supervene.
The inferior limit of the predatory culture is an industrial limit.
Predation can not become the habitual, conventional resource of any
group or any class until industrial methods have been developed to such
a degree of efficiency as to leave a margin worth fighting for, above
the subsistence of those engaged in getting a living. The transition
from peace to predation therefore depends on the growth of technical
knowledge and the use of tools. A predatory culture is similarly
impracticable in early times, until weapons have been developed to such
a point as to make man a formidable animal. The early development of
tools and of weapons is of course the same fact seen from two different
points of view.

The life of a given group would be characterised as peaceable so long
as habitual recourse to combat has not brought the fight into the
foreground in men's every day thoughts, as a dominant feature of the
life of man. A group may evidently attain such a predatory attitude with
a greater or less degree of completeness, so that its scheme of life and
canons of conduct may be controlled to a greater or less extent by the
predatory animus. The predatory phase of culture is therefore conceived
to come on gradually, through a cumulative growth of predatory aptitudes
habits, and traditions this growth being due to a change in the
circumstances of the group's life, of such a kind as to develop and
conserve those traits of human nature and those traditions and norms of
conduct that make for a predatory rather than a peaceable life.

The evidence for the hypothesis that there has been such a peaceable
stage of primitive culture is in great part drawn from psychology rather
than from ethnology, and cannot be detailed here. It will be recited in
part in a later chapter, in discussing the survival of archaic traits of
human nature under the modern culture.




\chapter{Pecuniary Emulation}
In the sequence of cultural evolution the emergence of a leisure class
coincides with the beginning of ownership. This is necessarily the case,
for these two institutions result from the same set of economic forces.
In the inchoate phase of their development they are but different
aspects of the same general facts of social structure.

It is as elements of social structure--conventional facts--that leisure
and ownership are matters of interest for the purpose in hand. An
habitual neglect of work does not constitute a leisure class; neither
does the mechanical fact of use and consumption constitute ownership.
The present inquiry, therefore, is not concerned with the beginning
of indolence, nor with the beginning of the appropriation of useful
articles to individual consumption. The point in question is the origin
and nature of a conventional leisure class on the one hand and the
beginnings of individual ownership as a conventional right or equitable
claim on the other hand.

The early differentiation out of which the distinction between a leisure
and a working class arises is a division maintained between men's and
women's work in the lower stages of barbarism. Likewise the earliest
form of ownership is an ownership of the women by the able bodied men
of the community. The facts may be expressed in more general terms, and
truer to the import of the barbarian theory of life, by saying that it
is an ownership of the woman by the man.

There was undoubtedly some appropriation of useful articles before the
custom of appropriating women arose. The usages of existing archaic
communities in which there is no ownership of women is warrant for such
a view. In all communities the members, both male and female, habitually
appropriate to their individual use a variety of useful things; but
these useful things are not thought of as owned by the person who
appropriates and consumes them. The habitual appropriation and
consumption of certain slight personal effects goes on without
raising the question of ownership; that is to say, the question of a
conventional, equitable claim to extraneous things.

The ownership of women begins in the lower barbarian stages of culture,
apparently with the seizure of female captives. The original reason
for the seizure and appropriation of women seems to have been their
usefulness as trophies. The practice of seizing women from the enemy
as trophies, gave rise to a form of ownership-marriage, resulting in a
household with a male head. This was followed by an extension of slavery
to other captives and inferiors, besides women, and by an extension of
ownership-marriage to other women than those seized from the enemy.
The outcome of emulation under the circumstances of a predatory life,
therefore, has been on the one hand a form of marriage resting on
coercion, and on the other hand the custom of ownership. The two
institutions are not distinguishable in the initial phase of their
development; both arise from the desire of the successful men to put
their prowess in evidence by exhibiting some durable result of their
exploits. Both also minister to that propensity for mastery which
pervades all predatory communities. From the ownership of women the
concept of ownership extends itself to include the products of their
industry, and so there arises the ownership of things as well as of
persons.

In this way a consistent system of property in goods is gradually
installed. And although in the latest stages of the development,
the serviceability of goods for consumption has come to be the most
obtrusive element of their value, still, wealth has by no means yet lost
its utility as a honorific evidence of the owner's prepotence.

Wherever the institution of private property is found, even in a
slightly developed form, the economic process bears the character of a
struggle between men for the possession of goods. It has been customary
in economic theory, and especially among those economists who adhere
with least faltering to the body of modernised classical doctrines, to
construe this struggle for wealth as being substantially a struggle for
subsistence. Such is, no doubt, its character in large part during
the earlier and less efficient phases of industry. Such is also its
character in all cases where the "niggardliness of nature" is so strict
as to afford but a scanty livelihood to the community in return for
strenuous and unremitting application to the business of getting the
means of subsistence. But in all progressing communities an advance is
presently made beyond this early stage of technological development.
Industrial efficiency is presently carried to such a pitch as to afford
something appreciably more than a bare livelihood to those engaged in
the industrial process. It has not been unusual for economic theory to
speak of the further struggle for wealth on this new industrial basis as
a competition for an increase of the comforts of life,--primarily for
an increase of the physical comforts which the consumption of goods
affords.

The end of acquisition and accumulation is conventionally held to be the
consumption of the goods accumulated--whether it is consumption directly
by the owner of the goods or by the household attached to him and for
this purpose identified with him in theory. This is at least felt to
be the economically legitimate end of acquisition, which alone it is
incumbent on the theory to take account of. Such consumption may of
course be conceived to serve the consumer's physical wants--his
physical comfort--or his so-called higher wants--spiritual, aesthetic,
intellectual, or what not; the latter class of wants being served
indirectly by an expenditure of goods, after the fashion familiar to all
economic readers.

But it is only when taken in a sense far removed from its naive meaning
that consumption of goods can be said to afford the incentive from which
accumulation invariably proceeds. The motive that lies at the root
of ownership is emulation; and the same motive of emulation continues
active in the further development of the institution to which it has
given rise and in the development of all those features of the social
structure which this institution of ownership touches. The possession of
wealth confers honour; it is an invidious distinction. Nothing equally
cogent can be said for the consumption of goods, nor for any other
conceivable incentive to acquisition, and especially not for any
incentive to accumulation of wealth.

It is of course not to be overlooked that in a community where nearly
all goods are private property the necessity of earning a livelihood
is a powerful and ever present incentive for the poorer members of
the community. The need of subsistence and of an increase of physical
comfort may for a time be the dominant motive of acquisition for those
classes who are habitually employed at manual labour, whose subsistence
is on a precarious footing, who possess little and ordinarily accumulate
little; but it will appear in the course of the discussion that even in
the case of these impecunious classes the predominance of the motive of
physical want is not so decided as has sometimes been assumed. On the
other hand, so far as regards those members and classes of the community
who are chiefly concerned in the accumulation of wealth, the incentive
of subsistence or of physical comfort never plays a considerable part.
Ownership began and grew into a human institution on grounds unrelated
to the subsistence minimum. The dominant incentive was from the outset
the invidious distinction attaching to wealth, and, save temporarily and
by exception, no other motive has usurped the primacy at any later stage
of the development.

Property set out with being booty held as trophies of the successful
raid. So long as the group had departed and so long as it still stood
in close contact with other hostile groups, the utility of things or
persons owned lay chiefly in an invidious comparison between their
possessor and the enemy from whom they were taken. The habit of
distinguishing between the interests of the individual and those of
the group to which he belongs is apparently a later growth. Invidious
comparison between the possessor of the honorific booty and his less
successful neighbours within the group was no doubt present early as an
element of the utility of the things possessed, though this was not at
the outset the chief element of their value. The man's prowess was
still primarily the group's prowess, and the possessor of the booty
felt himself to be primarily the keeper of the honour of his group. This
appreciation of exploit from the communal point of view is met with also
at later stages of social growth, especially as regards the laurels of
war.

But as soon as the custom of individual ownership begins to gain
consistency, the point of view taken in making the invidious comparison
on which private property rests will begin to change. Indeed, the one
change is but the reflex of the other. The initial phase of ownership,
the phase of acquisition by naive seizure and conversion, begins to pass
into the subsequent stage of an incipient organization of industry on
the basis of private property (in slaves); the horde develops into a
more or less self-sufficing industrial community; possessions then come
to be valued not so much as evidence of successful foray, but rather as
evidence of the prepotence of the possessor of these goods over other
individuals within the community. The invidious comparison now becomes
primarily a comparison of the owner with the other members of the
group. Property is still of the nature of trophy, but, with the cultural
advance, it becomes more and more a trophy of successes scored in the
game of ownership carried on between the members of the group under the
quasi-peaceable methods of nomadic life.

Gradually, as industrial activity further displaced predatory activity
in the community's everyday life and in men's habits of thought,
accumulated property more and more replaces trophies of predatory
exploit as the conventional exponent of prepotence and success. With the
growth of settled industry, therefore, the possession of wealth gains in
relative importance and effectiveness as a customary basis of repute and
esteem. Not that esteem ceases to be awarded on the basis of other, more
direct evidence of prowess; not that successful predatory aggression or
warlike exploit ceases to call out the approval and admiration of the
crowd, or to stir the envy of the less successful competitors; but
the opportunities for gaining distinction by means of this direct
manifestation of superior force grow less available both in scope and
frequency. At the same time opportunities for industrial aggression, and
for the accumulation of property, increase in scope and availability.
And it is even more to the point that property now becomes the
most easily recognised evidence of a reputable degree of success as
distinguished from heroic or signal achievement. It therefore becomes
the conventional basis of esteem. Its possession in some amount becomes
necessary in order to any reputable standing in the community. It
becomes indispensable to accumulate, to acquire property, in order to
retain one's good name. When accumulated goods have in this way once
become the accepted badge of efficiency, the possession of wealth
presently assumes the character of an independent and definitive basis
of esteem. The possession of goods, whether acquired aggressively by
one's own exertion or passively by transmission through inheritance from
others, becomes a conventional basis of reputability. The possession
of wealth, which was at the outset valued simply as an evidence of
efficiency, becomes, in popular apprehension, itself a meritorious act.
Wealth is now itself intrinsically honourable and confers honour on
its possessor. By a further refinement, wealth acquired passively by
transmission from ancestors or other antecedents presently becomes even
more honorific than wealth acquired by the possessor's own effort;
but this distinction belongs at a later stage in the evolution of the
pecuniary culture and will be spoken of in its place.

Prowess and exploit may still remain the basis of award of the highest
popular esteem, although the possession of wealth has become the basis
of common place reputability and of a blameless social standing.
The predatory instinct and the consequent approbation of predatory
efficiency are deeply ingrained in the habits of thought of those
peoples who have passed under the discipline of a protracted predatory
culture. According to popular award, the highest honours within human
reach may, even yet, be those gained by an unfolding of extraordinary
predatory efficiency in war, or by a quasi-predatory efficiency in
statecraft; but for the purposes of a commonplace decent standing in the
community these means of repute have been replaced by the acquisition
and accumulation of goods. In order to stand well in the eyes of the
community, it is necessary to come up to a certain, somewhat indefinite,
conventional standard of wealth; just as in the earlier predatory stage
it is necessary for the barbarian man to come up to the tribe's standard
of physical endurance, cunning, and skill at arms. A certain standard
of wealth in the one case, and of prowess in the other, is a necessary
condition of reputability, and anything in excess of this normal amount
is meritorious.

Those members of the community who fall short of this, somewhat
indefinite, normal degree of prowess or of property suffer in the esteem
of their fellow-men; and consequently they suffer also in their own
esteem, since the usual basis of self-respect is the respect accorded by
one's neighbours. Only individuals with an aberrant temperament can in
the long run retain their self-esteem in the face of the disesteem of
their fellows. Apparent exceptions to the rule are met with, especially
among people with strong religious convictions. But these apparent
exceptions are scarcely real exceptions, since such persons commonly
fall back on the putative approbation of some supernatural witness of
their deeds.

So soon as the possession of property becomes the basis of popular
esteem, therefore, it becomes also a requisite to the complacency which
we call self-respect. In any community where goods are held in severalty
it is necessary, in order to his own peace of mind, that an individual
should possess as large a portion of goods as others with whom he is
accustomed to class himself; and it is extremely gratifying to
possess something more than others. But as fast as a person makes new
acquisitions, and becomes accustomed to the resulting new standard of
wealth, the new standard forthwith ceases to afford appreciably greater
satisfaction than the earlier standard did. The tendency in any case is
constantly to make the present pecuniary standard the point of departure
for a fresh increase of wealth; and this in turn gives rise to a new
standard of sufficiency and a new pecuniary classification of one's
self as compared with one's neighbours. So far as concerns the present
question, the end sought by accumulation is to rank high in comparison
with the rest of the community in point of pecuniary strength. So long
as the comparison is distinctly unfavourable to himself, the normal,
average individual will live in chronic dissatisfaction with his present
lot; and when he has reached what may be called the normal pecuniary
standard of the community, or of his class in the community, this
chronic dissatisfaction will give place to a restless straining to place
a wider and ever-widening pecuniary interval between himself and
this average standard. The invidious comparison can never become so
favourable to the individual making it that he would not gladly rate
himself still higher relatively to his competitors in the struggle for
pecuniary reputability.

In the nature of the case, the desire for wealth can scarcely be
satiated in any individual instance, and evidently a satiation of the
average or general desire for wealth is out of the question. However
widely, or equally, or "fairly", it may be distributed, no general
increase of the community's wealth can make any approach to satiating
this need, the ground of which is the desire of every one to excel every
one else in the accumulation of goods. If, as is sometimes assumed, the
incentive to accumulation were the want of subsistence or of physical
comfort, then the aggregate economic wants of a community might
conceivably be satisfied at some point in the advance of industrial
efficiency; but since the struggle is substantially a race for
reputability on the basis of an invidious comparison, no approach to
a definitive attainment is possible.

What has just been said must not be taken to mean that there are no
other incentives to acquisition and accumulation than this desire to
excel in pecuniary standing and so gain the esteem and envy of one's
fellow-men. The desire for added comfort and security from want is
present as a motive at every stage of the process of accumulation in
a modern industrial community; although the standard of sufficiency in
these respects is in turn greatly affected by the habit of pecuniary
emulation. To a great extent this emulation shapes the methods and
selects the objects of expenditure for personal comfort and decent
livelihood.

Besides this, the power conferred by wealth also affords a motive
to accumulation. That propensity for purposeful activity and that
repugnance to all futility of effort which belong to man by virtue of
his character as an agent do not desert him when he emerges from the
naive communal culture where the dominant note of life is the unanalysed
and undifferentiated solidarity of the individual with the group with
which his life is bound up. When he enters upon the predatory stage,
where self-seeking in the narrower sense becomes the dominant note, this
propensity goes with him still, as the pervasive trait that shapes his
scheme of life. The propensity for achievement and the repugnance to
futility remain the underlying economic motive. The propensity changes
only in the form of its expression and in the proximate objects to which
it directs the man's activity. Under the regime of individual ownership
the most available means of visibly achieving a purpose is that afforded
by the acquisition and accumulation of goods; and as the self-regarding
antithesis between man and man reaches fuller consciousness, the
propensity for achievement--the instinct of workmanship--tends more
and more to shape itself into a straining to excel others in pecuniary
achievement. Relative success, tested by an invidious pecuniary
comparison with other men, becomes the conventional end of action. The
currently accepted legitimate end of effort becomes the achievement of
a favourable comparison with other men; and therefore the repugnance to
futility to a good extent coalesces with the incentive of emulation. It
acts to accentuate the struggle for pecuniary reputability by visiting
with a sharper disapproval all shortcoming and all evidence of
shortcoming in point of pecuniary success. Purposeful effort comes to
mean, primarily, effort directed to or resulting in a more creditable
showing of accumulated wealth. Among the motives which lead men to
accumulate wealth, the primacy, both in scope and intensity, therefore,
continues to belong to this motive of pecuniary emulation.

In making use of the term "invidious", it may perhaps be unnecessary to
remark, there is no intention to extol or depreciate, or to commend or
deplore any of the phenomena which the word is used to characterise. The
term is used in a technical sense as describing a comparison of persons
with a view to rating and grading them in respect of relative worth or
value--in an aesthetic or moral sense--and so awarding and defining
the relative degrees of complacency with which they may legitimately be
contemplated by themselves and by others. An invidious comparison is a
process of valuation of persons in respect of worth.




\chapter{Conspicuous Leisure}

If its working were not disturbed by other economic forces or other
features of the emulative process, the immediate effect of such a
pecuniary struggle as has just been described in outline would be to
make men industrious and frugal. This result actually follows, in some
measure, so far as regards the lower classes, whose ordinary means of
acquiring goods is productive labour. This is more especially true
of the labouring classes in a sedentary community which is at an
agricultural stage of industry, in which there is a considerable
subdivision of industry, and whose laws and customs secure to these
classes a more or less definite share of the product of their industry.
These lower classes can in any case not avoid labour, and the imputation
of labour is therefore not greatly derogatory to them, at least not
within their class. Rather, since labour is their recognised and
accepted mode of life, they take some emulative pride in a reputation
for efficiency in their work, this being often the only line of
emulation that is open to them. For those for whom acquisition and
emulation is possible only within the field of productive efficiency
and thrift, the struggle for pecuniary reputability will in some
measure work out in an increase of diligence and parsimony. But certain
secondary features of the emulative process, yet to be spoken of,
come in to very materially circumscribe and modify emulation in these
directions among the pecuniary inferior classes as well as among the
superior class.

But it is otherwise with the superior pecuniary class, with which we
are here immediately concerned. For this class also the incentive
to diligence and thrift is not absent; but its action is so greatly
qualified by the secondary demands of pecuniary emulation, that any
inclination in this direction is practically overborne and any incentive
to diligence tends to be of no effect. The most imperative of these
secondary demands of emulation, as well as the one of widest scope, is
the requirement of abstention from productive work. This is true in an
especial degree for the barbarian stage of culture. During the predatory
culture labour comes to be associated in men's habits of thought
with weakness and subjection to a master. It is therefore a mark of
inferiority, and therefore comes to be accounted unworthy of man in his
best estate. By virtue of this tradition labour is felt to be debasing,
and this tradition has never died out. On the contrary, with the advance
of social differentiation it has acquired the axiomatic force due to
ancient and unquestioned prescription.

In order to gain and to hold the esteem of men it is not sufficient
merely to possess wealth or power. The wealth or power must be put in
evidence, for esteem is awarded only on evidence. And not only does the
evidence of wealth serve to impress one's importance on others and
to keep their sense of his importance alive and alert, but it is of
scarcely less use in building up and preserving one's self-complacency.
In all but the lowest stages of culture the normally constituted man is
comforted and upheld in his self-respect by "decent surroundings" and
by exemption from "menial offices". Enforced departure from his habitual
standard of decency, either in the paraphernalia of life or in the kind
and amount of his everyday activity, is felt to be a slight upon his
human dignity, even apart from all conscious consideration of the
approval or disapproval of his fellows.

The archaic theoretical distinction between the base and the honourable
in the manner of a man's life retains very much of its ancient force
even today. So much so that there are few of the better class who are not
possessed of an instinctive repugnance for the vulgar forms of labour.
We have a realising sense of ceremonial uncleanness attaching in an
especial degree to the occupations which are associated in our habits of
thought with menial service. It is felt by all persons of refined taste
that a spiritual contamination is inseparable from certain offices that
are conventionally required of servants. Vulgar surroundings, mean (that
is to say, inexpensive) habitations, and vulgarly productive occupations
are unhesitatingly condemned and avoided. They are incompatible with
life on a satisfactory spiritual plane---with "high thinking". From the
days of the Greek philosophers to the present, a degree of leisure and
of exemption from contact with such industrial processes as serve the
immediate everyday purposes of human life has ever been recognised by
thoughtful men as a prerequisite to a worthy or beautiful, or even a
blameless, human life. In itself and in its consequences the life of
leisure is beautiful and ennobling in all civilised men's eyes.

This direct, subjective value of leisure and of other evidences of
wealth is no doubt in great part secondary and derivative. It is in part
a reflex of the utility of leisure as a means of gaining the respect
of others, and in part it is the result of a mental substitution. The
performance of labour has been accepted as a conventional evidence of
inferior force; therefore it comes itself, by a mental short-cut, to be
regarded as intrinsically base.

During the predatory stage proper, and especially during the earlier
stages of the quasi-peaceable development of industry that follows the
predatory stage, a life of leisure is the readiest and most conclusive
evidence of pecuniary strength, and therefore of superior force;
provided always that the gentleman of leisure can live in manifest ease
and comfort. At this stage wealth consists chiefly of slaves, and the
benefits accruing from the possession of riches and power take the
form chiefly of personal service and the immediate products of personal
service. Conspicuous abstention from labour therefore becomes the
conventional mark of superior pecuniary achievement and the conventional
index of reputability; and conversely, since application to productive
labour is a mark of poverty and subjection, it becomes inconsistent with
a reputable standing in the community. Habits of industry and thrift,
therefore, are not uniformly furthered by a prevailing pecuniary
emulation. On the contrary, this kind of emulation indirectly
discountenances participation in productive labour. Labour would
unavoidably become dishonourable, as being an evidence indecorous under
the ancient tradition handed down from an earlier cultural stage. The
ancient tradition of the predatory culture is that productive effort is
to be shunned as being unworthy of able-bodied men, and this tradition
is reinforced rather than set aside in the passage from the predatory to
the quasi-peaceable manner of life.

Even if the institution of a leisure class had not come in with the
first emergence of individual ownership, by force of the dishonour
attaching to productive employment, it would in any case have come in
as one of the early consequences of ownership. And it is to be remarked
that while the leisure class existed in theory from the beginning of
predatory culture, the institution takes on a new and fuller meaning
with the transition from the predatory to the next succeeding pecuniary
stage of culture. It is from this time forth a "leisure class" in fact
as well as in theory. From this point dates the institution of the
leisure class in its consummate form.

During the predatory stage proper the distinction between the leisure
and the labouring class is in some degree a ceremonial distinction only.
The able bodied men jealously stand aloof from whatever is in their
apprehension, menial drudgery; but their activity in fact contributes
appreciably to the sustenance of the group. The subsequent stage of
quasi-peaceable industry is usually characterised by an established
chattel slavery, herds of cattle, and a servile class of herdsmen and
shepherds; industry has advanced so far that the community is no longer
dependent for its livelihood on the chase or on any other form of
activity that can fairly be classed as exploit. From this point on, the
characteristic feature of leisure class life is a conspicuous exemption
from all useful employment.

The normal and characteristic occupations of the class in this mature
phase of its life history are in form very much the same as in its
earlier days. These occupations are government, war, sports, and devout
observances. Persons unduly given to difficult theoretical niceties
may hold that these occupations are still incidentally and indirectly
"productive"; but it is to be noted as decisive of the question in hand
that the ordinary and ostensible motive of the leisure class in
engaging in these occupations is assuredly not an increase of wealth by
productive effort. At this as at any other cultural stage, government
and war are, at least in part, carried on for the pecuniary gain of
those who engage in them; but it is gain obtained by the honourable
method of seizure and conversion. These occupations are of the nature of
predatory, not of productive, employment. Something similar may be said
of the chase, but with a difference. As the community passes out of the
hunting stage proper, hunting gradually becomes differentiated into two
distinct employments. On the one hand it is a trade, carried on chiefly
for gain; and from this the element of exploit is virtually absent,
or it is at any rate not present in a sufficient degree to clear the
pursuit of the imputation of gainful industry. On the other hand, the
chase is also a sport--an exercise of the predatory impulse simply.
As such it does not afford any appreciable pecuniary incentive, but it
contains a more or less obvious element of exploit. It is this latter
development of the chase--purged of all imputation of handicraft--that
alone is meritorious and fairly belongs in the scheme of life of the
developed leisure class.

Abstention from labour is not only a honorific or meritorious act,
but it presently comes to be a requisite of decency. The insistence on
property as the basis of reputability is very naive and very imperious
during the early stages of the accumulation of wealth. Abstention
from labour is the convenient evidence of wealth and is therefore
the conventional mark of social standing; and this insistence on the
meritoriousness of wealth leads to a more strenuous insistence on
leisure. \emph{Nota notae est nota rei ipsius}. According to well established
laws of human nature, prescription presently seizes upon this
conventional evidence of wealth and fixes it in men's habits of thought
as something that is in itself substantially meritorious and ennobling;
while productive labour at the same time and by a like process becomes
in a double sense intrinsically unworthy. Prescription ends by making
labour not only disreputable in the eyes of the community, but morally
impossible to the noble, freeborn man, and incompatible with a worthy
life.

This tabu on labour has a further consequence in the industrial
differentiation of classes. As the population increases in density
and the predatory group grows into a settled industrial community, the
constituted authorities and the customs governing ownership gain in
scope and consistency. It then presently becomes impracticable to
accumulate wealth by simple seizure, and, in logical consistency,
acquisition by industry is equally impossible for high minded and
impecunious men. The alternative open to them is beggary or privation.
Wherever the canon of conspicuous leisure has a chance undisturbed to
work out its tendency, there will therefore emerge a secondary, and in a
sense spurious, leisure class--abjectly poor and living in a precarious
life of want and discomfort, but morally unable to stoop to gainful
pursuits. The decayed gentleman and the lady who has seen better days
are by no means unfamiliar phenomena even now. This pervading sense
of the indignity of the slightest manual labour is familiar to all
civilized peoples, as well as to peoples of a less advanced pecuniary
culture. In persons of a delicate sensibility who have long been
habituated to gentle manners, the sense of the shamefulness of manual
labour may become so strong that, at a critical juncture, it will even
set aside the instinct of self-preservation. So, for instance, we are
told of certain Polynesian chiefs, who, under the stress of good form,
preferred to starve rather than carry their food to their mouths with
their own hands. It is true, this conduct may have been due, at least in
part, to an excessive sanctity or tabu attaching to the chief's person.
The tabu would have been communicated by the contact of his hands, and
so would have made anything touched by him unfit for human food. But the
tabu is itself a derivative of the unworthiness or moral incompatibility
of labour; so that even when construed in this sense the conduct of the
Polynesian chiefs is truer to the canon of honorific leisure than would
at first appear. A better illustration, or at least a more unmistakable
one, is afforded by a certain king of France, who is said to have lost
his life through an excess of moral stamina in the observance of good
form. In the absence of the functionary whose office it was to shift his
master's seat, the king sat uncomplaining before the fire and suffered
his royal person to be toasted beyond recovery. But in so doing he saved
his Most Christian Majesty from menial contamination. 
\begin{displayquote}
\emph{
Summum crede nefas animam praeferre pudori, \\
Et propter vitam vivendi perdere causas.}
\end{displayquote}
It has already been remarked that the term "leisure", as here used, does
not connote indolence or quiescence. What it connotes is non-productive
consumption of time. Time is consumed non-productively (1) from a
sense of the unworthiness of productive work, and (2) as an evidence
of pecuniary ability to afford a life of idleness. But the whole of the
life of the gentleman of leisure is not spent before the eyes of the
spectators who are to be impressed with that spectacle of honorific
leisure which in the ideal scheme makes up his life. For some part of
the time his life is perforce withdrawn from the public eye, and of this
portion which is spent in private the gentleman of leisure should, for
the sake of his good name, be able to give a convincing account. He
should find some means of putting in evidence the leisure that is not
spent in the sight of the spectators. This can be done only indirectly,
through the exhibition of some tangible, lasting results of the leisure
so spent--in a manner analogous to the familiar exhibition of tangible,
lasting products of the labour performed for the gentleman of leisure by
handicraftsmen and servants in his employ.

The lasting evidence of productive labour is its material
product--commonly some article of consumption. In the case of exploit it
is similarly possible and usual to procure some tangible result that may
serve for exhibition in the way of trophy or booty. At a later phase
of the development it is customary to assume some badge of insignia of
honour that will serve as a conventionally accepted mark of exploit, and
which at the same time indicates the quantity or degree of exploit of
which it is the symbol. As the population increases in density, and as
human relations grow more complex and numerous, all the details of life
undergo a process of elaboration and selection; and in this process of
elaboration the use of trophies develops into a system of rank, titles,
degrees and insignia, typical examples of which are heraldic devices,
medals, and honorary decorations.

As seen from the economic point of view, leisure, considered as an
employment, is closely allied in kind with the life of exploit; and the
achievements which characterise a life of leisure, and which remain as
its decorous criteria, have much in common with the trophies of exploit.
But leisure in the narrower sense, as distinct from exploit and from any
ostensibly productive employment of effort on objects which are of no
intrinsic use, does not commonly leave a material product. The criteria
of a past performance of leisure therefore commonly take the form
of "immaterial" goods. Such immaterial evidences of past leisure are
quasi-scholarly or quasi-artistic accomplishments and a knowledge of
processes and incidents which do not conduce directly to the furtherance
of human life. So, for instance, in our time there is the knowledge
of the dead languages and the occult sciences; of correct spelling; of
syntax and prosody; of the various forms of domestic music and other
household art; of the latest properties of dress, furniture, and
equipage; of games, sports, and fancy-bred animals, such as dogs and
race-horses. In all these branches of knowledge the initial motive from
which their acquisition proceeded at the outset, and through which they
first came into vogue, may have been something quite different from
the wish to show that one's time had not been spent in industrial
employment; but unless these accomplishments had approved themselves as
serviceable evidence of an unproductive expenditure of time, they would
not have survived and held their place as conventional accomplishments
of the leisure class.

These accomplishments may, in some sense, be classed as branches of
learning. Beside and beyond these there is a further range of social
facts which shade off from the region of learning into that of physical
habit and dexterity. Such are what is known as manners and breeding,
polite usage, decorum, and formal and ceremonial observances generally.
This class of facts are even more immediately and obtrusively presented
to the observation, and they therefore more widely and more imperatively
insisted on as required evidences of a reputable degree of leisure. It
is worth while to remark that all that class of ceremonial observances
which are classed under the general head of manners hold a more
important place in the esteem of men during the stage of culture
at which conspicuous leisure has the greatest vogue as a mark of
reputability, than at later stages of the cultural development. The
barbarian of the quasi-peaceable stage of industry is notoriously a more
high-bred gentleman, in all that concerns decorum, than any but the very
exquisite among the men of a later age. Indeed, it is well known, or
at least it is currently believed, that manners have progressively
deteriorated as society has receded from the patriarchal stage. Many a
gentleman of the old school has been provoked to remark regretfully upon
the under-bred manners and bearing of even the better classes in the
modern industrial communities; and the decay of the ceremonial code--or
as it is otherwise called, the vulgarisation of life--among the
industrial classes proper has become one of the chief enormities
of latter-day civilisation in the eyes of all persons of delicate
sensibilities. The decay which the code has suffered at the hands of a
busy people testifies--all depreciation apart--to the fact that decorum
is a product and an exponent of leisure class life and thrives in full
measure only under a regime of status.

The origin, or better the derivation, of manners is no doubt, to
be sought elsewhere than in a conscious effort on the part of the
well-mannered to show that much time has been spent in acquiring them.
The proximate end of innovation and elaboration has been the
higher effectiveness of the new departure in point of beauty or of
expressiveness. In great part the ceremonial code of decorous usages
owes its beginning and its growth to the desire to conciliate or to
show good-will, as anthropologists and sociologists are in the habit
of assuming, and this initial motive is rarely if ever absent from the
conduct of well-mannered persons at any stage of the later development.
Manners, we are told, are in part an elaboration of gesture, and in part
they are symbolical and conventionalised survivals representing former
acts of dominance or of personal service or of personal contact. In
large part they are an expression of the relation of status,--a symbolic
pantomime of mastery on the one hand and of subservience on the other.
Wherever at the present time the predatory habit of mind, and the
consequent attitude of mastery and of subservience, gives its character
to the accredited scheme of life, there the importance of all punctilios
of conduct is extreme, and the assiduity with which the ceremonial
observance of rank and titles is attended to approaches closely to the
ideal set by the barbarian of the quasi-peaceable nomadic culture. Some
of the Continental countries afford good illustrations of this spiritual
survival. In these communities the archaic ideal is similarly approached
as regards the esteem accorded to manners as a fact of intrinsic worth.

Decorum set out with being symbol and pantomime and with having utility
only as an exponent of the facts and qualities symbolised; but it
presently suffered the transmutation which commonly passes over
symbolical facts in human intercourse. Manners presently came, in
popular apprehension, to be possessed of a substantial utility in
themselves; they acquired a sacramental character, in great measure
independent of the facts which they originally prefigured. Deviations
from the code of decorum have become intrinsically odious to all
men, and good breeding is, in everyday apprehension, not simply an
adventitious mark of human excellence, but an integral feature of
the worthy human soul. There are few things that so touch us with
instinctive revulsion as a breach of decorum; and so far have we
progressed in the direction of imputing intrinsic utility to the
ceremonial observances of etiquette that few of us, if any, can
dissociate an offence against etiquette from a sense of the substantial
unworthiness of the offender. A breach of faith may be condoned, but a
breach of decorum can not. "Manners maketh man."

None the less, while manners have this intrinsic utility, in the
apprehension of the performer and the beholder alike, this sense of the
intrinsic rightness of decorum is only the proximate ground of the vogue
of manners and breeding. Their ulterior, economic ground is to be sought
in the honorific character of that leisure or non-productive employment
of time and effort without which good manners are not acquired. The
knowledge and habit of good form come only by long-continued use.
Refined tastes, manners, habits of life are a useful evidence of
gentility, because good breeding requires time, application and expense,
and can therefore not be compassed by those whose time and energy are
taken up with work. A knowledge of good form is \emph{prima facie} evidence
that that portion of the well-bred person's life which is not spent
under the observation of the spectator has been worthily spent in
acquiring accomplishments that are of no lucrative effect. In the last
analysis the value of manners lies in the fact that they are the voucher
of a life of leisure. Therefore, conversely, since leisure is the
conventional means of pecuniary repute, the acquisition of some
proficiency in decorum is incumbent on all who aspire to a modicum of
pecuniary decency.

So much of the honourable life of leisure as is not spent in the sight
of spectators can serve the purposes of reputability only in so far as
it leaves a tangible, visible result that can be put in evidence and can
be measured and compared with products of the same class exhibited
by competing aspirants for repute. Some such effect, in the way of
leisurely manners and carriage, etc., follows from simple persistent
abstention from work, even where the subject does not take thought
of the matter and studiously acquire an air of leisurely opulence and
mastery. Especially does it seem to be true that a life of leisure
in this way persisted in through several generations will leave a
persistent, ascertainable effect in the conformation of the person,
and still more in his habitual bearing and demeanour. But all the
suggestions of a cumulative life of leisure, and all the proficiency
in decorum that comes by the way of passive habituation, may be further
improved upon by taking thought and assiduously acquiring the marks
of honourable leisure, and then carrying the exhibition of these
adventitious marks of exemption from employment out in a strenuous and
systematic discipline. Plainly, this is a point at which a diligent
application of effort and expenditure may materially further the
attainment of a decent proficiency in the leisure-class properties.
Conversely, the greater the degree of proficiency and the more patent
the evidence of a high degree of habituation to observances which
serve no lucrative or other directly useful purpose, the greater
the consumption of time and substance impliedly involved in their
acquisition, and the greater the resultant good repute. Hence under the
competitive struggle for proficiency in good manners, it comes about
that much pains in taken with the cultivation of habits of decorum; and
hence the details of decorum develop into a comprehensive discipline,
conformity to which is required of all who would be held blameless in
point of repute. And hence, on the other hand, this conspicuous leisure
of which decorum is a ramification grows gradually into a laborious
drill in deportment and an education in taste and discrimination as
to what articles of consumption are decorous and what are the decorous
methods of consuming them.

In this connection it is worthy of notice that the possibility of
producing pathological and other idiosyncrasies of person and manner by
shrewd mimicry and a systematic drill have been turned to account in
the deliberate production of a cultured class--often with a very happy
effect. In this way, by the process vulgarly known as snobbery, a
syncopated evolution of gentle birth and breeding is achieved in
the case of a goodly number of families and lines of descent. This
syncopated gentle birth gives results which, in point of serviceability
as a leisure-class factor in the population, are in no wise
substantially inferior to others who may have had a longer but less
arduous training in the pecuniary properties.

There are, moreover, measureable degrees of conformity to the latest
accredited code of the punctilios as regards decorous means and methods
of consumption. Differences between one person and another in the
degree of conformity to the ideal in these respects can be compared,
and persons may be graded and scheduled with some accuracy and effect
according to a progressive scale of manners and breeding. The award
of reputability in this regard is commonly made in good faith, on
the ground of conformity to accepted canons of taste in the matters
concerned, and without conscious regard to the pecuniary standing or the
degree of leisure practised by any given candidate for reputability; but
the canons of taste according to which the award is made are constantly
under the surveillance of the law of conspicuous leisure, and are indeed
constantly undergoing change and revision to bring them into closer
conformity with its requirements. So that while the proximate ground of
discrimination may be of another kind, still the pervading principle and
abiding test of good breeding is the requirement of a substantial and
patent waste of time. There may be some considerable range of variation
in detail within the scope of this principle, but they are variations of
form and expression, not of substance.

Much of the courtesy of everyday intercourse is of course a direct
expression of consideration and kindly good-will, and this element
of conduct has for the most part no need of being traced back to any
underlying ground of reputability to explain either its presence or the
approval with which it is regarded; but the same is not true of the code
of properties. These latter are expressions of status. It is of course
sufficiently plain, to any one who cares to see, that our bearing
towards menials and other pecuniary dependent inferiors is the bearing
of the superior member in a relation of status, though its manifestation
is often greatly modified and softened from the original expression of
crude dominance. Similarly, our bearing towards superiors, and in
great measure towards equals, expresses a more or less conventionalised
attitude of subservience. Witness the masterful presence of the
high-minded gentleman or lady, which testifies to so much of dominance
and independence of economic circumstances, and which at the same time
appeals with such convincing force to our sense of what is right and
gracious. It is among this highest leisure class, who have no superiors
and few peers, that decorum finds its fullest and maturest expression;
and it is this highest class also that gives decorum that definite
formulation which serves as a canon of conduct for the classes beneath.
And there also the code is most obviously a code of status and shows
most plainly its incompatibility with all vulgarly productive work. A
divine assurance and an imperious complaisance, as of one habituated
to require subservience and to take no thought for the morrow, is the
birthright and the criterion of the gentleman at his best; and it is in
popular apprehension even more than that, for this demeanour is accepted
as an intrinsic attribute of superior worth, before which the base-born
commoner delights to stoop and yield.

As has been indicated in an earlier chapter, there is reason to believe
that the institution of ownership has begun with the ownership of
persons, primarily women. The incentives to acquiring such property have
apparently been: (1) a propensity for dominance and coercion; (2) the
utility of these persons as evidence of the prowess of the owner; (3)
the utility of their services.

Personal service holds a peculiar place in the economic development.
During the stage of quasi-peaceable industry, and especially during the
earlier development of industry within the limits of this general stage,
the utility of their services seems commonly to be the dominant motive
to the acquisition of property in persons. Servants are valued for their
services. But the dominance of this motive is not due to a decline
in the absolute importance of the other two utilities possessed by
servants. It is rather that the altered circumstance of life accentuate
the utility of servants for this last-named purpose. Women and other
slaves are highly valued, both as an evidence of wealth and as a means
of accumulating wealth. Together with cattle, if the tribe is a pastoral
one, they are the usual form of investment for a profit. To such an
extent may female slavery give its character to the economic life under
the quasi-peaceable culture that the women even comes to serve as a unit
of value among peoples occupying this cultural stage--as for instance in
Homeric times. Where this is the case there need be little question but
that the basis of the industrial system is chattel slavery and that the
women are commonly slaves. The great, pervading human relation in such a
system is that of master and servant. The accepted evidence of wealth is
the possession of many women, and presently also of other slaves engaged
in attendance on their master's person and in producing goods for him.

A division of labour presently sets in, whereby personal service and
attendance on the master becomes the special office of a portion of the
servants, while those who are wholly employed in industrial occupations
proper are removed more and more from all immediate relation to the
person of their owner. At the same time those servants whose office
is personal service, including domestic duties, come gradually to be
exempted from productive industry carried on for gain.

This process of progressive exemption from the common run of industrial
employment will commonly begin with the exemption of the wife, or the
chief wife. After the community has advanced to settled habits of life,
wife-capture from hostile tribes becomes impracticable as a customary
source of supply. Where this cultural advance has been achieved, the
chief wife is ordinarily of gentle blood, and the fact of her being so
will hasten her exemption from vulgar employment. The manner in which
the concept of gentle blood originates, as well as the place which it
occupies in the development of marriage, cannot be discussed in this
place. For the purpose in hand it will be sufficient to say that gentle
blood is blood which has been ennobled by protracted contact with
accumulated wealth or unbroken prerogative. The women with these
antecedents is preferred in marriage, both for the sake of a resulting
alliance with her powerful relatives and because a superior worth is
felt to inhere in blood which has been associated with many goods and
great power. She will still be her husband's chattel, as she was her
father's chattel before her purchase, but she is at the same time of
her father's gentle blood; and hence there is a moral incongruity in her
occupying herself with the debasing employments of her fellow-servants.
However completely she may be subject to her master, and however
inferior to the male members of the social stratum in which her birth
has placed her, the principle that gentility is transmissible will act
to place her above the common slave; and so soon as this principle has
acquired a prescriptive authority it will act to invest her in some
measure with that prerogative of leisure which is the chief mark of
gentility. Furthered by this principle of transmissible gentility the
wife's exemption gains in scope, if the wealth of her owner permits it,
until it includes exemption from debasing menial service as well as from
handicraft. As the industrial development goes on and property becomes
massed in relatively fewer hands, the conventional standard of wealth of
the upper class rises. The same tendency to exemption from handicraft,
and in the course of time from menial domestic employments, will then
assert itself as regards the other wives, if such there are, and also as
regards other servants in immediate attendance upon the person of their
master. The exemption comes more tardily the remoter the relation in
which the servant stands to the person of the master.

If the pecuniary situation of the master permits it, the development of
a special class of personal or body servants is also furthered by the
very grave importance which comes to attach to this personal service.
The master's person, being the embodiment of worth and honour, is of
the most serious consequence. Both for his reputable standing in the
community and for his self-respect, it is a matter of moment that he
should have at his call efficient specialised servants, whose attendance
upon his person is not diverted from this their chief office by any
by-occupation. These specialised servants are useful more for show
than for service actually performed. In so far as they are not kept for
exhibition simply, they afford gratification to their master chiefly in
allowing scope to his propensity for dominance. It is true, the care of
the continually increasing household apparatus may require added labour;
but since the apparatus is commonly increased in order to serve as
a means of good repute rather than as a means of comfort, this
qualification is not of great weight. All these lines of utility are
better served by a larger number of more highly specialised servants.
There results, therefore, a constantly increasing differentiation and
multiplication of domestic and body servants, along with a concomitant
progressive exemption of such servants from productive labour. By virtue
of their serving as evidence of ability to pay, the office of such
domestics regularly tends to include continually fewer duties, and their
service tends in the end to become nominal only. This is especially true
of those servants who are in most immediate and obvious attendance upon
their master. So that the utility of these comes to consist, in great
part, in their conspicuous exemption from productive labour and in
the evidence which this exemption affords of their master's wealth and
power.

After some considerable advance has been made in the practice of
employing a special corps of servants for the performance of a
conspicuous leisure in this manner, men begin to be preferred above
women for services that bring them obtrusively into view. Men,
especially lusty, personable fellows, such as footmen and other menials
should be, are obviously more powerful and more expensive than women.
They are better fitted for this work, as showing a larger waste of time
and of human energy. Hence it comes about that in the economy of the
leisure class the busy housewife of the early patriarchal days, with her
retinue of hard-working handmaidens, presently gives place to the lady
and the lackey.

In all grades and walks of life, and at any stage of the economic
development, the leisure of the lady and of the lackey differs from the
leisure of the gentleman in his own right in that it is an occupation of
an ostensibly laborious kind. It takes the form, in large measure, of
a painstaking attention to the service of the master, or to the
maintenance and elaboration of the household paraphernalia; so that
it is leisure only in the sense that little or no productive work is
performed by this class, not in the sense that all appearance of
labour is avoided by them. The duties performed by the lady, or by the
household or domestic servants, are frequently arduous enough, and they
are also frequently directed to ends which are considered extremely
necessary to the comfort of the entire household. So far as these
services conduce to the physical efficiency or comfort of the master
or the rest of the household, they are to be accounted productive work.
Only the residue of employment left after deduction of this effective
work is to be classed as a performance of leisure.

But much of the services classed as household cares in modern everyday
life, and many of the "utilities" required for a comfortable existence
by civilised man, are of a ceremonial character. They are, therefore,
properly to be classed as a performance of leisure in the sense in which
the term is here used. They may be none the less imperatively necessary
from the point of view of decent existence: they may be none the less
requisite for personal comfort even, although they may be chiefly or
wholly of a ceremonial character. But in so far as they partake of this
character they are imperative and requisite because we have been taught
to require them under pain of ceremonial uncleanness or unworthiness. We
feel discomfort in their absence, but not because their absence results
directly in physical discomfort; nor would a taste not trained to
discriminate between the conventionally good and the conventionally bad
take offence at their omission. In so far as this is true the labour
spent in these services is to be classed as leisure; and when performed
by others than the economically free and self-directed head of the
establishment, they are to be classed as vicarious leisure.

The vicarious leisure performed by housewives and menials, under
the head of household cares, may frequently develop into drudgery,
especially where the competition for reputability is close and
strenuous. This is frequently the case in modern life. Where this
happens, the domestic service which comprises the duties of this
servant class might aptly be designated as wasted effort, rather than as
vicarious leisure. But the latter term has the advantage of indicating
the line of derivation of these domestic offices, as well as of neatly
suggesting the substantial economic ground of their utility; for
these occupations are chiefly useful as a method of imputing pecuniary
reputability to the master or to the household on the ground that a
given amount of time and effort is conspicuously wasted in that behalf.

In this way, then, there arises a subsidiary or derivative leisure
class, whose office is the performance of a vicarious leisure for the
behoof of the reputability of the primary or legitimate leisure class.
This vicarious leisure class is distinguished from the leisure class
proper by a characteristic feature of its habitual mode of life. The
leisure of the master class is, at least ostensibly, an indulgence of
a proclivity for the avoidance of labour and is presumed to enhance
the master's own well-being and fulness of life; but the leisure of
the servant class exempt from productive labour is in some sort a
performance exacted from them, and is not normally or primarily directed
to their own comfort. The leisure of the servant is not his own leisure.
So far as he is a servant in the full sense, and not at the same time
a member of a lower order of the leisure class proper, his leisure
normally passes under the guise of specialised service directed to the
furtherance of his master's fulness of life. Evidence of this relation
of subservience is obviously present in the servant's carriage and
manner of life. The like is often true of the wife throughout the
protracted economic stage during which she is still primarily a
servant--that is to say, so long as the household with a male head
remains in force. In order to satisfy the requirements of the leisure
class scheme of life, the servant should show not only an attitude of
subservience, but also the effects of special training and practice
in subservience. The servant or wife should not only perform certain
offices and show a servile disposition, but it is quite as imperative
that they should show an acquired facility in the tactics of
subservience--a trained conformity to the canons of effectual and
conspicuous subservience. Even today it is this aptitude and acquired
skill in the formal manifestation of the servile relation that
constitutes the chief element of utility in our highly paid servants, as
well as one of the chief ornaments of the well-bred housewife.

The first requisite of a good servant is that he should conspicuously
know his place. It is not enough that he knows how to effect certain
desired mechanical results; he must above all, know how to effect these
results in due form. Domestic service might be said to be a spiritual
rather than a mechanical function. Gradually there grows up an elaborate
system of good form, specifically regulating the manner in which this
vicarious leisure of the servant class is to be performed. Any departure
from these canons of form is to be depreciated, not so much because it
evinces a shortcoming in mechanical efficiency, or even that it shows
an absence of the servile attitude and temperament, but because, in
the last analysis, it shows the absence of special training. Special
training in personal service costs time and effort, and where it is
obviously present in a high degree, it argues that the servant who
possesses it, neither is nor has been habitually engaged in any
productive occupation. It is \emph{prima facie} evidence of a vicarious leisure
extending far back in the past. So that trained service has utility, not
only as gratifying the master's instinctive liking for good and skilful
workmanship and his propensity for conspicuous dominance over those
whose lives are subservient to his own, but it has utility also as
putting in evidence a much larger consumption of human service than
would be shown by the mere present conspicuous leisure performed by an
untrained person. It is a serious grievance if a gentleman's butler or
footman performs his duties about his master's table or carriage in
such unformed style as to suggest that his habitual occupation may be
ploughing or sheepherding. Such bungling work would imply inability on
the master's part to procure the service of specially trained servants;
that is to say, it would imply inability to pay for the consumption
of time, effort, and instruction required to fit a trained servant for
special service under the exacting code of forms. If the performance of
the servant argues lack of means on the part of his master, it defeats
its chief substantial end; for the chief use of servants is the evidence
they afford of the master's ability to pay.

What has just been said might be taken to imply that the offence of an
under-trained servant lies in a direct suggestion of inexpensiveness or
of usefulness. Such, of course, is not the case. The connection is much
less immediate. What happens here is what happens generally. Whatever
approves itself to us on any ground at the outset, presently comes to
appeal to us as a gratifying thing in itself; it comes to rest in our
habits of though as substantially right. But in order that any specific
canon of deportment shall maintain itself in favour, it must continue to
have the support of, or at least not be incompatible with, the habit
or aptitude which constitutes the norm of its development. The need of
vicarious leisure, or conspicuous consumption of service, is a dominant
incentive to the keeping of servants. So long as this remains true it
may be set down without much discussion that any such departure from
accepted usage as would suggest an abridged apprenticeship in service
would presently be found insufferable. The requirement of an expensive
vicarious leisure acts indirectly, selectively, by guiding the formation
of our taste,--of our sense of what is right in these matters,--and so
weeds out unconformable departures by withholding approval of them.

As the standard of wealth recognized by common consent advances,
the possession and exploitation of servants as a means of showing
superfluity undergoes a refinement. The possession and maintenance of
slaves employed in the production of goods argues wealth and prowess,
but the maintenance of servants who produce nothing argues still higher
wealth and position. Under this principle there arises a class of
servants, the more numerous the better, whose sole office is fatuously
to wait upon the person of their owner, and so to put in evidence his
ability unproductively to consume a large amount of service. There
supervenes a division of labour among the servants or dependents whose
life is spent in maintaining the honour of the gentleman of leisure.
So that, while one group produces goods for him, another group, usually
headed by the wife, or chief, consumes for him in conspicuous leisure;
thereby putting in evidence his ability to sustain large pecuniary
damage without impairing his superior opulence.

This somewhat idealized and diagrammatic outline of the development and
nature of domestic service comes nearest being true for that cultural
stage which was here been named the "quasi-peaceable" stage of industry.
At this stage personal service first rises to the position of an
economic institution, and it is at this stage that it occupies the
largest place in the community's scheme of life. In the cultural
sequence, the quasi-peaceable stage follows the predatory stage proper,
the two being successive phases of barbarian life. Its characteristic
feature is a formal observance of peace and order, at the same time that
life at this stage still has too much of coercion and class antagonism
to be called peaceable in the full sense of the word. For many purposes,
and from another point of view than the economic one, it might as well
be named the stage of status. The method of human relation during this
stage, and the spiritual attitude of men at this level of culture, is
well summed up under the term. But as a descriptive term to characterise
the prevailing methods of industry, as well as to indicate the trend
of industrial development at this point in economic evolution, the term
"quasi-peaceable" seems preferable. So far as concerns the communities
of the Western culture, this phase of economic development probably
lies in the past; except for a numerically small though very conspicuous
fraction of the community in whom the habits of thought peculiar to the
barbarian culture have suffered but a relatively slight disintegration.

Personal service is still an element of great economic importance,
especially as regards the distribution and consumption of goods; but its
relative importance even in this direction is no doubt less than it once
was. The best development of this vicarious leisure lies in the past
rather than in the present; and its best expression in the present is to
be found in the scheme of life of the upper leisure class. To this
class the modern culture owes much in the way of the conservation of
traditions, usages, and habits of thought which belong on a more archaic
cultural plane, so far as regards their widest acceptance and their most
effective development.

In the modern industrial communities the mechanical contrivances
available for the comfort and convenience of everyday life are highly
developed. So much so that body servants, or, indeed, domestic servants
of any kind, would now scarcely be employed by anybody except on the
ground of a canon of reputability carried over by tradition from earlier
usage. The only exception would be servants employed to attend on the
persons of the infirm and the feeble-minded. But such servants properly
come under the head of trained nurses rather than under that of domestic
servants, and they are, therefore, an apparent rather than a real
exception to the rule.

The proximate reason for keeping domestic servants, for instance, in
the moderately well-to-do household of to-day, is (ostensibly) that the
members of the household are unable without discomfort to compass the
work required by such a modern establishment. And the reason for their
being unable to accomplish it is (1) that they have too many "social
duties", and (2) that the work to be done is too severe and that there
is too much of it. These two reasons may be restated as follows: (1)
Under the mandatory code of decency, the time and effort of the members
of such a household are required to be ostensibly all spent in a
performance of conspicuous leisure, in the way of calls, drives, clubs,
sewing-circles, sports, charity organisations, and other like social
functions. Those persons whose time and energy are employed in these
matters privately avow that all these observances, as well as the
incidental attention to dress and other conspicuous consumption, are
very irksome but altogether unavoidable. (2) Under the requirement of
conspicuous consumption of goods, the apparatus of living has grown so
elaborate and cumbrous, in the way of dwellings, furniture, bric-a-brac,
wardrobe and meals, that the consumers of these things cannot make way
with them in the required manner without help. Personal contact with the
hired persons whose aid is called in to fulfil the routine of decency is
commonly distasteful to the occupants of the house, but their presence
is endured and paid for, in order to delegate to them a share in
this onerous consumption of household goods. The presence of domestic
servants, and of the special class of body servants in an eminent
degree, is a concession of physical comfort to the moral need of
pecuniary decency.

The largest manifestation of vicarious leisure in modern life is made
up of what are called domestic duties. These duties are fast becoming a
species of services performed, not so much for the individual behoof of
the head of the household as for the reputability of the household taken
as a corporate unit--a group of which the housewife is a member on a
footing of ostensible equality. As fast as the household for which they
are performed departs from its archaic basis of ownership-marriage,
these household duties of course tend to fall out of the category of
vicarious leisure in the original sense; except so far as they are
performed by hired servants. That is to say, since vicarious leisure
is possible only on a basis of status or of hired service, the
disappearance of the relation of status from human intercourse at any
point carries with it the disappearance of vicarious leisure so far as
regards that much of life. But it is to be added, in qualification of
this qualification, that so long as the household subsists, even with a
divided head, this class of non-productive labour performed for the
sake of the household reputability must still be classed as vicarious
leisure, although in a slightly altered sense. It is now leisure
performed for the quasi-personal corporate household, instead of, as
formerly, for the proprietary head of the household.




\chapter{Conspicuous Consumption}

In what has been said of the evolution of the vicarious leisure class
and its differentiation from the general body of the working classes,
reference has been made to a further division of labour,--that between
the different servant classes. One portion of the servant class, chiefly
those persons whose occupation is vicarious leisure, come to undertake a
new, subsidiary range of duties--the vicarious consumption of goods.
The most obvious form in which this consumption occurs is seen in the
wearing of liveries and the occupation of spacious servants' quarters.
Another, scarcely less obtrusive or less effective form of vicarious
consumption, and a much more widely prevalent one, is the consumption of
food, clothing, dwelling, and furniture by the lady and the rest of the
domestic establishment.

But already at a point in economic evolution far antedating the
emergence of the lady, specialised consumption of goods as an evidence
of pecuniary strength had begun to work out in a more or less elaborate
system. The beginning of a differentiation in consumption even antedates
the appearance of anything that can fairly be called pecuniary strength.
It is traceable back to the initial phase of predatory culture, and
there is even a suggestion that an incipient differentiation in this
respect lies back of the beginnings of the predatory life. This most
primitive differentiation in the consumption of goods is like the later
differentiation with which we are all so intimately familiar, in that it
is largely of a ceremonial character, but unlike the latter it does not
rest on a difference in accumulated wealth. The utility of consumption
as an evidence of wealth is to be classed as a derivative growth. It
is an adaption to a new end, by a selective process, of a distinction
previously existing and well established in men's habits of thought.

In the earlier phases of the predatory culture the only economic
differentiation is a broad distinction between an honourable superior
class made up of the able-bodied men on the one side, and a base
inferior class of labouring women on the other. According to the ideal
scheme of life in force at the time it is the office of the men to
consume what the women produce. Such consumption as falls to the women
is merely incidental to their work; it is a means to their continued
labour, and not a consumption directed to their own comfort and fulness
of life. Unproductive consumption of goods is honourable, primarily as
a mark of prowess and a perquisite of human dignity; secondarily it
becomes substantially honourable to itself, especially the consumption
of the more desirable things. The consumption of choice articles of
food, and frequently also of rare articles of adornment, becomes tabu to
the women and children; and if there is a base (servile) class of men,
the tabu holds also for them. With a further advance in culture this
tabu may change into simple custom of a more or less rigorous character;
but whatever be the theoretical basis of the distinction which is
maintained, whether it be a tabu or a larger conventionality, the
features of the conventional scheme of consumption do not change
easily. When the quasi-peaceable stage of industry is reached, with its
fundamental institution of chattel slavery, the general principle, more
or less rigorously applied, is that the base, industrious class should
consume only what may be necessary to their subsistence. In the nature
of things, luxuries and the comforts of life belong to the leisure
class. Under the tabu, certain victuals, and more particularly certain
beverages, are strictly reserved for the use of the superior class.

The ceremonial differentiation of the dietary is best seen in the use of
intoxicating beverages and narcotics. If these articles of consumption
are costly, they are felt to be noble and honorific. Therefore the
base classes, primarily the women, practice an enforced continence
with respect to these stimulants, except in countries where they are
obtainable at a very low cost. From archaic times down through all the
length of the patriarchal regime it has been the office of the women to
prepare and administer these luxuries, and it has been the perquisite
of the men of gentle birth and breeding to consume them. Drunkenness
and the other pathological consequences of the free use of stimulants
therefore tend in their turn to become honorific, as being a mark,
at the second remove, of the superior status of those who are able to
afford the indulgence. Infirmities induced by over-indulgence are among
some peoples freely recognised as manly attributes. It has even happened
that the name for certain diseased conditions of the body arising from
such an origin has passed into everyday speech as a synonym for "noble"
or "gentle". It is only at a relatively early stage of culture that the
symptoms of expensive vice are conventionally accepted as marks of a
superior status, and so tend to become virtues and command the deference
of the community; but the reputability that attaches to certain
expensive vices long retains so much of its force as to appreciably
lesson the disapprobation visited upon the men of the wealthy or noble
class for any excessive indulgence. The same invidious distinction adds
force to the current disapproval of any indulgence of this kind on
the part of women, minors, and inferiors. This invidious traditional
distinction has not lost its force even among the more advanced peoples
of today. Where the example set by the leisure class retains its
imperative force in the regulation of the conventionalities, it is
observable that the women still in great measure practise the same
traditional continence with regard to stimulants.

This characterisation of the greater continence in the use of stimulants
practised by the women of the reputable classes may seem an excessive
refinement of logic at the expense of common sense. But facts within
easy reach of any one who cares to know them go to say that the
greater abstinence of women is in some part due to an imperative
conventionality; and this conventionality is, in a general way,
strongest where the patriarchal tradition--the tradition that the woman
is a chattel--has retained its hold in greatest vigour. In a sense which
has been greatly qualified in scope and rigour, but which has by no
means lost its meaning even yet, this tradition says that the
woman, being a chattel, should consume only what is necessary to her
sustenance,--except so far as her further consumption contributes to the
comfort or the good repute of her master. The consumption of luxuries,
in the true sense, is a consumption directed to the comfort of the
consumer himself, and is, therefore, a mark of the master. Any such
consumption by others can take place only on a basis of sufferance. In
communities where the popular habits of thought have been profoundly
shaped by the patriarchal tradition we may accordingly look for
survivals of the tabu on luxuries at least to the extent of a
conventional deprecation of their use by the unfree and dependent class.
This is more particularly true as regards certain luxuries, the use of
which by the dependent class would detract sensibly from the comfort
or pleasure of their masters, or which are held to be of doubtful
legitimacy on other grounds. In the apprehension of the great
conservative middle class of Western civilisation the use of these
various stimulants is obnoxious to at least one, if not both, of these
objections; and it is a fact too significant to be passed over that it
is precisely among these middle classes of the Germanic culture, with
their strong surviving sense of the patriarchal proprieties, that
the women are to the greatest extent subject to a qualified tabu on
narcotics and alcoholic beverages. With many qualifications--with more
qualifications as the patriarchal tradition has gradually weakened--the
general rule is felt to be right and binding that women should consume
only for the benefit of their masters. The objection of course presents
itself that expenditure on women's dress and household paraphernalia is
an obvious exception to this rule; but it will appear in the sequel that
this exception is much more obvious than substantial. During the earlier
stages of economic development, consumption of goods without stint,
especially consumption of the better grades of goods,--ideally all
consumption in excess of the subsistence minimum,--pertains normally
to the leisure class. This restriction tends to disappear, at least
formally, after the later peaceable stage has been reached, with private
ownership of goods and an industrial system based on wage labour or
on the petty household economy. But during the earlier quasi-peaceable
stage, when so many of the traditions through which the institution of a
leisure class has affected the economic life of later times were taking
form and consistency, this principle has had the force of a conventional
law. It has served as the norm to which consumption has tended to
conform, and any appreciable departure from it is to be regarded as
an aberrant form, sure to be eliminated sooner or later in the further
course of development.

The quasi-peaceable gentleman of leisure, then, not only consumes of the
staff of life beyond the minimum required for subsistence and physical
efficiency, but his consumption also undergoes a specialisation as
regards the quality of the goods consumed. He consumes freely and of the
best, in food, drink, narcotics, shelter, services, ornaments, apparel,
weapons and accoutrements, amusements, amulets, and idols or divinities.
In the process of gradual amelioration which takes place in the articles
of his consumption, the motive principle and proximate aim of innovation
is no doubt the higher efficiency of the improved and more elaborate
products for personal comfort and well-being. But that does not remain
the sole purpose of their consumption. The canon of reputability is at
hand and seizes upon such innovations as are, according to its standard,
fit to survive. Since the consumption of these more excellent goods is
an evidence of wealth, it becomes honorific; and conversely, the failure
to consume in due quantity and quality becomes a mark of inferiority and
demerit.

This growth of punctilious discrimination as to qualitative excellence
in eating, drinking, etc. presently affects not only the manner of life,
but also the training and intellectual activity of the gentleman of
leisure. He is no longer simply the successful, aggressive male,--the
man of strength, resource, and intrepidity. In order to avoid
stultification he must also cultivate his tastes, for it now becomes
incumbent on him to discriminate with some nicety between the noble and
the ignoble in consumable goods. He becomes a connoisseur in creditable
viands of various degrees of merit, in manly beverages and trinkets,
in seemly apparel and architecture, in weapons, games, dancers, and
the narcotics. This cultivation of aesthetic faculty requires time and
application, and the demands made upon the gentleman in this direction
therefore tend to change his life of leisure into a more or less arduous
application to the business of learning how to live a life of ostensible
leisure in a becoming way. Closely related to the requirement that the
gentleman must consume freely and of the right kind of goods, there
is the requirement that he must know how to consume them in a seemly
manner. His life of leisure must be conducted in due form. Hence arise
good manners in the way pointed out in an earlier chapter. High-bred
manners and ways of living are items of conformity to the norm of
conspicuous leisure and conspicuous consumption.

Conspicuous consumption of valuable goods is a means of reputability to
the gentleman of leisure. As wealth accumulates on his hands, his
own unaided effort will not avail to sufficiently put his opulence in
evidence by this method. The aid of friends and competitors is therefore
brought in by resorting to the giving of valuable presents and expensive
feasts and entertainments. Presents and feasts had probably another
origin than that of naive ostentation, but they required their utility
for this purpose very early, and they have retained that character to
the present; so that their utility in this respect has now long been the
substantial ground on which these usages rest. Costly entertainments,
such as the potlatch or the ball, are peculiarly adapted to serve this
end. The competitor with whom the entertainer wishes to institute a
comparison is, by this method, made to serve as a means to the end. He
consumes vicariously for his host at the same time that he is witness to
the consumption of that excess of good things which his host is unable
to dispose of single-handed, and he is also made to witness his host's
facility in etiquette.

In the giving of costly entertainments other motives, of more genial
kind, are of course also present. The custom of festive gatherings
probably originated in motives of conviviality and religion; these
motives are also present in the later development, but they do
not continue to be the sole motives. The latter-day leisure-class
festivities and entertainments may continue in some slight degree to
serve the religious need and in a higher degree the needs of recreation
and conviviality, but they also serve an invidious purpose; and they
serve it none the less effectually for having a colorable non-invidious
ground in these more avowable motives. But the economic effect of these
social amenities is not therefore lessened, either in the vicarious
consumption of goods or in the exhibition of difficult and costly
achievements in etiquette.

As wealth accumulates, the leisure class develops further in function
and structure, and there arises a differentiation within the class.
There is a more or less elaborate system of rank and grades. This
differentiation is furthered by the inheritance of wealth and the
consequent inheritance of gentility. With the inheritance of gentility
goes the inheritance of obligatory leisure; and gentility of a
sufficient potency to entail a life of leisure may be inherited without
the complement of wealth required to maintain a dignified leisure.
Gentle blood may be transmitted without goods enough to afford a
reputably free consumption at one's ease. Hence results a class of
impecunious gentlemen of leisure, incidentally referred to already.
These half-caste gentlemen of leisure fall into a system of hierarchical
gradations. Those who stand near the higher and the highest grades of
the wealthy leisure class, in point of birth, or in point of wealth, or
both, outrank the remoter-born and the pecuniarily weaker. These lower
grades, especially the impecunious, or marginal, gentlemen of leisure,
affiliate themselves by a system of dependence or fealty to the great
ones; by so doing they gain an increment of repute, or of the means
with which to lead a life of leisure, from their patron. They become
his courtiers or retainers, servants; and being fed and countenanced by
their patron they are indices of his rank and vicarious consumer of his
superfluous wealth. Many of these affiliated gentlemen of leisure are at
the same time lesser men of substance in their own right; so that some
of them are scarcely at all, others only partially, to be rated as
vicarious consumers. So many of them, however, as make up the retainer
and hangers-on of the patron may be classed as vicarious consumer
without qualification. Many of these again, and also many of the other
aristocracy of less degree, have in turn attached to their persons a
more or less comprehensive group of vicarious consumer in the persons of
their wives and children, their servants, retainers, etc.

Throughout this graduated scheme of vicarious leisure and vicarious
consumption the rule holds that these offices must be performed in some
such manner, or under some such circumstance or insignia, as shall point
plainly to the master to whom this leisure or consumption pertains,
and to whom therefore the resulting increment of good repute of right
inures. The consumption and leisure executed by these persons for their
master or patron represents an investment on his part with a view to an
increase of good fame. As regards feasts and largesses this is obvious
enough, and the imputation of repute to the host or patron here takes
place immediately, on the ground of common notoriety. Where leisure
and consumption is performed vicariously by henchmen and retainers,
imputation of the resulting repute to the patron is effected by their
residing near his person so that it may be plain to all men from what
source they draw. As the group whose good esteem is to be secured in
this way grows larger, more patent means are required to indicate the
imputation of merit for the leisure performed, and to this end uniforms,
badges, and liveries come into vogue. The wearing of uniforms or
liveries implies a considerable degree of dependence, and may even
be said to be a mark of servitude, real or ostensible. The wearers of
uniforms and liveries may be roughly divided into two classes-the free
and the servile, or the noble and the ignoble. The services performed
by them are likewise divisible into noble and ignoble. Of course the
distinction is not observed with strict consistency in practice; the
less debasing of the base services and the less honorific of the noble
functions are not infrequently merged in the same person. But the
general distinction is not on that account to be overlooked. What
may add some perplexity is the fact that this fundamental distinction
between noble and ignoble, which rests on the nature of the ostensible
service performed, is traversed by a secondary distinction into
honorific and humiliating, resting on the rank of the person for whom
the service is performed or whose livery is worn. So, those offices
which are by right the proper employment of the leisure class are
noble; such as government, fighting, hunting, the care of arms and
accoutrements, and the like--in short, those which may be classed as
ostensibly predatory employments. On the other hand, those employments
which properly fall to the industrious class are ignoble; such as
handicraft or other productive labor, menial services and the like. But
a base service performed for a person of very high degree may become a
very honorific office; as for instance the office of a Maid of Honor or
of a Lady in Waiting to the Queen, or the King's Master of the Horse or
his Keeper of the Hounds. The two offices last named suggest a principle
of some general bearing. Whenever, as in these cases, the menial service
in question has to do directly with the primary leisure employments
of fighting and hunting, it easily acquires a reflected honorific
character. In this way great honor may come to attach to an employment
which in its own nature belongs to the baser sort. In the later
development of peaceable industry, the usage of employing an idle corps
of uniformed men-at-arms gradually lapses. Vicarious consumption by
dependents bearing the insignia of their patron or master narrows down
to a corps of liveried menials. In a heightened degree, therefore, the
livery comes to be a badge of servitude, or rather servility. Something
of a honorific character always attached to the livery of the armed
retainer, but this honorific character disappears when the livery
becomes the exclusive badge of the menial. The livery becomes obnoxious
to nearly all who are required to wear it. We are yet so little removed
from a state of effective slavery as still to be fully sensitive to the
sting of any imputation of servility. This antipathy asserts itself
even in the case of the liveries or uniforms which some corporations
prescribe as the distinctive dress of their employees. In this country
the aversion even goes the length of discrediting--in a mild and
uncertain way--those government employments, military and civil, which
require the wearing of a livery or uniform.

With the disappearance of servitude, the number of vicarious consumers
attached to any one gentleman tends, on the whole, to decrease. The like
is of course true, and perhaps in a still higher degree, of the number
of dependents who perform vicarious leisure for him. In a general way,
though not wholly nor consistently, these two groups coincide. The
dependent who was first delegated for these duties was the wife, or the
chief wife; and, as would be expected, in the later development of
the institution, when the number of persons by whom these duties are
customarily performed gradually narrows, the wife remains the last.
In the higher grades of society a large volume of both these kinds of
service is required; and here the wife is of course still assisted in
the work by a more or less numerous corps of menials. But as we descend
the social scale, the point is presently reached where the duties of
vicarious leisure and consumption devolve upon the wife alone. In the
communities of the Western culture, this point is at present found among
the lower middle class.

And here occurs a curious inversion. It is a fact of common observance
that in this lower middle class there is no pretense of leisure on the
part of the head of the household. Through force of circumstances it
has fallen into disuse. But the middle-class wife still carries on the
business of vicarious leisure, for the good name of the household and
its master. In descending the social scale in any modern industrial
community, the primary fact-the conspicuous leisure of the master of
the household-disappears at a relatively high point. The head of the
middle-class household has been reduced by economic circumstances to
turn his hand to gaining a livelihood by occupations which often partake
largely of the character of industry, as in the case of the ordinary
business man of today. But the derivative fact-the vicarious leisure
and consumption rendered by the wife, and the auxiliary vicarious
performance of leisure by menials-remains in vogue as a conventionality
which the demands of reputability will not suffer to be slighted. It is
by no means an uncommon spectacle to find a man applying himself to work
with the utmost assiduity, in order that his wife may in due form render
for him that degree of vicarious leisure which the common sense of the
time demands.

The leisure rendered by the wife in such cases is, of course, not a
simple manifestation of idleness or indolence. It almost invariably
occurs disguised under some form of work or household duties or social
amenities, which prove on analysis to serve little or no ulterior end
beyond showing that she does not occupy herself with anything that is
gainful or that is of substantial use. As has already been noticed under
the head of manners, the greater part of the customary round of domestic
cares to which the middle-class housewife gives her time and effort is
of this character. Not that the results of her attention to household
matters, of a decorative and mundificatory character, are not pleasing
to the sense of men trained in middle-class proprieties; but the taste
to which these effects of household adornment and tidiness appeal is a
taste which has been formed under the selective guidance of a canon
of propriety that demands just these evidences of wasted effort. The
effects are pleasing to us chiefly because we have been taught to find
them pleasing. There goes into these domestic duties much solicitude for
a proper combination of form and color, and for other ends that are to
be classed as aesthetic in the proper sense of the term; and it is
not denied that effects having some substantial aesthetic value are
sometimes attained. Pretty much all that is here insisted on is that, as
regards these amenities of life, the housewife's efforts are under the
guidance of traditions that have been shaped by the law of conspicuously
wasteful expenditure of time and substance. If beauty or comfort is
achieved-and it is a more or less fortuitous circumstance if they
are-they must be achieved by means and methods that commend themselves
to the great economic law of wasted effort. The more reputable,
"presentable" portion of middle-class household paraphernalia are, on
the one hand, items of conspicuous consumption, and on the other hand,
apparatus for putting in evidence the vicarious leisure rendered by the
housewife.

The requirement of vicarious consumption at the hands of the wife
continues in force even at a lower point in the pecuniary scale than the
requirement of vicarious leisure. At a point below which little if any
pretense of wasted effort, in ceremonial cleanness and the like,
is observable, and where there is assuredly no conscious attempt at
ostensible leisure, decency still requires the wife to consume some
goods conspicuously for the reputability of the household and its head.
So that, as the latter-day outcome of this evolution of an archaic
institution, the wife, who was at the outset the drudge and chattel of
the man, both in fact and in theory--the producer of goods for him to
consume--has become the ceremonial consumer of goods which he produces.
But she still quite unmistakably remains his chattel in theory; for the
habitual rendering of vicarious leisure and consumption is the abiding
mark of the unfree servant.

This vicarious consumption practiced by the household of the middle
and lower classes can not be counted as a direct expression of the
leisure-class scheme of life, since the household of this pecuniary
grade does not belong within the leisure class. It is rather that the
leisure-class scheme of life here comes to an expression at the second
remove. The leisure class stands at the head of the social structure in
point of reputability; and its manner of life and its standards of
worth therefore afford the norm of reputability for the community. The
observance of these standards, in some degree of approximation, becomes
incumbent upon all classes lower in the scale. In modern civilized
communities the lines of demarcation between social classes have grown
vague and transient, and wherever this happens the norm of reputability
imposed by the upper class extends its coercive influence with but
slight hindrance down through the social structure to the lowest strata.
The result is that the members of each stratum accept as their ideal of
decency the scheme of life in vogue in the next higher stratum, and bend
their energies to live up to that ideal. On pain of forfeiting their
good name and their self-respect in case of failure, they must conform
to the accepted code, at least in appearance. The basis on which good
repute in any highly organized industrial community ultimately rests is
pecuniary strength; and the means of showing pecuniary strength, and
so of gaining or retaining a good name, are leisure and a conspicuous
consumption of goods. Accordingly, both of these methods are in vogue
as far down the scale as it remains possible; and in the lower strata
in which the two methods are employed, both offices are in great part
delegated to the wife and children of the household. Lower still, where
any degree of leisure, even ostensible, has become impracticable for the
wife, the conspicuous consumption of goods remains and is carried on by
the wife and children. The man of the household also can do something
in this direction, and indeed, he commonly does; but with a still lower
descent into the levels of indigence--along the margin of the slums--the
man, and presently also the children, virtually cease to consume
valuable goods for appearances, and the woman remains virtually the sole
exponent of the household's pecuniary decency. No class of society,
not even the most abjectly poor, forgoes all customary conspicuous
consumption. The last items of this category of consumption are not
given up except under stress of the direst necessity. Very much of
squalor and discomfort will be endured before the last trinket or the
last pretense of pecuniary decency is put away. There is no class and
no country that has yielded so abjectly before the pressure of physical
want as to deny themselves all gratification of this higher or spiritual
need.

From the foregoing survey of the growth of conspicuous leisure and
consumption, it appears that the utility of both alike for the purposes
of reputability lies in the element of waste that is common to both.
In the one case it is a waste of time and effort, in the other it is
a waste of goods. Both are methods of demonstrating the possession of
wealth, and the two are conventionally accepted as equivalents. The
choice between them is a question of advertising expediency simply,
except so far as it may be affected by other standards of propriety,
springing from a different source. On grounds of expediency the
preference may be given to the one or the other at different stages of
the economic development. The question is, which of the two methods will
most effectively reach the persons whose convictions it is desired
to affect. Usage has answered this question in different ways under
different circumstances.

So long as the community or social group is small enough and compact
enough to be effectually reached by common notoriety alone that is
to say, so long as the human environment to which the individual is
required to adapt himself in respect of reputability is comprised within
his sphere of personal acquaintance and neighborhood gossip--so long the
one method is about as effective as the other. Each will therefore serve
about equally well during the earlier stages of social growth. But when
the differentiation has gone farther and it becomes necessary to reach
a wider human environment, consumption begins to hold over leisure as
an ordinary means of decency. This is especially true during the later,
peaceable economic stage. The means of communication and the mobility
of the population now expose the individual to the observation of many
persons who have no other means of judging of his reputability than
the display of goods (and perhaps of breeding) which he is able to make
while he is under their direct observation.

The modern organization of industry works in the same direction also by
another line. The exigencies of the modern industrial system frequently
place individuals and households in juxtaposition between whom there
is little contact in any other sense than that of juxtaposition.
One's neighbors, mechanically speaking, often are socially not one's
neighbors, or even acquaintances; and still their transient good opinion
has a high degree of utility. The only practicable means of impressing
one's pecuniary ability on these unsympathetic observers of one's
everyday life is an unremitting demonstration of ability to pay. In
the modern community there is also a more frequent attendance at large
gatherings of people to whom one's everyday life is unknown; in such
places as churches, theaters, ballrooms, hotels, parks, shops, and the
like. In order to impress these transient observers, and to retain
one's self-complacency under their observation, the signature of one's
pecuniary strength should be written in characters which he who runs
may read. It is evident, therefore, that the present trend of
the development is in the direction of heightening the utility of
conspicuous consumption as compared with leisure.

It is also noticeable that the serviceability of consumption as a means
of repute, as well as the insistence on it as an element of decency, is
at its best in those portions of the community where the human contact
of the individual is widest and the mobility of the population is
greatest. Conspicuous consumption claims a relatively larger portion of
the income of the urban than of the rural population, and the claim is
also more imperative. The result is that, in order to keep up a decent
appearance, the former habitually live hand-to-mouth to a greater extent
than the latter. So it comes, for instance, that the American farmer and
his wife and daughters are notoriously less modish in their dress, as
well as less urbane in their manners, than the city artisan's family
with an equal income. It is not that the city population is by nature
much more eager for the peculiar complacency that comes of a conspicuous
consumption, nor has the rural population less regard for pecuniary
decency. But the provocation to this line of evidence, as well as its
transient effectiveness, is more decided in the city. This method is
therefore more readily resorted to, and in the struggle to outdo one
another the city population push their normal standard of conspicuous
consumption to a higher point, with the result that a relatively greater
expenditure in this direction is required to indicate a given degree
of pecuniary decency in the city. The requirement of conformity to this
higher conventional standard becomes mandatory. The standard of decency
is higher, class for class, and this requirement of decent appearance
must be lived up to on pain of losing caste.

Consumption becomes a larger element in the standard of living in the
city than in the country. Among the country population its place is to
some extent taken by savings and home comforts known through the medium
of neighborhood gossip sufficiently to serve the like general purpose of
Pecuniary repute. These home comforts and the leisure indulged in--where
the indulgence is found--are of course also in great part to be classed
as items of conspicuous consumption; and much the same is to be said of
the savings. The smaller amount of the savings laid by by the artisan
class is no doubt due, in some measure, to the fact that in the case
of the artisan the savings are a less effective means of advertisement,
relative to the environment in which he is placed, than are the savings
of the people living on farms and in the small villages. Among the
latter, everybody's affairs, especially everybody's pecuniary status,
are known to everybody else. Considered by itself simply--taken in the
first degree--this added provocation to which the artisan and the urban
laboring classes are exposed may not very seriously decrease the amount
of savings; but in its cumulative action, through raising the standard
of decent expenditure, its deterrent effect on the tendency to save
cannot but be very great.

A felicitous illustration of the manner in which this canon of
reputability works out its results is seen in the practice of
dram-drinking, "treating," and smoking in public places, which is
customary among the laborers and handicraftsmen of the towns, and among
the lower middle class of the urban population generally Journeymen
printers may be named as a class among whom this form of conspicuous
consumption has a great vogue, and among whom it carries with it certain
well-marked consequences that are often deprecated. The peculiar habits
of the class in this respect are commonly set down to some kind of an
ill-defined moral deficiency with which this class is credited, or to
a morally deleterious influence which their occupation is supposed to
exert, in some unascertainable way, upon the men employed in it. The
state of the case for the men who work in the composition and press
rooms of the common run of printing-houses may be summed up as follows.
Skill acquired in any printing-house or any city is easily turned to
account in almost any other house or city; that is to say, the inertia
due to special training is slight. Also, this occupation requires more
than the average of intelligence and general information, and the men
employed in it are therefore ordinarily more ready than many others to
take advantage of any slight variation in the demand for their labor
from one place to another. The inertia due to the home feeling is
consequently also slight. At the same time the wages in the trade are
high enough to make movement from place to place relatively easy. The
result is a great mobility of the labor employed in printing; perhaps
greater than in any other equally well-defined and considerable body of
workmen. These men are constantly thrown in contact with new groups
of acquaintances, with whom the relations established are transient or
ephemeral, but whose good opinion is valued none the less for the time
being. The human proclivity to ostentation, reenforced by sentiments of
good-fellowship, leads them to spend freely in those directions which
will best serve these needs. Here as elsewhere prescription seizes
upon the custom as soon as it gains a vogue, and incorporates it in the
accredited standard of decency. The next step is to make this standard
of decency the point of departure for a new move in advance in the same
direction--for there is no merit in simple spiritless conformity to a
standard of dissipation that is lived up to as a matter of course by
everyone in the trade.

The greater prevalence of dissipation among printers than among the
average of workmen is accordingly attributable, at least in some
measure, to the greater ease of movement and the more transient
character of acquaintance and human contact in this trade. But the
substantial ground of this high requirement in dissipation is in the
last analysis no other than that same propensity for a manifestation
of dominance and pecuniary decency which makes the French
peasant-proprietor parsimonious and frugal, and induces the American
millionaire to found colleges, hospitals and museums. If the canon of
conspicuous consumption were not offset to a considerable extent by
other features of human nature, alien to it, any saving should logically
be impossible for a population situated as the artisan and laboring
classes of the cities are at present, however high their wages or their
income might be.

But there are other standards of repute and other, more or less
imperative, canons of conduct, besides wealth and its manifestation, and
some of these come in to accentuate or to qualify the broad, fundamental
canon of conspicuous waste. Under the simple test of effectiveness
for advertising, we should expect to find leisure and the conspicuous
consumption of goods dividing the field of pecuniary emulation pretty
evenly between them at the outset. Leisure might then be expected
gradually to yield ground and tend to obsolescence as the economic
development goes forward, and the community increases in size; while the
conspicuous consumption of goods should gradually gain in importance,
both absolutely and relatively, until it had absorbed all the available
product, leaving nothing over beyond a bare livelihood. But the actual
course of development has been somewhat different from this ideal
scheme. Leisure held the first place at the start, and came to hold a
rank very much above wasteful consumption of goods, both as a direct
exponent of wealth and as an element in the standard of decency, during
the quasi-peaceable culture. From that point onward, consumption has
gained ground, until, at present, it unquestionably holds the primacy,
though it is still far from absorbing the entire margin of production
above the subsistence minimum.

The early ascendency of leisure as a means of reputability is traceable
to the archaic distinction between noble and ignoble employments.
Leisure is honorable and becomes imperative partly because it shows
exemption from ignoble labor. The archaic differentiation into noble and
ignoble classes is based on an invidious distinction between employments
as honorific or debasing; and this traditional distinction grows into an
imperative canon of decency during the early quasi-peaceable stage.
Its ascendency is furthered by the fact that leisure is still fully as
effective an evidence of wealth as consumption. Indeed, so effective
is it in the relatively small and stable human environment to which the
individual is exposed at that cultural stage, that, with the aid of the
archaic tradition which deprecates all productive labor, it gives rise
to a large impecunious leisure class, and it even tends to limit the
production of the community's industry to the subsistence minimum. This
extreme inhibition of industry is avoided because slave labor, working
under a compulsion more vigorous than that of reputability, is forced to
turn out a product in excess of the subsistence minimum of the working
class. The subsequent relative decline in the use of conspicuous
leisure as a basis of repute is due partly to an increasing relative
effectiveness of consumption as an evidence of wealth; but in part it is
traceable to another force, alien, and in some degree antagonistic, to
the usage of conspicuous waste.

This alien factor is the instinct of workmanship. Other circumstances
permitting, that instinct disposes men to look with favor upon
productive efficiency and on whatever is of human use. It disposes them
to deprecate waste of substance or effort. The instinct of workmanship
is present in all men, and asserts itself even under very adverse
circumstances. So that however wasteful a given expenditure may be in
reality, it must at least have some colorable excuse in the way of an
ostensible purpose. The manner in which, under special circumstances,
the instinct eventuates in a taste for exploit and an invidious
discrimination between noble and ignoble classes has been indicated in
an earlier chapter. In so far as it comes into conflict with the law of
conspicuous waste, the instinct of workmanship expresses itself not so
much in insistence on substantial usefulness as in an abiding sense of
the odiousness and aesthetic impossibility of what is obviously futile.
Being of the nature of an instinctive affection, its guidance touches
chiefly and immediately the obvious and apparent violations of its
requirements. It is only less promptly and with less constraining force
that it reaches such substantial violations of its requirements as are
appreciated only upon reflection.

So long as all labor continues to be performed exclusively or usually
by slaves, the baseness of all productive effort is too constantly
and deterrently present in the mind of men to allow the instinct of
workmanship seriously to take effect in the direction of industrial
usefulness; but when the quasi-peaceable stage (with slavery and status)
passes into the peaceable stage of industry (with wage labor and cash
payment) the instinct comes more effectively into play. It then begins
aggressively to shape men's views of what is meritorious, and asserts
itself at least as an auxiliary canon of self-complacency. All
extraneous considerations apart, those persons (adult) are but a
vanishing minority today who harbor no inclination to the accomplishment
of some end, or who are not impelled of their own motion to shape some
object or fact or relation for human use. The propensity may in large
measure be overborne by the more immediately constraining incentive to a
reputable leisure and an avoidance of indecorous usefulness, and it
may therefore work itself out in make-believe only; as for instance
in "social duties," and in quasi-artistic or quasi-scholarly
accomplishments, in the care and decoration of the house, in
sewing-circle activity or dress reform, in proficiency at dress, cards,
yachting, golf, and various sports. But the fact that it may under
stress of circumstances eventuate in inanities no more disproves the
presence of the instinct than the reality of the brooding instinct is
disproved by inducing a hen to sit on a nestful of china eggs.

This latter-day uneasy reaching-out for some form of purposeful activity
that shall at the same time not be indecorously productive of either
individual or collective gain marks a difference of attitude between
the modern leisure class and that of the quasi-peaceable stage. At the
earlier stage, as was said above, the all-dominating institution
of slavery and status acted resistlessly to discountenance exertion
directed to other than naively predatory ends. It was still possible to
find some habitual employment for the inclination to action in the way
of forcible aggression or repression directed against hostile groups or
against the subject classes within the group; and this served to relieve
the pressure and draw off the energy of the leisure class without a
resort to actually useful, or even ostensibly useful employments. The
practice of hunting also served the same purpose in some degree. When the
community developed into a peaceful industrial organization, and when
fuller occupation of the land had reduced the opportunities for the hunt
to an inconsiderable residue, the pressure of energy seeking purposeful
employment was left to find an outlet in some other direction. The
ignominy which attaches to useful effort also entered upon a less acute
phase with the disappearance of compulsory labor; and the instinct
of workmanship then came to assert itself with more persistence and
consistency.

The line of least resistance has changed in some measure, and the energy
which formerly found a vent in predatory activity, now in part takes the
direction of some ostensibly useful end. Ostensibly purposeless leisure
has come to be deprecated, especially among that large portion of the
leisure class whose plebeian origin acts to set them at variance with
the tradition of the otium cum dignitate. But that canon of reputability
which discountenances all employment that is of the nature of productive
effort is still at hand, and will permit nothing beyond the most
transient vogue to any employment that is substantially useful or
productive. The consequence is that a change has been wrought in the
conspicuous leisure practiced by the leisure class; not so much in
substance as in form. A reconciliation between the two conflicting
requirements is effected by a resort to make-believe. Many and intricate
polite observances and social duties of a ceremonial nature are
developed; many organizations are founded, with some specious object of
amelioration embodied in their official style and title; there is much
coming and going, and a deal of talk, to the end that the talkers may
not have occasion to reflect on what is the effectual economic value of
their traffic. And along with the make-believe of purposeful employment,
and woven inextricably into its texture, there is commonly, if not
invariably, a more or less appreciable element of purposeful effort
directed to some serious end.

In the narrower sphere of vicarious leisure a similar change has gone
forward. Instead of simply passing her time in visible idleness, as in
the best days of the patriarchal regime, the housewife of the advanced
peaceable stage applies herself assiduously to household cares. The
salient features of this development of domestic service have already
been indicated. Throughout the entire evolution of conspicuous
expenditure, whether of goods or of services or human life, runs the
obvious implication that in order to effectually mend the consumer's
good fame it must be an expenditure of superfluities. In order to
be reputable it must be wasteful. No merit would accrue from the
consumption of the bare necessaries of life, except by comparison with
the abjectly poor who fall short even of the subsistence minimum; and no
standard of expenditure could result from such a comparison, except the
most prosaic and unattractive level of decency. A standard of life would
still be possible which should admit of invidious comparison in other
respects than that of opulence; as, for instance, a comparison
in various directions in the manifestation of moral, physical,
intellectual, or aesthetic force. Comparison in all these directions is
in vogue today; and the comparison made in these respects is commonly
so inextricably bound up with the pecuniary comparison as to be scarcely
distinguishable from the latter. This is especially true as regards the
current rating of expressions of intellectual and aesthetic force
or proficiency' so that we frequently interpret as aesthetic or
intellectual a difference which in substance is pecuniary only.

The use of the term "waste" is in one respect an unfortunate one. As
used in the speech of everyday life the word carries an undertone
of deprecation. It is here used for want of a better term that will
adequately describe the same range of motives and of phenomena, and
it is not to be taken in an odious sense, as implying an illegitimate
expenditure of human products or of human life. In the view of economic
theory the expenditure in question is no more and no less legitimate
than any other expenditure. It is here called "waste" because this
expenditure does not serve human life or human well-being on the whole,
not because it is waste or misdirection of effort or expenditure as
viewed from the standpoint of the individual consumer who chooses it. If
he chooses it, that disposes of the question of its relative utility
to him, as compared with other forms of consumption that would not
be deprecated on account of their wastefulness. Whatever form of
expenditure the consumer chooses, or whatever end he seeks in making his
choice, has utility to him by virtue of his preference. As seen from the
point of view of the individual consumer, the question of wastefulness
does not arise within the scope of economic theory proper. The use of
the word "waste" as a technical term, therefore, implies no deprecation
of the motives or of the ends sought by the consumer under this canon of
conspicuous waste.

But it is, on other grounds, worth noting that the term "waste" in the
language of everyday life implies deprecation of what is characterized
as wasteful. This common-sense implication is itself an outcropping of
the instinct of workmanship. The popular reprobation of waste goes to
say that in order to be at peace with himself the common man must
be able to see in any and all human effort and human enjoyment an
enhancement of life and well-being on the whole. In order to meet with
unqualified approval, any economic fact must approve itself under the
test of impersonal usefulness--usefulness as seen from the point of
view of the generically human. Relative or competitive advantage of
one individual in comparison with another does not satisfy the economic
conscience, and therefore competitive expenditure has not the approval
of this conscience.

In strict accuracy nothing should be included under the head of
conspicuous waste but such expenditure as is incurred on the ground of
an invidious pecuniary comparison. But in order to bring any given item
or element in under this head it is not necessary that it should
be recognized as waste in this sense by the person incurring the
expenditure. It frequently happens that an element of the standard of
living which set out with being primarily wasteful, ends with becoming,
in the apprehension of the consumer, a necessary of life; and it may
in this way become as indispensable as any other item of the consumer's
habitual expenditure. As items which sometimes fall under this head,
and are therefore available as illustrations of the manner in which this
principle applies, may be cited carpets and tapestries, silver table
service, waiter's services, silk hats, starched linen, many articles
of jewelry and of dress. The indispensability of these things after the
habit and the convention have been formed, however, has little to say
in the classification of expenditures as waste or not waste in the
technical meaning of the word. The test to which all expenditure must
be brought in an attempt to decide that point is the question whether it
serves directly to enhance human life on the whole-whether it furthers
the life process taken impersonally. For this is the basis of award of
the instinct of workmanship, and that instinct is the court of final
appeal in any question of economic truth or adequacy. It is a question
as to the award rendered by a dispassionate common sense. The question
is, therefore, not whether, under the existing circumstances of
individual habit and social custom, a given expenditure conduces to the
particular consumer's gratification or peace of mind; but whether,
aside from acquired tastes and from the canons of usage and conventional
decency, its result is a net gain in comfort or in the fullness of life.
Customary expenditure must be classed under the head of waste in so far
as the custom on which it rests is traceable to the habit of making
an invidious pecuniary comparison-in so far as it is conceived that it
could not have become customary and prescriptive without the backing of
this principle of pecuniary reputability or relative economic success.
It is obviously not necessary that a given object of expenditure should
be exclusively wasteful in order to come in under the category of
conspicuous waste. An article may be useful and wasteful both, and its
utility to the consumer may be made up of use and waste in the most
varying proportions. Consumable goods, and even productive goods,
generally show the two elements in combination, as constituents of
their utility; although, in a general way, the element of waste tends
to predominate in articles of consumption, while the contrary is true of
articles designed for productive use. Even in articles which appear at
first glance to serve for pure ostentation only, it is always possible
to detect the presence of some, at least ostensible, useful purpose;
and on the other hand, even in special machinery and tools contrived for
some particular industrial process, as well as in the rudest appliances
of human industry, the traces of conspicuous waste, or at least of the
habit of ostentation, usually become evident on a close scrutiny. It
would be hazardous to assert that a useful purpose is ever absent from
the utility of any article or of any service, however obviously its
prime purpose and chief element is conspicuous waste; and it would be
only less hazardous to assert of any primarily useful product that the
element of waste is in no way concerned in its value, immediately or
remotely.




\chapter{The Pecuniary Standard of Living}
For the great body of the people in any modern community, the proximate
ground of expenditure in excess of what is required for physical comfort
is not a conscious effort to excel in the expensiveness of their visible
consumption, so much as it is a desire to live up to the conventional
standard of decency in the amount and grade of goods consumed. This
desire is not guided by a rigidly invariable standard, which must be
lived up to, and beyond which there is no incentive to go. The standard
is flexible; and especially it is indefinitely extensible, if only time
is allowed for habituation to any increase in pecuniary ability and
for acquiring facility in the new and larger scale of expenditure that
follows such an increase. It is much more difficult to recede from a
scale of expenditure once adopted than it is to extend the accustomed
scale in response to an accession of wealth. Many items of customary
expenditure prove on analysis to be almost purely wasteful, and they
are therefore honorific only, but after they have once been incorporated
into the scale of decent consumption, and so have become an integral
part of one's scheme of life, it is quite as hard to give up these as
it is to give up many items that conduce directly to one's physical
comfort, or even that may be necessary to life and health. That is
to say, the conspicuously wasteful honorific expenditure that confers
spiritual well-being may become more indispensable than much of that
expenditure which ministers to the "lower" wants of physical well-being
or sustenance only. It is notoriously just as difficult to recede from a
"high" standard of living as it is to lower a standard which is already
relatively low; although in the former case the difficulty is a moral
one, while in the latter it may involve a material deduction from the
physical comforts of life.

But while retrogression is difficult, a fresh advance in conspicuous
expenditure is relatively easy; indeed, it takes place almost as a
matter of course. In the rare cases where it occurs, a failure to
increase one's visible consumption when the means for an increase are
at hand is felt in popular apprehension to call for explanation, and
unworthy motives of miserliness are imputed to those who fall short in
this respect. A prompt response to the stimulus, on the other hand,
is accepted as the normal effect. This suggests that the standard
of expenditure which commonly guides our efforts is not the average,
ordinary expenditure already achieved; it is an ideal of consumption
that lies just beyond our reach, or to reach which requires some strain.
The motive is emulation--the stimulus of an invidious comparison which
prompts us to outdo those with whom we are in the habit of classing
ourselves. Substantially the same proposition is expressed in the
commonplace remark that each class envies and emulates the class next
above it in the social scale, while it rarely compares itself with those
below or with those who are considerably in advance. That is to say, in
other words, our standard of decency in expenditure, as in other ends of
emulation, is set by the usage of those next above us in reputability;
until, in this way, especially in any community where class distinctions
are somewhat vague, all canons of reputability and decency, and all
standards of consumption, are traced back by insensible gradations to
the usages and habits of thought of the highest social and pecuniary
class--the wealthy leisure class.

It is for this class to determine, in general outline, what scheme of
Life the community shall accept as decent or honorific; and it is
their office by precept and example to set forth this scheme of social
salvation in its highest, ideal form. But the higher leisure class
can exercise this quasi-sacerdotal office only under certain material
limitations. The class cannot at discretion effect a sudden revolution
or reversal of the popular habits of thought with respect to any of
these ceremonial requirements. It takes time for any change to permeate
the mass and change the habitual attitude of the people; and especially
it takes time to change the habits of those classes that are socially
more remote from the radiant body. The process is slower where the
mobility of the population is less or where the intervals between the
several classes are wider and more abrupt. But if time be allowed, the
scope of the discretion of the leisure class as regards questions of
form and detail in the community's scheme of life is large; while as
regards the substantial principles of reputability, the changes which
it can effect lie within a narrow margin of tolerance. Its example and
precept carries the force of prescription for all classes below it; but
in working out the precepts which are handed down as governing the form
and method of reputability--in shaping the usages and the spiritual
attitude of the lower classes--this authoritative prescription
constantly works under the selective guidance of the canon of
conspicuous waste, tempered in varying degree by the instinct of
workmanship. To those norms is to be added another broad principle of
human nature--the predatory animus--which in point of generality and of
psychological content lies between the two just named. The effect of the
latter in shaping the accepted scheme of life is yet to be discussed.
The canon of reputability, then, must adapt itself to the economic
circumstances, the traditions, and the degree of spiritual maturity
of the particular class whose scheme of life it is to regulate. It is
especially to be noted that however high its authority and however true
to the fundamental requirements of reputability it may have been at
its inception, a specific formal observance can under no circumstances
maintain itself in force if with the lapse of time or on its
transmission to a lower pecuniary class it is found to run counter
to the ultimate ground of decency among civilized peoples, namely,
serviceability for the purpose of an invidious comparison in pecuniary
success. It is evident that these canons of expenditure have much to
say in determining the standard of living for any community and for any
class. It is no less evident that the standard of living which prevails
at any time or at any given social altitude will in its turn have much
to say as to the forms which honorific expenditure will take, and as
to the degree to which this "higher" need will dominate a people's
consumption. In this respect the control exerted by the accepted
standard of living is chiefly of a negative character; it acts almost
solely to prevent recession from a scale of conspicuous expenditure that
has once become habitual.

A standard of living is of the nature of habit. It is an habitual scale
and method of responding to given stimuli. The difficulty in the way
of receding from an accustomed standard is the difficulty of breaking
a habit that has once been formed. The relative facility with which an
advance in the standard is made means that the life process is a process
of unfolding activity and that it will readily unfold in a new direction
whenever and wherever the resistance to self-expression decreases. But
when the habit of expression along such a given line of low resistance
has once been formed, the discharge will seek the accustomed outlet even
after a change has taken place in the environment whereby the external
resistance has appreciably risen. That heightened facility of expression
in a given direction which is called habit may offset a considerable
increase in the resistance offered by external circumstances to the
unfolding of life in the given direction. As between the various habits,
or habitual modes and directions of expression, which go to make up an
individual's standard of living, there is an appreciable difference in
point of persistence under counteracting circumstances and in point
of the degree of imperativeness with which the discharge seeks a given
direction.

That is to say, in the language of current economic theory, while men
are reluctant to retrench their expenditures in any direction, they are
more reluctant to retrench in some directions than in others; so that
while any accustomed consumption is reluctantly given up, there are
certain lines of consumption which are given up with relatively extreme
reluctance. The articles or forms of consumption to which the consumer
clings with the greatest tenacity are commonly the so-called necessaries
of life, or the subsistence minimum. The subsistence minimum is of
course not a rigidly determined allowance of goods, definite and
invariable in kind and quantity; but for the purpose in hand it may
be taken to comprise a certain, more or less definite, aggregate of
consumption required for the maintenance of life. This minimum, it
may be assumed, is ordinarily given up last in case of a progressive
retrenchment of expenditure. That is to say, in a general way, the
most ancient and ingrained of the habits which govern the individual's
life--those habits that touch his existence as an organism--are the
most persistent and imperative. Beyond these come the higher
wants--later-formed habits of the individual or the race--in a somewhat
irregular and by no means invariable gradation. Some of these higher
wants, as for instance the habitual use of certain stimulants, or the
need of salvation (in the eschatological sense), or of good repute, may
in some cases take precedence of the lower or more elementary wants. In
general, the longer the habituation, the more unbroken the habit, and
the more nearly it coincides with previous habitual forms of the life
process, the more persistently will the given habit assert itself. The
habit will be stronger if the particular traits of human nature which
its action involves, or the particular aptitudes that find exercise
in it, are traits or aptitudes that are already largely and profoundly
concerned in the life process or that are intimately bound up with the
life history of the particular racial stock. The varying degrees of ease
with which different habits are formed by different persons, as well as
the varying degrees of reluctance with which different habits are given
up, goes to say that the formation of specific habits is not a matter
of length of habituation simply. Inherited aptitudes and traits of
temperament count for quite as much as length of habituation in deciding
what range of habits will come to dominate any individual's scheme of
life. And the prevalent type of transmitted aptitudes, or in other words
the type of temperament belonging to the dominant ethnic element in
any community, will go far to decide what will be the scope and form
of expression of the community's habitual life process. How greatly the
transmitted idiosyncrasies of aptitude may count in the way of a rapid
and definitive formation of habit in individuals is illustrated by the
extreme facility with which an all-dominating habit of alcoholism
is sometimes formed; or in the similar facility and the similarly
inevitable formation of a habit of devout observances in the case of
persons gifted with a special aptitude in that direction. Much the same
meaning attaches to that peculiar facility of habituation to a specific
human environment that is called romantic love.

Men differ in respect of transmitted aptitudes, or in respect of
the relative facility with which they unfold their life activity in
particular directions; and the habits which coincide with or proceed
upon a relatively strong specific aptitude or a relatively great
specific facility of expression become of great consequence to the man's
well-being. The part played by this element of aptitude in determining
the relative tenacity of the several habits which constitute the
standard of living goes to explain the extreme reluctance with which men
give up any habitual expenditure in the way of conspicuous consumption.
The aptitudes or propensities to which a habit of this kind is to be
referred as its ground are those aptitudes whose exercise is comprised
in emulation; and the propensity for emulation--for invidious
comparison--is of ancient growth and is a pervading trait of human
nature. It is easily called into vigorous activity in any new form, and
it asserts itself with great insistence under any form under which it
has once found habitual expression. When the individual has once
formed the habit of seeking expression in a given line of honorific
expenditure--when a given set of stimuli have come to be habitually
responded to in activity of a given kind and direction under the
guidance of these alert and deep-reaching propensities of emulation--it
is with extreme reluctance that such an habitual expenditure is given
up. And on the other hand, whenever an accession of pecuniary strength
puts the individual in a position to unfold his life process in larger
scope and with additional reach, the ancient propensities of the race
will assert themselves in determining the direction which the new
unfolding of life is to take. And those propensities which are already
actively in the field under some related form of expression, which are
aided by the pointed suggestions afforded by a current accredited
scheme of life, and for the exercise of which the material means and
opportunities are readily available--these will especially have much to
say in shaping the form and direction in which the new accession to
the individual's aggregate force will assert itself. That is to say,
in concrete terms, in any community where conspicuous consumption is an
element of the scheme of life, an increase in an individual's ability
to pay is likely to take the form of an expenditure for some accredited
line of conspicuous consumption.

With the exception of the instinct of self-preservation, the propensity
for emulation is probably the strongest and most alert and persistent of
the economic motives proper. In an industrial community this propensity
for emulation expresses itself in pecuniary emulation; and this, so
far as regards the Western civilized communities of the present, is
virtually equivalent to saying that it expresses itself in some form
of conspicuous waste. The need of conspicuous waste, therefore, stands
ready to absorb any increase in the community's industrial efficiency
or output of goods, after the most elementary physical wants have
been provided for. Where this result does not follow, under modern
conditions, the reason for the discrepancy is commonly to be sought in
a rate of increase in the individual's wealth too rapid for the habit of
expenditure to keep abreast of it; or it may be that the individual in
question defers the conspicuous consumption of the increment to a later
date--ordinarily with a view to heightening the spectacular effect
of the aggregate expenditure contemplated. As increased industrial
efficiency makes it possible to procure the means of livelihood with
less labor, the energies of the industrious members of the community are
bent to the compassing of a higher result in conspicuous expenditure,
rather than slackened to a more comfortable pace. The strain is not
lightened as industrial efficiency increases and makes a lighter strain
possible, but the increment of output is turned to use to meet this
want, which is indefinitely expansible, after the manner commonly
imputed in economic theory to higher or spiritual wants. It is owing
chiefly to the presence of this element in the standard of living that
J. S. Mill was able to say that "hitherto it is questionable if all
the mechanical inventions yet made have lightened the day's toil of any
human being." The accepted standard of expenditure in the community
or in the class to which a person belongs largely determines what his
standard of living will be. It does this directly by commending
itself to his common sense as right and good, through his habitually
contemplating it and assimilating the scheme of life in which it
belongs; but it does so also indirectly through popular insistence
on conformity to the accepted scale of expenditure as a matter of
propriety, under pain of disesteem and ostracism. To accept and
practice the standard of living which is in vogue is both agreeable
and expedient, commonly to the point of being indispensable to personal
comfort and to success in life. The standard of living of any class, so
far as concerns the element of conspicuous waste, is commonly as high as
the earning capacity of the class will permit--with a constant tendency
to go higher. The effect upon the serious activities of men is therefore
to direct them with great singleness of purpose to the largest possible
acquisition of wealth, and to discountenance work that brings no
pecuniary gain. At the same time the effect on consumption is to
concentrate it upon the lines which are most patent to the observers
whose good opinion is sought; while the inclinations and aptitudes whose
exercise does not involve a honorific expenditure of time or substance
tend to fall into abeyance through disuse.
\clearpage
Through this discrimination in favor of visible consumption it has come
about that the domestic life of most classes is relatively shabby, as
compared with the éclat of that overt portion of their life that is
carried on before the eyes of observers. As a secondary consequence of
the same discrimination, people habitually screen their private life
from observation. So far as concerns that portion of their consumption
that may without blame be carried on in secret, they withdraw from all
contact with their neighbors, hence the exclusiveness of people, as
regards their domestic life, in most of the industrially developed
communities; and hence, by remoter derivation, the habit of privacy and
reserve that is so large a feature in the code of proprieties of the
better class in all communities. The low birthrate of the classes upon
whom the requirements of reputable expenditure fall with great urgency
is likewise traceable to the exigencies of a standard of living based
on conspicuous waste. The conspicuous consumption, and the consequent
increased expense, required in the reputable maintenance of a child is
very considerable and acts as a powerful deterrent. It is probably the
most effectual of the Malthusian prudential checks.

The effect of this factor of the standard of living, both in the way of
retrenchment in the obscurer elements of consumption that go to physical
comfort and maintenance, and also in the paucity or absence of children,
is perhaps seen at its best among the classes given to scholarly
pursuits. Because of a presumed superiority and scarcity of the gifts
and attainments that characterize their life, these classes are by
convention subsumed under a higher social grade than their pecuniary
grade should warrant. The scale of decent expenditure in their case
is pitched correspondingly high, and it consequently leaves an
exceptionally narrow margin disposable for the other ends of life. By
force of circumstances, their habitual sense of what is good and right
in these matters, as well as the expectations of the community in the
way of pecuniary decency among the learned, are excessively high--as
measured by the prevalent degree of opulence and earning capacity of the
class, relatively to the non-scholarly classes whose social equals
they nominally are. In any modern community where there is no priestly
monopoly of these occupations, the people of scholarly pursuits are
unavoidably thrown into contact with classes that are pecuniarily their
superiors. The high standard of pecuniary decency in force among these
superior classes is transfused among the scholarly classes with but
little mitigation of its rigor; and as a consequence there is no class
of the community that spends a larger proportion of its substance in
conspicuous waste than these.




\chapter{Pecuniary Canons of Taste}
The caution has already been repeated more than once, that while the
regulating norm of consumption is in large part the requirement of
conspicuous waste, it must not be understood that the motive on which
the consumer acts in any given case is this principle in its bald,
unsophisticated form. Ordinarily his motive is a wish to conform to
established usage, to avoid unfavorable notice and comment, to live
up to the accepted canons of decency in the kind, amount, and grade of
goods consumed, as well as in the decorous employment of his time and
effort. In the common run of cases this sense of prescriptive usage is
present in the motives of the consumer and exerts a direct constraining
force, especially as regards consumption carried on under the eyes of
observers. But a considerable element of prescriptive expensiveness is
observable also in consumption that does not in any appreciable degree
become known to outsiders--as, for instance, articles of underclothing,
some articles of food, kitchen utensils, and other household apparatus
designed for service rather than for evidence. In all such useful
articles a close scrutiny will discover certain features which add to
the cost and enhance the commercial value of the goods in question, but
do not proportionately increase the serviceability of these articles for
the material purposes which alone they ostensibly are designed to serve.

Under the selective surveillance of the law of conspicuous waste there
grows up a code of accredited canons of consumption, the effect of
which is to hold the consumer up to a standard of expensiveness and
wastefulness in his consumption of goods and in his employment of time
and effort. This growth of prescriptive usage has an immediate effect
upon economic life, but it has also an indirect and remoter effect upon
conduct in other respects as well. Habits of thought with respect to
the expression of life in any given direction unavoidably affect the
habitual view of what is good and right in life in other directions
also. In the organic complex of habits of thought which make up the
substance of an individual's conscious life the economic interest does
not lie isolated and distinct from all other interests. Something,
for instance, has already been said of its relation to the canons of
reputability.

The principle of conspicuous waste guides the formation of habits of
thought as to what is honest and reputable in life and in commodities.
In so doing, this principle will traverse other norms of conduct which
do not primarily have to do with the code of pecuniary honor, but
which have, directly or incidentally, an economic significance of some
magnitude. So the canon of honorific waste may, immediately or remotely,
influence the sense of duty, the sense of beauty, the sense of utility,
the sense of devotional or ritualistic fitness, and the scientific sense
of truth.

It is scarcely necessary to go into a discussion here of the particular
points at which, or the particular manner in which, the canon of
honorific expenditure habitually traverses the canons of moral conduct.
The matter is one which has received large attention and illustration at
the hands of those whose office it is to watch and admonish with
respect to any departures from the accepted code of morals. In modern
communities, where the dominant economic and legal feature of the
community's life is the institution of private property, one of the
salient features of the code of morals is the sacredness of property.
There needs no insistence or illustration to gain assent to the
proposition that the habit of holding private property inviolate is
traversed by the other habit of seeking wealth for the sake of the good
repute to be gained through its conspicuous consumption. Most offenses
against property, especially offenses of an appreciable magnitude, come
under this head. It is also a matter of common notoriety and byword
that in offenses which result in a large accession of property to the
offender he does not ordinarily incur the extreme penalty or the extreme
obloquy with which his offenses would be visited on the ground of the
naive moral code alone. The thief or swindler who has gained great
wealth by his delinquency has a better chance than the small thief of
escaping the rigorous penalty of the law and some good repute accrues
to him from his increased wealth and from his spending the irregularly
acquired possessions in a seemly manner. A well-bred expenditure of his
booty especially appeals with great effect to persons of a cultivated
sense of the proprieties, and goes far to mitigate the sense of moral
turpitude with which his dereliction is viewed by them. It may be noted
also--and it is more immediately to the point--that we are all inclined
to condone an offense against property in the case of a man whose motive
is the worthy one of providing the means of a "decent" manner of
life for his wife and children. If it is added that the wife has been
"nurtured in the lap of luxury," that is accepted as an additional
extenuating circumstance. That is to say, we are prone to condone such
an offense where its aim is the honorific one of enabling the offender's
wife to perform for him such an amount of vicarious consumption of time
and substance as is demanded by the standard of pecuniary decency. In
such a case the habit of approving the accustomed degree of conspicuous
waste traverses the habit of deprecating violations of ownership, to the
extent even of sometimes leaving the award of praise or blame uncertain.
This is peculiarly true where the dereliction involves an appreciable
predatory or piratical element.

This topic need scarcely be pursued further here; but the remark may not
be out of place that all that considerable body of morals that clusters
about the concept of an inviolable ownership is itself a psychological
precipitate of the traditional meritoriousness of wealth. And it should
be added that this wealth which is held sacred is valued primarily
for the sake of the good repute to be got through its conspicuous
consumption. The bearing of pecuniary decency upon the scientific spirit
or the quest of knowledge will be taken up in some detail in a separate
chapter. Also as regards the sense of devout or ritual merit and
adequacy in this connection, little need be said in this place. That
topic will also come up incidentally in a later chapter. Still, this
usage of honorific expenditure has much to say in shaping popular tastes
as to what is right and meritorious in sacred matters, and the bearing
of the principle of conspicuous waste upon some of the commonplace
devout observances and conceits may therefore be pointed out.

Obviously, the canon of conspicuous waste is accountable for a great
portion of what may be called devout consumption; as, \emph{e.g.}, the
consumption of sacred edifices, vestments, and other goods of the same
class. Even in those modern cults to whose divinities is imputed a
predilection for temples not built with hands, the sacred buildings and
the other properties of the cult are constructed and decorated with some
view to a reputable degree of wasteful expenditure. And it needs but
little either of observation or introspection--and either will serve the
turn--to assure us that the expensive splendor of the house of worship
has an appreciable uplifting and mellowing effect upon the worshipper's
frame of mind. It will serve to enforce the same fact if we reflect upon
the sense of abject shamefulness with which any evidence of indigence or
squalor about the sacred place affects all beholders. The accessories
of any devout observance should be pecuniarily above reproach. This
requirement is imperative, whatever latitude may be allowed with regard
to these accessories in point of aesthetic or other serviceability. It
may also be in place to notice that in all communities, especially in
neighborhoods where the standard of pecuniary decency for dwellings
is not high, the local sanctuary is more ornate, more conspicuously
wasteful in its architecture and decoration, than the dwelling houses
of the congregation. This is true of nearly all denominations and cults,
whether Christian or Pagan, but it is true in a peculiar degree of
the older and maturer cults. At the same time the sanctuary commonly
contributes little if anything to the physical comfort of the members.
Indeed, the sacred structure not only serves the physical well-being
of the members to but a slight extent, as compared with their humbler
dwelling-houses; but it is felt by all men that a right and enlightened
sense of the true, the beautiful, and the good demands that in all
expenditure on the sanctuary anything that might serve the comfort of
the worshipper should be conspicuously absent. If any element of comfort
is admitted in the fittings of the sanctuary, it should be at least
scrupulously screened and masked under an ostensible austerity. In the
most reputable latter-day houses of worship, where no expense is spared,
the principle of austerity is carried to the length of making the
fittings of the place a means of mortifying the flesh, especially in
appearance. There are few persons of delicate tastes, in the matter of
devout consumption to whom this austerely wasteful discomfort does not
appeal as intrinsically right and good. Devout consumption is of the
nature of vicarious consumption. This canon of devout austerity is based
on the pecuniary reputability of conspicuously wasteful consumption,
backed by the principle that vicarious consumption should conspicuously
not conduce to the comfort of the vicarious consumer.

The sanctuary and its fittings have something of this austerity in all
the cults in which the saint or divinity to whom the sanctuary pertains
is not conceived to be present and make personal use of the property for
the gratification of luxurious tastes imputed to him. The character of
the sacred paraphernalia is somewhat different in this respect in those
cults where the habits of life imputed to the divinity more nearly
approach those of an earthly patriarchal potentate--where he is
conceived to make use of these consumable goods in person. In the latter
case the sanctuary and its fittings take on more of the fashion given to
goods destined for the conspicuous consumption of a temporal master or
owner. On the other hand, where the sacred apparatus is simply employed
in the divinity's service, that is to say, where it is consumed
vicariously on his account by his servants, there the sacred properties
take the character suited to goods that are destined for vicarious
consumption only.

In the latter case the sanctuary and the sacred apparatus are so
contrived as not to enhance the comfort or fullness of life of the
vicarious consumer, or at any rate not to convey the impression that
the end of their consumption is the consumer's comfort. For the end of
vicarious consumption is to enhance, not the fullness of life of the
consumer, but the pecuniary repute of the master for whose behoof the
consumption takes place. Therefore priestly vestments are notoriously
expensive, ornate, and inconvenient; and in the cults where the priestly
servitor of the divinity is not conceived to serve him in the capacity
of consort, they are of an austere, comfortless fashion. And such it is
felt that they should be.

It is not only in establishing a devout standard of decent expensiveness
that the principle of waste invades the domain of the canons of ritual
serviceability. It touches the ways as well as the means, and draws on
vicarious leisure as well as on vicarious consumption. Priestly demeanor
at its best is aloof, leisurely, perfunctory, and uncontaminated with
suggestions of sensuous pleasure. This holds true, in different degrees
of course, for the different cults and denominations; but in the
priestly life of all anthropomorphic cults the marks of a vicarious
consumption of time are visible.

The same pervading canon of vicarious leisure is also visibly present in
the exterior details of devout observances and need only be pointed out
in order to become obvious to all beholders. All ritual has a notable
tendency to reduce itself to a rehearsal of formulas. This development
of formula is most noticeable in the maturer cults, which have at the
same time a more austere, ornate, and severe priestly life and garb; but
it is perceptible also in the forms and methods of worship of the newer
and fresher sects, whose tastes in respect of priests, vestments, and
sanctuaries are less exacting. The rehearsal of the service (the term
"service" carries a suggestion significant for the point in question)
grows more perfunctory as the cult gains in age and consistency, and
this perfunctoriness of the rehearsal is very pleasing to the correct
devout taste. And with a good reason, for the fact of its being
perfunctory goes to say pointedly that the master for whom it is
performed is exalted above the vulgar need of actually proficuous
service on the part of his servants. They are unprofitable servants, and
there is an honorific implication for their master in their remaining
unprofitable. It is needless to point out the close analogy at this
point between the priestly office and the office of the footman. It is
pleasing to our sense of what is fitting in these matters, in either
case, to recognize in the obvious perfunctoriness of the service that it
is a \emph{pro forma} execution only. There should be no show of agility or of
dexterous manipulation in the execution of the priestly office, such as
might suggest a capacity for turning off the work.

In all this there is of course an obvious implication as to the
temperament, tastes, propensities, and habits of life imputed to the
divinity by worshippers who live under the tradition of these pecuniary
canons of reputability. Through its pervading men's habits of thought,
the principle of conspicuous waste has colored the worshippers' notions
of the divinity and of the relation in which the human subject stands
to him. It is of course in the more naive cults that this suffusion
of pecuniary beauty is most patent, but it is visible throughout. All
peoples, at whatever stage of culture or degree of enlightenment, are
fain to eke out a sensibly scant degree of authentic formation regarding
the personality and habitual surroundings of their divinities. In so
calling in the aid of fancy to enrich and fill in their picture of the
divinity's presence and manner of life they habitually impute to him
such traits as go to make up their ideal of a worthy man. And in
seeking communion with the divinity the ways and means of approach are
assimilated as nearly as may be to the divine ideal that is in men's
minds at the time. It is felt that the divine presence is entered with
the best grace, and with the best effect, according to certain accepted
methods and with the accompaniment of certain material circumstances
which in popular apprehension are peculiarly consonant with the divine
nature. This popularly accepted ideal of the bearing and paraphernalia
adequate to such occasions of communion is, of course, to a good extent
shaped by the popular apprehension of what is intrinsically worthy
and beautiful in human carriage and surroundings on all occasions of
dignified intercourse. It would on this account be misleading to
attempt an analysis of devout demeanor by referring all evidences of
the presence of a pecuniary standard of reputability back directly and
baldly to the underlying norm of pecuniary emulation. So it would also
be misleading to ascribe to the divinity, as popularly conceived, a
jealous regard for his pecuniary standing and a habit of avoiding and
condemning squalid situations and surroundings simply because they are
under grade in the pecuniary respect.

And still, after all allowance has been made, it appears that the canons
of pecuniary reputability do, directly or indirectly, materially affect
our notions of the attributes of divinity, as well as our notions
of what are the fit and adequate manner and circumstances of divine
communion. It is felt that the divinity must be of a peculiarly serene
and leisurely habit of life. And whenever his local habitation is
pictured in poetic imagery, for edification or in appeal to the devout
fancy, the devout word-painter, as a matter of course, brings out before
his auditors' imagination a throne with a profusion of the insignia of
opulence and power, and surrounded by a great number of servitors. In
the common run of such presentations of the celestial abodes, the office
of this corps of servants is a vicarious leisure, their time and efforts
being in great measure taken up with an industrially unproductive
rehearsal of the meritorious characteristics and exploits of the
divinity; while the background of the presentation is filled with the
shimmer of the precious metals and of the more expensive varieties of
precious stones. It is only in the crasser expressions of devout fancy
that this intrusion of pecuniary canons into the devout ideals reaches
such an extreme. An extreme case occurs in the devout imagery of the
Negro population of the South. Their word-painters are unable to descend
to anything cheaper than gold; so that in this case the insistence on
pecuniary beauty gives a startling effect in yellow--such as would be
unbearable to a soberer taste. Still, there is probably no cult in which
ideals of pecuniary merit have not been called in to supplement the
ideals of ceremonial adequacy that guide men's conception of what is
right in the matter of sacred apparatus.

Similarly it is felt--and the sentiment is acted upon--that the priestly
servitors of the divinity should not engage in industrially productive
work; that work of any kind--any employment which is of tangible human
use--must not be carried on in the divine presence, or within the
precincts of the sanctuary; that whoever comes into the presence should
come cleansed of all profane industrial features in his apparel
or person, and should come clad in garments of more than everyday
expensiveness; that on holidays set apart in honor of or for communion
with the divinity no work that is of human use should be performed by
any one. Even the remoter, lay dependents should render a vicarious
leisure to the extent of one day in seven. In all these deliverances of
men's uninstructed sense of what is fit and proper in devout observance
and in the relations of the divinity, the effectual presence of the
canons of pecuniary reputability is obvious enough, whether these canons
have had their effect on the devout judgment in this respect immediately
or at the second remove.

These canons of reputability have had a similar, but more far-reaching
and more specifically determinable, effect upon the popular sense
of beauty or serviceability in consumable goods. The requirements of
pecuniary decency have, to a very appreciable extent, influenced the
sense of beauty and of utility in articles of use or beauty.
Articles are to an extent preferred for use on account of their being
conspicuously wasteful; they are felt to be serviceable somewhat in
proportion as they are wasteful and ill adapted to their ostensible use.

The utility of articles valued for their beauty depends closely upon the
expensiveness of the articles. A homely illustration will bring out this
dependence. A hand-wrought silver spoon, of a commercial value of some
ten to twenty dollars, is not ordinarily more serviceable--in the first
sense of the word--than a machine-made spoon of the same material.
It may not even be more serviceable than a machine-made spoon of some
"base" metal, such as aluminum, the value of which may be no more than
some ten to twenty cents. The former of the two utensils is, in fact,
commonly a less effective contrivance for its ostensible purpose than
the latter. The objection is of course ready to hand that, in taking
this view of the matter, one of the chief uses, if not the chief use,
of the costlier spoon is ignored; the hand-wrought spoon gratifies our
taste, our sense of the beautiful, while that made by machinery out of
the base metal has no useful office beyond a brute efficiency. The facts
are no doubt as the objection states them, but it will be evident
on rejection that the objection is after all more plausible than
conclusive. It appears (1) that while the different materials of which
the two spoons are made each possesses beauty and serviceability for the
purpose for which it is used, the material of the hand-wrought spoon is
some one hundred times more valuable than the baser metal, without very
greatly excelling the latter in intrinsic beauty of grain or color, and
without being in any appreciable degree superior in point of mechanical
serviceability; (2) if a close inspection should show that the supposed
hand-wrought spoon were in reality only a very clever citation of
hand-wrought goods, but an imitation so cleverly wrought as to give the
same impression of line and surface to any but a minute examination by
a trained eye, the utility of the article, including the gratification
which the user derives from its contemplation as an object of beauty,
would immediately decline by some eighty or ninety per cent, or even
more; (3) if the two spoons are, to a fairly close observer, so nearly
identical in appearance that the lighter weight of the spurious article
alone betrays it, this identity of form and color will scarcely add
to the value of the machine-made spoon, nor appreciably enhance the
gratification of the user's "sense of beauty" in contemplating it, so
long as the cheaper spoon is not a novelty, ad so long as it can be
procured at a nominal cost. The case of the spoons is typical. The
superior gratification derived from the use and contemplation of costly
and supposedly beautiful products is, commonly, in great measure a
gratification of our sense of costliness masquerading under the name
of beauty. Our higher appreciation of the superior article is an
appreciation of its superior honorific character, much more frequently
than it is an unsophisticated appreciation of its beauty. The
requirement of conspicuous wastefulness is not commonly present,
consciously, in our canons of taste, but it is none the less present as
a constraining norm selectively shaping and sustaining our sense of what
is beautiful, and guiding our discrimination with respect to what may
legitimately be approved as beautiful and what may not.

It is at this point, where the beautiful and the honorific meet and
blend, that a discrimination between serviceability and wastefulness
is most difficult in any concrete case. It frequently happens that an
article which serves the honorific purpose of conspicuous waste is at
the same time a beautiful object; and the same application of labor to
which it owes its utility for the former purpose may, and often does,
give beauty of form and color to the article. The question is further
complicated by the fact that many objects, as, for instance, the
precious stones and the metals and some other materials used for
adornment and decoration, owe their utility as items of conspicuous
waste to an antecedent utility as objects of beauty. Gold, for instance,
has a high degree of sensuous beauty very many if not most of the highly
prized works of art are intrinsically beautiful, though often with
material qualification; the like is true of some stuffs used for
clothing, of some landscapes, and of many other things in less degree.
Except for this intrinsic beauty which they possess, these objects
would scarcely have been coveted as they are, or have become monopolized
objects of pride to their possessors and users. But the utility of these
things to the possessor is commonly due less to their intrinsic beauty
than to the honor which their possession and consumption confers, or to
the obloquy which it wards off.

Apart from their serviceability in other respects, these objects are
beautiful and have a utility as such; they are valuable on this account
if they can be appropriated or monopolized; they are, therefore, coveted
as valuable possessions, and their exclusive enjoyment gratifies the
possessor's sense of pecuniary superiority at the same time that their
contemplation gratifies his sense of beauty. But their beauty, in the
naive sense of the word, is the occasion rather than the ground of their
monopolization or of their commercial value. "Great as is the sensuous
beauty of gems, their rarity and price adds an expression of distinction
to them, which they would never have if they were cheap." There is,
indeed, in the common run of cases under this head, relatively little
incentive to the exclusive possession and use of these beautiful
things, except on the ground of their honorific character as items of
conspicuous waste. Most objects of this general class, with the partial
exception of articles of personal adornment, would serve all other
purposes than the honorific one equally well, whether owned by the
person viewing them or not; and even as regards personal ornaments it is
to be added that their chief purpose is to lend éclat to the person
of their wearer (or owner) by comparison with other persons who are
compelled to do without. The aesthetic serviceability of objects of
beauty is not greatly nor universally heightened by possession.

The generalization for which the discussion so far affords ground is
that any valuable object in order to appeal to our sense of beauty must
conform to the requirements of beauty and of expensiveness both. But
this is not all. Beyond this the canon of expensiveness also affects
our tastes in such a way as to inextricably blend the marks of
expensiveness, in our appreciation, with the beautiful features of
the object, and to subsume the resultant effect under the head of an
appreciation of beauty simply. The marks of expensiveness come to be
accepted as beautiful features of the expensive articles. They are
pleasing as being marks of honorific costliness, and the pleasure which
they afford on this score blends with that afforded by the beautiful
form and color of the object; so that we often declare that an article
of apparel, for instance, is "perfectly lovely," when pretty much all
that an analysis of the aesthetic value of the article would leave
ground for is the declaration that it is pecuniarily honorific.

This blending and confusion of the elements of expensiveness and
of beauty is, perhaps, best exemplified in articles of dress and of
household furniture. The code of reputability in matters of dress
decides what shapes, colors, materials, and general effects in human
apparel are for the time to be accepted as suitable; and departures from
the code are offensive to our taste, supposedly as being departures from
aesthetic truth. The approval with which we look upon fashionable attire
is by no means to be accounted pure make-believe. We readily, and for
the most part with utter sincerity, find those things pleasing that
are in vogue. Shaggy dress-stuffs and pronounced color effects, for
instance, offend us at times when the vogue is goods of a high,
glossy finish and neutral colors. A fancy bonnet of this year's model
unquestionably appeals to our sensibilities today much more forcibly
than an equally fancy bonnet of the model of last year; although
when viewed in the perspective of a quarter of a century, it would, I
apprehend, be a matter of the utmost difficulty to award the palm
for intrinsic beauty to the one rather than to the other of these
structures. So, again, it may be remarked that, considered simply in
their physical juxtaposition with the human form, the high gloss of a
gentleman's hat or of a patent-leather shoe has no more of intrinsic
beauty than a similarly high gloss on a threadbare sleeve; and yet
there is no question but that all well-bred people (in the Occidental
civilized communities) instinctively and unaffectedly cleave to the one
as a phenomenon of great beauty, and eschew the other as offensive to
every sense to which it can appeal. It is extremely doubtful if any one
could be induced to wear such a contrivance as the high hat of civilized
society, except for some urgent reason based on other than aesthetic
grounds.

By further habituation to an appreciative perception of the marks
of expensiveness in goods, and by habitually identifying beauty with
reputability, it comes about that a beautiful article which is not
expensive is accounted not beautiful. In this way it has happened, for
instance, that some beautiful flowers pass conventionally for offensive
weeds; others that can be cultivated with relative ease are accepted
and admired by the lower middle class, who can afford no more expensive
luxuries of this kind; but these varieties are rejected as vulgar by
those people who are better able to pay for expensive flowers and who
are educated to a higher schedule of pecuniary beauty in the florist's
products; while still other flowers, of no greater intrinsic beauty than
these, are cultivated at great cost and call out much admiration from
flower-lovers whose tastes have been matured under the critical guidance
of a polite environment.

The same variation in matters of taste, from one class of society to
another, is visible also as regards many other kinds of consumable
goods, as, for example, is the case with furniture, houses, parks,
and gardens. This diversity of views as to what is beautiful in these
various classes of goods is not a diversity of the norm according to
which the unsophisticated sense of the beautiful works. It is not a
constitutional difference of endowments in the aesthetic respect, but
rather a difference in the code of reputability which specifies what
objects properly lie within the scope of honorific consumption for the
class to which the critic belongs. It is a difference in the traditions
of propriety with respect to the kinds of things which may, without
derogation to the consumer, be consumed under the head of objects of
taste and art. With a certain allowance for variations to be accounted
for on other grounds, these traditions are determined, more or less
rigidly, by the pecuniary plane of life of the class.

Everyday life affords many curious illustrations of the way in which the
code of pecuniary beauty in articles of use varies from class to class,
as well as of the way in which the conventional sense of beauty departs
in its deliverances from the sense untutored by the requirements of
pecuniary repute. Such a fact is the lawn, or the close-cropped yard or
park, which appeals so unaffectedly to the taste of the Western peoples.
It appears especially to appeal to the tastes of the well-to-do classes
in those communities in which the dolicho-blond element predominates
in an appreciable degree. The lawn unquestionably has an element of
sensuous beauty, simply as an object of apperception, and as such no
doubt it appeals pretty directly to the eye of nearly all races and all
classes; but it is, perhaps, more unquestionably beautiful to the eye
of the dolicho-blond than to most other varieties of men. This higher
appreciation of a stretch of greensward in this ethnic element than
in the other elements of the population, goes along with certain other
features of the dolicho-blond temperament that indicate that this racial
element had once been for a long time a pastoral people inhabiting a
region with a humid climate. The close-cropped lawn is beautiful in the
eyes of a people whose inherited bent it is to readily find pleasure in
contemplating a well-preserved pasture or grazing land.

For the aesthetic purpose the lawn is a cow pasture; and in some cases
today--where the expensiveness of the attendant circumstances bars out
any imputation of thrift--the idyl of the dolicho-blond is rehabilitated
in the introduction of a cow into a lawn or private ground. In such
cases the cow made use of is commonly of an expensive breed. The vulgar
suggestion of thrift, which is nearly inseparable from the cow, is a
standing objection to the decorative use of this animal. So that in all
cases, except where luxurious surroundings negate this suggestion,
the use of the cow as an object of taste must be avoided. Where the
predilection for some grazing animal to fill out the suggestion of the
pasture is too strong to be suppressed, the cow's place is often given
to some more or less inadequate substitute, such as deer, antelopes, or
some such exotic beast. These substitutes, although less beautiful
to the pastoral eye of Western man than the cow, are in such cases
preferred because of their superior expensiveness or futility, and their
consequent repute. They are not vulgarly lucrative either in fact or in
suggestion.

Public parks of course fall in the same category with the lawn; they
too, at their best, are imitations of the pasture. Such a park is of
course best kept by grazing, and the cattle on the grass are themselves
no mean addition to the beauty of the thing, as need scarcely be
insisted on with anyone who has once seen a well-kept pasture. But it
is worth noting, as an expression of the pecuniary element in popular
taste, that such a method of keeping public grounds is seldom resorted
to. The best that is done by skilled workmen under the supervision of a
trained keeper is a more or less close imitation of a pasture, but
the result invariably falls somewhat short of the artistic effect of
grazing. But to the average popular apprehension a herd of cattle so
pointedly suggests thrift and usefulness that their presence in the
public pleasure ground would be intolerably cheap. This method
of keeping grounds is comparatively inexpensive, therefore it is
indecorous.

Of the same general bearing is another feature of public grounds. There
is a studious exhibition of expensiveness coupled with a make-believe of
simplicity and crude serviceability. Private grounds also show the same
physiognomy wherever they are in the management or ownership of persons
whose tastes have been formed under middle-class habits of life or under
the upper-class traditions of no later a date than the childhood of the
generation that is now passing. Grounds which conform to the instructed
tastes of the latter-day upper class do not show these features in so
marked a degree. The reason for this difference in tastes between the
past and the incoming generation of the well-bred lies in the changing
economic situation. A similar difference is perceptible in other
respects, as well as in the accepted ideals of pleasure grounds. In this
country as in most others, until the last half century but a very small
proportion of the population were possessed of such wealth as would
exempt them from thrift. Owing to imperfect means of communication,
this small fraction were scattered and out of effective touch with one
another. There was therefore no basis for a growth of taste in disregard
of expensiveness. The revolt of the well-bred taste against vulgar
thrift was unchecked. Wherever the unsophisticated sense of beauty
might show itself sporadically in an approval of inexpensive or thrifty
surroundings, it would lack the "social confirmation" which nothing
but a considerable body of like-minded people can give. There was,
therefore, no effective upper-class opinion that would overlook
evidences of possible inexpensiveness in the management of grounds;
and there was consequently no appreciable divergence between the
leisure-class and the lower middle-class ideal in the physiognomy of
pleasure grounds. Both classes equally constructed their ideals with the
fear of pecuniary disrepute before their eyes.

Today a divergence in ideals is beginning to be apparent. The portion of
the leisure class that has been consistently exempt from work and from
pecuniary cares for a generation or more is now large enough to form and
sustain opinion in matters of taste. Increased mobility of the members
has also added to the facility with which a "social confirmation" can be
attained within the class. Within this select class the exemption from
thrift is a matter so commonplace as to have lost much of its utility
as a basis of pecuniary decency. Therefore the latter-day upper-class
canons of taste do not so consistently insist on an unremitting
demonstration of expensiveness and a strict exclusion of the appearance
of thrift. So, a predilection for the rustic and the "natural" in parks
and grounds makes its appearance on these higher social and intellectual
levels. This predilection is in large part an outcropping of the
instinct of workmanship; and it works out its results with varying
degrees of consistency. It is seldom altogether unaffected, and at times
it shades off into something not widely different from that make-believe
of rusticity which has been referred to above.

A weakness for crudely serviceable contrivances that pointedly suggest
immediate and wasteless use is present even in the middle-class tastes;
but it is there kept well in hand under the unbroken dominance of the
canon of reputable futility. Consequently it works out in a variety
of ways and means for shamming serviceability--in such contrivances
as rustic fences, bridges, bowers, pavilions, and the like decorative
features. An expression of this affectation of serviceability, at what
is perhaps its widest divergence from the first promptings of the
sense of economic beauty, is afforded by the cast-iron rustic fence and
trellis or by a circuitous drive laid across level ground.

The select leisure class has outgrown the use of these
pseudo-serviceable variants of pecuniary beauty, at least at some
points. But the taste of the more recent accessions to the leisure class
proper and of the middle and lower classes still requires a pecuniary
beauty to supplement the aesthetic beauty, even in those objects which
are primarily admired for the beauty that belongs to them as natural
growths.

The popular taste in these matters is to be seen in the prevalent high
appreciation of topiary work and of the conventional flower-beds of
public grounds. Perhaps as happy an illustration as may be had of this
dominance of pecuniary beauty over aesthetic beauty in middle-class
tastes is seen in the reconstruction of the grounds lately occupied by
the Columbian Exposition. The evidence goes to show that the requirement
of reputable expensiveness is still present in good vigor even where
all ostensibly lavish display is avoided. The artistic effects actually
wrought in this work of reconstruction diverge somewhat widely from
the effect to which the same ground would have lent itself in hands not
guided by pecuniary canons of taste. And even the better class of the
city's population view the progress of the work with an unreserved
approval which suggests that there is in this case little if any
discrepancy between the tastes of the upper and the lower or middle
classes of the city. The sense of beauty in the population of this
representative city of the advanced pecuniary culture is very chary of
any departure from its great cultural principle of conspicuous waste.

The love of nature, perhaps itself borrowed from a higher-class code of
taste, sometimes expresses itself in unexpected ways under the guidance
of this canon of pecuniary beauty, and leads to results that may seem
incongruous to an unreflecting beholder. The well-accepted practice of
planting trees in the treeless areas of this country, for instance, has
been carried over as an item of honorific expenditure into the heavily
wooded areas; so that it is by no means unusual for a village or a
farmer in the wooded country to clear the land of its native trees and
immediately replant saplings of certain introduced varieties about the
farmyard or along the streets. In this way a forest growth of oak, elm,
beech, butternut, hemlock, basswood, and birch is cleared off to give
room for saplings of soft maple, cottonwood, and brittle willow. It is
felt that the inexpensiveness of leaving the forest trees standing
would derogate from the dignity that should invest an article which is
intended to serve a decorative and honorific end.

The like pervading guidance of taste by pecuniary repute is traceable
in the prevalent standards of beauty in animals. The part played by this
canon of taste in assigning her place in the popular aesthetic scale to
the cow has already been spokes of. Something to the same effect is
true of the other domestic animals, so far as they are in an appreciable
degree industrially useful to the community--as, for instance, barnyard
fowl, hogs, cattle, sheep, goats, draught-horses. They are of the
nature of productive goods, and serve a useful, often a lucrative end;
therefore beauty is not readily imputed to them. The case is different
with those domestic animals which ordinarily serve no industrial end;
such as pigeons, parrots and other cage-birds, cats, dogs, and fast
horses. These commonly are items of conspicuous consumption, and are
therefore honorific in their nature and may legitimately be accounted
beautiful. This class of animals are conventionally admired by the body
of the upper classes, while the pecuniarily lower classes--and that
select minority of the leisure class among whom the rigorous canon that
abjures thrift is in a measure obsolescent--find beauty in one class of
animals as in another, without drawing a hard and fast line of pecuniary
demarcation between the beautiful and the ugly. In the case of those
domestic animals which are honorific and are reputed beautiful, there
is a subsidiary basis of merit that should be spokes of. Apart from the
birds which belong in the honorific class of domestic animals, and which
owe their place in this class to their non-lucrative character alone,
the animals which merit particular attention are cats, dogs, and fast
horses. The cat is less reputable than the other two just named, because
she is less wasteful; she may even serve a useful end. At the same time
the cat's temperament does not fit her for the honorific purpose. She
lives with man on terms of equality, knows nothing of that relation of
status which is the ancient basis of all distinctions of worth, honor,
and repute, and she does not lend herself with facility to an invidious
comparison between her owner and his neighbors. The exception to this
last rule occurs in the case of such scarce and fanciful products as
the Angora cat, which have some slight honorific value on the ground
of expensiveness, and have, therefore, some special claim to beauty on
pecuniary grounds.

The dog has advantages in the way of uselessness as well as in special
gifts of temperament. He is often spoken of, in an eminent sense, as
the friend of man, and his intelligence and fidelity are praised. The
meaning of this is that the dog is man's servant and that he has
the gift of an unquestioning subservience and a slave's quickness in
guessing his master's mood. Coupled with these traits, which fit him
well for the relation of status--and which must for the present purpose
be set down as serviceable traits--the dog has some characteristics
which are of a more equivocal aesthetic value. He is the filthiest of
the domestic animals in his person and the nastiest in his habits. For
this he makes up is a servile, fawning attitude towards his master, and
a readiness to inflict damage and discomfort on all else. The dog, then,
commends himself to our favor by affording play to our propensity for
mastery, and as he is also an item of expense, and commonly serves no
industrial purpose, he holds a well-assured place in men's regard as
a thing of good repute. The dog is at the same time associated in our
imagination with the chase--a meritorious employment and an expression
of the honorable predatory impulse. Standing on this vantage ground,
whatever beauty of form and motion and whatever commendable mental
traits he may possess are conventionally acknowledged and magnified.
And even those varieties of the dog which have been bred into grotesque
deformity by the dog-fancier are in good faith accounted beautiful by
many. These varieties of dogs--and the like is true of other fancy-bred
animals--are rated and graded in aesthetic value somewhat in proportion
to the degree of grotesqueness and instability of the particular fashion
which the deformity takes in the given case. For the purpose in hand,
this differential utility on the ground of grotesqueness and instability
of structure is reducible to terms of a greater scarcity and consequent
expense. The commercial value of canine monstrosities, such as the
prevailing styles of pet dogs both for men's and women's use, rests
on their high cost of production, and their value to their owners
lies chiefly in their utility as items of conspicuous consumption.
Indirectly, through reflection upon their honorific expensiveness,
a social worth is imputed to them; and so, by an easy substitution of
words and ideas, they come to be admired and reputed beautiful. Since
any attention bestowed upon these animals is in no sense gainful
or useful, it is also reputable; and since the habit of giving them
attention is consequently not deprecated, it may grow into an habitual
attachment of great tenacity and of a most benevolent character. So that
in the affection bestowed on pet animals the canon of expensiveness
is present more or less remotely as a norm which guides and shapes the
sentiment and the selection of its object. The like is true, as will be
noticed presently, with respect to affection for persons also; although
the manner in which the norm acts in that case is somewhat different.

The case of the fast horse is much like that of the dog. He is on the
whole expensive, or wasteful and useless--for the industrial purpose.
What productive use he may possess, in the way of enhancing the
well-being of the community or making the way of life easier for men,
takes the form of exhibitions of force and facility of motion that
gratify the popular aesthetic sense. This is of course a substantial
serviceability. The horse is not endowed with the spiritual aptitude
for servile dependence in the same measure as the dog; but he ministers
effectually to his master's impulse to convert the "animate" forces of
the environment to his own use and discretion and so express his own
dominating individuality through them. The fast horse is at least
potentially a race-horse, of high or low degree; and it is as such that
he is peculiarly serviceable to his owner. The utility of the fast horse
lies largely in his efficiency as a means of emulation; it gratifies the
owner's sense of aggression and dominance to have his own horse outstrip
his neighbor's. This use being not lucrative, but on the whole pretty
consistently wasteful, and quite conspicuously so, it is honorific,
and therefore gives the fast horse a strong presumptive position of
reputability. Beyond this, the race-horse proper has also a similarly
non-industrial but honorific use as a gambling instrument.

The fast horse, then, is aesthetically fortunate, in that the canon of
pecuniary good repute legitimates a free appreciation of whatever beauty
or serviceability he may possess. His pretensions have the countenance
of the principle of conspicuous waste and the backing of the predatory
aptitude for dominance and emulation. The horse is, moreover, a
beautiful animal, although the race-horse is so in no peculiar degree to
the uninstructed taste of those persons who belong neither in the class
of race-horse fanciers nor in the class whose sense of beauty is held in
abeyance by the moral constraint of the horse fancier's award. To this
untutored taste the most beautiful horse seems to be a form which has
suffered less radical alteration than the race-horse under the
breeder's selective development of the animal. Still, when a writer
or speaker--especially of those whose eloquence is most consistently
commonplace wants an illustration of animal grace and serviceability,
for rhetorical use, he habitually turns to the horse; and he commonly
makes it plain before he is done that what he has in mind is the
race-horse.

It should be noted that in the graduated appreciation of varieties
of horses and of dogs, such as one meets with among people of even
moderately cultivated tastes in these matters, there is also discernible
another and more direct line of influence of the leisure-class canons of
reputability. In this country, for instance, leisure-class tastes are
to some extent shaped on usages and habits which prevail, or which are
apprehended to prevail, among the leisure class of Great Britain. In
dogs this is true to a less extent than in horses. In horses, more
particularly in saddle horses--which at their best serve the purpose of
wasteful display simply--it will hold true in a general way that a
horse is more beautiful in proportion as he is more English; the English
leisure class being, for purposes of reputable usage, the upper leisure
class of this country, and so the exemplar for the lower grades. This
mimicry in the methods of the apperception of beauty and in the forming
of judgments of taste need not result in a spurious, or at any rate not
a hypocritical or affected, predilection. The predilection is as serious
and as substantial an award of taste when it rests on this basis as
when it rests on any other, the difference is that this taste is and
as substantial an award of taste when it rests on this basis as when it
rests on any other; the difference is that this taste is a taste for the
reputably correct, not for the aesthetically true.

The mimicry, it should be said, extends further than to the sense of
beauty in horseflesh simply. It includes trappings and horsemanship as
well, so that the correct or reputably beautiful seat or posture is also
decided by English usage, as well as the equestrian gait. To show how
fortuitous may sometimes be the circumstances which decide what shall
be becoming and what not under the pecuniary canon of beauty, it may be
noted that this English seat, and the peculiarly distressing gait which
has made an awkward seat necessary, are a survival from the time when
the English roads were so bad with mire and mud as to be virtually
impassable for a horse travelling at a more comfortable gait; so that
a person of decorous tastes in horsemanship today rides a punch with
docked tail, in an uncomfortable posture and at a distressing gait,
because the English roads during a great part of the last century were
impassable for a horse travelling at a more horse-like gait, or for
an animal built for moving with ease over the firm and open country to
which the horse is indigenous. It is not only with respect to consumable
goods--including domestic animals--that the canons of taste have been
colored by the canons of pecuniary reputability. Something to the like
effect is to be said for beauty in persons. In order to avoid whatever
may be matter of controversy, no weight will be given in this connection
to such popular predilection as there may be for the dignified
(leisurely) bearing and poly presence that are by vulgar tradition
associated with opulence in mature men. These traits are in some measure
accepted as elements of personal beauty. But there are certain elements
of feminine beauty, on the other hand, which come in under this head,
and which are of so concrete and specific a character as to admit of
itemized appreciation. It is more or less a rule that in communities
which are at the stage of economic development at which women are valued
by the upper class for their service, the ideal of female beauty is a
robust, large-limbed woman. The ground of appreciation is the physique,
while the conformation of the face is of secondary weight only. A
well-known instance of this ideal of the early predatory culture is that
of the maidens of the Homeric poems.

This ideal suffers a change in the succeeding development, when, in the
conventional scheme, the office of the high-class wife comes to be a
vicarious leisure simply. The ideal then includes the characteristics
which are supposed to result from or to go with a life of leisure
consistently enforced. The ideal accepted under these circumstances may
be gathered from descriptions of beautiful women by poets and writers of
the chivalric times. In the conventional scheme of those days ladies
of high degree were conceived to be in perpetual tutelage, and to be
scrupulously exempt from all useful work. The resulting chivalric or
romantic ideal of beauty takes cognizance chiefly of the face, and
dwells on its delicacy, and on the delicacy of the hands and feet,
the slender figure, and especially the slender waist. In the pictured
representations of the women of that time, and in modern romantic
imitators of the chivalric thought and feeling, the waist is attenuated
to a degree that implies extreme debility. The same ideal is still
extant among a considerable portion of the population of modern
industrial communities; but it is to be said that it has retained
its hold most tenaciously in those modern communities which are least
advanced in point of economic and civil development, and which show the
most considerable survivals of status and of predatory institutions.
That is to say, the chivalric ideal is best preserved in those existing
communities which are substantially least modern. Survivals of this
lackadaisical or romantic ideal occur freely in the tastes of the
well-to-do classes of Continental countries. In modern communities which
have reached the higher levels of industrial development, the upper
leisure class has accumulated so great a mass of wealth as to place its
women above all imputation of vulgarly productive labor. Here the status
of women as vicarious consumers is beginning to lose its place in the
sections of the body of the people; and as a consequence the ideal of
feminine beauty is beginning to change back again from the infirmly
delicate, translucent, and hazardously slender, to a woman of the
archaic type that does not disown her hands and feet, nor, indeed, the
other gross material facts of her person. In the course of economic
development the ideal of beauty among the peoples of the Western culture
has shifted from the woman of physical presence to the lady, and it is
beginning to shift back again to the woman; and all in obedience to the
changing conditions of pecuniary emulation. The exigencies of emulation
at one time required lusty slaves; at another time they required a
conspicuous performance of vicarious leisure and consequently an obvious
disability; but the situation is now beginning to outgrow this last
requirement, since, under the higher efficiency of modern industry,
leisure in women is possible so far down the scale of reputability that
it will no longer serve as a definitive mark of the highest pecuniary
grade.

Apart from this general control exercised by the norm of conspicuous
waste over the ideal of feminine beauty, there are one or two details
which merit specific mention as showing how it may exercise an extreme
constraint in detail over men's sense of beauty in women. It has
already been noticed that at the stages of economic evolution at which
conspicuous leisure is much regarded as a means of good repute, the
ideal requires delicate and diminutive hands and feet and a slender
waist. These features, together with the other, related faults of
structure that commonly go with them, go to show that the person so
affected is incapable of useful effort and must therefore be supported
in idleness by her owner. She is useless and expensive, and she is
consequently valuable as evidence of pecuniary strength. It results that
at this cultural stage women take thought to alter their persons, so as
to conform more nearly to the requirements of the instructed taste of
the time; and under the guidance of the canon of pecuniary decency,
the men find the resulting artificially induced pathological features
attractive. So, for instance, the constricted waist which has had so
wide and persistent a vogue in the communities of the Western culture,
and so also the deformed foot of the Chinese. Both of these are
mutilations of unquestioned repulsiveness to the untrained sense. It
requires habituation to become reconciled to them. Yet there is no room
to question their attractiveness to men into whose scheme of life they
fit as honorific items sanctioned by the requirements of pecuniary
reputability. They are items of pecuniary and cultural beauty which have
come to do duty as elements of the ideal of womanliness.

The connection here indicated between the aesthetic value and the
invidious pecuniary value of things is of course not present in the
consciousness of the valuer. So far as a person, in forming a judgment
of taste, takes thought and reflects that the object of beauty under
consideration is wasteful and reputable, and therefore may legitimately
be accounted beautiful; so far the judgment is not a \emph{bona fide} judgment
of taste and does not come up for consideration in this connection. The
connection which is here insisted on between the reputability and the
apprehended beauty of objects lies through the effect which the fact of
reputability has upon the valuer's habits of thought. He is in the
habit of forming judgments of value of various kinds-economic, moral,
aesthetic, or reputable concerning the objects with which he has to do,
and his attitude of commendation towards a given object on any other
ground will affect the degree of his appreciation of the object when he
comes to value it for the aesthetic purpose. This is more particularly
true as regards valuation on grounds so closely related to the aesthetic
ground as that of reputability. The valuation for the aesthetic purpose
and for the purpose of repute are not held apart as distinctly as might
be. Confusion is especially apt to arise between these two kinds of
valuation, because the value of objects for repute is not habitually
distinguished in speech by the use of a special descriptive term. The
result is that the terms in familiar use to designate categories
or elements of beauty are applied to cover this unnamed element of
pecuniary merit, and the corresponding confusion of ideas follows by
easy consequence. The demands of reputability in this way coalesce in
the popular apprehension with the demands of the sense of beauty, and
beauty which is not accompanied by the accredited marks of good repute
is not accepted. But the requirements of pecuniary reputability and
those of beauty in the naive sense do not in any appreciable degree
coincide. The elimination from our surroundings of the pecuniarily
unfit, therefore, results in a more or less thorough elimination of that
considerable range of elements of beauty which do not happen to conform
to the pecuniary requirement. The underlying norms of taste are of very
ancient growth, probably far antedating the advent of the pecuniary
institutions that are here under discussion. Consequently, by force of
the past selective adaptation of men's habits of thought, it happens
that the requirements of beauty, simply, are for the most part best
satisfied by inexpensive contrivances and structures which in a
straightforward manner suggest both the office which they are to perform
and the method of serving their end. It may be in place to recall the
modern psychological position. Beauty of form seems to be a question of
facility of apperception. The proposition could perhaps safely be made
broader than this. If abstraction is made from association, suggestion,
and "expression," classed as elements of beauty, then beauty in any
perceived object means that the mind readily unfolds its apperceptive
activity in the directions which the object in question affords. But the
directions in which activity readily unfolds or expresses itself are the
directions to which long and close habituation has made the mind prone.
So far as concerns the essential elements of beauty, this habituation
is an habituation so close and long as to have induced not only a
proclivity to the apperceptive form in question, but an adaptation of
physiological structure and function as well. So far as the economic
interest enters into the constitution of beauty, it enters as a
suggestion or expression of adequacy to a purpose, a manifest and
readily inferable subservience to the life process. This expression of
economic facility or economic serviceability in any object--what may
be called the economic beauty of the object-is best served by neat and
unambiguous suggestion of its office and its efficiency for the material
ends of life.

On this ground, among objects of use the simple and unadorned article
is aesthetically the best. But since the pecuniary canon of reputability
rejects the inexpensive in articles appropriated to individual
consumption, the satisfaction of our craving for beautiful things
must be sought by way of compromise. The canons of beauty must be
circumvented by some contrivance which will give evidence of a reputably
wasteful expenditure, at the same time that it meets the demands of our
critical sense of the useful and the beautiful, or at least meets the
demand of some habit which has come to do duty in place of that sense.
Such an auxiliary sense of taste is the sense of novelty; and this
latter is helped out in its surrogateship by the curiosity with which
men view ingenious and puzzling contrivances. Hence it comes that
most objects alleged to be beautiful, and doing duty as such, show
considerable ingenuity of design and are calculated to puzzle the
beholder--to bewilder him with irrelevant suggestions and hints of the
improbable--at the same time that they give evidence of an expenditure
of labor in excess of what would give them their fullest efficency for
their ostensible economic end.

This may be shown by an illustration taken from outside the range of our
everyday habits and everyday contact, and so outside the range of
our bias. Such are the remarkable feather mantles of Hawaii, or the
well-known cawed handles of the ceremonial adzes of several Polynesian
islands. These are undeniably beautiful, both in the sense that they
offer a pleasing composition of form, lines, and color, and in the sense
that they evince great skill and ingenuity in design and construction.
At the same time the articles are manifestly ill fitted to serve any
other economic purpose. But it is not always that the evolution of
ingenious and puzzling contrivances under the guidance of the canon of
wasted effort works out so happy a result. The result is quite as
often a virtually complete suppression of all elements that would
bear scrutiny as expressions of beauty, or of serviceability, and the
substitution of evidences of misspent ingenuity and labor, backed by a
conspicuous ineptitude; until many of the objects with which we surround
ourselves in everyday life, and even many articles of everyday dress and
ornament, are such as would not be tolerated except under the stress of
prescriptive tradition. Illustrations of this substitution of ingenuity
and expense in place of beauty and serviceability are to be seen, for
instance, in domestic architecture, in domestic art or fancy work,
in various articles of apparel, especially of feminine and priestly
apparel.

The canon of beauty requires expression of the generic. The "novelty"
due to the demands of conspicuous waste traverses this canon of beauty,
in that it results in making the physiognomy of our objects of taste a
congeries of idiosyncrasies; and the idiosyncrasies are, moreover, under
the selective surveillance of the canon of expensiveness.

This process of selective adaptation of designs to the end of
conspicuous waste, and the substitution of pecuniary beauty for
aesthetic beauty, has been especially effective in the development of
architecture. It would be extremely difficult to find a modern civilized
residence or public building which can claim anything better than
relative inoffensiveness in the eyes of anyone who will dissociate the
elements of beauty from those of honorific waste. The endless variety of
fronts presented by the better class of tenements and apartment houses
in our cities is an endless variety of architectural distress and of
suggestions of expensive discomfort. Considered as objects of beauty,
the dead walls of the sides and back of these structures, left untouched
by the hands of the artist, are commonly the best feature of the
building.

What has been said of the influence of the law of conspicuous waste upon
the canons of taste will hold true, with but a slight change of terms,
of its influence upon our notions of the serviceability of goods for
other ends than the aesthetic one. Goods are produced and consumed as a
means to the fuller unfolding of human life; and their utility consists,
in the first instance, in their efficiency as means to this end. The end
is, in the first instance, the fullness of life of the individual, taken
in absolute terms. But the human proclivity to emulation has seized upon
the consumption of goods as a means to an invidious comparison, and has
thereby invested consumable goods with a secondary utility as evidence
of relative ability to pay. This indirect or secondary use of consumable
goods lends an honorific character to consumption and presently also
to the goods which best serve the emulative end of consumption. The
consumption of expensive goods is meritorious, and the goods which
contain an appreciable element of cost in excess of what goes to
give them serviceability for their ostensible mechanical purpose
are honorific. The marks of superfluous costliness in the goods are
therefore marks of worth--of high efficency for the indirect, invidious
end to be served by their consumption; and conversely, goods are
humilific, and therefore unattractive, if they show too thrifty an
adaptation to the mechanical end sought and do not include a margin of
expensiveness on which to rest a complacent invidious comparison. This
indirect utility gives much of their value to the "better" grades of
goods. In order to appeal to the cultivated sense of utility, an article
must contain a modicum of this indirect utility.

While men may have set out with disapproving an inexpensive manner of
living because it indicated inability to spend much, and so indicated
a lack of pecuniary success, they end by falling into the habit of
disapproving cheap things as being intrinsically dishonorable or
unworthy because they are cheap. As time has gone on, each succeeding
generation has received this tradition of meritorious expenditure from
the generation before it, and has in its turn further elaborated and
fortified the traditional canon of pecuniary reputability in goods
consumed; until we have finally reached such a degree of conviction as
to the unworthiness of all inexpensive things, that we have no
longer any misgivings in formulating the maxim, "Cheap and nasty." So
thoroughly has the habit of approving the expensive and disapproving
the inexpensive been ingrained into our thinking that we instinctively
insist upon at least some measure of wasteful expensiveness in all our
consumption, even in the case of goods which are consumed in strict
privacy and without the slightest thought of display. We all feel,
sincerely and without misgiving, that we are the more lifted up in
spirit for having, even in the privacy of our own household, eaten
our daily meal by the help of hand-wrought silver utensils, from
hand-painted china (often of dubious artistic value) laid on high-priced
table linen. Any retrogression from the standard of living which we are
accustomed to regard as worthy in this respect is felt to be a grievous
violation of our human dignity. So, also, for the last dozen years
candles have been a more pleasing source of light at dinner than any
other. Candlelight is now softer, less distressing to well-bred eyes,
than oil, gas, or electric light. The same could not have been said
thirty years ago, when candles were, or recently had been, the cheapest
available light for domestic use. Nor are candles even now found to
give an acceptable or effective light for any other than a ceremonial
illumination.

A political sage still living has summed up the conclusion of this whole
matter in the dictum: "A cheap coat makes a cheap man," and there is
probably no one who does not feel the convincing force of the maxim.

The habit of looking for the marks of superfluous expensiveness in
goods, and of requiring that all goods should afford some utility of the
indirect or invidious sort, leads to a change in the standards by which
the utility of goods is gauged. The honorific element and the element
of brute efficiency are not held apart in the consumer's appreciation of
commodities, and the two together go to make up the unanalyzed
aggregate serviceability of the goods. Under the resulting standard of
serviceability, no article will pass muster on the strength of material
sufficiency alone. In order to completeness and full acceptability to
the consumer it must also show the honorific element. It results that
the producers of articles of consumption direct their efforts to the
production of goods that shall meet this demand for the honorific
element. They will do this with all the more alacrity and effect, since
they are themselves under the dominance of the same standard of worth in
goods, and would be sincerely grieved at the sight of goods which lack
the proper honorific finish. Hence it has come about that there are
today no goods supplied in any trade which do not contain the
honorific element in greater or less degree. Any consumer who might,
Diogenes-like, insist on the elimination of all honorific or wasteful
elements from his consumption, would be unable to supply his most
trivial wants in the modern market. Indeed, even if he resorted to
supplying his wants directly by his own efforts, he would find it
difficult if not impossible to divest himself of the current habits of
thought on this head; so that he could scarcely compass a supply of the
necessaries of life for a day's consumption without instinctively and
by oversight incorporating in his home-made product something of this
honorific, quasi-decorative element of wasted labor.

It is notorious that in their selection of serviceable goods in the
retail market purchasers are guided more by the finish and workmanship
of the goods than by any marks of substantial serviceability. Goods,
in order to sell, must have some appreciable amount of labor spent in
giving them the marks of decent expensiveness, in addition to what goes
to give them efficiency for the material use which they are to serve.
This habit of making obvious costliness a canon of serviceability of
course acts to enhance the aggregate cost of articles of consumption.
It puts us on our guard against cheapness by identifying merit in some
degree with cost. There is ordinarily a consistent effort on the part
of the consumer to obtain goods of the required serviceability at as
advantageous a bargain as may be; but the conventional requirement of
obvious costliness, as a voucher and a constituent of the serviceability
of the goods, leads him to reject as under grade such goods as do not
contain a large element of conspicuous waste.

It is to be added that a large share of those features of consumable
goods which figure in popular apprehension as marks of serviceability,
and to which reference is here had as elements of conspicuous waste,
commend themselves to the consumer also on other grounds than that of
expensiveness alone. They usually give evidence of skill and effective
workmanship, even if they do not contribute to the substantial
serviceability of the goods; and it is no doubt largely on some such
ground that any particular mark of honorific serviceability first comes
into vogue and afterward maintains its footing as a normal constituent
element of the worth of an article. A display of efficient workmanship
is pleasing simply as such, even where its remoter, for the time
unconsidered, outcome is futile. There is a gratification of the
artistic sense in the contemplation of skillful work. But it is also to
be added that no such evidence of skillful workmanship, or of ingenious
and effective adaptation of means to an end, will, in the long run,
enjoy the approbation of the modern civilized consumer unless it has the
sanction of the Canon of conspicuous waste.

The position here taken is enforced in a felicitous manner by the place
assigned in the economy of consumption to machine products. The point
of material difference between machine-made goods and the hand-wrought
goods which serve the same purposes is, ordinarily, that the former
serve their primary purpose more adequately. They are a more perfect
product--show a more perfect adaptation of means to end. This does not
save them from disesteem and deprecation, for they fall short under
the test of honorific waste. Hand labor is a more wasteful method
of production; hence the goods turned out by this method are more
serviceable for the purpose of pecuniary reputability; hence the marks
of hand labor come to be honorific, and the goods which exhibit these
marks take rank as of higher grade than the corresponding machine
product. Commonly, if not invariably, the honorific marks of hand
labor are certain imperfections and irregularities in the lines of the
hand-wrought article, showing where the workman has fallen short in the
execution of the design. The ground of the superiority of hand-wrought
goods, therefore, is a certain margin of crudeness. This margin must
never be so wide as to show bungling workmanship, since that would be
evidence of low cost, nor so narrow as to suggest the ideal precision
attained only by the machine, for that would be evidence of low cost.

The appreciation of those evidences of honorific crudeness to which
hand-wrought goods owe their superior worth and charm in the eyes
of well-bred people is a matter of nice discrimination. It requires
training and the formation of right habits of thought with respect to
what may be called the physiognomy of goods. Machine-made goods of
daily use are often admired and preferred precisely on account of their
excessive perfection by the vulgar and the underbred who have not given
due thought to the punctilios of elegant consumption. The ceremonial
inferiority of machine products goes to show that the perfection of
skill and workmanship embodied in any costly innovations in the finish
of goods is not sufficient of itself to secure them acceptance and
permanent favor. The innovation must have the support of the canon of
conspicuous waste. Any feature in the physiognomy of goods, however
pleasing in itself, and however well it may approve itself to the taste
for effective work, will not be tolerated if it proves obnoxious to this
norm of pecuniary reputability.

The ceremonial inferiority or uncleanness in consumable goods due to
"commonness," or in other words to their slight cost of production,
has been taken very seriously by many persons. The objection to machine
products is often formulated as an objection to the commonness of such
goods. What is common is within the (pecuniary) reach of many people.
Its consumption is therefore not honorific, since it does not serve the
purpose of a favorable invidious comparison with other consumers. Hence
the consumption, or even the sight of such goods, is inseparable from an
odious suggestion of the lower levels of human life, and one comes away
from their contemplation with a pervading sense of meanness that is
extremely distasteful and depressing to a person of sensibility. In
persons whose tastes assert themselves imperiously, and who have not the
gift, habit, or incentive to discriminate between the grounds of
their various judgments of taste, the deliverances of the sense of the
honorific coalesce with those of the sense of beauty and of the sense of
serviceability--in the manner already spoken of; the resulting
composite valuation serves as a judgment of the object's beauty or its
serviceability, according as the valuer's bias or interest inclines him
to apprehend the object in the one or the other of these aspects. It
follows not infrequently that the marks of cheapness or commonness
are accepted as definitive marks of artistic unfitness, and a code or
schedule of aesthetic proprieties on the one hand, and of aesthetic
abominations on the other, is constructed on this basis for guidance in
questions of taste.

As has already been pointed out, the cheap, and therefore indecorous,
articles of daily consumption in modern industrial communities are
commonly machine products; and the generic feature of the physiognomy
of machine-made goods as compared with the hand-wrought article is their
greater perfection in workmanship and greater accuracy in the detail
execution of the design. Hence it comes about that the visible
imperfections of the hand-wrought goods, being honorific, are accounted
marks of superiority in point of beauty, or serviceability, or both.
Hence has arisen that exaltation of the defective, of which John Ruskin
and William Morris were such eager spokesmen in their time; and on this
ground their propaganda of crudity and wasted effort has been taken up
and carried forward since their time. And hence also the propaganda for
a return to handicraft and household industry. So much of the work
and speculations of this group of men as fairly comes under the
characterization here given would have been impossible at a time when
the visibly more perfect goods were not the cheaper.

It is of course only as to the economic value of this school of
aesthetic teaching that anything is intended to be said or can be said
here. What is said is not to be taken in the sense of depreciation, but
chiefly as a characterization of the tendency of this teaching in its
effect on consumption and on the production of consumable goods.

The manner in which the bias of this growth of taste has worked itself
out in production is perhaps most cogently exemplified in the book
manufacture with which Morris busied himself during the later years of
his life; but what holds true of the work of the Kelmscott Press in an
eminent degree, holds true with but slightly abated force when applied
to latter-day artistic book-making generally--as to type, paper,
illustration, binding materials, and binder's work. The claims to
excellence put forward by the later products of the bookmaker's industry
rest in some measure on the degree of its approximation to the crudities
of the time when the work of book-making was a doubtful struggle with
refractory materials carried on by means of insufficient appliances.
These products, since they require hand labor, are more expensive; they
are also less convenient for use than the books turned out with a view
to serviceability alone; they therefore argue ability on the part of
the purchaser to consume freely, as well as ability to waste time and
effort. It is on this basis that the printers of today are returning to
"old-style," and other more or less obsolete styles of type which are
less legible and give a cruder appearance to the page than the "modern."
Even a scientific periodical, with ostensibly no purpose but the most
effective presentation of matter with which its science is concerned,
will concede so much to the demands of this pecuniary beauty as to
publish its scientific discussions in oldstyle type, on laid paper, and
with uncut edges. But books which are not ostensibly concerned with the
effective presentation of their contents alone, of course go farther
in this direction. Here we have a somewhat cruder type, printed on
hand-laid, deckel-edged paper, with excessive margins and uncut leaves,
with bindings of a painstaking crudeness and elaborate ineptitude. The
Kelmscott Press reduced the matter to an absurdity--as seen from the
point of view of brute serviceability alone--by issuing books for modern
use, edited with the obsolete spelling, printed in black-letter, and
bound in limp vellum fitted with thongs. As a further characteristic
feature which fixes the economic place of artistic book-making, there
is the fact that these more elegant books are, at their best, printed in
limited editions. A limited edition is in effect a guarantee--somewhat
crude, it is true--that this book is scarce and that it therefore is
costly and lends pecuniary distinction to its consumer.

The special attractiveness of these book-products to the book-buyer of
cultivated taste lies, of course, not in a conscious, naive recognition
of their costliness and superior clumsiness. Here, as in the parallel
case of the superiority of hand-wrought articles over machine products,
the conscious ground of preference is an intrinsic excellence imputed to
the costlier and more awkward article. The superior excellence imputed
to the book which imitates the products of antique and obsolete
processes is conceived to be chiefly a superior utility in the aesthetic
respect; but it is not unusual to find a well-bred book-lover insisting
that the clumsier product is also more serviceable as a vehicle of
printed speech. So far as regards the superior aesthetic value of the
decadent book, the chances are that the book-lover's contention has some
ground. The book is designed with an eye single to its beauty, and the
result is commonly some measure of success on the part of the designer.
What is insisted on here, however, is that the canon of taste under
which the designer works is a canon formed under the surveillance of
the law of conspicuous waste, and that this law acts selectively to
eliminate any canon of taste that does not conform to its demands. That
is to say, while the decadent book may be beautiful, the limits within
which the designer may work are fixed by requirements of a non-aesthetic
kind. The product, if it is beautiful, must also at the same time be
costly and ill adapted to its ostensible use. This mandatory canon of
taste in the case of the book-designer, however, is not shaped entirely
by the law of waste in its first form; the canon is to some extent
shaped in conformity to that secondary expression of the predatory
temperament, veneration for the archaic or obsolete, which in one of its
special developments is called classicism. In aesthetic theory it might
be extremely difficult, if not quite impracticable, to draw a line
between the canon of classicism, or regard for the archaic, and the
canon of beauty. For the aesthetic purpose such a distinction need
scarcely be drawn, and indeed it need not exist. For a theory of taste
the expression of an accepted ideal of archaism, on whatever basis it
may have been accepted, is perhaps best rated as an element of beauty;
there need be no question of its legitimation. But for the present
purpose--for the purpose of determining what economic grounds are
present in the accepted canons of taste and what is their significance
for the distribution and consumption of goods--the distinction is not
similarly beside the point. The position of machine products in the
civilized scheme of consumption serves to point out the nature of the
relation which subsists between the canon of conspicuous waste and the
code of proprieties in consumption. Neither in matters of art and taste
proper, nor as regards the current sense of the serviceability of goods,
does this canon act as a principle of innovation or initiative. It does
not go into the future as a creative principle which makes innovations
and adds new items of consumption and new elements of cost. The
principle in question is, in a certain sense, a negative rather than a
positive law. It is a regulative rather than a creative principle. It
very rarely initiates or originates any usage or custom directly. Its
action is selective only. Conspicuous wastefulness does not directly
afford ground for variation and growth, but conformity to its
requirements is a condition to the survival of such innovations as may
be made on other grounds. In whatever way usages and customs and methods
of expenditure arise, they are all subject to the selective action of
this norm of reputability; and the degree in which they conform to its
requirements is a test of their fitness to survive in the competition
with other similar usages and customs. Other thing being equal, the more
obviously wasteful usage or method stands the better chance of survival
under this law. The law of conspicuous waste does not account for the
origin of variations, but only for the persistence of such forms as are
fit to survive under its dominance. It acts to conserve the fit, not to
originate the acceptable. Its office is to prove all things and to hold
fast that which is good for its purpose.




\chapter{Dress as an Expression of the Pecuniary Culture}

It will in place, by way of illustration, to show in some detail how the
economic principles so far set forth apply to everyday facts in some one
direction of the life process. For this purpose no line of consumption
affords a more apt illustration than expenditure on dress. It is
especially the rule of the conspicuous waste of goods that finds
expression in dress, although the other, related principles of pecuniary
repute are also exemplified in the same contrivances. Other methods
of putting one's pecuniary standing in evidence serve their end
effectually, and other methods are in vogue always and everywhere; but
expenditure on dress has this advantage over most other methods, that
our apparel is always in evidence and affords an indication of our
pecuniary standing to all observers at the first glance. It is also true
that admitted expenditure for display is more obviously present, and is,
perhaps, more universally practiced in the matter of dress than in any
other line of consumption. No one finds difficulty in assenting to the
commonplace that the greater part of the expenditure incurred by all
classes for apparel is incurred for the sake of a respectable appearance
rather than for the protection of the person. And probably at no other
point is the sense of shabbiness so keenly felt as it is if we fall
short of the standard set by social usage in this matter of dress. It
is true of dress in even a higher degree than of most other items of
consumption, that people will undergo a very considerable degree of
privation in the comforts or the necessaries of life in order to afford
what is considered a decent amount of wasteful consumption; so that
it is by no means an uncommon occurrence, in an inclement climate,
for people to go ill clad in order to appear well dressed. And the
commercial value of the goods used for clotting in any modern community
is made up to a much larger extent of the fashionableness, the
reputability of the goods than of the mechanical service which they
render in clothing the person of the wearer. The need of dress is
eminently a "higher" or spiritual need.

This spiritual need of dress is not wholly, nor even chiefly, a naive
propensity for display of expenditure. The law of conspicuous waste
guides consumption in apparel, as in other things, chiefly at the second
remove, by shaping the canons of taste and decency. In the common run of
cases the conscious motive of the wearer or purchaser of conspicuously
wasteful apparel is the need of conforming to established usage, and of
living up to the accredited standard of taste and reputability. It is
not only that one must be guided by the code of proprieties in dress in
order to avoid the mortification that comes of unfavorable notice and
comment, though that motive in itself counts for a great deal; but
besides that, the requirement of expensiveness is so ingrained into
our habits of thought in matters of dress that any other than expensive
apparel is instinctively odious to us. Without reflection or analysis,
we feel that what is inexpensive is unworthy. "A cheap coat makes a
cheap man." "Cheap and nasty" is recognized to hold true in dress with
even less mitigation than in other lines of consumption. On the ground
both of taste and of serviceability, an inexpensive article of apparel
is held to be inferior, under the maxim "cheap and nasty." We find
things beautiful, as well as serviceable, somewhat in proportion as
they are costly. With few and inconsequential exceptions, we all find
a costly hand-wrought article of apparel much preferable, in point
of beauty and of serviceability, to a less expensive imitation of it,
however cleverly the spurious article may imitate the costly original;
and what offends our sensibilities in the spurious article is not that
it falls short in form or color, or, indeed, in visual effect in any
way. The offensive object may be so close an imitation as to defy
any but the closest scrutiny; and yet so soon as the counterfeit
is detected, its aesthetic value, and its commercial value as well,
declines precipitately. Not only that, but it may be asserted with
but small risk of contradiction that the aesthetic value of a detected
counterfeit in dress declines somewhat in the same proportion as the
counterfeit is cheaper than its original. It loses caste aesthetically
because it falls to a lower pecuniary grade.

But the function of dress as an evidence of ability to pay does not end
with simply showing that the wearer consumes valuable goods in excess of
what is required for physical comfort. Simple conspicuous waste of goods
is effective and gratifying as far as it goes; it is good \emph{prima facie}
evidence of pecuniary success, and consequently \emph{prima facie} evidence of
social worth. But dress has subtler and more far-reaching possibilities
than this crude, first-hand evidence of wasteful consumption only. If,
in addition to showing that the wearer can afford to consume freely and
uneconomically, it can also be shown in the same stroke that he or she
is not under the necessity of earning a livelihood, the evidence of
social worth is enhanced in a very considerable degree. Our dress,
therefore, in order to serve its purpose effectually, should not only
he expensive, but it should also make plain to all observers that
the wearer is not engaged in any kind of productive labor. In the
evolutionary process by which our system of dress has been elaborated
into its present admirably perfect adaptation to its purpose, this
subsidiary line of evidence has received due attention. A detailed
examination of what passes in popular apprehension for elegant apparel
will show that it is contrived at every point to convey the impression
that the wearer does not habitually put forth any useful effort. It
goes without saying that no apparel can be considered elegant, or
even decent, if it shows the effect of manual labor on the part of the
wearer, in the way of soil or wear. The pleasing effect of neat and
spotless garments is chiefly, if not altogether, due to their carrying
the suggestion of leisure-exemption from personal contact with
industrial processes of any kind. Much of the charm that invests the
patent-leather shoe, the stainless linen, the lustrous cylindrical hat,
and the walking-stick, which so greatly enhance the native dignity of
a gentleman, comes of their pointedly suggesting that the wearer cannot
when so attired bear a hand in any employment that is directly and
immediately of any human use. Elegant dress serves its purpose of
elegance not only in that it is expensive, but also because it is
the insignia of leisure. It not only shows that the wearer is able to
consume a relatively large value, but it argues at the same time that he
consumes without producing.

The dress of women goes even farther than that of men in the way of
demonstrating the wearer's abstinence from productive employment. It
needs no argument to enforce the generalization that the more elegant
styles of feminine bonnets go even farther towards making work
impossible than does the man's high hat. The woman's shoe adds the
so-called French heel to the evidence of enforced leisure afforded
by its polish; because this high heel obviously makes any, even the
simplest and most necessary manual work extremely difficult. The like
is true even in a higher degree of the skirt and the rest of the drapery
which characterizes woman's dress. The substantial reason for our
tenacious attachment to the skirt is just this; it is expensive and it
hampers the wearer at every turn and incapacitates her for all useful
exertion. The like is true of the feminine custom of wearing the hair
excessively long.

But the woman's apparel not only goes beyond that of the modern man
in the degree in which it argues exemption from labor; it also adds a
peculiar and highly characteristic feature which differs in kind from
anything habitually practiced by the men. This feature is the class of
contrivances of which the corset is the typical example. The corset
is, in economic theory, substantially a mutilation, undergone for the
purpose of lowering the subject's vitality and rendering her permanently
and obviously unfit for work. It is true, the corset impairs the
personal attractions of the wearer, but the loss suffered on that
score is offset by the gain in reputability which comes of her visibly
increased expensiveness and infirmity. It may broadly be set down
that the womanliness of woman's apparel resolves itself, in point of
substantial fact, into the more effective hindrance to useful exertion
offered by the garments peculiar to women. This difference between
masculine and feminine apparel is here simply pointed out as a
characteristic feature. The ground of its occurrence will be discussed
presently.

So far, then, we have, as the great and dominant norm of dress, the
broad principle of conspicuous waste. Subsidiary to this principle,
and as a corollary under it, we get as a second norm the principle of
conspicuous leisure. In dress construction this norm works out in the
shape of divers contrivances going to show that the wearer does not and,
as far as it may conveniently be shown, can not engage in productive
labor. Beyond these two principles there is a third of scarcely less
constraining force, which will occur to any one who reflects at all
on the subject. Dress must not only be conspicuously expensive and
inconvenient, it must at the same time be up to date. No explanation at
all satisfactory has hitherto been offered of the phenomenon of
changing fashions. The imperative requirement of dressing in the latest
accredited manner, as well as the fact that this accredited fashion
constantly changes from season to season, is sufficiently familiar to
every one, but the theory of this flux and change has not been worked
out. We may of course say, with perfect consistency and truthfulness,
that this principle of novelty is another corollary under the law of
conspicuous waste. Obviously, if each garment is permitted to serve for
but a brief term, and if none of last season's apparel is carried
over and made further use of during the present season, the wasteful
expenditure on dress is greatly increased. This is good as far as it
goes, but it is negative only. Pretty much all that this consideration
warrants us in saying is that the norm of conspicuous waste exercises a
controlling surveillance in all matters of dress, so that any change in
the fashions must conspicuous waste exercises a controlling surveillance
in all matters of dress, so that any change in the fashions must conform
to the requirement of wastefulness; it leaves unanswered the question
as to the motive for making and accepting a change in the prevailing
styles, and it also fails to explain why conformity to a given style at
a given time is so imperatively necessary as we know it to be.

For a creative principle, capable of serving as motive to invention
and innovation in fashions, we shall have to go back to the primitive,
non-economic motive with which apparel originated--the motive of
adornment. Without going into an extended discussion of how and why this
motive asserts itself under the guidance of the law of expensiveness, it
may be stated broadly that each successive innovation in the fashions is
an effort to reach some form of display which shall be more acceptable
to our sense of form and color or of effectiveness, than that which it
displaces. The changing styles are the expression of a restless search
for something which shall commend itself to our aesthetic sense; but
as each innovation is subject to the selective action of the norm of
conspicuous waste, the range within which innovation can take place is
somewhat restricted. The innovation must not only be more beautiful,
or perhaps oftener less offensive, than that which it displaces, but it
must also come up to the accepted standard of expensiveness.

It would seem at first sight that the result of such an unremitting
struggle to attain the beautiful in dress should be a gradual approach
to artistic perfection. We might naturally expect that the fashions
should show a well-marked trend in the direction of some one or more
types of apparel eminently becoming to the human form; and we might even
feel that we have substantial ground for the hope that today, after
all the ingenuity and effort which have been spent on dress these many
years, the fashions should have achieved a relative perfection and
a relative stability, closely approximating to a permanently tenable
artistic ideal. But such is not the case. It would be very hazardous
indeed to assert that the styles of today are intrinsically more
becoming than those of ten years ago, or than those of twenty, or fifty,
or one hundred years ago. On the other hand, the assertion freely goes
uncontradicted that styles in vogue two thousand years ago are more
becoming than the most elaborate and painstaking constructions of today.

The explanation of the fashions just offered, then, does not fully
explain, and we shall have to look farther. It is well known that
certain relatively stable styles and types of costume have been worked
out in various parts of the world; as, for instance, among the Japanese,
Chinese, and other Oriental nations; likewise among the Greeks, Romans,
and other Eastern peoples of antiquity so also, in later times, among
the peasants of nearly every country of Europe. These national or
popular costumes are in most cases adjudged by competent critics to
be more becoming, more artistic, than the fluctuating styles of modern
civilized apparel. At the same time they are also, at least usually,
less obviously wasteful; that is to say, other elements than that of a
display of expense are more readily detected in their structure.

These relatively stable costumes are, commonly, pretty strictly and
narrowly localized, and they vary by slight and systematic gradations
from place to place. They have in every case been worked out by peoples
or classes which are poorer than we, and especially they belong in
countries and localities and times where the population, or at least
the class to which the costume in question belongs, is relatively
homogeneous, stable, and immobile. That is to say, stable costumes
which will bear the test of time and perspective are worked out under
circumstances where the norm of conspicuous waste asserts itself less
imperatively than it does in the large modern civilized cities, whose
relatively mobile wealthy population today sets the pace in matters of
fashion. The countries and classes which have in this way worked out
stable and artistic costumes have been so placed that the pecuniary
emulation among them has taken the direction of a competition in
conspicuous leisure rather than in conspicuous consumption of goods. So
that it will hold true in a general way that fashions are least stable
and least becoming in those communities where the principle of a
conspicuous waste of goods asserts itself most imperatively, as among
ourselves. All this points to an antagonism between expensiveness and
artistic apparel. In point of practical fact, the norm of conspicuous
waste is incompatible with the requirement that dress should be
beautiful or becoming. And this antagonism offers an explanation of that
restless change in fashion which neither the canon of expensiveness nor
that of beauty alone can account for.

The standard of reputability requires that dress should show wasteful
expenditure; but all wastefulness is offensive to native taste. The
psychological law has already been pointed out that all men--and women
perhaps even in a higher degree abhor futility, whether of effort or
of expenditure--much as Nature was once said to abhor a vacuum. But the
principle of conspicuous waste requires an obviously futile expenditure;
and the resulting conspicuous expensiveness of dress is therefore
intrinsically ugly. Hence we find that in all innovations in dress, each
added or altered detail strives to avoid condemnation by showing some
ostensible purpose, at the same time that the requirement of conspicuous
waste prevents the purposefulness of these innovations from becoming
anything more than a somewhat transparent pretense. Even in its freest
flights, fashion rarely if ever gets away from a simulation of some
ostensible use. The ostensible usefulness of the fashionable details
of dress, however, is always so transparent a make-believe, and
their substantial futility presently forces itself so baldly upon our
attention as to become unbearable, and then we take refuge in a new
style. But the new style must conform to the requirement of reputable
wastefulness and futility. Its futility presently becomes as odious
as that of its predecessor; and the only remedy which the law of waste
allows us is to seek relief in some new construction, equally futile and
equally untenable. Hence the essential ugliness and the unceasing change
of fashionable attire.

Having so explained the phenomenon of shifting fashions, the next
thing is to make the explanation tally with everyday facts. Among these
everyday facts is the well-known liking which all men have for the
styles that are in vogue at any given time. A new style comes into vogue
and remains in favor for a season, and, at least so long as it is
a novelty, people very generally find the new style attractive. The
prevailing fashion is felt to be beautiful. This is due partly to the
relief it affords in being different from what went before it, partly
to its being reputable. As indicated in the last chapter, the canon
of reputability to some extent shapes our tastes, so that under its
guidance anything will be accepted as becoming until its novelty wears
off, or until the warrant of reputability is transferred to a new and
novel structure serving the same general purpose. That the alleged
beauty, or "loveliness," of the styles in vogue at any given time is
transient and spurious only is attested by the fact that none of the
many shifting fashions will bear the test of time. When seen in the
perspective of half-a-dozen years or more, the best of our fashions
strike us as grotesque, if not unsightly. Our transient attachment to
whatever happens to be the latest rests on other than aesthetic grounds,
and lasts only until our abiding aesthetic sense has had time to assert
itself and reject this latest indigestible contrivance.

The process of developing an aesthetic nausea takes more or less time;
the length of time required in any given case being inversely as the
degree of intrinsic odiousness of the style in question. This time
relation between odiousness and instability in fashions affords ground
for the inference that the more rapidly the styles succeed and
displace one another, the more offensive they are to sound taste. The
presumption, therefore, is that the farther the community, especially
the wealthy classes of the community, develop in wealth and mobility and
in the range of their human contact, the more imperatively will the law
of conspicuous waste assert itself in matters of dress, the more will
the sense of beauty tend to fall into abeyance or be overborne by the
canon of pecuniary reputability, the more rapidly will fashions shift
and change, and the more grotesque and intolerable will be the varying
styles that successively come into vogue.

There remains at least one point in this theory of dress yet to be
discussed. Most of what has been said applies to men's attire as well
as to that of women; although in modern times it applies at nearly all
points with greater force to that of women. But at one point the dress
of women differs substantially from that of men. In woman's dress there
is obviously greater insistence on such features as testify to the
wearer's exemption from or incapacity for all vulgarly productive
employment. This characteristic of woman's apparel is of interest, not
only as completing the theory of dress, but also as confirming what has
already been said of the economic status of women, both in the past and
in the present.

As has been seen in the discussion of woman's status under the heads
of Vicarious Leisure and Vicarious Consumption, it has in the course
of economic development become the office of the woman to consume
vicariously for the head of the household; and her apparel is contrived
with this object in view. It has come about that obviously productive
labor is in a peculiar degree derogatory to respectable women, and
therefore special pains should be taken in the construction of women's
dress, to impress upon the beholder the fact (often indeed a fiction)
that the wearer does not and can not habitually engage in useful work.
Propriety requires respectable women to abstain more consistently from
useful effort and to make more of a show of leisure than the men of the
same social classes. It grates painfully on our nerves to contemplate
the necessity of any well-bred woman's earning a livelihood by useful
work. It is not "woman's sphere." Her sphere is within the household,
which she should "beautify," and of which she should be the "chief
ornament." The male head of the household is not currently spoken of as
its ornament. This feature taken in conjunction with the other fact that
propriety requires more unremitting attention to expensive display in
the dress and other paraphernalia of women, goes to enforce the view
already implied in what has gone before. By virtue of its descent from a
patriarchal past, our social system makes it the woman's function in
an especial degree to put in evidence her household's ability to pay.
According to the modern civilized scheme of life, the good name of the
household to which she belongs should be the special care of the woman;
and the system of honorific expenditure and conspicuous leisure by which
this good name is chiefly sustained is therefore the woman's sphere.
In the ideal scheme, as it tends to realize itself in the life of
the higher pecuniary classes, this attention to conspicuous waste of
substance and effort should normally be the sole economic function of
the woman.

At the stage of economic development at which the women were still in
the full sense the property of the men, the performance of conspicuous
leisure and consumption came to be part of the services required of
them. The women being not their own masters, obvious expenditure and
leisure on their part would redound to the credit of their master rather
than to their own credit; and therefore the more expensive and the
more obviously unproductive the women of the household are, the more
creditable and more effective for the purpose of reputability of the
household or its head will their life be. So much so that the women have
been required not only to afford evidence of a life of leisure, but even
to disable themselves for useful activity.

It is at this point that the dress of men falls short of that of women,
and for sufficient reason. Conspicuous waste and conspicuous leisure
are reputable because they are evidence of pecuniary strength; pecuniary
strength is reputable or honorific because, in the last analysis, it
argues success and superior force; therefore the evidence of waste
and leisure put forth by any individual in his own behalf cannot
consistently take such a form or be carried to such a pitch as to argue
incapacity or marked discomfort on his part; as the exhibition would in
that case show not superior force, but inferiority, and so defeat its
own purpose. So, then, wherever wasteful expenditure and the show of
abstention from effort is normally, or on an average, carried to the
extent of showing obvious discomfort or voluntarily induced physical
disability. There the immediate inference is that the individual in
question does not perform this wasteful expenditure and undergo this
disability for her own personal gain in pecuniary repute, but in
behalf of some one else to whom she stands in a relation of economic
dependence; a relation which in the last analysis must, in economic
theory, reduce itself to a relation of servitude.

To apply this generalization to women's dress, and put the matter in
concrete terms: the high heel, the skirt, the impracticable bonnet, the
corset, and the general disregard of the wearer's comfort which is an
obvious feature of all civilized women's apparel, are so many items of
evidence to the effect that in the modern civilized scheme of life the
woman is still, in theory, the economic dependent of the man--that,
perhaps in a highly idealized sense, she still is the man's chattel. The
homely reason for all this conspicuous leisure and attire on the part
of women lies in the fact that they are servants to whom, in the
differentiation of economic functions, has been delegated the office
of putting in evidence their master's ability to pay. There is a marked
similarity in these respects between the apparel of women and that of
domestic servants, especially liveried servants. In both there is a very
elaborate show of unnecessary expensiveness, and in both cases there is
also a notable disregard of the physical comfort of the wearer. But
the attire of the lady goes farther in its elaborate insistence on the
idleness, if not on the physical infirmity of the wearer, than does that
of the domestic. And this is as it should be; for in theory, according
to the ideal scheme of the pecuniary culture, the lady of the house is
the chief menial of the household.

Besides servants, currently recognized as such, there is at least one
other class of persons whose garb assimilates them to the class
of servants and shows many of the features that go to make up the
womanliness of woman's dress. This is the priestly class. Priestly
vestments show, in accentuated form, all the features that have been
shown to be evidence of a servile status and a vicarious life. Even
more strikingly than the everyday habit of the priest, the vestments,
properly so called, are ornate, grotesque, inconvenient, and, at least
ostensibly, comfortless to the point of distress. The priest is at the
same time expected to refrain from useful effort and, when before the
public eye, to present an impassively disconsolate countenance, very
much after the manner of a well-trained domestic servant. The
shaven face of the priest is a further item to the same effect. This
assimilation of the priestly class to the class of body servants, in
demeanor and apparel, is due to the similarity of the two classes as
regards economic function. In economic theory, the priest is a body
servant, constructively in attendance upon the person of the divinity
whose livery he wears. His livery is of a very expensive character, as
it should be in order to set forth in a beseeming manner the dignity of
his exalted master; but it is contrived to show that the wearing of it
contributes little or nothing to the physical comfort of the wearer,
for it is an item of vicarious consumption, and the repute which accrues
from its consumption is to be imputed to the absent master, not to the
servant.

The line of demarcation between the dress of women, priests, and
servants, on the one hand, and of men, on the other hand, is not always
consistently observed in practice, but it will scarcely be disputed
that it is always present in a more or less definite way in the popular
habits of thought. There are of course also free men, and not a few
of them, who, in their blind zeal for faultless reputable attire,
transgress the theoretical line between man's and woman's dress, to the
extent of arraying themselves in apparel that is obviously designed to
vex the mortal frame; but everyone recognizes without hesitation that
such apparel for men is a departure from the normal. We are in the habit
of saying that such dress is "effeminate"; and one sometimes hears the
remark that such or such an exquisitely attired gentleman is as well
dressed as a footman.

Certain apparent discrepancies under this theory of dress merit a more
detailed examination, especially as they mark a more or less evident
trend in the later and maturer development of dress. The vogue of the
corset offers an apparent exception from the rule of which it has here
been cited as an illustration. A closer examination, however, will show
that this apparent exception is really a verification of the rule that
the vogue of any given element or feature in dress rests on its utility
as an evidence of pecuniary standing. It is well known that in the
industrially more advanced communities the corset is employed only
within certain fairly well defined social strata. The women of the
poorer classes, especially of the rural population, do not habitually
use it, except as a holiday luxury. Among these classes the women have
to work hard, and it avails them little in the way of a pretense of
leisure to so crucify the flesh in everyday life. The holiday use of
the contrivance is due to imitation of a higher-class canon of decency.
Upwards from this low level of indigence and manual labor, the corset
was until within a generation or two nearly indispensable to a socially
blameless standing for all women, including the wealthiest and most
reputable. This rule held so long as there still was no large class of
people wealthy enough to be above the imputation of any necessity
for manual labor and at the same time large enough to form a
self-sufficient, isolated social body whose mass would afford a
foundation for special rules of conduct within the class, enforced by
the current opinion of the class alone. But now there has grown up a
large enough leisure class possessed of such wealth that any aspersion
on the score of enforced manual employment would be idle and harmless
calumny; and the corset has therefore in large measure fallen into
disuse within this class. The exceptions under this rule of exemption
from the corset are more apparent than real. They are the wealthy
classes of countries with a lower industrial structure--nearer the
archaic, quasi-industrial type--together with the later accessions of
the wealthy classes in the more advanced industrial communities. The
latter have not yet had time to divest themselves of the plebeian canons
of taste and of reputability carried over from their former, lower
pecuniary grade. Such survival of the corset is not infrequent among the
higher social classes of those American cities, for instance, which
have recently and rapidly risen into opulence. If the word be used as a
technical term, without any odious implication, it may be said that the
corset persists in great measure through the period of snobbery--the
interval of uncertainty and of transition from a lower to the upper
levels of pecuniary culture. That is to say, in all countries which
have inherited the corset it continues in use wherever and so long as
it serves its purpose as an evidence of honorific leisure by arguing
physical disability in the wearer. The same rule of course applies to
other mutilations and contrivances for decreasing the visible efficiency
of the individual.

Something similar should hold true with respect to divers items of
conspicuous consumption, and indeed something of the kind does seem to
hold to a slight degree of sundry features of dress, especially if such
features involve a marked discomfort or appearance of discomfort to
the wearer. During the past one hundred years there is a tendency
perceptible, in the development of men's dress especially, to
discontinue methods of expenditure and the use of symbols of leisure
which must have been irksome, which may have served a good purpose in
their time, but the continuation of which among the upper classes today
would be a work of supererogation; as, for instance, the use of powdered
wigs and of gold lace, and the practice of constantly shaving the face.
There has of late years been some slight recrudescence of the shaven
face in polite society, but this is probably a transient and unadvised
mimicry of the fashion imposed upon body servants, and it may fairly be
expected to go the way of the powdered wig of our grandfathers.

These indices and others which resemble them in point of the boldness
with which they point out to all observers the habitual uselessness
of those persons who employ them, have been replaced by other, more
dedicate methods of expressing the same fact; methods which are no less
evident to the trained eyes of that smaller, select circle whose
good opinion is chiefly sought. The earlier and cruder method of
advertisement held its ground so long as the public to which the
exhibitor had to appeal comprised large portions of the community who
were not trained to detect delicate variations in the evidences of
wealth and leisure. The method of advertisement undergoes a refinement
when a sufficiently large wealthy class has developed, who have the
leisure for acquiring skill in interpreting the subtler signs of
expenditure. "Loud" dress becomes offensive to people of taste,
as evincing an undue desire to reach and impress the untrained
sensibilities of the vulgar. To the individual of high breeding, it is
only the more honorific esteem accorded by the cultivated sense of the
members of his own high class that is of material consequence. Since
the wealthy leisure class has grown so large, or the contact of the
leisure-class individual with members of his own class has grown so
wide, as to constitute a human environment sufficient for the honorific
purpose, there arises a tendency to exclude the baser elements of
the population from the scheme even as spectators whose applause or
mortification should be sought. The result of all this is a refinement
of methods, a resort to subtler contrivances, and a spiritualization of
the scheme of symbolism in dress. And as this upper leisure class sets
the pace in all matters of decency, the result for the rest of society
also is a gradual amelioration of the scheme of dress. As the community
advances in wealth and culture, the ability to pay is put in evidence
by means which require a progressively nicer discrimination in the
beholder. This nicer discrimination between advertising media is in fact
a very large element of the higher pecuniary culture.




\chapter{Industrial Exemption and Conservatism}

The life of man in society, just like the life of other species, is
a struggle for existence, and therefore it is a process of selective
adaptation. The evolution of social structure has been a process of
natural selection of institutions. The progress which has been and is
being made in human institutions and in human character may be set down,
broadly, to a natural selection of the fittest habits of thought and to
a process of enforced adaptation of individuals to an environment which
has progressively changed with the growth of the community and with the
changing institutions under which men have lived. Institutions are not
only themselves the result of a selective and adaptive process which
shapes the prevailing or dominant types of spiritual attitude and
aptitudes; they are at the same time special methods of life and of
human relations, and are therefore in their turn efficient factors of
selection. So that the changing institutions in their turn make for a
further selection of individuals endowed with the fittest temperament,
and a further adaptation of individual temperament and habits to the
changing environment through the formation of new institutions.

The forces which have shaped the development of human life and of social
structure are no doubt ultimately reducible to terms of living tissue
and material environment; but proximately for the purpose in hand, these
forces may best be stated in terms of an environment, partly human,
partly non-human, and a human subject with a more or less definite
physical and intellectual constitution. Taken in the aggregate or
average, this human subject is more or less variable; chiefly, no doubt,
under a rule of selective conservation of favorable variations.
The selection of favorable variations is perhaps in great measure a
selective conservation of ethnic types. In the life history of any
community whose population is made up of a mixture of divers ethnic
elements, one or another of several persistent and relatively stable
types of body and of temperament rises into dominance at any given
point. The situation, including the institutions in force at any given
time, will favor the survival and dominance of one type of character in
preference to another; and the type of man so selected to continue and
to further elaborate the institutions handed down from the past will in
some considerable measure shape these institutions in his own likeness.
But apart from selection as between relatively stable types of character
and habits of mind, there is no doubt simultaneously going on a process
of selective adaptation of habits of thought within the general range of
aptitudes which is characteristic of the dominant ethnic type or types.
There may be a variation in the fundamental character of any population
by selection between relatively stable types; but there is also a
variation due to adaptation in detail within the range of the type, and
to selection between specific habitual views regarding any given social
relation or group of relations.

For the present purpose, however, the question as to the nature of the
adaptive process--whether it is chiefly a selection between stable types
of temperament and character, or chiefly an adaptation of men's habits
of thought to changing circumstances--is of less importance than the
fact that, by one method or another, institutions change and develop.
Institutions must change with changing circumstances, since they are
of the nature of an habitual method of responding to the stimuli
which these changing circumstances afford. The development of these
institutions is the development of society. The institutions are,
in substance, prevalent habits of thought with respect to particular
relations and particular functions of the individual and of the
community; and the scheme of life, which is made up of the aggregate
of institutions in force at a given time or at a given point in the
development of any society, may, on the psychological side, be broadly
characterized as a prevalent spiritual attitude or a prevalent theory of
life. As regards its generic features, this spiritual attitude or theory
of life is in the last analysis reducible to terms of a prevalent type
of character.

The situation of today shapes the institutions of tomorrow through
a selective, coercive process, by acting upon men's habitual view
of things, and so altering or fortifying a point of view or a mental
attitude handed down from the past. The institutions--that is to say the
habits of thought--under the guidance of which men live are in this way
received from an earlier time; more or less remotely earlier, but in
any event they have been elaborated in and received from the past.
Institutions are products of the past process, are adapted to past
circumstances, and are therefore never in full accord with the
requirements of the present. In the nature of the case, this process of
selective adaptation can never catch up with the progressively changing
situation in which the community finds itself at any given time; for
the environment, the situation, the exigencies of life which enforce the
adaptation and exercise the selection, change from day to day; and each
successive situation of the community in its turn tends to obsolescence
as soon as it has been established. When a step in the development has
been taken, this step itself constitutes a change of situation which
requires a new adaptation; it becomes the point of departure for a new
step in the adjustment, and so on interminably.

It is to be noted then, although it may be a tedious truism, that the
institutions of today--the present accepted scheme of life--do not
entirely fit the situation of today. At the same time, men's present
habits of thought tend to persist indefinitely, except as circumstances
enforce a change. These institutions which have thus been handed down,
these habits of thought, points of view, mental attitudes and aptitudes,
or what not, are therefore themselves a conservative factor. This is the
factor of social inertia, psychological inertia, conservatism. Social
structure changes, develops, adapts itself to an altered situation, only
through a change in the habits of thought of the several classes of the
community, or in the last analysis, through a change in the habits of
thought of the individuals which make up the community. The evolution of
society is substantially a process of mental adaptation on the part
of individuals under the stress of circumstances which will no longer
tolerate habits of thought formed under and conforming to a different
set of circumstances in the past. For the immediate purpose it need not
be a question of serious importance whether this adaptive process is
a process of selection and survival of persistent ethnic types or a
process of individual adaptation and an inheritance of acquired traits.

Social advance, especially as seen from the point of view of economic
theory, consists in a continued progressive approach to an approximately
exact "adjustment of inner relations to outer relations", but this
adjustment is never definitively established, since the "outer
relations" are subject to constant change as a consequence of the
progressive change going on in the "inner relations." But the degree
of approximation may be greater or less, depending on the facility with
which an adjustment is made. A readjustment of men's habits of thought
to conform with the exigencies of an altered situation is in any case
made only tardily and reluctantly, and only under the coercion exercised
by a stipulation which has made the accredited views untenable.
The readjustment of institutions and habitual views to an altered
environment is made in response to pressure from without; it is of the
nature of a response to stimulus. Freedom and facility of readjustment,
that is to say capacity for growth in social structure, therefore
depends in great measure on the degree of freedom with which the
situation at any given time acts on the individual members of the
community-the degree of exposure of the individual members to the
constraining forces of the environment. If any portion or class of
society is sheltered from the action of the environment in any essential
respect, that portion of the community, or that class, will adapt
its views and its scheme of life more tardily to the altered general
situation; it will in so far tend to retard the process of social
transformation. The wealthy leisure class is in such a sheltered
position with respect to the economic forces that make for change
and readjustment. And it may be said that the forces which make for
a readjustment of institutions, especially in the case of a modern
industrial community, are, in the last analysis, almost entirely of an
economic nature.

Any community may be viewed as an industrial or economic mechanism,
the structure of which is made up of what is called its economic
institutions. These institutions are habitual methods of carrying on the
life process of the community in contact with the material environment
in which it lives. When given methods of unfolding human activity in
this given environment have been elaborated in this way, the life of
the community will express itself with some facility in these habitual
directions. The community will make use of the forces of the environment
for the purposes of its life according to methods learned in the past
and embodied in these institutions. But as population increases, and as
men's knowledge and skill in directing the forces of nature widen, the
habitual methods of relation between the members of the group, and the
habitual method of carrying on the life process of the group as a
whole, no longer give the same result as before; nor are the resulting
conditions of life distributed and apportioned in the same manner or
with the same effect among the various members as before. If the scheme
according to which the life process of the group was carried on under
the earlier conditions gave approximately the highest attainable
result--under the circumstances--in the way of efficiency or facility
of the life process of the group; then the same scheme of life unaltered
will not yield the highest result attainable in this respect under the
altered conditions. Under the altered conditions of population, skill,
and knowledge, the facility of life as carried on according to the
traditional scheme may not be lower than under the earlier conditions;
but the chances are always that it is less than might be if the scheme
were altered to suit the altered conditions.

The group is made up of individuals, and the group's life is the life
of individuals carried on in at least ostensible severalty. The group's
accepted scheme of life is the consensus of views held by the body of
these individuals as to what is right, good, expedient, and beautiful in
the way of human life. In the redistribution of the conditions of life
that comes of the altered method of dealing with the environment, the
outcome is not an equable change in the facility of life throughout the
group. The altered conditions may increase the facility of life for
the group as a whole, but the redistribution will usually result in a
decrease of facility or fullness of life for some members of the
group. An advance in technical methods, in population, or in industrial
organization will require at least some of the members of the community
to change their habits of life, if they are to enter with facility and
effect into the altered industrial methods; and in doing so they will be
unable to live up to the received notions as to what are the right and
beautiful habits of life.

Any one who is required to change his habits of life and his habitual
relations to his fellow men will feel the discrepancy between the
method of life required of him by the newly arisen exigencies, and
the traditional scheme of life to which he is accustomed. It is the
individuals placed in this position who have the liveliest incentive to
reconstruct the received scheme of life and are most readily persuaded
to accept new standards; and it is through the need of the means of
livelihood that men are placed in such a position. The pressure exerted
by the environment upon the group, and making for a readjustment of the
group's scheme of life, impinges upon the members of the group in
the form of pecuniary exigencies; and it is owing to this fact--that
external forces are in great part translated into the form of pecuniary
or economic exigencies--it is owing to this fact that we can say that
the forces which count toward a readjustment of institutions in any
modern industrial community are chiefly economic forces; or more
specifically, these forces take the form of pecuniary pressure. Such a
readjustment as is here contemplated is substantially a change in men's
views as to what is good and right, and the means through which a change
is wrought in men's apprehension of what is good and right is in large
part the pressure of pecuniary exigencies.

Any change in men's views as to what is good and right in human life
make its way but tardily at the best. Especially is this true of any
change in the direction of what is called progress; that is to say, in
the direction of divergence from the archaic position--from the position
which may be accounted the point of departure at any step in the social
evolution of the community. Retrogression, reapproach to a standpoint to
which the race has been long habituated in the past, is easier. This is
especially true in case the development away from this past standpoint
has not been due chiefly to a substitution of an ethnic type whose
temperament is alien to the earlier standpoint. The cultural stage which
lies immediately back of the present in the life history of Western
civilization is what has here been called the quasi-peaceable stage. At
this quasi-peaceable stage the law of status is the dominant feature in
the scheme of life. There is no need of pointing out how prone the
men of today are to revert to the spiritual attitude of mastery and of
personal subservience which characterizes that stage. It may rather be
said to be held in an uncertain abeyance by the economic exigencies of
today, than to have been definitely supplanted by a habit of mind that
is in full accord with these later-developed exigencies. The predatory
and quasi-peaceable stages of economic evolution seem to have been of
long duration in life history of all the chief ethnic elements which go
to make up the populations of the Western culture. The temperament
and the propensities proper to those cultural stages have, therefore,
attained such a persistence as to make a speedy reversion to the broad
features of the corresponding psychological constitution inevitable in
the case of any class or community which is removed from the action of
those forces that make for a maintenance of the later-developed habits
of thought.

It is a matter of common notoriety that when individuals, or even
considerable groups of men, are segregated from a higher industrial
culture and exposed to a lower cultural environment, or to an economic
situation of a more primitive character, they quickly show evidence of
reversion toward the spiritual features which characterize the predatory
type; and it seems probable that the dolicho-blond type of European man
is possessed of a greater facility for such reversion to barbarism than
the other ethnic elements with which that type is associated in the
Western culture. Examples of such a reversion on a small scale abound in
the later history of migration and colonization. Except for the fear
of offending that chauvinistic patriotism which is so characteristic
a feature of the predatory culture, and the presence of which is
frequently the most striking mark of reversion in modern communities,
the case of the American colonies might be cited as an example of such a
reversion on an unusually large scale, though it was not a reversion of
very large scope.

The leisure class is in great measure sheltered from the stress of
those economic exigencies which prevail in any modern, highly organized
industrial community. The exigencies of the struggle for the means
of life are less exacting for this class than for any other; and as a
consequence of this privileged position we should expect to find it one
of the least responsive of the classes of society to the demands
which the situation makes for a further growth of institutions and a
readjustment to an altered industrial situation. The leisure class is
the conservative class. The exigencies of the general economic situation
of the community do not freely or directly impinge upon the members of
this class. They are not required under penalty of forfeiture to change
their habits of life and their theoretical views of the external world
to suit the demands of an altered industrial technique, since they
are not in the full sense an organic part of the industrial community.
Therefore these exigencies do not readily produce, in the members of
this class, that degree of uneasiness with the existing order which
alone can lead any body of men to give up views and methods of life that
have become habitual to them. The office of the leisure class in social
evolution is to retard the movement and to conserve what is obsolescent.
This proposition is by no means novel; it has long been one of the
commonplaces of popular opinion.

The prevalent conviction that the wealthy class is by nature
conservative has been popularly accepted without much aid from any
theoretical view as to the place and relation of that class in the
cultural development. When an explanation of this class conservatism is
offered, it is commonly the invidious one that the wealthy class opposes
innovation because it has a vested interest, of an unworthy sort, in
maintaining the present conditions. The explanation here put forward
imputes no unworthy motive. The opposition of the class to changes in
the cultural scheme is instinctive, and does not rest primarily on an
interested calculation of material advantages; it is an instinctive
revulsion at any departure from the accepted way of doing and of looking
at things--a revulsion common to all men and only to be overcome by
stress of circumstances. All change in habits of life and of thought
is irksome. The difference in this respect between the wealthy and the
common run of mankind lies not so much in the motive which prompts to
conservatism as in the degree of exposure to the economic forces that
urge a change. The members of the wealthy class do not yield to the
demand for innovation as readily as other men because they are not
constrained to do so.

This conservatism of the wealthy class is so obvious a feature that
it has even come to be recognized as a mark of respectability. Since
conservatism is a characteristic of the wealthier and therefore more
reputable portion of the community, it has acquired a certain honorific
or decorative value. It has become prescriptive to such an extent that
an adherence to conservative views is comprised as a matter of course in
our notions of respectability; and it is imperatively incumbent on all
who would lead a blameless life in point of social repute. Conservatism,
being an upper-class characteristic, is decorous; and conversely,
innovation, being a lower-class phenomenon, is vulgar. The first and
most unreflected element in that instinctive revulsion and reprobation
with which we turn from all social innovators is this sense of the
essential vulgarity of the thing. So that even in cases where one
recognizes the substantial merits of the case for which the innovator
is spokesman--as may easily happen if the evils which he seeks to
remedy are sufficiently remote in point of time or space or personal
contact--still one cannot but be sensible of the fact that the innovator
is a person with whom it is at least distasteful to be associated, and
from whose social contact one must shrink. Innovation is bad form.

The fact that the usages, actions, and views of the well-to-do leisure
class acquire the character of a prescriptive canon of conduct for
the rest of society, gives added weight and reach to the conservative
influence of that class. It makes it incumbent upon all reputable people
to follow their lead. So that, by virtue of its high position as the
avatar of good form, the wealthier class comes to exert a retarding
influence upon social development far in excess of that which the
simple numerical strength of the class would assign it. Its prescriptive
example acts to greatly stiffen the resistance of all other classes
against any innovation, and to fix men's affections upon the good
institutions handed down from an earlier generation. There is a second
way in which the influence of the leisure class acts in the same
direction, so far as concerns hindrance to the adoption of a
conventional scheme of life more in accord with the exigencies of
the time. This second method of upper-class guidance is not in strict
consistency to be brought under the same category as the instinctive
conservatism and aversion to new modes of thought just spoken of; but
it may as well be dealt with here, since it has at least this much
in common with the conservative habit of mind that it acts to retard
innovation and the growth of social structure. The code of proprieties,
conventionalities, and usages in vogue at any given time and among any
given people has more or less of the character of an organic whole;
so that any appreciable change in one point of the scheme involves
something of a change or readjustment at other points also, if not
a reorganization all along the line. When a change is made which
immediately touches only a minor point in the scheme, the consequent
derangement of the structure of conventionalities may be inconspicuous;
but even in such a case it is safe to say that some derangement of the
general scheme, more or less far-reaching, will follow. On the
other hand, when an attempted reform involves the suppression or
thorough-going remodelling of an institution of first-rate importance
in the conventional scheme, it is immediately felt that a serious
derangement of the entire scheme would result; it is felt that a
readjustment of the structure to the new form taken on by one of
its chief elements would be a painful and tedious, if not a doubtful
process.

In order to realize the difficulty which such a radical change in any
one feature of the conventional scheme of life would involve, it is only
necessary to suggest the suppression of the monogamic family, or of
the agnatic system of consanguinity, or of private property, or of the
theistic faith, in any country of the Western civilization; or suppose
the suppression of ancestor worship in China, or of the caste system in
india, or of slavery in Africa, or the establishment of equality of the
sexes in Mohammedan countries. It needs no argument to show that the
derangement of the general structure of conventionalities in any of
these cases would be very considerable. In order to effect such an
innovation a very far-reaching alteration of men's habits of thought
would be involved also at other points of the scheme than the one
immediately in question. The aversion to any such innovation amounts to
a shrinking from an essentially alien scheme of life.

The revulsion felt by good people at any proposed departure from the
accepted methods of life is a familiar fact of everyday experience. It
is not unusual to hear those persons who dispense salutary advice
and admonition to the community express themselves forcibly upon the
far-reaching pernicious effects which the community would suffer from
such relatively slight changes as the disestablishment of the Anglican
Church, an increased facility of divorce, adoption of female suffrage,
prohibition of the manufacture and sale of intoxicating beverages,
abolition or restriction of inheritances, etc. Any one of these
innovations would, we are told, "shake the social structure to its
base," "reduce society to chaos," "subvert the foundations of morality,"
"make life intolerable," "confound the order of nature," etc. These
various locutions are, no doubt, of the nature of hyperbole; but, at the
same time, like all overstatement, they are evidence of a lively sense
of the gravity of the consequences which they are intended to describe.
The effect of these and like innovations in deranging the accepted
scheme of life is felt to be of much graver consequence than the simple
alteration of an isolated item in a series of contrivances for the
convenience of men in society. What is true in so obvious a degree of
innovations of first-rate importance is true in a less degree of changes
of a smaller immediate importance. The aversion to change is in large
part an aversion to the bother of making the readjustment which any
given change will necessitate; and this solidarity of the system of
institutions of any given culture or of any given people strengthens the
instinctive resistance offered to any change in men's habits of thought,
even in matters which, taken by themselves, are of minor importance. A
consequence of this increased reluctance, due to the solidarity of human
institutions, is that any innovation calls for a greater expenditure of
nervous energy in making the necessary readjustment than would otherwise
be the case. It is not only that a change in established habits of
thought is distasteful. The process of readjustment of the accepted
theory of life involves a degree of mental effort--a more or less
protracted and laborious effort to find and to keep one's bearings under
the altered circumstances. This process requires a certain expenditure
of energy, and so presumes, for its successful accomplishment, some
surplus of energy beyond that absorbed in the daily struggle for
subsistence. Consequently it follows that progress is hindered by
underfeeding and excessive physical hardship, no less effectually than
by such a luxurious life as will shut out discontent by cutting off the
occasion for it. The abjectly poor, and all those persons whose
energies are entirely absorbed by the struggle for daily sustenance, are
conservative because they cannot afford the effort of taking thought for
the day after tomorrow; just as the highly prosperous are conservative
because they have small occasion to be discontented with the situation
as it stands today.

From this proposition it follows that the institution of a leisure class
acts to make the lower classes conservative by withdrawing from them
as much as it may of the means of sustenance, and so reducing their
consumption, and consequently their available energy, to such a point
as to make them incapable of the effort required for the learning and
adoption of new habits of thought. The accumulation of wealth at the
upper end of the pecuniary scale implies privation at the lower end of
the scale. It is a commonplace that, wherever it occurs, a considerable
degree of privation among the body of the people is a serious obstacle
to any innovation.

This direct inhibitory effect of the unequal distribution of wealth
is seconded by an indirect effect tending to the same result. As has
already been seen, the imperative example set by the upper class in
fixing the canons of reputability fosters the practice of conspicuous
consumption. The prevalence of conspicuous consumption as one of the
main elements in the standard of decency among all classes is of course
not traceable wholly to the example of the wealthy leisure class, but
the practice and the insistence on it are no doubt strengthened by the
example of the leisure class. The requirements of decency in this matter
are very considerable and very imperative; so that even among classes
whose pecuniary position is sufficiently strong to admit a consumption
of goods considerably in excess of the subsistence minimum, the
disposable surplus left over after the more imperative physical
needs are satisfied is not infrequently diverted to the purpose of a
conspicuous decency, rather than to added physical comfort and fullness
of life. Moreover, such surplus energy as is available is also likely to
be expended in the acquisition of goods for conspicuous consumption or
conspicuous boarding. The result is that the requirements of pecuniary
reputability tend (1) to leave but a scanty subsistence minimum
available for other than conspicuous consumption, and (2) to absorb
any surplus energy which may be available after the bare physical
necessities of life have been provided for. The outcome of the whole is
a strengthening of the general conservative attitude of the community.
The institution of a leisure class hinders cultural development
immediately (1) by the inertia proper to the class itself, (2) through
its prescriptive example of conspicuous waste and of conservatism, and
(3) indirectly through that system of unequal distribution of wealth and
sustenance on which the institution itself rests. To this is to be added
that the leisure class has also a material interest in leaving things
as they are. Under the circumstances prevailing at any given time this
class is in a privileged position, and any departure from the existing
order may be expected to work to the detriment of the class rather than
the reverse. The attitude of the class, simply as influenced by its
class interest, should therefore be to let well-enough alone. This
interested motive comes in to supplement the strong instinctive bias of
the class, and so to render it even more consistently conservative than
it otherwise would be.

All this, of course, has nothing to say in the way of eulogy or
deprecation of the office of the leisure class as an exponent and
vehicle of conservatism or reversion in social structure. The inhibition
which it exercises may be salutary or the reverse. Wether it is the one
or the other in any given case is a question of casuistry rather than of
general theory. There may be truth in the view (as a question of policy)
so often expressed by the spokesmen of the conservative element, that
without some such substantial and consistent resistance to innovation as
is offered by the conservative well-to-do classes, social innovation
and experiment would hurry the community into untenable and intolerable
situations; the only possible result of which would be discontent and
disastrous reaction. All this, however, is beside the present argument.

But apart from all deprecation, and aside from all question as to the
indispensability of some such check on headlong innovation, the leisure
class, in the nature of things, consistently acts to retard that
adjustment to the environment which is called social advance or
development. The characteristic attitude of the class may be summed
up in the maxim: "Whatever is, is right" whereas the law of natural
selection, as applied to human institutions, gives the axiom: "Whatever
is, is wrong." Not that the institutions of today are wholly wrong
for the purposes of the life of today, but they are, always and in the
nature of things, wrong to some extent. They are the result of a more or
less inadequate adjustment of the methods of living to a situation which
prevailed at some point in the past development; and they are therefore
wrong by something more than the interval which separates the present
situation from that of the past. "Right" and "wrong" are of course here
used without conveying any rejection as to what ought or ought not to
be. They are applied simply from the (morally colorless) evolutionary
standpoint, and are intended to designate compatibility or
incompatibility with the effective evolutionary process. The institution
of a leisure class, by force or class interest and instinct, and by
precept and prescriptive example, makes for the perpetuation of the
existing maladjustment of institutions, and even favors a reversion to
a somewhat more archaic scheme of life; a scheme which would be still
farther out of adjustment with the exigencies of life under the existing
situation even than the accredited, obsolescent scheme that has come
down from the immediate past.

But after all has been said on the head of conservation of the good old
ways, it remains true that institutions change and develop. There is
a cumulative growth of customs and habits of thought; a selective
adaptation of conventions and methods of life. Something is to be said
of the office of the leisure class in guiding this growth as well as
in retarding it; but little can be said here of its relation to
institutional growth except as it touches the institutions that
are primarily and immediately of an economic character. These
institutions--the economic structure--may be roughly distinguished into
two classes or categories, according as they serve one or the other of
two divergent purposes of economic life.

To adapt the classical terminology, they are institutions of acquisition
or of production; or to revert to terms already employed in a different
connection in earlier chapters, they are pecuniary or industrial
institutions; or in still other terms, they are institutions serving
either the invidious or the non-invidious economic interest. The former
category have to do with "business," the latter with industry, taking
the latter word in the mechanical sense. The latter class are not
often recognized as institutions, in great part because they do not
immediately concern the ruling class, and are, therefore, seldom the
subject of legislation or of deliberate convention. When they do receive
attention they are commonly approached from the pecuniary or business
side; that being the side or phase of economic life that chiefly
occupies men's deliberations in our time, especially the deliberations
of the upper classes. These classes have little else than a business
interest in things economic, and on them at the same time it is chiefly
incumbent to deliberate upon the community's affairs.

The relation of the leisure (that is, propertied non-industrial)
class to the economic process is a pecuniary relation--a relation of
acquisition, not of production; of exploitation, not of serviceability.
Indirectly their economic office may, of course, be of the utmost
importance to the economic life process; and it is by no means here
intended to depreciate the economic function of the propertied class or
of the captains of industry. The purpose is simply to point out what is
the nature of the relation of these classes to the industrial process
and to economic institutions. Their office is of a parasitic character,
and their interest is to divert what substance they may to their own
use, and to retain whatever is under their hand. The conventions of the
business world have grown up under the selective surveillance of this
principle of predation or parasitism. They are conventions of ownership;
derivatives, more or less remote, of the ancient predatory culture. But
these pecuniary institutions do not entirely fit the situation of today,
for they have grown up under a past situation differing somewhat from
the present. Even for effectiveness in the pecuniary way, therefore,
they are not as apt as might be. The changed industrial life requires
changed methods of acquisition; and the pecuniary classes have some
interest in so adapting the pecuniary institutions as to give them the
best effect for acquisition of private gain that is compatible with the
continuance of the industrial process out of which this gain arises.
Hence there is a more or less consistent trend in the leisure-class
guidance of institutional growth, answering to the pecuniary ends which
shape leisure-class economic life.

The effect of the pecuniary interest and the pecuniary habit of
mind upon the growth of institutions is seen in those enactments
and conventions that make for security of property, enforcement of
contracts, facility of pecuniary transactions, vested interests. Of
such bearing are changes affecting bankruptcy and receiverships, limited
liability, banking and currency, coalitions of laborers or employers,
trusts and pools. The community's institutional furniture of this kind
is of immediate consequence only to the propertied classes, and in
proportion as they are propertied; that is to say, in proportion as
they are to be ranked with the leisure class. But indirectly these
conventions of business life are of the gravest consequence for the
industrial process and for the life of the community. And in guiding the
institutional growth in this respect, the pecuniary classes, therefore,
serve a purpose of the most serious importance to the community, not
only in the conservation of the accepted social scheme, but also
in shaping the industrial process proper. The immediate end of this
pecuniary institutional structure and of its amelioration is the greater
facility of peaceable and orderly exploitation; but its remoter effects
far outrun this immediate object. Not only does the more facile conduct
of business permit industry and extra-industrial life to go on with
less perturbation; but the resulting elimination of disturbances and
complications calling for an exercise of astute discrimination in
everyday affairs acts to make the pecuniary class itself superfluous.
As fast as pecuniary transactions are reduced to routine, the captain
of industry can be dispensed with. This consummation, it is needless
to say, lies yet in the indefinite future. The ameliorations wrought in
favor of the pecuniary interest in modern institutions tend, in another
field, to substitute the "soulless" joint-stock corporation for the
captain, and so they make also for the dispensability, of the great
leisure-class function of ownership. Indirectly, therefore, the bent
given to the growth of economic institutions by the leisure-class
influence is of very considerable industrial consequence.




\chapter{The Conservation of Archaic Traits}
The institution of a leisure class has an effect not only upon social
structure but also upon the individual character of the members of
society. So soon as a given proclivity or a given point of view has won
acceptance as an authoritative standard or norm of life it will react
upon the character of the members of the society which has accepted it
as a norm. It will to some extent shape their habits of thought and
will exercise a selective surveillance over the development of men's
aptitudes and inclinations. This effect is wrought partly by a coercive,
educational adaptation of the habits of all individuals, partly by a
selective elimination of the unfit individuals and lines of descent.
Such human material as does not lend itself to the methods of life
imposed by the accepted scheme suffers more or less elimination as well
as repression. The principles of pecuniary emulation and of industrial
exemption have in this way been erected into canons of life, and have
become coercive factors of some importance in the situation to which men
have to adapt themselves.

These two broad principles of conspicuous waste and industrial exemption
affect the cultural development both by guiding men's habits of thought,
and so controlling the growth of institutions, and by selectively
conserving certain traits of human nature that conduce to facility of
life under the leisure-class scheme, and so controlling the effective
temper of the community. The proximate tendency of the institution of
a leisure class in shaping human character runs in the direction of
spiritual survival and reversion. Its effect upon the temper of a
community is of the nature of an arrested spiritual development. In
the later culture especially, the institution has, on the whole, a
conservative trend. This proposition is familiar enough in substance,
but it may to many have the appearance of novelty in its present
application. Therefore a summary review of its logical grounds may
not be uncalled for, even at the risk of some tedious repetition and
formulation of commonplaces.

Social evolution is a process of selective adaptation of temperament and
habits of thought under the stress of the circumstances of associated
life. The adaptation of habits of thought is the growth of institutions.
But along with the growth of institutions has gone a change of a more
substantial character. Not only have the habits of men changed with the
changing exigencies of the situation, but these changing exigencies
have also brought about a correlative change in human nature. The human
material of society itself varies with the changing conditions of life.
This variation of human nature is held by the later ethnologists to be
a process of selection between several relatively stable and persistent
ethnic types or ethnic elements. Men tend to revert or to breed true,
more or less closely, to one or another of certain types of human nature
that have in their main features been fixed in approximate conformity
to a situation in the past which differed from the situation of today.
There are several of these relatively stable ethnic types of mankind
comprised in the populations of the Western culture. These ethnic types
survive in the race inheritance today, not as rigid and invariable
moulds, each of a single precise and specific pattern, but in the form
of a greater or smaller number of variants. Some variation of the ethnic
types has resulted under the protracted selective process to which
the several types and their hybrids have been subjected during the
prehistoric and historic growth of culture.

This necessary variation of the types themselves, due to a selective
process of considerable duration and of a consistent trend, has not been
sufficiently noticed by the writers who have discussed ethnic survival.
The argument is here concerned with two main divergent variants of human
nature resulting from this, relatively late, selective adaptation of
the ethnic types comprised in the Western culture; the point of interest
being the probable effect of the situation of today in furthering
variation along one or the other of these two divergent lines.

The ethnological position may be briefly summed up; and in order to
avoid any but the most indispensable detail the schedule of types and
variants and the scheme of reversion and survival in which they
are concerned are here presented with a diagrammatic meagerness and
simplicity which would not be admissible for any other purpose. The man
of our industrial communities tends to breed true to one or the other
of three main ethic types; the dolichocephalic-blond, the
brachycephalic-brunette, and the Mediterranean--disregarding minor and
outlying elements of our culture. But within each of these main ethnic
types the reversion tends to one or the other of at least two main
directions of variation; the peaceable or antepredatory variant and the
predatory variant. The former of these two characteristic variants
is nearer to the generic type in each case, being the reversional
representative of its type as it stood at the earliest stage
of associated life of which there is available evidence, either
archaeological or psychological. This variant is taken to represent the
ancestors of existing civilized man at the peaceable, savage phase of
life which preceded the predatory culture, the regime of status, and the
growth of pecuniary emulation. The second or predatory variant of the
types is taken to be a survival of a more recent modification of
the main ethnic types and their hybrids--of these types as they were
modified, mainly by a selective adaptation, under the discipline of
the predatory culture and the latter emulative culture of the
quasi-peaceable stage, or the pecuniary culture proper.

Under the recognized laws of heredity there may be a survival from a
more or less remote past phase. In the ordinary, average, or normal
case, if the type has varied, the traits of the type are transmitted
approximately as they have stood in the recent past--which may be called
the hereditary present. For the purpose in hand this hereditary present
is represented by the later predatory and the quasi-peaceable culture.

It is to the variant of human nature which is characteristic of this
recent--hereditarily still existing--predatory or quasi-predatory
culture that the modern civilized man tends to breed true in the common
run of cases. This proposition requires some qualification so far
as concerns the descendants of the servile or repressed classes of
barbarian times, but the qualification necessary is probably not so
great as might at first thought appear. Taking the population as a
whole, this predatory, emulative variant does not seem to have attained
a high degree of consistency or stability. That is to say, the human
nature inherited by modern Occidental man is not nearly uniform in
respect of the range or the relative strength of the various aptitudes
and propensities which go to make it up. The man of the hereditary
present is slightly archaic as judged for the purposes of the latest
exigencies of associated life. And the type to which the modern man
chiefly tends to revert under the law of variation is a somewhat more
archaic human nature. On the other hand, to judge by the reversional
traits which show themselves in individuals that vary from the
prevailing predatory style of temperament, the ante-predatory
variant seems to have a greater stability and greater symmetry in the
distribution or relative force of its temperamental elements.

This divergence of inherited human nature, as between an earlier and a
later variant of the ethnic type to which the individual tends to breed
true, is traversed and obscured by a similar divergence between the
two or three main ethnic types that go to make up the Occidental
populations. The individuals in these communities are conceived to be,
in virtually every instance, hybrids of the prevailing ethnic elements
combined in the most varied proportions; with the result that they tend
to take back to one or the other of the component ethnic types. These
ethnic types differ in temperament in a way somewhat similar to the
difference between the predatory and the antepredatory variants of the
types; the dolicho-blond type showing more of the characteristics of the
predatory temperament--or at least more of the violent disposition--than
the brachycephalic-brunette type, and especially more than the
Mediterranean. When the growth of institutions or of the effective
sentiment of a given community shows a divergence from the predatory
human nature, therefore, it is impossible to say with certainty that
such a divergence indicates a reversion to the ante-predatory variant.
It may be due to an increasing dominance of the one or the other of the
"lower" ethnic elements in the population. Still, although the evidence
is not as conclusive as might be desired, there are indications that
the variations in the effective temperament of modern communities is not
altogether due to a selection between stable ethnic types. It seems to
be to some appreciable extent a selection between the predatory and the
peaceable variants of the several types. This conception of contemporary
human evolution is not indispensable to the discussion. The general
conclusions reached by the use of these concepts of selective
adaptation would remain substantially true if the earlier, Darwinian
and Spencerian, terms and concepts were substituted. Under the
circumstances, some latitude may be admissible in the use of terms. The
word "type" is used loosely, to denote variations of temperament which
the ethnologists would perhaps recognize only as trivial variants of
the type rather than as distinct ethnic types. Wherever a closer
discrimination seems essential to the argument, the effort to make such
a closer discrimination will be evident from the context.

The ethnic types of today, then, are variants of the primitive racial
types. They have suffered some alteration, and have attained some degree
of fixity in their altered form, under the discipline of the barbarian
culture. The man of the hereditary present is the barbarian variant,
servile or aristocratic, of the ethnic elements that constitute him.
But this barbarian variant has not attained the highest degree of
homogeneity or of stability. The barbarian culture--the predatory and
quasi-peaceable cultural stages--though of great absolute duration, has
been neither protracted enough nor invariable enough in character to
give an extreme fixity of type. Variations from the barbarian human
nature occur with some frequency, and these cases of variation are
becoming more noticeable today, because the conditions of modern life no
longer act consistently to repress departures from the barbarian normal.
The predatory temperament does not lead itself to all the purposes of
modern life, and more especially not to modern industry.

Departures from the human nature of the hereditary present are most
frequently of the nature of reversions to an earlier variant of the
type. This earlier variant is represented by the temperament
which characterizes the primitive phase of peaceable savagery. The
circumstances of life and the ends of effort that prevailed before the
advent of the barbarian culture, shaped human nature and fixed it as
regards certain fundamental traits. And it is to these ancient, generic
features that modern men are prone to take back in case of variation
from the human nature of the hereditary present. The conditions under
which men lived in the most primitive stages of associated life that can
properly be called human, seem to have been of a peaceful kind; and the
character--the temperament and spiritual attitude of men under these
early conditions or environment and institutions seems to have been of
a peaceful and unaggressive, not to say an indolent, cast. For the
immediate purpose this peaceable cultural stage may be taken to mark
the initial phase of social development. So far as concerns the present
argument, the dominant spiritual feature of this presumptive initial
phase of culture seems to have been an unreflecting, unformulated sense
of group solidarity, largely expressing itself in a complacent, but by
no means strenuous, sympathy with all facility of human life, and an
uneasy revulsion against apprehended inhibition or futility of life.
Through its ubiquitous presence in the habits of thought of the
ante-predatory savage man, this pervading but uneager sense of the
generically useful seems to have exercised an appreciable constraining
force upon his life and upon the manner of his habitual contact with
other members of the group.

The traces of this initial, undifferentiated peaceable phase of culture
seem faint and doubtful if we look merely to such categorical evidence
of its existence as is afforded by usages and views in vogue within the
historical present, whether in civilized or in rude communities; but
less dubious evidence of its existence is to be found in psychological
survivals, in the way of persistent and pervading traits of human
character. These traits survive perhaps in an especial degree among
those ethic elements which were crowded into the background during the
predatory culture. Traits that were suited to the earlier habits of life
then became relatively useless in the individual struggle for existence.
And those elements of the population, or those ethnic groups, which
were by temperament less fitted to the predatory life were repressed and
pushed into the background. On the transition to the predatory culture
the character of the struggle for existence changed in some degree from
a struggle of the group against a non-human environment to a struggle
against a human environment. This change was accompanied by an
increasing antagonism and consciousness of antagonism between the
individual members of the group. The conditions of success within the
group, as well as the conditions of the survival of the group, changed
in some measure; and the dominant spiritual attitude for the group
gradually changed, and brought a different range of aptitudes and
propensities into the position of legitimate dominance in the accepted
scheme of life. Among these archaic traits that are to be regarded as
survivals from the peaceable cultural phase, are that instinct of race
solidarity which we call conscience, including the sense of truthfulness
and equity, and the instinct of workmanship, in its naive, non-invidious
expression.

Under the guidance of the later biological and psychological science,
human nature will have to be restated in terms of habit; and in the
restatement, this, in outline, appears to be the only assignable place
and ground of these traits. These habits of life are of too pervading a
character to be ascribed to the influence of a late or brief discipline.
The ease with which they are temporarily overborne by the special
exigencies of recent and modern life argues that these habits are the
surviving effects of a discipline of extremely ancient date, from the
teachings of which men have frequently been constrained to depart in
detail under the altered circumstances of a later time; and the almost
ubiquitous fashion in which they assert themselves whenever the pressure
of special exigencies is relieved, argues that the process by which the
traits were fixed and incorporated into the spiritual make-up of the
type must have lasted for a relatively very long time and without
serious intermission. The point is not seriously affected by any
question as to whether it was a process of habituation in the
old-fashioned sense of the word or a process of selective adaptation of
the race.

The character and exigencies of life, under that regime of status and
of individual and class antithesis which covers the entire interval from
the beginning of predatory culture to the present, argue that the traits
of temperament here under discussion could scarcely have arisen and
acquired fixity during that interval. It is entirely probable that these
traits have come down from an earlier method of life, and have survived
through the interval of predatory and quasi-peaceable culture in a
condition of incipient, or at least imminent, desuetude, rather than
that they have been brought out and fixed by this later culture.
They appear to be hereditary characteristics of the race, and to have
persisted in spite of the altered requirements of success under the
predatory and the later pecuniary stages of culture. They seem to have
persisted by force of the tenacity of transmission that belongs to an
hereditary trait that is present in some degree in every member of the
species, and which therefore rests on a broad basis of race continuity.

Such a generic feature is not readily eliminated, even under a process
of selection so severe and protracted as that to which the traits here
under discussion were subjected during the predatory and quasi-peaceable
stages. These peaceable traits are in great part alien to the methods
and the animus of barbarian life. The salient characteristic of the
barbarian culture is an unremitting emulation and antagonism between
classes and between individuals. This emulative discipline favors those
individuals and lines of descent which possess the peaceable savage
traits in a relatively slight degree. It therefore tends to eliminate
these traits, and it has apparently weakened them, in an appreciable
degree, in the populations that have been subject to it. Even where the
extreme penalty for non-conformity to the barbarian type of temperament
is not paid, there results at least a more or less consistent repression
of the non-conforming individuals and lines of descent. Where life is
largely a struggle between individuals within the group, the possession
of the ancient peaceable traits in a marked degree would hamper an
individual in the struggle for life.

Under any known phase of culture, other or later than the presumptive
initial phase here spoken of, the gifts of good-nature, equity, and
indiscriminate sympathy do not appreciably further the life of the
individual. Their possession may serve to protect the individual from
hard usage at the hands of a majority that insists on a modicum of
these ingredients in their ideal of a normal man; but apart from their
indirect and negative effect in this way, the individual fares better
under the regime of competition in proportion as he has less of these
gifts. Freedom from scruple, from sympathy, honesty and regard for life,
may, within fairly wide limits, be said to further the success of the
individual in the pecuniary culture. The highly successful men of all
times have commonly been of this type; except those whose success has
not been scored in terms of either wealth or power. It is only within
narrow limits, and then only in a Pickwickian sense, that honesty is the
best policy.

As seen from the point of view of life under modern civilized conditions
in an enlightened community of the Western culture, the primitive,
ante-predatory savage, whose character it has been attempted to trace
in outline above, was not a great success. Even for the purposes of
that hypothetical culture to which his type of human nature owes what
stability it has--even for the ends of the peaceable savage group--this
primitive man has quite as many and as conspicuous economic failings as
he has economic virtues--as should be plain to any one whose sense of
the case is not biased by leniency born of a fellow-feeling. At his
best he is "a clever, good-for-nothing fellow." The shortcomings of this
presumptively primitive type of character are weakness, inefficiency,
lack of initiative and ingenuity, and a yielding and indolent
amiability, together with a lively but inconsequential animistic sense.
Along with these traits go certain others which have some value for the
collective life process, in the sense that they further the facility
of life in the group. These traits are truthfulness, peaceableness,
good-will, and a non-emulative, non-invidious interest in men and
things.

With the advent of the predatory stage of life there comes a change in
the requirements of the successful human character. Men's habits of life
are required to adapt themselves to new exigencies under a new scheme
of human relations. The same unfolding of energy, which had previously
found expression in the traits of savage life recited above, is now
required to find expression along a new line of action, in a new group
of habitual responses to altered stimuli. The methods which, as counted
in terms of facility of life, answered measurably under the earlier
conditions, are no longer adequate under the new conditions. The earlier
situation was characterized by a relative absence of antagonism or
differentiation of interests, the later situation by an emulation
constantly increasing in relative absence of antagonism or
differentiation of interests, the later situation by an emulation
constantly increasing in intensity and narrowing in scope. The traits
which characterize the predatory and subsequent stages of culture, and
which indicate the types of man best fitted to survive under the regime
of status, are (in their primary expression) ferocity, self-seeking,
clannishness, and disingenuousness--a free resort to force and fraud.

Under the severe and protracted discipline of the regime of competition,
the selection of ethnic types has acted to give a somewhat pronounced
dominance to these traits of character, by favoring the survival of
those ethnic elements which are most richly endowed in these respects.
At the same time the earlier--acquired, more generic habits of the race
have never ceased to have some usefulness for the purpose of the life of
the collectivity and have never fallen into definitive abeyance. It may
be worth while to point out that the dolicho-blond type of European man
seems to owe much of its dominating influence and its masterful position
in the recent culture to its possessing the characteristics of predatory
man in an exceptional degree. These spiritual traits, together with
a large endowment of physical energy--itself probably a result of
selection between groups and between lines of descent--chiefly go to
place any ethnic element in the position of a leisure or master
class, especially during the earlier phases of the development of the
institution of a leisure class. This need not mean that precisely the
same complement of aptitudes in any individual would insure him an
eminent personal success. Under the competitive regime, the conditions
of success for the individual are not necessarily the same as those for
a class. The success of a class or party presumes a strong element of
clannishness, or loyalty to a chief, or adherence to a tenet; whereas
the competitive individual can best achieve his ends if he combines the
barbarian's energy, initiative, self-seeking and disingenuousness with
the savage's lack of loyalty or clannishness. It may be remarked by the
way, that the men who have scored a brilliant (Napoleonic) success on
the basis of an impartial self-seeking and absence of scruple, have
not uncommonly shown more of the physical characteristics of the
brachycephalic-brunette than of the dolicho-blond. The greater
proportion of moderately successful individuals, in a self-seeking way,
however, seem, in physique, to belong to the last-named ethnic element.

The temperament induced by the predatory habit of life makes for the
survival and fullness of life of the individual under a regime of
emulation; at the same time it makes for the survival and success of the
group if the group's life as a collectivity is also predominantly a life
of hostile competition with other groups. But the evolution of economic
life in the industrially more mature communities has now begun to take
such a turn that the interest of the community no longer coincides with
the emulative interests of the individual. In their corporate capacity,
these advanced industrial communities are ceasing to be competitors
for the means of life or for the right to live--except in so far as the
predatory propensities of their ruling classes keep up the tradition of
war and rapine. These communities are no longer hostile to one another
by force of circumstances, other than the circumstances of tradition
and temperament. Their material interests--apart, possibly, from
the interests of the collective good fame--are not only no longer
incompatible, but the success of any one of the communities
unquestionably furthers the fullness of life of any other community in
the group, for the present and for an incalculable time to come. No one
of them any longer has any material interest in getting the better
of any other. The same is not true in the same degree as regards
individuals and their relations to one another.

The collective interests of any modern community center in industrial
efficiency. The individual is serviceable for the ends of the community
somewhat in proportion to his efficiency in the productive employments
vulgarly so called. This collective interest is best served by honesty,
diligence, peacefulness, good-will, an absence of self-seeking, and
an habitual recognition and apprehension of causal sequence, without
admixture of animistic belief and without a sense of dependence on any
preternatural intervention in the course of events. Not much is to
be said for the beauty, moral excellence, or general worthiness and
reputability of such a prosy human nature as these traits imply; and
there is little ground of enthusiasm for the manner of collective life
that would result from the prevalence of these traits in unmitigated
dominance. But that is beside the point. The successful working of a
modern industrial community is best secured where these traits concur,
and it is attained in the degree in which the human material is
characterized by their possession. Their presence in some measure is
required in order to have a tolerable adjustment to the circumstances of
the modern industrial situation. The complex, comprehensive, essentially
peaceable, and highly organized mechanism of the modern industrial
community works to the best advantage when these traits, or most of
them, are present in the highest practicable degree. These traits are
present in a markedly less degree in the man of the predatory type than
is useful for the purposes of the modern collective life.

On the other hand, the immediate interest of the individual under the
competitive regime is best served by shrewd trading and unscrupulous
management. The characteristics named above as serving the interests
of the community are disserviceable to the individual, rather than
otherwise. The presence of these aptitudes in his make-up diverts his
energies to other ends than those of pecuniary gain; and also in
his pursuit of gain they lead him to seek gain by the indirect and
ineffectual channels of industry, rather than by a free and unfaltering
career of sharp practice. The industrial aptitudes are pretty
consistently a hindrance to the individual. Under the regime of
emulation the members of a modern industrial community are rivals, each
of whom will best attain his individual and immediate advantage if,
through an exceptional exemption from scruple, he is able serenely to
overreach and injure his fellows when the chance offers.

It has already been noticed that modern economic institutions fall into
two roughly distinct categories--the pecuniary and the industrial. The
like is true of employments. Under the former head are employments that
have to do with ownership or acquisition; under the latter head, those
that have to do with workmanship or production. As was found in speaking
of the growth of institutions, so with regard to employments.
The economic interests of the leisure class lie in the pecuniary
employments; those of the working classes lie in both classes of
employments, but chiefly in the industrial. Entrance to the leisure
class lies through the pecuniary employments.

These two classes of employment differ materially in respect of the
aptitudes required for each; and the training which they give similarly
follows two divergent lines. The discipline of the pecuniary employments
acts to conserve and to cultivate certain of the predatory aptitudes and
the predatory animus. It does this both by educating those individuals
and classes who are occupied with these employments and by selectively
repressing and eliminating those individuals and lines of descent that
are unfit in this respect. So far as men's habits of thought are shaped
by the competitive process of acquisition and tenure; so far as their
economic functions are comprised within the range of ownership of
wealth as conceived in terms of exchange value, and its management and
financiering through a permutation of values; so far their experience
in economic life favors the survival and accentuation of the predatory
temperament and habits of thought. Under the modern, peaceable system,
it is of course the peaceable range of predatory habits and aptitudes
that is chiefly fostered by a life of acquisition. That is to say, the
pecuniary employments give proficiency in the general line of practices
comprised under fraud, rather than in those that belong under the more
archaic method of forcible seizure.

These pecuniary employments, tending to conserve the predatory
temperament, are the employments which have to do with ownership--the
immediate function of the leisure class proper--and the subsidiary
functions concerned with acquisition and accumulation. These cover the
class of persons and that range of duties in the economic process which
have to do with the ownership of enterprises engaged in competitive
industry; especially those fundamental lines of economic management
which are classed as financiering operations. To these may be added
the greater part of mercantile occupations. In their best and clearest
development these duties make up the economic office of the "captain
of industry." The captain of industry is an astute man rather than
an ingenious one, and his captaincy is a pecuniary rather than an
industrial captaincy. Such administration of industry as he exercises
is commonly of a permissive kind. The mechanically effective details of
production and of industrial organization are delegated to subordinates
of a less "practical" turn of mind--men who are possessed of a gift for
workmanship rather than administrative ability. So far as regards their
tendency in shaping human nature by education and selection, the common
run of non-economic employments are to be classed with the pecuniary
employments. Such are politics and ecclesiastical and military
employments.

The pecuniary employments have also the sanction of reputability in
a much higher degree than the industrial employments. In this way the
leisure-class standards of good repute come in to sustain the
prestige of those aptitudes that serve the invidious purpose; and the
leisure-class scheme of decorous living, therefore, also furthers the
survival and culture of the predatory traits. Employments fall into
a hierarchical gradation of reputability. Those which have to do
immediately with ownership on a large scale are the most reputable of
economic employments proper. Next to these in good repute come
those employments that are immediately subservient to ownership and
financiering--such as banking and the law. Banking employments also
carry a suggestion of large ownership, and this fact is doubtless
accountable for a share of the prestige that attaches to the business.
The profession of the law does not imply large ownership; but since no
taint of usefulness, for other than the competitive purpose, attaches
to the lawyer's trade, it grades high in the conventional scheme. The
lawyer is exclusively occupied with the details of predatory fraud,
either in achieving or in checkmating chicanery, and success in the
profession is therefore accepted as marking a large endowment of that
barbarian astuteness which has always commanded men's respect and fear.
Mercantile pursuits are only half-way reputable, unless they involve a
large element of ownership and a small element of usefulness. They grade
high or low somewhat in proportion as they serve the higher or the lower
needs; so that the business of retailing the vulgar necessaries of
life descends to the level of the handicrafts and factory labor. Manual
labor, or even the work of directing mechanical processes, is of course
on a precarious footing as regards respectability. A qualification is
necessary as regards the discipline given by the pecuniary employments.
As the scale of industrial enterprise grows larger, pecuniary management
comes to bear less of the character of chicanery and shrewd competition
in detail. That is to say, for an ever-increasing proportion of the
persons who come in contact with this phase of economic life, business
reduces itself to a routine in which there is less immediate suggestion
of overreaching or exploiting a competitor. The consequent exemption
from predatory habits extends chiefly to subordinates employed in
business. The duties of ownership and administration are virtually
untouched by this qualification. The case is different as regards those
individuals or classes who are immediately occupied with the technique
and manual operations of production. Their daily life is not in the same
degree a course of habituation to the emulative and invidious motives
and maneuvers of the pecuniary side of industry. They are consistently
held to the apprehension and coordination of mechanical facts and
sequences, and to their appreciation and utilization for the purposes
of human life. So far as concerns this portion of the population, the
educative and selective action of the industrial process with which they
are immediately in contact acts to adapt their habits of thought to the
non-invidious purposes of the collective life. For them, therefore, it
hastens the obsolescence of the distinctively predatory aptitudes and
propensities carried over by heredity and tradition from the barbarian
past of the race.

The educative action of the economic life of the community, therefore,
is not of a uniform kind throughout all its manifestations. That range
of economic activities which is concerned immediately with pecuniary
competition has a tendency to conserve certain predatory traits; while
those industrial occupations which have to do immediately with the
production of goods have in the main the contrary tendency. But with
regard to the latter class of employments it is to be noticed in
qualification that the persons engaged in them are nearly all to some
extent also concerned with matters of pecuniary competition (as, for
instance, in the competitive fixing of wages and salaries, in the
purchase of goods for consumption, etc.). Therefore the distinction
here made between classes of employments is by no means a hard and fast
distinction between classes of persons.

The employments of the leisure classes in modern industry are such as to
keep alive certain of the predatory habits and aptitudes. So far as
the members of those classes take part in the industrial process, their
training tends to conserve in them the barbarian temperament. But there
is something to be said on the other side. Individuals so placed as to
be exempt from strain may survive and transmit their characteristics
even if they differ widely from the average of the species both in
physique and in spiritual make-up. The chances for a survival and
transmission of atavistic traits are greatest in those classes that are
most sheltered from the stress of circumstances. The leisure class is in
some degree sheltered from the stress of the industrial situation,
and should, therefore, afford an exceptionally great proportion of
reversions to the peaceable or savage temperament. It should be possible
for such aberrant or atavistic individuals to unfold their life activity
on ante-predatory lines without suffering as prompt a repression or
elimination as in the lower walks of life.

Something of the sort seems to be true in fact. There is, for instance,
an appreciable proportion of the upper classes whose inclinations
lead them into philanthropic work, and there is a considerable body
of sentiment in the class going to support efforts of reform and
amelioration. And much of this philanthropic and reformatory effort,
moreover, bears the marks of that amiable "cleverness" and incoherence
that is characteristic of the primitive savage. But it may still be
doubtful whether these facts are evidence of a larger proportion of
reversions in the higher than in the lower strata, even if the same
inclinations were present in the impecunious classes, it would not as
easily find expression there; since those classes lack the means and the
time and energy to give effect to their inclinations in this respect.
The \emph{prima facie} evidence of the facts can scarcely go unquestioned.

In further qualification it is to be noted that the leisure class of
today is recruited from those who have been successful in a pecuniary
way, and who, therefore, are presumably endowed with more than an even
complement of the predatory traits. Entrance into the leisure class lies
through the pecuniary employments, and these employments, by selection
and adaptation, act to admit to the upper levels only those lines of
descent that are pecuniarily fit to survive under the predatory test.
And so soon as a case of reversion to non-predatory human nature shows
itself on these upper levels, it is commonly weeded out and thrown back
to the lower pecuniary levels. In order to hold its place in the class,
a stock must have the pecuniary temperament; otherwise its fortune would
be dissipated and it would presently lose caste. Instances of this kind
are sufficiently frequent. The constituency of the leisure class is kept
up by a continual selective process, whereby the individuals and
lines of descent that are eminently fitted for an aggressive pecuniary
competition are withdrawn from the lower classes. In order to reach the
upper levels the aspirant must have, not only a fair average complement
of the pecuniary aptitudes, but he must have these gifts in such an
eminent degree as to overcome very material difficulties that stand in
the way of his ascent. Barring accidents, the \emph{nouveaux arrivés} are a
picked body.

This process of selective admission has, of course, always been going
on; ever since the fashion of pecuniary emulation set in--which is much
the same as saying, ever since the institution of a leisure class was
first installed. But the precise ground of selection has not always been
the same, and the selective process has therefore not always given the
same results. In the early barbarian, or predatory stage proper, the
test of fitness was prowess, in the naive sense of the word. To gain
entrance to the class, the candidate had to be gifted with clannishness,
massiveness, ferocity, unscrupulousness, and tenacity of purpose. These
were the qualities that counted toward the accumulation and continued
tenure of wealth. The economic basis of the leisure class, then as
later, was the possession of wealth; but the methods of accumulating
wealth, and the gifts required for holding it, have changed in some
degree since the early days of the predatory culture. In consequence of
the selective process the dominant traits of the early barbarian leisure
class were bold aggression, an alert sense of status, and a free
resort to fraud. The members of the class held their place by tenure of
prowess. In the later barbarian culture society attained settled methods
of acquisition and possession under the quasi-peaceable regime of
status. Simple aggression and unrestrained violence in great measure
gave place to shrewd practice and chicanery, as the best approved method
of accumulating wealth. A different range of aptitudes and propensities
would then be conserved in the leisure class. Masterful aggression, and
the correlative massiveness, together with a ruthlessly consistent
sense of status, would still count among the most splendid traits of
the class. These have remained in our traditions as the typical
"aristocratic virtues." But with these were associated an increasing
complement of the less obtrusive pecuniary virtues; such as providence,
prudence, and chicanery. As time has gone on, and the modern peaceable
stage of pecuniary culture has been approached, the last-named range of
aptitudes and habits has gained in relative effectiveness for pecuniary
ends, and they have counted for relatively more in the selective process
under which admission is gained and place is held in the leisure class.

The ground of selection has changed, until the aptitudes which now
qualify for admission to the class are the pecuniary aptitudes only.
What remains of the predatory barbarian traits is the tenacity of
purpose or consistency of aim which distinguished the successful
predatory barbarian from the peaceable savage whom he supplanted.
But this trait can not be said characteristically to distinguish the
pecuniarily successful upper-class man from the rank and file of the
industrial classes. The training and the selection to which the latter
are exposed in modern industrial life give a similarly decisive weight
to this trait. Tenacity of purpose may rather be said to distinguish
both these classes from two others; the shiftless ne'er do-well and the
lower-class delinquent. In point of natural endowment the pecuniary man
compares with the delinquent in much the same way as the industrial man
compares with the good-natured shiftless dependent. The ideal pecuniary
man is like the ideal delinquent in his unscrupulous conversion of goods
and persons to his own ends, and in a callous disregard of the feelings
and wishes of others and of the remoter effects of his actions; but he
is unlike him in possessing a keener sense of status, and in working
more consistently and farsightedly to a remoter end. The kinship of the
two types of temperament is further shown in a proclivity to "sport"
and gambling, and a relish of aimless emulation. The ideal pecuniary
man also shows a curious kinship with the delinquent in one of the
concomitant variations of the predatory human nature. The delinquent is
very commonly of a superstitious habit of mind; he is a great believer
in luck, spells, divination and destiny, and in omens and shamanistic
ceremony. Where circumstances are favorable, this proclivity is apt to
express itself in a certain servile devotional fervor and a punctilious
attention to devout observances; it may perhaps be better characterized
as devoutness than as religion. At this point the temperament of the
delinquent has more in common with the pecuniary and leisure classes
than with the industrial man or with the class of shiftless dependents.

Life in a modern industrial community, or in other words life under
the pecuniary culture, acts by a process of selection to develop and
conserve a certain range of aptitudes and propensities. The present
tendency of this selective process is not simply a reversion to a given,
immutable ethnic type. It tends rather to a modification of human nature
differing in some respects from any of the types or variants transmitted
out of the past. The objective point of the evolution is not a single
one. The temperament which the evolution acts to establish as normal
differs from any one of the archaic variants of human nature in its
greater stability of aim--greater singleness of purpose and greater
persistence in effort. So far as concerns economic theory, the objective
point of the selective process is on the whole single to this extent;
although there are minor tendencies of considerable importance diverging
from this line of development. But apart from this general trend the
line of development is not single. As concerns economic theory, the
development in other respects runs on two divergent lines. So far
as regards the selective conservation of capacities or aptitudes
in individuals, these two lines may be called the pecuniary and the
industrial. As regards the conservation of propensities, spiritual
attitude, or animus, the two may be called the invidious or
self-regarding and the non-invidious or economical. As regards the
intellectual or cognitive bent of the two directions of growth, the
former may be characterized as the personal standpoint, of conation,
qualitative relation, status, or worth; the latter as the impersonal
standpoint, of sequence, quantitative relation, mechanical efficiency,
or use.

The pecuniary employments call into action chiefly the former of
these two ranges of aptitudes and propensities, and act selectively
to conserve them in the population. The industrial employments, on the
other hand, chiefly exercise the latter range, and act to conserve them.
An exhaustive psychological analysis will show that each of these two
ranges of aptitudes and propensities is but the multiform expression of
a given temperamental bent. By force of the unity or singleness of
the individual, the aptitudes, animus, and interests comprised in the
first-named range belong together as expressions of a given variant
of human nature. The like is true of the latter range. The two may be
conceived as alternative directions of human life, in such a way that
a given individual inclines more or less consistently to the one or
the other. The tendency of the pecuniary life is, in a general way, to
conserve the barbarian temperament, but with the substitution of fraud
and prudence, or administrative ability, in place of that predilection
for physical damage that characterizes the early barbarian. This
substitution of chicanery in place of devastation takes place only in an
uncertain degree. Within the pecuniary employments the selective action
runs pretty consistently in this direction, but the discipline of
pecuniary life, outside the competition for gain, does not work
consistently to the same effect. The discipline of modern life in the
consumption of time and goods does not act unequivocally to eliminate
the aristocratic virtues or to foster the bourgeois virtues. The
conventional scheme of decent living calls for a considerable exercise
of the earlier barbarian traits. Some details of this traditional scheme
of life, bearing on this point, have been noticed in earlier chapters
under the head of leisure, and further details will be shown in later
chapters.

From what has been said, it appears that the leisure-class life and
the leisure-class scheme of life should further the conservation of the
barbarian temperament; chiefly of the quasi-peaceable, or bourgeois,
variant, but also in some measure of the predatory variant. In the
absence of disturbing factors, therefore, it should be possible to
trace a difference of temperament between the classes of society. The
aristocratic and the bourgeois virtues--that is to say the destructive
and pecuniary traits--should be found chiefly among the upper classes,
and the industrial virtues--that is to say the peaceable traits--chiefly
among the classes given to mechanical industry.

In a general and uncertain way this holds true, but the test is not so
readily applied nor so conclusive as might be wished. There are several
assignable reasons for its partial failure. All classes are in a measure
engaged in the pecuniary struggle, and in all classes the possession
of the pecuniary traits counts towards the success and survival of
the individual. Wherever the pecuniary culture prevails, the selective
process by which men's habits of thought are shaped, and by which the
survival of rival lines of descent is decided, proceeds proximately on
the basis of fitness for acquisition. Consequently, if it were not for
the fact that pecuniary efficiency is on the whole incompatible with
industrial efficiency, the selective action of all occupations would
tend to the unmitigated dominance of the pecuniary temperament. The
result would be the installation of what has been known as the "economic
man," as the normal and definitive type of human nature. But the
"economic man," whose only interest is the self-regarding one and whose
only human trait is prudence is useless for the purposes of modern
industry.

The modern industry requires an impersonal, non-invidious interest in
the work in hand. Without this the elaborate processes of industry
would be impossible, and would, indeed, never have been conceived. This
interest in work differentiates the workman from the criminal on the one
hand, and from the captain of industry on the other. Since work must be
done in order to the continued life of the community, there results a
qualified selection favoring the spiritual aptitude for work, within
a certain range of occupations. This much, however, is to be conceded,
that even within the industrial occupations the selective elimination
of the pecuniary traits is an uncertain process, and that there is
consequently an appreciable survival of the barbarian temperament even
within these occupations. On this account there is at present no broad
distinction in this respect between the leisure-class character and the
character of the common run of the population.

The whole question as to a class distinction in respect to spiritual
make-up is also obscured by the presence, in all classes of society, of
acquired habits of life that closely simulate inherited traits and at
the same time act to develop in the entire body of the population the
traits which they simulate. These acquired habits, or assumed traits of
character, are most commonly of an aristocratic cast. The prescriptive
position of the leisure class as the exemplar of reputability has
imposed many features of the leisure-class theory of life upon the
lower classes; with the result that there goes on, always and throughout
society, a more or less persistent cultivation of these aristocratic
traits. On this ground also these traits have a better chance of
survival among the body of the people than would be the case if it were
not for the precept and example of the leisure class. As one channel,
and an important one, through which this transfusion of aristocratic
views of life, and consequently more or less archaic traits of character
goes on, may be mentioned the class of domestic servants. These have
their notions of what is good and beautiful shaped by contact with the
master class and carry the preconceptions so acquired back among their
low-born equals, and so disseminate the higher ideals abroad through
the community without the loss of time which this dissemination might
otherwise suffer. The saying "Like master, like man," has a greater
significance than is commonly appreciated for the rapid popular
acceptance of many elements of upper-class culture.

There is also a further range of facts that go to lessen class
differences as regards the survival of the pecuniary virtues. The
pecuniary struggle produces an underfed class, of large proportions.
This underfeeding consists in a deficiency of the necessaries of life or
of the necessaries of a decent expenditure. In either case the result is
a closely enforced struggle for the means with which to meet the daily
needs; whether it be the physical or the higher needs. The strain of
self-assertion against odds takes up the whole energy of the individual;
he bends his efforts to compass his own invidious ends alone, and
becomes continually more narrowly self-seeking. The industrial traits in
this way tend to obsolescence through disuse. Indirectly, therefore, by
imposing a scheme of pecuniary decency and by withdrawing as much as
may be of the means of life from the lower classes, the institution of
a leisure class acts to conserve the pecuniary traits in the body of the
population. The result is an assimilation of the lower classes to the
type of human nature that belongs primarily to the upper classes only.
It appears, therefore, that there is no wide difference in temperament
between the upper and the lower classes; but it appears also that the
absence of such a difference is in good part due to the prescriptive
example of the leisure class and to the popular acceptance of those
broad principles of conspicuous waste and pecuniary emulation on which
the institution of a leisure class rests. The institution acts to lower
the industrial efficiency of the community and retard the adaptation of
human nature to the exigencies of modern industrial life. It affects the
prevalent or effective human nature in a conservative direction, (1) by
direct transmission of archaic traits, through inheritance within the
class and wherever the leisure-class blood is transfused outside the
class, and (2) by conserving and fortifying the traditions of the
archaic regime, and so making the chances of survival of barbarian
traits greater also outside the range of transfusion of leisure-class
blood.

But little if anything has been done towards collecting or digesting
data that are of special significance for the question of survival or
elimination of traits in the modern populations. Little of a tangible
character can therefore be offered in support of the view here taken,
beyond a discursive review of such everyday facts as lie ready to hand.
Such a recital can scarcely avoid being commonplace and tedious, but for
all that it seems necessary to the completeness of the argument, even in
the meager outline in which it is here attempted. A degree of indulgence
may therefore fairly be bespoken for the succeeding chapters, which
offer a fragmentary recital of this kind.




\chapter{Modern Survivals of Prowess}

The leisure class lives by the industrial community rather than in it.
Its relations to industry are of a pecuniary rather than an industrial
kind. Admission to the class is gained by exercise of the pecuniary
aptitudes--aptitudes for acquisition rather than for serviceability.
There is, therefore, a continued selective sifting of the human material
that makes up the leisure class, and this selection proceeds on the
ground of fitness for pecuniary pursuits. But the scheme of life of the
class is in large part a heritage from the past, and embodies much of
the habits and ideals of the earlier barbarian period. This archaic,
barbarian scheme of life imposes itself also on the lower orders, with
more or less mitigation. In its turn the scheme of life, of conventions,
acts selectively and by education to shape the human material, and its
action runs chiefly in the direction of conserving traits, habits, and
ideals that belong to the early barbarian age--the age of prowess and
predatory life.

The most immediate and unequivocal expression of that archaic human
nature which characterizes man in the predatory stage is the fighting
propensity proper. In cases where the predatory activity is a collective
one, this propensity is frequently called the martial spirit, or,
latterly, patriotism. It needs no insistence to find assent to the
proposition that in the countries of civilized Europe the hereditary
leisure class is endowed with this martial spirit in a higher
degree than the middle classes. Indeed, the leisure class claims the
distinction as a matter of pride, and no doubt with some grounds. War is
honorable, and warlike prowess is eminently honorific in the eyes of the
generality of men; and this admiration of warlike prowess is itself
the best voucher of a predatory temperament in the admirer of war. The
enthusiasm for war, and the predatory temper of which it is the index,
prevail in the largest measure among the upper classes, especially
among the hereditary leisure class. Moreover, the ostensible serious
occupation of the upper class is that of government, which, in point of
origin and developmental content, is also a predatory occupation.

The only class which could at all dispute with the hereditary leisure
class the honor of an habitual bellicose frame of mind is that of
the lower-class delinquents. In ordinary times, the large body of the
industrial classes is relatively apathetic touching warlike interests.
When unexcited, this body of the common people, which makes up the
effective force of the industrial community, is rather averse to any
other than a defensive fight; indeed, it responds a little tardily even
to a provocation which makes for an attitude of defense. In the more
civilized communities, or rather in the communities which have reached
an advanced industrial development, the spirit of warlike aggression
may be said to be obsolescent among the common people. This does not
say that there is not an appreciable number of individuals among
the industrial classes in whom the martial spirit asserts itself
obtrusively. Nor does it say that the body of the people may not be
fired with martial ardor for a time under the stimulus of some special
provocation, such as is seen in operation today in more than one of the
countries of Europe, and for the time in America. But except for such
seasons of temporary exaltation, and except for those individuals who
are endowed with an archaic temperament of the predatory type, together
with the similarly endowed body of individuals among the higher and
the lowest classes, the inertness of the mass of any modern civilized
community in this respect is probably so great as would make war
impracticable, except against actual invasion. The habits and aptitudes
of the common run of men make for an unfolding of activity in other,
less picturesque directions than that of war.

This class difference in temperament may be due in part to a difference
in the inheritance of acquired traits in the several classes, but it
seems also, in some measure, to correspond with a difference in ethnic
derivation. The class difference is in this respect visibly less in
those countries whose population is relatively homogeneous, ethnically,
than in the countries where there is a broader divergence between the
ethnic elements that make up the several classes of the community. In
the same connection it may be noted that the later accessions to the
leisure class in the latter countries, in a general way, show less of
the martial spirit than contemporary representatives of the aristocracy
of the ancient line. These nouveaux arrivés have recently emerged from
the commonplace body of the population and owe their emergence into the
leisure class to the exercise of traits and propensities which are not
to be classed as prowess in the ancient sense.

Apart from warlike activity proper, the institution of the duel is also
an expression of the same superior readiness for combat; and the duel
is a leisure-class institution. The duel is in substance a more or less
deliberate resort to a fight as a final settlement of a difference of
opinion. In civilized communities it prevails as a normal phenomenon
only where there is an hereditary leisure class, and almost exclusively
among that class. The exceptions are (1) military and naval officers
who are ordinarily members of the leisure class, and who are at the
same time specially trained to predatory habits of mind and (2) the
lower-class delinquents--who are by inheritance, or training, or both,
of a similarly predatory disposition and habit. It is only the high-bred
gentleman and the rowdy that normally resort to blows as the universal
solvent of differences of opinion. The plain man will ordinarily fight
only when excessive momentary irritation or alcoholic exaltation act to
inhibit the more complex habits of response to the stimuli that make
for provocation. He is then thrown back upon the simpler, less
differentiated forms of the instinct of self-assertion; that is to say,
he reverts temporarily and without reflection to an archaic habit of
mind.

This institution of the duel as a mode of finally settling disputes
and serious questions of precedence shades off into the obligatory,
unprovoked private fight, as a social obligation due to one's good
repute. As a leisure-class usage of this kind we have, particularly,
that bizarre survival of bellicose chivalry, the German student duel. In
the lower or spurious leisure class of the delinquents there is in all
countries a similar, though less formal, social obligation incumbent on
the rowdy to assert his manhood in unprovoked combat with his fellows.
And spreading through all grades of society, a similar usage prevails
among the boys of the community. The boy usually knows to nicety, from
day to day, how he and his associates grade in respect of relative
fighting capacity; and in the community of boys there is ordinarily no
secure basis of reputability for any one who, by exception, will not or
can not fight on invitation.

All this applies especially to boys above a certain somewhat vague limit
of maturity. The child's temperament does not commonly answer to this
description during infancy and the years of close tutelage, when the
child still habitually seeks contact with its mother at every turn of
its daily life. During this earlier period there is little aggression
and little propensity for antagonism. The transition from this
peaceable temper to the predaceous, and in extreme cases malignant,
mischievousness of the boy is a gradual one, and it is accomplished
with more completeness, covering a larger range of the individual's
aptitudes, in some cases than in others. In the earlier stage of his
growth, the child, whether boy or girl, shows less of initiative and
aggressive self-assertion and less of an inclination to isolate himself
and his interests from the domestic group in which he lives, and he
shows more of sensitiveness to rebuke, bashfulness, timidity, and the
need of friendly human contact. In the common run of cases this early
temperament passes, by a gradual but somewhat rapid obsolescence of the
infantile features, into the temperament of the boy proper; though there
are also cases where the predaceous futures of boy life do not emerge at
all, or at the most emerge in but a slight and obscure degree.

In girls the transition to the predaceous stage is seldom accomplished
with the same degree of completeness as in boys; and in a relatively
large proportion of cases it is scarcely undergone at all. In such cases
the transition from infancy to adolescence and maturity is a gradual and
unbroken process of the shifting of interest from infantile purposes and
aptitudes to the purposes, functions, and relations of adult life. In
the girls there is a less general prevalence of a predaceous interval
in the development; and in the cases where it occurs, the predaceous and
isolating attitude during the interval is commonly less accentuated.

In the male child the predaceous interval is ordinarily fairly well
marked and lasts for some time, but it is commonly terminated (if at
all) with the attainment of maturity. This last statement may need very
material qualification. The cases are by no means rare in which the
transition from the boyish to the adult temperament is not made, or
is made only partially--understanding by the "adult" temperament the
average temperament of those adult individuals in modern industrial life
who have some serviceability for the purposes of the collective life
process, and who may therefore be said to make up the effective average
of the industrial community.

The ethnic composition of the European populations varies. In some
cases even the lower classes are in large measure made up of the
peace-disturbing dolicho-blond; while in others this ethnic element is
found chiefly among the hereditary leisure class. The fighting habit
seems to prevail to a less extent among the working-class boys in the
latter class of populations than among the boys of the upper classes or
among those of the populations first named.

If this generalization as to the temperament of the boy among the
working classes should be found true on a fuller and closer scrutiny of
the field, it would add force to the view that the bellicose temperament
is in some appreciable degree a race characteristic; it appears to
enter more largely into the make-up of the dominant, upper-class
ethnic type--the dolicho-blond--of the European countries than into the
subservient, lower-class types of man which are conceived to constitute
the body of the population of the same communities.

The case of the boy may seem not to bear seriously on the question of
the relative endowment of prowess with which the several classes of
society are gifted; but it is at least of some value as going to show
that this fighting impulse belongs to a more archaic temperament than
that possessed by the average adult man of the industrious classes. In
this, as in many other features of child life, the child reproduces,
temporarily and in miniature, some of the earlier phases of the
development of adult man. Under this interpretation, the boy's
predilection for exploit and for isolation of his own interest is to be
taken as a transient reversion to the human nature that is normal to the
early barbarian culture--the predatory culture proper. In this respect,
as in much else, the leisure-class and the delinquent-class character
shows a persistence into adult life of traits that are normal to
childhood and youth, and that are likewise normal or habitual to the
earlier stages of culture. Unless the difference is traceable entirely
to a fundamental difference between persistent ethnic types, the traits
that distinguish the swaggering delinquent and the punctilious gentleman
of leisure from the common crowd are, in some measure, marks of an
arrested spiritual development. They mark an immature phase, as compared
with the stage of development attained by the average of the adults in
the modern industrial community. And it will appear presently that the
puerile spiritual make-up of these representatives of the upper and the
lowest social strata shows itself also in the presence of other archaic
traits than this proclivity to ferocious exploit and isolation.

As if to leave no doubt about the essential immaturity of the fighting
temperament, we have, bridging the interval between legitimate boyhood
and adult manhood, the aimless and playful, but more or less systematic
and elaborate, disturbances of the peace in vogue among schoolboys of a
slightly higher age. In the common run of cases, these disturbances
are confined to the period of adolescence. They recur with decreasing
frequency and acuteness as youth merges into adult life, and so they
reproduce, in a general way, in the life of the individual, the sequence
by which the group has passed from the predatory to a more settled habit
of life. In an appreciable number of cases the spiritual growth of the
individual comes to a close before he emerges from this puerile
phase; in these cases the fighting temper persists through life. Those
individuals who in spiritual development eventually reach man's
estate, therefore, ordinarily pass through a temporary archaic phase
corresponding to the permanent spiritual level of the fighting and
sporting men. Different individuals will, of course, achieve spiritual
maturity and sobriety in this respect in different degrees; and those
who fail of the average remain as an undissolved residue of crude
humanity in the modern industrial community and as a foil for that
selective process of adaptation which makes for a heightened industrial
efficiency and the fullness of life of the collectivity. This
arrested spiritual development may express itself not only in a direct
participation by adults in youthful exploits of ferocity, but also
indirectly in aiding and abetting disturbances of this kind on the
part of younger persons. It thereby furthers the formation of habits of
ferocity which may persist in the later life of the growing generation,
and so retard any movement in the direction of a more peaceable
effective temperament on the part of the community. If a person so
endowed with a proclivity for exploits is in a position to guide the
development of habits in the adolescent members of the community, the
influence which he exerts in the direction of conservation and reversion
to prowess may be very considerable. This is the significance, for
instance, of the fostering care latterly bestowed by many clergymen
and other pillars of society upon "boys' brigades" and similar
pseudo-military organizations. The same is true of the encouragement
given to the growth of "college spirit," college athletics, and the
like, in the higher institutions of learning.

These manifestations of the predatory temperament are all to be classed
under the head of exploit. They are partly simple and unreflected
expressions of an attitude of emulative ferocity, partly activities
deliberately entered upon with a view to gaining repute for prowess.
Sports of all kinds are of the same general character, including
prize-fights, bull-fights, athletics, shooting, angling, yachting,
and games of skill, even where the element of destructive physical
efficiency is not an obtrusive feature. Sports shade off from the basis
of hostile combat, through skill, to cunning and chicanery, without its
being possible to draw a line at any point. The ground of an addiction
to sports is an archaic spiritual constitution--the possession of the
predatory emulative propensity in a relatively high potency, a strong
proclivity to adventuresome exploit and to the infliction of damage is
especially pronounced in those employments which are in colloquial usage
specifically called sportsmanship.

It is perhaps truer, or at least more evident, as regards sports than as
regards the other expressions of predatory emulation already spoken of,
that the temperament which inclines men to them is essentially a boyish
temperament. The addiction to sports, therefore, in a peculiar degree
marks an arrested development of the man's moral nature. This peculiar
boyishness of temperament in sporting men immediately becomes apparent
when attention is directed to the large element of make-believe that
is present in all sporting activity. Sports share this character of
make-believe with the games and exploits to which children, especially
boys, are habitually inclined. Make-believe does not enter in the same
proportion into all sports, but it is present in a very appreciable
degree in all. It is apparently present in a larger measure in
sportsmanship proper and in athletic contests than in set games of skill
of a more sedentary character; although this rule may not be found to
apply with any great uniformity. It is noticeable, for instance, that
even very mild-mannered and matter-of-fact men who go out shooting are
apt to carry an excess of arms and accoutrements in order to impress
upon their own imagination the seriousness of their undertaking.
These huntsmen are also prone to a histrionic, prancing gait and to
an elaborate exaggeration of the motions, whether of stealth or of
onslaught, involved in their deeds of exploit. Similarly in athletic
sports there is almost invariably present a good share of rant and
swagger and ostensible mystification--features which mark the histrionic
nature of these employments. In all this, of course, the reminder of
boyish make-believe is plain enough. The slang of athletics, by the way,
is in great part made up of extremely sanguinary locutions borrowed from
the terminology of warfare. Except where it is adopted as a necessary
means of secret communication, the use of a special slang in any
employment is probably to be accepted as evidence that the occupation in
question is substantially make-believe.

A further feature in which sports differ from the duel and similar
disturbances of the peace is the peculiarity that they admit of other
motives being assigned for them besides the impulses of exploit and
ferocity. There is probably little if any other motive present in any
given case, but the fact that other reasons for indulging in sports are
frequently assigned goes to say that other grounds are sometimes present
in a subsidiary way. Sportsmen--hunters and anglers--are more or less in
the habit of assigning a love of nature, the need of recreation, and the
like, as the incentives to their favorite pastime. These motives are no
doubt frequently present and make up a part of the attractiveness of
the sportsman's life; but these can not be the chief incentives. These
ostensible needs could be more readily and fully satisfied without the
accompaniment of a systematic effort to take the life of those creatures
that make up an essential feature of that "nature" that is beloved
by the sportsman. It is, indeed, the most noticeable effect of the
sportsman's activity to keep nature in a state of chronic desolation by
killing off all living thing whose destruction he can compass.

Still, there is ground for the sportsman's claim that under the existing
conventionalities his need of recreation and of contact with nature can
best be satisfied by the course which he takes. Certain canons of good
breeding have been imposed by the prescriptive example of a predatory
leisure class in the past and have been somewhat painstakingly conserved
by the usage of the latter-day representatives of that class; and these
canons will not permit him, without blame, to seek contact with nature
on other terms. From being an honorable employment handed down from the
predatory culture as the highest form of everyday leisure, sports have
come to be the only form of outdoor activity that has the full sanction
of decorum. Among the proximate incentives to shooting and angling,
then, may be the need of recreation and outdoor life. The remoter cause
which imposes the necessity of seeking these objects under the cover of
systematic slaughter is a prescription that can not be violated except
at the risk of disrepute and consequent lesion to one's self-respect.

The case of other kinds of sport is somewhat similar. Of these, athletic
games are the best example. Prescriptive usage with respect to what
forms of activity, exercise, and recreation are permissible under the
code of reputable living is of course present here also. Those who are
addicted to athletic sports, or who admire them, set up the claim that
these afford the best available means of recreation and of "physical
culture." And prescriptive usage gives countenance to the claim. The
canons of reputable living exclude from the scheme of life of the
leisure class all activity that can not be classed as conspicuous
leisure. And consequently they tend by prescription to exclude it also
from the scheme of life of the community generally. At the same
time purposeless physical exertion is tedious and distasteful beyond
tolerance. As has been noticed in another connection, recourse is in
such a case had to some form of activity which shall at least afford
a colorable pretense of purpose, even if the object assigned be only a
make-believe. Sports satisfy these requirements of substantial futility
together with a colorable make-believe of purpose. In addition to
this they afford scope for emulation, and are attractive also on that
account. In order to be decorous, an employment must conform to the
leisure-class canon of reputable waste; at the same time all activity,
in order to be persisted in as an habitual, even if only partial,
expression of life, must conform to the generically human canon of
efficiency for some serviceable objective end. The leisure-class canon
demands strict and comprehensive futility, the instinct of workmanship
demands purposeful action. The leisure-class canon of decorum acts
slowly and pervasively, by a selective elimination of all substantially
useful or purposeful modes of action from the accredited scheme of
life; the instinct of workmanship acts impulsively and may be satisfied,
provisionally, with a proximate purpose. It is only as the apprehended
ulterior futility of a given line of action enters the reflective
complex of consciousness as an element essentially alien to the normally
purposeful trend of the life process that its disquieting and deterrent
effect on the consciousness of the agent is wrought.

The individual's habits of thought make an organic complex, the trend
of which is necessarily in the direction of serviceability to the
life process. When it is attempted to assimilate systematic waste or
futility, as an end in life, into this organic complex, there presently
supervenes a revulsion. But this revulsion of the organism may be
avoided if the attention can be confined to the proximate, unreflected
purpose of dexterous or emulative exertion. Sports--hunting, angling,
athletic games, and the like--afford an exercise for dexterity and for
the emulative ferocity and astuteness characteristic of predatory life.
So long as the individual is but slightly gifted with reflection or
with a sense of the ulterior trend of his actions so long as his life
is substantially a life of naive impulsive action--so long the immediate
and unreflected purposefulness of sports, in the way of an expression of
dominance, will measurably satisfy his instinct of workmanship. This is
especially true if his dominant impulses are the unreflecting emulative
propensities of the predaceous temperament. At the same time the canons
of decorum will commend sports to him as expressions of a pecuniarily
blameless life. It is by meeting these two requirements, of ulterior
wastefulness and proximate purposefulness, that any given employment
holds its place as a traditional and habitual mode of decorous
recreation. In the sense that other forms of recreation and exercise
are morally impossible to persons of good breeding and delicate
sensibilities, then, sports are the best available means of recreation
under existing circumstances.

But those members of respectable society who advocate athletic games
commonly justify their attitude on this head to themselves and to their
neighbors on the ground that these games serve as an invaluable means of
development. They not only improve the contestant's physique, but it
is commonly added that they also foster a manly spirit, both in the
participants and in the spectators. Football is the particular game
which will probably first occur to any one in this community when the
question of the serviceability of athletic games is raised, as this form
of athletic contest is at present uppermost in the mind of those who
plead for or against games as a means of physical or moral salvation.
This typical athletic sport may, therefore, serve to illustrate the
bearing of athletics upon the development of the contestant's character
and physique. It has been said, not inaptly, that the relation of
football to physical culture is much the same as that of the bull-fight
to agriculture. Serviceability for these lusory institutions requires
sedulous training or breeding. The material used, whether brute or
human, is subjected to careful selection and discipline, in order to
secure and accentuate certain aptitudes and propensities which are
characteristic of the ferine state, and which tend to obsolescence under
domestication. This does not mean that the result in either case is
an all around and consistent rehabilitation of the ferine or barbarian
habit of mind and body. The result is rather a one-sided return to
barbarism or to the \emph{feroe natura}--a rehabilitation and accentuation
of those ferine traits which make for damage and desolation, without
a corresponding development of the traits which would serve the
individual's self-preservation and fullness of life in a ferine
environment. The culture bestowed in football gives a product of exotic
ferocity and cunning. It is a rehabilitation of the early barbarian
temperament, together with a suppression of those details of
temperament, which, as seen from the standpoint of the social and
economic exigencies, are the redeeming features of the savage character.

The physical vigor acquired in the training for athletic games--so far
as the training may be said to have this effect--is of advantage both
to the individual and to the collectivity, in that, other things being
equal, it conduces to economic serviceability. The spiritual traits
which go with athletic sports are likewise economically advantageous
to the individual, as contradistinguished from the interests of the
collectivity. This holds true in any community where these traits are
present in some degree in the population. Modern competition is in
large part a process of self-assertion on the basis of these traits of
predatory human nature. In the sophisticated form in which they enter
into the modern, peaceable emulation, the possession of these traits
in some measure is almost a necessary of life to the civilized man. But
while they are indispensable to the competitive individual, they are
not directly serviceable to the community. So far as regards the
serviceability of the individual for the purposes of the collective
life, emulative efficiency is of use only indirectly if at all. Ferocity
and cunning are of no use to the community except in its hostile
dealings with other communities; and they are useful to the individual
only because there is so large a proportion of the same traits actively
present in the human environment to which he is exposed. Any individual
who enters the competitive struggle without the due endowment of these
traits is at a disadvantage, somewhat as a hornless steer would find
himself at a disadvantage in a drove of horned cattle.

The possession and the cultivation of the predatory traits of character
may, of course, be desirable on other than economic grounds. There is a
prevalent aesthetic or ethical predilection for the barbarian aptitudes,
and the traits in question minister so effectively to this predilection
that their serviceability in the aesthetic or ethical respect probably
offsets any economic unserviceability which they may give. But for the
present purpose that is beside the point. Therefore nothing is said here
as to the desirability or advisability of sports on the whole, or as to
their value on other than economic grounds.

In popular apprehension there is much that is admirable in the type
of manhood which the life of sport fosters. There is self-reliance and
good-fellowship, so termed in the somewhat loose colloquial use of
the words. From a different point of view the qualities currently so
characterized might be described as truculence and clannishness. The
reason for the current approval and admiration of these manly qualities,
as well as for their being called manly, is the same as the reason for
their usefulness to the individual. The members of the community, and
especially that class of the community which sets the pace in canons of
taste, are endowed with this range of propensities in sufficient measure
to make their absence in others felt as a shortcoming, and to make
their possession in an exceptional degree appreciated as an attribute of
superior merit. The traits of predatory man are by no means obsolete in
the common run of modern populations. They are present and can be called
out in bold relief at any time by any appeal to the sentiments in
which they express themselves--unless this appeal should clash with the
specific activities that make up our habitual occupations and comprise
the general range of our everyday interests. The common run of the
population of any industrial community is emancipated from these,
economically considered, untoward propensities only in the sense
that, through partial and temporary disuse, they have lapsed into the
background of sub-conscious motives. With varying degrees of potency in
different individuals, they remain available for the aggressive shaping
of men's actions and sentiments whenever a stimulus of more than
everyday intensity comes in to call them forth. And they assert
themselves forcibly in any case where no occupation alien to the
predatory culture has usurped the individual's everyday range of
interest and sentiment. This is the case among the leisure class and
among certain portions of the population which are ancillary to that
class. Hence the facility with which any new accessions to the leisure
class take to sports; and hence the rapid growth of sports and of
the sporting sentient in any industrial community where wealth has
accumulated sufficiently to exempt a considerable part of the population
from work.

A homely and familiar fact may serve to show that the predaceous impulse
does not prevail in the same degree in all classes. Taken simply as a
feature of modern life, the habit of carrying a walking-stick may seem
at best a trivial detail; but the usage has a significance for the point
in question. The classes among whom the habit most prevails--the classes
with whom the walking-stick is associated in popular apprehension--are
the men of the leisure class proper, sporting men, and the lower-class
delinquents. To these might perhaps be added the men engaged in the
pecuniary employments. The same is not true of the common run of men
engaged in industry and it may be noted by the way that women do not
carry a stick except in case of infirmity, where it has a use of a
different kind. The practice is of course in great measure a matter
of polite usage; but the basis of polite usage is, in turn, the
proclivities of the class which sets the pace in polite usage. The
walking-stick serves the purpose of an advertisement that the bearer's
hands are employed otherwise than in useful effort, and it therefore has
utility as an evidence of leisure. But it is also a weapon, and it meets
a felt need of barbarian man on that ground. The handling of so tangible
and primitive a means of offense is very comforting to any one who is
gifted with even a moderate share of ferocity. The exigencies of
the language make it impossible to avoid an apparent implication of
disapproval of the aptitudes, propensities, and expressions of life here
under discussion. It is, however, not intended to imply anything in the
way of deprecation or commendation of any one of these phases of human
character or of the life process. The various elements of the prevalent
human nature are taken up from the point of view of economic theory,
and the traits discussed are gauged and graded with regard to their
immediate economic bearing on the facility of the collective life
process. That is to say, these phenomena are here apprehended from
the economic point of view and are valued with respect to their direct
action in furtherance or hindrance of a more perfect adjustment of the
human collectivity to the environment and to the institutional structure
required by the economic situation of the collectivity for the present
and for the immediate future. For these purposes the traits handed down
from the predatory culture are less serviceable than might be. Although
even in this connection it is not to be overlooked that the energetic
aggressiveness and pertinacity of predatory man is a heritage of no mean
value. The economic value--with some regard also to the social value in
the narrower sense--of these aptitudes and propensities is attempted to
be passed upon without reflecting on their value as seen from another
point of view. When contrasted with the prosy mediocrity of the
latter-day industrial scheme of life, and judged by the accredited
standards of morality, and more especially by the standards of
aesthetics and of poetry, these survivals from a more primitive type of
manhood may have a very different value from that here assigned them.
But all this being foreign to the purpose in hand, no expression
of opinion on this latter head would be in place here. All that is
admissible is to enter the caution that these standards of excellence,
which are alien to the present purpose, must not be allowed to influence
our economic appreciation of these traits of human character or of the
activities which foster their growth. This applies both as regards those
persons who actively participate in sports and those whose sporting
experience consists in contemplation only. What is here said of
the sporting propensity is likewise pertinent to sundry reflections
presently to be made in this connection on what would colloquially be
known as the religious life.

The last paragraph incidentally touches upon the fact that everyday
speech can scarcely be employed in discussing this class of aptitudes
and activities without implying deprecation or apology. The fact is
significant as showing the habitual attitude of the dispassionate common
man toward the propensities which express themselves in sports and in
exploit generally. And this is perhaps as convenient a place as any
to discuss that undertone of deprecation which runs through all the
voluminous discourse in defense or in laudation of athletic sports, as
well as of other activities of a predominantly predatory character. The
same apologetic frame of mind is at least beginning to be observable in
the spokesmen of most other institutions handed down from the barbarian
phase of life. Among these archaic institutions which are felt to need
apology are comprised, with others, the entire existing system of the
distribution of wealth, together with the resulting class distinction of
status; all or nearly all forms of consumption that come under the head
of conspicuous waste; the status of women under the patriarchal system;
and many features of the traditional creeds and devout observances,
especially the exoteric expressions of the creed and the naive
apprehension of received observances. What is to be said in this
connection of the apologetic attitude taken in commending sports and
the sporting character will therefore apply, with a suitable change in
phraseology, to the apologies offered in behalf of these other, related
elements of our social heritage.

There is a feeling--usually vague and not commonly avowed in so many
words by the apologist himself, but ordinarily perceptible in the manner
of his discourse--that these sports, as well as the general range of
predaceous impulses and habits of thought which underlie the sporting
character, do not altogether commend themselves to common sense. "As
to the majority of murderers, they are very incorrect characters." This
aphorism offers a valuation of the predaceous temperament, and of the
disciplinary effects of its overt expression and exercise, as seen from
the moralist's point of view. As such it affords an indication of what
is the deliverance of the sober sense of mature men as to the degree
of availability of the predatory habit of mind for the purposes of the
collective life. It is felt that the presumption is against any activity
which involves habituation to the predatory attitude, and that the
burden of proof lies with those who speak for the rehabilitation of the
predaceous temper and for the practices which strengthen it. There is a
strong body of popular sentiment in favor of diversions and enterprises
of the kind in question; but there is at the same time present in
the community a pervading sense that this ground of sentiment wants
legitimation. The required legitimation is ordinarily sought by
showing that although sports are substantially of a predatory, socially
disintegrating effect; although their proximate effect runs in
the direction of reversion to propensities that are industrially
disserviceable; yet indirectly and remotely--by some not readily
comprehensible process of polar induction, or counter-irritation
perhaps--sports are conceived to foster a habit of mind that is
serviceable for the social or industrial purpose. That is to say,
although sports are essentially of the nature of invidious exploit, it
is presumed that by some remote and obscure effect they result in the
growth of a temperament conducive to non-invidious work. It is commonly
attempted to show all this empirically or it is rather assumed that this
is the empirical generalization which must be obvious to any one who
cares to see it. In conducting the proof of this thesis the treacherous
ground of inference from cause to effect is somewhat shrewdly avoided,
except so far as to show that the "manly virtues" spoken of above
are fostered by sports. But since it is these manly virtues that are
(economically) in need of legitimation, the chain of proof breaks
off where it should begin. In the most general economic terms, these
apologies are an effort to show that, in spite of the logic of the
thing, sports do in fact further what may broadly be called workmanship.
So long as he has not succeeded in persuading himself or others that
this is their effect the thoughtful apologist for sports will not rest
content, and commonly, it is to be admitted, he does not rest content.
His discontent with his own vindication of the practice in question is
ordinarily shown by his truculent tone and by the eagerness with which
he heaps up asseverations in support of his position. But why are
apologies needed? If there prevails a body of popular sentient in
favor of sports, why is not that fact a sufficient legitimation? The
protracted discipline of prowess to which the race has been subjected
under the predatory and quasi-peaceable culture has transmitted to the
men of today a temperament that finds gratification in these expressions
of ferocity and cunning. So, why not accept these sports as legitimate
expressions of a normal and wholesome human nature? What other norm is
there that is to be lived up to than that given in the aggregate range
of propensities that express themselves in the sentiments of this
generation, including the hereditary strain of prowess? The ulterior
norm to which appeal is taken is the instinct of workmanship, which is
an instinct more fundamental, of more ancient prescription, than
the propensity to predatory emulation. The latter is but a special
development of the instinct of workmanship, a variant, relatively late
and ephemeral in spite of its great absolute antiquity. The emulative
predatory impulse--or the instinct of sportsmanship, as it might well
be called--is essentially unstable in comparison with the primordial
instinct of workmanship out of which it has been developed and
differentiated. Tested by this ulterior norm of life, predatory
emulation, and therefore the life of sports, falls short.

The manner and the measure in which the institution of a leisure class
conduces to the conservation of sports and invidious exploit can of
course not be succinctly stated. From the evidence already recited it
appears that, in sentient and inclinations, the leisure class is more
favorable to a warlike attitude and animus than the industrial classes.
Something similar seems to be true as regards sports. But it is chiefly
in its indirect effects, though the canons of decorous living, that the
institution has its influence on the prevalent sentiment with respect to
the sporting life. This indirect effect goes almost unequivocally in
the direction of furthering a survival of the predatory temperament
and habits; and this is true even with respect to those variants of
the sporting life which the higher leisure-class code of proprieties
proscribes; as, \emph{e.g.}, prize-fighting, cock-fighting, and other
like vulgar expressions of the sporting temper. Whatever the latest
authenticated schedule of detail proprieties may say, the accredited
canons of decency sanctioned by the institution say without equivocation
that emulation and waste are good and their opposites are disreputable.
In the crepuscular light of the social nether spaces the details of the
code are not apprehended with all the facility that might be desired,
and these broad underlying canons of decency are therefore applied
somewhat unreflectingly, with little question as to the scope of their
competence or the exceptions that have been sanctioned in detail.

Addiction to athletic sports, not only in the way of direct
participation, but also in the way of sentiment and moral support, is,
in a more or less pronounced degree, a characteristic of the leisure
class; and it is a trait which that class shares with the lower-class
delinquents, and with such atavistic elements throughout the body of
the community as are endowed with a dominant predaceous trend. Few
individuals among the populations of Western civilized countries are
so far devoid of the predaceous instinct as to find no diversion in
contemplating athletic sports and games, but with the common run of
individuals among the industrial classes the inclination to sports
does not assert itself to the extent of constituting what may fairly
be called a sporting habit. With these classes sports are an occasional
diversion rather than a serious feature of life. This common body of the
people can therefore not be said to cultivate the sporting propensity.
Although it is not obsolete in the average of them, or even in any
appreciable number of individuals, yet the predilection for sports in
the commonplace industrial classes is of the nature of a reminiscence,
more or less diverting as an occasional interest, rather than a vital
and permanent interest that counts as a dominant factor in shaping
the organic complex of habits of thought into which it enters. As it
manifests itself in the sporting life of today, this propensity may not
appear to be an economic factor of grave consequence. Taken simply by
itself it does not count for a great deal in its direct effects on the
industrial efficiency or the consumption of any given individual; but
the prevalence and the growth of the type of human nature of which this
propensity is a characteristic feature is a matter of some consequence.
It affects the economic life of the collectivity both as regards the
rate of economic development and as regards the character of the results
attained by the development. For better or worse, the fact that the
popular habits of thought are in any degree dominated by this type of
character can not but greatly affect the scope, direction, standards,
and ideals of the collective economic life, as well as the degree of
adjustment of the collective life to the environment.

Something to a like effect is to be said of other traits that go to make
up the barbarian character. For the purposes of economic theory, these
further barbarian traits may be taken as concomitant variations of that
predaceous temper of which prowess is an expression. In great measure
they are not primarily of an economic character, nor do they have much
direct economic bearing. They serve to indicate the stage of economic
evolution to which the individual possessed of them is adapted. They
are of importance, therefore, as extraneous tests of the degree of
adaptation of the character in which they are comprised to the economic
exigencies of today, but they are also to some extent important as
being aptitudes which themselves go to increase or diminish the economic
serviceability of the individual.

As it finds expression in the life of the barbarian, prowess manifests
itself in two main directions--force and fraud. In varying degrees these
two forms of expression are similarly present in modern warfare, in the
pecuniary occupations, and in sports and games. Both lines of aptitudes
are cultivated and strengthened by the life of sport as well as by the
more serious forms of emulative life. Strategy or cunning is an element
invariably present in games, as also in warlike pursuits and in the
chase. In all of these employments strategy tends to develop into
finesse and chicanery. Chicanery, falsehood, browbeating, hold a
well-secured place in the method of procedure of any athletic contest
and in games generally. The habitual employment of an umpire, and
the minute technical regulations governing the limits and details of
permissible fraud and strategic advantage, sufficiently attest the fact
that fraudulent practices and attempts to overreach one's opponents
are not adventitious features of the game. In the nature of the case
habituation to sports should conduce to a fuller development of
the aptitude for fraud; and the prevalence in the community of that
predatory temperament which inclines men to sports connotes a prevalence
of sharp practice and callous disregard of the interests of others,
individually and collectively. Resort to fraud, in any guise and under
any legitimation of law or custom, is an expression of a narrowly
self-regarding habit of mind. It is needless to dwell at any length on
the economic value of this feature of the sporting character.

In this connection it is to be noted that the most obvious
characteristic of the physiognomy affected by athletic and other
sporting men is that of an extreme astuteness. The gifts and exploits
of Ulysses are scarcely second to those of Achilles, either in their
substantial furtherance of the game or in the éclat which they give the
astute sporting man among his associates. The pantomime of astuteness
is commonly the first step in that assimilation to the professional
sporting man which a youth undergoes after matriculation in any
reputable school, of the secondary or the higher education, as the case
may be. And the physiognomy of astuteness, as a decorative feature,
never ceases to receive the thoughtful attention of men whose serious
interest lies in athletic games, races, or other contests of a similar
emulative nature. As a further indication of their spiritual kinship,
it may be pointed out that the members of the lower delinquent class
usually show this physiognomy of astuteness in a marked degree, and that
they very commonly show the same histrionic exaggeration of it that is
often seen in the young candidate for athletic honors. This, by the
way, is the most legible mark of what is vulgarly called "toughness" in
youthful aspirants for a bad name.

The astute man, it may be remarked, is of no economic value to the
community--unless it be for the purpose of sharp practice in dealings
with other communities. His functioning is not a furtherance of the
generic life process. At its best, in its direct economic bearing, it is
a conversion of the economic substance of the collectivity to a growth
alien to the collective life process--very much after the analogy of
what in medicine would be called a benign tumor, with some tendency to
transgress the uncertain line that divides the benign from the malign
growths. The two barbarian traits, ferocity and astuteness, go to make
up the predaceous temper or spiritual attitude. They are the expressions
of a narrowly self-regarding habit of mind. Both are highly serviceable
for individual expediency in a life looking to invidious success. Both
also have a high aesthetic value. Both are fostered by the pecuniary
culture. But both alike are of no use for the purposes of the collective
life.




\chapter{The Belief in Luck}
The gambling propensity is another subsidiary trait of the barbarian
temperament. It is a concomitant variation of character of almost
universal prevalence among sporting men and among men given to warlike
and emulative activities generally. This trait also has a direct
economic value. It is recognized to be a hindrance to the highest
industrial efficiency of the aggregate in any community where it
prevails in an appreciable degree. The gambling proclivity is doubtfully
to be classed as a feature belonging exclusively to the predatory type
of human nature. The chief factor in the gambling habit is the belief in
luck; and this belief is apparently traceable, at least in its elements,
to a stage in human evolution antedating the predatory culture. It may
well have been under the predatory culture that the belief in luck was
developed into the form in which it is present, as the chief element of
the gambling proclivity, in the sporting temperament. It probably owes
the specific form under which it occurs in the modern culture to the
predatory discipline. But the belief in luck is in substance a habit
of more ancient date than the predatory culture. It is one form of the
artistic apprehension of things. The belief seems to be a trait carried
over in substance from an earlier phase into the barbarian culture,
and transmuted and transmitted through that culture to a later stage
of human development under a specific form imposed by the predatory
discipline. But in any case, it is to be taken as an archaic trait,
inherited from a more or less remote past, more or less incompatible
with the requirements of the modern industrial process, and more or less
of a hindrance to the fullest efficiency of the collective economic life
of the present.

While the belief in luck is the basis of the gambling habit, it is not
the only element that enters into the habit of betting. Betting on the
issue of contests of strength and skill proceeds on a further motive,
without which the belief in luck would scarcely come in as a prominent
feature of sporting life. This further motive is the desire of the
anticipated winner, or the partisan of the anticipated winning side, to
heighten his side's ascendency at the cost of the loser. Not only does
the stronger side score a more signal victory, and the losing side
suffer a more painful and humiliating defeat, in proportion as the
pecuniary gain and loss in the wager is large; although this alone is
a consideration of material weight. But the wager is commonly laid also
with a view, not avowed in words nor even recognized in set terms in
petto, to enhancing the chances of success for the contestant on which
it is laid. It is felt that substance and solicitude expended to
this end can not go for naught in the issue. There is here a special
manifestation of the instinct of workmanship, backed by an even more
manifest sense that the animistic congruity of things must decide for a
victorious outcome for the side in whose behalf the propensity inherent
in events has been propitiated and fortified by so much of conative
and kinetic urging. This incentive to the wager expresses itself freely
under the form of backing one's favorite in any contest, and it is
unmistakably a predatory feature. It is as ancillary to the predaceous
impulse proper that the belief in luck expresses itself in a wager. So
that it may be set down that in so far as the belief in luck comes
to expression in the form of laying a wager, it is to be accounted an
integral element of the predatory type of character. The belief is, in
its elements, an archaic habit which belongs substantially to early,
undifferentiated human nature; but when this belief is helped out by the
predatory emulative impulse, and so is differentiated into the specific
form of the gambling habit, it is, in this higher-developed and specific
form, to be classed as a trait of the barbarian character.

The belief in luck is a sense of fortuitous necessity in the sequence
of phenomena. In its various mutations and expressions, it is of very
serious importance for the economic efficiency of any community in which
it prevails to an appreciable extent. So much so as to warrant a more
detailed discussion of its origin and content and of the bearing of its
various ramifications upon economic structure and function, as well as
a discussion of the relation of the leisure class to its growth,
differentiation, and persistence. In the developed, integrated form
in which it is most readily observed in the barbarian of the predatory
culture or in the sporting man of modern communities, the belief
comprises at least two distinguishable elements--which are to be taken
as two different phases of the same fundamental habit of thought, or as
the same psychological factor in two successive phases of its evolution.
The fact that these two elements are successive phases of the same
general line of growth of belief does not hinder their coexisting in the
habits of thought of any given individual. The more primitive form
(or the more archaic phase) is an incipient animistic belief, or an
animistic sense of relations and things, that imputes a quasi-personal
character to facts. To the archaic man all the obtrusive and obviously
consequential objects and facts in his environment have a quasi-personal
individuality. They are conceived to be possessed of volition, or rather
of propensities, which enter into the complex of causes and affect
events in an inscrutable manner. The sporting man's sense of luck and
chance, or of fortuitous necessity, is an inarticulate or inchoate
animism. It applies to objects and situations, often in a very vague
way; but it is usually so far defined as to imply the possibility of
propitiating, or of deceiving and cajoling, or otherwise disturbing the
holding of propensities resident in the objects which constitute the
apparatus and accessories of any game of skill or chance. There are few
sporting men who are not in the habit of wearing charms or talismans to
which more or less of efficacy is felt to belong. And the proportion is
not much less of those who instinctively dread the "hoodooing" of the
contestants or the apparatus engaged in any contest on which they lay a
wager; or who feel that the fact of their backing a given contestant or
side in the game does and ought to strengthen that side; or to whom the
"mascot" which they cultivate means something more than a jest.

In its simple form the belief in luck is this instinctive sense of an
inscrutable teleological propensity in objects or situations. Objects or
events have a propensity to eventuate in a given end, whether this end
or objective point of the sequence is conceived to be fortuitously given
or deliberately sought. From this simple animism the belief shades off
by insensible gradations into the second, derivative form or phase above
referred to, which is a more or less articulate belief in an inscrutable
preternatural agency. The preternatural agency works through the visible
objects with which it is associated, but is not identified with these
objects in point of individuality. The use of the term "preternatural
agency" here carries no further implication as to the nature of the
agency spoken of as preternatural. This is only a farther development of
animistic belief. The preternatural agency is not necessarily conceived
to be a personal agent in the full sense, but it is an agency which
partakes of the attributes of personality to the extent of somewhat
arbitrarily influencing the outcome of any enterprise, and especially
of any contest. The pervading belief in the \emph{hamingia} or \emph{gipta}
(\emph{gaefa, authna}) which lends so much of color to the Icelandic sagas
specifically, and to early Germanic folk-legends, is an illustration of
this sense of an extra-physical propensity in the course of events.

In this expression or form of the belief the propensity is scarcely
personified although to a varying extent an individuality is imputed to
it; and this individuated propensity is sometimes conceived to yield to
circumstances, commonly to circumstances of a spiritual or preternatural
character. A well-known and striking exemplification of the belief--in
a fairly advanced stage of differentiation and involving an
anthropomorphic personification of the preternatural agent appealed
to--is afforded by the wager of battle. Here the preternatural agent was
conceived to act on request as umpire, and to shape the outcome of the
contest in accordance with some stipulated ground of decision, such as
the equity or legality of the respective contestants' claims. The like
sense of an inscrutable but spiritually necessary tendency in events
is still traceable as an obscure element in current popular belief, as
shown, for instance, by the well-accredited maxim, "Thrice is he
armed who knows his quarrel just,"--a maxim which retains much of its
significance for the average unreflecting person even in the civilized
communities of today. The modern reminiscence of the belief in the
\emph{hamingia}, or in the guidance of an unseen hand, which is traceable in
the acceptance of this maxim is faint and perhaps uncertain; and it
seems in any case to be blended with other psychological moments that
are not clearly of an animistic character.

For the purpose in hand it is unnecessary to look more closely into the
psychological process or the ethnological line of descent by which the
later of these two animistic apprehensions of propensity is derived
from the earlier. This question may be of the gravest importance to
folk-psychology or to the theory of the evolution of creeds and cults.
The same is true of the more fundamental question whether the two
are related at all as successive phases in a sequence of development.
Reference is here made to the existence of these questions only to
remark that the interest of the present discussion does not lie in that
direction. So far as concerns economic theory, these two elements or
phases of the belief in luck, or in an extra-causal trend or propensity
in things, are of substantially the same character. They have an
economic significance as habits of thought which affect the individual's
habitual view of the facts and sequences with which he comes in contact,
and which thereby affect the individual's serviceability for the
industrial purpose. Therefore, apart from all question of the beauty,
worth, or beneficence of any animistic belief, there is place for
a discussion of their economic bearing on the serviceability of the
individual as an economic factor, and especially as an industrial agent.

It has already been noted in an earlier connection, that in order to
have the highest serviceability in the complex industrial processes of
today, the individual must be endowed with the aptitude and the habit
of readily apprehending and relating facts in terms of causal sequence.
Both as a whole and in its details, the industrial process is a process
of quantitative causation. The "intelligence" demanded of the workman,
as well as of the director of an industrial process, is little else
than a degree of facility in the apprehension of and adaptation to a
quantitatively determined causal sequence. This facility of apprehension
and adaptation is what is lacking in stupid workmen, and the growth
of this facility is the end sought in their education--so far as their
education aims to enhance their industrial efficiency.

In so far as the individual's inherited aptitudes or his training
incline him to account for facts and sequences in other terms than those
of causation or matter-of-fact, they lower his productive efficiency or
industrial usefulness. This lowering of efficiency through a penchant
for animistic methods of apprehending facts is especially apparent when
taken in the mass-when a given population with an animistic turn is
viewed as a whole. The economic drawbacks of animism are more patent and
its consequences are more far-reaching under the modern system of large
industry than under any other. In the modern industrial communities,
industry is, to a constantly increasing extent, being organized in a
comprehensive system of organs and functions mutually conditioning one
another; and therefore freedom from all bias in the causal apprehension
of phenomena grows constantly more requisite to efficiency on the
part of the men concerned in industry. Under a system of handicraft an
advantage in dexterity, diligence, muscular force, or endurance may, in
a very large measure, offset such a bias in the habits of thought of the
workmen.

Similarly in agricultural industry of the traditional kind, which
closely resembles handicraft in the nature of the demands made upon
the workman. In both, the workman is himself the prime mover chiefly
depended upon, and the natural forces engaged are in large part
apprehended as inscrutable and fortuitous agencies, whose working lies
beyond the workman's control or discretion. In popular apprehension
there is in these forms of industry relatively little of the industrial
process left to the fateful swing of a comprehensive mechanical sequence
which must be comprehended in terms of causation and to which the
operations of industry and the movements of the workmen must be adapted.
As industrial methods develop, the virtues of the handicraftsman count
for less and less as an offset to scanty intelligence or a halting
acceptance of the sequence of cause and effect. The industrial
organization assumes more and more of the character of a mechanism, in
which it is man's office to discriminate and select what natural forces
shall work out their effects in his service. The workman's part in
industry changes from that of a prime mover to that of discrimination
and valuation of quantitative sequences and mechanical facts. The
faculty of a ready apprehension and unbiased appreciation of causes in
his environment grows in relative economic importance and any element in
the complex of his habits of thought which intrudes a bias at
variance with this ready appreciation of matter-of-fact sequence gains
proportionately in importance as a disturbing element acting to lower
his industrial usefulness. Through its cumulative effect upon the
habitual attitude of the population, even a slight or inconspicuous bias
towards accounting for everyday facts by recourse to other ground than
that of quantitative causation may work an appreciable lowering of the
collective industrial efficiency of a community.

The animistic habit of mind may occur in the early, undifferentiated
form of an inchoate animistic belief, or in the later and more highly
integrated phase in which there is an anthropomorphic personification of
the propensity imputed to facts. The industrial value of such a lively
animistic sense, or of such recourse to a preternatural agency or the
guidance of an unseen hand, is of course very much the same in either
case. As affects the industrial serviceability of the individual, the
effect is of the same kind in either case; but the extent to which
this habit of thought dominates or shapes the complex of his habits of
thought varies with the degree of immediacy, urgency, or exclusiveness
with which the individual habitually applies the animistic or
anthropomorphic formula in dealing with the facts of his environment.
The animistic habit acts in all cases to blur the appreciation of causal
sequence; but the earlier, less reflected, less defined animistic sense
of propensity may be expected to affect the intellectual processes
of the individual in a more pervasive way than the higher forms of
anthropomorphism. Where the animistic habit is present in the naive
form, its scope and range of application are not defined or limited.
It will therefore palpably affect his thinking at every turn of the
person's life--wherever he has to do with the material means of life.
In the later, maturer development of animism, after it has been defined
through the process of anthropomorphic elaboration, when its application
has been limited in a somewhat consistent fashion to the remote and the
invisible, it comes about that an increasing range of everyday facts are
provisionally accounted for without recourse to the preternatural agency
in which a cultivated animism expresses itself. A highly integrated,
personified preternatural agency is not a convenient means of handling
the trivial occurrences of life, and a habit is therefore easily fallen
into of accounting for many trivial or vulgar phenomena in terms of
sequence. The provisional explanation so arrived at is by neglect
allowed to stand as definitive, for trivial purposes, until special
provocation or perplexity recalls the individual to his allegiance. But
when special exigencies arise, that is to say, when there is peculiar
need of a full and free recourse to the law of cause and effect, then
the individual commonly has recourse to the preternatural agency as a
universal solvent, if he is possessed of an anthropomorphic belief.

The extra-causal propensity or agent has a very high utility as a
recourse in perplexity, but its utility is altogether of a non-economic
kind. It is especially a refuge and a fund of comfort where it has
attained the degree of consistency and specialization that belongs to
an anthropomorphic divinity. It has much to commend it even on other
grounds than that of affording the perplexed individual a means of
escape from the difficulty of accounting for phenomena in terms of
causal sequence. It would scarcely be in place here to dwell on the
obvious and well-accepted merits of an anthropomorphic divinity, as seen
from the point of view of the aesthetic, moral, or spiritual interest,
or even as seen from the less remote standpoint of political, military,
or social policy. The question here concerns the less picturesque and
less urgent economic value of the belief in such a preternatural agency,
taken as a habit of thought which affects the industrial serviceability
of the believer. And even within this narrow, economic range, the
inquiry is perforce confined to the immediate bearing of this habit
of thought upon the believer's workmanlike serviceability, rather than
extended to include its remoter economic effects. These remoter effects
are very difficult to trace. The inquiry into them is so encumbered with
current preconceptions as to the degree in which life is enhanced by
spiritual contact with such a divinity, that any attempt to inquire into
their economic value must for the present be fruitless.

The immediate, direct effect of the animistic habit of thought upon the
general frame of mind of the believer goes in the direction of lowering
his effective intelligence in the respect in which intelligence is of
especial consequence for modern industry. The effect follows, in varying
degree, whether the preternatural agent or propensity believed in is
of a higher or a lower cast. This holds true of the barbarian's and
the sporting man's sense of luck and propensity, and likewise of the
somewhat higher developed belief in an anthropomorphic divinity, such as
is commonly possessed by the same class. It must be taken to hold true
also--though with what relative degree of cogency is not easy to say--of
the more adequately developed anthropomorphic cults, such as appeal
to the devout civilized man. The industrial disability entailed by a
popular adherence to one of the higher anthropomorphic cults may be
relatively slight, but it is not to be overlooked. And even these
high-class cults of the Western culture do not represent the last
dissolving phase of this human sense of extra-causal propensity. Beyond
these the same animistic sense shows itself also in such attenuations of
anthropomorphism as the eighteenth-century appeal to an order of nature
and natural rights, and in their modern representative, the ostensibly
post-Darwinian concept of a meliorative trend in the process of
evolution. This animistic explanation of phenomena is a form of the
fallacy which the logicians knew by the name of \emph{ignava ratio}. For
the purposes of industry or of science it counts as a blunder in the
apprehension and valuation of facts. Apart from its direct industrial
consequences, the animistic habit has a certain significance for
economic theory on other grounds. (1) It is a fairly reliable indication
of the presence, and to some extent even of the degree of potency,
of certain other archaic traits that accompany it and that are of
substantial economic consequence; and (2) the material consequences of
that code of devout proprieties to which the animistic habit gives rise
in the development of an anthropomorphic cult are of importance both
(a) as affecting the community's consumption of goods and the prevalent
canons of taste, as already suggested in an earlier chapter, and (b) by
inducing and conserving a certain habitual recognition of the relation
to a superior, and so stiffening the current sense of status and
allegiance.

As regards the point last named (b), that body of habits of thought
which makes up the character of any individual is in some sense an
organic whole. A marked variation in a given direction at any one point
carries with it, as its correlative, a concomitant variation in the
habitual expression of life in other directions or other groups of
activities. These various habits of thought, or habitual expressions
of life, are all phases of the single life sequence of the individual;
therefore a habit formed in response to a given stimulus will
necessarily affect the character of the response made to other stimuli.
A modification of human nature at any one point is a modification of
human nature as a whole. On this ground, and perhaps to a still greater
extent on obscurer grounds that can not be discussed here, there are
these concomitant variations as between the different traits of human
nature. So, for instance, barbarian peoples with a well-developed
predatory scheme of life are commonly also possessed of a strong
prevailing animistic habit, a well-formed anthropomorphic cult, and
a lively sense of status. On the other hand, anthropomorphism and
the realizing sense of an animistic propensity in material are less
obtrusively present in the life of the peoples at the cultural stages
which precede and which follow the barbarian culture. The sense of
status is also feebler; on the whole, in peaceable communities. It is to
be remarked that a lively, but slightly specialized, animistic belief
is to be found in most if not all peoples living in the ante-predatory,
savage stage of culture. The primitive savage takes his animism less
seriously than the barbarian or the degenerate savage. With him
it eventuates in fantastic myth-making, rather than in coercive
superstition. The barbarian culture shows sportsmanship, status, and
anthropomorphism. There is commonly observable a like concomitance of
variations in the same respects in the individual temperament of men in
the civilized communities of today. Those modern representatives of
the predaceous barbarian temper that make up the sporting element are
commonly believers in luck; at least they have a strong sense of an
animistic propensity in things, by force of which they are given to
gambling. So also as regards anthropomorphism in this class. Such of
them as give in their adhesion to some creed commonly attach themselves
to one of the naively and consistently anthropomorphic creeds; there
are relatively few sporting men who seek spiritual comfort in the less
anthropomorphic cults, such as the Unitarian or the Universalist.

Closely bound up with this correlation of anthropomorphism and prowess
is the fact that anthropomorphic cults act to conserve, if not to
initiate, habits of mind favorable to a regime of status. As regards
this point, it is quite impossible to say where the disciplinary effect
of the cult ends and where the evidence of a concomitance of variations
in inherited traits begins. In their finest development, the predatory
temperament, the sense of status, and the anthropomorphic cult all
together belong to the barbarian culture; and something of a mutual
causal relation subsists between the three phenomena as they come into
sight in communities on that cultural level. The way in which they recur
in correlation in the habits and attitudes of individuals and classes
today goes far to imply a like causal or organic relation between the
same psychological phenomena considered as traits or habits of the
individual. It has appeared at an earlier point in the discussion
that the relation of status, as a feature of social structure, is a
consequence of the predatory habit of life. As regards its line
of derivation, it is substantially an elaborated expression of the
predatory attitude. On the other hand, an anthropomorphic cult is a
code of detailed relations of status superimposed upon the concept of
a preternatural, inscrutable propensity in material things. So that, as
regards the external facts of its derivation, the cult may be taken as
an outgrowth of archaic man's pervading animistic sense, defined and in
some degree transformed by the predatory habit of life, the result being
a personified preternatural agency, which is by imputation endowed with
a full complement of the habits of thought that characterize the man of
the predatory culture.

The grosser psychological features in the case, which have an immediate
bearing on economic theory and are consequently to be taken account
of here, are therefore: (a) as has appeared in an earlier chapter,
the predatory, emulative habit of mind here called prowess is but the
barbarian variant of the generically human instinct of workmanship,
which has fallen into this specific form under the guidance of a habit
of invidious comparison of persons; (b) the relation of status is a
formal expression of such an invidious comparison duly gauged and graded
according to a sanctioned schedule; (c) an anthropomorphic cult, in the
days of its early vigor at least, is an institution the characteristic
element of which is a relation of status between the human subject as
inferior and the personified preternatural agency as superior. With
this in mind, there should be no difficulty in recognizing the intimate
relation which subsists between these three phenomena of human nature
and of human life; the relation amounts to an identity in some of their
substantial elements. On the one hand, the system of status and the
predatory habit of life are an expression of the instinct of workmanship
as it takes form under a custom of invidious comparison; on the other
hand, the anthropomorphic cult and the habit of devout observances
are an expression of men's animistic sense of a propensity in material
things, elaborated under the guidance of substantially the same general
habit of invidious comparison. The two categories--the emulative habit
of life and the habit of devout observances--are therefore to be taken
as complementary elements of the barbarian type of human nature and of
its modern barbarian variants. They are expressions of much the same
range of aptitudes, made in response to different sets of stimuli.




\chapter{Devout Observances}
A discoursive rehearsal of certain incidents of modern life will show
the organic relation of the anthropomorphic cults to the barbarian
culture and temperament. It will likewise serve to show how the survival
and efficacy of the cults and he prevalence of their schedule of devout
observances are related to the institution of a leisure class and to the
springs of action underlying that institution. Without any intention to
commend or to deprecate the practices to be spoken of under the head of
devout observances, or the spiritual and intellectual traits of which
these observances are the expression, the everyday phenomena of current
anthropomorphic cults may be taken up from the point of view of the
interest which they have for economic theory. What can properly
be spoken of here are the tangible, external features of devout
observances. The moral, as well as the devotional value of the life of
faith lies outside of the scope of the present inquiry. Of course no
question is here entertained as to the truth or beauty of the creeds on
which the cults proceed. And even their remoter economic bearing can not
be taken up here; the subject is too recondite and of too grave import
to find a place in so slight a sketch.

Something has been said in an earlier chapter as to the influence which
pecuniary standards of value exert upon the processes of valuation
carried out on other bases, not related to the pecuniary interest. The
relation is not altogether one-sided. The economic standards or canons
of valuation are in their turn influenced by extra-economic standards of
value. Our judgments of the economic bearing of facts are to some extent
shaped by the dominant presence of these weightier interests. There is
a point of view, indeed, from which the economic interest is of weight
only as being ancillary to these higher, non-economic interests. For the
present purpose, therefore, some thought must be taken to isolate
the economic interest or the economic hearing of these phenomena of
anthropomorphic cults. It takes some effort to divest oneself of the
more serious point of view, and to reach an economic appreciation
of these facts, with as little as may be of the bias due to higher
interests extraneous to economic theory. In the discussion of the
sporting temperament, it has appeared that the sense of an animistic
propensity in material things and events is what affords the spiritual
basis of the sporting man's gambling habit. For the economic purpose,
this sense of propensity is substantially the same psychological element
as expresses itself, under a variety of forms, in animistic beliefs and
anthropomorphic creeds. So far as concerns those tangible psychological
features with which economic theory has to deal, the gambling spirit
which pervades the sporting element shades off by insensible gradations
into that frame of mind which finds gratification in devout observances.
As seen from the point of view of economic theory, the sporting
character shades off into the character of a religious devotee. Where
the betting man's animistic sense is helped out by a somewhat consistent
tradition, it has developed into a more or less articulate belief in
a preternatural or hyperphysical agency, with something of an
anthropomorphic content. And where this is the case, there is commonly
a perceptible inclination to make terms with the preternatural agency
by some approved method of approach and conciliation. This element of
propitiation and cajoling has much in common with the crasser forms
of worship--if not in historical derivation, at least in actual
psychological content. It obviously shades off in unbroken continuity
into what is recognized as superstitious practice and belief, and so
asserts its claim to kinship with the grosser anthropomorphic cults.

The sporting or gambling temperament, then, comprises some of the
substantial psychological elements that go to make a believer in creeds
and an observer of devout forms, the chief point of coincidence being
the belief in an inscrutable propensity or a preternatural interposition
in the sequence of events. For the purpose of the gambling practice the
belief in preternatural agency may be, and ordinarily is, less closely
formulated, especially as regards the habits of thought and the scheme
of life imputed to the preternatural agent; or, in other words, as
regards his moral character and his purposes in interfering in events.
With respect to the individuality or personality of the agency whose
presence as luck, or chance, or hoodoo, or mascot, etc., he feels and
sometimes dreads and endeavors to evade, the sporting man's views are
also less specific, less integrated and differentiated. The basis of his
gambling activity is, in great measure, simply an instinctive sense
of the presence of a pervasive extraphysical and arbitrary force or
propensity in things or situations, which is scarcely recognized as a
personal agent. The betting man is not infrequently both a believer
in luck, in this naive sense, and at the same time a pretty staunch
adherent of some form of accepted creed. He is especially prone to
accept so much of the creed as concerts the inscrutable power and the
arbitrary habits of the divinity which has won his confidence. In such a
case he is possessed of two, or sometimes more than two, distinguishable
phases of animism. Indeed, the complete series of successive phases of
animistic belief is to be found unbroken in the spiritual furniture
of any sporting community. Such a chain of animistic conceptions will
comprise the most elementary form of an instinctive sense of luck and
chance and fortuitous necessity at one end of the series, together with
the perfectly developed anthropomorphic divinity at the other end, with
all intervening stages of integration. Coupled with these beliefs in
preternatural agency goes an instinctive shaping of conduct to conform
with the surmised requirements of the lucky chance on the one hand,
and a more or less devout submission to the inscrutable decrees of the
divinity on the other hand.

There is a relationship in this respect between the sporting temperament
and the temperament of the delinquent classes; and the two are related
to the temperament which inclines to an anthropomorphic cult. Both
the delinquent and the sporting man are on the average more apt to be
adherents of some accredited creed, and are also rather more inclined
to devout observances, than the general average of the community. It is
also noticeable that unbelieving members of these classes show more of
a proclivity to become proselytes to some accredited faith than the
average of unbelievers. This fact of observation is avowed by the
spokesmen of sports, especially in apologizing for the more naively
predatory athletic sports. Indeed, it is somewhat insistently claimed as
a meritorious feature of sporting life that the habitual participants in
athletic games are in some degree peculiarly given to devout practices.
And it is observable that the cult to which sporting men and the
predaceous delinquent classes adhere, or to which proselytes from
these classes commonly attach themselves, is ordinarily not one of the
so-called higher faiths, but a cult which has to do with a thoroughly
anthropomorphic divinity. Archaic, predatory human nature is not
satisfied with abstruse conceptions of a dissolving personality that
shades off into the concept of quantitative causal sequence, such as the
speculative, esoteric creeds of Christendom impute to the First Cause,
Universal Intelligence, World Soul, or Spiritual Aspect. As an instance
of a cult of the character which the habits of mind of the athlete and
the delinquent require, may be cited that branch of the church militant
known as the Salvation Army. This is to some extent recruited from the
lower-class delinquents, and it appears to comprise also, among its
officers especially, a larger proportion of men with a sporting record
than the proportion of such men in the aggregate population of the
community.

College athletics afford a case in point. It is contended by exponents
of the devout element in college life--and there seems to be no ground
for disputing the claim--that the desirable athletic material afforded
by any student body in this country is at the same time predominantly
religious; or that it is at least given to devout observances to a
greater degree than the average of those students whose interest in
athletics and other college sports is less. This is what might be
expected on theoretical grounds. It may be remarked, by the way, that
from one point of view this is felt to reflect credit on the college
sporting life, on athletic games, and on those persons who occupy
themselves with these matters. It happens not frequently that college
sporting men devote themselves to religious propaganda, either as a
vocation or as a by-occupation; and it is observable that when this
happens they are likely to become propagandists of some one of the more
anthropomorphic cults. In their teaching they are apt to insist
chiefly on the personal relation of status which subsists between an
anthropomorphic divinity and the human subject.

This intimate relation between athletics and devout observance among
college men is a fact of sufficient notoriety; but it has a special
feature to which attention has not been called, although it is obvious
enough. The religious zeal which pervades much of the college sporting
element is especially prone to express itself in an unquestioning
devoutness and a naive and complacent submission to an inscrutable
Providence. It therefore by preference seeks affiliation with some one
of those lay religious organizations which occupy themselves with
the spread of the exoteric forms of faith--as, \emph{e.g.}, the Young Men's
Christian Association or the Young People's Society for Christian
Endeavor. These lay bodies are organized to further "practical"
religion; and as if to enforce the argument and firmly establish the
close relationship between the sporting temperament and the archaic
devoutness, these lay religious bodies commonly devote some appreciable
portion of their energies to the furtherance of athletic contests and
similar games of chance and skill. It might even be said that sports
of this kind are apprehended to have some efficacy as a means of grace.
They are apparently useful as a means of proselyting, and as a means of
sustaining the devout attitude in converts once made. That is to
say, the games which give exercise to the animistic sense and to the
emulative propensity help to form and to conserve that habit of mind to
which the more exoteric cults are congenial. Hence, in the hands of
the lay organizations, these sporting activities come to do duty as a
novitiate or a means of induction into that fuller unfolding of the
life of spiritual status which is the privilege of the full communicant
along.

That the exercise of the emulative and lower animistic proclivities are
substantially useful for the devout purpose seems to be placed beyond
question by the fact that the priesthood of many denominations is
following the lead of the lay organizations in this respect. Those
ecclesiastical organizations especially which stand nearest the lay
organizations in their insistence on practical religion have gone some
way towards adopting these or analogous practices in connection with the
traditional devout observances. So there are "boys' brigades," and other
organizations, under clerical sanction, acting to develop the emulative
proclivity and the sense of status in the youthful members of the
congregation. These pseudo-military organizations tend to elaborate and
accentuate the proclivity to emulation and invidious comparison, and so
strengthen the native facility for discerning and approving the relation
of personal mastery and subservience. And a believer is eminently a
person who knows how to obey and accept chastisement with good grace.
But the habits of thought which these practices foster and conserve
make up but one half of the substance of the anthropomorphic cults.
The other, complementary element of devout life--the animistic habit
of mind--is recruited and conserved by a second range of practices
organized under clerical sanction. These are the class of gambling
practices of which the church bazaar or raffle may be taken as the type.
As indicating the degree of legitimacy of these practices in connection
with devout observances proper, it is to be remarked that these raffles,
and the like trivial opportunities for gambling, seem to appeal with
more effect to the common run of the members of religious organizations
than they do to persons of a less devout habit of mind.

All this seems to argue, on the one hand, that the same temperament
inclines people to sports as inclines them to the anthropomorphic cults,
and on the other hand that the habituation to sports, perhaps especially
to athletic sports, acts to develop the propensities which find
satisfaction in devout observances. Conversely; it also appears that
habituation to these observances favors the growth of a proclivity
for athletic sports and for all games that give play to the habit of
invidious comparison and of the appeal to luck. Substantially the same
range of propensities finds expression in both these directions of
the spiritual life. That barbarian human nature in which the predatory
instinct and the animistic standpoint predominate is normally prone
to both. The predatory habit of mind involves an accentuated sense of
personal dignity and of the relative standing of individuals. The social
structure in which the predatory habit has been the dominant factor
in the shaping of institutions is a structure based on status. The
pervading norm in the predatory community's scheme of life is the
relation of superior and inferior, noble and base, dominant and
subservient persons and classes, master and slave. The anthropomorphic
cults have come down from that stage of industrial development and
have been shaped by the same scheme of economic differentiation--a
differentiation into consumer and producer--and they are pervaded by the
same dominant principle of mastery and subservience. The cults impute to
their divinity the habits of thought answering to the stage of economic
differentiation at which the cults took shape. The anthropomorphic
divinity is conceived to be punctilious in all questions of precedence
and is prone to an assertion of mastery and an arbitrary exercise of
power--an habitual resort to force as the final arbiter.

In the later and maturer formulations of the anthropomorphic creed this
imputed habit of dominance on the part of a divinity of awful presence
and inscrutable power is chastened into "the fatherhood of God." The
spiritual attitude and the aptitudes imputed to the preternatural agent
are still such as belong under the regime of status, but they now assume
the patriarchal cast characteristic of the quasi-peaceable stage of
culture. Still it is to be noted that even in this advanced phase of the
cult the observances in which devoutness finds expression consistently
aim to propitiate the divinity by extolling his greatness and glory and
by professing subservience and fealty. The act of propitiation or
of worship is designed to appeal to a sense of status imputed to the
inscrutable power that is thus approached. The propitiatory formulas
most in vogue are still such as carry or imply an invidious comparison.
A loyal attachment to the person of an anthropomorphic divinity endowed
with such an archaic human nature implies the like archaic propensities
in the devotee. For the purposes of economic theory, the relation of
fealty, whether to a physical or to an extraphysical person, is to be
taken as a variant of that personal subservience which makes up so large
a share of the predatory and the quasi-peaceable scheme of life.

The barbarian conception of the divinity, as a warlike chieftain
inclined to an overbearing manner of government, has been greatly
softened through the milder manners and the soberer habits of life that
characterize those cultural phases which lie between the early predatory
stage and the present. But even after this chastening of the devout
fancy, and the consequent mitigation of the harsher traits of conduct
and character that are currently imputed to the divinity, there still
remains in the popular apprehension of the divine nature and temperament
a very substantial residue of the barbarian conception. So it comes
about, for instance, that in characterizing the divinity and his
relations to the process of human life, speakers and writers are still
able to make effective use of similes borrowed from the vocabulary of
war and of the predatory manner of life, as well as of locutions which
involve an invidious comparison. Figures of speech of this import
are used with good effect even in addressing the less warlike modern
audiences, made up of adherents of the blander variants of the creed.
This effective use of barbarian epithets and terms of comparison by
popular speakers argues that the modern generation has retained a lively
appreciation of the dignity and merit of the barbarian virtues; and
it argues also that there is a degree of congruity between the devout
attitude and the predatory habit of mind. It is only on second thought,
if at all, that the devout fancy of modern worshippers revolts at the
imputation of ferocious and vengeful emotions and actions to the object
of their adoration. It is a matter of common observation that sanguinary
epithets applied to the divinity have a high aesthetic and honorific
value in the popular apprehension. That is to say, suggestions
which these epithets carry are very acceptable to our unreflecting
apprehension.
\begin{displayquote}
  Mine eyes have seen the glory of the coming of the Lord: \\
  He is trampling out the vintage where the grapes of wrath are stored; \\
  He hath loosed the fateful lightning of his terrible swift sword; \\
  His truth is marching on.
\end{displayquote}
The guiding habits of thought of a devout person move on the plane of an
archaic scheme of life which has outlived much of its usefulness for the
economic exigencies of the collective life of today. In so far as the
economic organization fits the exigencies of the collective life of
today, it has outlived the regime of status, and has no use and no place
for a relation of personal subserviency. So far as concerns the economic
efficiency of the community, the sentiment of personal fealty, and the
general habit of mind of which that sentiment is an expression, are
survivals which cumber the ground and hinder an adequate adjustment of
human institutions to the existing situation. The habit of mind which
best lends itself to the purposes of a peaceable, industrial community,
is that matter-of-fact temper which recognizes the value of material
facts simply as opaque items in the mechanical sequence. It is
that frame of mind which does not instinctively impute an animistic
propensity to things, nor resort to preternatural intervention as an
explanation of perplexing phenomena, nor depend on an unseen hand to
shape the course of events to human use. To meet the requirements of the
highest economic efficiency under modern conditions, the world process
must habitually be apprehended in terms of quantitative, dispassionate
force and sequence.

As seen from the point of view of the later economic exigencies,
devoutness is, perhaps in all cases, to be looked upon as a survival
from an earlier phase of associated life--a mark of arrested spiritual
development. Of course it remains true that in a community where the
economic structure is still substantially a system of status; where
the attitude of the average of persons in the community is consequently
shaped by and adapted to the relation of personal dominance and
personal subservience; or where for any other reason--of tradition or
of inherited aptitude--the population as a whole is strongly inclined to
devout observances; there a devout habit of mind in any individual, not
in excess of the average of the community, must be taken simply as
a detail of the prevalent habit of life. In this light, a devout
individual in a devout community can not be called a case of reversion,
since he is abreast of the average of the community. But as seen from
the point of view of the modern industrial situation, exceptional
devoutness--devotional zeal that rises appreciably above the average
pitch of devoutness in the community--may safely be set down as in all
cases an atavistic trait.

It is, of course, equally legitimate to consider these phenomena from
a different point of view. They may be appreciated for a different
purpose, and the characterization here offered may be turned about.
In speaking from the point of view of the devotional interest, or the
interest of devout taste, it may, with equal cogency, be said that
the spiritual attitude bred in men by the modern industrial life is
unfavorable to a free development of the life of faith. It might fairly
be objected to the later development of the industrial process that its
discipline tends to "materialism," to the elimination of filial piety.
From the aesthetic point of view, again, something to a similar purport
might be said. But, however legitimate and valuable these and the like
reflections may be for their purpose, they would not be in place in the
present inquiry, which is exclusively concerned with the valuation of
these phenomena from the economic point of view.

The grave economic significance of the anthropomorphic habit of mind
and of the addiction to devout observances must serve as apology for
speaking further on a topic which it can not but be distasteful to
discuss at all as an economic phenomenon in a community so devout as
ours. Devout observances are of economic importance as an index of a
concomitant variation of temperament, accompanying the predatory habit
of mind and so indicating the presence of industrially disserviceable
traits. They indicate the presence of a mental attitude which has a
certain economic value of its own by virtue of its influence upon
the industrial serviceability of the individual. But they are also of
importance more directly, in modifying the economic activities of the
community, especially as regards the distribution and consumption of
goods.

The most obvious economic bearing of these observances is seen in the
devout consumption of goods and services. The consumption of ceremonial
paraphernalia required by any cult, in the way of shrines, temples,
churches, vestments, sacrifices, sacraments, holiday attire, etc.,
serves no immediate material end. All this material apparatus may,
therefore, without implying deprecation, be broadly characterized as
items of conspicuous waste. The like is true in a general way of the
personal service consumed under this head; such as priestly education,
priestly service, pilgrimages, fasts, holidays, household devotions,
and the like. At the same time the observances in the execution of which
this consumption takes place serve to extend and protract the vogue of
those habits of thought on which an anthropomorphic cult rests. That is
to say, they further the habits of thought characteristic of the regime
of status. They are in so far an obstruction to the most effective
organization of industry under modern circumstances; and are, in the
first instance, antagonistic to the development of economic institutions
in the direction required by the situation of today. For the present
purpose, the indirect as well as the direct effects of this consumption
are of the nature of a curtailment of the community's economic
efficiency. In economic theory, then, and considered in its proximate
consequences, the consumption of goods and effort in the service of
an anthropomorphic divinity means a lowering of the vitality of the
community. What may be the remoter, indirect, moral effects of this
class of consumption does not admit of a succinct answer, and it is a
question which can not be taken up here.

It will be to the point, however, to note the general economic character
of devout consumption, in comparison with consumption for other
purposes. An indication of the range of motives and purposes from which
devout consumption of goods proceeds will help toward an appreciation
of the value both of this consumption itself and of the general habit of
mind to which it is congenial. There is a striking parallelism, if not
rather a substantial identity of motive, between the consumption which
goes to the service of an anthropomorphic divinity and that which goes
to the service of a gentleman of leisure chieftain or patriarch--in the
upper class of society during the barbarian culture. Both in the case of
the chieftain and in that of the divinity there are expensive edifices
set apart for the behoof of the person served. These edifices, as well
as the properties which supplement them in the service, must not be
common in kind or grade; they always show a large element of conspicuous
waste. It may also be noted that the devout edifices are invariably of
an archaic cast in their structure and fittings. So also the servants,
both of the chieftain and of the divinity, must appear in the presence
clothed in garments of a special, ornate character. The characteristic
economic feature of this apparel is a more than ordinarily accentuated
conspicuous waste, together with the secondary feature--more accentuated
in the case of the priestly servants than in that of the servants or
courtiers of the barbarian potentate--that this court dress must always
be in some degree of an archaic fashion. Also the garments worn by the
lay members of the community when they come into the presence, should be
of a more expensive kind than their everyday apparel. Here, again, the
parallelism between the usage of the chieftain's audience hall and
that of the sanctuary is fairly well marked. In this respect there
is required a certain ceremonial "cleanness" of attire, the essential
feature of which, in the economic respect, is that the garments worn
on these occasions should carry as little suggestion as may be of any
industrial occupation or of any habitual addiction to such employments
as are of material use.

This requirement of conspicuous waste and of ceremonial cleanness from
the traces of industry extends also to the apparel, and in a less degree
to the food, which is consumed on sacred holidays; that is to say, on
days set apart--tabu--for the divinity or for some member of the lower
ranks of the preternatural leisure class. In economic theory, sacred
holidays are obviously to be construed as a season of vicarious leisure
performed for the divinity or saint in whose name the tabu is imposed
and to whose good repute the abstention from useful effort on these days
is conceived to inure. The characteristic feature of all such seasons
of devout vicarious leisure is a more or less rigid tabu on all
activity that is of human use. In the case of fast-days the conspicuous
abstention from gainful occupations and from all pursuits that
(materially) further human life is further accentuated by compulsory
abstinence from such consumption as would conduce to the comfort or the
fullness of life of the consumer.

It may be remarked, parenthetically, that secular holidays are of the
same origin, by slightly remoter derivation. They shade off by degrees
from the genuinely sacred days, through an intermediate class of
semi-sacred birthdays of kings and great men who have been in some
measure canonized, to the deliberately invented holiday set apart to
further the good repute of some notable event or some striking fact, to
which it is intended to do honor, or the good fame of which is felt
to be in need of repair. The remoter refinement in the employment
of vicarious leisure as a means of augmenting the good repute of a
phenomenon or datum is seen at its best in its very latest application.
A day of vicarious leisure has in some communities been set apart as
Labor Day. This observance is designed to augment the prestige of
the fact of labor, by the archaic, predatory method of a compulsory
abstention from useful effort. To this datum of labor-in-general is
imputed the good repute attributable to the pecuniary strength put
in evidence by abstaining from labor. Sacred holidays, and holidays
generally, are of the nature of a tribute levied on the body of the
people. The tribute is paid in vicarious leisure, and the honorific
effect which emerges is imputed to the person or the fact for whose
good repute the holiday has been instituted. Such a tithe of vicarious
leisure is a perquisite of all members of the preternatural leisure
class and is indispensable to their good fame. \emph{Un saint qu'on ne chôme
pas} is indeed a saint fallen on evil days.

Besides this tithe of vicarious leisure levied on the laity, there
are also special classes of persons--the various grades of priests and
hierodules--whose time is wholly set apart for a similar service. It is
not only incumbent on the priestly class to abstain from vulgar labor,
especially so far as it is lucrative or is apprehended to contribute to
the temporal well-being of mankind. The tabu in the case of the priestly
class goes farther and adds a refinement in the form of an injunction
against their seeking worldly gain even where it may be had without
debasing application to industry. It is felt to be unworthy of the
servant of the divinity, or rather unworthy the dignity of the divinity
whose servant he is, that he should seek material gain or take thought
for temporal matters. "Of all contemptible things a man who pretends to
be a priest of God and is a priest to his own comforts and ambitions
is the most contemptible." There is a line of discrimination, which a
cultivated taste in matters of devout observance finds little difficulty
in drawing, between such actions and conduct as conduce to the
fullness of human life and such as conduce to the good fame of the
anthropomorphic divinity; and the activity of the priestly class, in the
ideal barbarian scheme, falls wholly on the hither side of this line.
What falls within the range of economics falls below the proper level
of solicitude of the priesthood in its best estate. Such apparent
exceptions to this rule as are afforded, for instance, by some of the
medieval orders of monks (the members of which actually labored to some
useful end), scarcely impugn the rule. These outlying orders of the
priestly class are not a sacerdotal element in the full sense of the
term. And it is noticeable also that these doubtfully sacerdotal
orders, which countenanced their members in earning a living, fell into
disrepute through offending the sense of propriety in the communities
where they existed.

The priest should not put his hand to mechanically productive work; but
he should consume in large measure. But even as regards his consumption
it is to be noted that it should take such forms as do not obviously
conduce to his own comfort or fullness of life; it should conform to the
rules governing vicarious consumption, as explained under that head in
an earlier chapter. It is not ordinarily in good form for the priestly
class to appear well fed or in hilarious spirits. Indeed, in many of
the more elaborate cults the injunction against other than vicarious
consumption by this class frequently goes so far as to enjoin
mortification of the flesh. And even in those modern denominations which
have been organized under the latest formulations of the creed, in a
modern industrial community, it is felt that all levity and avowed zest
in the enjoyment of the good things of this world is alien to the true
clerical decorum. Whatever suggests that these servants of an invisible
master are living a life, not of devotion to their master's good fame,
but of application to their own ends, jars harshly on our sensibilities
as something fundamentally and eternally wrong. They are a servant
class, although, being servants of a very exalted master, they rank high
in the social scale by virtue of this borrowed light. Their consumption
is vicarious consumption; and since, in the advanced cults, their master
has no need of material gain, their occupation is vicarious leisure in
the full sense. "Whether therefore ye eat, or drink, or whatsoever ye
do, do all to the glory of God." It may be added that so far as the
laity is assimilated to the priesthood in the respect that they are
conceived to be servants of the divinity. So far this imputed vicarious
character attaches also to the layman's life. The range of application
of this corollary is somewhat wide. It applies especially to such
movements for the reform or rehabilitation of the religious life as
are of an austere, pietistic, ascetic cast--where the human subject is
conceived to hold his life by a direct servile tenure from his spiritual
sovereign. That is to say, where the institution of the priesthood
lapses, or where there is an exceptionally lively sense of the immediate
and masterful presence of the divinity in the affairs of life, there
the layman is conceived to stand in an immediate servile relation to
the divinity, and his life is construed to be a performance of vicarious
leisure directed to the enhancement of his master's repute. In such
cases of reversion there is a return to the unmediated relation of
subservience, as the dominant fact of the devout attitude. The emphasis
is thereby thrown on an austere and discomforting vicarious leisure, to
the neglect of conspicuous consumption as a means of grace.

A doubt will present itself as to the full legitimacy of this
characterization of the sacerdotal scheme of life, on the ground that a
considerable proportion of the modern priesthood departs from the scheme
in many details. The scheme does not hold good for the clergy of
those denominations which have in some measure diverged from the old
established schedule of beliefs or observances. These take thought, at
least ostensibly or permissively, for the temporal welfare of the laity,
as well as for their own. Their manner of life, not only in the privacy
of their own household, but often even before the public, does not
differ in an extreme degree from that of secular-minded persons, either
in its ostensible austerity or in the archaism of its apparatus. This is
truest for those denominations that have wandered the farthest. To
this objection it is to be said that we have here to do not with a
discrepancy in the theory of sacerdotal life, but with an imperfect
conformity to the scheme on the part of this body of clergy. They are
but a partial and imperfect representative of the priesthood, and must
not be taken as exhibiting the sacerdotal scheme of life in an authentic
and competent manner. The clergy of the sects and denominations might be
characterized as a half-caste priesthood, or a priesthood in process of
becoming or of reconstitution. Such a priesthood may be expected to
show the characteristics of the sacerdotal office only as blended
and obscured with alien motives and traditions, due to the disturbing
presence of other factors than those of animism and status in the
purposes of the organizations to which this non-conforming fraction of
the priesthood belongs.

Appeal may be taken direct to the taste of any person with a
discriminating and cultivated sense of the sacerdotal proprieties, or
to the prevalent sense of what constitutes clerical decorum in any
community at all accustomed to think or to pass criticism on what a
clergyman may or may not do without blame. Even in the most extremely
secularized denominations, there is some sense of a distinction that
should be observed between the sacerdotal and the lay scheme of life.
There is no person of sensibility but feels that where the members of
this denominational or sectarian clergy depart from traditional usage,
in the direction of a less austere or less archaic demeanor and apparel,
they are departing from the ideal of priestly decorum. There is probably
no community and no sect within the range of the Western culture in
which the bounds of permissible indulgence are not drawn appreciably
closer for the incumbent of the priestly office than for the common
layman. If the priest's own sense of sacerdotal propriety does not
effectually impose a limit, the prevalent sense of the proprieties on
the part of the community will commonly assert itself so obtrusively as
to lead to his conformity or his retirement from office.

Few if any members of any body of clergy, it may be added, would
avowedly seek an increase of salary for gain's sake; and if such avowal
were openly made by a clergyman, it would be found obnoxious to the
sense of propriety among his congregation. It may also be noted in this
connection that no one but the scoffers and the very obtuse are not
instinctively grieved inwardly at a jest from the pulpit; and that there
are none whose respect for their pastor does not suffer through any mark
of levity on his part in any conjuncture of life, except it be levity
of a palpably histrionic kind--a constrained unbending of dignity. The
diction proper to the sanctuary and to the priestly office should also
carry little if any suggestion of effective everyday life, and should
not draw upon the vocabulary of modern trade or industry. Likewise,
one's sense of the proprieties is readily offended by too detailed and
intimate a handling of industrial and other purely human questions at
the hands of the clergy. There is a certain level of generality below
which a cultivated sense of the proprieties in homiletical discourse
will not permit a well-bred clergyman to decline in his discussion
of temporal interests. These matters that are of human and secular
consequence simply, should properly be handled with such a degree of
generality and aloofness as may imply that the speaker represents
a master whose interest in secular affairs goes only so far as to
permissively countenance them.

It is further to be noticed that the non-conforming sects and variants
whose priesthood is here under discussion, vary among themselves in the
degree of their conformity to the ideal scheme of sacerdotal life. In
a general way it will be found that the divergence in this respect is
widest in the case of the relatively young denominations, and especially
in the case of such of the newer denominations as have chiefly a lower
middle-class constituency. They commonly show a large admixture of
humanitarian, philanthropic, or other motives which can not be classed
as expressions of the devotional attitude; such as the desire of
learning or of conviviality, which enter largely into the effective
interest shown by members of these organizations. The non-conforming or
sectarian movements have commonly proceeded from a mixture of motives,
some of which are at variance with that sense of status on which the
priestly office rests. Sometimes, indeed, the motive has been in good
part a revulsion against a system of status. Where this is the case
the institution of the priesthood has broken down in the transition, at
least partially. The spokesman of such an organization is at the outset
a servant and representative of the organization, rather than a member
of a special priestly class and the spokesman of a divine master. And
it is only by a process of gradual specialization that, in succeeding
generations, this spokesman regains the position of priest, with a full
investiture of sacerdotal authority, and with its accompanying austere,
archaic and vicarious manner of life. The like is true of the breakdown
and redintegration of devout ritual after such a revulsion. The priestly
office, the scheme of sacerdotal life, and the schedule of devout
observances are rehabilitated only gradually, insensibly, and with more
or less variation in details, as a persistent human sense of devout
propriety reasserts its primacy in questions touching the interest in
the preternatural--and it may be added, as the organization increases
in wealth, and so acquires more of the point of view and the habits of
thought of a leisure class.

Beyond the priestly class, and ranged in an ascending hierarchy,
ordinarily comes a superhuman vicarious leisure class of saints, angels,
etc.--or their equivalents in the ethnic cults. These rise in grade, one
above another, according to elaborate system of status. The principle of
status runs through the entire hierarchical system, both visible and
invisible. The good fame of these several orders of the supernatural
hierarchy also commonly requires a certain tribute of vicarious
consumption and vicarious leisure. In many cases they accordingly have
devoted to their service sub-orders of attendants or dependents who
perform a vicarious leisure for them, after much the same fashion as was
found in an earlier chapter to be true of the dependent leisure class
under the patriarchal system.

It may not appear without reflection how these devout observances and
the peculiarity of temperament which they imply, or the consumption of
goods and services which is comprised in the cult, stand related to the
leisure class of a modern community, or to the economic motives of which
that class is the exponent in the modern scheme of life to this end a
summary review of certain facts bearing on this relation will be useful.
It appears from an earlier passage in this discussion that for the
purpose of the collective life of today, especially so far as concerns
the industrial efficiency of the modern community, the characteristic
traits of the devout temperament are a hindrance rather than a help.
It should accordingly be found that the modern industrial life tends
selectively to eliminate these traits of human nature from the spiritual
constitution of the classes that are immediately engaged in the
industrial process. It should hold true, approximately, that devoutness
is declining or tending to obsolescence among the members of what may
be called the effective industrial community. At the same time it should
appear that this aptitude or habit survives in appreciably greater vigor
among those classes which do not immediately or primarily enter into the
community's life process as an industrial factor.

It has already been pointed out that these latter classes, which live
by, rather than in, the industrial process, are roughly comprised under
two categories (1) the leisure class proper, which is shielded from
the stress of the economic situation; and (2) the indigent classes,
including the lower-class delinquents, which are unduly exposed to
the stress. In the case of the former class an archaic habit of mind
persists because no effectual economic pressure constrains this class to
an adaptation of its habits of thought to the changing situation; while
in the latter the reason for a failure to adjust their habits of thought
to the altered requirements of industrial efficiency is innutrition,
absence of such surplus of energy as is needed in order to make the
adjustment with facility, together with a lack of opportunity to acquire
and become habituated to the modern point of view. The trend of the
selective process runs in much the same direction in both cases.

From the point of view which the modern industrial life inculcates,
phenomena are habitually subsumed under the quantitative relation of
mechanical sequence. The indigent classes not only fall short of the
modicum of leisure necessary in order to appropriate and assimilate
the more recent generalizations of science which this point of view
involves, but they also ordinarily stand in such a relation of personal
dependence or subservience to their pecuniary superiors as materially to
retard their emancipation from habits of thought proper to the regime
of status. The result is that these classes in some measure retain that
general habit of mind the chief expression of which is a strong sense of
personal status, and of which devoutness is one feature.

In the older communities of the European culture, the hereditary leisure
class, together with the mass of the indigent population, are given to
devout observances in an appreciably higher degree than the average
of the industrious middle class, wherever a considerable class of
the latter character exists. But in some of these countries, the two
categories of conservative humanity named above comprise virtually the
whole population. Where these two classes greatly preponderate, their
bent shapes popular sentiment to such an extent as to bear down any
possible divergent tendency in the inconsiderable middle class, and
imposes a devout attitude upon the whole community.

This must, of course, not be construed to say that such communities or
such classes as are exceptionally prone to devout observances tend to
conform in any exceptional degree to the specifications of any code
of morals that we may be accustomed to associate with this or that
confession of faith. A large measure of the devout habit of mind
need not carry with it a strict observance of the injunctions of the
Decalogue or of the common law. Indeed, it is becoming somewhat of a
commonplace with observers of criminal life in European communities that
the criminal and dissolute classes are, if anything, rather more devout,
and more naively so, than the average of the population. It is among
those who constitute the pecuniary middle class and the body of
law-abiding citizens that a relative exemption from the devotional
attitude is to be looked for. Those who best appreciate the merits of
the higher creeds and observances would object to all this and say that
the devoutness of the low-class delinquents is a spurious, or at the
best a superstitious devoutness; and the point is no doubt well taken
and goes directly and cogently to the purpose intended. But for the
purpose of the present inquiry these extra-economic, extra-psychological
distinctions must perforce be neglected, however valid and however
decisive they may be for the purpose for which they are made.

What has actually taken place with regard to class emancipation from the
habit of devout observance is shown by the latter-day complaint of
the clergy--that the churches are losing the sympathy of the artisan
classes, and are losing their hold upon them. At the same time it is
currently believed that the middle class, commonly so called, is also
falling away in the cordiality of its support of the church, especially
so far as regards the adult male portion of that class. These are
currently recognized phenomena, and it might seem that a simple
reference to these facts should sufficiently substantiate the general
position outlined. Such an appeal to the general phenomena of popular
church attendance and church membership may be sufficiently convincing
for the proposition here advanced. But it will still be to the purpose
to trace in some detail the course of events and the particular forces
which have wrought this change in the spiritual attitude of the more
advanced industrial communities of today. It will serve to illustrate
the manner in which economic causes work towards a secularization of
men's habits of thought. In this respect the American community should
afford an exceptionally convincing illustration, since this community
has been the least trammelled by external circumstances of any equally
important industrial aggregate.

After making due allowance for exceptions and sporadic departures from
the normal, the situation here at the present time may be summarized
quite briefly. As a general rule the classes that are low in economic
efficiency, or in intelligence, or both, are peculiarly devout--as, for
instance, the Negro population of the South, much of the lower-class
foreign population, much of the rural population, especially in those
sections which are backward in education, in the stage of development of
their industry, or in respect of their industrial contact with the rest
of the community. So also such fragments as we possess of a specialized
or hereditary indigent class, or of a segregated criminal or dissolute
class; although among these latter the devout habit of mind is apt to
take the form of a naive animistic belief in luck and in the efficacy of
shamanistic practices perhaps more frequently than it takes the form of
a formal adherence to any accredited creed. The artisan class, on
the other hand, is notoriously falling away from the accredited
anthropomorphic creeds and from all devout observances. This class is
in an especial degree exposed to the characteristic intellectual and
spiritual stress of modern organized industry, which requires a constant
recognition of the undisguised phenomena of impersonal, matter-of-fact
sequence and an unreserved conformity to the law of cause and effect.
This class is at the same time not underfed nor over-worked to such an
extent as to leave no margin of energy for the work of adaptation.

The case of the lower or doubtful leisure class in America--the middle
class commonly so called--is somewhat peculiar. It differs in respect
of its devotional life from its European counterpart, but it differs in
degree and method rather than in substance. The churches still have the
pecuniary support of this class; although the creeds to which the
class adheres with the greatest facility are relatively poor in
anthropomorphic content. At the same time the effective middle-class
congregation tends, in many cases, more or less remotely perhaps, to
become a congregation of women and minors. There is an appreciable lack
of devotional fervor among the adult males of the middle class, although
to a considerable extent there survives among them a certain complacent,
reputable assent to the outlines of the accredited creed under which
they were born. Their everyday life is carried on in a more or less
close contact with the industrial process.

This peculiar sexual differentiation, which tends to delegate devout
observances to the women and their children, is due, at least in
part, to the fact that the middle-class women are in great measure a
(vicarious) leisure class. The same is true in a less degree of the
women of the lower, artisan classes. They live under a regime of status
handed down from an earlier stage of industrial development, and thereby
they preserve a frame of mind and habits of thought which incline them
to an archaic view of things generally. At the same time they stand in
no such direct organic relation to the industrial process at large as
would tend strongly to break down those habits of thought which, for the
modern industrial purpose, are obsolete. That is to say, the peculiar
devoutness of women is a particular expression of that conservatism
which the women of civilized communities owe, in great measure, to their
economic position. For the modern man the patriarchal relation of status
is by no means the dominant feature of life; but for the women on the
other hand, and for the upper middle-class women especially, confined
as they are by prescription and by economic circumstances to their
"domestic sphere," this relation is the most real and most formative
factor of life. Hence a habit of mind favorable to devout observances
and to the interpretation of the facts of life generally in terms of
personal status. The logic, and the logical processes, of her everyday
domestic life are carried over into the realm of the supernatural, and
the woman finds herself at home and content in a range of ideas which to
the man are in great measure alien and imbecile.

Still the men of this class are also not devoid of piety, although it
is commonly not piety of an aggressive or exuberant kind. The men of
the upper middle class commonly take a more complacent attitude towards
devout observances than the men of the artisan class. This may perhaps
be explained in part by saying that what is true of the women of
the class is true to a less extent also of the men. They are to an
appreciable extent a sheltered class; and the patriarchal relation of
status which still persists in their conjugal life and in their habitual
use of servants, may also act to conserve an archaic habit of mind and
may exercise a retarding influence upon the process of secularization
which their habits of thought are undergoing. The relations of the
American middle-class man to the economic community, however, are
usually pretty close and exacting; although it may be remarked, by the
way and in qualification, that their economic activity frequently also
partakes in some degree of the patriarchal or quasi-predatory character.
The occupations which are in good repute among this class and which have
most to do with shaping the class habits of thought, are the pecuniary
occupations which have been spoken of in a similar connection in an
earlier chapter. There is a good deal of the relation of arbitrary
command and submission, and not a little of shrewd practice, remotely
akin to predatory fraud. All this belongs on the plane of life of the
predatory barbarian, to whom a devotional attitude is habitual. And in
addition to this, the devout observances also commend themselves to this
class on the ground of reputability. But this latter incentive to piety
deserves treatment by itself and will be spoken of presently. There
is no hereditary leisure class of any consequence in the American
community, except in the South. This Southern leisure class is somewhat
given to devout observances; more so than any class of corresponding
pecuniary standing in other parts of the country. It is also well known
that the creeds of the South are of a more old-fashioned cast than their
counterparts in the North. Corresponding to this more archaic devotional
life of the South is the lower industrial development of that section.
The industrial organization of the South is at present, and especially
it has been until quite recently, of a more primitive character than
that of the American community taken as a whole. It approaches nearer
to handicraft, in the paucity and rudeness of its mechanical appliances,
and there is more of the element of mastery and subservience. It may
also be noted that, owing to the peculiar economic circumstances of this
section, the greater devoutness of the Southern population, both white
and black, is correlated with a scheme of life which in many ways
recalls the barbarian stages of industrial development. Among this
population offenses of an archaic character also are and have been
relatively more prevalent and are less deprecated than they are
elsewhere; as, for example, duels, brawls, feuds, drunkenness,
horse-racing, cock-fighting, gambling, male sexual incontinence
(evidenced by the considerable number of mulattoes). There is also a
livelier sense of honor--an expression of sportsmanship and a derivative
of predatory life.

As regards the wealthier class of the North, the American leisure class
in the best sense of the term, it is, to begin with, scarcely possible
to speak of an hereditary devotional attitude. This class is of too
recent growth to be possessed of a well-formed transmitted habit in this
respect, or even of a special home-grown tradition. Still, it may be
noted in passing that there is a perceptible tendency among this class
to give in at least a nominal, and apparently something of a real,
adherence to some one of the accredited creeds. Also, weddings,
funerals, and the like honorific events among this class are
pretty uniformly solemnized with some especial degree of religious
circumstance. It is impossible to say how far this adherence to a creed
is a \emph{bona fide} reversion to a devout habit of mind, and how far it is to
be classed as a case of protective mimicry assumed for the purpose of
an outward assimilation to canons of reputability borrowed from foreign
ideals. Something of a substantial devotional propensity seems to
be present, to judge especially by the somewhat peculiar degree of
ritualistic observance which is in process of development in the
upper-class cults. There is a tendency perceptible among the upper-class
worshippers to affiliate themselves with those cults which lay
relatively great stress on ceremonial and on the spectacular accessories
of worship; and in the churches in which an upper-class membership
predominates, there is at the same time a tendency to accentuate the
ritualistic, at the cost of the intellectual features in the service and
in the apparatus of the devout observances. This holds true even where
the church in question belongs to a denomination with a relatively
slight general development of ritual and paraphernalia. This peculiar
development of the ritualistic element is no doubt due in part to a
predilection for conspicuously wasteful spectacles, but it probably
also in part indicates something of the devotional attitude of the
worshippers. So far as the latter is true, it indicates a relatively
archaic form of the devotional habit. The predominance of spectacular
effects in devout observances is noticeable in all devout communities at
a relatively primitive stage of culture and with a slight intellectual
development. It is especially characteristic of the barbarian culture.
Here there is pretty uniformly present in the devout observances a
direct appeal to the emotions through all the avenues of sense. And
a tendency to return to this naive, sensational method of appeal is
unmistakable in the upper-class churches of today. It is perceptible
in a less degree in the cults which claim the allegiance of the lower
leisure class and of the middle classes. There is a reversion to the
use of colored lights and brilliant spectacles, a freer use of symbols,
orchestral music and incense, and one may even detect in "processionals"
and "recessionals" and in richly varied genuflexional evolutions, an
incipient reversion to so antique an accessory of worship as the sacred
dance. This reversion to spectacular observances is not confined to the
upper-class cults, although it finds its best exemplification and its
highest accentuation in the higher pecuniary and social altitudes. The
cults of the lower-class devout portion of the community, such as the
Southern Negroes and the backward foreign elements of the population,
of course also show a strong inclination to ritual, symbolism, and
spectacular effects; as might be expected from the antecedents and the
cultural level of those classes. With these classes the prevalence of
ritual and anthropomorphism are not so much a matter of reversion as of
continued development out of the past. But the use of ritual and related
features of devotion are also spreading in other directions. In the
early days of the American community the prevailing denominations
started out with a ritual and paraphernalia of an austere simplicity;
but it is a matter familiar to every one that in the course of time
these denominations have, in a varying degree, adopted much of the
spectacular elements which they once renounced. In a general way, this
development has gone hand in hand with the growth of the wealth and the
ease of life of the worshippers and has reached its fullest expression
among those classes which grade highest in wealth and repute.

The causes to which this pecuniary stratification of devoutness is
due have already been indicated in a general way in speaking of
class differences in habits of thought. Class differences as regards
devoutness are but a special expression of a generic fact. The lax
allegiance of the lower middle class, or what may broadly be called the
failure of filial piety among this class, is chiefly perceptible among
the town populations engaged in the mechanical industries. In a general
way, one does not, at the present time, look for a blameless filial
piety among those classes whose employment approaches that of the
engineer and the mechanician. These mechanical employments are in a
degree a modern fact. The handicraftsmen of earlier times, who served
an industrial end of a character similar to that now served by the
mechanician, were not similarly refractory under the discipline of
devoutness. The habitual activity of the men engaged in these branches
of industry has greatly changed, as regards its intellectual discipline,
since the modern industrial processes have come into vogue; and the
discipline to which the mechanician is exposed in his daily employment
affects the methods and standards of his thinking also on topics which
lie outside his everyday work. Familiarity with the highly organized and
highly impersonal industrial processes of the present acts to derange
the animistic habits of thought. The workman's office is becoming more
and more exclusively that of discretion and supervision in a process of
mechanical, dispassionate sequences. So long as the individual is the
chief and typical prime mover in the process; so long as the obtrusive
feature of the industrial process is the dexterity and force of the
individual handicraftsman; so long the habit of interpreting phenomena
in terms of personal motive and propensity suffers no such considerable
and consistent derangement through facts as to lead to its elimination.
But under the later developed industrial processes, when the prime
movers and the contrivances through which they work are of an
impersonal, non-individual character, the grounds of generalization
habitually present in the workman's mind and the point of view from
which he habitually apprehends phenomena is an enforced cognizance of
matter-of-fact sequence. The result, so far as concerts the workman's
life of faith, is a proclivity to undevout scepticism.

It appears, then, that the devout habit of mind attains its best
development under a relatively archaic culture; the term "devout" being
of course here used in its anthropological sense simply, and not
as implying anything with respect to the spiritual attitude so
characterized, beyond the fact of a proneness to devout observances.
It appears also that this devout attitude marks a type of human nature
which is more in consonance with the predatory mode of life than with
the later-developed, more consistently and organically industrial life
process of the community. It is in large measure an expression of the
archaic habitual sense of personal status--the relation of mastery and
subservience--and it therefore fits into the industrial scheme of the
predatory and the quasi-peaceable culture, but does not fit into the
industrial scheme of the present. It also appears that this habit
persists with greatest tenacity among those classes in the modern
communities whose everyday life is most remote from the mechanical
processes of industry and which are the most conservative also in other
respects; while for those classes that are habitually in immediate
contact with modern industrial processes, and whose habits of thought
are therefore exposed to the constraining force of technological
necessities, that animistic interpretation of phenomena and that
respect of persons on which devout observance proceeds are in process
of obsolescence. And also--as bearing especially on the present
discussion--it appears that the devout habit to some extent
progressively gains in scope and elaboration among those classes in
the modern communities to whom wealth and leisure accrue in the most
pronounced degree. In this as in other relations, the institution of a
leisure class acts to conserve, and even to rehabilitate, that archaic
type of human nature and those elements of the archaic culture which the
industrial evolution of society in its later stages acts to eliminate.




\chapter{Survivals of the Non-Invidious Interests}

In an increasing proportion as time goes on, the anthropomorphic
cult, with its code of devout observations, suffers a progressive
disintegration through the stress of economic exigencies and the decay
of the system of status. As this disintegration proceeds, there come to
be associated and blended with the devout attitude certain other motives
and impulses that are not always of an anthropomorphic origin, nor
traceable to the habit of personal subservience. Not all of these
subsidiary impulses that blend with the habit of devoutness in the later
devotional life are altogether congruous with the devout attitude or
with the anthropomorphic apprehension of the sequence of phenomena. The
origin being not the same, their action upon the scheme of devout
life is also not in the same direction. In many ways they traverse the
underlying norm of subservience or vicarious life to which the code of
devout observations and the ecclesiastical and sacerdotal institutions
are to be traced as their substantial basis. Through the presence of
these alien motives the social and industrial regime of status gradually
disintegrates, and the canon of personal subservience loses the support
derived from an unbroken tradition. Extraneous habits and proclivities
encroach upon the field of action occupied by this canon, and it
presently comes about that the ecclesiastical and sacerdotal structures
are partially converted to other uses, in some measure alien to the
purposes of the scheme of devout life as it stood in the days of the
most vigorous and characteristic development of the priesthood.

Among these alien motives which affect the devout scheme in its
later growth, may be mentioned the motives of charity and of social
good-fellowship, or conviviality; or, in more general terms, the various
expressions of the sense of human solidarity and sympathy. It may
be added that these extraneous uses of the ecclesiastical structure
contribute materially to its survival in name and form even among
people who may be ready to give up the substance of it. A still more
characteristic and more pervasive alien element in the motives
which have gone to formally uphold the scheme of devout life is that
non-reverent sense of aesthetic congruity with the environment, which is
left as a residue of the latter-day act of worship after elimination
of its anthropomorphic content. This has done good service for the
maintenance of the sacerdotal institution through blending with the
motive of subservience. This sense of impulse of aesthetic congruity
is not primarily of an economic character, but it has a considerable
indirect effect in shaping the habit of mind of the individual for
economic purposes in the later stages of industrial development;
its most perceptible effect in this regard goes in the direction of
mitigating the somewhat pronounced self-regarding bias that has been
transmitted by tradition from the earlier, more competent phases of the
regime of status. The economic bearing of this impulse is therefore seen
to transverse that of the devout attitude; the former goes to qualify,
if not eliminate, the self-regarding bias, through sublation of the
antithesis or antagonism of self and not-self; while the latter, being
and expression of the sense of personal subservience and mastery, goes
to accentuate this antithesis and to insist upon the divergence between
the self-regarding interest and the interests of the generically human
life process.

This non-invidious residue of the religious life--the sense of communion
with the environment, or with the generic life process--as well as the
impulse of charity or of sociability, act in a pervasive way to shape
men's habits of thought for the economic purpose. But the action of
all this class of proclivities is somewhat vague, and their effects are
difficult to trace in detail. So much seems clear, however, as that the
action of this entire class of motives or aptitudes tends in a direction
contrary to the underlying principles of the institution of the leisure
class as already formulated. The basis of that institution, as well
as of the anthropomorphic cults associated with it in the cultural
development, is the habit of invidious comparison; and this habit is
incongruous with the exercise of the aptitudes now in question. The
substantial canons of the leisure-class scheme of life are a conspicuous
waste of time and substance and a withdrawal from the industrial
process; while the particular aptitudes here in question assert
themselves, on the economic side, in a deprecation of waste and of
a futile manner of life, and in an impulse to participation in or
identification with the life process, whether it be on the economic side
or in any other of its phases or aspects.

It is plain that these aptitudes and habits of life to which they give
rise where circumstances favor their expression, or where they assert
themselves in a dominant way, run counter to the leisure-class scheme of
life; but it is not clear that life under the leisure-class scheme, as
seen in the later stages of its development, tends consistently to the
repression of these aptitudes or to exemption from the habits of
thought in which they express themselves. The positive discipline of the
leisure-class scheme of life goes pretty much all the other way. In its
positive discipline, by prescription and by selective elimination, the
leisure-class scheme favors the all-pervading and all-dominating primacy
of the canons of waste and invidious comparison at every conjuncture
of life. But in its negative effects the tendency of the leisure-class
discipline is not so unequivocally true to the fundamental canons of the
scheme. In its regulation of human activity for the purpose of
pecuniary decency the leisure-class canon insists on withdrawal from
the industrial process. That is to say, it inhibits activity in the
directions in which the impecunious members of the community habitually
put forth their efforts. Especially in the case of women, and more
particularly as regards the upper-class and upper-middle-class women
of advanced industrial communities, this inhibition goes so far as to
insist on withdrawal even from the emulative process of accumulation by
the quasi-predator methods of the pecuniary occupations.

The pecuniary or the leisure-class culture, which set out as an
emulative variant of the impulse of workmanship, is in its latest
development beginning to neutralize its own ground, by eliminating
the habit of invidious comparison in respect of efficiency, or even
of pecuniary standing. On the other hand, the fact that members of the
leisure class, both men and women, are to some extent exempt from the
necessity of finding a livelihood in a competitive struggle with
their fellows, makes it possible for members of this class not only to
survive, but even, within bounds, to follow their bent in case they are
not gifted with the aptitudes which make for success in the competitive
struggle. That is to say, in the latest and fullest development of the
institution, the livelihood of members of this class does not depend
on the possession and the unremitting exercise of those aptitudes are
therefore greater in the higher grades of the leisure class than in the
general average of a population living under the competitive system.

In an earlier chapter, in discussing the conditions of survival of
archaic traits, it has appeared that the peculiar position of the
leisure class affords exceptionally favorable chances for the survival
of traits which characterize the type of human nature proper to an
earlier and obsolete cultural stage. The class is sheltered from the
stress of economic exigencies, and is in this sense withdrawn from
the rude impact of forces which make for adaptation to the economic
situation. The survival in the leisure class, and under the
leisure-class scheme of life, of traits and types that are reminiscent
of the predatory culture has already been discussed. These aptitudes
and habits have an exceptionally favorable chance of survival under the
leisure-class regime. Not only does the sheltered pecuniary position of
the leisure class afford a situation favorable to the survival of such
individuals as are not gifted with the complement of aptitudes
required for serviceability in the modern industrial process; but
the leisure-class canons of reputability at the same time enjoin the
conspicuous exercise of certain predatory aptitudes. The employments
in which the predatory aptitudes find exercise serve as an evidence of
wealth, birth, and withdrawal from the industrial process. The survival
of the predatory traits under the leisure-class culture is furthered
both negatively, through the industrial exemption of the class, and
positively, through the sanction of the leisure-class canons of decency.

With respect to the survival of traits characteristic of the
ante-predatory savage culture the case is in some degree different.
The sheltered position of the leisure class favors the survival also of
these traits; but the exercise of the aptitudes for peace and good-will
does not have the affirmative sanction of the code of proprieties.
Individuals gifted with a temperament that is reminiscent of the
ante-predatory culture are placed at something of an advantage within
the leisure class, as compared with similarly gifted individuals outside
the class, in that they are not under a pecuniary necessity to
thwart these aptitudes that make for a non-competitive life; but such
individuals are still exposed to something of a moral constraint
which urges them to disregard these inclinations, in that the code of
proprieties enjoins upon them habits of life based on the predatory
aptitudes. So long as the system of status remains intact, and so long
as the leisure class has other lines of non-industrial activity to take
to than obvious killing of time in aimless and wasteful fatigation,
so long no considerable departure from the leisure-class scheme of
reputable life is to be looked for. The occurrence of non-predatory
temperament with the class at that stage is to be looked upon as a case
of sporadic reversion. But the reputable non-industrial outlets for
the human propensity to action presently fail, through the advance of
economic development, the disappearance of large game, the decline of
war, the obsolescence of proprietary government, and the decay of the
priestly office. When this happens, the situation begins to change.
Human life must seek expression in one direction if it may not in
another; and if the predatory outlet fails, relief is sought elsewhere.

As indicated above, the exemption from pecuniary stress has been
carried farther in the case of the leisure-class women of the advanced
industrial communities than in that of any other considerable group of
persons. The women may therefore be expected to show a more pronounced
reversion to a non-invidious temperament than the men. But there is also
among men of the leisure class a perceptible increase in the range and
scope of activities that proceed from aptitudes which are not to be
classed as self-regarding, and the end of which is not an invidious
distinction. So, for instance, the greater number of men who have to do
with industry in the way of pecuniarily managing an enterprise take
some interest and some pride in seeing that the work is well done and
is industrially effective, and this even apart from the profit which
may result from any improvement of this kind. The efforts of
commercial clubs and manufacturers' organizations in this direction of
non-invidious advancement of industrial efficiency are also well know.

The tendency to some other than an invidious purpose in life has worked
out in a multitude of organizations, the purpose of which is some work
of charity or of social amelioration. These organizations are often of
a quasi-religious or pseudo-religious character, and are participated in
by both men and women. Examples will present themselves in abundance
on reflection, but for the purpose of indicating the range of the
propensities in question and of characterizing them, some of the
more obvious concrete cases may be cited. Such, for instance, are the
agitation for temperance and similar social reforms, for prison reform,
for the spread of education, for the suppression of vice, and for the
avoidance of war by arbitration, disarmament, or other means; such
are, in some measure, university settlements, neighborhood guilds, the
various organizations typified by the Young Men's Christian Association
and Young People's Society for Christian Endeavor, sewing-clubs, art
clubs, and even commercial clubs; such are also, in some slight measure,
the pecuniary foundations of semi-public establishments for charity,
education, or amusement, whether they are endowed by wealthy individuals
or by contributions collected from persons of smaller means--in so far
as these establishments are not of a religious character.

It is of course not intended to say that these efforts proceed entirely
from other motives than those of a self-regarding kind. What can be
claimed is that other motives are present in the common run of cases,
and that the perceptibly greater prevalence of effort of this kind under
the circumstances of the modern industrial life than under the unbroken
regime of the principle of status, indicates the presence in modern life
of an effective scepticism with respect to the full legitimacy of an
emulative scheme of life. It is a matter of sufficient notoriety to have
become a commonplace jest that extraneous motives are commonly present
among the incentives to this class of work--motives of a self-regarding
kind, and especially the motive of an invidious distinction. To such an
extent is this true, that many ostensible works of disinterested public
spirit are no doubt initiated and carried on with a view primarily to
the enhance repute or even to the pecuniary gain, of their promoters. In
the case of some considerable groups of organizations or establishments
of this kind the invidious motive is apparently the dominant motive both
with the initiators of the work and with their supporters. This last
remark would hold true especially with respect to such works as lend
distinction to their doer through large and conspicuous expenditure; as,
for example, the foundation of a university or of a public library
or museum; but it is also, and perhaps equally, true of the more
commonplace work of participation in such organizations. These serve
to authenticate the pecuniary reputability of their members, as well as
gratefully to keep them in mind of their superior status by pointing
the contrast between themselves and the lower-lying humanity in whom the
work of amelioration is to be wrought; as, for example, the university
settlement, which now has some vogue. But after all allowances and
deductions have been made, there is left some remainder of motives of
a non-emulative kind. The fact itself that distinction or a decent good
fame is sought by this method is evidence of a prevalent sense of
the legitimacy, and of the presumptive effectual presence, of a
non-emulative, non-invidious interest, as a consistent factor in the
habits of thought of modern communities.

In all this latter-day range of leisure-class activities that proceed
on the basis of a non-invidious and non-religious interest, it is to
be noted that the women participate more actively and more persistently
than the men--except, of course, in the case of such works as require
a large expenditure of means. The dependent pecuniary position of the
women disables them for work requiring large expenditure. As regards
the general range of ameliorative work, the members of the priesthood
or clergy of the less naively devout sects, or the secularized
denominations, are associated with the class of women. This is as the
theory would have it. In other economic relations, also, this clergy
stands in a somewhat equivocal position between the class of women and
that of the men engaged in economic pursuits. By tradition and by the
prevalent sense of the proprieties, both the clergy and the women of
the well-to-do classes are placed in the position of a vicarious leisure
class; with both classes the characteristic relation which goes to form
the habits of thought of the class is a relation of subservience--that
is to say, an economic relation conceived in personal terms; in both
classes there is consequently perceptible a special proneness to
construe phenomena in terms of personal relation rather than of causal
sequence; both classes are so inhibited by the canons of decency from
the ceremonially unclean processes of the lucrative or productive
occupations as to make participation in the industrial life process
of today a moral impossibility for them. The result of this ceremonial
exclusion from productive effort of the vulgar sort is to draft a
relatively large share of the energies of the modern feminine
and priestly classes into the service of other interests than the
self-regarding one. The code leaves no alternative direction in which
the impulse to purposeful action may find expression. The effect of a
consistent inhibition on industrially useful activity in the case of the
leisure-class women shows itself in a restless assertion of the impulse
to workmanship in other directions than that of business activity. As
has been noticed already, the everyday life of the well-to-do women and
the clergy contains a larger element of status than that of the average
of the men, especially than that of the men engaged in the modern
industrial occupations proper. Hence the devout attitude survives in a
better state of preservation among these classes than among the common
run of men in the modern communities. Hence an appreciable share of the
energy which seeks expression in a non-lucrative employment among these
members of the vicarious leisure classes may be expected to eventuate in
devout observances and works of piety. Hence, in part, the excess of
the devout proclivity in women, spoken of in the last chapter. But it
is more to the present point to note the effect of this proclivity
in shaping the action and coloring the purposes of the non-lucrative
movements and organizations here under discussion. Where this
devout coloring is present it lowers the immediate efficiency of
the organizations for any economic end to which their efforts may be
directed. Many organizations, charitable and ameliorative, divide their
attention between the devotional and the secular well-being of the
people whose interests they aim to further. It can scarcely be doubted
that if they were to give an equally serious attention and effort
undividedly to the secular interests of these people, the immediate
economic value of their work should be appreciably higher than it is.
It might of course similarly be said, if this were the place to say it,
that the immediate efficiency of these works of amelioration for the
devout might be greater if it were not hampered with the secular motives
and aims which are usually present.

Some deduction is to be made from the economic value of this class of
non-invidious enterprise, on account of the intrusion of the devotional
interest. But there are also deductions to be made on account of the
presence of other alien motives which more or less broadly traverse
the economic trend of this non-emulative expression of the instinct
of workmanship. To such an extent is this seen to be true on a closer
scrutiny, that, when all is told, it may even appear that this general
class of enterprises is of an altogether dubious economic value--as
measured in terms of the fullness or facility of life of the individuals
or classes to whose amelioration the enterprise is directed.
For instance, many of the efforts now in reputable vogue for the
amelioration of the indigent population of large cities are of the
nature, in great part, of a mission of culture. It is by this means
sought to accelerate the rate of speed at which given elements of the
upper-class culture find acceptance in the everyday scheme of life of
the lower classes. The solicitude of "settlements," for example, is in
part directed to enhance the industrial efficiency of the poor and to
teach them the more adequate utilization of the means at hand; but it
is also no less consistently directed to the inculcation, by precept and
example, of certain punctilios of upper-class propriety in manners and
customs. The economic substance of these proprieties will commonly be
found on scrutiny to be a conspicuous waste of time and goods. Those
good people who go out to humanize the poor are commonly, and advisedly,
extremely scrupulous and silently insistent in matters of decorum and
the decencies of life. They are commonly persons of an exemplary life
and gifted with a tenacious insistence on ceremonial cleanness in the
various items of their daily consumption. The cultural or civilizing
efficacy of this inculcation of correct habits of thought with respect
to the consumption of time and commodities is scarcely to be overrated;
nor is its economic value to the individual who acquires these higher
and more reputable ideals inconsiderable. Under the circumstances of
the existing pecuniary culture, the reputability, and consequently
the success, of the individual is in great measure dependent on his
proficiency in demeanor and methods of consumption that argue habitual
waste of time and goods. But as regards the ulterior economic bearing
of this training in worthier methods of life, it is to be said that
the effect wrought is in large part a substitution of costlier or
less efficient methods of accomplishing the same material results, in
relations where the material result is the fact of substantial economic
value. The propaganda of culture is in great part an inculcation of
new tastes, or rather of a new schedule of proprieties, which have been
adapted to the upper-class scheme of life under the guidance of the
leisure-class formulation of the principles of status and pecuniary
decency. This new schedule of proprieties is intruded into the
lower-class scheme of life from the code elaborated by an element of
the population whose life lies outside the industrial process; and this
intrusive schedule can scarcely be expected to fit the exigencies of
life for these lower classes more adequately than the schedule already
in vogue among them, and especially not more adequately than the
schedule which they are themselves working out under the stress of
modern industrial life.

All this of course does not question the fact that the proprieties
of the substituted schedule are more decorous than those which they
displace. The doubt which presents itself is simply a doubt as to the
economic expediency of this work of regeneration--that is to say, the
economic expediency in that immediate and material bearing in which the
effects of the change can be ascertained with some degree of confidence,
and as viewed from the standpoint not of the individual but of the
facility of life of the collectivity. For an appreciation of the
economic expediency of these enterprises of amelioration, therefore,
their effective work is scarcely to be taken at its face value, even
where the aim of the enterprise is primarily an economic one and where
the interest on which it proceeds is in no sense self-regarding or
invidious. The economic reform wrought is largely of the nature of a
permutation in the methods of conspicuous waste.

But something further is to be said with respect to the character of the
disinterested motives and canons of procedure in all work of this
class that is affected by the habits of thought characteristic of the
pecuniary culture; and this further consideration may lead to a further
qualification of the conclusions already reached. As has been seen in
an earlier chapter, the canons of reputability or decency under the
pecuniary culture insist on habitual futility of effort as the mark of a
pecuniarily blameless life. There results not only a habit of disesteem
of useful occupations, but there results also what is of more decisive
consequence in guiding the action of any organized body of people that
lays claim to social good repute. There is a tradition which requires
that one should not be vulgarly familiar with any of the processes or
details that have to do with the material necessities of life. One may
meritoriously show a quantitative interest in the well-being of the
vulgar, through subscriptions or through work on managing committees and
the like. One may, perhaps even more meritoriously, show solicitude in
general and in detail for the cultural welfare of the vulgar, in the
way of contrivances for elevating their tastes and affording them
opportunities for spiritual amelioration. But one should not betray an
intimate knowledge of the material circumstances of vulgar life, or of
the habits of thought of the vulgar classes, such as would effectually
direct the efforts of these organizations to a materially useful end.
This reluctance to avow an unduly intimate knowledge of the lower-class
conditions of life in detail of course prevails in very different
degrees in different individuals; but there is commonly enough of
it present collectively in any organization of the kind in question
profoundly to influence its course of action. By its cumulative action
in shaping the usage and precedents of any such body, this shrinking
from an imputation of unseemly familiarity with vulgar life tends
gradually to set aside the initial motives of the enterprise, in favor
of certain guiding principles of good repute, ultimately reducible to
terms of pecuniary merit. So that in an organization of long standing
the initial motive of furthering the facility of life in these classes
comes gradually to be an ostensible motive only, and the vulgarly
effective work of the organization tends to obsolescence.

What is true of the efficiency of organizations for non-invidious
work in this respect is true also as regards the work of individuals
proceeding on the same motives; though it perhaps holds true with more
qualification for individuals than for organized enterprises. The habit
of gauging merit by the leisure-class canons of wasteful expenditure and
unfamiliarity with vulgar life, whether on the side of production or of
consumption, is necessarily strong in the individuals who aspire to do
some work of public utility. And if the individual should forget his
station and turn his efforts to vulgar effectiveness, the common sense
of the community-the sense of pecuniary decency--would presently
reject his work and set him right. An example of this is seen in the
administration of bequests made by public-spirited men for the single
purpose (at least ostensibly) of furthering the facility of human life
in some particular respect. The objects for which bequests of this class
are most frequently made at present are
are schools, libraries, hospitals, and asylums for the infirm or
unfortunate. The avowed purpose of the donor in these cases is the
amelioration of human life in the particular respect which is named
in the bequest; but it will be found an invariable rule that in
the execution of the work not a little of other motives, frequency
incompatible with the initial motive, is present and determines the
particular disposition eventually made of a good share of the means
which have been set apart by the bequest. Certain funds, for instance,
may have been set apart as a foundation for a foundling asylum or a
retreat for invalids. The diversion of expenditure to honorific waste in
such cases is not uncommon enough to cause surprise or even to raise a
smile. An appreciable share of the funds is spent in the construction
of an edifice faced with some aesthetically objectionable but expensive
stone, covered with grotesque and incongruous details, and designed, in
its battlemented walls and turrets and its massive portals and strategic
approaches, to suggest certain barbaric methods of warfare. The interior
of the structure shows the same pervasive guidance of the canons of
conspicuous waste and predatory exploit. The windows, for instance,
to go no farther into detail, are placed with a view to impress their
pecuniary excellence upon the chance beholder from the outside, rather
than with a view to effectiveness for their ostensible end in the
convenience or comfort of the beneficiaries within; and the detail of
interior arrangement is required to conform itself as best it may to
this alien but imperious requirement of pecuniary beauty.

In all this, of course, it is not to be presumed that the donor would
have found fault, or that he would have done otherwise if he had taken
control in person; it appears that in those cases where such a personal
direction is exercised--where the enterprise is conducted by direct
expenditure and superintendence instead of by bequest--the aims and
methods of management are not different in this respect. Nor would the
beneficiaries, or the outside observers whose ease or vanity are not
immediately touched, be pleased with a different disposition of the
funds. It would suit no one to have the enterprise conducted with a view
directly to the most economical and effective use of the means at hand
for the initial, material end of the foundation. All concerned, whether
their interest is immediate and self-regarding, or contemplative only,
agree that some considerable share of the expenditure should go to
the higher or spiritual needs derived from the habit of an invidious
comparison in predatory exploit and pecuniary waste. But this only goes
to say that the canons of emulative and pecuniary reputability so far
pervade the common sense of the community as to permit no escape or
evasion, even in the case of an enterprise which ostensibly proceeds
entirely on the basis of a non-invidious interest.

It may even be that the enterprise owes its honorific virtue, as a means
of enhancing the donor's good repute, to the imputed presence of this
non-invidious motive; but that does not hinder the invidious interest
from guiding the expenditure. The effectual presence of motives of an
emulative or invidious origin in non-emulative works of this kind
might be shown at length and with detail, in any one of the classes of
enterprise spoken of above. Where these honorific details occur, in such
cases, they commonly masquerade under designations that belong in the
field of the aesthetic, ethical or economic interest. These special
motives, derived from the standards and canons of the pecuniary culture,
act surreptitiously to divert effort of a non-invidious kind from
effective service, without disturbing the agent's sense of good
intention or obtruding upon his consciousness the substantial futility
of his work. Their effect might be traced through the entire range
of that schedule of non-invidious, meliorative enterprise that is so
considerable a feature, and especially so conspicuous a feature, in the
overt scheme of life of the well-to-do. But the theoretical bearing is
perhaps clear enough and may require no further illustration; especially
as some detailed attention will be given to one of these lines of
enterprise--the establishments for the higher learning--in another
connection.

Under the circumstances of the sheltered situation in which the leisure
class is placed there seems, therefore, to be something of a reversion
to the range of non-invidious impulses that characterizes the
ante-predatory savage culture. The reversion comprises both the sense of
workmanship and the proclivity to indolence and good-fellowship. But
in the modern scheme of life canons of conduct based on pecuniary or
invidious merit stand in the way of a free exercise of these impulses;
and the dominant presence of these canons of conduct goes far to divert
such efforts as are made on the basis of the non-invidious interest to
the service of that invidious interest on which the pecuniary culture
rests. The canons of pecuniary decency are reducible for the present
purpose to the principles of waste, futility, and ferocity. The
requirements of decency are imperiously present in meliorative
enterprise as in other lines of conduct, and exercise a selective
surveillance over the details of conduct and management in any
enterprise. By guiding and adapting the method in detail, these canons
of decency go far to make all non-invidious aspiration or effort
nugatory. The pervasive, impersonal, un-eager principle of futility is
at hand from day to day and works obstructively to hinder the effectual
expression of so much of the surviving ante-predatory aptitudes as is to
be classed under the instinct of workmanship; but its presence does not
preclude the transmission of those aptitudes or the continued recurrence
of an impulse to find expression for them.

In the later and farther development of the pecuniary culture, the
requirement of withdrawal from the industrial process in order to
avoid social odium is carried so far as to comprise abstention from
the emulative employments. At this advanced stage the pecuniary culture
negatively favors the assertion of the non-invidious propensities
by relaxing the stress laid on the merit of emulative, predatory,
or pecuniary occupations, as compared with those of an industrial
or productive kind. As was noticed above, the requirement of such
withdrawal from all employment that is of human use applies more
rigorously to the upper-class women than to any other class, unless the
priesthood of certain cults might be cited as an exception, perhaps
more apparent than real, to this rule. The reason for the more extreme
insistence on a futile life for this class of women than for the men
of the same pecuniary and social grade lies in their being not only an
upper-grade leisure class but also at the same time a vicarious
leisure class. There is in their case a double ground for a consistent
withdrawal from useful effort.

It has been well and repeatedly said by popular writers and speakers who
reflect the common sense of intelligent people on questions of social
structure and function that the position of woman in any community
is the most striking index of the level of culture attained by the
community, and it might be added, by any given class in the community.
This remark is perhaps truer as regards the stage of economic
development than as regards development in any other respect. At the
same time the position assigned to the woman in the accepted scheme of
life, in any community or under any culture, is in a very great degree
an expression of traditions which have been shaped by the circumstances
of an earlier phase of development, and which have been but partially
adapted to the existing economic circumstances, or to the existing
exigencies of temperament and habits of mind by which the women living
under this modern economic situation are actuated.

The fact has already been remarked upon incidentally in the course of
the discussion of the growth of economic institutions generally, and
in particular in speaking of vicarious leisure and of dress, that the
position of women in the modern economic scheme is more widely and
more consistently at variance with the promptings of the instinct of
workmanship than is the position of the men of the same classes. It
is also apparently true that the woman's temperament includes a larger
share of this instinct that approves peace and disapproves futility.
It is therefore not a fortuitous circumstance that the women of modern
industrial communities show a livelier sense of the discrepancy
between the accepted scheme of life and the exigencies of the economic
situation.

The several phases of the "woman question" have brought out in
intelligible form the extent to which the life of women in modern
society, and in the polite circles especially, is regulated by a body of
common sense formulated under the economic circumstances of an earlier
phase of development. It is still felt that woman's life, in its civil,
economic, and social bearing, is essentially and normally a vicarious
life, the merit or demerit of which is, in the nature of things, to
be imputed to some other individual who stands in some relation of
ownership or tutelage to the woman. So, for instance, any action on the
part of a woman which traverses an injunction of the accepted schedule
of proprieties is felt to reflect immediately upon the honor of the man
whose woman she is. There may of course be some sense of incongruity
in the mind of any one passing an opinion of this kind on the woman's
frailty or perversity; but the common-sense judgment of the community in
such matters is, after all, delivered without much hesitation, and few
men would question the legitimacy of their sense of an outraged tutelage
in any case that might arise. On the other hand, relatively little
discredit attaches to a woman through the evil deeds of the man with
whom her life is associated.

The good and beautiful scheme of life, then--that is to say the scheme
to which we are habituated--assigns to the woman a "sphere" ancillary
to the activity of the man; and it is felt that any departure from the
traditions of her assigned round of duties is unwomanly. If the
question is as to civil rights or the suffrage, our common sense in the
matter--that is to say the logical deliverance of our general scheme
of life upon the point in question--says that the woman should be
represented in the body politic and before the law, not immediately in
her own person, but through the mediation of the head of the
household to which she belongs. It is unfeminine in her to aspire to a
self-directing, self-centered life; and our common sense tells us that
her direct participation in the affairs of the community, civil or
industrial, is a menace to that social order which expresses our habits
of thought as they have been formed under the guidance of the traditions
of the pecuniary culture. "All this fume and froth of 'emancipating
woman from the slavery of man' and so on, is, to use the chaste and
expressive language of Elizabeth Cady Stanton inversely, 'utter rot.'
The social relations of the sexes are fixed by nature. Our entire
civilization--that is whatever is good in it--is based on the home."
The "home" is the household with a male head. This view, but commonly
expressed even more chastely, is the prevailing view of the woman's
status, not only among the common run of the men of civilized
communities, but among the women as well. Women have a very alert sense
of what the scheme of proprieties requires, and while it is true that
many of them are ill at ease under the details which the code imposes,
there are few who do not recognize that the existing moral order, of
necessity and by the divine right of prescription, places the woman in
a position ancillary to the man. In the last analysis, according to her
own sense of what is good and beautiful, the woman's life is, and in
theory must be, an expression of the man's life at the second remove.

But in spite of this pervading sense of what is the good and natural
place for the woman, there is also perceptible an incipient development
of sentiment to the effect that this whole arrangement of tutelage and
vicarious life and imputation of merit and demerit is somehow a mistake.
Or, at least, that even if it may be a natural growth and a good
arrangement in its time and place, and in spite of its patent aesthetic
value, still it does not adequately serve the more everyday ends of life
in a modern industrial community. Even that large and substantial body
of well-bred, upper and middle-class women to whose dispassionate,
matronly sense of the traditional proprieties this relation of status
commends itself as fundamentally and eternally right-even these, whose
attitude is conservative, commonly find some slight discrepancy in
detail between things as they are and things as they should be in this
respect. But that less manageable body of modern women who, by force of
youth, education, or temperament, are in some degree out of touch with
the traditions of status received from the barbarian culture, and
in whom there is, perhaps, an undue reversion to the impulse of
self-expression and workmanship--these are touched with a sense of
grievance too vivid to leave them at rest.

In this "New-Woman" movement--as these blind and incoherent efforts to
rehabilitate the woman's pre-glacial standing have been named--there
are at least two elements discernible, both of which are of an economic
character. These two elements or motives are expressed by the double
watchword, "Emancipation" and "Work." Each of these words is recognized
to stand for something in the way of a wide-spread sense of grievance.
The prevalence of the sentiment is recognized even by people who do not
see that there is any real ground for a grievance in the situation as
it stands today. It is among the women of the well-to-do classes, in the
communities which are farthest advanced in industrial development, that
this sense of a grievance to be redressed is most alive and finds most
frequent expression. That is to say, in other words, there is a demand,
more or less serious, for emancipation from all relation of status,
tutelage, or vicarious life; and the revulsion asserts itself especially
among the class of women upon whom the scheme of life handed down from
the regime of status imposes with least litigation a vicarious life, and
in those communities whose economic development has departed farthest
from the circumstances to which this traditional scheme is adapted. The
demand comes from that portion of womankind which is excluded by the
canons of good repute from all effectual work, and which is closely
reserved for a life of leisure and conspicuous consumption.

More than one critic of this new-woman movement has misapprehended its
motive. The case of the American "new woman" has lately been summed
up with some warmth by a popular observer of social phenomena: "She is
petted by her husband, the most devoted and hard-working of husbands in
the world.... She is the superior of her husband in education, and
in almost every respect. She is surrounded by the most numerous and
delicate attentions. Yet she is not satisfied.... The Anglo-Saxon 'new
woman' is the most ridiculous production of modern times, and destined
to be the most ghastly failure of the century." Apart from the
deprecation--perhaps well placed--which is contained in this
presentment, it adds nothing but obscurity to the woman question. The
grievance of the new woman is made up of those things which this typical
characterization of the movement urges as reasons why she should be
content. She is petted, and is permitted, or even required, to consume
largely and conspicuously--vicariously for her husband or other
natural guardian. She is exempted, or debarred, from vulgarly useful
employment--in order to perform leisure vicariously for the good repute
of her natural (pecuniary) guardian. These offices are the conventional
marks of the un-free, at the same time that they are incompatible with
the human impulse to purposeful activity. But the woman is endowed
with her share-which there is reason to believe is more than an even
share--of the instinct of workmanship, to which futility of life or of
expenditure is obnoxious. She must unfold her life activity in response
to the direct, unmediated stimuli of the economic environment with which
she is in contact. The impulse is perhaps stronger upon the woman
than upon the man to live her own life in her own way and to enter the
industrial process of the community at something nearer than the second
remove.

So long as the woman's place is consistently that of a drudge, she is,
in the average of cases, fairly contented with her lot. She not only
has something tangible and purposeful to do, but she has also no time or
thought to spare for a rebellious assertion of such human propensity to
self-direction as she has inherited. And after the stage of universal
female drudgery is passed, and a vicarious leisure without strenuous
application becomes the accredited employment of the women of the
well-to-do classes, the prescriptive force of the canon of pecuniary
decency, which requires the observance of ceremonial futility on their
part, will long preserve high-minded women from any sentimental leaning
to self-direction and a "sphere of usefulness." This is especially true
during the earlier phases of the pecuniary culture, while the leisure
of the leisure class is still in great measure a predatory activity, an
active assertion of mastery in which there is enough of tangible
purpose of an invidious kind to admit of its being taken seriously as an
employment to which one may without shame put one's hand. This condition
of things has obviously lasted well down into the present in some
communities. It continues to hold to a different extent for different
individuals, varying with the vividness of the sense of status and with
the feebleness of the impulse to workmanship with which the individual
is endowed. But where the economic structure of the community has so
far outgrown the scheme of life based on status that the relation of
personal subservience is no longer felt to be the sole "natural" human
relation; there the ancient habit of purposeful activity will begin
to assert itself in the less conformable individuals against the more
recent, relatively superficial, relatively ephemeral habits and views
which the predatory and the pecuniary culture have contributed to our
scheme of life. These habits and views begin to lose their coercive
force for the community or the class in question so soon as the habit of
mind and the views of life due to the predatory and the quasi-peaceable
discipline cease to be in fairly close accord with the later-developed
economic situation. This is evident in the case of the industrious
classes of modern communities; for them the leisure-class scheme of life
has lost much of its binding force, especially as regards the element of
status. But it is also visibly being verified in the case of the upper
classes, though not in the same manner.

The habits derived from the predatory and quasi-peaceable culture are
relatively ephemeral variants of certain underlying propensities and
mental characteristics of the race; which it owes to the protracted
discipline of the earlier, proto-anthropoid cultural stage of peaceable,
relatively undifferentiated economic life carried on in contact with a
relatively simple and invariable material environment. When the habits
superinduced by the emulative method of life have ceased to enjoy the
section of existing economic exigencies, a process of disintegration
sets in whereby the habits of thought of more recent growth and of a
less generic character to some extent yield the ground before the more
ancient and more pervading spiritual characteristics of the race.

In a sense, then, the new-woman movement marks a reversion to a more
generic type of human character, or to a less differentiated
expression of human nature. It is a type of human nature which is to be
characterized as proto-anthropoid, and, as regards the substance if not
the form of its dominant traits, it belongs to a cultural stage that may
be classed as possibly sub-human. The particular movement or evolutional
feature in question of course shares this characterization with the rest
of the later social development, in so far as this social development
shows evidence of a reversion to the spiritual attitude that
characterizes the earlier, undifferentiated stage of economic
revolution. Such evidence of a general tendency to reversion from the
dominance of the invidious interest is not entirely wanting, although it
is neither plentiful nor unquestionably convincing. The general decay
of the sense of status in modern industrial communities goes some way as
evidence in this direction; and the perceptible return to a disapproval
of futility in human life, and a disapproval of such activities as serve
only the individual gain at the cost of the collectivity or at the
cost of other social groups, is evidence to a like effect. There is a
perceptible tendency to deprecate the infliction of pain, as well as to
discredit all marauding enterprises, even where these expressions of the
invidious interest do not tangibly work to the material detriment of
the community or of the individual who passes an opinion on them. It
may even be said that in the modern industrial communities the average,
dispassionate sense of men says that the ideal character is a character
which makes for peace, good-will, and economic efficiency, rather than
for a life of self-seeking, force, fraud, and mastery.

The influence of the leisure class is not consistently for or against
the rehabilitation of this proto-anthropoid human nature. So far
as concerns the chance of survival of individuals endowed with an
exceptionally large share of the primitive traits, the sheltered
position of the class favors its members directly by withdrawing them
from the pecuniary struggle; but indirectly, through the leisure-class
canons of conspicuous waste of goods and effort, the institution of a
leisure class lessens the chance of survival of such individuals in the
entire body of the population. The decent requirements of waste absorb
the surplus energy of the population in an invidious struggle and leave
no margin for the non-invidious expression of life. The remoter, less
tangible, spiritual effects of the discipline of decency go in the same
direction and work perhaps more effectually to the same end. The
canons of decent life are an elaboration of the principle of invidious
comparison, and they accordingly act consistently to inhibit all
non-invidious effort and to inculcate the self-regarding attitude.




\chapter{The Higher Learning as an Expression of
the Pecuniary Culture}

To the end that suitable habits of thought on certain heads may be
conserved in the incoming generation, a scholastic discipline is
sanctioned by the common sense of the community and incorporated into
the accredited scheme of life. The habits of thought which are so
formed under the guidance of teachers and scholastic traditions have
an economic value--a value as affecting the serviceability of the
individual--no less real than the similar economic value of the habits
of thought formed without such guidance under the discipline of everyday
life. Whatever characteristics of the accredited scholastic scheme and
discipline are traceable to the predilections of the leisure class or to
the guidance of the canons of pecuniary merit are to be set down to the
account of that institution, and whatever economic value these features
of the educational scheme possess are the expression in detail of the
value of that institution. It will be in place, therefore, to point out
any peculiar features of the educational system which are traceable to
the leisure-class scheme of life, whether as regards the aim and method
of the discipline, or as regards the compass and character of the body
of knowledge inculcated. It is in learning proper, and more particularly
in the higher learning, that the influence of leisure-class ideals is
most patent; and since the purpose here is not to make an exhaustive
collation of data showing the effect of the pecuniary culture upon
education, but rather to illustrate the method and trend of the
leisure-class influence in education, a survey of certain salient
features of the higher learning, such as may serve this purpose, is all
that will be attempted.

In point of derivation and early development, learning is somewhat
closely related to the devotional function of the community,
particularly to the body of observances in which the service rendered
the supernatural leisure class expresses itself. The service by which it
is sought to conciliate supernatural agencies in the primitive cults is
not an industrially profitable employment of the community's time and
effort. It is, therefore, in great part, to be classed as a vicarious
leisure performed for the supernatural powers with whom negotiations
are carried on and whose good-will the service and the professions of
subservience are conceived to procure. In great part, the early learning
consisted in an acquisition of knowledge and facility in the service of
a supernatural agent. It was therefore closely analogous in character to
the training required for the domestic service of a temporal master. To
a great extent, the knowledge acquired under the priestly teachers of
the primitive community was knowledge of ritual and ceremonial; that
is to say, a knowledge of the most proper, most effective, or most
acceptable manner of approaching and of serving the preternatural
agents. What was learned was how to make oneself indispensable to these
powers, and so to put oneself in a position to ask, or even to require,
their intercession in the course of events or their abstention from
interference in any given enterprise. Propitiation was the end, and this
end was sought, in great part, by acquiring facility in subservience.
It appears to have been only gradually that other elements than those
of efficient service of the master found their way into the stock of
priestly or shamanistic instruction.

The priestly servitor of the inscrutable powers that move in the
external world came to stand in the position of a mediator between these
powers and the common run of unrestricted humanity; for he was possessed
of a knowledge of the supernatural etiquette which would admit him into
the presence. And as commonly happens with mediators between the vulgar
and their masters, whether the masters be natural or preternatural, he
found it expedient to have the means at hand tangibly to impress upon
the vulgar the fact that these inscrutable powers would do what he might
ask of them. Hence, presently, a knowledge of certain natural processes
which could be turned to account for spectacular effect, together with
some sleight of hand, came to be an integral part of priestly lore.
Knowledge of this kind passes for knowledge of the "unknowable", and
it owes its serviceability for the sacerdotal purpose to its recondite
character. It appears to have been from this source that learning, as an
institution, arose, and its differentiation from this its parent stock
of magic ritual and shamanistic fraud has been slow and tedious, and is
scarcely yet complete even in the most advanced of the higher seminaries
of learning.

The recondite element in learning is still, as it has been in all ages,
a very attractive and effective element for the purpose of impressing,
or even imposing upon, the unlearned; and the standing of the savant in
the mind of the altogether unlettered is in great measure rated in terms
of intimacy with the occult forces. So, for instance, as a typical case,
even so late as the middle of this century, the Norwegian peasants have
instinctively formulated their sense of the superior erudition of such
doctors of divinity as Luther, Malanchthon, Peder Dass, and even so late
a scholar in divinity as Grundtvig, in terms of the Black Art. These,
together with a very comprehensive list of minor celebrities, both
living and dead, have been reputed masters in all magical arts; and a
high position in the ecclesiastical personnel has carried with it,
in the apprehension of these good people, an implication of profound
familiarity with magical practice and the occult sciences. There is
a parallel fact nearer home, similarly going to show the close
relationship, in popular apprehension, between erudition and the
unknowable; and it will at the same time serve to illustrate, in
somewhat coarse outline, the bent which leisure-class life gives to
the cognitive interest. While the belief is by no means confined to the
leisure class, that class today comprises a disproportionately large
number of believers in occult sciences of all kinds and shades. By those
whose habits of thought are not shaped by contact with modern industry,
the knowledge of the unknowable is still felt to the ultimate if not the
only true knowledge.

Learning, then, set out by being in some sense a by-product of the
priestly vicarious leisure class; and, at least until a recent date,
the higher learning has since remained in some sense a by-product or
by-occupation of the priestly classes. As the body of systematized
knowledge increased, there presently arose a distinction, traceable
very far back in the history of education, between esoteric and exoteric
knowledge, the former--so far as there is a substantial difference
between the two--comprising such knowledge as is primarily of no
economic or industrial effect, and the latter comprising chiefly
knowledge of industrial processes and of natural phenomena which were
habitually turned to account for the material purposes of life.
This line of demarcation has in time become, at least in popular
apprehension, the normal line between the higher learning and the lower.

It is significant, not only as an evidence of their close affiliation
with the priestly craft, but also as indicating that their activity to
a good extent falls under that category of conspicuous leisure known
as manners and breeding, that the learned class in all primitive
communities are great sticklers for form, precedent, gradations of rank,
ritual, ceremonial vestments, and learned paraphernalia generally.
This is of course to be expected, and it goes to say that the higher
learning, in its incipient phase, is a leisure-class occupation--more
specifically an occupation of the vicarious leisure class employed in
the service of the supernatural leisure class. But this predilection for
the paraphernalia of learning goes also to indicate a further point of
contact or of continuity between the priestly office and the office of
the savant. In point of derivation, learning, as well as the priestly
office, is largely an outgrowth of sympathetic magic; and this magical
apparatus of form and ritual therefore finds its place with the learned
class of the primitive community as a matter of course. The ritual and
paraphernalia have an occult efficacy for the magical purpose; so
that their presence as an integral factor in the earlier phases of the
development of magic and science is a matter of expediency, quite as
much as of affectionate regard for symbolism simply.

This sense of the efficacy of symbolic ritual, and of sympathetic effect
to be wrought through dexterous rehearsal of the traditional accessories
of the act or end to be compassed, is of course present more obviously
and in larger measure in magical practice than in the discipline of the
sciences, even of the occult sciences. But there are, I apprehend,
few persons with a cultivated sense of scholastic merit to whom the
ritualistic accessories of science are altogether an idle matter. The
very great tenacity with which these ritualistic paraphernalia persist
through the later course of the development is evident to any one
who will reflect on what has been the history of learning in our
civilization. Even today there are such things in the usage of the
learned community as the cap and gown, matriculation, initiation,
and graduation ceremonies, and the conferring of scholastic degrees,
dignities, and prerogatives in a way which suggests some sort of a
scholarly apostolic succession. The usage of the priestly orders is
no doubt the proximate source of all these features of learned ritual,
vestments, sacramental initiation, the transmission of peculiar
dignities and virtues by the imposition of hands, and the like; but
their derivation is traceable back of this point, to the source from
which the specialized priestly class proper came to be distinguished
from the sorcerer on the one hand and from the menial servant of
a temporal master on the other hand. So far as regards both their
derivation and their psychological content, these usages and the
conceptions on which they rest belong to a stage in cultural development
no later than that of the angekok and the rain-maker. Their place in the
later phases of devout observance, as well as in the higher educational
system, is that of a survival from a very early animistic phase of the
development of human nature.

These ritualistic features of the educational system of the present and
of the recent past, it is quite safe to say, have their place primarily
in the higher, liberal, and classic institutions and grades of learning,
rather than in the lower, technological, or practical grades, and
branches of the system. So far as they possess them, the lower and less
reputable branches of the educational scheme have evidently borrowed
these things from the higher grades; and their continued persistence
among the practical schools, without the sanction of the continued
example of the higher and classic grades, would be highly improbable,
to say the least. With the lower and practical schools and scholars, the
adoption and cultivation of these usages is a case of mimicry--due to
a desire to conform as far as may be to the standards of scholastic
reputability maintained by the upper grades and classes, who have
come by these accessory features legitimately, by the right of lineal
devolution.

The analysis may even be safely carried a step farther. Ritualistic
survivals and reversions come out in fullest vigor and with the freest
air of spontaneity among those seminaries of learning which have to
do primarily with the education of the priestly and leisure classes.
Accordingly it should appear, and it does pretty plainly appear, on
a survey of recent developments in college and university life, that
wherever schools founded for the instruction of the lower classes in the
immediately useful branches of knowledge grow into institutions of the
higher learning, the growth of ritualistic ceremonial and paraphernalia
and of elaborate scholastic "functions" goes hand in hand with
the transition of the schools in question from the field of homely
practicality into the higher, classical sphere. The initial purpose of
these schools, and the work with which they have chiefly had to do at
the earlier of these two stages of their evolution, has been that of
fitting the young of the industrious classes for work. On the higher,
classical plane of learning to which they commonly tend, their dominant
aim becomes the preparation of the youth of the priestly and the leisure
classes--or of an incipient leisure class--for the consumption of
goods, material and immaterial, according to a conventionally accepted,
reputable scope and method. This happy issue has commonly been the fate
of schools founded by "friends of the people" for the aid of struggling
young men, and where this transition is made in good form there is
commonly, if not invariably, a coincident change to a more ritualistic
life in the schools.

In the school life of today, learned ritual is in a general way best at
home in schools whose chief end is the cultivation of the "humanities".
This correlation is shown, perhaps more neatly than anywhere else, in
the life-history of the American colleges and universities of recent
growth. There may be many exceptions from the rule, especially among
those schools which have been founded by the typically reputable and
ritualistic churches, and which, therefore, started on the conservative
and classical plane or reached the classical position by a short-cut;
but the general rule as regards the colleges founded in the newer
American communities during the present century has been that so long
as the constituency from which the colleges have drawn their pupils
has been dominated by habits of industry and thrift, so long the
reminiscences of the medicine-man have found but a scant and precarious
acceptance in the scheme of college life. But so soon as wealth begins
appreciably to accumulate in the community, and so soon as a given
school begins to lean on a leisure-class constituency, there comes
also a perceptibly increased insistence on scholastic ritual and on
conformity to the ancient forms as regards vestments and social and
scholastic solemnities. So, for instance, there has been an approximate
coincidence between the growth of wealth among the constituency
which supports any given college of the Middle West and the date of
acceptance--first into tolerance and then into imperative vogue--of
evening dress for men and of the décolleté for women, as the scholarly
vestments proper to occasions of learned solemnity or to the seasons
of social amenity within the college circle. Apart from the mechanical
difficulty of so large a task, it would scarcely be a difficult matter
to trace this correlation. The like is true of the vogue of the cap and
gown.

Cap and gown have been adopted as learned insignia by many colleges of
this section within the last few years; and it is safe to say that this
could scarcely have occurred at a much earlier date, or until there had
grown up a leisure-class sentiment of sufficient volume in the community
to support a strong movement of reversion towards an archaic view as to
the legitimate end of education. This particular item of learned ritual,
it may be noted, would not only commend itself to the leisure-class
sense of the fitness of things, as appealing to the archaic propensity
for spectacular effect and the predilection for antique symbolism;
but it at the same time fits into the leisure-class scheme of life as
involving a notable element of conspicuous waste. The precise date at
which the reversion to cap and gown took place, as well as the fact that
it affected so large a number of schools at about the same time,
seems to have been due in some measure to a wave of atavistic sense
of conformity and reputability that passed over the community at that
period.

It may not be entirely beside the point to note that in point of time
this curious reversion seems to coincide with the culmination of a
certain vogue of atavistic sentiment and tradition in other directions
also. The wave of reversion seems to have received its initial impulse
in the psychologically disintegrating effects of the Civil War.
Habituation to war entails a body of predatory habits of thought,
whereby clannishness in some measure replaces the sense of solidarity,
and a sense of invidious distinction supplants the impulse to equitable,
everyday serviceability. As an outcome of the cumulative action of these
factors, the generation which follows a season of war is apt to witness
a rehabilitation of the element of status, both in its social life and
in its scheme of devout observances and other symbolic or ceremonial
forms. Throughout the eighties, and less plainly traceable through the
seventies also, there was perceptible a gradually advancing wave of
sentiment favoring quasi-predatory business habits, insistence on
status, anthropomorphism, and conservatism generally. The more direct
and unmediated of these expressions of the barbarian temperament, such
as the recrudescence of outlawry and the spectacular quasi-predatory
careers of fraud run by certain "captains of industry", came to a
head earlier and were appreciably on the decline by the close of the
seventies. The recrudescence of anthropomorphic sentiment also seems to
have passed its most acute stage before the close of the eighties. But
the learned ritual and paraphernalia here spoken of are a still remoter
and more recondite expression of the barbarian animistic sense; and
these, therefore, gained vogue and elaboration more slowly and reached
their most effective development at a still later date. There is reason
to believe that the culmination is now already past. Except for the new
impetus given by a new war experience, and except for the support which
the growth of a wealthy class affords to all ritual, and especially to
whatever ceremonial is wasteful and pointedly suggests gradations of
status, it is probable that the late improvements and augmentation of
scholastic insignia and ceremonial would gradually decline. But while it
may be true that the cap and gown, and the more strenuous observance
of scholastic proprieties which came with them, were floated in on this
post-bellum tidal wave of reversion to barbarism, it is also no doubt
true that such a ritualistic reversion could not have been effected in
the college scheme of life until the accumulation of wealth in the
hands of a propertied class had gone far enough to afford the requisite
pecuniary ground for a movement which should bring the colleges of the
country up to the leisure-class requirements in the higher learning. The
adoption of the cap and gown is one of the striking atavistic features
of modern college life, and at the same time it marks the fact that
these colleges have definitely become leisure-class establishments,
either in actual achievement or in aspiration.

As further evidence of the close relation between the educational system
and the cultural standards of the community, it may be remarked that
there is some tendency latterly to substitute the captain of industry in
place of the priest, as the head of seminaries of the higher learning.
The substitution is by no means complete or unequivocal. Those heads of
institutions are best accepted who combine the sacerdotal office with
a high degree of pecuniary efficiency. There is a similar but less
pronounced tendency to intrust the work of instruction in the higher
learning to men of some pecuniary qualification. Administrative ability
and skill in advertising the enterprise count for rather more than
they once did, as qualifications for the work of teaching. This applies
especially in those sciences that have most to do with the everyday
facts of life, and it is particularly true of schools in the
economically single-minded communities. This partial substitution of
pecuniary for sacerdotal efficiency is a concomitant of the modern
transition from conspicuous leisure to conspicuous consumption, as
the chief means of reputability. The correlation of the two facts is
probably clear without further elaboration.

The attitude of the schools and of the learned class towards the
education of women serves to show in what manner and to what extent
learning has departed from its ancient station of priestly and
leisure-class prerogatives, and it indicates also what approach has
been made by the truly learned to the modern, economic or industrial,
matter-of-fact standpoint. The higher schools and the learned
professions were until recently tabu to the women. These establishments
were from the outset, and have in great measure continued to be, devoted
to the education of the priestly and leisure classes.

The women, as has been shown elsewhere, were the original subservient
class, and to some extent, especially so far as regards their nominal
or ceremonial position, they have remained in that relation down to the
present. There has prevailed a strong sense that the admission of
women to the privileges of the higher learning (as to the Eleusianin
mysteries) would be derogatory to the dignity of the learned craft. It
is therefore only very recently, and almost solely in the industrially
most advanced communities, that the higher grades of schools have
been freely opened to women. And even under the urgent circumstances
prevailing in the modern industrial communities, the highest and most
reputable universities show an extreme reluctance in making the move.
The sense of class worthiness, that is to say of status, of a honorific
differentiation of the sexes according to a distinction between superior
and inferior intellectual dignity, survives in a vigorous form in these
corporations of the aristocracy of learning. It is felt that the woman
should, in all propriety, acquire only such knowledge as may be classed
under one or the other of two heads: (1) such knowledge as conduces
immediately to a better performance of domestic service--the domestic
sphere; (2) such accomplishments and dexterity, quasi-scholarly and
quasi-artistic, as plainly come in under the head of a performance of
vicarious leisure. Knowledge is felt to be unfeminine if it is knowledge
which expresses the unfolding of the learner's own life, the acquisition
of which proceeds on the learner's own cognitive interest, without
prompting from the canons of propriety, and without reference back to a
master whose comfort or good repute is to be enhanced by the employment
or the exhibition of it. So, also, all knowledge which is useful as
evidence of leisure, other than vicarious leisure, is scarcely feminine.

For an appreciation of the relation which these higher seminaries of
learning bear to the economic life of the community, the phenomena which
have been reviewed are of importance rather as indications of a general
attitude than as being in themselves facts of first-rate economic
consequence. They go to show what is the instinctive attitude and
animus of the learned class towards the life process of an industrial
community. They serve as an exponent of the stage of development, for
the industrial purpose, attained by the higher learning and by the
learned class, and so they afford an indication as to what may fairly be
looked for from this class at points where the learning and the life of
the class bear more immediately upon the economic life and efficiency
of the community, and upon the adjustment of its scheme of life to
the requirements of the time. What these ritualistic survivals go
to indicate is a prevalence of conservatism, if not of reactionary
sentiment, especially among the higher schools where the conventional
learning is cultivated.

To these indications of a conservative attitude is to be added another
characteristic which goes in the same direction, but which is a symptom
of graver consequence that this playful inclination to trivialities
of form and ritual. By far the greater number of American colleges
and universities, for instance, are affiliated to some religious
denomination and are somewhat given to devout observances. Their
putative familiarity with scientific methods and the scientific point
of view should presumably exempt the faculties of these schools
from animistic habits of thought; but there is still a considerable
proportion of them who profess an attachment to the anthropomorphic
beliefs and observances of an earlier culture. These professions
of devotional zeal are, no doubt, to a good extent expedient and
perfunctory, both on the part of the schools in their corporate
capacity, and on the part of the individual members of the corps of
instructors; but it can not be doubted that there is after all a very
appreciable element of anthropomorphic sentiment present in the
higher schools. So far as this is the case it must be set down as the
expression of an archaic, animistic habit of mind. This habit of
mind must necessarily assert itself to some extent in the instruction
offered, and to this extent its influence in shaping the habits of
thought of the student makes for conservatism and reversion; it acts
to hinder his development in the direction of matter-of-fact knowledge,
such as best serves the ends of industry.

The college sports, which have so great a vogue in the reputable
seminaries of learning today, tend in a similar direction; and, indeed,
sports have much in common with the devout attitude of the colleges,
both as regards their psychological basis and as regards their
disciplinary effect. But this expression of the barbarian temperament
is to be credited primarily to the body of students, rather than to the
temper of the schools as such; except in so far as the colleges or the
college officials--as sometimes happens--actively countenance and foster
the growth of sports. The like is true of college fraternities as
of college sports, but with a difference. The latter are chiefly
an expression of the predatory impulse simply; the former are more
specifically an expression of that heritage of clannishness which is
so large a feature in the temperament of the predatory barbarian. It is
also noticeable that a close relation subsists between the fraternities
and the sporting activity of the schools. After what has already been
said in an earlier chapter on the sporting and gambling habit, it
is scarcely necessary further to discuss the economic value of this
training in sports and in factional organization and activity.

But all these features of the scheme of life of the learned class,
and of the establishments dedicated to the conservation of the higher
learning, are in a great measure incidental only. They are scarcely
to be accounted organic elements of the professed work of research and
instruction for the ostensible pursuit of which the schools exists. But
these symptomatic indications go to establish a presumption as to the
character of the work performed--as seen from the economic point of
view--and as to the bent which the serious work carried on under their
auspices gives to the youth who resort to the schools. The presumption
raised by the considerations already offered is that in their work also,
as well as in their ceremonial, the higher schools may be expected to
take a conservative position; but this presumption must be checked by a
comparison of the economic character of the work actually performed, and
by something of a survey of the learning whose conservation is
intrusted to the higher schools. On this head, it is well known that
the accredited seminaries of learning have, until a recent date, held
a conservative position. They have taken an attitude of depreciation
towards all innovations. As a general rule a new point of view or a new
formulation of knowledge have been countenanced and taken up within the
schools only after these new things have made their way outside of
the schools. As exceptions from this rule are chiefly to be mentioned
innovations of an inconspicuous kind and departures which do not bear
in any tangible way upon the conventional point of view or upon the
conventional scheme of life; as, for instance, details of fact in the
mathematico-physical sciences, and new readings and interpretations of
the classics, especially such as have a philological or literary bearing
only. Except within the domain of the "humanities", in the narrow sense,
and except so far as the traditional point of view of the humanities has
been left intact by the innovators, it has generally held true that the
accredited learned class and the seminaries of the higher learning
have looked askance at all innovation. New views, new departures in
scientific theory, especially in new departures which touch the theory
of human relations at any point, have found a place in the scheme of
the university tardily and by a reluctant tolerance, rather than by
a cordial welcome; and the men who have occupied themselves with such
efforts to widen the scope of human knowledge have not commonly been
well received by their learned contemporaries. The higher schools have
not commonly given their countenance to a serious advance in the methods
or the content of knowledge until the innovations have outlived their
youth and much of their usefulness--after they have become commonplaces
of the intellectual furniture of a new generation which has grown
up under, and has had its habits of thought shaped by, the new,
extra-scholastic body of knowledge and the new standpoint. This is true
of the recent past. How far it may be true of the immediate present it
would be hazardous to say, for it is impossible to see present-day
facts in such perspective as to get a fair conception of their relative
proportions.

So far, nothing has been said of the Maecenas function of the
well-to-do, which is habitually dwelt on at some length by writers
and speakers who treat of the development of culture and of social
structure. This leisure-class function is not without an important
bearing on the higher and on the spread of knowledge and culture. The
manner and the degree in which the class furthers learning through
patronage of this kind is sufficiently familiar. It has been frequently
presented in affectionate and effective terms by spokesmen whose
familiarity with the topic fits them to bring home to their hearers the
profound significance of this cultural factor. These spokesmen, however,
have presented the matter from the point of view of the cultural
interest, or of the interest of reputability, rather than from that of
the economic interest. As apprehended from the economic point of view,
and valued for the purpose of industrial serviceability, this function
of the well-to-do, as well as the intellectual attitude of members of
the well-to-do class, merits some attention and will bear illustration.

By way of characterization of the Maecenas relation, it is to be noted
that, considered externally, as an economic or industrial relation
simply, it is a relation of status. The scholar under the patronage
performs the duties of a learned life vicariously for his patron, to
whom a certain repute inures after the manner of the good repute imputed
to a master for whom any form of vicarious leisure is performed. It is
also to be noted that, in point of historical fact, the furtherance of
learning or the maintenance of scholarly activity through the Maecenas
relation has most commonly been a furtherance of proficiency in
classical lore or in the humanities. The knowledge tends to lower rather
than to heighten the industrial efficiency of the community.

Further, as regards the direct participation of the members of the
leisure class in the furtherance of knowledge, the canons of reputable
living act to throw such intellectual interest as seeks expression among
the class on the side of classical and formal erudition, rather than
on the side of the sciences that bear some relation to the community's
industrial life. The most frequent excursions into other than classical
fields of knowledge on the part of members of the leisure class are made
into the discipline of law and the political, and more especially the
administrative, sciences. These so-called sciences are substantially
bodies of maxims of expediency for guidance in the leisure-class office
of government, as conducted on a proprietary basis. The interest with
which this discipline is approached is therefore not commonly the
intellectual or cognitive interest simply. It is largely the practical
interest of the exigencies of that relation of mastery in which the
members of the class are placed. In point of derivation, the office of
government is a predatory function, pertaining integrally to the archaic
leisure-class scheme of life. It is an exercise of control and coercion
over the population from which the class draws its sustenance. This
discipline, as well as the incidents of practice which give it its
content, therefore has some attraction for the class apart from all
questions of cognition. All this holds true wherever and so long as
the governmental office continues, in form or in substance, to be a
proprietary office; and it holds true beyond that limit, in so far as
the tradition of the more archaic phase of governmental evolution has
lasted on into the later life of those modern communities for whom
proprietary government by a leisure class is now beginning to pass away.

For that field of learning within which the cognitive or intellectual
interest is dominant--the sciences properly so called--the case is
somewhat different, not only as regards the attitude of the leisure
class, but as regards the whole drift of the pecuniary culture.
Knowledge for its own sake, the exercise of the faculty of comprehensive
without ulterior purpose, should, it might be expected, be sought by
men whom no urgent material interest diverts from such a quest. The
sheltered industrial position of the leisure class should give free
play to the cognitive interest in members of this class, and we should
consequently have, as many writers confidently find that we do have, a
very large proportion of scholars, scientists, savants derived from
this class and deriving their incentive to scientific investigation and
speculation from the discipline of a life of leisure. Some such result
is to be looked for, but there are features of the leisure-class
scheme of life, already sufficiently dwelt upon, which go to divert the
intellectual interest of this class to other subjects than that causal
sequence in phenomena which makes the content of the sciences. The
habits of thought which characterize the life of the class run on
the personal relation of dominance, and on the derivative, invidious
concepts of honor, worth, merit, character, and the like. The casual
sequence which makes up the subject matter of science is not visible
from this point of view. Neither does good repute attach to knowledge of
facts that are vulgarly useful. Hence it should appear probable that the
interest of the invidious comparison with respect to pecuniary or other
honorific merit should occupy the attention of the leisure class, to the
neglect of the cognitive interest. Where this latter interest asserts
itself it should commonly be diverted to fields of speculation or
investigation which are reputable and futile, rather than to the quest
of scientific knowledge. Such indeed has been the history of priestly
and leisure-class learning so long as no considerable body of
systematized knowledge had been intruded into the scholastic discipline
from an extra-scholastic source. But since the relation of mastery and
subservience is ceasing to be the dominant and formative factor in the
community's life process, other features of the life process and other
points of view are forcing themselves upon the scholars. The true-bred
gentleman of leisure should, and does, see the world from the point of
view of the personal relation; and the cognitive interest, so far as
it asserts itself in him, should seek to systematize phenomena on this
basis. Such indeed is the case with the gentleman of the old school, in
whom the leisure-class ideals have suffered no disintegration; and such
is the attitude of his latter-day descendant, in so far as he has fallen
heir to the full complement of upper-class virtues. But the ways of
heredity are devious, and not every gentleman's son is to the manor
born. Especially is the transmission of the habits of thought which
characterize the predatory master somewhat precarious in the case of a
line of descent in which but one or two of the latest steps have lain
within the leisure-class discipline. The chances of occurrence of a
strong congenital or acquired bent towards the exercise of the cognitive
aptitudes are apparently best in those members of the leisure class who
are of lower class or middle class antecedents--that is to say, those
who have inherited the complement of aptitudes proper to the industrious
classes, and who owe their place in the leisure class to the possession
of qualities which count for more today than they did in the times when
the leisure-class scheme of life took shape. But even outside the range
of these later accessions to the leisure class there are an appreciable
number of individuals in whom the invidious interest is not sufficiently
dominant to shape their theoretical views, and in whom the proclivity to
theory is sufficiently strong to lead them into the scientific quest.

The higher learning owes the intrusion of the sciences in part to these
aberrant scions of the leisure class, who have come under the dominant
influence of the latter-day tradition of impersonal relation and who
have inherited a complement of human aptitudes differing in certain
salient features from the temperament which is characteristic of
the regime of status. But it owes the presence of this alien body of
scientific knowledge also in part, and in a higher degree, to members of
the industrious classes who have been in sufficiently easy circumstances
to turn their attention to other interests than that of finding daily
sustenance, and whose inherited aptitudes and anthropomorphic point of
view does not dominate their intellectual processes. As between
these two groups, which approximately comprise the effective force of
scientific progress, it is the latter that has contributed the most. And
with respect to both it seems to be true that they are not so much
the source as the vehicle, or at the most they are the instrument of
commutation, by which the habits of thought enforced upon the community,
through contact with its environment under the exigencies of modern
associated life and the mechanical industries, are turned to account for
theoretical knowledge.

Science, in the sense of an articulate recognition of causal sequence in
phenomena, whether physical or social, has been a feature of the Western
culture only since the industrial process in the Western communities has
come to be substantially a process of mechanical contrivances in which
man's office is that of discrimination and valuation of material forces.
Science has flourished somewhat in the same degree as the industrial
life of the community has conformed to this pattern, and somewhat in
the same degree as the industrial interest has dominated the community's
life. And science, and scientific theory especially, has made headway
in the several departments of human life and knowledge in proportion
as each of these several departments has successively come into closer
contact with the industrial process and the economic interest;
or perhaps it is truer to say, in proportion as each of them has
successively escaped from the dominance of the conceptions of personal
relation or status, and of the derivative canons of anthropomorphic
fitness and honorific worth.

It is only as the exigencies of modern industrial life have enforced the
recognition of causal sequence in the practical contact of mankind with
their environment, that men have come to systematize the phenomena of
this environment and the facts of their own contact with it in terms
of causal sequence. So that while the higher learning in its best
development, as the perfect flower of scholasticism and classicism, was
a by-product of the priestly office and the life of leisure, so modern
science may be said to be a by-product of the industrial process.
Through these groups of men, then--investigators, savants, scientists,
inventors, speculators--most of whom have done their most telling work
outside the shelter of the schools, the habits of thought enforced
by the modern industrial life have found coherent expression and
elaboration as a body of theoretical science having to do with the
causal sequence of phenomena. And from this extra-scholastic field of
scientific speculation, changes of method and purpose have from time to
time been intruded into the scholastic discipline.

In this connection it is to be remarked that there is a very perceptible
difference of substance and purpose between the instruction offered in
the primary and secondary schools, on the one hand, and in the higher
seminaries of learning, on the other hand. The difference in point
of immediate practicality of the information imparted and of the
proficiency acquired may be of some consequence and may merit the
attention which it has from time to time received; but there is more
substantial difference in the mental and spiritual bent which is favored
by the one and the other discipline. This divergent trend in discipline
between the higher and the lower learning is especially noticeable as
regards the primary education in its latest development in the advanced
industrial communities. Here the instruction is directed chiefly to
proficiency or dexterity, intellectual and manual, in the apprehension
and employment of impersonal facts, in their casual rather than in their
honorific incidence. It is true, under the traditions of the earlier
days, when the primary education was also predominantly a leisure-class
commodity, a free use is still made of emulation as a spur to diligence
in the common run of primary schools; but even this use of emulation as
an expedient is visibly declining in the primary grades of instruction
in communities where the lower education is not under the guidance
of the ecclesiastical or military tradition. All this holds true in
a peculiar degree, and more especially on the spiritual side, of such
portions of the educational system as have been immediately affected by
kindergarten methods and ideals.

The peculiarly non-invidious trend of the kindergarten discipline, and
the similar character of the kindergarten influence in primary education
beyond the limits of the kindergarten proper, should be taken in
connection with what has already been said of the peculiar spiritual
attitude of leisure-class womankind under the circumstances of the
modern economic situation. The kindergarten discipline is at its
best--or at its farthest remove from ancient patriarchal and pedagogical
ideals--in the advanced industrial communities, where there is a
considerable body of intelligent and idle women, and where the system of
status has somewhat abated in rigor under the disintegrating influence
of industrial life and in the absence of a consistent body of
military and ecclesiastical traditions. It is from these women in easy
circumstances that it gets its moral support. The aims and methods of
the kindergarten commend themselves with especial effect to this class
of women who are ill at ease under the pecuniary code of reputable life.
The kindergarten, and whatever the kindergarten spirit counts for
in modern education, therefore, is to be set down, along with the
"new-woman movement," to the account of that revulsion against futility
and invidious comparison which the leisure-class life under modern
circumstances induces in the women most immediately exposed to its
discipline. In this way it appears that, by indirection, the institution
of a leisure class here again favors the growth of a non-invidious
attitude, which may, in the long run, prove a menace to the stability
of the institution itself, and even to the institution of individual
ownership on which it rests.

During the recent past some tangible changes have taken place in the
scope of college and university teaching. These changes have in the main
consisted in a partial displacement of the humanities--those branches
of learning which are conceived to make for the traditional "culture",
character, tastes, and ideals--by those more matter-of-fact branches
which make for civic and industrial efficiency. To put the same thing
in other words, those branches of knowledge which make for efficiency
(ultimately productive efficiency) have gradually been gaining ground
against those branches which make for a heightened consumption or a
lowered industrial efficiency and for a type of character suited to the
regime of status. In this adaptation of the scheme of instruction the
higher schools have commonly been found on the conservative side; each
step which they have taken in advance has been to some extent of
the nature of a concession. The sciences have been intruded into
the scholar's discipline from without, not to say from below. It is
noticeable that the humanities which have so reluctantly yielded ground
to the sciences are pretty uniformly adapted to shape the character
of the student in accordance with a traditional self-centred scheme of
consumption; a scheme of contemplation and enjoyment of the true,
the beautiful, and the good, according to a conventional standard of
propriety and excellence, the salient feature of which is leisure--otium
cum dignitate. In language veiled by their own habituation to the
archaic, decorous point of view, the spokesmen of the humanities have
insisted upon the ideal embodied in the maxim, fruges consumere nati.
This attitude should occasion no surprise in the case of schools which
are shaped by and rest upon a leisure-class culture.

The professed grounds on which it has been sought, as far as might be,
to maintain the received standards and methods of culture intact
are likewise characteristic of the archaic temperament and of the
leisure-class theory of life. The enjoyment and the bent derived from
habitual contemplation of the life, ideals, speculations, and methods of
consuming time and goods, in vogue among the leisure class of classical
antiquity, for instance, is felt to be "higher", "nobler", "worthier",
than what results in these respects from a like familiarity with the
everyday life and the knowledge and aspirations of commonplace humanity
in a modern community, that learning the content of which is an
unmitigated knowledge of latter-day men and things is by comparison
"lower", "base", "ignoble"--one even hears the epithet "sub-human"
applied to this matter-of-fact knowledge of mankind and of everyday
life.
\clearpage
This contention of the leisure-class spokesmen of the humanities
seems to be substantially sound. In point of substantial fact, the
gratification and the culture, or the spiritual attitude or habit of
mind, resulting from an habitual contemplation of the anthropomorphism,
clannishness, and leisurely self-complacency of the gentleman of an
early day, or from a familiarity with the animistic superstitions
and the exuberant truculence of the Homeric heroes, for instance, is,
aesthetically considered, more legitimate than the corresponding results
derived from a matter-of-fact knowledge of things and a contemplation
of latter-day civic or workmanlike efficiency. There can be but little
question that the first-named habits have the advantage in respect of
aesthetic or honorific value, and therefore in respect of the "worth"
which is made the basis of award in the comparison. The content of the
canons of taste, and more particularly of the canons of honor, is in the
nature of things a resultant of the past life and circumstances of
the race, transmitted to the later generation by inheritance or by
tradition; and the fact that the protracted dominance of a predatory,
leisure-class scheme of life has profoundly shaped the habit of mind and
the point of view of the race in the past, is a sufficient basis for an
aesthetically legitimate dominance of such a scheme of life in very much
of what concerns matters of taste in the present. For the purpose in
hand, canons of taste are race habits, acquired through a more or less
protracted habituation to the approval or disapproval of the kind
of things upon which a favorable or unfavorable judgment of taste is
passed. Other things being equal, the longer and more unbroken the
habituation, the more legitimate is the canon of taste in question. All
this seems to be even truer of judgments regarding worth or honor than
of judgments of taste generally.

But whatever may be the aesthetic legitimacy of the derogatory judgment
passed on the newer learning by the spokesmen of the humanities, and
however substantial may be the merits of the contention that the
classic lore is worthier and results in a more truly human culture and
character, it does not concern the question in hand. The question in
hand is as to how far these branches of learning, and the point of
view for which they stand in the educational system, help or hinder an
efficient collective life under modern industrial circumstances--how
far they further a more facile adaptation to the economic situation
of today. The question is an economic, not an aesthetic one; and
the leisure-class standards of learning which find expression in the
deprecatory attitude of the higher schools towards matter-of-fact
knowledge are, for the present purpose, to be valued from this point of
view only. For this purpose the use of such epithets as "noble", "base",
"higher", "lower", etc., is significant only as showing the animus
and the point of view of the disputants; whether they contend for the
worthiness of the new or of the old. All these epithets are honorific or
humilific terms; that is to say, they are terms of invidious comparison,
which in the last analysis fall under the category of the reputable or
the disreputable; that is, they belong within the range of ideas that
characterizes the scheme of life of the regime of status; that is, they
are in substance an expression of sportsmanship--of the predatory and
animistic habit of mind; that is, they indicate an archaic point of view
and theory of life, which may fit the predatory stage of culture and of
economic organization from which they have sprung, but which are,
from the point of view of economic efficiency in the broader sense,
disserviceable anachronisms.

The classics, and their position of prerogative in the scheme of
education to which the higher seminaries of learning cling with such a
fond predilection, serve to shape the intellectual attitude and lower
the economic efficiency of the new learned generation. They do this
not only by holding up an archaic ideal of manhood, but also by the
discrimination which they inculcate with respect to the reputable and
the disreputable in knowledge. This result is accomplished in two ways:
(1) by inspiring an habitual aversion to what is merely useful, as
contrasted with what is merely honorific in learning, and so shaping the
tastes of the novice that he comes in good faith to find gratification
of his tastes solely, or almost solely, in such exercise of the
intellect as normally results in no industrial or social gain; and (2)
by consuming the learner's time and effort in acquiring knowledge which
is of no use except in so far as this learning has by convention become
incorporated into the sum of learning required of the scholar, and has
thereby affected the terminology and diction employed in the useful
branches of knowledge. Except for this terminological difficulty--which
is itself a consequence of the vogue of the classics of the past--a
knowledge of the ancient languages, for instance, would have no
practical bearing for any scientist or any scholar not engaged on work
primarily of a linguistic character. Of course, all this has nothing to
say as to the cultural value of the classics, nor is there any intention
to disparage the discipline of the classics or the bent which their
study gives to the student. That bent seems to be of an economically
disserviceable kind, but this fact--somewhat notorious indeed--need
disturb no one who has the good fortune to find comfort and strength in
the classical lore. The fact that classical learning acts to derange
the learner's workmanlike attitudes should fall lightly upon the
apprehension of those who hold workmanship of small account in
comparison with the cultivation of decorous ideals: 
\begin{displayquote}
\emph{Iam fides et pax et honos pudorque \\
Priscus et neglecta redire virtus \\
Audet.}
\end{displayquote}
Owing to the circumstance that this knowledge has become part of the
elementary requirements in our system of education, the ability to use
and to understand certain of the dead languages of southern Europe
is not only gratifying to the person who finds occasion to parade his
accomplishments in this respect, but the evidence of such knowledge
serves at the same time to recommend any savant to his audience, both
lay and learned. It is currently expected that a certain number of
years shall have been spent in acquiring this substantially useless
information, and its absence creates a presumption of hasty and
precarious learning, as well as of a vulgar practicality that is
equally obnoxious to the conventional standards of sound scholarship and
intellectual force.

The case is analogous to what happens in the purchase of any article of
consumption by a purchaser who is not an expert judge of materials or
of workmanship. He makes his estimate of value of the article chiefly
on the ground of the apparent expensiveness of the finish of those
decorative parts and features which have no immediate relation to the
intrinsic usefulness of the article; the presumption being that some
sort of ill-defined proportion subsists between the substantial value of
an article and the expense of adornment added in order to sell it. The
presumption that there can ordinarily be no sound scholarship where
a knowledge of the classics and humanities is wanting leads to a
conspicuous waste of time and labor on the part of the general body of
students in acquiring such knowledge. The conventional insistence on a
modicum of conspicuous waste as an incident of all reputable scholarship
has affected our canons of taste and of serviceability in matters of
scholarship in much the same way as the same principle has influenced
our judgment of the serviceability of manufactured goods.

It is true, since conspicuous consumption has gained more and more on
conspicuous leisure as a means of repute, the acquisition of the dead
languages is no longer so imperative a requirement as it once was,
and its talismanic virtue as a voucher of scholarship has suffered a
concomitant impairment. But while this is true, it is also true that the
classics have scarcely lost in absolute value as a voucher of scholastic
respectability, since for this purpose it is only necessary that
the scholar should be able to put in evidence some learning which is
conventionally recognized as evidence of wasted time; and the classics
lend themselves with great facility to this use. Indeed, there can be
little doubt that it is their utility as evidence of wasted time and
effort, and hence of the pecuniary strength necessary in order to
afford this waste, that has secured to the classics their position of
prerogative in the scheme of higher learning, and has led to their being
esteemed the most honorific of all learning. They serve the decorative
ends of leisure-class learning better than any other body of knowledge,
and hence they are an effective means of reputability.

In this respect the classics have until lately had scarcely a rival.
They still have no dangerous rival on the continent of Europe, but
lately, since college athletics have won their way into a recognized
standing as an accredited field of scholarly accomplishment, this latter
branch of learning--if athletics may be freely classed as learning--has
become a rival of the classics for the primacy in leisure-class
education in American and English schools. Athletics have an obvious
advantage over the classics for the purpose of leisure-class learning,
since success as an athlete presumes, not only waste of time, but also
waste of money, as well as the possession of certain highly unindustrial
archaic traits of character and temperament. In the German universities
the place of athletics and Greek-letter fraternities, as a leisure-class
scholarly occupation, has in some measure been supplied by a skilled and
graded inebriety and a perfunctory duelling.

The leisure class and its standard of virtue--archaism and waste--can
scarcely have been concerned in the introduction of the classics into
the scheme of the higher learning; but the tenacious retention of the
classics by the higher schools, and the high degree of reputability
which still attaches to them, are no doubt due to their conforming so
closely to the requirements of archaism and waste.

"Classic" always carries this connotation of wasteful and archaic,
whether it is used to denote the dead languages or the obsolete or
obsolescent forms of thought and diction in the living language, or to
denote other items of scholarly activity or apparatus to which it is
applied with less aptness. So the archaic idiom of the English language
is spoken of as "classic" English. Its use is imperative in all speaking
and writing upon serious topics, and a facile use of it lends dignity to
even the most commonplace and trivial string of talk. The newest form
of English diction is of course never written; the sense of that
leisure-class propriety which requires archaism in speech is present
even in the most illiterate or sensational writers in sufficient
force to prevent such a lapse. On the other hand, the highest and
most conventionalized style of archaic diction is--quite
characteristically--properly employed only in communications between an
anthropomorphic divinity and his subjects. Midway between these extremes
lies the everyday speech of leisure-class conversation and literature.

Elegant diction, whether in writing or speaking, is an effective means
of reputability. It is of moment to know with some precision what is
the degree of archaism conventionally required in speaking on any given
topic. Usage differs appreciably from the pulpit to the market-place;
the latter, as might be expected, admits the use of relatively new and
effective words and turns of expression, even by fastidious persons. A
discriminative avoidance of neologisms is honorific, not only because it
argues that time has been wasted in acquiring the obsolescent habit of
speech, but also as showing that the speaker has from infancy habitually
associated with persons who have been familiar with the obsolescent
idiom. It thereby goes to show his leisure-class antecedents. Great
purity of speech is presumptive evidence of several lives spent in other
than vulgarly useful occupations; although its evidence is by no means
entirely conclusive to this point.

As felicitous an instance of futile classicism as can well be found,
outside of the Far East, is the conventional spelling of the English
language. A breach of the proprieties in spelling is extremely annoying
and will discredit any writer in the eyes of all persons who are
possessed of a developed sense of the true and beautiful. English
orthography satisfies all the requirements of the canons of reputability
under the law of conspicuous waste. It is archaic, cumbrous, and
ineffective; its acquisition consumes much time and effort; failure to
acquire it is easy of detection. Therefore it is the first and readiest
test of reputability in learning, and conformity to its ritual is
indispensable to a blameless scholastic life.

On this head of purity of speech, as at other points where a
conventional usage rests on the canons of archaism and waste, the
spokesmen for the usage instinctively take an apologetic attitude. It
is contended, in substance, that a punctilious use of ancient and
accredited locutions will serve to convey thought more adequately and
more precisely than would be the straightforward use of the latest form
of spoken English; whereas it is notorious that the ideas of today are
effectively expressed in the slang of today. Classic speech has the
honorific virtue of dignity; it commands attention and respect as being
the accredited method of communication under the leisure-class scheme
of life, because it carries a pointed suggestion of the industrial
exemption of the speaker. The advantage of the accredited locutions lies
in their reputability; they are reputable because they are cumbrous and
out of date, and therefore argue waste of time and exemption from the
use and the need of direct and forcible speech.

\end{document}