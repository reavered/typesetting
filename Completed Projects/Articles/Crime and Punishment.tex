\documentclass[12pt]{report}
\usepackage[12pt]{moresize}
\usepackage[utf8]{inputenc}
\usepackage[english]{babel}
\usepackage[top=2.5cm, bottom=2.5cm, left=2.5cm, right=2.5cm]{geometry}
\usepackage{ebgaramond}

%=======SECTION HEADERS=========%
\usepackage{titlesec}
\usepackage{titletoc}

%========QUOTES=========%
\usepackage{epigraph}
\usepackage[autostyle, english = american]{csquotes}
\MakeOuterQuote{"}

%=======PARAGRAPH FORMATTING=========%

\renewcommand{\chaptermark}[1]{\markboth{\MakeUppercase{Book \thechapter}}{}} %Book format- heading
\titleformat
{\chapter} 
[display]
{\Huge\centering} 
{\thechapter} 
{\leftmargin}{}[]

\newcommand{\mychapter}[2]{
\setcounter{chapter}{#1}
    \setcounter{section}{0}
    \chapter*{#2}
    \addcontentsline{toc}{chapter}{\thechapter}
}

\renewcommand\thechapter{{\Roman{chapter}}}

%=======SECTION HEADER SPACING=========%
\titlespacing{\chapter}{0mm}{-2em}{0em}
\titlespacing{\section}{0mm}{3mm}{2mm}

%=======TITLE PAGE=========%
\title{\HUGE\bfseries{Crime and Punishment}}
\author{\Large by Fyodor Dostoevsky}
\date{\vspace{-1mm}Translated 1914 by Constance Garnett}

%=======FOOTNOTES=========%
\renewcommand{\thefootnote}{[\fnsymbol{footnote}]}
\setlength{\skip\footins}{1cm}
\usepackage[symbol]{footmisc}
\renewcommand{\footnotemargin}{3mm} %Setting left margin
\renewcommand{\footnotelayout}{\hspace{2mm}} %spacing between the footnote number and the text of footnote

\usepackage{hyperref}
\hypersetup{bookmarksnumbered} %Bookmarks are numbered in the ToC when converted to PDF or EPUB

\titlecontents{part}
    [0pt]% <left-indent>
    {\vspace{24pt}\centering}
    {}
    {}
    {}

\titlecontents{chapter}% formatting-toc-chapters
    [0pt]% <left-indent>
    {\vspace{8pt}}% <above-code>
    {\thechapter}% <numbered-entry-format>
    {}% <numberless-entry-format>
    {\titlerule*[1pc]{.}\contentspage}% <filler-page-format>
\begin{document}

\begin{titlepage}
    \maketitle
\end{titlepage}

%=======TABLE OF CONTENTS=========%
\tableofcontents

%=======MAIN DOCUMENT=========%
\part*{PART I}
\addcontentsline{toc}{part}{PART I}

\titlespacing{\chapter}{0mm}{-2em}{2em}

\mychapter{1}{Chapter I}\label{chapter-i}

On an exceptionally hot evening early in July a young man came out of
the garret in which he lodged in S. Place and walked slowly, as though
in hesitation, towards K. bridge.

He had successfully avoided meeting his landlady on the staircase. His
garret was under the roof of a high, five-storied house and was more
like a cupboard than a room. The landlady who provided him with garret,
dinners, and attendance, lived on the floor below, and every time he
went out he was obliged to pass her kitchen, the door of which
invariably stood open. And each time he passed, the young man had a
sick, frightened feeling, which made him scowl and feel ashamed. He was
hopelessly in debt to his landlady, and was afraid of meeting her.

This was not because he was cowardly and abject, quite the contrary; but
for some time past he had been in an overstrained irritable condition,
verging on hypochondria. He had become so completely absorbed in
himself, and isolated from his fellows that he dreaded meeting, not only
his landlady, but anyone at all. He was crushed by poverty, but the
anxieties of his position had of late ceased to weigh upon him. He had
given up attending to matters of practical importance; he had lost all
desire to do so. Nothing that any landlady could do had a real terror
for him. But to be stopped on the stairs, to be forced to listen to her
trivial, irrelevant gossip, to pestering demands for payment, threats
and complaints, and to rack his brains for excuses, to prevaricate, to
lie---no, rather than that, he would creep down the stairs like a cat
and slip out unseen.

This evening, however, on coming out into the street, he became acutely
aware of his fears.

``I want to attempt a thing \emph{like that} and am frightened by these
trifles,'' he thought, with an odd smile. ``Hm\ldots{} yes, all is in a
man's hands and he lets it all slip from cowardice, that's an axiom. It
would be interesting to know what it is men are most afraid of. Taking a
new step, uttering a new word is what they fear most\ldots{}. But I am
talking too much. It's because I chatter that I do nothing. Or perhaps
it is that I chatter because I do nothing. I've learned to chatter this
last month, lying for days together in my den thinking\ldots{} of Jack
the Giant-killer. Why am I going there now? Am I capable of \emph{that}?
Is \emph{that} serious? It is not serious at all. It's simply a fantasy
to amuse myself; a plaything! Yes, maybe it is a plaything.''

The heat in the street was terrible: and the airlessness, the bustle and
the plaster, scaffolding, bricks, and dust all about him, and that
special Petersburg stench, so familiar to all who are unable to get out
of town in summer---all worked painfully upon the young man's already
overwrought nerves. The insufferable stench from the pot-houses, which
are particularly numerous in that part of the town, and the drunken men
whom he met continually, although it was a working day, completed the
revolting misery of the picture. An expression of the profoundest
disgust gleamed for a moment in the young man's refined face. He was, by
the way, exceptionally handsome, above the average in height, slim,
well-built, with beautiful dark eyes and dark brown hair. Soon he sank
into deep thought, or more accurately speaking into a complete blankness
of mind; he walked along not observing what was about him and not caring
to observe it. From time to time, he would mutter something, from the
habit of talking to himself, to which he had just confessed. At these
moments he would become conscious that his ideas were sometimes in a
tangle and that he was very weak; for two days he had scarcely tasted
food.

He was so badly dressed that even a man accustomed to shabbiness would
have been ashamed to be seen in the street in such rags. In that quarter
of the town, however, scarcely any shortcoming in dress would have
created surprise. Owing to the proximity of the Hay Market, the number
of establishments of bad character, the preponderance of the trading and
working class population crowded in these streets and alleys in the
heart of Petersburg, types so various were to be seen in the streets
that no figure, however queer, would have caused surprise. But there was
such accumulated bitterness and contempt in the young man's heart, that,
in spite of all the fastidiousness of youth, he minded his rags least of
all in the street. It was a different matter when he met with
acquaintances or with former fellow students, whom, indeed, he disliked
meeting at any time. And yet when a drunken man who, for some unknown
reason, was being taken somewhere in a huge waggon dragged by a heavy
dray horse, suddenly shouted at him as he drove past: ``Hey there,
German hatter'' bawling at the top of his voice and pointing at
him---the young man stopped suddenly and clutched tremulously at his
hat. It was a tall round hat from Zimmerman's, but completely worn out,
rusty with age, all torn and bespattered, brimless and bent on one side
in a most unseemly fashion. Not shame, however, but quite another
feeling akin to terror had overtaken him.

``I knew it,'' he muttered in confusion, ``I thought so! That's the
worst of all! Why, a stupid thing like this, the most trivial detail
might spoil the whole plan. Yes, my hat is too noticeable\ldots{}. It
looks absurd and that makes it noticeable\ldots{}. With my rags I ought
to wear a cap, any sort of old pancake, but not this grotesque thing.
Nobody wears such a hat, it would be noticed a mile off, it would be
remembered\ldots{}. What matters is that people would remember it, and
that would give them a clue. For this business one should be as little
conspicuous as possible\ldots{}. Trifles, trifles are what matter! Why,
it's just such trifles that always ruin everything\ldots{}.''

He had not far to go; he knew indeed how many steps it was from the gate
of his lodging house: exactly seven hundred and thirty. He had counted
them once when he had been lost in dreams. At the time he had put no
faith in those dreams and was only tantalising himself by their hideous
but daring recklessness. Now, a month later, he had begun to look upon
them differently, and, in spite of the monologues in which he jeered at
his own impotence and indecision, he had involuntarily come to regard
this ``hideous'' dream as an exploit to be attempted, although he still
did not realise this himself. He was positively going now for a
``rehearsal'' of his project, and at every step his excitement grew more
and more violent.

With a sinking heart and a nervous tremor, he went up to a huge house
which on one side looked on to the canal, and on the other into the
street. This house was let out in tiny tenements and was inhabited by
working people of all kinds---tailors, locksmiths, cooks, Germans of
sorts, girls picking up a living as best they could, petty clerks, etc.
There was a continual coming and going through the two gates and in the
two courtyards of the house. Three or four door-keepers were employed on
the building. The young man was very glad to meet none of them, and at
once slipped unnoticed through the door on the right, and up the
staircase. It was a back staircase, dark and narrow, but he was familiar
with it already, and knew his way, and he liked all these surroundings:
in such darkness even the most inquisitive eyes were not to be dreaded.

``If I am so scared now, what would it be if it somehow came to pass
that I were really going to do it?'' he could not help asking himself as
he reached the fourth storey. There his progress was barred by some
porters who were engaged in moving furniture out of a flat. He knew that
the flat had been occupied by a German clerk in the civil service, and
his family. This German was moving out then, and so the fourth floor on
this staircase would be untenanted except by the old woman. ``That's a
good thing anyway,'' he thought to himself, as he rang the bell of the
old woman's flat. The bell gave a faint tinkle as though it were made of
tin and not of copper. The little flats in such houses always have bells
that ring like that. He had forgotten the note of that bell, and now its
peculiar tinkle seemed to remind him of something and to bring it
clearly before him\ldots{}. He started, his nerves were terribly
overstrained by now. In a little while, the door was opened a tiny
crack: the old woman eyed her visitor with evident distrust through the
crack, and nothing could be seen but her little eyes, glittering in the
darkness. But, seeing a number of people on the landing, she grew
bolder, and opened the door wide. The young man stepped into the dark
entry, which was partitioned off from the tiny kitchen. The old woman
stood facing him in silence and looking inquiringly at him. She was a
diminutive, withered up old woman of sixty, with sharp malignant eyes
and a sharp little nose. Her colourless, somewhat grizzled hair was
thickly smeared with oil, and she wore no kerchief over it. Round her
thin long neck, which looked like a hen's leg, was knotted some sort of
flannel rag, and, in spite of the heat, there hung flapping on her
shoulders, a mangy fur cape, yellow with age. The old woman coughed and
groaned at every instant. The young man must have looked at her with a
rather peculiar expression, for a gleam of mistrust came into her eyes
again.

``Raskolnikov, a student, I came here a month ago,'' the young man made
haste to mutter, with a half bow, remembering that he ought to be more
polite.

``I remember, my good sir, I remember quite well your coming here,'' the
old woman said distinctly, still keeping her inquiring eyes on his face.

``And here\ldots{} I am again on the same errand,'' Raskolnikov
continued, a little disconcerted and surprised at the old woman's
mistrust. ``Perhaps she is always like that though, only I did not
notice it the other time,'' he thought with an uneasy feeling.

The old woman paused, as though hesitating; then stepped on one side,
and pointing to the door of the room, she said, letting her visitor pass
in front of her:

``Step in, my good sir.''

The little room into which the young man walked, with yellow paper on
the walls, geraniums and muslin curtains in the windows, was brightly
lighted up at that moment by the setting sun.

``So the sun will shine like this \emph{then} too!'' flashed as it were
by chance through Raskolnikov's mind, and with a rapid glance he scanned
everything in the room, trying as far as possible to notice and remember
its arrangement. But there was nothing special in the room. The
furniture, all very old and of yellow wood, consisted of a sofa with a
huge bent wooden back, an oval table in front of the sofa, a
dressing-table with a looking-glass fixed on it between the windows,
chairs along the walls and two or three half-penny prints in yellow
frames, representing German damsels with birds in their hands---that was
all. In the corner a light was burning before a small ikon. Everything
was very clean; the floor and the furniture were brightly polished;
everything shone.

``Lizaveta's work,'' thought the young man. There was not a speck of
dust to be seen in the whole flat.

``It's in the houses of spiteful old widows that one finds such
cleanliness,'' Raskolnikov thought again, and he stole a curious glance
at the cotton curtain over the door leading into another tiny room, in
which stood the old woman's bed and chest of drawers and into which he
had never looked before. These two rooms made up the whole flat.

``What do you want?'' the old woman said severely, coming into the room
and, as before, standing in front of him so as to look him straight in
the face.

``I've brought something to pawn here,'' and he drew out of his pocket
an old-fashioned flat silver watch, on the back of which was engraved a
globe; the chain was of steel.

``But the time is up for your last pledge. The month was up the day
before yesterday.''

``I will bring you the interest for another month; wait a little.''

``But that's for me to do as I please, my good sir, to wait or to sell
your pledge at once.''

``How much will you give me for the watch, Alyona Ivanovna?''

``You come with such trifles, my good sir, it's scarcely worth anything.
I gave you two roubles last time for your ring and one could buy it
quite new at a jeweler's for a rouble and a half.''

``Give me four roubles for it, I shall redeem it, it was my father's. I
shall be getting some money soon.''

``A rouble and a half, and interest in advance, if you like!''

``A rouble and a half!'' cried the young man.

``Please yourself''---and the old woman handed him back the watch. The
young man took it, and was so angry that he was on the point of going
away; but checked himself at once, remembering that there was nowhere
else he could go, and that he had had another object also in coming.

``Hand it over,'' he said roughly.

The old woman fumbled in her pocket for her keys, and disappeared behind
the curtain into the other room. The young man, left standing alone in
the middle of the room, listened inquisitively, thinking. He could hear
her unlocking the chest of drawers.

``It must be the top drawer,'' he reflected. ``So she carries the keys
in a pocket on the right. All in one bunch on a steel ring\ldots{}. And
there's one key there, three times as big as all the others, with deep
notches; that can't be the key of the chest of drawers\ldots{} then
there must be some other chest or strong-box\ldots{} that's worth
knowing. Strong-boxes always have keys like that\ldots{} but how
degrading it all is.''

The old woman came back.

``Here, sir: as we say ten copecks the rouble a month, so I must take
fifteen copecks from a rouble and a half for the month in advance. But
for the two roubles I lent you before, you owe me now twenty copecks on
the same reckoning in advance. That makes thirty-five copecks
altogether. So I must give you a rouble and fifteen copecks for the
watch. Here it is.''

``What! only a rouble and fifteen copecks now!''

``Just so.''

The young man did not dispute it and took the money. He looked at the
old woman, and was in no hurry to get away, as though there was still
something he wanted to say or to do, but he did not himself quite know
what.

``I may be bringing you something else in a day or two, Alyona
Ivanovna---a valuable thing---silver---a cigarette-box, as soon as I get
it back from a friend\ldots{}'' he broke off in confusion.

``Well, we will talk about it then, sir.''

``Good-bye---are you always at home alone, your sister is not here with
you?'' He asked her as casually as possible as he went out into the
passage.

``What business is she of yours, my good sir?''

``Oh, nothing particular, I simply asked. You are too quick\ldots{}.
Good-day, Alyona Ivanovna.''

Raskolnikov went out in complete confusion. This confusion became more
and more intense. As he went down the stairs, he even stopped short, two
or three times, as though suddenly struck by some thought. When he was
in the street he cried out, ``Oh, God, how loathsome it all is! and can
I, can I possibly\ldots{}. No, it's nonsense, it's rubbish!'' he added
resolutely. ``And how could such an atrocious thing come into my head?
What filthy things my heart is capable of. Yes, filthy above all,
disgusting, loathsome, loathsome!---and for a whole month I've
been\ldots{}.'' But no words, no exclamations, could express his
agitation. The feeling of intense repulsion, which had begun to oppress
and torture his heart while he was on his way to the old woman, had by
now reached such a pitch and had taken such a definite form that he did
not know what to do with himself to escape from his wretchedness. He
walked along the pavement like a drunken man, regardless of the
passers-by, and jostling against them, and only came to his senses when
he was in the next street. Looking round, he noticed that he was
standing close to a tavern which was entered by steps leading from the
pavement to the basement. At that instant two drunken men came out at
the door, and abusing and supporting one another, they mounted the
steps. Without stopping to think, Raskolnikov went down the steps at
once. Till that moment he had never been into a tavern, but now he felt
giddy and was tormented by a burning thirst. He longed for a drink of
cold beer, and attributed his sudden weakness to the want of food. He
sat down at a sticky little table in a dark and dirty corner; ordered
some beer, and eagerly drank off the first glassful. At once he felt
easier; and his thoughts became clear.

``All that's nonsense,'' he said hopefully, ``and there is nothing in it
all to worry about! It's simply physical derangement. Just a glass of
beer, a piece of dry bread---and in one moment the brain is stronger,
the mind is clearer and the will is firm! Phew, how utterly petty it all
is!''

But in spite of this scornful reflection, he was by now looking cheerful
as though he were suddenly set free from a terrible burden: and he gazed
round in a friendly way at the people in the room. But even at that
moment he had a dim foreboding that this happier frame of mind was also
not normal.

There were few people at the time in the tavern. Besides the two drunken
men he had met on the steps, a group consisting of about five men and a
girl with a concertina had gone out at the same time. Their departure
left the room quiet and rather empty. The persons still in the tavern
were a man who appeared to be an artisan, drunk, but not extremely so,
sitting before a pot of beer, and his companion, a huge, stout man with
a grey beard, in a short full-skirted coat. He was very drunk: and had
dropped asleep on the bench; every now and then, he began as though in
his sleep, cracking his fingers, with his arms wide apart and the upper
part of his body bounding about on the bench, while he hummed some
meaningless refrain, trying to recall some such lines as these:

\begin{quote}
\itshape
``His wife a year he fondly loved\\
His wife a---a year he---fondly loved.''
\end{quote}

Or suddenly waking up again:

\begin{quote}
\itshape
``Walking along the crowded row\\
He met the one he used to know.''
\end{quote}

But no one shared his enjoyment: his silent companion looked with
positive hostility and mistrust at all these manifestations. There was
another man in the room who looked somewhat like a retired government
clerk. He was sitting apart, now and then sipping from his pot and
looking round at the company. He, too, appeared to be in some agitation.

\protect\hypertarget{link2HCH0002}{}{}

\mychapter{2}{Chapter II}\label{chapter-ii}

Raskolnikov was not used to crowds, and, as we said before, he avoided
society of every sort, more especially of late. But now all at once he
felt a desire to be with other people. Something new seemed to be taking
place within him, and with it he felt a sort of thirst for company. He
was so weary after a whole month of concentrated wretchedness and gloomy
excitement that he longed to rest, if only for a moment, in some other
world, whatever it might be; and, in spite of the filthiness of the
surroundings, he was glad now to stay in the tavern.

The master of the establishment was in another room, but he frequently
came down some steps into the main room, his jaunty, tarred boots with
red turn-over tops coming into view each time before the rest of his
person. He wore a full coat and a horribly greasy black satin waistcoat,
with no cravat, and his whole face seemed smeared with oil like an iron
lock. At the counter stood a boy of about fourteen, and there was
another boy somewhat younger who handed whatever was wanted. On the
counter lay some sliced cucumber, some pieces of dried black bread, and
some fish, chopped up small, all smelling very bad. It was insufferably
close, and so heavy with the fumes of spirits that five minutes in such
an atmosphere might well make a man drunk.

There are chance meetings with strangers that interest us from the first
moment, before a word is spoken. Such was the impression made on
Raskolnikov by the person sitting a little distance from him, who looked
like a retired clerk. The young man often recalled this impression
afterwards, and even ascribed it to presentiment. He looked repeatedly
at the clerk, partly no doubt because the latter was staring
persistently at him, obviously anxious to enter into conversation. At
the other persons in the room, including the tavern-keeper, the clerk
looked as though he were used to their company, and weary of it, showing
a shade of condescending contempt for them as persons of station and
culture inferior to his own, with whom it would be useless for him to
converse. He was a man over fifty, bald and grizzled, of medium height,
and stoutly built. His face, bloated from continual drinking, was of a
yellow, even greenish, tinge, with swollen eyelids out of which keen
reddish eyes gleamed like little chinks. But there was something very
strange in him; there was a light in his eyes as though of intense
feeling---perhaps there were even thought and intelligence, but at the
same time there was a gleam of something like madness. He was wearing an
old and hopelessly ragged black dress coat, with all its buttons missing
except one, and that one he had buttoned, evidently clinging to this
last trace of respectability. A crumpled shirt front, covered with spots
and stains, protruded from his canvas waistcoat. Like a clerk, he wore
no beard, nor moustache, but had been so long unshaven that his chin
looked like a stiff greyish brush. And there was something respectable
and like an official about his manner too. But he was restless; he
ruffled up his hair and from time to time let his head drop into his
hands dejectedly resting his ragged elbows on the stained and sticky
table. At last he looked straight at Raskolnikov, and said loudly and
resolutely:

``May I venture, honoured sir, to engage you in polite conversation?
Forasmuch as, though your exterior would not command respect, my
experience admonishes me that you are a man of education and not
accustomed to drinking. I have always respected education when in
conjunction with genuine sentiments, and I am besides a titular
counsellor in rank. Marmeladov---such is my name; titular counsellor. I
make bold to inquire---have you been in the service?''

``No, I am studying,'' answered the young man, somewhat surprised at the
grandiloquent style of the speaker and also at being so directly
addressed. In spite of the momentary desire he had just been feeling for
company of any sort, on being actually spoken to he felt immediately his
habitual irritable and uneasy aversion for any stranger who approached
or attempted to approach him.

``A student then, or formerly a student,'' cried the clerk. ``Just what
I thought! I'm a man of experience, immense experience, sir,'' and he
tapped his forehead with his fingers in self-approval. ``You've been a
student or have attended some learned institution!\ldots{} But allow
me\ldots{}.'' He got up, staggered, took up his jug and glass, and sat
down beside the young man, facing him a little sideways. He was drunk,
but spoke fluently and boldly, only occasionally losing the thread of
his sentences and drawling his words. He pounced upon Raskolnikov as
greedily as though he too had not spoken to a soul for a month.

``Honoured sir,'' he began almost with solemnity, ``poverty is not a
vice, that's a true saying. Yet I know too that drunkenness is not a
virtue, and that that's even truer. But beggary, honoured sir, beggary
is a vice. In poverty you may still retain your innate nobility of soul,
but in beggary---never---no one. For beggary a man is not chased out of
human society with a stick, he is swept out with a broom, so as to make
it as humiliating as possible; and quite right, too, forasmuch as in
beggary I am ready to be the first to humiliate myself. Hence the
pot-house! Honoured sir, a month ago Mr. Lebeziatnikov gave my wife a
beating, and my wife is a very different matter from me! Do you
understand? Allow me to ask you another question out of simple
curiosity: have you ever spent a night on a hay barge, on the Neva?''

``No, I have not happened to,'' answered Raskolnikov. ``What do you
mean?''

``Well, I've just come from one and it's the fifth night I've slept
so\ldots{}.'' He filled his glass, emptied it and paused. Bits of hay
were in fact clinging to his clothes and sticking to his hair. It seemed
quite probable that he had not undressed or washed for the last five
days. His hands, particularly, were filthy. They were fat and red, with
black nails.

His conversation seemed to excite a general though languid interest. The
boys at the counter fell to sniggering. The innkeeper came down from the
upper room, apparently on purpose to listen to the ``funny fellow'' and
sat down at a little distance, yawning lazily, but with dignity.
Evidently Marmeladov was a familiar figure here, and he had most likely
acquired his weakness for high-flown speeches from the habit of
frequently entering into conversation with strangers of all sorts in the
tavern. This habit develops into a necessity in some drunkards, and
especially in those who are looked after sharply and kept in order at
home. Hence in the company of other drinkers they try to justify
themselves and even if possible obtain consideration.

``Funny fellow!'' pronounced the innkeeper. ``And why don't you work,
why aren't you at your duty, if you are in the service?''

``Why am I not at my duty, honoured sir,'' Marmeladov went on,
addressing himself exclusively to Raskolnikov, as though it had been he
who put that question to him. ``Why am I not at my duty? Does not my
heart ache to think what a useless worm I am? A month ago when Mr.
Lebeziatnikov beat my wife with his own hands, and I lay drunk, didn't I
suffer? Excuse me, young man, has it ever happened to you\ldots{}
hm\ldots{} well, to petition hopelessly for a loan?''

``Yes, it has. But what do you mean by hopelessly?''

``Hopelessly in the fullest sense, when you know beforehand that you
will get nothing by it. You know, for instance, beforehand with positive
certainty that this man, this most reputable and exemplary citizen, will
on no consideration give you money; and indeed I ask you why should he?
For he knows of course that I shan't pay it back. From compassion? But
Mr. Lebeziatnikov who keeps up with modern ideas explained the other day
that compassion is forbidden nowadays by science itself, and that that's
what is done now in England, where there is political economy. Why, I
ask you, should he give it to me? And yet though I know beforehand that
he won't, I set off to him and\ldots{}''

``Why do you go?'' put in Raskolnikov.

``Well, when one has no one, nowhere else one can go! For every man must
have somewhere to go. Since there are times when one absolutely must go
somewhere! When my own daughter first went out with a yellow ticket,
then I had to go\ldots{} (for my daughter has a yellow passport),'' he
added in parenthesis, looking with a certain uneasiness at the young
man. ``No matter, sir, no matter!'' he went on hurriedly and with
apparent composure when both the boys at the counter guffawed and even
the innkeeper smiled---``No matter, I am not confounded by the wagging
of their heads; for everyone knows everything about it already, and all
that is secret is made open. And I accept it all, not with contempt, but
with humility. So be it! So be it! `Behold the man!' Excuse me, young
man, can you\ldots{}. No, to put it more strongly and more distinctly;
not \emph{can} you but \emph{dare} you, looking upon me, assert that I
am not a pig?''

The young man did not answer a word.

``Well,'' the orator began again stolidly and with even increased
dignity, after waiting for the laughter in the room to subside. ``Well,
so be it, I am a pig, but she is a lady! I have the semblance of a
beast, but Katerina Ivanovna, my spouse, is a person of education and an
officer's daughter. Granted, granted, I am a scoundrel, but she is a
woman of a noble heart, full of sentiments, refined by education. And
yet\ldots{} oh, if only she felt for me! Honoured sir, honoured sir, you
know every man ought to have at least one place where people feel for
him! But Katerina Ivanovna, though she is magnanimous, she is
unjust\ldots{}. And yet, although I realise that when she pulls my hair
she only does it out of pity---for I repeat without being ashamed, she
pulls my hair, young man,'' he declared with redoubled dignity, hearing
the sniggering again---``but, my God, if she would but once\ldots{}. But
no, no! It's all in vain and it's no use talking! No use talking! For
more than once, my wish did come true and more than once she has felt
for me but\ldots{} such is my fate and I am a beast by nature!''

``Rather!'' assented the innkeeper yawning. Marmeladov struck his fist
resolutely on the table.

``Such is my fate! Do you know, sir, do you know, I have sold her very
stockings for drink? Not her shoes---that would be more or less in the
order of things, but her stockings, her stockings I have sold for drink!
Her mohair shawl I sold for drink, a present to her long ago, her own
property, not mine; and we live in a cold room and she caught cold this
winter and has begun coughing and spitting blood too. We have three
little children and Katerina Ivanovna is at work from morning till
night; she is scrubbing and cleaning and washing the children, for she's
been used to cleanliness from a child. But her chest is weak and she has
a tendency to consumption and I feel it! Do you suppose I don't feel it?
And the more I drink the more I feel it. That's why I drink too. I try
to find sympathy and feeling in drink\ldots{}. I drink so that I may
suffer twice as much!'' And as though in despair he laid his head down
on the table.

``Young man,'' he went on, raising his head again, ``in your face I seem
to read some trouble of mind. When you came in I read it, and that was
why I addressed you at once. For in unfolding to you the story of my
life, I do not wish to make myself a laughing-stock before these idle
listeners, who indeed know all about it already, but I am looking for a
man of feeling and education. Know then that my wife was educated in a
high-class school for the daughters of noblemen, and on leaving she
danced the shawl dance before the governor and other personages for
which she was presented with a gold medal and a certificate of merit.
The medal\ldots{} well, the medal of course was sold---long ago,
hm\ldots{} but the certificate of merit is in her trunk still and not
long ago she showed it to our landlady. And although she is most
continually on bad terms with the landlady, yet she wanted to tell
someone or other of her past honours and of the happy days that are
gone. I don't condemn her for it, I don't blame her, for the one thing
left her is recollection of the past, and all the rest is dust and
ashes. Yes, yes, she is a lady of spirit, proud and determined. She
scrubs the floors herself and has nothing but black bread to eat, but
won't allow herself to be treated with disrespect. That's why she would
not overlook Mr. Lebeziatnikov's rudeness to her, and so when he gave
her a beating for it, she took to her bed more from the hurt to her
feelings than from the blows. She was a widow when I married her, with
three children, one smaller than the other. She married her first
husband, an infantry officer, for love, and ran away with him from her
father's house. She was exceedingly fond of her husband; but he gave way
to cards, got into trouble and with that he died. He used to beat her at
the end: and although she paid him back, of which I have authentic
documentary evidence, to this day she speaks of him with tears and she
throws him up to me; and I am glad, I am glad that, though only in
imagination, she should think of herself as having once been
happy\ldots{}. And she was left at his death with three children in a
wild and remote district where I happened to be at the time; and she was
left in such hopeless poverty that, although I have seen many ups and
downs of all sort, I don't feel equal to describing it even. Her
relations had all thrown her off. And she was proud, too, excessively
proud\ldots{}. And then, honoured sir, and then, I, being at the time a
widower, with a daughter of fourteen left me by my first wife, offered
her my hand, for I could not bear the sight of such suffering. You can
judge the extremity of her calamities, that she, a woman of education
and culture and distinguished family, should have consented to be my
wife. But she did! Weeping and sobbing and wringing her hands, she
married me! For she had nowhere to turn! Do you understand, sir, do you
understand what it means when you have absolutely nowhere to turn? No,
that you don't understand yet\ldots{}. And for a whole year, I performed
my duties conscientiously and faithfully, and did not touch this'' (he
tapped the jug with his finger), ``for I have feelings. But even so, I
could not please her; and then I lost my place too, and that through no
fault of mine but through changes in the office; and then I did touch
it!\ldots{} It will be a year and a half ago soon since we found
ourselves at last after many wanderings and numerous calamities in this
magnificent capital, adorned with innumerable monuments. Here I obtained
a situation\ldots{}. I obtained it and I lost it again. Do you
understand? This time it was through my own fault I lost it: for my
weakness had come out\ldots{}. We have now part of a room at Amalia
Fyodorovna Lippevechsel's; and what we live upon and what we pay our
rent with, I could not say. There are a lot of people living there
besides ourselves. Dirt and disorder, a perfect Bedlam\ldots{}
hm\ldots{} yes\ldots{} And meanwhile my daughter by my first wife has
grown up; and what my daughter has had to put up with from her
step-mother whilst she was growing up, I won't speak of. For, though
Katerina Ivanovna is full of generous feelings, she is a spirited lady,
irritable and short-tempered\ldots{}. Yes. But it's no use going over
that! Sonia, as you may well fancy, has had no education. I did make an
effort four years ago to give her a course of geography and universal
history, but as I was not very well up in those subjects myself and we
had no suitable books, and what books we had\ldots{} hm, anyway we have
not even those now, so all our instruction came to an end. We stopped at
Cyrus of Persia. Since she has attained years of maturity, she has read
other books of romantic tendency and of late she had read with great
interest a book she got through Mr. Lebeziatnikov, Lewes'
Physiology---do you know it?---and even recounted extracts from it to
us: and that's the whole of her education. And now may I venture to
address you, honoured sir, on my own account with a private question. Do
you suppose that a respectable poor girl can earn much by honest work?
Not fifteen farthings a day can she earn, if she is respectable and has
no special talent and that without putting her work down for an instant!
And what's more, Ivan Ivanitch Klopstock the civil counsellor---have you
heard of him?---has not to this day paid her for the half-dozen linen
shirts she made him and drove her roughly away, stamping and reviling
her, on the pretext that the shirt collars were not made like the
pattern and were put in askew. And there are the little ones
hungry\ldots{}. And Katerina Ivanovna walking up and down and wringing
her hands, her cheeks flushed red, as they always are in that disease:
`Here you live with us,' says she, `you eat and drink and are kept warm
and you do nothing to help.' And much she gets to eat and drink when
there is not a crust for the little ones for three days! I was lying at
the time\ldots{} well, what of it! I was lying drunk and I heard my
Sonia speaking (she is a gentle creature with a soft little
voice\ldots{} fair hair and such a pale, thin little face). She said:
`Katerina Ivanovna, am I really to do a thing like that?' And Darya
Frantsovna, a woman of evil character and very well known to the police,
had two or three times tried to get at her through the landlady. `And
why not?' said Katerina Ivanovna with a jeer, `you are something mighty
precious to be so careful of!' But don't blame her, don't blame her,
honoured sir, don't blame her! She was not herself when she spoke, but
driven to distraction by her illness and the crying of the hungry
children; and it was said more to wound her than anything else\ldots{}.
For that's Katerina Ivanovna's character, and when children cry, even
from hunger, she falls to beating them at once. At six o'clock I saw
Sonia get up, put on her kerchief and her cape, and go out of the room
and about nine o'clock she came back. She walked straight up to Katerina
Ivanovna and she laid thirty roubles on the table before her in silence.
She did not utter a word, she did not even look at her, she simply
picked up our big green \emph{drap de dames} shawl (we have a shawl,
made of \emph{drap de dames}), put it over her head and face and lay
down on the bed with her face to the wall; only her little shoulders and
her body kept shuddering\ldots{}. And I went on lying there, just as
before\ldots{}. And then I saw, young man, I saw Katerina Ivanovna, in
the same silence go up to Sonia's little bed; she was on her knees all
the evening kissing Sonia's feet, and would not get up, and then they
both fell asleep in each other's arms\ldots{} together, together\ldots{}
yes\ldots{} and I\ldots{} lay drunk.''

Marmeladov stopped short, as though his voice had failed him. Then he
hurriedly filled his glass, drank, and cleared his throat.

``Since then, sir,'' he went on after a brief pause---``Since then,
owing to an unfortunate occurrence and through information given by
evil-intentioned persons---in all which Darya Frantsovna took a leading
part on the pretext that she had been treated with want of
respect---since then my daughter Sofya Semyonovna has been forced to
take a yellow ticket, and owing to that she is unable to go on living
with us. For our landlady, Amalia Fyodorovna would not hear of it
(though she had backed up Darya Frantsovna before) and Mr. Lebeziatnikov
too\ldots{} hm\ldots{}. All the trouble between him and Katerina
Ivanovna was on Sonia's account. At first he was for making up to Sonia
himself and then all of a sudden he stood on his dignity: `how,' said
he, `can a highly educated man like me live in the same rooms with a
girl like that?' And Katerina Ivanovna would not let it pass, she stood
up for her\ldots{} and so that's how it happened. And Sonia comes to us
now, mostly after dark; she comforts Katerina Ivanovna and gives her all
she can\ldots{}. She has a room at the Kapernaumovs' the tailors, she
lodges with them; Kapernaumov is a lame man with a cleft palate and all
of his numerous family have cleft palates too. And his wife, too, has a
cleft palate. They all live in one room, but Sonia has her own,
partitioned off\ldots{}. Hm\ldots{} yes\ldots{} very poor people and all
with cleft palates\ldots{} yes. Then I got up in the morning, and put on
my rags, lifted up my hands to heaven and set off to his excellency Ivan
Afanasyvitch. His excellency Ivan Afanasyvitch, do you know him? No?
Well, then, it's a man of God you don't know. He is wax\ldots{} wax
before the face of the Lord; even as wax melteth!\ldots{} His eyes were
dim when he heard my story. `Marmeladov, once already you have deceived
my expectations\ldots{} I'll take you once more on my own
responsibility'---that's what he said, `remember,' he said, `and now you
can go.' I kissed the dust at his feet---in thought only, for in reality
he would not have allowed me to do it, being a statesman and a man of
modern political and enlightened ideas. I returned home, and when I
announced that I'd been taken back into the service and should receive a
salary, heavens, what a to-do there was!\ldots{}''

Marmeladov stopped again in violent excitement. At that moment a whole
party of revellers already drunk came in from the street, and the sounds
of a hired concertina and the cracked piping voice of a child of seven
singing ``The Hamlet'' were heard in the entry. The room was filled with
noise. The tavern-keeper and the boys were busy with the new-comers.
Marmeladov paying no attention to the new arrivals continued his story.
He appeared by now to be extremely weak, but as he became more and more
drunk, he became more and more talkative. The recollection of his recent
success in getting the situation seemed to revive him, and was
positively reflected in a sort of radiance on his face. Raskolnikov
listened attentively.

``That was five weeks ago, sir. Yes\ldots{}. As soon as Katerina
Ivanovna and Sonia heard of it, mercy on us, it was as though I stepped
into the kingdom of Heaven. It used to be: you can lie like a beast,
nothing but abuse. Now they were walking on tiptoe, hushing the
children. `Semyon Zaharovitch is tired with his work at the office, he
is resting, shh!' They made me coffee before I went to work and boiled
cream for me! They began to get real cream for me, do you hear that? And
how they managed to get together the money for a decent outfit---eleven
roubles, fifty copecks, I can't guess. Boots, cotton shirt-fronts---most
magnificent, a uniform, they got up all in splendid style, for eleven
roubles and a half. The first morning I came back from the office I
found Katerina Ivanovna had cooked two courses for dinner---soup and
salt meat with horse radish---which we had never dreamed of till then.
She had not any dresses\ldots{} none at all, but she got herself up as
though she were going on a visit; and not that she'd anything to do it
with, she smartened herself up with nothing at all, she'd done her hair
nicely, put on a clean collar of some sort, cuffs, and there she was,
quite a different person, she was younger and better looking. Sonia, my
little darling, had only helped with money `for the time,' she said, `it
won't do for me to come and see you too often. After dark maybe when no
one can see.' Do you hear, do you hear? I lay down for a nap after
dinner and what do you think: though Katerina Ivanovna had quarrelled to
the last degree with our landlady Amalia Fyodorovna only a week before,
she could not resist then asking her in to coffee. For two hours they
were sitting, whispering together. `Semyon Zaharovitch is in the service
again, now, and receiving a salary,' says she, `and he went himself to
his excellency and his excellency himself came out to him, made all the
others wait and led Semyon Zaharovitch by the hand before everybody into
his study.' Do you hear, do you hear? `To be sure,' says he, `Semyon
Zaharovitch, remembering your past services,' says he, `and in spite of
your propensity to that foolish weakness, since you promise now and
since moreover we've got on badly without you,' (do you hear, do you
hear;) `and so,' says he, `I rely now on your word as a gentleman.' And
all that, let me tell you, she has simply made up for herself, and not
simply out of wantonness, for the sake of bragging; no, she believes it
all herself, she amuses herself with her own fancies, upon my word she
does! And I don't blame her for it, no, I don't blame her!\ldots{} Six
days ago when I brought her my first earnings in full---twenty-three
roubles forty copecks altogether---she called me her poppet: `poppet,'
said she, `my little poppet.' And when we were by ourselves, you
understand? You would not think me a beauty, you would not think much of
me as a husband, would you?\ldots{} Well, she pinched my cheek, `my
little poppet,' said she.''

Marmeladov broke off, tried to smile, but suddenly his chin began to
twitch. He controlled himself however. The tavern, the degraded
appearance of the man, the five nights in the hay barge, and the pot of
spirits, and yet this poignant love for his wife and children bewildered
his listener. Raskolnikov listened intently but with a sick sensation.
He felt vexed that he had come here.

``Honoured sir, honoured sir,'' cried Marmeladov recovering
himself---``Oh, sir, perhaps all this seems a laughing matter to you, as
it does to others, and perhaps I am only worrying you with the stupidity
of all the trivial details of my home life, but it is not a laughing
matter to me. For I can feel it all\ldots{}. And the whole of that
heavenly day of my life and the whole of that evening I passed in
fleeting dreams of how I would arrange it all, and how I would dress all
the children, and how I should give her rest, and how I should rescue my
own daughter from dishonour and restore her to the bosom of her
family\ldots{}. And a great deal more\ldots{}. Quite excusable, sir.
Well, then, sir'' (Marmeladov suddenly gave a sort of start, raised his
head and gazed intently at his listener) ``well, on the very next day
after all those dreams, that is to say, exactly five days ago, in the
evening, by a cunning trick, like a thief in the night, I stole from
Katerina Ivanovna the key of her box, took out what was left of my
earnings, how much it was I have forgotten, and now look at me, all of
you! It's the fifth day since I left home, and they are looking for me
there and it's the end of my employment, and my uniform is lying in a
tavern on the Egyptian bridge. I exchanged it for the garments I have
on\ldots{} and it's the end of everything!''

Marmeladov struck his forehead with his fist, clenched his teeth, closed
his eyes and leaned heavily with his elbow on the table. But a minute
later his face suddenly changed and with a certain assumed slyness and
affectation of bravado, he glanced at Raskolnikov, laughed and said:

``This morning I went to see Sonia, I went to ask her for a pick-me-up!
He-he-he!''

``You don't say she gave it to you?'' cried one of the new-comers; he
shouted the words and went off into a guffaw.

``This very quart was bought with her money,'' Marmeladov declared,
addressing himself exclusively to Raskolnikov. ``Thirty copecks she gave
me with her own hands, her last, all she had, as I saw\ldots{}. She said
nothing, she only looked at me without a word\ldots{}. Not on earth, but
up yonder\ldots{} they grieve over men, they weep, but they don't blame
them, they don't blame them! But it hurts more, it hurts more when they
don't blame! Thirty copecks yes! And maybe she needs them now, eh? What
do you think, my dear sir? For now she's got to keep up her appearance.
It costs money, that smartness, that special smartness, you know? Do you
understand? And there's pomatum, too, you see, she must have things;
petticoats, starched ones, shoes, too, real jaunty ones to show off her
foot when she has to step over a puddle. Do you understand, sir, do you
understand what all that smartness means? And here I, her own father,
here I took thirty copecks of that money for a drink! And I am drinking
it! And I have already drunk it! Come, who will have pity on a man like
me, eh? Are you sorry for me, sir, or not? Tell me, sir, are you sorry
or not? He-he-he!''

He would have filled his glass, but there was no drink left. The pot was
empty.

``What are you to be pitied for?'' shouted the tavern-keeper who was
again near them.

Shouts of laughter and even oaths followed. The laughter and the oaths
came from those who were listening and also from those who had heard
nothing but were simply looking at the figure of the discharged
government clerk.

``To be pitied! Why am I to be pitied?'' Marmeladov suddenly declaimed,
standing up with his arm outstretched, as though he had been only
waiting for that question.

``Why am I to be pitied, you say? Yes! there's nothing to pity me for! I
ought to be crucified, crucified on a cross, not pitied! Crucify me, oh
judge, crucify me but pity me! And then I will go of myself to be
crucified, for it's not merry-making I seek but tears and
tribulation!\ldots{} Do you suppose, you that sell, that this pint of
yours has been sweet to me? It was tribulation I sought at the bottom of
it, tears and tribulation, and have found it, and I have tasted it; but
He will pity us Who has had pity on all men, Who has understood all men
and all things, He is the One, He too is the judge. He will come in that
day and He will ask: `Where is the daughter who gave herself for her
cross, consumptive step-mother and for the little children of another?
Where is the daughter who had pity upon the filthy drunkard, her earthly
father, undismayed by his beastliness?' And He will say, `Come to me! I
have already forgiven thee once\ldots{}. I have forgiven thee
once\ldots{}. Thy sins which are many are forgiven thee for thou hast
loved much\ldots{}.' And he will forgive my Sonia, He will forgive, I
know it\ldots{} I felt it in my heart when I was with her just now! And
He will judge and will forgive all, the good and the evil, the wise and
the meek\ldots{}. And when He has done with all of them, then He will
summon us. `You too come forth,' He will say, `Come forth ye drunkards,
come forth, ye weak ones, come forth, ye children of shame!' And we
shall all come forth, without shame and shall stand before him. And He
will say unto us, `Ye are swine, made in the Image of the Beast and with
his mark; but come ye also!' And the wise ones and those of
understanding will say, `Oh Lord, why dost Thou receive these men?' And
He will say, `This is why I receive them, oh ye wise, this is why I
receive them, oh ye of understanding, that not one of them believed
himself to be worthy of this.' And He will hold out His hands to us and
we shall fall down before him\ldots{} and we shall weep\ldots{} and we
shall understand all things! Then we shall understand all!\ldots{} and
all will understand, Katerina Ivanovna even\ldots{} she will
understand\ldots{}. Lord, Thy kingdom come!'' And he sank down on the
bench exhausted, and helpless, looking at no one, apparently oblivious
of his surroundings and plunged in deep thought. His words had created a
certain impression; there was a moment of silence; but soon laughter and
oaths were heard again.

``That's his notion!''

``Talked himself silly!''

``A fine clerk he is!''

And so on, and so on.

``Let us go, sir,'' said Marmeladov all at once, raising his head and
addressing Raskolnikov---``come along with me\ldots{} Kozel's house,
looking into the yard. I'm going to Katerina Ivanovna---time I did.''

Raskolnikov had for some time been wanting to go and he had meant to
help him. Marmeladov was much unsteadier on his legs than in his speech
and leaned heavily on the young man. They had two or three hundred paces
to go. The drunken man was more and more overcome by dismay and
confusion as they drew nearer the house.

``It's not Katerina Ivanovna I am afraid of now,'' he muttered in
agitation---``and that she will begin pulling my hair. What does my hair
matter! Bother my hair! That's what I say! Indeed it will be better if
she does begin pulling it, that's not what I am afraid of\ldots{} it's
her eyes I am afraid of\ldots{} yes, her eyes\ldots{} the red on her
cheeks, too, frightens me\ldots{} and her breathing too\ldots{}. Have
you noticed how people in that disease breathe\ldots{} when they are
excited? I am frightened of the children's crying, too\ldots{}. For if
Sonia has not taken them food\ldots{} I don't know what's happened! I
don't know! But blows I am not afraid of\ldots{}. Know, sir, that such
blows are not a pain to me, but even an enjoyment. In fact I can't get
on without it\ldots{}. It's better so. Let her strike me, it relieves
her heart\ldots{} it's better so\ldots{} There is the house. The house
of Kozel, the cabinet-maker\ldots{} a German, well-to-do. Lead the
way!''

They went in from the yard and up to the fourth storey. The staircase
got darker and darker as they went up. It was nearly eleven o'clock and
although in summer in Petersburg there is no real night, yet it was
quite dark at the top of the stairs.

A grimy little door at the very top of the stairs stood ajar. A very
poor-looking room about ten paces long was lighted up by a candle-end;
the whole of it was visible from the entrance. It was all in disorder,
littered up with rags of all sorts, especially children's garments.
Across the furthest corner was stretched a ragged sheet. Behind it
probably was the bed. There was nothing in the room except two chairs
and a sofa covered with American leather, full of holes, before which
stood an old deal kitchen-table, unpainted and uncovered. At the edge of
the table stood a smoldering tallow-candle in an iron candlestick. It
appeared that the family had a room to themselves, not part of a room,
but their room was practically a passage. The door leading to the other
rooms, or rather cupboards, into which Amalia Lippevechsel's flat was
divided stood half open, and there was shouting, uproar and laughter
within. People seemed to be playing cards and drinking tea there. Words
of the most unceremonious kind flew out from time to time.

Raskolnikov recognised Katerina Ivanovna at once. She was a rather tall,
slim and graceful woman, terribly emaciated, with magnificent dark brown
hair and with a hectic flush in her cheeks. She was pacing up and down
in her little room, pressing her hands against her chest; her lips were
parched and her breathing came in nervous broken gasps. Her eyes
glittered as in fever and looked about with a harsh immovable stare. And
that consumptive and excited face with the last flickering light of the
candle-end playing upon it made a sickening impression. She seemed to
Raskolnikov about thirty years old and was certainly a strange wife for
Marmeladov\ldots{}. She had not heard them and did not notice them
coming in. She seemed to be lost in thought, hearing and seeing nothing.
The room was close, but she had not opened the window; a stench rose
from the staircase, but the door on to the stairs was not closed. From
the inner rooms clouds of tobacco smoke floated in, she kept coughing,
but did not close the door. The youngest child, a girl of six, was
asleep, sitting curled up on the floor with her head on the sofa. A boy
a year older stood crying and shaking in the corner, probably he had
just had a beating. Beside him stood a girl of nine years old, tall and
thin, wearing a thin and ragged chemise with an ancient cashmere pelisse
flung over her bare shoulders, long outgrown and barely reaching her
knees. Her arm, as thin as a stick, was round her brother's neck. She
was trying to comfort him, whispering something to him, and doing all
she could to keep him from whimpering again. At the same time her large
dark eyes, which looked larger still from the thinness of her frightened
face, were watching her mother with alarm. Marmeladov did not enter the
door, but dropped on his knees in the very doorway, pushing Raskolnikov
in front of him. The woman seeing a stranger stopped indifferently
facing him, coming to herself for a moment and apparently wondering what
he had come for. But evidently she decided that he was going into the
next room, as he had to pass through hers to get there. Taking no
further notice of him, she walked towards the outer door to close it and
uttered a sudden scream on seeing her husband on his knees in the
doorway.

``Ah!'' she cried out in a frenzy, ``he has come back! The criminal! the
monster!\ldots{} And where is the money? What's in your pocket, show me!
And your clothes are all different! Where are your clothes? Where is the
money! Speak!''

And she fell to searching him. Marmeladov submissively and obediently
held up both arms to facilitate the search. Not a farthing was there.

``Where is the money?'' she cried---``Mercy on us, can he have drunk it
all? There were twelve silver roubles left in the chest!'' and in a fury
she seized him by the hair and dragged him into the room. Marmeladov
seconded her efforts by meekly crawling along on his knees.

``And this is a consolation to me! This does not hurt me, but is a
positive con-so-la-tion, ho-nou-red sir,'' he called out, shaken to and
fro by his hair and even once striking the ground with his forehead. The
child asleep on the floor woke up, and began to cry. The boy in the
corner losing all control began trembling and screaming and rushed to
his sister in violent terror, almost in a fit. The eldest girl was
shaking like a leaf.

``He's drunk it! he's drunk it all,'' the poor woman screamed in
despair---``and his clothes are gone! And they are hungry,
hungry!''---and wringing her hands she pointed to the children. ``Oh,
accursed life! And you, are you not ashamed?''---she pounced all at once
upon Raskolnikov---``from the tavern! Have you been drinking with him?
You have been drinking with him, too! Go away!''

The young man was hastening away without uttering a word. The inner door
was thrown wide open and inquisitive faces were peering in at it. Coarse
laughing faces with pipes and cigarettes and heads wearing caps thrust
themselves in at the doorway. Further in could be seen figures in
dressing gowns flung open, in costumes of unseemly scantiness, some of
them with cards in their hands. They were particularly diverted, when
Marmeladov, dragged about by his hair, shouted that it was a consolation
to him. They even began to come into the room; at last a sinister shrill
outcry was heard: this came from Amalia Lippevechsel herself pushing her
way amongst them and trying to restore order after her own fashion and
for the hundredth time to frighten the poor woman by ordering her with
coarse abuse to clear out of the room next day. As he went out,
Raskolnikov had time to put his hand into his pocket, to snatch up the
coppers he had received in exchange for his rouble in the tavern and to
lay them unnoticed on the window. Afterwards on the stairs, he changed
his mind and would have gone back.

``What a stupid thing I've done,'' he thought to himself, ``they have
Sonia and I want it myself.'' But reflecting that it would be impossible
to take it back now and that in any case he would not have taken it, he
dismissed it with a wave of his hand and went back to his lodging.
``Sonia wants pomatum too,'' he said as he walked along the street, and
he laughed malignantly---``such smartness costs money\ldots{}. Hm! And
maybe Sonia herself will be bankrupt to-day, for there is always a risk,
hunting big game\ldots{} digging for gold\ldots{} then they would all be
without a crust to-morrow except for my money. Hurrah for Sonia! What a
mine they've dug there! And they're making the most of it! Yes, they are
making the most of it! They've wept over it and grown used to it. Man
grows used to everything, the scoundrel!''

He sank into thought.

``And what if I am wrong,'' he cried suddenly after a moment's thought.
``What if man is not really a scoundrel, man in general, I mean, the
whole race of mankind---then all the rest is prejudice, simply
artificial terrors and there are no barriers and it's all as it should
be.''

\protect\hypertarget{link2HCH0003}{}{}

\mychapter{3}{Chapter III}\label{chapter-iii}

He waked up late next day after a broken sleep. But his sleep had not
refreshed him; he waked up bilious, irritable, ill-tempered, and looked
with hatred at his room. It was a tiny cupboard of a room about six
paces in length. It had a poverty-stricken appearance with its dusty
yellow paper peeling off the walls, and it was so low-pitched that a man
of more than average height was ill at ease in it and felt every moment
that he would knock his head against the ceiling. The furniture was in
keeping with the room: there were three old chairs, rather rickety; a
painted table in the corner on which lay a few manuscripts and books;
the dust that lay thick upon them showed that they had been long
untouched. A big clumsy sofa occupied almost the whole of one wall and
half the floor space of the room; it was once covered with chintz, but
was now in rags and served Raskolnikov as a bed. Often he went to sleep
on it, as he was, without undressing, without sheets, wrapped in his old
student's overcoat, with his head on one little pillow, under which he
heaped up all the linen he had, clean and dirty, by way of a bolster. A
little table stood in front of the sofa.

It would have been difficult to sink to a lower ebb of disorder, but to
Raskolnikov in his present state of mind this was positively agreeable.
He had got completely away from everyone, like a tortoise in its shell,
and even the sight of a servant girl who had to wait upon him and looked
sometimes into his room made him writhe with nervous irritation. He was
in the condition that overtakes some monomaniacs entirely concentrated
upon one thing. His landlady had for the last fortnight given up sending
him in meals, and he had not yet thought of expostulating with her,
though he went without his dinner. Nastasya, the cook and only servant,
was rather pleased at the lodger's mood and had entirely given up
sweeping and doing his room, only once a week or so she would stray into
his room with a broom. She waked him up that day.

``Get up, why are you asleep?'' she called to him. ``It's past nine, I
have brought you some tea; will you have a cup? I should think you're
fairly starving?''

Raskolnikov opened his eyes, started and recognised Nastasya.

``From the landlady, eh?'' he asked, slowly and with a sickly face
sitting up on the sofa.

``From the landlady, indeed!''

She set before him her own cracked teapot full of weak and stale tea and
laid two yellow lumps of sugar by the side of it.

``Here, Nastasya, take it please,'' he said, fumbling in his pocket (for
he had slept in his clothes) and taking out a handful of coppers---``run
and buy me a loaf. And get me a little sausage, the cheapest, at the
pork-butcher's.''

``The loaf I'll fetch you this very minute, but wouldn't you rather have
some cabbage soup instead of sausage? It's capital soup, yesterday's. I
saved it for you yesterday, but you came in late. It's fine soup.''

When the soup had been brought, and he had begun upon it, Nastasya sat
down beside him on the sofa and began chatting. She was a country
peasant-woman and a very talkative one.

``Praskovya Pavlovna means to complain to the police about you,'' she
said.

He scowled.

``To the police? What does she want?''

``You don't pay her money and you won't turn out of the room. That's
what she wants, to be sure.''

``The devil, that's the last straw,'' he muttered, grinding his teeth,
``no, that would not suit me\ldots{} just now. She is a fool,'' he added
aloud. ``I'll go and talk to her to-day.''

``Fool she is and no mistake, just as I am. But why, if you are so
clever, do you lie here like a sack and have nothing to show for it? One
time you used to go out, you say, to teach children. But why is it you
do nothing now?''

``I am doing\ldots{}'' Raskolnikov began sullenly and reluctantly.

``What are you doing?''

``Work\ldots{}''

``What sort of work?''

``I am thinking,'' he answered seriously after a pause.

Nastasya was overcome with a fit of laughter. She was given to laughter
and when anything amused her, she laughed inaudibly, quivering and
shaking all over till she felt ill.

``And have you made much money by your thinking?'' she managed to
articulate at last.

``One can't go out to give lessons without boots. And I'm sick of it.''

``Don't quarrel with your bread and butter.''

``They pay so little for lessons. What's the use of a few coppers?'' he
answered, reluctantly, as though replying to his own thought.

``And you want to get a fortune all at once?''

He looked at her strangely.

``Yes, I want a fortune,'' he answered firmly, after a brief pause.

``Don't be in such a hurry, you quite frighten me! Shall I get you the
loaf or not?''

``As you please.''

``Ah, I forgot! A letter came for you yesterday when you were out.''

``A letter? for me! from whom?''

``I can't say. I gave three copecks of my own to the postman for it.
Will you pay me back?''

``Then bring it to me, for God's sake, bring it,'' cried Raskolnikov
greatly excited---``good God!''

A minute later the letter was brought him. That was it: from his mother,
from the province of R------. He turned pale when he took it. It was a
long while since he had received a letter, but another feeling also
suddenly stabbed his heart.

``Nastasya, leave me alone, for goodness' sake; here are your three
copecks, but for goodness' sake, make haste and go!''

The letter was quivering in his hand; he did not want to open it in her
presence; he wanted to be left \emph{alone} with this letter. When
Nastasya had gone out, he lifted it quickly to his lips and kissed it;
then he gazed intently at the address, the small, sloping handwriting,
so dear and familiar, of the mother who had once taught him to read and
write. He delayed; he seemed almost afraid of something. At last he
opened it; it was a thick heavy letter, weighing over two ounces, two
large sheets of note paper were covered with very small handwriting.

``My dear Rodya,'' wrote his mother---``it's two months since I last had
a talk with you by letter which has distressed me and even kept me awake
at night, thinking. But I am sure you will not blame me for my
inevitable silence. You know how I love you; you are all we have to look
to, Dounia and I, you are our all, our one hope, our one stay. What a
grief it was to me when I heard that you had given up the university
some months ago, for want of means to keep yourself and that you had
lost your lessons and your other work! How could I help you out of my
hundred and twenty roubles a year pension? The fifteen roubles I sent
you four months ago I borrowed, as you know, on security of my pension,
from Vassily Ivanovitch Vahrushin a merchant of this town. He is a
kind-hearted man and was a friend of your father's too. But having given
him the right to receive the pension, I had to wait till the debt was
paid off and that is only just done, so that I've been unable to send
you anything all this time. But now, thank God, I believe I shall be
able to send you something more and in fact we may congratulate
ourselves on our good fortune now, of which I hasten to inform you. In
the first place, would you have guessed, dear Rodya, that your sister
has been living with me for the last six weeks and we shall not be
separated in the future. Thank God, her sufferings are over, but I will
tell you everything in order, so that you may know just how everything
has happened and all that we have hitherto concealed from you. When you
wrote to me two months ago that you had heard that Dounia had a great
deal to put up with in the Svidrigraïlovs' house, when you wrote that
and asked me to tell you all about it---what could I write in answer to
you? If I had written the whole truth to you, I dare say you would have
thrown up everything and have come to us, even if you had to walk all
the way, for I know your character and your feelings, and you would not
let your sister be insulted. I was in despair myself, but what could I
do? And, besides, I did not know the whole truth myself then. What made
it all so difficult was that Dounia received a hundred roubles in
advance when she took the place as governess in their family, on
condition of part of her salary being deducted every month, and so it
was impossible to throw up the situation without repaying the debt. This
sum (now I can explain it all to you, my precious Rodya) she took
chiefly in order to send you sixty roubles, which you needed so terribly
then and which you received from us last year. We deceived you then,
writing that this money came from Dounia's savings, but that was not so,
and now I tell you all about it, because, thank God, things have
suddenly changed for the better, and that you may know how Dounia loves
you and what a heart she has. At first indeed Mr. Svidrigaïlov treated
her very rudely and used to make disrespectful and jeering remarks at
table\ldots{}. But I don't want to go into all those painful details, so
as not to worry you for nothing when it is now all over. In short, in
spite of the kind and generous behaviour of Marfa Petrovna, Mr.
Svidrigaïlov's wife, and all the rest of the household, Dounia had a
very hard time, especially when Mr. Svidrigaïlov, relapsing into his old
regimental habits, was under the influence of Bacchus. And how do you
think it was all explained later on? Would you believe that the crazy
fellow had conceived a passion for Dounia from the beginning, but had
concealed it under a show of rudeness and contempt. Possibly he was
ashamed and horrified himself at his own flighty hopes, considering his
years and his being the father of a family; and that made him angry with
Dounia. And possibly, too, he hoped by his rude and sneering behaviour
to hide the truth from others. But at last he lost all control and had
the face to make Dounia an open and shameful proposal, promising her all
sorts of inducements and offering, besides, to throw up everything and
take her to another estate of his, or even abroad. You can imagine all
she went through! To leave her situation at once was impossible not only
on account of the money debt, but also to spare the feelings of Marfa
Petrovna, whose suspicions would have been aroused: and then Dounia
would have been the cause of a rupture in the family. And it would have
meant a terrible scandal for Dounia too; that would have been
inevitable. There were various other reasons owing to which Dounia could
not hope to escape from that awful house for another six weeks. You know
Dounia, of course; you know how clever she is and what a strong will she
has. Dounia can endure a great deal and even in the most difficult cases
she has the fortitude to maintain her firmness. She did not even write
to me about everything for fear of upsetting me, although we were
constantly in communication. It all ended very unexpectedly. Marfa
Petrovna accidentally overheard her husband imploring Dounia in the
garden, and, putting quite a wrong interpretation on the position, threw
the blame upon her, believing her to be the cause of it all. An awful
scene took place between them on the spot in the garden; Marfa Petrovna
went so far as to strike Dounia, refused to hear anything and was
shouting at her for a whole hour and then gave orders that Dounia should
be packed off at once to me in a plain peasant's cart, into which they
flung all her things, her linen and her clothes, all pell-mell, without
folding it up and packing it. And a heavy shower of rain came on, too,
and Dounia, insulted and put to shame, had to drive with a peasant in an
open cart all the seventeen versts into town. Only think now what answer
could I have sent to the letter I received from you two months ago and
what could I have written? I was in despair; I dared not write to you
the truth because you would have been very unhappy, mortified and
indignant, and yet what could you do? You could only perhaps ruin
yourself, and, besides, Dounia would not allow it; and fill up my letter
with trifles when my heart was so full of sorrow, I could not. For a
whole month the town was full of gossip about this scandal, and it came
to such a pass that Dounia and I dared not even go to church on account
of the contemptuous looks, whispers, and even remarks made aloud about
us. All our acquaintances avoided us, nobody even bowed to us in the
street, and I learnt that some shopmen and clerks were intending to
insult us in a shameful way, smearing the gates of our house with pitch,
so that the landlord began to tell us we must leave. All this was set
going by Marfa Petrovna who managed to slander Dounia and throw dirt at
her in every family. She knows everyone in the neighbourhood, and that
month she was continually coming into the town, and as she is rather
talkative and fond of gossiping about her family affairs and
particularly of complaining to all and each of her husband---which is
not at all right---so in a short time she had spread her story not only
in the town, but over the whole surrounding district. It made me ill,
but Dounia bore it better than I did, and if only you could have seen
how she endured it all and tried to comfort me and cheer me up! She is
an angel! But by God's mercy, our sufferings were cut short: Mr.
Svidrigaïlov returned to his senses and repented and, probably feeling
sorry for Dounia, he laid before Marfa Petrovna a complete and
unmistakable proof of Dounia's innocence, in the form of a letter Dounia
had been forced to write and give to him, before Marfa Petrovna came
upon them in the garden. This letter, which remained in Mr.
Svidrigaïlov's hands after her departure, she had written to refuse
personal explanations and secret interviews, for which he was entreating
her. In that letter she reproached him with great heat and indignation
for the baseness of his behaviour in regard to Marfa Petrovna, reminding
him that he was the father and head of a family and telling him how
infamous it was of him to torment and make unhappy a defenceless girl,
unhappy enough already. Indeed, dear Rodya, the letter was so nobly and
touchingly written that I sobbed when I read it and to this day I cannot
read it without tears. Moreover, the evidence of the servants, too,
cleared Dounia's reputation; they had seen and known a great deal more
than Mr. Svidrigaïlov had himself supposed---as indeed is always the
case with servants. Marfa Petrovna was completely taken aback, and
`again crushed' as she said herself to us, but she was completely
convinced of Dounia's innocence. The very next day, being Sunday, she
went straight to the Cathedral, knelt down and prayed with tears to Our
Lady to give her strength to bear this new trial and to do her duty.
Then she came straight from the Cathedral to us, told us the whole
story, wept bitterly and, fully penitent, she embraced Dounia and
besought her to forgive her. The same morning without any delay, she
went round to all the houses in the town and everywhere, shedding tears,
she asserted in the most flattering terms Dounia's innocence and the
nobility of her feelings and her behavior. What was more, she showed and
read to everyone the letter in Dounia's own handwriting to Mr.
Svidrigaïlov and even allowed them to take copies of it---which I must
say I think was superfluous. In this way she was busy for several days
in driving about the whole town, because some people had taken offence
through precedence having been given to others. And therefore they had
to take turns, so that in every house she was expected before she
arrived, and everyone knew that on such and such a day Marfa Petrovna
would be reading the letter in such and such a place and people
assembled for every reading of it, even many who had heard it several
times already both in their own houses and in other people's. In my
opinion a great deal, a very great deal of all this was unnecessary; but
that's Marfa Petrovna's character. Anyway she succeeded in completely
re-establishing Dounia's reputation and the whole ignominy of this
affair rested as an indelible disgrace upon her husband, as the only
person to blame, so that I really began to feel sorry for him; it was
really treating the crazy fellow too harshly. Dounia was at once asked
to give lessons in several families, but she refused. All of a sudden
everyone began to treat her with marked respect and all this did much to
bring about the event by which, one may say, our whole fortunes are now
transformed. You must know, dear Rodya, that Dounia has a suitor and
that she has already consented to marry him. I hasten to tell you all
about the matter, and though it has been arranged without asking your
consent, I think you will not be aggrieved with me or with your sister
on that account, for you will see that we could not wait and put off our
decision till we heard from you. And you could not have judged all the
facts without being on the spot. This was how it happened. He is already
of the rank of a counsellor, Pyotr Petrovitch Luzhin, and is distantly
related to Marfa Petrovna, who has been very active in bringing the
match about. It began with his expressing through her his desire to make
our acquaintance. He was properly received, drank coffee with us and the
very next day he sent us a letter in which he very courteously made an
offer and begged for a speedy and decided answer. He is a very busy man
and is in a great hurry to get to Petersburg, so that every moment is
precious to him. At first, of course, we were greatly surprised, as it
had all happened so quickly and unexpectedly. We thought and talked it
over the whole day. He is a well-to-do man, to be depended upon, he has
two posts in the government and has already made his fortune. It is true
that he is forty-five years old, but he is of a fairly prepossessing
appearance and might still be thought attractive by women, and he is
altogether a very respectable and presentable man, only he seems a
little morose and somewhat conceited. But possibly that may only be the
impression he makes at first sight. And beware, dear Rodya, when he
comes to Petersburg, as he shortly will do, beware of judging him too
hastily and severely, as your way is, if there is anything you do not
like in him at first sight. I give you this warning, although I feel
sure that he will make a favourable impression upon you. Moreover, in
order to understand any man one must be deliberate and careful to avoid
forming prejudices and mistaken ideas, which are very difficult to
correct and get over afterwards. And Pyotr Petrovitch, judging by many
indications, is a thoroughly estimable man. At his first visit, indeed,
he told us that he was a practical man, but still he shares, as he
expressed it, many of the convictions `of our most rising generation'
and he is an opponent of all prejudices. He said a good deal more, for
he seems a little conceited and likes to be listened to, but this is
scarcely a vice. I, of course, understood very little of it, but Dounia
explained to me that, though he is not a man of great education, he is
clever and seems to be good-natured. You know your sister's character,
Rodya. She is a resolute, sensible, patient and generous girl, but she
has a passionate heart, as I know very well. Of course, there is no
great love either on his side, or on hers, but Dounia is a clever girl
and has the heart of an angel, and will make it her duty to make her
husband happy who on his side will make her happiness his care. Of that
we have no good reason to doubt, though it must be admitted the matter
has been arranged in great haste. Besides he is a man of great prudence
and he will see, to be sure, of himself, that his own happiness will be
the more secure, the happier Dounia is with him. And as for some defects
of character, for some habits and even certain differences of
opinion---which indeed are inevitable even in the happiest
marriages---Dounia has said that, as regards all that, she relies on
herself, that there is nothing to be uneasy about, and that she is ready
to put up with a great deal, if only their future relationship can be an
honourable and straightforward one. He struck me, for instance, at
first, as rather abrupt, but that may well come from his being an
outspoken man, and that is no doubt how it is. For instance, at his
second visit, after he had received Dounia's consent, in the course of
conversation, he declared that before making Dounia's acquaintance, he
had made up his mind to marry a girl of good reputation, without dowry
and, above all, one who had experienced poverty, because, as he
explained, a man ought not to be indebted to his wife, but that it is
better for a wife to look upon her husband as her benefactor. I must add
that he expressed it more nicely and politely than I have done, for I
have forgotten his actual phrases and only remember the meaning. And,
besides, it was obviously not said of design, but slipped out in the
heat of conversation, so that he tried afterwards to correct himself and
smooth it over, but all the same it did strike me as somewhat rude, and
I said so afterwards to Dounia. But Dounia was vexed, and answered that
`words are not deeds,' and that, of course, is perfectly true. Dounia
did not sleep all night before she made up her mind, and, thinking that
I was asleep, she got out of bed and was walking up and down the room
all night; at last she knelt down before the ikon and prayed long and
fervently and in the morning she told me that she had decided.

``I have mentioned already that Pyotr Petrovitch is just setting off for
Petersburg, where he has a great deal of business, and he wants to open
a legal bureau. He has been occupied for many years in conducting civil
and commercial litigation, and only the other day he won an important
case. He has to be in Petersburg because he has an important case before
the Senate. So, Rodya dear, he may be of the greatest use to you, in
every way indeed, and Dounia and I have agreed that from this very day
you could definitely enter upon your career and might consider that your
future is marked out and assured for you. Oh, if only this comes to
pass! This would be such a benefit that we could only look upon it as a
providential blessing. Dounia is dreaming of nothing else. We have even
ventured already to drop a few words on the subject to Pyotr Petrovitch.
He was cautious in his answer, and said that, of course, as he could not
get on without a secretary, it would be better to be paying a salary to
a relation than to a stranger, if only the former were fitted for the
duties (as though there could be doubt of your being fitted!) but then
he expressed doubts whether your studies at the university would leave
you time for work at his office. The matter dropped for the time, but
Dounia is thinking of nothing else now. She has been in a sort of fever
for the last few days, and has already made a regular plan for your
becoming in the end an associate and even a partner in Pyotr
Petrovitch's business, which might well be, seeing that you are a
student of law. I am in complete agreement with her, Rodya, and share
all her plans and hopes, and think there is every probability of
realising them. And in spite of Pyotr Petrovitch's evasiveness, very
natural at present (since he does not know you), Dounia is firmly
persuaded that she will gain everything by her good influence over her
future husband; this she is reckoning upon. Of course we are careful not
to talk of any of these more remote plans to Pyotr Petrovitch,
especially of your becoming his partner. He is a practical man and might
take this very coldly, it might all seem to him simply a day-dream. Nor
has either Dounia or I breathed a word to him of the great hopes we have
of his helping us to pay for your university studies; we have not spoken
of it in the first place, because it will come to pass of itself, later
on, and he will no doubt without wasting words offer to do it of
himself, (as though he could refuse Dounia that) the more readily since
you may by your own efforts become his right hand in the office, and
receive this assistance not as a charity, but as a salary earned by your
own work. Dounia wants to arrange it all like this and I quite agree
with her. And we have not spoken of our plans for another reason, that
is, because I particularly wanted you to feel on an equal footing when
you first meet him. When Dounia spoke to him with enthusiasm about you,
he answered that one could never judge of a man without seeing him
close, for oneself, and that he looked forward to forming his own
opinion when he makes your acquaintance. Do you know, my precious Rodya,
I think that perhaps for some reasons (nothing to do with Pyotr
Petrovitch though, simply for my own personal, perhaps old-womanish,
fancies) I should do better to go on living by myself, apart, than with
them, after the wedding. I am convinced that he will be generous and
delicate enough to invite me and to urge me to remain with my daughter
for the future, and if he has said nothing about it hitherto, it is
simply because it has been taken for granted; but I shall refuse. I have
noticed more than once in my life that husbands don't quite get on with
their mothers-in-law, and I don't want to be the least bit in anyone's
way, and for my own sake, too, would rather be quite independent, so
long as I have a crust of bread of my own, and such children as you and
Dounia. If possible, I would settle somewhere near you, for the most
joyful piece of news, dear Rodya, I have kept for the end of my letter:
know then, my dear boy, that we may, perhaps, be all together in a very
short time and may embrace one another again after a separation of
almost three years! It is settled \emph{for certain} that Dounia and I
are to set off for Petersburg, exactly when I don't know, but very, very
soon, possibly in a week. It all depends on Pyotr Petrovitch who will
let us know when he has had time to look round him in Petersburg. To
suit his own arrangements he is anxious to have the ceremony as soon as
possible, even before the fast of Our Lady, if it could be managed, or
if that is too soon to be ready, immediately after. Oh, with what
happiness I shall press you to my heart! Dounia is all excitement at the
joyful thought of seeing you, she said one day in joke that she would be
ready to marry Pyotr Petrovitch for that alone. She is an angel! She is
not writing anything to you now, and has only told me to write that she
has so much, so much to tell you that she is not going to take up her
pen now, for a few lines would tell you nothing, and it would only mean
upsetting herself; she bids me send you her love and innumerable kisses.
But although we shall be meeting so soon, perhaps I shall send you as
much money as I can in a day or two. Now that everyone has heard that
Dounia is to marry Pyotr Petrovitch, my credit has suddenly improved and
I know that Afanasy Ivanovitch will trust me now even to seventy-five
roubles on the security of my pension, so that perhaps I shall be able
to send you twenty-five or even thirty roubles. I would send you more,
but I am uneasy about our travelling expenses; for though Pyotr
Petrovitch has been so kind as to undertake part of the expenses of the
journey, that is to say, he has taken upon himself the conveyance of our
bags and big trunk (which will be conveyed through some acquaintances of
his), we must reckon upon some expense on our arrival in Petersburg,
where we can't be left without a halfpenny, at least for the first few
days. But we have calculated it all, Dounia and I, to the last penny,
and we see that the journey will not cost very much. It is only ninety
versts from us to the railway and we have come to an agreement with a
driver we know, so as to be in readiness; and from there Dounia and I
can travel quite comfortably third class. So that I may very likely be
able to send to you not twenty-five, but thirty roubles. But enough; I
have covered two sheets already and there is no space left for more; our
whole history, but so many events have happened! And now, my precious
Rodya, I embrace you and send you a mother's blessing till we meet. Love
Dounia your sister, Rodya; love her as she loves you and understand that
she loves you beyond everything, more than herself. She is an angel and
you, Rodya, you are everything to us---our one hope, our one
consolation. If only you are happy, we shall be happy. Do you still say
your prayers, Rodya, and believe in the mercy of our Creator and our
Redeemer? I am afraid in my heart that you may have been visited by the
new spirit of infidelity that is abroad to-day; If it is so, I pray for
you. Remember, dear boy, how in your childhood, when your father was
living, you used to lisp your prayers at my knee, and how happy we all
were in those days. Good-bye, till we meet then---I embrace you warmly,
warmly, with many kisses.

``Yours till death,

``PULCHERIA RASKOLNIKOV.''

Almost from the first, while he read the letter, Raskolnikov's face was
wet with tears; but when he finished it, his face was pale and distorted
and a bitter, wrathful and malignant smile was on his lips. He laid his
head down on his threadbare dirty pillow and pondered, pondered a long
time. His heart was beating violently, and his brain was in a turmoil.
At last he felt cramped and stifled in the little yellow room that was
like a cupboard or a box. His eyes and his mind craved for space. He
took up his hat and went out, this time without dread of meeting anyone;
he had forgotten his dread. He turned in the direction of the
Vassilyevsky Ostrov, walking along Vassilyevsky Prospect, as though
hastening on some business, but he walked, as his habit was, without
noticing his way, muttering and even speaking aloud to himself, to the
astonishment of the passers-by. Many of them took him to be drunk.

\protect\hypertarget{link2HCH0004}{}{}

\mychapter{4}{Chapter IV}\label{chapter-iv}

His mother's letter had been a torture to him, but as regards the chief
fact in it, he had felt not one moment's hesitation, even whilst he was
reading the letter. The essential question was settled, and irrevocably
settled, in his mind: ``Never such a marriage while I am alive and Mr.
Luzhin be damned!'' ``The thing is perfectly clear,'' he muttered to
himself, with a malignant smile anticipating the triumph of his
decision. ``No, mother, no, Dounia, you won't deceive me! and then they
apologise for not asking my advice and for taking the decision without
me! I dare say! They imagine it is arranged now and can't be broken off;
but we will see whether it can or not! A magnificent excuse: `Pyotr
Petrovitch is such a busy man that even his wedding has to be in
post-haste, almost by express.' No, Dounia, I see it all and I know what
you want to say to me; and I know too what you were thinking about, when
you walked up and down all night, and what your prayers were like before
the Holy Mother of Kazan who stands in mother's bedroom. Bitter is the
ascent to Golgotha\ldots{}. Hm\ldots{} so it is finally settled; you
have determined to marry a sensible business man, Avdotya Romanovna, one
who has a fortune (has \emph{already} made his fortune, that is so much
more solid and impressive) a man who holds two government posts and who
shares the ideas of our most rising generation, as mother writes, and
who \emph{seems} to be kind, as Dounia herself observes. That
\emph{seems} beats everything! And that very Dounia for that very
`\emph{seems}' is marrying him! Splendid! splendid!

``\ldots{} But I should like to know why mother has written to me about
`our most rising generation'? Simply as a descriptive touch, or with the
idea of prepossessing me in favour of Mr. Luzhin? Oh, the cunning of
them! I should like to know one thing more: how far they were open with
one another that day and night and all this time since? Was it all put
into \emph{words}, or did both understand that they had the same thing
at heart and in their minds, so that there was no need to speak of it
aloud, and better not to speak of it. Most likely it was partly like
that, from mother's letter it's evident: he struck her as rude \emph{a
little}, and mother in her simplicity took her observations to Dounia.
And she was sure to be vexed and `answered her angrily.' I should think
so! Who would not be angered when it was quite clear without any naïve
questions and when it was understood that it was useless to discuss it.
And why does she write to me, `love Dounia, Rodya, and she loves you
more than herself'? Has she a secret conscience-prick at sacrificing her
daughter to her son? `You are our one comfort, you are everything to
us.' Oh, mother!''

His bitterness grew more and more intense, and if he had happened to
meet Mr. Luzhin at the moment, he might have murdered him.

``Hm\ldots{} yes, that's true,'' he continued, pursuing the whirling
ideas that chased each other in his brain, ``it is true that `it needs
time and care to get to know a man,' but there is no mistake about Mr.
Luzhin. The chief thing is he is `a man of business and \emph{seems}
kind,' that was something, wasn't it, to send the bags and big box for
them! A kind man, no doubt after that! But his \emph{bride} and her
mother are to drive in a peasant's cart covered with sacking (I know, I
have been driven in it). No matter! It is only ninety versts and then
they can `travel very comfortably, third class,' for a thousand versts!
Quite right, too. One must cut one's coat according to one's cloth, but
what about you, Mr. Luzhin? She is your bride\ldots{}. And you must be
aware that her mother has to raise money on her pension for the journey.
To be sure it's a matter of business, a partnership for mutual benefit,
with equal shares and expenses;---food and drink provided, but pay for
your tobacco. The business man has got the better of them, too. The
luggage will cost less than their fares and very likely go for nothing.
How is it that they don't both see all that, or is it that they don't
want to see? And they are pleased, pleased! And to think that this is
only the first blossoming, and that the real fruits are to come! But
what really matters is not the stinginess, is not the meanness, but the
\emph{tone} of the whole thing. For that will be the tone after
marriage, it's a foretaste of it. And mother too, why should she be so
lavish? What will she have by the time she gets to Petersburg? Three
silver roubles or two `paper ones' as \emph{she} says\ldots{}. that old
woman\ldots{} hm. What does she expect to live upon in Petersburg
afterwards? She has her reasons already for guessing that she
\emph{could not} live with Dounia after the marriage, even for the first
few months. The good man has no doubt let slip something on that subject
also, though mother would deny it: `I shall refuse,' says she. On whom
is she reckoning then? Is she counting on what is left of her hundred
and twenty roubles of pension when Afanasy Ivanovitch's debt is paid?
She knits woollen shawls and embroiders cuffs, ruining her old eyes. And
all her shawls don't add more than twenty roubles a year to her hundred
and twenty, I know that. So she is building all her hopes all the time
on Mr. Luzhin's generosity; `he will offer it of himself, he will press
it on me.' You may wait a long time for that! That's how it always is
with these Schilleresque noble hearts; till the last moment every goose
is a swan with them, till the last moment, they hope for the best and
will see nothing wrong, and although they have an inkling of the other
side of the picture, yet they won't face the truth till they are forced
to; the very thought of it makes them shiver; they thrust the truth away
with both hands, until the man they deck out in false colours puts a
fool's cap on them with his own hands. I should like to know whether Mr.
Luzhin has any orders of merit; I bet he has the Anna in his buttonhole
and that he puts it on when he goes to dine with contractors or
merchants. He will be sure to have it for his wedding, too! Enough of
him, confound him!

``Well,\ldots{} mother I don't wonder at, it's like her, God bless her,
but how could Dounia? Dounia darling, as though I did not know you! You
were nearly twenty when I saw you last: I understood you then. Mother
writes that `Dounia can put up with a great deal.' I know that very
well. I knew that two years and a half ago, and for the last two and a
half years I have been thinking about it, thinking of just that, that
`Dounia can put up with a great deal.' If she could put up with Mr.
Svidrigaïlov and all the rest of it, she certainly can put up with a
great deal. And now mother and she have taken it into their heads that
she can put up with Mr. Luzhin, who propounds the theory of the
superiority of wives raised from destitution and owing everything to
their husband's bounty---who propounds it, too, almost at the first
interview. Granted that he `let it slip,' though he is a sensible man,
(yet maybe it was not a slip at all, but he meant to make himself clear
as soon as possible) but Dounia, Dounia? She understands the man, of
course, but she will have to live with the man. Why! she'd live on black
bread and water, she would not sell her soul, she would not barter her
moral freedom for comfort; she would not barter it for all
Schleswig-Holstein, much less Mr. Luzhin's money. No, Dounia was not
that sort when I knew her and\ldots{} she is still the same, of course!
Yes, there's no denying, the Svidrigaïlovs are a bitter pill! It's a
bitter thing to spend one's life a governess in the provinces for two
hundred roubles, but I know she would rather be a nigger on a plantation
or a Lett with a German master than degrade her soul, and her moral
dignity, by binding herself for ever to a man whom she does not respect
and with whom she has nothing in common---for her own advantage. And if
Mr. Luzhin had been of unalloyed gold, or one huge diamond, she would
never have consented to become his legal concubine. Why is she
consenting then? What's the point of it? What's the answer? It's clear
enough: for herself, for her comfort, to save her life she would not
sell herself, but for someone else she is doing it! For one she loves,
for one she adores, she will sell herself! That's what it all amounts
to; for her brother, for her mother, she will sell herself! She will
sell everything! In such cases, `we overcome our moral feeling if
necessary,' freedom, peace, conscience even, all, all are brought into
the market. Let my life go, if only my dear ones may be happy! More than
that, we become casuists, we learn to be Jesuitical and for a time maybe
we can soothe ourselves, we can persuade ourselves that it is one's duty
for a good object. That's just like us, it's as clear as daylight. It's
clear that Rodion Romanovitch Raskolnikov is the central figure in the
business, and no one else. Oh, yes, she can ensure his happiness, keep
him in the university, make him a partner in the office, make his whole
future secure; perhaps he may even be a rich man later on, prosperous,
respected, and may even end his life a famous man! But my mother? It's
all Rodya, precious Rodya, her first born! For such a son who would not
sacrifice such a daughter! Oh, loving, over-partial hearts! Why, for his
sake we would not shrink even from Sonia's fate. Sonia, Sonia
Marmeladov, the eternal victim so long as the world lasts. Have you
taken the measure of your sacrifice, both of you? Is it right? Can you
bear it? Is it any use? Is there sense in it? And let me tell you,
Dounia, Sonia's life is no worse than life with Mr. Luzhin. `There can
be no question of love,' mother writes. And what if there can be no
respect either, if on the contrary there is aversion, contempt,
repulsion, what then? So you will have to `keep up your appearance,'
too. Is not that so? Do you understand what that smartness means? Do you
understand that the Luzhin smartness is just the same thing as Sonia's
and may be worse, viler, baser, because in your case, Dounia, it's a
bargain for luxuries, after all, but with Sonia it's simply a question
of starvation. It has to be paid for, it has to be paid for, Dounia,
this smartness. And what if it's more than you can bear afterwards, if
you regret it? The bitterness, the misery, the curses, the tears hidden
from all the world, for you are not a Marfa Petrovna. And how will your
mother feel then? Even now she is uneasy, she is worried, but then, when
she sees it all clearly? And I? Yes, indeed, what have you taken me for?
I won't have your sacrifice, Dounia, I won't have it, mother! It shall
not be, so long as I am alive, it shall not, it shall not! I won't
accept it!''

He suddenly paused in his reflection and stood still.

``It shall not be? But what are you going to do to prevent it? You'll
forbid it? And what right have you? What can you promise them on your
side to give you such a right? Your whole life, your whole future, you
will devote to them \emph{when you have finished your studies and
obtained a post}? Yes, we have heard all that before, and that's all
\emph{words}, but now? Now something must be done, now, do you
understand that? And what are you doing now? You are living upon them.
They borrow on their hundred roubles pension. They borrow from the
Svidrigaïlovs. How are you going to save them from Svidrigaïlovs, from
Afanasy Ivanovitch Vahrushin, oh, future millionaire Zeus who would
arrange their lives for them? In another ten years? In another ten
years, mother will be blind with knitting shawls, maybe with weeping
too. She will be worn to a shadow with fasting; and my sister? Imagine
for a moment what may have become of your sister in ten years? What may
happen to her during those ten years? Can you fancy?''

So he tortured himself, fretting himself with such questions, and
finding a kind of enjoyment in it. And yet all these questions were not
new ones suddenly confronting him, they were old familiar aches. It was
long since they had first begun to grip and rend his heart. Long, long
ago his present anguish had its first beginnings; it had waxed and
gathered strength, it had matured and concentrated, until it had taken
the form of a fearful, frenzied and fantastic question, which tortured
his heart and mind, clamouring insistently for an answer. Now his
mother's letter had burst on him like a thunderclap. It was clear that
he must not now suffer passively, worrying himself over unsolved
questions, but that he must do something, do it at once, and do it
quickly. Anyway he must decide on something, or else\ldots{}

``Or throw up life altogether!'' he cried suddenly, in a
frenzy---``accept one's lot humbly as it is, once for all and stifle
everything in oneself, giving up all claim to activity, life and love!''

``Do you understand, sir, do you understand what it means when you have
absolutely nowhere to turn?'' Marmeladov's question came suddenly into
his mind, ``for every man must have somewhere to turn\ldots{}.''

He gave a sudden start; another thought, that he had had yesterday,
slipped back into his mind. But he did not start at the thought
recurring to him, for he knew, he had \emph{felt beforehand}, that it
must come back, he was expecting it; besides it was not only yesterday's
thought. The difference was that a month ago, yesterday even, the
thought was a mere dream: but now\ldots{} now it appeared not a dream at
all, it had taken a new menacing and quite unfamiliar shape, and he
suddenly became aware of this himself\ldots{}. He felt a hammering in
his head, and there was a darkness before his eyes.

He looked round hurriedly, he was searching for something. He wanted to
sit down and was looking for a seat; he was walking along the K------
Boulevard. There was a seat about a hundred paces in front of him. He
walked towards it as fast he could; but on the way he met with a little
adventure which absorbed all his attention. Looking for the seat, he had
noticed a woman walking some twenty paces in front of him, but at first
he took no more notice of her than of other objects that crossed his
path. It had happened to him many times going home not to notice the
road by which he was going, and he was accustomed to walk like that. But
there was at first sight something so strange about the woman in front
of him, that gradually his attention was riveted upon her, at first
reluctantly and, as it were, resentfully, and then more and more
intently. He felt a sudden desire to find out what it was that was so
strange about the woman. In the first place, she appeared to be a girl
quite young, and she was walking in the great heat bareheaded and with
no parasol or gloves, waving her arms about in an absurd way. She had on
a dress of some light silky material, but put on strangely awry, not
properly hooked up, and torn open at the top of the skirt, close to the
waist: a great piece was rent and hanging loose. A little kerchief was
flung about her bare throat, but lay slanting on one side. The girl was
walking unsteadily, too, stumbling and staggering from side to side. She
drew Raskolnikov's whole attention at last. He overtook the girl at the
seat, but, on reaching it, she dropped down on it, in the corner; she
let her head sink on the back of the seat and closed her eyes,
apparently in extreme exhaustion. Looking at her closely, he saw at once
that she was completely drunk. It was a strange and shocking sight. He
could hardly believe that he was not mistaken. He saw before him the
face of a quite young, fair-haired girl---sixteen, perhaps not more than
fifteen, years old, pretty little face, but flushed and heavy looking
and, as it were, swollen. The girl seemed hardly to know what she was
doing; she crossed one leg over the other, lifting it indecorously, and
showed every sign of being unconscious that she was in the street.

Raskolnikov did not sit down, but he felt unwilling to leave her, and
stood facing her in perplexity. This boulevard was never much
frequented; and now, at two o'clock, in the stifling heat, it was quite
deserted. And yet on the further side of the boulevard, about fifteen
paces away, a gentleman was standing on the edge of the pavement. He,
too, would apparently have liked to approach the girl with some object
of his own. He, too, had probably seen her in the distance and had
followed her, but found Raskolnikov in his way. He looked angrily at
him, though he tried to escape his notice, and stood impatiently biding
his time, till the unwelcome man in rags should have moved away. His
intentions were unmistakable. The gentleman was a plump, thickly-set
man, about thirty, fashionably dressed, with a high colour, red lips and
moustaches. Raskolnikov felt furious; he had a sudden longing to insult
this fat dandy in some way. He left the girl for a moment and walked
towards the gentleman.

``Hey! You Svidrigaïlov! What do you want here?'' he shouted, clenching
his fists and laughing, spluttering with rage.

``What do you mean?'' the gentleman asked sternly, scowling in haughty
astonishment.

``Get away, that's what I mean.''

``How dare you, you low fellow!''

He raised his cane. Raskolnikov rushed at him with his fists, without
reflecting that the stout gentleman was a match for two men like
himself. But at that instant someone seized him from behind, and a
police constable stood between them.

``That's enough, gentlemen, no fighting, please, in a public place. What
do you want? Who are you?'' he asked Raskolnikov sternly, noticing his
rags.

Raskolnikov looked at him intently. He had a straight-forward, sensible,
soldierly face, with grey moustaches and whiskers.

``You are just the man I want,'' Raskolnikov cried, catching at his arm.
``I am a student, Raskolnikov\ldots{}. You may as well know that too,''
he added, addressing the gentleman, ``come along, I have something to
show you.''

And taking the policeman by the hand he drew him towards the seat.

``Look here, hopelessly drunk, and she has just come down the boulevard.
There is no telling who and what she is, she does not look like a
professional. It's more likely she has been given drink and deceived
somewhere\ldots{} for the first time\ldots{} you understand? and they've
put her out into the street like that. Look at the way her dress is
torn, and the way it has been put on: she has been dressed by somebody,
she has not dressed herself, and dressed by unpractised hands, by a
man's hands; that's evident. And now look there: I don't know that dandy
with whom I was going to fight, I see him for the first time, but he,
too, has seen her on the road, just now, drunk, not knowing what she is
doing, and now he is very eager to get hold of her, to get her away
somewhere while she is in this state\ldots{} that's certain, believe me,
I am not wrong. I saw him myself watching her and following her, but I
prevented him, and he is just waiting for me to go away. Now he has
walked away a little, and is standing still, pretending to make a
cigarette\ldots{}. Think how can we keep her out of his hands, and how
are we to get her home?''

The policeman saw it all in a flash. The stout gentleman was easy to
understand, he turned to consider the girl. The policeman bent over to
examine her more closely, and his face worked with genuine compassion.

``Ah, what a pity!'' he said, shaking his head---``why, she is quite a
child! She has been deceived, you can see that at once. Listen, lady,''
he began addressing her, ``where do you live?'' The girl opened her
weary and sleepy-looking eyes, gazed blankly at the speaker and waved
her hand.

``Here,'' said Raskolnikov feeling in his pocket and finding twenty
copecks, ``here, call a cab and tell him to drive her to her address.
The only thing is to find out her address!''

``Missy, missy!'' the policeman began again, taking the money. ``I'll
fetch you a cab and take you home myself. Where shall I take you, eh?
Where do you live?''

``Go away! They won't let me alone,'' the girl muttered, and once more
waved her hand.

``Ach, ach, how shocking! It's shameful, missy, it's a shame!'' He shook
his head again, shocked, sympathetic and indignant.

``It's a difficult job,'' the policeman said to Raskolnikov, and as he
did so, he looked him up and down in a rapid glance. He, too, must have
seemed a strange figure to him: dressed in rags and handing him money!

``Did you meet her far from here?'' he asked him.

``I tell you she was walking in front of me, staggering, just here, in
the boulevard. She only just reached the seat and sank down on it.''

``Ah, the shameful things that are done in the world nowadays, God have
mercy on us! An innocent creature like that, drunk already! She has been
deceived, that's a sure thing. See how her dress has been torn
too\ldots{}. Ah, the vice one sees nowadays! And as likely as not she
belongs to gentlefolk too, poor ones maybe\ldots{}. There are many like
that nowadays. She looks refined, too, as though she were a lady,'' and
he bent over her once more.

Perhaps he had daughters growing up like that, ``looking like ladies and
refined'' with pretensions to gentility and smartness\ldots{}.

``The chief thing is,'' Raskolnikov persisted, ``to keep her out of this
scoundrel's hands! Why should he outrage her! It's as clear as day what
he is after; ah, the brute, he is not moving off!''

Raskolnikov spoke aloud and pointed to him. The gentleman heard him, and
seemed about to fly into a rage again, but thought better of it, and
confined himself to a contemptuous look. He then walked slowly another
ten paces away and again halted.

``Keep her out of his hands we can,'' said the constable thoughtfully,
``if only she'd tell us where to take her, but as it is\ldots{}. Missy,
hey, missy!'' he bent over her once more.

She opened her eyes fully all of a sudden, looked at him intently, as
though realising something, got up from the seat and walked away in the
direction from which she had come. ``Oh shameful wretches, they won't
let me alone!'' she said, waving her hand again. She walked quickly,
though staggering as before. The dandy followed her, but along another
avenue, keeping his eye on her.

``Don't be anxious, I won't let him have her,'' the policeman said
resolutely, and he set off after them.

``Ah, the vice one sees nowadays!'' he repeated aloud, sighing.

At that moment something seemed to sting Raskolnikov; in an instant a
complete revulsion of feeling came over him.

``Hey, here!'' he shouted after the policeman.

The latter turned round.

``Let them be! What is it to do with you? Let her go! Let him amuse
himself.'' He pointed at the dandy, ``What is it to do with you?''

The policeman was bewildered, and stared at him open-eyed. Raskolnikov
laughed.

``Well!'' ejaculated the policeman, with a gesture of contempt, and he
walked after the dandy and the girl, probably taking Raskolnikov for a
madman or something even worse.

``He has carried off my twenty copecks,'' Raskolnikov murmured angrily
when he was left alone. ``Well, let him take as much from the other
fellow to allow him to have the girl and so let it end. And why did I
want to interfere? Is it for me to help? Have I any right to help? Let
them devour each other alive---what is to me? How did I dare to give him
twenty copecks? Were they mine?''

In spite of those strange words he felt very wretched. He sat down on
the deserted seat. His thoughts strayed aimlessly\ldots{}. He found it
hard to fix his mind on anything at that moment. He longed to forget
himself altogether, to forget everything, and then to wake up and begin
life anew\ldots{}.

``Poor girl!'' he said, looking at the empty corner where she had
sat---``She will come to herself and weep, and then her mother will find
out\ldots{}. She will give her a beating, a horrible, shameful beating
and then maybe, turn her out of doors\ldots{}. And even if she does not,
the Darya Frantsovnas will get wind of it, and the girl will soon be
slipping out on the sly here and there. Then there will be the hospital
directly (that's always the luck of those girls with respectable
mothers, who go wrong on the sly) and then\ldots{} again the
hospital\ldots{} drink\ldots{} the taverns\ldots{} and more hospital, in
two or three years---a wreck, and her life over at eighteen or
nineteen\ldots{}. Have not I seen cases like that? And how have they
been brought to it? Why, they've all come to it like that. Ugh! But what
does it matter? That's as it should be, they tell us. A certain
percentage, they tell us, must every year go\ldots{} that way\ldots{} to
the devil, I suppose, so that the rest may remain chaste, and not be
interfered with. A percentage! What splendid words they have; they are
so scientific, so consolatory\ldots{}. Once you've said `percentage'
there's nothing more to worry about. If we had any other word\ldots{}
maybe we might feel more uneasy\ldots{}. But what if Dounia were one of
the percentage! Of another one if not that one?

``But where am I going?'' he thought suddenly. ``Strange, I came out for
something. As soon as I had read the letter I came out\ldots{}. I was
going to Vassilyevsky Ostrov, to Razumihin. That's what it was\ldots{}
now I remember. What for, though? And what put the idea of going to
Razumihin into my head just now? That's curious.''

He wondered at himself. Razumihin was one of his old comrades at the
university. It was remarkable that Raskolnikov had hardly any friends at
the university; he kept aloof from everyone, went to see no one, and did
not welcome anyone who came to see him, and indeed everyone soon gave
him up. He took no part in the students' gatherings, amusements or
conversations. He worked with great intensity without sparing himself,
and he was respected for this, but no one liked him. He was very poor,
and there was a sort of haughty pride and reserve about him, as though
he were keeping something to himself. He seemed to some of his comrades
to look down upon them all as children, as though he were superior in
development, knowledge and convictions, as though their beliefs and
interests were beneath him.

With Razumihin he had got on, or, at least, he was more unreserved and
communicative with him. Indeed it was impossible to be on any other
terms with Razumihin. He was an exceptionally good-humoured and candid
youth, good-natured to the point of simplicity, though both depth and
dignity lay concealed under that simplicity. The better of his comrades
understood this, and all were fond of him. He was extremely intelligent,
though he was certainly rather a simpleton at times. He was of striking
appearance---tall, thin, blackhaired and always badly shaved. He was
sometimes uproarious and was reputed to be of great physical strength.
One night, when out in a festive company, he had with one blow laid a
gigantic policeman on his back. There was no limit to his drinking
powers, but he could abstain from drink altogether; he sometimes went
too far in his pranks; but he could do without pranks altogether.
Another thing striking about Razumihin, no failure distressed him, and
it seemed as though no unfavourable circumstances could crush him. He
could lodge anywhere, and bear the extremes of cold and hunger. He was
very poor, and kept himself entirely on what he could earn by work of
one sort or another. He knew of no end of resources by which to earn
money. He spent one whole winter without lighting his stove, and used to
declare that he liked it better, because one slept more soundly in the
cold. For the present he, too, had been obliged to give up the
university, but it was only for a time, and he was working with all his
might to save enough to return to his studies again. Raskolnikov had not
been to see him for the last four months, and Razumihin did not even
know his address. About two months before, they had met in the street,
but Raskolnikov had turned away and even crossed to the other side that
he might not be observed. And though Razumihin noticed him, he passed
him by, as he did not want to annoy him.

\protect\hypertarget{link2HCH0005}{}{}

\mychapter{5}{Chapter V}\label{chapter-v}

``Of course, I've been meaning lately to go to Razumihin's to ask for
work, to ask him to get me lessons or something\ldots{}'' Raskolnikov
thought, ``but what help can he be to me now? Suppose he gets me
lessons, suppose he shares his last farthing with me, if he has any
farthings, so that I could get some boots and make myself tidy enough to
give lessons\ldots{} hm\ldots{} Well and what then? What shall I do with
the few coppers I earn? That's not what I want now. It's really absurd
for me to go to Razumihin\ldots{}.''

The question why he was now going to Razumihin agitated him even more
than he was himself aware; he kept uneasily seeking for some sinister
significance in this apparently ordinary action.

``Could I have expected to set it all straight and to find a way out by
means of Razumihin alone?'' he asked himself in perplexity.

He pondered and rubbed his forehead, and, strange to say, after long
musing, suddenly, as if it were spontaneously and by chance, a fantastic
thought came into his head.

``Hm\ldots{} to Razumihin's,'' he said all at once, calmly, as though he
had reached a final determination. ``I shall go to Razumihin's of
course, but\ldots{} not now. I shall go to him\ldots{} on the next day
after It, when It will be over and everything will begin
afresh\ldots{}.''

And suddenly he realised what he was thinking.

``After It,'' he shouted, jumping up from the seat, ``but is It really
going to happen? Is it possible it really will happen?'' He left the
seat, and went off almost at a run; he meant to turn back, homewards,
but the thought of going home suddenly filled him with intense loathing;
in that hole, in that awful little cupboard of his, all \emph{this} had
for a month past been growing up in him; and he walked on at random.

His nervous shudder had passed into a fever that made him feel
shivering; in spite of the heat he felt cold. With a kind of effort he
began almost unconsciously, from some inner craving, to stare at all the
objects before him, as though looking for something to distract his
attention; but he did not succeed, and kept dropping every moment into
brooding. When with a start he lifted his head again and looked round,
he forgot at once what he had just been thinking about and even where he
was going. In this way he walked right across Vassilyevsky Ostrov, came
out on to the Lesser Neva, crossed the bridge and turned towards the
islands. The greenness and freshness were at first restful to his weary
eyes after the dust of the town and the huge houses that hemmed him in
and weighed upon him. Here there were no taverns, no stifling closeness,
no stench. But soon these new pleasant sensations passed into morbid
irritability. Sometimes he stood still before a brightly painted summer
villa standing among green foliage, he gazed through the fence, he saw
in the distance smartly dressed women on the verandahs and balconies,
and children running in the gardens. The flowers especially caught his
attention; he gazed at them longer than at anything. He was met, too, by
luxurious carriages and by men and women on horseback; he watched them
with curious eyes and forgot about them before they had vanished from
his sight. Once he stood still and counted his money; he found he had
thirty copecks. ``Twenty to the policeman, three to Nastasya for the
letter, so I must have given forty-seven or fifty to the Marmeladovs
yesterday,'' he thought, reckoning it up for some unknown reason, but he
soon forgot with what object he had taken the money out of his pocket.
He recalled it on passing an eating-house or tavern, and felt that he
was hungry\ldots{}. Going into the tavern he drank a glass of vodka and
ate a pie of some sort. He finished eating it as he walked away. It was
a long while since he had taken vodka and it had an effect upon him at
once, though he only drank a wineglassful. His legs felt suddenly heavy
and a great drowsiness came upon him. He turned homewards, but reaching
Petrovsky Ostrov he stopped completely exhausted, turned off the road
into the bushes, sank down upon the grass and instantly fell asleep.

In a morbid condition of the brain, dreams often have a singular
actuality, vividness, and extraordinary semblance of reality. At times
monstrous images are created, but the setting and the whole picture are
so truth-like and filled with details so delicate, so unexpectedly, but
so artistically consistent, that the dreamer, were he an artist like
Pushkin or Turgenev even, could never have invented them in the waking
state. Such sick dreams always remain long in the memory and make a
powerful impression on the overwrought and deranged nervous system.

Raskolnikov had a fearful dream. He dreamt he was back in his childhood
in the little town of his birth. He was a child about seven years old,
walking into the country with his father on the evening of a holiday. It
was a grey and heavy day, the country was exactly as he remembered it;
indeed he recalled it far more vividly in his dream than he had done in
memory. The little town stood on a level flat as bare as the hand, not
even a willow near it; only in the far distance, a copse lay, a dark
blur on the very edge of the horizon. A few paces beyond the last market
garden stood a tavern, a big tavern, which had always aroused in him a
feeling of aversion, even of fear, when he walked by it with his father.
There was always a crowd there, always shouting, laughter and abuse,
hideous hoarse singing and often fighting. Drunken and horrible-looking
figures were hanging about the tavern. He used to cling close to his
father, trembling all over when he met them. Near the tavern the road
became a dusty track, the dust of which was always black. It was a
winding road, and about a hundred paces further on, it turned to the
right to the graveyard. In the middle of the graveyard stood a stone
church with a green cupola where he used to go to mass two or three
times a year with his father and mother, when a service was held in
memory of his grandmother, who had long been dead, and whom he had never
seen. On these occasions they used to take on a white dish tied up in a
table napkin a special sort of rice pudding with raisins stuck in it in
the shape of a cross. He loved that church, the old-fashioned, unadorned
ikons and the old priest with the shaking head. Near his grandmother's
grave, which was marked by a stone, was the little grave of his younger
brother who had died at six months old. He did not remember him at all,
but he had been told about his little brother, and whenever he visited
the graveyard he used religiously and reverently to cross himself and to
bow down and kiss the little grave. And now he dreamt that he was
walking with his father past the tavern on the way to the graveyard; he
was holding his father's hand and looking with dread at the tavern. A
peculiar circumstance attracted his attention: there seemed to be some
kind of festivity going on, there were crowds of gaily dressed
townspeople, peasant women, their husbands, and riff-raff of all sorts,
all singing and all more or less drunk. Near the entrance of the tavern
stood a cart, but a strange cart. It was one of those big carts usually
drawn by heavy cart-horses and laden with casks of wine or other heavy
goods. He always liked looking at those great cart-horses, with their
long manes, thick legs, and slow even pace, drawing along a perfect
mountain with no appearance of effort, as though it were easier going
with a load than without it. But now, strange to say, in the shafts of
such a cart he saw a thin little sorrel beast, one of those peasants'
nags which he had often seen straining their utmost under a heavy load
of wood or hay, especially when the wheels were stuck in the mud or in a
rut. And the peasants would beat them so cruelly, sometimes even about
the nose and eyes, and he felt so sorry, so sorry for them that he
almost cried, and his mother always used to take him away from the
window. All of a sudden there was a great uproar of shouting, singing
and the balalaïka, and from the tavern a number of big and very drunken
peasants came out, wearing red and blue shirts and coats thrown over
their shoulders.

``Get in, get in!'' shouted one of them, a young thick-necked peasant
with a fleshy face red as a carrot. ``I'll take you all, get in!''

But at once there was an outbreak of laughter and exclamations in the
crowd.

``Take us all with a beast like that!''

``Why, Mikolka, are you crazy to put a nag like that in such a cart?''

``And this mare is twenty if she is a day, mates!''

``Get in, I'll take you all,'' Mikolka shouted again, leaping first into
the cart, seizing the reins and standing straight up in front. ``The bay
has gone with Matvey,'' he shouted from the cart---``and this brute,
mates, is just breaking my heart, I feel as if I could kill her. She's
just eating her head off. Get in, I tell you! I'll make her gallop!
She'll gallop!'' and he picked up the whip, preparing himself with
relish to flog the little mare.

``Get in! Come along!'' The crowd laughed. ``D'you hear, she'll
gallop!''

``Gallop indeed! She has not had a gallop in her for the last ten
years!''

``She'll jog along!''

``Don't you mind her, mates, bring a whip each of you, get ready!''

``All right! Give it to her!''

They all clambered into Mikolka's cart, laughing and making jokes. Six
men got in and there was still room for more. They hauled in a fat,
rosy-cheeked woman. She was dressed in red cotton, in a pointed, beaded
headdress and thick leather shoes; she was cracking nuts and laughing.
The crowd round them was laughing too and indeed, how could they help
laughing? That wretched nag was to drag all the cartload of them at a
gallop! Two young fellows in the cart were just getting whips ready to
help Mikolka. With the cry of ``now,'' the mare tugged with all her
might, but far from galloping, could scarcely move forward; she
struggled with her legs, gasping and shrinking from the blows of the
three whips which were showered upon her like hail. The laughter in the
cart and in the crowd was redoubled, but Mikolka flew into a rage and
furiously thrashed the mare, as though he supposed she really could
gallop.

``Let me get in, too, mates,'' shouted a young man in the crowd whose
appetite was aroused.

``Get in, all get in,'' cried Mikolka, ``she will draw you all. I'll
beat her to death!'' And he thrashed and thrashed at the mare, beside
himself with fury.

``Father, father,'' he cried, ``father, what are they doing? Father,
they are beating the poor horse!''

``Come along, come along!'' said his father. ``They are drunken and
foolish, they are in fun; come away, don't look!'' and he tried to draw
him away, but he tore himself away from his hand, and, beside himself
with horror, ran to the horse. The poor beast was in a bad way. She was
gasping, standing still, then tugging again and almost falling.

``Beat her to death,'' cried Mikolka, ``it's come to that. I'll do for
her!''

``What are you about, are you a Christian, you devil?'' shouted an old
man in the crowd.

``Did anyone ever see the like? A wretched nag like that pulling such a
cartload,'' said another.

``You'll kill her,'' shouted the third.

``Don't meddle! It's my property, I'll do what I choose. Get in, more of
you! Get in, all of you! I will have her go at a gallop!\ldots{}''

All at once laughter broke into a roar and covered everything: the mare,
roused by the shower of blows, began feebly kicking. Even the old man
could not help smiling. To think of a wretched little beast like that
trying to kick!

Two lads in the crowd snatched up whips and ran to the mare to beat her
about the ribs. One ran each side.

``Hit her in the face, in the eyes, in the eyes,'' cried Mikolka.

``Give us a song, mates,'' shouted someone in the cart and everyone in
the cart joined in a riotous song, jingling a tambourine and whistling.
The woman went on cracking nuts and laughing.

\ldots{} He ran beside the mare, ran in front of her, saw her being
whipped across the eyes, right in the eyes! He was crying, he felt
choking, his tears were streaming. One of the men gave him a cut with
the whip across the face, he did not feel it. Wringing his hands and
screaming, he rushed up to the grey-headed old man with the grey beard,
who was shaking his head in disapproval. One woman seized him by the
hand and would have taken him away, but he tore himself from her and ran
back to the mare. She was almost at the last gasp, but began kicking
once more.

``I'll teach you to kick,'' Mikolka shouted ferociously. He threw down
the whip, bent forward and picked up from the bottom of the cart a long,
thick shaft, he took hold of one end with both hands and with an effort
brandished it over the mare.

``He'll crush her,'' was shouted round him. ``He'll kill her!''

``It's my property,'' shouted Mikolka and brought the shaft down with a
swinging blow. There was a sound of a heavy thud.

``Thrash her, thrash her! Why have you stopped?'' shouted voices in the
crowd.

And Mikolka swung the shaft a second time and it fell a second time on
the spine of the luckless mare. She sank back on her haunches, but
lurched forward and tugged forward with all her force, tugged first on
one side and then on the other, trying to move the cart. But the six
whips were attacking her in all directions, and the shaft was raised
again and fell upon her a third time, then a fourth, with heavy measured
blows. Mikolka was in a fury that he could not kill her at one blow.

``She's a tough one,'' was shouted in the crowd.

``She'll fall in a minute, mates, there will soon be an end of her,''
said an admiring spectator in the crowd.

``Fetch an axe to her! Finish her off,'' shouted a third.

``I'll show you! Stand off,'' Mikolka screamed frantically; he threw
down the shaft, stooped down in the cart and picked up an iron crowbar.
``Look out,'' he shouted, and with all his might he dealt a stunning
blow at the poor mare. The blow fell; the mare staggered, sank back,
tried to pull, but the bar fell again with a swinging blow on her back
and she fell on the ground like a log.

``Finish her off,'' shouted Mikolka and he leapt beside himself, out of
the cart. Several young men, also flushed with drink, seized anything
they could come across---whips, sticks, poles, and ran to the dying
mare. Mikolka stood on one side and began dealing random blows with the
crowbar. The mare stretched out her head, drew a long breath and died.

``You butchered her,'' someone shouted in the crowd.

``Why wouldn't she gallop then?''

``My property!'' shouted Mikolka, with bloodshot eyes, brandishing the
bar in his hands. He stood as though regretting that he had nothing more
to beat.

``No mistake about it, you are not a Christian,'' many voices were
shouting in the crowd.

But the poor boy, beside himself, made his way, screaming, through the
crowd to the sorrel nag, put his arms round her bleeding dead head and
kissed it, kissed the eyes and kissed the lips\ldots{}. Then he jumped
up and flew in a frenzy with his little fists out at Mikolka. At that
instant his father, who had been running after him, snatched him up and
carried him out of the crowd.

``Come along, come! Let us go home,'' he said to him.

``Father! Why did they\ldots{} kill\ldots{} the poor horse!'' he sobbed,
but his voice broke and the words came in shrieks from his panting
chest.

``They are drunk\ldots{}. They are brutal\ldots{} it's not our
business!'' said his father. He put his arms round his father but he
felt choked, choked. He tried to draw a breath, to cry out---and woke
up.

He waked up, gasping for breath, his hair soaked with perspiration, and
stood up in terror.

``Thank God, that was only a dream,'' he said, sitting down under a tree
and drawing deep breaths. ``But what is it? Is it some fever coming on?
Such a hideous dream!''

He felt utterly broken: darkness and confusion were in his soul. He
rested his elbows on his knees and leaned his head on his hands.

``Good God!'' he cried, ``can it be, can it be, that I shall really take
an axe, that I shall strike her on the head, split her skull
open\ldots{} that I shall tread in the sticky warm blood, break the
lock, steal and tremble; hide, all spattered in the blood\ldots{} with
the axe\ldots{}. Good God, can it be?''

He was shaking like a leaf as he said this.

``But why am I going on like this?'' he continued, sitting up again, as
it were in profound amazement. ``I knew that I could never bring myself
to it, so what have I been torturing myself for till now? Yesterday,
yesterday, when I went to make that\ldots{} \emph{experiment}, yesterday
I realised completely that I could never bear to do it\ldots{}. Why am I
going over it again, then? Why am I hesitating? As I came down the
stairs yesterday, I said myself that it was base, loathsome, vile,
vile\ldots{} the very thought of it made me feel sick and filled me with
horror.

``No, I couldn't do it, I couldn't do it! Granted, granted that there is
no flaw in all that reasoning, that all that I have concluded this last
month is clear as day, true as arithmetic\ldots{}. My God! Anyway I
couldn't bring myself to it! I couldn't do it, I couldn't do it! Why,
why then am I still\ldots{}?''

He rose to his feet, looked round in wonder as though surprised at
finding himself in this place, and went towards the bridge. He was pale,
his eyes glowed, he was exhausted in every limb, but he seemed suddenly
to breathe more easily. He felt he had cast off that fearful burden that
had so long been weighing upon him, and all at once there was a sense of
relief and peace in his soul. ``Lord,'' he prayed, ``show me my path---I
renounce that accursed\ldots{} dream of mine.''

Crossing the bridge, he gazed quietly and calmly at the Neva, at the
glowing red sun setting in the glowing sky. In spite of his weakness he
was not conscious of fatigue. It was as though an abscess that had been
forming for a month past in his heart had suddenly broken. Freedom,
freedom! He was free from that spell, that sorcery, that obsession!

Later on, when he recalled that time and all that happened to him during
those days, minute by minute, point by point, he was superstitiously
impressed by one circumstance, which, though in itself not very
exceptional, always seemed to him afterwards the predestined
turning-point of his fate. He could never understand and explain to
himself why, when he was tired and worn out, when it would have been
more convenient for him to go home by the shortest and most direct way,
he had returned by the Hay Market where he had no need to go. It was
obviously and quite unnecessarily out of his way, though not much so. It
is true that it happened to him dozens of times to return home without
noticing what streets he passed through. But why, he was always asking
himself, why had such an important, such a decisive and at the same time
such an absolutely chance meeting happened in the Hay Market (where he
had moreover no reason to go) at the very hour, the very minute of his
life when he was just in the very mood and in the very circumstances in
which that meeting was able to exert the gravest and most decisive
influence on his whole destiny? As though it had been lying in wait for
him on purpose!

It was about nine o'clock when he crossed the Hay Market. At the tables
and the barrows, at the booths and the shops, all the market people were
closing their establishments or clearing away and packing up their wares
and, like their customers, were going home. Rag pickers and
costermongers of all kinds were crowding round the taverns in the dirty
and stinking courtyards of the Hay Market. Raskolnikov particularly
liked this place and the neighbouring alleys, when he wandered aimlessly
in the streets. Here his rags did not attract contemptuous attention,
and one could walk about in any attire without scandalising people. At
the corner of an alley a huckster and his wife had two tables set out
with tapes, thread, cotton handkerchiefs, etc. They, too, had got up to
go home, but were lingering in conversation with a friend, who had just
come up to them. This friend was Lizaveta Ivanovna, or, as everyone
called her, Lizaveta, the younger sister of the old pawnbroker, Alyona
Ivanovna, whom Raskolnikov had visited the previous day to pawn his
watch and make his \emph{experiment}\ldots{}. He already knew all about
Lizaveta and she knew him a little too. She was a single woman of about
thirty-five, tall, clumsy, timid, submissive and almost idiotic. She was
a complete slave and went in fear and trembling of her sister, who made
her work day and night, and even beat her. She was standing with a
bundle before the huckster and his wife, listening earnestly and
doubtfully. They were talking of something with special warmth. The
moment Raskolnikov caught sight of her, he was overcome by a strange
sensation as it were of intense astonishment, though there was nothing
astonishing about this meeting.

``You could make up your mind for yourself, Lizaveta Ivanovna,'' the
huckster was saying aloud. ``Come round to-morrow about seven. They will
be here too.''

``To-morrow?'' said Lizaveta slowly and thoughtfully, as though unable
to make up her mind.

``Upon my word, what a fright you are in of Alyona Ivanovna,'' gabbled
the huckster's wife, a lively little woman. ``I look at you, you are
like some little babe. And she is not your own sister either---nothing
but a step-sister and what a hand she keeps over you!''

``But this time don't say a word to Alyona Ivanovna,'' her husband
interrupted; ``that's my advice, but come round to us without asking. It
will be worth your while. Later on your sister herself may have a
notion.''

``Am I to come?''

``About seven o'clock to-morrow. And they will be here. You will be able
to decide for yourself.''

``And we'll have a cup of tea,'' added his wife.

``All right, I'll come,'' said Lizaveta, still pondering, and she began
slowly moving away.

Raskolnikov had just passed and heard no more. He passed softly,
unnoticed, trying not to miss a word. His first amazement was followed
by a thrill of horror, like a shiver running down his spine. He had
learnt, he had suddenly quite unexpectedly learnt, that the next day at
seven o'clock Lizaveta, the old woman's sister and only companion, would
be away from home and that therefore at seven o'clock precisely the old
woman \emph{would be left alone}.

He was only a few steps from his lodging. He went in like a man
condemned to death. He thought of nothing and was incapable of thinking;
but he felt suddenly in his whole being that he had no more freedom of
thought, no will, and that everything was suddenly and irrevocably
decided.

Certainly, if he had to wait whole years for a suitable opportunity, he
could not reckon on a more certain step towards the success of the plan
than that which had just presented itself. In any case, it would have
been difficult to find out beforehand and with certainty, with greater
exactness and less risk, and without dangerous inquiries and
investigations, that next day at a certain time an old woman, on whose
life an attempt was contemplated, would be at home and entirely alone.

\mychapter{6}{Chapter VI}\label{chapter-vi}

Later on Raskolnikov happened to find out why the huckster and his wife
had invited Lizaveta. It was a very ordinary matter and there was
nothing exceptional about it. A family who had come to the town and been
reduced to poverty were selling their household goods and clothes, all
women's things. As the things would have fetched little in the market,
they were looking for a dealer. This was Lizaveta's business. She
undertook such jobs and was frequently employed, as she was very honest
and always fixed a fair price and stuck to it. She spoke as a rule
little and, as we have said already, she was very submissive and timid.

But Raskolnikov had become superstitious of late. The traces of
superstition remained in him long after, and were almost ineradicable.
And in all this he was always afterwards disposed to see something
strange and mysterious, as it were, the presence of some peculiar
influences and coincidences. In the previous winter a student he knew
called Pokorev, who had left for Harkov, had chanced in conversation to
give him the address of Alyona Ivanovna, the old pawnbroker, in case he
might want to pawn anything. For a long while he did not go to her, for
he had lessons and managed to get along somehow. Six weeks ago he had
remembered the address; he had two articles that could be pawned: his
father's old silver watch and a little gold ring with three red stones,
a present from his sister at parting. He decided to take the ring. When
he found the old woman he had felt an insurmountable repulsion for her
at the first glance, though he knew nothing special about her. He got
two roubles from her and went into a miserable little tavern on his way
home. He asked for tea, sat down and sank into deep thought. A strange
idea was pecking at his brain like a chicken in the egg, and very, very
much absorbed him.

Almost beside him at the next table there was sitting a student, whom he
did not know and had never seen, and with him a young officer. They had
played a game of billiards and began drinking tea. All at once he heard
the student mention to the officer the pawnbroker Alyona Ivanovna and
give him her address. This of itself seemed strange to Raskolnikov; he
had just come from her and here at once he heard her name. Of course it
was a chance, but he could not shake off a very extraordinary
impression, and here someone seemed to be speaking expressly for him;
the student began telling his friend various details about Alyona
Ivanovna.

``She is first-rate,'' he said. ``You can always get money from her. She
is as rich as a Jew, she can give you five thousand roubles at a time
and she is not above taking a pledge for a rouble. Lots of our fellows
have had dealings with her. But she is an awful old harpy\ldots{}.''

And he began describing how spiteful and uncertain she was, how if you
were only a day late with your interest the pledge was lost; how she
gave a quarter of the value of an article and took five and even seven
percent a month on it and so on. The student chattered on, saying that
she had a sister Lizaveta, whom the wretched little creature was
continually beating, and kept in complete bondage like a small child,
though Lizaveta was at least six feet high.

``There's a phenomenon for you,'' cried the student and he laughed.

They began talking about Lizaveta. The student spoke about her with a
peculiar relish and was continually laughing and the officer listened
with great interest and asked him to send Lizaveta to do some mending
for him. Raskolnikov did not miss a word and learned everything about
her. Lizaveta was younger than the old woman and was her half-sister,
being the child of a different mother. She was thirty-five. She worked
day and night for her sister, and besides doing the cooking and the
washing, she did sewing and worked as a charwoman and gave her sister
all she earned. She did not dare to accept an order or job of any kind
without her sister's permission. The old woman had already made her
will, and Lizaveta knew of it, and by this will she would not get a
farthing; nothing but the movables, chairs and so on; all the money was
left to a monastery in the province of N------, that prayers might be
said for her in perpetuity. Lizaveta was of lower rank than her sister,
unmarried and awfully uncouth in appearance, remarkably tall with long
feet that looked as if they were bent outwards. She always wore battered
goatskin shoes, and was clean in her person. What the student expressed
most surprise and amusement about was the fact that Lizaveta was
continually with child.

``But you say she is hideous?'' observed the officer.

``Yes, she is so dark-skinned and looks like a soldier dressed up, but
you know she is not at all hideous. She has such a good-natured face and
eyes. Strikingly so. And the proof of it is that lots of people are
attracted by her. She is such a soft, gentle creature, ready to put up
with anything, always willing, willing to do anything. And her smile is
really very sweet.''

``You seem to find her attractive yourself,'' laughed the officer.

``From her queerness. No, I'll tell you what. I could kill that damned
old woman and make off with her money, I assure you, without the
faintest conscience-prick,'' the student added with warmth. The officer
laughed again while Raskolnikov shuddered. How strange it was!

``Listen, I want to ask you a serious question,'' the student said
hotly. ``I was joking of course, but look here; on one side we have a
stupid, senseless, worthless, spiteful, ailing, horrid old woman, not
simply useless but doing actual mischief, who has not an idea what she
is living for herself, and who will die in a day or two in any case. You
understand? You understand?''

``Yes, yes, I understand,'' answered the officer, watching his excited
companion attentively.

``Well, listen then. On the other side, fresh young lives thrown away
for want of help and by thousands, on every side! A hundred thousand
good deeds could be done and helped, on that old woman's money which
will be buried in a monastery! Hundreds, thousands perhaps, might be set
on the right path; dozens of families saved from destitution, from ruin,
from vice, from the Lock hospitals---and all with her money. Kill her,
take her money and with the help of it devote oneself to the service of
humanity and the good of all. What do you think, would not one tiny
crime be wiped out by thousands of good deeds? For one life thousands
would be saved from corruption and decay. One death, and a hundred lives
in exchange---it's simple arithmetic! Besides, what value has the life
of that sickly, stupid, ill-natured old woman in the balance of
existence! No more than the life of a louse, of a black-beetle, less in
fact because the old woman is doing harm. She is wearing out the lives
of others; the other day she bit Lizaveta's finger out of spite; it
almost had to be amputated.''

``Of course she does not deserve to live,'' remarked the officer, ``but
there it is, it's nature.''

``Oh, well, brother, but we have to correct and direct nature, and, but
for that, we should drown in an ocean of prejudice. But for that, there
would never have been a single great man. They talk of duty,
conscience---I don't want to say anything against duty and
conscience;---but the point is, what do we mean by them? Stay, I have
another question to ask you. Listen!''

``No, you stay, I'll ask you a question. Listen!''

``Well?''

``You are talking and speechifying away, but tell me, would you kill the
old woman \emph{yourself}?''

``Of course not! I was only arguing the justice of it\ldots{}. It's
nothing to do with me\ldots{}.''

``But I think, if you would not do it yourself, there's no justice about
it\ldots{}. Let us have another game.''

Raskolnikov was violently agitated. Of course, it was all ordinary
youthful talk and thought, such as he had often heard before in
different forms and on different themes. But why had he happened to hear
such a discussion and such ideas at the very moment when his own brain
was just conceiving\ldots{} \emph{the very same ideas}? And why, just at
the moment when he had brought away the embryo of his idea from the old
woman had he dropped at once upon a conversation about her? This
coincidence always seemed strange to him. This trivial talk in a tavern
had an immense influence on him in his later action; as though there had
really been in it something preordained, some guiding hint\ldots{}.

\begin{center}\rule{0.5\linewidth}{\linethickness}\end{center}

On returning from the Hay Market he flung himself on the sofa and sat
for a whole hour without stirring. Meanwhile it got dark; he had no
candle and, indeed, it did not occur to him to light up. He could never
recollect whether he had been thinking about anything at that time. At
last he was conscious of his former fever and shivering, and he realised
with relief that he could lie down on the sofa. Soon heavy, leaden sleep
came over him, as it were crushing him.

He slept an extraordinarily long time and without dreaming. Nastasya,
coming into his room at ten o'clock the next morning, had difficulty in
rousing him. She brought him in tea and bread. The tea was again the
second brew and again in her own tea-pot.

``My goodness, how he sleeps!'' she cried indignantly. ``And he is
always asleep.''

He got up with an effort. His head ached, he stood up, took a turn in
his garret and sank back on the sofa again.

``Going to sleep again,'' cried Nastasya. ``Are you ill, eh?''

He made no reply.

``Do you want some tea?''

``Afterwards,'' he said with an effort, closing his eyes again and
turning to the wall.

Nastasya stood over him.

``Perhaps he really is ill,'' she said, turned and went out. She came in
again at two o'clock with soup. He was lying as before. The tea stood
untouched. Nastasya felt positively offended and began wrathfully
rousing him.

``Why are you lying like a log?'' she shouted, looking at him with
repulsion.

He got up, and sat down again, but said nothing and stared at the floor.

``Are you ill or not?'' asked Nastasya and again received no answer.
``You'd better go out and get a breath of air,'' she said after a pause.
``Will you eat it or not?''

``Afterwards,'' he said weakly. ``You can go.''

And he motioned her out.

She remained a little longer, looked at him with compassion and went
out.

A few minutes afterwards, he raised his eyes and looked for a long while
at the tea and the soup. Then he took the bread, took up a spoon and
began to eat.

He ate a little, three or four spoonfuls, without appetite, as it were
mechanically. His head ached less. After his meal he stretched himself
on the sofa again, but now he could not sleep; he lay without stirring,
with his face in the pillow. He was haunted by day-dreams and such
strange day-dreams; in one, that kept recurring, he fancied that he was
in Africa, in Egypt, in some sort of oasis. The caravan was resting, the
camels were peacefully lying down; the palms stood all around in a
complete circle; all the party were at dinner. But he was drinking water
from a spring which flowed gurgling close by. And it was so cool, it was
wonderful, wonderful, blue, cold water running among the parti-coloured
stones and over the clean sand which glistened here and there like
gold\ldots{}. Suddenly he heard a clock strike. He started, roused
himself, raised his head, looked out of the window, and seeing how late
it was, suddenly jumped up wide awake as though someone had pulled him
off the sofa. He crept on tiptoe to the door, stealthily opened it and
began listening on the staircase. His heart beat terribly. But all was
quiet on the stairs as if everyone was asleep\ldots{}. It seemed to him
strange and monstrous that he could have slept in such forgetfulness
from the previous day and had done nothing, had prepared nothing
yet\ldots{}. And meanwhile perhaps it had struck six. And his drowsiness
and stupefaction were followed by an extraordinary, feverish, as it were
distracted haste. But the preparations to be made were few. He
concentrated all his energies on thinking of everything and forgetting
nothing; and his heart kept beating and thumping so that he could hardly
breathe. First he had to make a noose and sew it into his overcoat---a
work of a moment. He rummaged under his pillow and picked out amongst
the linen stuffed away under it, a worn out, old unwashed shirt. From
its rags he tore a long strip, a couple of inches wide and about sixteen
inches long. He folded this strip in two, took off his wide, strong
summer overcoat of some stout cotton material (his only outer garment)
and began sewing the two ends of the rag on the inside, under the left
armhole. His hands shook as he sewed, but he did it successfully so that
nothing showed outside when he put the coat on again. The needle and
thread he had got ready long before and they lay on his table in a piece
of paper. As for the noose, it was a very ingenious device of his own;
the noose was intended for the axe. It was impossible for him to carry
the axe through the street in his hands. And if hidden under his coat he
would still have had to support it with his hand, which would have been
noticeable. Now he had only to put the head of the axe in the noose, and
it would hang quietly under his arm on the inside. Putting his hand in
his coat pocket, he could hold the end of the handle all the way, so
that it did not swing; and as the coat was very full, a regular sack in
fact, it could not be seen from outside that he was holding something
with the hand that was in the pocket. This noose, too, he had designed a
fortnight before.

When he had finished with this, he thrust his hand into a little opening
between his sofa and the floor, fumbled in the left corner and drew out
the \emph{pledge}, which he had got ready long before and hidden there.
This pledge was, however, only a smoothly planed piece of wood the size
and thickness of a silver cigarette case. He picked up this piece of
wood in one of his wanderings in a courtyard where there was some sort
of a workshop. Afterwards he had added to the wood a thin smooth piece
of iron, which he had also picked up at the same time in the street.
Putting the iron which was a little the smaller on the piece of wood, he
fastened them very firmly, crossing and re-crossing the thread round
them; then wrapped them carefully and daintily in clean white paper and
tied up the parcel so that it would be very difficult to untie it. This
was in order to divert the attention of the old woman for a time, while
she was trying to undo the knot, and so to gain a moment. The iron strip
was added to give weight, so that the woman might not guess the first
minute that the ``thing'' was made of wood. All this had been stored by
him beforehand under the sofa. He had only just got the pledge out when
he heard someone suddenly about in the yard.

``It struck six long ago.''

``Long ago! My God!''

He rushed to the door, listened, caught up his hat and began to descend
his thirteen steps cautiously, noiselessly, like a cat. He had still the
most important thing to do---to steal the axe from the kitchen. That the
deed must be done with an axe he had decided long ago. He had also a
pocket pruning-knife, but he could not rely on the knife and still less
on his own strength, and so resolved finally on the axe. We may note in
passing, one peculiarity in regard to all the final resolutions taken by
him in the matter; they had one strange characteristic: the more final
they were, the more hideous and the more absurd they at once became in
his eyes. In spite of all his agonising inward struggle, he never for a
single instant all that time could believe in the carrying out of his
plans.

And, indeed, if it had ever happened that everything to the least point
could have been considered and finally settled, and no uncertainty of
any kind had remained, he would, it seems, have renounced it all as
something absurd, monstrous and impossible. But a whole mass of
unsettled points and uncertainties remained. As for getting the axe,
that trifling business cost him no anxiety, for nothing could be easier.
Nastasya was continually out of the house, especially in the evenings;
she would run in to the neighbours or to a shop, and always left the
door ajar. It was the one thing the landlady was always scolding her
about. And so, when the time came, he would only have to go quietly into
the kitchen and to take the axe, and an hour later (when everything was
over) go in and put it back again. But these were doubtful points.
Supposing he returned an hour later to put it back, and Nastasya had
come back and was on the spot. He would of course have to go by and wait
till she went out again. But supposing she were in the meantime to miss
the axe, look for it, make an outcry---that would mean suspicion or at
least grounds for suspicion.

But those were all trifles which he had not even begun to consider, and
indeed he had no time. He was thinking of the chief point, and put off
trifling details, until \emph{he could believe in it all}. But that
seemed utterly unattainable. So it seemed to himself at least. He could
not imagine, for instance, that he would sometime leave off thinking,
get up and simply go there\ldots{}. Even his late experiment (i.e. his
visit with the object of a final survey of the place) was simply an
attempt at an experiment, far from being the real thing, as though one
should say ``come, let us go and try it---why dream about it!''---and at
once he had broken down and had run away cursing, in a frenzy with
himself. Meanwhile it would seem, as regards the moral question, that
his analysis was complete; his casuistry had become keen as a razor, and
he could not find rational objections in himself. But in the last resort
he simply ceased to believe in himself, and doggedly, slavishly sought
arguments in all directions, fumbling for them, as though someone were
forcing and drawing him to it.

At first---long before indeed---he had been much occupied with one
question; why almost all crimes are so badly concealed and so easily
detected, and why almost all criminals leave such obvious traces? He had
come gradually to many different and curious conclusions, and in his
opinion the chief reason lay not so much in the material impossibility
of concealing the crime, as in the criminal himself. Almost every
criminal is subject to a failure of will and reasoning power by a
childish and phenomenal heedlessness, at the very instant when prudence
and caution are most essential. It was his conviction that this eclipse
of reason and failure of will power attacked a man like a disease,
developed gradually and reached its highest point just before the
perpetration of the crime, continued with equal violence at the moment
of the crime and for longer or shorter time after, according to the
individual case, and then passed off like any other disease. The
question whether the disease gives rise to the crime, or whether the
crime from its own peculiar nature is always accompanied by something of
the nature of disease, he did not yet feel able to decide.

When he reached these conclusions, he decided that in his own case there
could not be such a morbid reaction, that his reason and will would
remain unimpaired at the time of carrying out his design, for the simple
reason that his design was ``not a crime\ldots{}.'' We will omit all the
process by means of which he arrived at this last conclusion; we have
run too far ahead already\ldots{}. We may add only that the practical,
purely material difficulties of the affair occupied a secondary position
in his mind. ``One has but to keep all one's will-power and reason to
deal with them, and they will all be overcome at the time when once one
has familiarised oneself with the minutest details of the
business\ldots{}.'' But this preparation had never been begun. His final
decisions were what he came to trust least, and when the hour struck, it
all came to pass quite differently, as it were accidentally and
unexpectedly.

One trifling circumstance upset his calculations, before he had even
left the staircase. When he reached the landlady's kitchen, the door of
which was open as usual, he glanced cautiously in to see whether, in
Nastasya's absence, the landlady herself was there, or if not, whether
the door to her own room was closed, so that she might not peep out when
he went in for the axe. But what was his amazement when he suddenly saw
that Nastasya was not only at home in the kitchen, but was occupied
there, taking linen out of a basket and hanging it on a line. Seeing
him, she left off hanging the clothes, turned to him and stared at him
all the time he was passing. He turned away his eyes, and walked past as
though he noticed nothing. But it was the end of everything; he had not
the axe! He was overwhelmed.

``What made me think,'' he reflected, as he went under the gateway,
``what made me think that she would be sure not to be at home at that
moment! Why, why, why did I assume this so certainly?''

He was crushed and even humiliated. He could have laughed at himself in
his anger\ldots{}. A dull animal rage boiled within him.

He stood hesitating in the gateway. To go into the street, to go a walk
for appearance' sake was revolting; to go back to his room, even more
revolting. ``And what a chance I have lost for ever!'' he muttered,
standing aimlessly in the gateway, just opposite the porter's little
dark room, which was also open. Suddenly he started. From the porter's
room, two paces away from him, something shining under the bench to the
right caught his eye\ldots{}. He looked about him---nobody. He
approached the room on tiptoe, went down two steps into it and in a
faint voice called the porter. ``Yes, not at home! Somewhere near
though, in the yard, for the door is wide open.'' He dashed to the axe
(it was an axe) and pulled it out from under the bench, where it lay
between two chunks of wood; at once, before going out, he made it fast
in the noose, he thrust both hands into his pockets and went out of the
room; no one had noticed him! ``When reason fails, the devil helps!'' he
thought with a strange grin. This chance raised his spirits
extraordinarily.

He walked along quietly and sedately, without hurry, to avoid awakening
suspicion. He scarcely looked at the passers-by, tried to escape looking
at their faces at all, and to be as little noticeable as possible.
Suddenly he thought of his hat. ``Good heavens! I had the money the day
before yesterday and did not get a cap to wear instead!'' A curse rose
from the bottom of his soul.

Glancing out of the corner of his eye into a shop, he saw by a clock on
the wall that it was ten minutes past seven. He had to make haste and at
the same time to go someway round, so as to approach the house from the
other side\ldots{}.

When he had happened to imagine all this beforehand, he had sometimes
thought that he would be very much afraid. But he was not very much
afraid now, was not afraid at all, indeed. His mind was even occupied by
irrelevant matters, but by nothing for long. As he passed the Yusupov
garden, he was deeply absorbed in considering the building of great
fountains, and of their refreshing effect on the atmosphere in all the
squares. By degrees he passed to the conviction that if the summer
garden were extended to the field of Mars, and perhaps joined to the
garden of the Mihailovsky Palace, it would be a splendid thing and a
great benefit to the town. Then he was interested by the question why in
all great towns men are not simply driven by necessity, but in some
peculiar way inclined to live in those parts of the town where there are
no gardens nor fountains; where there is most dirt and smell and all
sorts of nastiness. Then his own walks through the Hay Market came back
to his mind, and for a moment he waked up to reality. ``What nonsense!''
he thought, ``better think of nothing at all!''

``So probably men led to execution clutch mentally at every object that
meets them on the way,'' flashed through his mind, but simply flashed,
like lightning; he made haste to dismiss this thought\ldots{}. And by
now he was near; here was the house, here was the gate. Suddenly a clock
somewhere struck once. ``What! can it be half-past seven? Impossible, it
must be fast!''

Luckily for him, everything went well again at the gates. At that very
moment, as though expressly for his benefit, a huge waggon of hay had
just driven in at the gate, completely screening him as he passed under
the gateway, and the waggon had scarcely had time to drive through into
the yard, before he had slipped in a flash to the right. On the other
side of the waggon he could hear shouting and quarrelling; but no one
noticed him and no one met him. Many windows looking into that huge
quadrangular yard were open at that moment, but he did not raise his
head---he had not the strength to. The staircase leading to the old
woman's room was close by, just on the right of the gateway. He was
already on the stairs\ldots{}.

Drawing a breath, pressing his hand against his throbbing heart, and
once more feeling for the axe and setting it straight, he began softly
and cautiously ascending the stairs, listening every minute. But the
stairs, too, were quite deserted; all the doors were shut; he met no
one. One flat indeed on the first floor was wide open and painters were
at work in it, but they did not glance at him. He stood still, thought a
minute and went on. ``Of course it would be better if they had not been
here, but\ldots{} it's two storeys above them.''

And there was the fourth storey, here was the door, here was the flat
opposite, the empty one. The flat underneath the old woman's was
apparently empty also; the visiting card nailed on the door had been
torn off---they had gone away!\ldots{} He was out of breath. For one
instant the thought floated through his mind ``Shall I go back?'' But he
made no answer and began listening at the old woman's door, a dead
silence. Then he listened again on the staircase, listened long and
intently\ldots{} then looked about him for the last time, pulled himself
together, drew himself up, and once more tried the axe in the noose.
``Am I very pale?'' he wondered. ``Am I not evidently agitated? She is
mistrustful\ldots{}. Had I better wait a little longer\ldots{} till my
heart leaves off thumping?''

But his heart did not leave off. On the contrary, as though to spite
him, it throbbed more and more violently. He could stand it no longer,
he slowly put out his hand to the bell and rang. Half a minute later he
rang again, more loudly.

No answer. To go on ringing was useless and out of place. The old woman
was, of course, at home, but she was suspicious and alone. He had some
knowledge of her habits\ldots{} and once more he put his ear to the
door. Either his senses were peculiarly keen (which it is difficult to
suppose), or the sound was really very distinct. Anyway, he suddenly
heard something like the cautious touch of a hand on the lock and the
rustle of a skirt at the very door. Someone was standing stealthily
close to the lock and just as he was doing on the outside was secretly
listening within, and seemed to have her ear to the door\ldots{}. He
moved a little on purpose and muttered something aloud that he might not
have the appearance of hiding, then rang a third time, but quietly,
soberly, and without impatience, Recalling it afterwards, that moment
stood out in his mind vividly, distinctly, for ever; he could not make
out how he had had such cunning, for his mind was as it were clouded at
moments and he was almost unconscious of his body\ldots{}. An instant
later he heard the latch unfastened.

\mychapter{7}{Chapter VII}\label{chapter-vii}

The door was as before opened a tiny crack, and again two sharp and
suspicious eyes stared at him out of the darkness. Then Raskolnikov lost
his head and nearly made a great mistake.

Fearing the old woman would be frightened by their being alone, and not
hoping that the sight of him would disarm her suspicions, he took hold
of the door and drew it towards him to prevent the old woman from
attempting to shut it again. Seeing this she did not pull the door back,
but she did not let go the handle so that he almost dragged her out with
it on to the stairs. Seeing that she was standing in the doorway not
allowing him to pass, he advanced straight upon her. She stepped back in
alarm, tried to say something, but seemed unable to speak and stared
with open eyes at him.

``Good evening, Alyona Ivanovna,'' he began, trying to speak easily, but
his voice would not obey him, it broke and shook. ``I have come\ldots{}
I have brought something\ldots{} but we'd better come in\ldots{} to the
light\ldots{}.''

And leaving her, he passed straight into the room uninvited. The old
woman ran after him; her tongue was unloosed.

``Good heavens! What it is? Who is it? What do you want?''

``Why, Alyona Ivanovna, you know me\ldots{} Raskolnikov\ldots{} here, I
brought you the pledge I promised the other day\ldots{}'' And he held
out the pledge.

The old woman glanced for a moment at the pledge, but at once stared in
the eyes of her uninvited visitor. She looked intently, maliciously and
mistrustfully. A minute passed; he even fancied something like a sneer
in her eyes, as though she had already guessed everything. He felt that
he was losing his head, that he was almost frightened, so frightened
that if she were to look like that and not say a word for another half
minute, he thought he would have run away from her.

``Why do you look at me as though you did not know me?'' he said
suddenly, also with malice. ``Take it if you like, if not I'll go
elsewhere, I am in a hurry.''

He had not even thought of saying this, but it was suddenly said of
itself. The old woman recovered herself, and her visitor's resolute tone
evidently restored her confidence.

``But why, my good sir, all of a minute\ldots{}. What is it?'' she
asked, looking at the pledge.

``The silver cigarette case; I spoke of it last time, you know.''

She held out her hand.

``But how pale you are, to be sure\ldots{} and your hands are trembling
too? Have you been bathing, or what?''

``Fever,'' he answered abruptly. ``You can't help getting pale\ldots{}
if you've nothing to eat,'' he added, with difficulty articulating the
words.

His strength was failing him again. But his answer sounded like the
truth; the old woman took the pledge.

``What is it?'' she asked once more, scanning Raskolnikov intently, and
weighing the pledge in her hand.

``A thing\ldots{} cigarette case\ldots{}. Silver\ldots{}. Look at it.''

``It does not seem somehow like silver\ldots{}. How he has wrapped it
up!''

Trying to untie the string and turning to the window, to the light (all
her windows were shut, in spite of the stifling heat), she left him
altogether for some seconds and stood with her back to him. He
unbuttoned his coat and freed the axe from the noose, but did not yet
take it out altogether, simply holding it in his right hand under the
coat. His hands were fearfully weak, he felt them every moment growing
more numb and more wooden. He was afraid he would let the axe slip and
fall\ldots{}. A sudden giddiness came over him.

``But what has he tied it up like this for?'' the old woman cried with
vexation and moved towards him.

He had not a minute more to lose. He pulled the axe quite out, swung it
with both arms, scarcely conscious of himself, and almost without
effort, almost mechanically, brought the blunt side down on her head. He
seemed not to use his own strength in this. But as soon as he had once
brought the axe down, his strength returned to him.

The old woman was as always bareheaded. Her thin, light hair, streaked
with grey, thickly smeared with grease, was plaited in a rat's tail and
fastened by a broken horn comb which stood out on the nape of her neck.
As she was so short, the blow fell on the very top of her skull. She
cried out, but very faintly, and suddenly sank all of a heap on the
floor, raising her hands to her head. In one hand she still held ``the
pledge.'' Then he dealt her another and another blow with the blunt side
and on the same spot. The blood gushed as from an overturned glass, the
body fell back. He stepped back, let it fall, and at once bent over her
face; she was dead. Her eyes seemed to be starting out of their sockets,
the brow and the whole face were drawn and contorted convulsively.

He laid the axe on the ground near the dead body and felt at once in her
pocket (trying to avoid the streaming body)---the same right-hand pocket
from which she had taken the key on his last visit. He was in full
possession of his faculties, free from confusion or giddiness, but his
hands were still trembling. He remembered afterwards that he had been
particularly collected and careful, trying all the time not to get
smeared with blood\ldots{}. He pulled out the keys at once, they were
all, as before, in one bunch on a steel ring. He ran at once into the
bedroom with them. It was a very small room with a whole shrine of holy
images. Against the other wall stood a big bed, very clean and covered
with a silk patchwork wadded quilt. Against a third wall was a chest of
drawers. Strange to say, so soon as he began to fit the keys into the
chest, so soon as he heard their jingling, a convulsive shudder passed
over him. He suddenly felt tempted again to give it all up and go away.
But that was only for an instant; it was too late to go back. He
positively smiled at himself, when suddenly another terrifying idea
occurred to his mind. He suddenly fancied that the old woman might be
still alive and might recover her senses. Leaving the keys in the chest,
he ran back to the body, snatched up the axe and lifted it once more
over the old woman, but did not bring it down. There was no doubt that
she was dead. Bending down and examining her again more closely, he saw
clearly that the skull was broken and even battered in on one side. He
was about to feel it with his finger, but drew back his hand and indeed
it was evident without that. Meanwhile there was a perfect pool of
blood. All at once he noticed a string on her neck; he tugged at it, but
the string was strong and did not snap and besides, it was soaked with
blood. He tried to pull it out from the front of the dress, but
something held it and prevented its coming. In his impatience he raised
the axe again to cut the string from above on the body, but did not
dare, and with difficulty, smearing his hand and the axe in the blood,
after two minutes' hurried effort, he cut the string and took it off
without touching the body with the axe; he was not mistaken---it was a
purse. On the string were two crosses, one of Cyprus wood and one of
copper, and an image in silver filigree, and with them a small greasy
chamois leather purse with a steel rim and ring. The purse was stuffed
very full; Raskolnikov thrust it in his pocket without looking at it,
flung the crosses on the old woman's body and rushed back into the
bedroom, this time taking the axe with him.

He was in terrible haste, he snatched the keys, and began trying them
again. But he was unsuccessful. They would not fit in the locks. It was
not so much that his hands were shaking, but that he kept making
mistakes; though he saw for instance that a key was not the right one
and would not fit, still he tried to put it in. Suddenly he remembered
and realised that the big key with the deep notches, which was hanging
there with the small keys could not possibly belong to the chest of
drawers (on his last visit this had struck him), but to some strong box,
and that everything perhaps was hidden in that box. He left the chest of
drawers, and at once felt under the bedstead, knowing that old women
usually keep boxes under their beds. And so it was; there was a
good-sized box under the bed, at least a yard in length, with an arched
lid covered with red leather and studded with steel nails. The notched
key fitted at once and unlocked it. At the top, under a white sheet, was
a coat of red brocade lined with hareskin; under it was a silk dress,
then a shawl and it seemed as though there was nothing below but
clothes. The first thing he did was to wipe his blood-stained hands on
the red brocade. ``It's red, and on red blood will be less noticeable,''
the thought passed through his mind; then he suddenly came to himself.
``Good God, am I going out of my senses?'' he thought with terror.

But no sooner did he touch the clothes than a gold watch slipped from
under the fur coat. He made haste to turn them all over. There turned
out to be various articles made of gold among the clothes---probably all
pledges, unredeemed or waiting to be redeemed---bracelets, chains,
ear-rings, pins and such things. Some were in cases, others simply
wrapped in newspaper, carefully and exactly folded, and tied round with
tape. Without any delay, he began filling up the pockets of his trousers
and overcoat without examining or undoing the parcels and cases; but he
had not time to take many\ldots{}.

He suddenly heard steps in the room where the old woman lay. He stopped
short and was still as death. But all was quiet, so it must have been
his fancy. All at once he heard distinctly a faint cry, as though
someone had uttered a low broken moan. Then again dead silence for a
minute or two. He sat squatting on his heels by the box and waited
holding his breath. Suddenly he jumped up, seized the axe and ran out of
the bedroom.

In the middle of the room stood Lizaveta with a big bundle in her arms.
She was gazing in stupefaction at her murdered sister, white as a sheet
and seeming not to have the strength to cry out. Seeing him run out of
the bedroom, she began faintly quivering all over, like a leaf, a
shudder ran down her face; she lifted her hand, opened her mouth, but
still did not scream. She began slowly backing away from him into the
corner, staring intently, persistently at him, but still uttered no
sound, as though she could not get breath to scream. He rushed at her
with the axe; her mouth twitched piteously, as one sees babies' mouths,
when they begin to be frightened, stare intently at what frightens them
and are on the point of screaming. And this hapless Lizaveta was so
simple and had been so thoroughly crushed and scared that she did not
even raise a hand to guard her face, though that was the most necessary
and natural action at the moment, for the axe was raised over her face.
She only put up her empty left hand, but not to her face, slowly holding
it out before her as though motioning him away. The axe fell with the
sharp edge just on the skull and split at one blow all the top of the
head. She fell heavily at once. Raskolnikov completely lost his head,
snatching up her bundle, dropped it again and ran into the entry.

Fear gained more and more mastery over him, especially after this
second, quite unexpected murder. He longed to run away from the place as
fast as possible. And if at that moment he had been capable of seeing
and reasoning more correctly, if he had been able to realise all the
difficulties of his position, the hopelessness, the hideousness and the
absurdity of it, if he could have understood how many obstacles and,
perhaps, crimes he had still to overcome or to commit, to get out of
that place and to make his way home, it is very possible that he would
have flung up everything, and would have gone to give himself up, and
not from fear, but from simple horror and loathing of what he had done.
The feeling of loathing especially surged up within him and grew
stronger every minute. He would not now have gone to the box or even
into the room for anything in the world.

But a sort of blankness, even dreaminess, had begun by degrees to take
possession of him; at moments he forgot himself, or rather, forgot what
was of importance, and caught at trifles. Glancing, however, into the
kitchen and seeing a bucket half full of water on a bench, he bethought
him of washing his hands and the axe. His hands were sticky with blood.
He dropped the axe with the blade in the water, snatched a piece of soap
that lay in a broken saucer on the window, and began washing his hands
in the bucket. When they were clean, he took out the axe, washed the
blade and spent a long time, about three minutes, washing the wood where
there were spots of blood rubbing them with soap. Then he wiped it all
with some linen that was hanging to dry on a line in the kitchen and
then he was a long while attentively examining the axe at the window.
There was no trace left on it, only the wood was still damp. He
carefully hung the axe in the noose under his coat. Then as far as was
possible, in the dim light in the kitchen, he looked over his overcoat,
his trousers and his boots. At the first glance there seemed to be
nothing but stains on the boots. He wetted the rag and rubbed the boots.
But he knew he was not looking thoroughly, that there might be something
quite noticeable that he was overlooking. He stood in the middle of the
room, lost in thought. Dark agonising ideas rose in his mind---the idea
that he was mad and that at that moment he was incapable of reasoning,
of protecting himself, that he ought perhaps to be doing something
utterly different from what he was now doing. ``Good God!'' he muttered
``I must fly, fly,'' and he rushed into the entry. But here a shock of
terror awaited him such as he had never known before.

He stood and gazed and could not believe his eyes: the door, the outer
door from the stairs, at which he had not long before waited and rung,
was standing unfastened and at least six inches open. No lock, no bolt,
all the time, all that time! The old woman had not shut it after him
perhaps as a precaution. But, good God! Why, he had seen Lizaveta
afterwards! And how could he, how could he have failed to reflect that
she must have come in somehow! She could not have come through the wall!

He dashed to the door and fastened the latch.

``But no, the wrong thing again! I must get away, get away\ldots{}.''

He unfastened the latch, opened the door and began listening on the
staircase.

He listened a long time. Somewhere far away, it might be in the gateway,
two voices were loudly and shrilly shouting, quarrelling and scolding.
``What are they about?'' He waited patiently. At last all was still, as
though suddenly cut off; they had separated. He was meaning to go out,
but suddenly, on the floor below, a door was noisily opened and someone
began going downstairs humming a tune. ``How is it they all make such a
noise?'' flashed through his mind. Once more he closed the door and
waited. At last all was still, not a soul stirring. He was just taking a
step towards the stairs when he heard fresh footsteps.

The steps sounded very far off, at the very bottom of the stairs, but he
remembered quite clearly and distinctly that from the first sound he
began for some reason to suspect that this was someone coming
\emph{there}, to the fourth floor, to the old woman. Why? Were the
sounds somehow peculiar, significant? The steps were heavy, even and
unhurried. Now \emph{he} had passed the first floor, now he was mounting
higher, it was growing more and more distinct! He could hear his heavy
breathing. And now the third storey had been reached. Coming here! And
it seemed to him all at once that he was turned to stone, that it was
like a dream in which one is being pursued, nearly caught and will be
killed, and is rooted to the spot and cannot even move one's arms.

At last when the unknown was mounting to the fourth floor, he suddenly
started, and succeeded in slipping neatly and quickly back into the flat
and closing the door behind him. Then he took the hook and softly,
noiselessly, fixed it in the catch. Instinct helped him. When he had
done this, he crouched holding his breath, by the door. The unknown
visitor was by now also at the door. They were now standing opposite one
another, as he had just before been standing with the old woman, when
the door divided them and he was listening.

The visitor panted several times. ``He must be a big, fat man,'' thought
Raskolnikov, squeezing the axe in his hand. It seemed like a dream
indeed. The visitor took hold of the bell and rang it loudly.

As soon as the tin bell tinkled, Raskolnikov seemed to be aware of
something moving in the room. For some seconds he listened quite
seriously. The unknown rang again, waited and suddenly tugged violently
and impatiently at the handle of the door. Raskolnikov gazed in horror
at the hook shaking in its fastening, and in blank terror expected every
minute that the fastening would be pulled out. It certainly did seem
possible, so violently was he shaking it. He was tempted to hold the
fastening, but \emph{he} might be aware of it. A giddiness came over him
again. ``I shall fall down!'' flashed through his mind, but the unknown
began to speak and he recovered himself at once.

``What's up? Are they asleep or murdered? D-damn them!'' he bawled in a
thick voice, ``Hey, Alyona Ivanovna, old witch! Lizaveta Ivanovna, hey,
my beauty! open the door! Oh, damn them! Are they asleep or what?''

And again, enraged, he tugged with all his might a dozen times at the
bell. He must certainly be a man of authority and an intimate
acquaintance.

At this moment light hurried steps were heard not far off, on the
stairs. Someone else was approaching. Raskolnikov had not heard them at
first.

``You don't say there's no one at home,'' the new-comer cried in a
cheerful, ringing voice, addressing the first visitor, who still went on
pulling the bell. ``Good evening, Koch.''

``From his voice he must be quite young,'' thought Raskolnikov.

``Who the devil can tell? I've almost broken the lock,'' answered Koch.
``But how do you come to know me?''

``Why! The day before yesterday I beat you three times running at
billiards at Gambrinus'.''

``Oh!''

``So they are not at home? That's queer. It's awfully stupid though.
Where could the old woman have gone? I've come on business.''

``Yes; and I have business with her, too.''

``Well, what can we do? Go back, I suppose, Aie---aie! And I was hoping
to get some money!'' cried the young man.

``We must give it up, of course, but what did she fix this time for? The
old witch fixed the time for me to come herself. It's out of my way. And
where the devil she can have got to, I can't make out. She sits here
from year's end to year's end, the old hag; her legs are bad and yet
here all of a sudden she is out for a walk!''

``Hadn't we better ask the porter?''

``What?''

``Where she's gone and when she'll be back.''

``Hm\ldots{}. Damn it all!\ldots{} We might ask\ldots{}. But you know
she never does go anywhere.''

And he once more tugged at the door-handle.

``Damn it all. There's nothing to be done, we must go!''

``Stay!'' cried the young man suddenly. ``Do you see how the door shakes
if you pull it?''

``Well?''

``That shows it's not locked, but fastened with the hook! Do you hear
how the hook clanks?''

``Well?''

``Why, don't you see? That proves that one of them is at home. If they
were all out, they would have locked the door from the outside with the
key and not with the hook from inside. There, do you hear how the hook
is clanking? To fasten the hook on the inside they must be at home,
don't you see. So there they are sitting inside and don't open the
door!''

``Well! And so they must be!'' cried Koch, astonished. ``What are they
about in there?'' And he began furiously shaking the door.

``Stay!'' cried the young man again. ``Don't pull at it! There must be
something wrong\ldots{}. Here, you've been ringing and pulling at the
door and still they don't open! So either they've both fainted
or\ldots{}''

``What?''

``I tell you what. Let's go fetch the porter, let him wake them up.''

``All right.''

Both were going down.

``Stay. You stop here while I run down for the porter.''

``What for?''

``Well, you'd better.''

``All right.''

``I'm studying the law you see! It's evident, e-vi-dent there's
something wrong here!'' the young man cried hotly, and he ran
downstairs.

Koch remained. Once more he softly touched the bell which gave one
tinkle, then gently, as though reflecting and looking about him, began
touching the door-handle pulling it and letting it go to make sure once
more that it was only fastened by the hook. Then puffing and panting he
bent down and began looking at the keyhole: but the key was in the lock
on the inside and so nothing could be seen.

Raskolnikov stood keeping tight hold of the axe. He was in a sort of
delirium. He was even making ready to fight when they should come in.
While they were knocking and talking together, the idea several times
occurred to him to end it all at once and shout to them through the
door. Now and then he was tempted to swear at them, to jeer at them,
while they could not open the door! ``Only make haste!'' was the thought
that flashed through his mind.

``But what the devil is he about?\ldots{}'' Time was passing, one
minute, and another---no one came. Koch began to be restless.

``What the devil?'' he cried suddenly and in impatience deserting his
sentry duty, he, too, went down, hurrying and thumping with his heavy
boots on the stairs. The steps died away.

``Good heavens! What am I to do?''

Raskolnikov unfastened the hook, opened the door---there was no sound.
Abruptly, without any thought at all, he went out, closing the door as
thoroughly as he could, and went downstairs.

He had gone down three flights when he suddenly heard a loud voice
below---where could he go! There was nowhere to hide. He was just going
back to the flat.

``Hey there! Catch the brute!''

Somebody dashed out of a flat below, shouting, and rather fell than ran
down the stairs, bawling at the top of his voice.

``Mitka! Mitka! Mitka! Mitka! Mitka! Blast him!''

The shout ended in a shriek; the last sounds came from the yard; all was
still. But at the same instant several men talking loud and fast began
noisily mounting the stairs. There were three or four of them. He
distinguished the ringing voice of the young man. ``Hey!''

Filled with despair he went straight to meet them, feeling ``come what
must!'' If they stopped him---all was lost; if they let him pass---all
was lost too; they would remember him. They were approaching; they were
only a flight from him---and suddenly deliverance! A few steps from him
on the right, there was an empty flat with the door wide open, the flat
on the second floor where the painters had been at work, and which, as
though for his benefit, they had just left. It was they, no doubt, who
had just run down, shouting. The floor had only just been painted, in
the middle of the room stood a pail and a broken pot with paint and
brushes. In one instant he had whisked in at the open door and hidden
behind the wall and only in the nick of time; they had already reached
the landing. Then they turned and went on up to the fourth floor,
talking loudly. He waited, went out on tiptoe and ran down the stairs.

No one was on the stairs, nor in the gateway. He passed quickly through
the gateway and turned to the left in the street.

He knew, he knew perfectly well that at that moment they were at the
flat, that they were greatly astonished at finding it unlocked, as the
door had just been fastened, that by now they were looking at the
bodies, that before another minute had passed they would guess and
completely realise that the murderer had just been there, and had
succeeded in hiding somewhere, slipping by them and escaping. They would
guess most likely that he had been in the empty flat, while they were
going upstairs. And meanwhile he dared not quicken his pace much, though
the next turning was still nearly a hundred yards away. ``Should he slip
through some gateway and wait somewhere in an unknown street? No,
hopeless! Should he fling away the axe? Should he take a cab? Hopeless,
hopeless!''

At last he reached the turning. He turned down it more dead than alive.
Here he was half way to safety, and he understood it; it was less risky
because there was a great crowd of people, and he was lost in it like a
grain of sand. But all he had suffered had so weakened him that he could
scarcely move. Perspiration ran down him in drops, his neck was all wet.
``My word, he has been going it!'' someone shouted at him when he came
out on the canal bank.

He was only dimly conscious of himself now, and the farther he went the
worse it was. He remembered however, that on coming out on to the canal
bank, he was alarmed at finding few people there and so being more
conspicuous, and he had thought of turning back. Though he was almost
falling from fatigue, he went a long way round so as to get home from
quite a different direction.

He was not fully conscious when he passed through the gateway of his
house! He was already on the staircase before he recollected the axe.
And yet he had a very grave problem before him, to put it back and to
escape observation as far as possible in doing so. He was of course
incapable of reflecting that it might perhaps be far better not to
restore the axe at all, but to drop it later on in somebody's yard. But
it all happened fortunately, the door of the porter's room was closed
but not locked, so that it seemed most likely that the porter was at
home. But he had so completely lost all power of reflection that he
walked straight to the door and opened it. If the porter had asked him,
``What do you want?'' he would perhaps have simply handed him the axe.
But again the porter was not at home, and he succeeded in putting the
axe back under the bench, and even covering it with the chunk of wood as
before. He met no one, not a soul, afterwards on the way to his room;
the landlady's door was shut. When he was in his room, he flung himself
on the sofa just as he was---he did not sleep, but sank into blank
forgetfulness. If anyone had come into his room then, he would have
jumped up at once and screamed. Scraps and shreds of thoughts were
simply swarming in his brain, but he could not catch at one, he could
not rest on one, in spite of all his efforts\ldots{}.

\part*{PART II}
\addcontentsline{toc}{part}{PART II}

\mychapter{1}{Chapter I}\label{chapter-i-1}

So he lay a very long while. Now and then he seemed to wake up, and at
such moments he noticed that it was far into the night, but it did not
occur to him to get up. At last he noticed that it was beginning to get
light. He was lying on his back, still dazed from his recent oblivion.
Fearful, despairing cries rose shrilly from the street, sounds which he
heard every night, indeed, under his window after two o'clock. They woke
him up now.

``Ah! the drunken men are coming out of the taverns,'' he thought,
``it's past two o'clock,'' and at once he leaped up, as though someone
had pulled him from the sofa.

``What! Past two o'clock!''

He sat down on the sofa---and instantly recollected everything! All at
once, in one flash, he recollected everything.

For the first moment he thought he was going mad. A dreadful chill came
over him; but the chill was from the fever that had begun long before in
his sleep. Now he was suddenly taken with violent shivering, so that his
teeth chattered and all his limbs were shaking. He opened the door and
began listening---everything in the house was asleep. With amazement he
gazed at himself and everything in the room around him, wondering how he
could have come in the night before without fastening the door, and have
flung himself on the sofa without undressing, without even taking his
hat off. It had fallen off and was lying on the floor near his pillow.

``If anyone had come in, what would he have thought? That I'm drunk
but\ldots{}''

He rushed to the window. There was light enough, and he began hurriedly
looking himself all over from head to foot, all his clothes; were there
no traces? But there was no doing it like that; shivering with cold, he
began taking off everything and looking over again. He turned everything
over to the last threads and rags, and mistrusting himself, went through
his search three times.

But there seemed to be nothing, no trace, except in one place, where
some thick drops of congealed blood were clinging to the frayed edge of
his trousers. He picked up a big claspknife and cut off the frayed
threads. There seemed to be nothing more.

Suddenly he remembered that the purse and the things he had taken out of
the old woman's box were still in his pockets! He had not thought till
then of taking them out and hiding them! He had not even thought of them
while he was examining his clothes! What next? Instantly he rushed to
take them out and fling them on the table. When he had pulled out
everything, and turned the pocket inside out to be sure there was
nothing left, he carried the whole heap to the corner. The paper had
come off the bottom of the wall and hung there in tatters. He began
stuffing all the things into the hole under the paper: ``They're in! All
out of sight, and the purse too!'' he thought gleefully, getting up and
gazing blankly at the hole which bulged out more than ever. Suddenly he
shuddered all over with horror; ``My God!'' he whispered in despair:
``what's the matter with me? Is that hidden? Is that the way to hide
things?''

He had not reckoned on having trinkets to hide. He had only thought of
money, and so had not prepared a hiding-place.

``But now, now, what am I glad of?'' he thought, ``Is that hiding
things? My reason's deserting me---simply!''

He sat down on the sofa in exhaustion and was at once shaken by another
unbearable fit of shivering. Mechanically he drew from a chair beside
him his old student's winter coat, which was still warm though almost in
rags, covered himself up with it and once more sank into drowsiness and
delirium. He lost consciousness.

Not more than five minutes had passed when he jumped up a second time,
and at once pounced in a frenzy on his clothes again.

``How could I go to sleep again with nothing done? Yes, yes; I have not
taken the loop off the armhole! I forgot it, forgot a thing like that!
Such a piece of evidence!''

He pulled off the noose, hurriedly cut it to pieces and threw the bits
among his linen under the pillow.

``Pieces of torn linen couldn't rouse suspicion, whatever happened; I
think not, I think not, any way!'' he repeated, standing in the middle
of the room, and with painful concentration he fell to gazing about him
again, at the floor and everywhere, trying to make sure he had not
forgotten anything. The conviction that all his faculties, even memory,
and the simplest power of reflection were failing him, began to be an
insufferable torture.

``Surely it isn't beginning already! Surely it isn't my punishment
coming upon me? It is!''

The frayed rags he had cut off his trousers were actually lying on the
floor in the middle of the room, where anyone coming in would see them!

``What is the matter with me!'' he cried again, like one distraught.

Then a strange idea entered his head; that, perhaps, all his clothes
were covered with blood, that, perhaps, there were a great many stains,
but that he did not see them, did not notice them because his
perceptions were failing, were going to pieces\ldots{} his reason was
clouded\ldots{}. Suddenly he remembered that there had been blood on the
purse too. ``Ah! Then there must be blood on the pocket too, for I put
the wet purse in my pocket!''

In a flash he had turned the pocket inside out and, yes!---there were
traces, stains on the lining of the pocket!

``So my reason has not quite deserted me, so I still have some sense and
memory, since I guessed it of myself,'' he thought triumphantly, with a
deep sigh of relief; ``it's simply the weakness of fever, a moment's
delirium,'' and he tore the whole lining out of the left pocket of his
trousers. At that instant the sunlight fell on his left boot; on the
sock which poked out from the boot, he fancied there were traces! He
flung off his boots; ``traces indeed! The tip of the sock was soaked
with blood;'' he must have unwarily stepped into that pool\ldots{}.
``But what am I to do with this now? Where am I to put the sock and rags
and pocket?''

He gathered them all up in his hands and stood in the middle of the
room.

``In the stove? But they would ransack the stove first of all. Burn
them? But what can I burn them with? There are no matches even. No,
better go out and throw it all away somewhere. Yes, better throw it
away,'' he repeated, sitting down on the sofa again, ``and at once, this
minute, without lingering\ldots{}''

But his head sank on the pillow instead. Again the unbearable icy
shivering came over him; again he drew his coat over him.

And for a long while, for some hours, he was haunted by the impulse to
``go off somewhere at once, this moment, and fling it all away, so that
it may be out of sight and done with, at once, at once!'' Several times
he tried to rise from the sofa, but could not.

He was thoroughly waked up at last by a violent knocking at his door.

``Open, do, are you dead or alive? He keeps sleeping here!'' shouted
Nastasya, banging with her fist on the door. ``For whole days together
he's snoring here like a dog! A dog he is too. Open I tell you. It's
past ten.''

``Maybe he's not at home,'' said a man's voice.

``Ha! that's the porter's voice\ldots{}. What does he want?''

He jumped up and sat on the sofa. The beating of his heart was a
positive pain.

``Then who can have latched the door?'' retorted Nastasya. ``He's taken
to bolting himself in! As if he were worth stealing! Open, you stupid,
wake up!''

``What do they want? Why the porter? All's discovered. Resist or open?
Come what may!\ldots{}''

He half rose, stooped forward and unlatched the door.

His room was so small that he could undo the latch without leaving the
bed. Yes; the porter and Nastasya were standing there.

Nastasya stared at him in a strange way. He glanced with a defiant and
desperate air at the porter, who without a word held out a grey folded
paper sealed with bottle-wax.

``A notice from the office,'' he announced, as he gave him the paper.

``From what office?''

``A summons to the police office, of course. You know which office.''

``To the police?\ldots{} What for?\ldots{}''

``How can I tell? You're sent for, so you go.''

The man looked at him attentively, looked round the room and turned to
go away.

``He's downright ill!'' observed Nastasya, not taking her eyes off him.
The porter turned his head for a moment. ``He's been in a fever since
yesterday,'' she added.

Raskolnikov made no response and held the paper in his hands, without
opening it. ``Don't you get up then,'' Nastasya went on compassionately,
seeing that he was letting his feet down from the sofa. ``You're ill,
and so don't go; there's no such hurry. What have you got there?''

He looked; in his right hand he held the shreds he had cut from his
trousers, the sock, and the rags of the pocket. So he had been asleep
with them in his hand. Afterwards reflecting upon it, he remembered that
half waking up in his fever, he had grasped all this tightly in his hand
and so fallen asleep again.

``Look at the rags he's collected and sleeps with them, as though he has
got hold of a treasure\ldots{}''

And Nastasya went off into her hysterical giggle.

Instantly he thrust them all under his great coat and fixed his eyes
intently upon her. Far as he was from being capable of rational
reflection at that moment, he felt that no one would behave like that
with a person who was going to be arrested. ``But\ldots{} the police?''

``You'd better have some tea! Yes? I'll bring it, there's some left.''

``No\ldots{} I'm going; I'll go at once,'' he muttered, getting on to
his feet.

``Why, you'll never get downstairs!''

``Yes, I'll go.''

``As you please.''

She followed the porter out.

At once he rushed to the light to examine the sock and the rags.

``There are stains, but not very noticeable; all covered with dirt, and
rubbed and already discoloured. No one who had no suspicion could
distinguish anything. Nastasya from a distance could not have noticed,
thank God!'' Then with a tremor he broke the seal of the notice and
began reading; he was a long while reading, before he understood. It was
an ordinary summons from the district police-station to appear that day
at half-past nine at the office of the district superintendent.

``But when has such a thing happened? I never have anything to do with
the police! And why just to-day?'' he thought in agonising bewilderment.
``Good God, only get it over soon!''

He was flinging himself on his knees to pray, but broke into
laughter---not at the idea of prayer, but at himself.

He began, hurriedly dressing. ``If I'm lost, I am lost, I don't care!
Shall I put the sock on?'' he suddenly wondered, ``it will get dustier
still and the traces will be gone.''

But no sooner had he put it on than he pulled it off again in loathing
and horror. He pulled it off, but reflecting that he had no other socks,
he picked it up and put it on again---and again he laughed.

``That's all conventional, that's all relative, merely a way of looking
at it,'' he thought in a flash, but only on the top surface of his mind,
while he was shuddering all over, ``there, I've got it on! I have
finished by getting it on!''

But his laughter was quickly followed by despair.

``No, it's too much for me\ldots{}'' he thought. His legs shook. ``From
fear,'' he muttered. His head swam and ached with fever. ``It's a trick!
They want to decoy me there and confound me over everything,'' he mused,
as he went out on to the stairs---``the worst of it is I'm almost
light-headed\ldots{} I may blurt out something stupid\ldots{}''

On the stairs he remembered that he was leaving all the things just as
they were in the hole in the wall, ``and very likely, it's on purpose to
search when I'm out,'' he thought, and stopped short. But he was
possessed by such despair, such cynicism of misery, if one may so call
it, that with a wave of his hand he went on. ``Only to get it over!''

In the street the heat was insufferable again; not a drop of rain had
fallen all those days. Again dust, bricks and mortar, again the stench
from the shops and pot-houses, again the drunken men, the Finnish
pedlars and half-broken-down cabs. The sun shone straight in his eyes,
so that it hurt him to look out of them, and he felt his head going
round---as a man in a fever is apt to feel when he comes out into the
street on a bright sunny day.

When he reached the turning into \emph{the} street, in an agony of
trepidation he looked down it\ldots{} at \emph{the} house\ldots{} and at
once averted his eyes.

``If they question me, perhaps I'll simply tell,'' he thought, as he
drew near the police-station.

The police-station was about a quarter of a mile off. It had lately been
moved to new rooms on the fourth floor of a new house. He had been once
for a moment in the old office but long ago. Turning in at the gateway,
he saw on the right a flight of stairs which a peasant was mounting with
a book in his hand. ``A house-porter, no doubt; so then, the office is
here,'' and he began ascending the stairs on the chance. He did not want
to ask questions of anyone.

``I'll go in, fall on my knees, and confess everything\ldots{}'' he
thought, as he reached the fourth floor.

The staircase was steep, narrow and all sloppy with dirty water. The
kitchens of the flats opened on to the stairs and stood open almost the
whole day. So there was a fearful smell and heat. The staircase was
crowded with porters going up and down with their books under their
arms, policemen, and persons of all sorts and both sexes. The door of
the office, too, stood wide open. Peasants stood waiting within. There,
too, the heat was stifling and there was a sickening smell of fresh
paint and stale oil from the newly decorated rooms.

After waiting a little, he decided to move forward into the next room.
All the rooms were small and low-pitched. A fearful impatience drew him
on and on. No one paid attention to him. In the second room some clerks
sat writing, dressed hardly better than he was, and rather a
queer-looking set. He went up to one of them.

``What is it?''

He showed the notice he had received.

``You are a student?'' the man asked, glancing at the notice.

``Yes, formerly a student.''

The clerk looked at him, but without the slightest interest. He was a
particularly unkempt person with the look of a fixed idea in his eye.

``There would be no getting anything out of him, because he has no
interest in anything,'' thought Raskolnikov.

``Go in there to the head clerk,'' said the clerk, pointing towards the
furthest room.

He went into that room---the fourth in order; it was a small room and
packed full of people, rather better dressed than in the outer rooms.
Among them were two ladies. One, poorly dressed in mourning, sat at the
table opposite the chief clerk, writing something at his dictation. The
other, a very stout, buxom woman with a purplish-red, blotchy face,
excessively smartly dressed with a brooch on her bosom as big as a
saucer, was standing on one side, apparently waiting for something.
Raskolnikov thrust his notice upon the head clerk. The latter glanced at
it, said: ``Wait a minute,'' and went on attending to the lady in
mourning.

He breathed more freely. ``It can't be that!''

By degrees he began to regain confidence, he kept urging himself to have
courage and be calm.

``Some foolishness, some trifling carelessness, and I may betray myself!
Hm\ldots{} it's a pity there's no air here,'' he added, ``it's
stifling\ldots{}. It makes one's head dizzier than ever\ldots{} and
one's mind too\ldots{}''

He was conscious of a terrible inner turmoil. He was afraid of losing
his self-control; he tried to catch at something and fix his mind on it,
something quite irrelevant, but he could not succeed in this at all. Yet
the head clerk greatly interested him, he kept hoping to see through him
and guess something from his face.

He was a very young man, about two and twenty, with a dark mobile face
that looked older than his years. He was fashionably dressed and
foppish, with his hair parted in the middle, well combed and pomaded,
and wore a number of rings on his well-scrubbed fingers and a gold chain
on his waistcoat. He said a couple of words in French to a foreigner who
was in the room, and said them fairly correctly.

``Luise Ivanovna, you can sit down,'' he said casually to the
gaily-dressed, purple-faced lady, who was still standing as though not
venturing to sit down, though there was a chair beside her.

``Ich danke,'' said the latter, and softly, with a rustle of silk she
sank into the chair. Her light blue dress trimmed with white lace
floated about the table like an air-balloon and filled almost half the
room. She smelt of scent. But she was obviously embarrassed at filling
half the room and smelling so strongly of scent; and though her smile
was impudent as well as cringing, it betrayed evident uneasiness.

The lady in mourning had done at last, and got up. All at once, with
some noise, an officer walked in very jauntily, with a peculiar swing of
his shoulders at each step. He tossed his cockaded cap on the table and
sat down in an easy-chair. The small lady positively skipped from her
seat on seeing him, and fell to curtsying in a sort of ecstasy; but the
officer took not the smallest notice of her, and she did not venture to
sit down again in his presence. He was the assistant superintendent. He
had a reddish moustache that stood out horizontally on each side of his
face, and extremely small features, expressive of nothing much except a
certain insolence. He looked askance and rather indignantly at
Raskolnikov; he was so very badly dressed, and in spite of his
humiliating position, his bearing was by no means in keeping with his
clothes. Raskolnikov had unwarily fixed a very long and direct look on
him, so that he felt positively affronted.

``What do you want?'' he shouted, apparently astonished that such a
ragged fellow was not annihilated by the majesty of his glance.

``I was summoned\ldots{} by a notice\ldots{}'' Raskolnikov faltered.

``For the recovery of money due, from \emph{the student},'' the head
clerk interfered hurriedly, tearing himself from his papers. ``Here!''
and he flung Raskolnikov a document and pointed out the place. ``Read
that!''

``Money? What money?'' thought Raskolnikov, ``but\ldots{} then\ldots{}
it's certainly not \emph{that}.''

And he trembled with joy. He felt sudden intense indescribable relief. A
load was lifted from his back.

``And pray, what time were you directed to appear, sir?'' shouted the
assistant superintendent, seeming for some unknown reason more and more
aggrieved. ``You are told to come at nine, and now it's twelve!''

``The notice was only brought me a quarter of an hour ago,'' Raskolnikov
answered loudly over his shoulder. To his own surprise he, too, grew
suddenly angry and found a certain pleasure in it. ``And it's enough
that I have come here ill with fever.''

``Kindly refrain from shouting!''

``I'm not shouting, I'm speaking very quietly, it's you who are shouting
at me. I'm a student, and allow no one to shout at me.''

The assistant superintendent was so furious that for the first minute he
could only splutter inarticulately. He leaped up from his seat.

``Be silent! You are in a government office. Don't be impudent, sir!''

``You're in a government office, too,'' cried Raskolnikov, ``and you're
smoking a cigarette as well as shouting, so you are showing disrespect
to all of us.''

He felt an indescribable satisfaction at having said this.

The head clerk looked at him with a smile. The angry assistant
superintendent was obviously disconcerted.

``That's not your business!'' he shouted at last with unnatural
loudness. ``Kindly make the declaration demanded of you. Show him.
Alexandr Grigorievitch. There is a complaint against you! You don't pay
your debts! You're a fine bird!''

But Raskolnikov was not listening now; he had eagerly clutched at the
paper, in haste to find an explanation. He read it once, and a second
time, and still did not understand.

``What is this?'' he asked the head clerk.

``It is for the recovery of money on an I O U, a writ. You must either
pay it, with all expenses, costs and so on, or give a written
declaration when you can pay it, and at the same time an undertaking not
to leave the capital without payment, and nor to sell or conceal your
property. The creditor is at liberty to sell your property, and proceed
against you according to the law.''

``But I\ldots{} am not in debt to anyone!''

``That's not our business. Here, an I O U for a hundred and fifteen
roubles, legally attested, and due for payment, has been brought us for
recovery, given by you to the widow of the assessor Zarnitsyn, nine
months ago, and paid over by the widow Zarnitsyn to one Mr. Tchebarov.
We therefore summon you, hereupon.''

``But she is my landlady!''

``And what if she is your landlady?''

The head clerk looked at him with a condescending smile of compassion,
and at the same time with a certain triumph, as at a novice under fire
for the first time---as though he would say: ``Well, how do you feel
now?'' But what did he care now for an I O U, for a writ of recovery!
Was that worth worrying about now, was it worth attention even! He
stood, he read, he listened, he answered, he even asked questions
himself, but all mechanically. The triumphant sense of security, of
deliverance from overwhelming danger, that was what filled his whole
soul that moment without thought for the future, without analysis,
without suppositions or surmises, without doubts and without
questioning. It was an instant of full, direct, purely instinctive joy.
But at that very moment something like a thunderstorm took place in the
office. The assistant superintendent, still shaken by Raskolnikov's
disrespect, still fuming and obviously anxious to keep up his wounded
dignity, pounced on the unfortunate smart lady, who had been gazing at
him ever since he came in with an exceedingly silly smile.

``You shameful hussy!'' he shouted suddenly at the top of his voice.
(The lady in mourning had left the office.) ``What was going on at your
house last night? Eh! A disgrace again, you're a scandal to the whole
street. Fighting and drinking again. Do you want the house of
correction? Why, I have warned you ten times over that I would not let
you off the eleventh! And here you are again, again, you\ldots{}
you\ldots{}!''

The paper fell out of Raskolnikov's hands, and he looked wildly at the
smart lady who was so unceremoniously treated. But he soon saw what it
meant, and at once began to find positive amusement in the scandal. He
listened with pleasure, so that he longed to laugh and laugh\ldots{} all
his nerves were on edge.

``Ilya Petrovitch!'' the head clerk was beginning anxiously, but stopped
short, for he knew from experience that the enraged assistant could not
be stopped except by force.

As for the smart lady, at first she positively trembled before the
storm. But, strange to say, the more numerous and violent the terms of
abuse became, the more amiable she looked, and the more seductive the
smiles she lavished on the terrible assistant. She moved uneasily, and
curtsied incessantly, waiting impatiently for a chance of putting in her
word: and at last she found it.

``There was no sort of noise or fighting in my house, Mr. Captain,'' she
pattered all at once, like peas dropping, speaking Russian confidently,
though with a strong German accent, ``and no sort of scandal, and his
honour came drunk, and it's the whole truth I am telling, Mr. Captain,
and I am not to blame\ldots{}. Mine is an honourable house, Mr. Captain,
and honourable behaviour, Mr. Captain, and I always, always dislike any
scandal myself. But he came quite tipsy, and asked for three bottles
again, and then he lifted up one leg, and began playing the pianoforte
with one foot, and that is not at all right in an honourable house, and
he \emph{ganz} broke the piano, and it was very bad manners indeed and I
said so. And he took up a bottle and began hitting everyone with it. And
then I called the porter, and Karl came, and he took Karl and hit him in
the eye; and he hit Henriette in the eye, too, and gave me five slaps on
the cheek. And it was so ungentlemanly in an honourable house, Mr.
Captain, and I screamed. And he opened the window over the canal, and
stood in the window, squealing like a little pig; it was a disgrace. The
idea of squealing like a little pig at the window into the street! Fie
upon him! And Karl pulled him away from the window by his coat, and it
is true, Mr. Captain, he tore \emph{sein rock}. And then he shouted that
\emph{man muss} pay him fifteen roubles damages. And I did pay him, Mr.
Captain, five roubles for \emph{sein rock}. And he is an ungentlemanly
visitor and caused all the scandal. `I will show you up,' he said, `for
I can write to all the papers about you.'''

``Then he was an author?''

``Yes, Mr. Captain, and what an ungentlemanly visitor in an honourable
house\ldots{}.''

``Now then! Enough! I have told you already\ldots{}''

``Ilya Petrovitch!'' the head clerk repeated significantly.

The assistant glanced rapidly at him; the head clerk slightly shook his
head.

``\ldots{} So I tell you this, most respectable Luise Ivanovna, and I
tell it you for the last time,'' the assistant went on. ``If there is a
scandal in your honourable house once again, I will put you yourself in
the lock-up, as it is called in polite society. Do you hear? So a
literary man, an author took five roubles for his coat-tail in an
`honourable house'? A nice set, these authors!''

And he cast a contemptuous glance at Raskolnikov. ``There was a scandal
the other day in a restaurant, too. An author had eaten his dinner and
would not pay; `I'll write a satire on you,' says he. And there was
another of them on a steamer last week used the most disgraceful
language to the respectable family of a civil councillor, his wife and
daughter. And there was one of them turned out of a confectioner's shop
the other day. They are like that, authors, literary men, students,
town-criers\ldots{}. Pfoo! You get along! I shall look in upon you
myself one day. Then you had better be careful! Do you hear?''

With hurried deference, Luise Ivanovna fell to curtsying in all
directions, and so curtsied herself to the door. But at the door, she
stumbled backwards against a good-looking officer with a fresh, open
face and splendid thick fair whiskers. This was the superintendent of
the district himself, Nikodim Fomitch. Luise Ivanovna made haste to
curtsy almost to the ground, and with mincing little steps, she
fluttered out of the office.

``Again thunder and lightning---a hurricane!'' said Nikodim Fomitch to
Ilya Petrovitch in a civil and friendly tone. ``You are aroused again,
you are fuming again! I heard it on the stairs!''

``Well, what then!'' Ilya Petrovitch drawled with gentlemanly
nonchalance; and he walked with some papers to another table, with a
jaunty swing of his shoulders at each step. ``Here, if you will kindly
look: an author, or a student, has been one at least, does not pay his
debts, has given an I O U, won't clear out of his room, and complaints
are constantly being lodged against him, and here he has been pleased to
make a protest against my smoking in his presence! He behaves like a cad
himself, and just look at him, please. Here's the gentleman, and very
attractive he is!''

``Poverty is not a vice, my friend, but we know you go off like powder,
you can't bear a slight, I daresay you took offence at something and
went too far yourself,'' continued Nikodim Fomitch, turning affably to
Raskolnikov. ``But you were wrong there; he is a capital fellow, I
assure you, but explosive, explosive! He gets hot, fires up, boils over,
and no stopping him! And then it's all over! And at the bottom he's a
heart of gold! His nickname in the regiment was the Explosive
Lieutenant\ldots{}.''

``And what a regiment it was, too,'' cried Ilya Petrovitch, much
gratified at this agreeable banter, though still sulky.

Raskolnikov had a sudden desire to say something exceptionally pleasant
to them all. ``Excuse me, Captain,'' he began easily, suddenly
addressing Nikodim Fomitch, ``will you enter into my position?\ldots{} I
am ready to ask pardon, if I have been ill-mannered. I am a poor
student, sick and shattered (shattered was the word he used) by poverty.
I am not studying, because I cannot keep myself now, but I shall get
money\ldots{}. I have a mother and sister in the province of X. They
will send it to me, and I will pay. My landlady is a good-hearted woman,
but she is so exasperated at my having lost my lessons, and not paying
her for the last four months, that she does not even send up my
dinner\ldots{} and I don't understand this I O U at all. She is asking
me to pay her on this I O U. How am I to pay her? Judge for
yourselves!\ldots{}''

``But that is not our business, you know,'' the head clerk was
observing.

``Yes, yes. I perfectly agree with you. But allow me to
explain\ldots{}'' Raskolnikov put in again, still addressing Nikodim
Fomitch, but trying his best to address Ilya Petrovitch also, though the
latter persistently appeared to be rummaging among his papers and to be
contemptuously oblivious of him. ``Allow me to explain that I have been
living with her for nearly three years and at first\ldots{} at
first\ldots{} for why should I not confess it, at the very beginning I
promised to marry her daughter, it was a verbal promise, freely
given\ldots{} she was a girl\ldots{} indeed, I liked her, though I was
not in love with her\ldots{} a youthful affair in fact\ldots{} that is,
I mean to say, that my landlady gave me credit freely in those days, and
I led a life of\ldots{} I was very heedless\ldots{}''

``Nobody asks you for these personal details, sir, we've no time to
waste,'' Ilya Petrovitch interposed roughly and with a note of triumph;
but Raskolnikov stopped him hotly, though he suddenly found it
exceedingly difficult to speak.

``But excuse me, excuse me. It is for me to explain\ldots{} how it all
happened\ldots{} In my turn\ldots{} though I agree with you\ldots{} it
is unnecessary. But a year ago, the girl died of typhus. I remained
lodging there as before, and when my landlady moved into her present
quarters, she said to me\ldots{} and in a friendly way\ldots{} that she
had complete trust in me, but still, would I not give her an I O U for
one hundred and fifteen roubles, all the debt I owed her. She said if
only I gave her that, she would trust me again, as much as I liked, and
that she would never, never---those were her own words---make use of
that I O U till I could pay of myself\ldots{} and now, when I have lost
my lessons and have nothing to eat, she takes action against me. What am
I to say to that?''

``All these affecting details are no business of ours.'' Ilya Petrovitch
interrupted rudely. ``You must give a written undertaking but as for
your love affairs and all these tragic events, we have nothing to do
with that.''

``Come now\ldots{} you are harsh,'' muttered Nikodim Fomitch, sitting
down at the table and also beginning to write. He looked a little
ashamed.

``Write!'' said the head clerk to Raskolnikov.

``Write what?'' the latter asked, gruffly.

``I will dictate to you.''

Raskolnikov fancied that the head clerk treated him more casually and
contemptuously after his speech, but strange to say he suddenly felt
completely indifferent to anyone's opinion, and this revulsion took
place in a flash, in one instant. If he had cared to think a little, he
would have been amazed indeed that he could have talked to them like
that a minute before, forcing his feelings upon them. And where had
those feelings come from? Now if the whole room had been filled, not
with police officers, but with those nearest and dearest to him, he
would not have found one human word for them, so empty was his heart. A
gloomy sensation of agonising, everlasting solitude and remoteness, took
conscious form in his soul. It was not the meanness of his sentimental
effusions before Ilya Petrovitch, nor the meanness of the latter's
triumph over him that had caused this sudden revulsion in his heart. Oh,
what had he to do now with his own baseness, with all these petty
vanities, officers, German women, debts, police-offices? If he had been
sentenced to be burnt at that moment, he would not have stirred, would
hardly have heard the sentence to the end. Something was happening to
him entirely new, sudden and unknown. It was not that he understood, but
he felt clearly with all the intensity of sensation that he could never
more appeal to these people in the police-office with sentimental
effusions like his recent outburst, or with anything whatever; and that
if they had been his own brothers and sisters and not police-officers,
it would have been utterly out of the question to appeal to them in any
circumstance of life. He had never experienced such a strange and awful
sensation. And what was most agonising---it was more a sensation than a
conception or idea, a direct sensation, the most agonising of all the
sensations he had known in his life.

The head clerk began dictating to him the usual form of declaration,
that he could not pay, that he undertook to do so at a future date, that
he would not leave the town, nor sell his property, and so on.

``But you can't write, you can hardly hold the pen,'' observed the head
clerk, looking with curiosity at Raskolnikov. ``Are you ill?''

``Yes, I am giddy. Go on!''

``That's all. Sign it.''

The head clerk took the paper, and turned to attend to others.

Raskolnikov gave back the pen; but instead of getting up and going away,
he put his elbows on the table and pressed his head in his hands. He
felt as if a nail were being driven into his skull. A strange idea
suddenly occurred to him, to get up at once, to go up to Nikodim
Fomitch, and tell him everything that had happened yesterday, and then
to go with him to his lodgings and to show him the things in the hole in
the corner. The impulse was so strong that he got up from his seat to
carry it out. ``Hadn't I better think a minute?'' flashed through his
mind. ``No, better cast off the burden without thinking.'' But all at
once he stood still, rooted to the spot. Nikodim Fomitch was talking
eagerly with Ilya Petrovitch, and the words reached him:

``It's impossible, they'll both be released. To begin with, the whole
story contradicts itself. Why should they have called the porter, if it
had been their doing? To inform against themselves? Or as a blind? No,
that would be too cunning! Besides, Pestryakov, the student, was seen at
the gate by both the porters and a woman as he went in. He was walking
with three friends, who left him only at the gate, and he asked the
porters to direct him, in the presence of the friends. Now, would he
have asked his way if he had been going with such an object? As for
Koch, he spent half an hour at the silversmith's below, before he went
up to the old woman and he left him at exactly a quarter to eight. Now
just consider\ldots{}''

``But excuse me, how do you explain this contradiction? They state
themselves that they knocked and the door was locked; yet three minutes
later when they went up with the porter, it turned out the door was
unfastened.''

``That's just it; the murderer must have been there and bolted himself
in; and they'd have caught him for a certainty if Koch had not been an
ass and gone to look for the porter too. \emph{He} must have seized the
interval to get downstairs and slip by them somehow. Koch keeps crossing
himself and saying: `If I had been there, he would have jumped out and
killed me with his axe.' He is going to have a thanksgiving
service---ha, ha!''

``And no one saw the murderer?''

``They might well not see him; the house is a regular Noah's Ark,'' said
the head clerk, who was listening.

``It's clear, quite clear,'' Nikodim Fomitch repeated warmly.

``No, it is anything but clear,'' Ilya Petrovitch maintained.

Raskolnikov picked up his hat and walked towards the door, but he did
not reach it\ldots{}.

When he recovered consciousness, he found himself sitting in a chair,
supported by someone on the right side, while someone else was standing
on the left, holding a yellowish glass filled with yellow water, and
Nikodim Fomitch standing before him, looking intently at him. He got up
from the chair.

``What's this? Are you ill?'' Nikodim Fomitch asked, rather sharply.

``He could hardly hold his pen when he was signing,'' said the head
clerk, settling back in his place, and taking up his work again.

``Have you been ill long?'' cried Ilya Petrovitch from his place, where
he, too, was looking through papers. He had, of course, come to look at
the sick man when he fainted, but retired at once when he recovered.

``Since yesterday,'' muttered Raskolnikov in reply.

``Did you go out yesterday?''

``Yes.''

``Though you were ill?''

``Yes.''

``At what time?''

``About seven.''

``And where did you go, may I ask?''

``Along the street.''

``Short and clear.''

Raskolnikov, white as a handkerchief, had answered sharply, jerkily,
without dropping his black feverish eyes before Ilya Petrovitch's stare.

``He can scarcely stand upright. And you\ldots{}'' Nikodim Fomitch was
beginning.

``No matter,'' Ilya Petrovitch pronounced rather peculiarly.

Nikodim Fomitch would have made some further protest, but glancing at
the head clerk who was looking very hard at him, he did not speak. There
was a sudden silence. It was strange.

``Very well, then,'' concluded Ilya Petrovitch, ``we will not detain
you.''

Raskolnikov went out. He caught the sound of eager conversation on his
departure, and above the rest rose the questioning voice of Nikodim
Fomitch. In the street, his faintness passed off completely.

``A search---there will be a search at once,'' he repeated to himself,
hurrying home. ``The brutes! they suspect.''

His former terror mastered him completely again.

\mychapter{2}{Chapter II}\label{chapter-ii-1}

``And what if there has been a search already? What if I find them in my
room?''

But here was his room. Nothing and no one in it. No one had peeped in.
Even Nastasya had not touched it. But heavens! how could he have left
all those things in the hole?

He rushed to the corner, slipped his hand under the paper, pulled the
things out and lined his pockets with them. There were eight articles in
all: two little boxes with ear-rings or something of the sort, he hardly
looked to see; then four small leather cases. There was a chain, too,
merely wrapped in newspaper and something else in newspaper, that looked
like a decoration\ldots{}. He put them all in the different pockets of
his overcoat, and the remaining pocket of his trousers, trying to
conceal them as much as possible. He took the purse, too. Then he went
out of his room, leaving the door open. He walked quickly and
resolutely, and though he felt shattered, he had his senses about him.
He was afraid of pursuit, he was afraid that in another half-hour,
another quarter of an hour perhaps, instructions would be issued for his
pursuit, and so at all costs, he must hide all traces before then. He
must clear everything up while he still had some strength, some
reasoning power left him\ldots{}. Where was he to go?

That had long been settled: ``Fling them into the canal, and all traces
hidden in the water, the thing would be at an end.'' So he had decided
in the night of his delirium when several times he had had the impulse
to get up and go away, to make haste, and get rid of it all. But to get
rid of it, turned out to be a very difficult task. He wandered along the
bank of the Ekaterininsky Canal for half an hour or more and looked
several times at the steps running down to the water, but he could not
think of carrying out his plan; either rafts stood at the steps' edge,
and women were washing clothes on them, or boats were moored there, and
people were swarming everywhere. Moreover he could be seen and noticed
from the banks on all sides; it would look suspicious for a man to go
down on purpose, stop, and throw something into the water. And what if
the boxes were to float instead of sinking? And of course they would.
Even as it was, everyone he met seemed to stare and look round, as if
they had nothing to do but to watch him. ``Why is it, or can it be my
fancy?'' he thought.

At last the thought struck him that it might be better to go to the
Neva. There were not so many people there, he would be less observed,
and it would be more convenient in every way, above all it was further
off. He wondered how he could have been wandering for a good half-hour,
worried and anxious in this dangerous past without thinking of it
before. And that half-hour he had lost over an irrational plan, simply
because he had thought of it in delirium! He had become extremely absent
and forgetful and he was aware of it. He certainly must make haste.

He walked towards the Neva along V------ Prospect, but on the way
another idea struck him. ``Why to the Neva? Would it not be better to go
somewhere far off, to the Islands again, and there hide the things in
some solitary place, in a wood or under a bush, and mark the spot
perhaps?'' And though he felt incapable of clear judgment, the idea
seemed to him a sound one. But he was not destined to go there. For
coming out of V------ Prospect towards the square, he saw on the left a
passage leading between two blank walls to a courtyard. On the right
hand, the blank unwhitewashed wall of a four-storied house stretched far
into the court; on the left, a wooden hoarding ran parallel with it for
twenty paces into the court, and then turned sharply to the left. Here
was a deserted fenced-off place where rubbish of different sorts was
lying. At the end of the court, the corner of a low, smutty, stone shed,
apparently part of some workshop, peeped from behind the hoarding. It
was probably a carriage builder's or carpenter's shed; the whole place
from the entrance was black with coal dust. Here would be the place to
throw it, he thought. Not seeing anyone in the yard, he slipped in, and
at once saw near the gate a sink, such as is often put in yards where
there are many workmen or cab-drivers; and on the hoarding above had
been scribbled in chalk the time-honoured witticism, ``Standing here
strictly forbidden.'' This was all the better, for there would be
nothing suspicious about his going in. ``Here I could throw it all in a
heap and get away!''

Looking round once more, with his hand already in his pocket, he noticed
against the outer wall, between the entrance and the sink, a big unhewn
stone, weighing perhaps sixty pounds. The other side of the wall was a
street. He could hear passers-by, always numerous in that part, but he
could not be seen from the entrance, unless someone came in from the
street, which might well happen indeed, so there was need of haste.

He bent down over the stone, seized the top of it firmly in both hands,
and using all his strength turned it over. Under the stone was a small
hollow in the ground, and he immediately emptied his pocket into it. The
purse lay at the top, and yet the hollow was not filled up. Then he
seized the stone again and with one twist turned it back, so that it was
in the same position again, though it stood a very little higher. But he
scraped the earth about it and pressed it at the edges with his foot.
Nothing could be noticed.

Then he went out, and turned into the square. Again an intense, almost
unbearable joy overwhelmed him for an instant, as it had in the
police-office. ``I have buried my tracks! And who, who can think of
looking under that stone? It has been lying there most likely ever since
the house was built, and will lie as many years more. And if it were
found, who would think of me? It is all over! No clue!'' And he laughed.
Yes, he remembered that he began laughing a thin, nervous noiseless
laugh, and went on laughing all the time he was crossing the square. But
when he reached the K------ Boulevard where two days before he had come
upon that girl, his laughter suddenly ceased. Other ideas crept into his
mind. He felt all at once that it would be loathsome to pass that seat
on which after the girl was gone, he had sat and pondered, and that it
would be hateful, too, to meet that whiskered policeman to whom he had
given the twenty copecks: ``Damn him!''

He walked, looking about him angrily and distractedly. All his ideas now
seemed to be circling round some single point, and he felt that there
really was such a point, and that now, now, he was left facing that
point---and for the first time, indeed, during the last two months.

``Damn it all!'' he thought suddenly, in a fit of ungovernable fury.
``If it has begun, then it has begun. Hang the new life! Good Lord, how
stupid it is!\ldots{} And what lies I told to-day! How despicably I
fawned upon that wretched Ilya Petrovitch! But that is all folly! What
do I care for them all, and my fawning upon them! It is not that at all!
It is not that at all!''

Suddenly he stopped; a new utterly unexpected and exceedingly simple
question perplexed and bitterly confounded him.

``If it all has really been done deliberately and not idiotically, if I
really had a certain and definite object, how is it I did not even
glance into the purse and don't know what I had there, for which I have
undergone these agonies, and have deliberately undertaken this base,
filthy degrading business? And here I wanted at once to throw into the
water the purse together with all the things which I had not seen
either\ldots{} how's that?''

Yes, that was so, that was all so. Yet he had known it all before, and
it was not a new question for him, even when it was decided in the night
without hesitation and consideration, as though so it must be, as though
it could not possibly be otherwise\ldots{}. Yes, he had known it all,
and understood it all; it surely had all been settled even yesterday at
the moment when he was bending over the box and pulling the jewel-cases
out of it\ldots{}. Yes, so it was.

``It is because I am very ill,'' he decided grimly at last, ``I have
been worrying and fretting myself, and I don't know what I am
doing\ldots{}. Yesterday and the day before yesterday and all this time
I have been worrying myself\ldots{}. I shall get well and I shall not
worry\ldots{}. But what if I don't get well at all? Good God, how sick I
am of it all!''

He walked on without resting. He had a terrible longing for some
distraction, but he did not know what to do, what to attempt. A new
overwhelming sensation was gaining more and more mastery over him every
moment; this was an immeasurable, almost physical, repulsion for
everything surrounding him, an obstinate, malignant feeling of hatred.
All who met him were loathsome to him---he loathed their faces, their
movements, their gestures. If anyone had addressed him, he felt that he
might have spat at him or bitten him\ldots{}.

He stopped suddenly, on coming out on the bank of the Little Neva, near
the bridge to Vassilyevsky Ostrov. ``Why, he lives here, in that
house,'' he thought, ``why, I have not come to Razumihin of my own
accord! Here it's the same thing over again\ldots{}. Very interesting to
know, though; have I come on purpose or have I simply walked here by
chance? Never mind, I said the day before yesterday that I would go and
see him the day \emph{after}; well, and so I will! Besides I really
cannot go further now.''

He went up to Razumihin's room on the fifth floor.

The latter was at home in his garret, busily writing at the moment, and
he opened the door himself. It was four months since they had seen each
other. Razumihin was sitting in a ragged dressing-gown, with slippers on
his bare feet, unkempt, unshaven and unwashed. His face showed surprise.

``Is it you?'' he cried. He looked his comrade up and down; then after a
brief pause, he whistled. ``As hard up as all that! Why, brother, you've
cut me out!'' he added, looking at Raskolnikov's rags. ``Come sit down,
you are tired, I'll be bound.''

And when he had sunk down on the American leather sofa, which was in
even worse condition than his own, Razumihin saw at once that his
visitor was ill.

``Why, you are seriously ill, do you know that?'' He began feeling his
pulse. Raskolnikov pulled away his hand.

``Never mind,'' he said, ``I have come for this: I have no
lessons\ldots{}. I wanted,\ldots{} but I don't really want
lessons\ldots{}.''

``But I say! You are delirious, you know!'' Razumihin observed, watching
him carefully.

``No, I am not.''

Raskolnikov got up from the sofa. As he had mounted the stairs to
Razumihin's, he had not realised that he would be meeting his friend
face to face. Now, in a flash, he knew, that what he was least of all
disposed for at that moment was to be face to face with anyone in the
wide world. His spleen rose within him. He almost choked with rage at
himself as soon as he crossed Razumihin's threshold.

``Good-bye,'' he said abruptly, and walked to the door.

``Stop, stop! You queer fish.''

``I don't want to,'' said the other, again pulling away his hand.

``Then why the devil have you come? Are you mad, or what? Why, this
is\ldots{} almost insulting! I won't let you go like that.''

``Well, then, I came to you because I know no one but you who could
help\ldots{} to begin\ldots{} because you are kinder than
anyone---cleverer, I mean, and can judge\ldots{} and now I see that I
want nothing. Do you hear? Nothing at all\ldots{} no one's
services\ldots{} no one's sympathy. I am by myself\ldots{} alone. Come,
that's enough. Leave me alone.''

``Stay a minute, you sweep! You are a perfect madman. As you like for
all I care. I have no lessons, do you see, and I don't care about that,
but there's a bookseller, Heruvimov---and he takes the place of a
lesson. I would not exchange him for five lessons. He's doing publishing
of a kind, and issuing natural science manuals and what a circulation
they have! The very titles are worth the money! You always maintained
that I was a fool, but by Jove, my boy, there are greater fools than I
am! Now he is setting up for being advanced, not that he has an inkling
of anything, but, of course, I encourage him. Here are two signatures of
the German text---in my opinion, the crudest charlatanism; it discusses
the question, `Is woman a human being?' And, of course, triumphantly
proves that she is. Heruvimov is going to bring out this work as a
contribution to the woman question; I am translating it; he will expand
these two and a half signatures into six, we shall make up a gorgeous
title half a page long and bring it out at half a rouble. It will do! He
pays me six roubles the signature, it works out to about fifteen roubles
for the job, and I've had six already in advance. When we have finished
this, we are going to begin a translation about whales, and then some of
the dullest scandals out of the second part of \emph{Les Confessions} we
have marked for translation; somebody has told Heruvimov, that Rousseau
was a kind of Radishchev. You may be sure I don't contradict him, hang
him! Well, would you like to do the second signature of `\emph{Is woman
a human being?}' If you would, take the German and pens and paper---all
those are provided, and take three roubles; for as I have had six
roubles in advance on the whole thing, three roubles come to you for
your share. And when you have finished the signature there will be
another three roubles for you. And please don't think I am doing you a
service; quite the contrary, as soon as you came in, I saw how you could
help me; to begin with, I am weak in spelling, and secondly, I am
sometimes utterly adrift in German, so that I make it up as I go along
for the most part. The only comfort is, that it's bound to be a change
for the better. Though who can tell, maybe it's sometimes for the worse.
Will you take it?''

Raskolnikov took the German sheets in silence, took the three roubles
and without a word went out. Razumihin gazed after him in astonishment.
But when Raskolnikov was in the next street, he turned back, mounted the
stairs to Razumihin's again and laying on the table the German article
and the three roubles, went out again, still without uttering a word.

``Are you raving, or what?'' Razumihin shouted, roused to fury at last.
``What farce is this? You'll drive me crazy too\ldots{} what did you
come to see me for, damn you?''

``I don't want\ldots{} translation,'' muttered Raskolnikov from the
stairs.

``Then what the devil do you want?'' shouted Razumihin from above.
Raskolnikov continued descending the staircase in silence.

``Hey, there! Where are you living?''

No answer.

``Well, confound you then!''

But Raskolnikov was already stepping into the street. On the Nikolaevsky
Bridge he was roused to full consciousness again by an unpleasant
incident. A coachman, after shouting at him two or three times, gave him
a violent lash on the back with his whip, for having almost fallen under
his horses' hoofs. The lash so infuriated him that he dashed away to the
railing (for some unknown reason he had been walking in the very middle
of the bridge in the traffic). He angrily clenched and ground his teeth.
He heard laughter, of course.

``Serves him right!''

``A pickpocket I dare say.''

``Pretending to be drunk, for sure, and getting under the wheels on
purpose; and you have to answer for him.''

``It's a regular profession, that's what it is.''

But while he stood at the railing, still looking angry and bewildered
after the retreating carriage, and rubbing his back, he suddenly felt
someone thrust money into his hand. He looked. It was an elderly woman
in a kerchief and goatskin shoes, with a girl, probably her daughter,
wearing a hat, and carrying a green parasol.

``Take it, my good man, in Christ's name.''

He took it and they passed on. It was a piece of twenty copecks. From
his dress and appearance they might well have taken him for a beggar
asking alms in the streets, and the gift of the twenty copecks he
doubtless owed to the blow, which made them feel sorry for him.

He closed his hand on the twenty copecks, walked on for ten paces, and
turned facing the Neva, looking towards the palace. The sky was without
a cloud and the water was almost bright blue, which is so rare in the
Neva. The cupola of the cathedral, which is seen at its best from the
bridge about twenty paces from the chapel, glittered in the sunlight,
and in the pure air every ornament on it could be clearly distinguished.
The pain from the lash went off, and Raskolnikov forgot about it; one
uneasy and not quite definite idea occupied him now completely. He stood
still, and gazed long and intently into the distance; this spot was
especially familiar to him. When he was attending the university, he had
hundreds of times---generally on his way home---stood still on this
spot, gazed at this truly magnificent spectacle and almost always
marvelled at a vague and mysterious emotion it roused in him. It left
him strangely cold; this gorgeous picture was for him blank and
lifeless. He wondered every time at his sombre and enigmatic impression
and, mistrusting himself, put off finding the explanation of it. He
vividly recalled those old doubts and perplexities, and it seemed to him
that it was no mere chance that he recalled them now. It struck him as
strange and grotesque, that he should have stopped at the same spot as
before, as though he actually imagined he could think the same thoughts,
be interested in the same theories and pictures that had interested
him\ldots{} so short a time ago. He felt it almost amusing, and yet it
wrung his heart. Deep down, hidden far away out of sight all that seemed
to him now---all his old past, his old thoughts, his old problems and
theories, his old impressions and that picture and himself and all,
all\ldots{}. He felt as though he were flying upwards, and everything
were vanishing from his sight. Making an unconscious movement with his
hand, he suddenly became aware of the piece of money in his fist. He
opened his hand, stared at the coin, and with a sweep of his arm flung
it into the water; then he turned and went home. It seemed to him, he
had cut himself off from everyone and from everything at that moment.

Evening was coming on when he reached home, so that he must have been
walking about six hours. How and where he came back he did not remember.
Undressing, and quivering like an overdriven horse, he lay down on the
sofa, drew his greatcoat over him, and at once sank into
oblivion\ldots{}.

It was dusk when he was waked up by a fearful scream. Good God, what a
scream! Such unnatural sounds, such howling, wailing, grinding, tears,
blows and curses he had never heard.

He could never have imagined such brutality, such frenzy. In terror he
sat up in bed, almost swooning with agony. But the fighting, wailing and
cursing grew louder and louder. And then to his intense amazement he
caught the voice of his landlady. She was howling, shrieking and
wailing, rapidly, hurriedly, incoherently, so that he could not make out
what she was talking about; she was beseeching, no doubt, not to be
beaten, for she was being mercilessly beaten on the stairs. The voice of
her assailant was so horrible from spite and rage that it was almost a
croak; but he, too, was saying something, and just as quickly and
indistinctly, hurrying and spluttering. All at once Raskolnikov
trembled; he recognised the voice---it was the voice of Ilya Petrovitch.
Ilya Petrovitch here and beating the landlady! He is kicking her,
banging her head against the steps---that's clear, that can be told from
the sounds, from the cries and the thuds. How is it, is the world
topsy-turvy? He could hear people running in crowds from all the storeys
and all the staircases; he heard voices, exclamations, knocking, doors
banging. ``But why, why, and how could it be?'' he repeated, thinking
seriously that he had gone mad. But no, he heard too distinctly! And
they would come to him then next, ``for no doubt\ldots{} it's all about
that\ldots{} about yesterday\ldots{}. Good God!'' He would have fastened
his door with the latch, but he could not lift his hand\ldots{} besides,
it would be useless. Terror gripped his heart like ice, tortured him and
numbed him\ldots{}. But at last all this uproar, after continuing about
ten minutes, began gradually to subside. The landlady was moaning and
groaning; Ilya Petrovitch was still uttering threats and curses\ldots{}.
But at last he, too, seemed to be silent, and now he could not be heard.
``Can he have gone away? Good Lord!'' Yes, and now the landlady is going
too, still weeping and moaning\ldots{} and then her door
slammed\ldots{}. Now the crowd was going from the stairs to their rooms,
exclaiming, disputing, calling to one another, raising their voices to a
shout, dropping them to a whisper. There must have been numbers of
them---almost all the inmates of the block. ``But, good God, how could
it be! And why, why had he come here!''

Raskolnikov sank worn out on the sofa, but could not close his eyes. He
lay for half an hour in such anguish, such an intolerable sensation of
infinite terror as he had never experienced before. Suddenly a bright
light flashed into his room. Nastasya came in with a candle and a plate
of soup. Looking at him carefully and ascertaining that he was not
asleep, she set the candle on the table and began to lay out what she
had brought---bread, salt, a plate, a spoon.

``You've eaten nothing since yesterday, I warrant. You've been trudging
about all day, and you're shaking with fever.''

``Nastasya\ldots{} what were they beating the landlady for?''

She looked intently at him.

``Who beat the landlady?''

``Just now\ldots{} half an hour ago, Ilya Petrovitch, the assistant
superintendent, on the stairs\ldots{}. Why was he ill-treating her like
that, and\ldots{} why was he here?''

Nastasya scrutinised him, silent and frowning, and her scrutiny lasted a
long time. He felt uneasy, even frightened at her searching eyes.

``Nastasya, why don't you speak?'' he said timidly at last in a weak
voice.

``It's the blood,'' she answered at last softly, as though speaking to
herself.

``Blood? What blood?'' he muttered, growing white and turning towards
the wall.

Nastasya still looked at him without speaking.

``Nobody has been beating the landlady,'' she declared at last in a
firm, resolute voice.

He gazed at her, hardly able to breathe.

``I heard it myself\ldots{}. I was not asleep\ldots{} I was sitting
up,'' he said still more timidly. ``I listened a long while. The
assistant superintendent came\ldots{}. Everyone ran out on to the stairs
from all the flats.''

``No one has been here. That's the blood crying in your ears. When
there's no outlet for it and it gets clotted, you begin fancying
things\ldots{}. Will you eat something?''

He made no answer. Nastasya still stood over him, watching him.

``Give me something to drink\ldots{} Nastasya.''

She went downstairs and returned with a white earthenware jug of water.
He remembered only swallowing one sip of the cold water and spilling
some on his neck. Then followed forgetfulness.

\mychapter{3}{Chapter III}\label{chapter-iii-1}

He was not completely unconscious, however, all the time he was ill; he
was in a feverish state, sometimes delirious, sometimes half conscious.
He remembered a great deal afterwards. Sometimes it seemed as though
there were a number of people round him; they wanted to take him away
somewhere, there was a great deal of squabbling and discussing about
him. Then he would be alone in the room; they had all gone away afraid
of him, and only now and then opened the door a crack to look at him;
they threatened him, plotted something together, laughed, and mocked at
him. He remembered Nastasya often at his bedside; he distinguished
another person, too, whom he seemed to know very well, though he could
not remember who he was, and this fretted him, even made him cry.
Sometimes he fancied he had been lying there a month; at other times it
all seemed part of the same day. But of \emph{that}---of \emph{that} he
had no recollection, and yet every minute he felt that he had forgotten
something he ought to remember. He worried and tormented himself trying
to remember, moaned, flew into a rage, or sank into awful, intolerable
terror. Then he struggled to get up, would have run away, but someone
always prevented him by force, and he sank back into impotence and
forgetfulness. At last he returned to complete consciousness.

It happened at ten o'clock in the morning. On fine days the sun shone
into the room at that hour, throwing a streak of light on the right wall
and the corner near the door. Nastasya was standing beside him with
another person, a complete stranger, who was looking at him very
inquisitively. He was a young man with a beard, wearing a full,
short-waisted coat, and looked like a messenger. The landlady was
peeping in at the half-opened door. Raskolnikov sat up.

``Who is this, Nastasya?'' he asked, pointing to the young man.

``I say, he's himself again!'' she said.

``He is himself,'' echoed the man.

Concluding that he had returned to his senses, the landlady closed the
door and disappeared. She was always shy and dreaded conversations or
discussions. She was a woman of forty, not at all bad-looking, fat and
buxom, with black eyes and eyebrows, good-natured from fatness and
laziness, and absurdly bashful.

``Who\ldots{} are you?'' he went on, addressing the man. But at that
moment the door was flung open, and, stooping a little, as he was so
tall, Razumihin came in.

``What a cabin it is!'' he cried. ``I am always knocking my head. You
call this a lodging! So you are conscious, brother? I've just heard the
news from Pashenka.''

``He has just come to,'' said Nastasya.

``Just come to,'' echoed the man again, with a smile.

``And who are you?'' Razumihin asked, suddenly addressing him. ``My name
is Vrazumihin, at your service; not Razumihin, as I am always called,
but Vrazumihin, a student and gentleman; and he is my friend. And who
are you?''

``I am the messenger from our office, from the merchant Shelopaev, and
I've come on business.''

``Please sit down.'' Razumihin seated himself on the other side of the
table. ``It's a good thing you've come to, brother,'' he went on to
Raskolnikov. ``For the last four days you have scarcely eaten or drunk
anything. We had to give you tea in spoonfuls. I brought Zossimov to see
you twice. You remember Zossimov? He examined you carefully and said at
once it was nothing serious---something seemed to have gone to your
head. Some nervous nonsense, the result of bad feeding, he says you have
not had enough beer and radish, but it's nothing much, it will pass and
you will be all right. Zossimov is a first-rate fellow! He is making
quite a name. Come, I won't keep you,'' he said, addressing the man
again. ``Will you explain what you want? You must know, Rodya, this is
the second time they have sent from the office; but it was another man
last time, and I talked to him. Who was it came before?''

``That was the day before yesterday, I venture to say, if you please,
sir. That was Alexey Semyonovitch; he is in our office, too.''

``He was more intelligent than you, don't you think so?''

``Yes, indeed, sir, he is of more weight than I am.''

``Quite so; go on.''

``At your mamma's request, through Afanasy Ivanovitch Vahrushin, of whom
I presume you have heard more than once, a remittance is sent to you
from our office,'' the man began, addressing Raskolnikov. ``If you are
in an intelligible condition, I've thirty-five roubles to remit to you,
as Semyon Semyonovitch has received from Afanasy Ivanovitch at your
mamma's request instructions to that effect, as on previous occasions.
Do you know him, sir?''

``Yes, I remember\ldots{} Vahrushin,'' Raskolnikov said dreamily.

``You hear, he knows Vahrushin,'' cried Razumihin. ``He is in `an
intelligible condition'! And I see you are an intelligent man too. Well,
it's always pleasant to hear words of wisdom.''

``That's the gentleman, Vahrushin, Afanasy Ivanovitch. And at the
request of your mamma, who has sent you a remittance once before in the
same manner through him, he did not refuse this time also, and sent
instructions to Semyon Semyonovitch some days since to hand you
thirty-five roubles in the hope of better to come.''

``That `hoping for better to come' is the best thing you've said, though
`your mamma' is not bad either. Come then, what do you say? Is he fully
conscious, eh?''

``That's all right. If only he can sign this little paper.''

``He can scrawl his name. Have you got the book?''

``Yes, here's the book.''

``Give it to me. Here, Rodya, sit up. I'll hold you. Take the pen and
scribble `Raskolnikov' for him. For just now, brother, money is sweeter
to us than treacle.''

``I don't want it,'' said Raskolnikov, pushing away the pen.

``Not want it?''

``I won't sign it.''

``How the devil can you do without signing it?''

``I don't want\ldots{} the money.''

``Don't want the money! Come, brother, that's nonsense, I bear witness.
Don't trouble, please, it's only that he is on his travels again. But
that's pretty common with him at all times though\ldots{}. You are a man
of judgment and we will take him in hand, that is, more simply, take his
hand and he will sign it. Here.''

``But I can come another time.''

``No, no. Why should we trouble you? You are a man of judgment\ldots{}.
Now, Rodya, don't keep your visitor, you see he is waiting,'' and he
made ready to hold Raskolnikov's hand in earnest.

``Stop, I'll do it alone,'' said the latter, taking the pen and signing
his name.

The messenger took out the money and went away.

``Bravo! And now, brother, are you hungry?''

``Yes,'' answered Raskolnikov.

``Is there any soup?''

``Some of yesterday's,'' answered Nastasya, who was still standing
there.

``With potatoes and rice in it?''

``Yes.''

``I know it by heart. Bring soup and give us some tea.''

``Very well.''

Raskolnikov looked at all this with profound astonishment and a dull,
unreasoning terror. He made up his mind to keep quiet and see what would
happen. ``I believe I am not wandering. I believe it's reality,'' he
thought.

In a couple of minutes Nastasya returned with the soup, and announced
that the tea would be ready directly. With the soup she brought two
spoons, two plates, salt, pepper, mustard for the beef, and so on. The
table was set as it had not been for a long time. The cloth was clean.

``It would not be amiss, Nastasya, if Praskovya Pavlovna were to send us
up a couple of bottles of beer. We could empty them.''

``Well, you are a cool hand,'' muttered Nastasya, and she departed to
carry out his orders.

Raskolnikov still gazed wildly with strained attention. Meanwhile
Razumihin sat down on the sofa beside him, as clumsily as a bear put his
left arm round Raskolnikov's head, although he was able to sit up, and
with his right hand gave him a spoonful of soup, blowing on it that it
might not burn him. But the soup was only just warm. Raskolnikov
swallowed one spoonful greedily, then a second, then a third. But after
giving him a few more spoonfuls of soup, Razumihin suddenly stopped, and
said that he must ask Zossimov whether he ought to have more.

Nastasya came in with two bottles of beer.

``And will you have tea?''

``Yes.''

``Cut along, Nastasya, and bring some tea, for tea we may venture on
without the faculty. But here is the beer!'' He moved back to his chair,
pulled the soup and meat in front of him, and began eating as though he
had not touched food for three days.

``I must tell you, Rodya, I dine like this here every day now,'' he
mumbled with his mouth full of beef, ``and it's all Pashenka, your dear
little landlady, who sees to that; she loves to do anything for me. I
don't ask for it, but, of course, I don't object. And here's Nastasya
with the tea. She is a quick girl. Nastasya, my dear, won't you have
some beer?''

``Get along with your nonsense!''

``A cup of tea, then?''

``A cup of tea, maybe.''

``Pour it out. Stay, I'll pour it out myself. Sit down.''

He poured out two cups, left his dinner, and sat on the sofa again. As
before, he put his left arm round the sick man's head, raised him up and
gave him tea in spoonfuls, again blowing each spoonful steadily and
earnestly, as though this process was the principal and most effective
means towards his friend's recovery. Raskolnikov said nothing and made
no resistance, though he felt quite strong enough to sit up on the sofa
without support and could not merely have held a cup or a spoon, but
even perhaps could have walked about. But from some queer, almost
animal, cunning he conceived the idea of hiding his strength and lying
low for a time, pretending if necessary not to be yet in full possession
of his faculties, and meanwhile listening to find out what was going on.
Yet he could not overcome his sense of repugnance. After sipping a dozen
spoonfuls of tea, he suddenly released his head, pushed the spoon away
capriciously, and sank back on the pillow. There were actually real
pillows under his head now, down pillows in clean cases, he observed
that, too, and took note of it.

``Pashenka must give us some raspberry jam to-day to make him some
raspberry tea,'' said Razumihin, going back to his chair and attacking
his soup and beer again.

``And where is she to get raspberries for you?'' asked Nastasya,
balancing a saucer on her five outspread fingers and sipping tea through
a lump of sugar.

``She'll get it at the shop, my dear. You see, Rodya, all sorts of
things have been happening while you have been laid up. When you
decamped in that rascally way without leaving your address, I felt so
angry that I resolved to find you out and punish you. I set to work that
very day. How I ran about making inquiries for you! This lodging of
yours I had forgotten, though I never remembered it, indeed, because I
did not know it; and as for your old lodgings, I could only remember it
was at the Five Corners, Harlamov's house. I kept trying to find that
Harlamov's house, and afterwards it turned out that it was not
Harlamov's, but Buch's. How one muddles up sound sometimes! So I lost my
temper, and I went on the chance to the address bureau next day, and
only fancy, in two minutes they looked you up! Your name is down
there.''

``My name!''

``I should think so; and yet a General Kobelev they could not find while
I was there. Well, it's a long story. But as soon as I did land on this
place, I soon got to know all your affairs---all, all, brother, I know
everything; Nastasya here will tell you. I made the acquaintance of
Nikodim Fomitch and Ilya Petrovitch, and the house-porter and Mr.
Zametov, Alexandr Grigorievitch, the head clerk in the police office,
and, last, but not least, of Pashenka; Nastasya here knows\ldots{}.''

``He's got round her,'' Nastasya murmured, smiling slyly.

``Why don't you put the sugar in your tea, Nastasya Nikiforovna?''

``You are a one!'' Nastasya cried suddenly, going off into a giggle. ``I
am not Nikiforovna, but Petrovna,'' she added suddenly, recovering from
her mirth.

``I'll make a note of it. Well, brother, to make a long story short, I
was going in for a regular explosion here to uproot all malignant
influences in the locality, but Pashenka won the day. I had not
expected, brother, to find her so\ldots{} prepossessing. Eh, what do you
think?''

Raskolnikov did not speak, but he still kept his eyes fixed upon him,
full of alarm.

``And all that could be wished, indeed, in every respect,'' Razumihin
went on, not at all embarrassed by his silence.

``Ah, the sly dog!'' Nastasya shrieked again. This conversation afforded
her unspeakable delight.

``It's a pity, brother, that you did not set to work in the right way at
first. You ought to have approached her differently. She is, so to
speak, a most unaccountable character. But we will talk about her
character later\ldots{}. How could you let things come to such a pass
that she gave up sending you your dinner? And that I O U? You must have
been mad to sign an I O U. And that promise of marriage when her
daughter, Natalya Yegorovna, was alive?\ldots{} I know all about it! But
I see that's a delicate matter and I am an ass; forgive me. But, talking
of foolishness, do you know Praskovya Pavlovna is not nearly so foolish
as you would think at first sight?''

``No,'' mumbled Raskolnikov, looking away, but feeling that it was
better to keep up the conversation.

``She isn't, is she?'' cried Razumihin, delighted to get an answer out
of him. ``But she is not very clever either, eh? She is essentially,
essentially an unaccountable character! I am sometimes quite at a loss,
I assure you\ldots{}. She must be forty; she says she is thirty-six, and
of course she has every right to say so. But I swear I judge her
intellectually, simply from the metaphysical point of view; there is a
sort of symbolism sprung up between us, a sort of algebra or what not! I
don't understand it! Well, that's all nonsense. Only, seeing that you
are not a student now and have lost your lessons and your clothes, and
that through the young lady's death she has no need to treat you as a
relation, she suddenly took fright; and as you hid in your den and
dropped all your old relations with her, she planned to get rid of you.
And she's been cherishing that design a long time, but was sorry to lose
the I O U, for you assured her yourself that your mother would pay.''

``It was base of me to say that\ldots{}. My mother herself is almost a
beggar\ldots{} and I told a lie to keep my lodging\ldots{} and be fed,''
Raskolnikov said loudly and distinctly.

``Yes, you did very sensibly. But the worst of it is that at that point
Mr. Tchebarov turns up, a business man. Pashenka would never have
thought of doing anything on her own account, she is too retiring; but
the business man is by no means retiring, and first thing he puts the
question, `Is there any hope of realising the I O U?' Answer: there is,
because he has a mother who would save her Rodya with her hundred and
twenty-five roubles pension, if she has to starve herself; and a sister,
too, who would go into bondage for his sake. That's what he was building
upon\ldots{}. Why do you start? I know all the ins and outs of your
affairs now, my dear boy---it's not for nothing that you were so open
with Pashenka when you were her prospective son-in-law, and I say all
this as a friend\ldots{}. But I tell you what it is; an honest and
sensitive man is open; and a business man `listens and goes on eating'
you up. Well, then she gave the I O U by way of payment to this
Tchebarov, and without hesitation he made a formal demand for payment.
When I heard of all this I wanted to blow him up, too, to clear my
conscience, but by that time harmony reigned between me and Pashenka,
and I insisted on stopping the whole affair, engaging that you would
pay. I went security for you, brother. Do you understand? We called
Tchebarov, flung him ten roubles and got the I O U back from him, and
here I have the honour of presenting it to you. She trusts your word
now. Here, take it, you see I have torn it.''

Razumihin put the note on the table. Raskolnikov looked at him and
turned to the wall without uttering a word. Even Razumihin felt a
twinge.

``I see, brother,'' he said a moment later, ``that I have been playing
the fool again. I thought I should amuse you with my chatter, and I
believe I have only made you cross.''

``Was it you I did not recognise when I was delirious?'' Raskolnikov
asked, after a moment's pause without turning his head.

``Yes, and you flew into a rage about it, especially when I brought
Zametov one day.''

``Zametov? The head clerk? What for?'' Raskolnikov turned round quickly
and fixed his eyes on Razumihin.

``What's the matter with you?\ldots{} What are you upset about? He
wanted to make your acquaintance because I talked to him a lot about
you\ldots{}. How could I have found out so much except from him? He is a
capital fellow, brother, first-rate\ldots{} in his own way, of course.
Now we are friends---see each other almost every day. I have moved into
this part, you know. I have only just moved. I've been with him to Luise
Ivanovna once or twice\ldots{}. Do you remember Luise, Luise Ivanovna?

``Did I say anything in delirium?''

``I should think so! You were beside yourself.''

``What did I rave about?''

``What next? What did you rave about? What people do rave about\ldots{}.
Well, brother, now I must not lose time. To work.'' He got up from the
table and took up his cap.

``What did I rave about?''

``How he keeps on! Are you afraid of having let out some secret? Don't
worry yourself; you said nothing about a countess. But you said a lot
about a bulldog, and about ear-rings and chains, and about Krestovsky
Island, and some porter, and Nikodim Fomitch and Ilya Petrovitch, the
assistant superintendent. And another thing that was of special interest
to you was your own sock. You whined, `Give me my sock.' Zametov hunted
all about your room for your socks, and with his own scented,
ring-bedecked fingers he gave you the rag. And only then were you
comforted, and for the next twenty-four hours you held the wretched
thing in your hand; we could not get it from you. It is most likely
somewhere under your quilt at this moment. And then you asked so
piteously for fringe for your trousers. We tried to find out what sort
of fringe, but we could not make it out. Now to business! Here are
thirty-five roubles; I take ten of them, and shall give you an account
of them in an hour or two. I will let Zossimov know at the same time,
though he ought to have been here long ago, for it is nearly twelve. And
you, Nastasya, look in pretty often while I am away, to see whether he
wants a drink or anything else. And I will tell Pashenka what is wanted
myself. Good-bye!''

``He calls her Pashenka! Ah, he's a deep one!'' said Nastasya as he went
out; then she opened the door and stood listening, but could not resist
running downstairs after him. She was very eager to hear what he would
say to the landlady. She was evidently quite fascinated by Razumihin.

No sooner had she left the room than the sick man flung off the
bedclothes and leapt out of bed like a madman. With burning, twitching
impatience he had waited for them to be gone so that he might set to
work. But to what work? Now, as though to spite him, it eluded him.

``Good God, only tell me one thing: do they know of it yet or not? What
if they know it and are only pretending, mocking me while I am laid up,
and then they will come in and tell me that it's been discovered long
ago and that they have only\ldots{} What am I to do now? That's what
I've forgotten, as though on purpose; forgotten it all at once, I
remembered a minute ago.''

He stood in the middle of the room and gazed in miserable bewilderment
about him; he walked to the door, opened it, listened; but that was not
what he wanted. Suddenly, as though recalling something, he rushed to
the corner where there was a hole under the paper, began examining it,
put his hand into the hole, fumbled---but that was not it. He went to
the stove, opened it and began rummaging in the ashes; the frayed edges
of his trousers and the rags cut off his pocket were lying there just as
he had thrown them. No one had looked, then! Then he remembered the sock
about which Razumihin had just been telling him. Yes, there it lay on
the sofa under the quilt, but it was so covered with dust and grime that
Zametov could not have seen anything on it.

``Bah, Zametov! The police office! And why am I sent for to the police
office? Where's the notice? Bah! I am mixing it up; that was then. I
looked at my sock then, too, but now\ldots{} now I have been ill. But
what did Zametov come for? Why did Razumihin bring him?'' he muttered,
helplessly sitting on the sofa again. ``What does it mean? Am I still in
delirium, or is it real? I believe it is real\ldots{}. Ah, I remember; I
must escape! Make haste to escape. Yes, I must, I must escape!
Yes\ldots{} but where? And where are my clothes? I've no boots. They've
taken them away! They've hidden them! I understand! Ah, here is my
coat---they passed that over! And here is money on the table, thank God!
And here's the I O U\ldots{} I'll take the money and go and take another
lodging. They won't find me!\ldots{} Yes, but the address bureau?
They'll find me, Razumihin will find me. Better escape
altogether\ldots{} far away\ldots{} to America, and let them do their
worst! And take the I O U\ldots{} it would be of use there\ldots{}. What
else shall I take? They think I am ill! They don't know that I can walk,
ha-ha-ha! I could see by their eyes that they know all about it! If only
I could get downstairs! And what if they have set a watch
there---policemen! What's this tea? Ah, and here is beer left, half a
bottle, cold!''

He snatched up the bottle, which still contained a glassful of beer, and
gulped it down with relish, as though quenching a flame in his breast.
But in another minute the beer had gone to his head, and a faint and
even pleasant shiver ran down his spine. He lay down and pulled the
quilt over him. His sick and incoherent thoughts grew more and more
disconnected, and soon a light, pleasant drowsiness came upon him. With
a sense of comfort he nestled his head into the pillow, wrapped more
closely about him the soft, wadded quilt which had replaced the old,
ragged greatcoat, sighed softly and sank into a deep, sound, refreshing
sleep.

He woke up, hearing someone come in. He opened his eyes and saw
Razumihin standing in the doorway, uncertain whether to come in or not.
Raskolnikov sat up quickly on the sofa and gazed at him, as though
trying to recall something.

``Ah, you are not asleep! Here I am! Nastasya, bring in the parcel!''
Razumihin shouted down the stairs. ``You shall have the account
directly.''

``What time is it?'' asked Raskolnikov, looking round uneasily.

``Yes, you had a fine sleep, brother, it's almost evening, it will be
six o'clock directly. You have slept more than six hours.''

``Good heavens! Have I?''

``And why not? It will do you good. What's the hurry? A tryst, is it?
We've all time before us. I've been waiting for the last three hours for
you; I've been up twice and found you asleep. I've called on Zossimov
twice; not at home, only fancy! But no matter, he will turn up. And I've
been out on my own business, too. You know I've been moving to-day,
moving with my uncle. I have an uncle living with me now. But that's no
matter, to business. Give me the parcel, Nastasya. We will open it
directly. And how do you feel now, brother?''

``I am quite well, I am not ill. Razumihin, have you been here long?''

``I tell you I've been waiting for the last three hours.''

``No, before.''

``How do you mean?''

``How long have you been coming here?''

``Why I told you all about it this morning. Don't you remember?''

Raskolnikov pondered. The morning seemed like a dream to him. He could
not remember alone, and looked inquiringly at Razumihin.

``Hm!'' said the latter, ``he has forgotten. I fancied then that you
were not quite yourself. Now you are better for your sleep\ldots{}. You
really look much better. First-rate! Well, to business. Look here, my
dear boy.''

He began untying the bundle, which evidently interested him.

``Believe me, brother, this is something specially near my heart. For we
must make a man of you. Let's begin from the top. Do you see this cap?''
he said, taking out of the bundle a fairly good though cheap and
ordinary cap. ``Let me try it on.''

``Presently, afterwards,'' said Raskolnikov, waving it off pettishly.

``Come, Rodya, my boy, don't oppose it, afterwards will be too late; and
I shan't sleep all night, for I bought it by guess, without measure.
Just right!'' he cried triumphantly, fitting it on, ``just your size! A
proper head-covering is the first thing in dress and a recommendation in
its own way. Tolstyakov, a friend of mine, is always obliged to take off
his pudding basin when he goes into any public place where other people
wear their hats or caps. People think he does it from slavish
politeness, but it's simply because he is ashamed of his bird's nest; he
is such a boastful fellow! Look, Nastasya, here are two specimens of
headgear: this Palmerston''---he took from the corner Raskolnikov's old,
battered hat, which for some unknown reason, he called a
Palmerston---``or this jewel! Guess the price, Rodya, what do you
suppose I paid for it, Nastasya!'' he said, turning to her, seeing that
Raskolnikov did not speak.

``Twenty copecks, no more, I dare say,'' answered Nastasya.

``Twenty copecks, silly!'' he cried, offended. ``Why, nowadays you would
cost more than that---eighty copecks! And that only because it has been
worn. And it's bought on condition that when's it's worn out, they will
give you another next year. Yes, on my word! Well, now let us pass to
the United States of America, as they called them at school. I assure
you I am proud of these breeches,'' and he exhibited to Raskolnikov a
pair of light, summer trousers of grey woollen material. ``No holes, no
spots, and quite respectable, although a little worn; and a waistcoat to
match, quite in the fashion. And its being worn really is an
improvement, it's softer, smoother\ldots{}. You see, Rodya, to my
thinking, the great thing for getting on in the world is always to keep
to the seasons; if you don't insist on having asparagus in January, you
keep your money in your purse; and it's the same with this purchase.
It's summer now, so I've been buying summer things---warmer materials
will be wanted for autumn, so you will have to throw these away in any
case\ldots{} especially as they will be done for by then from their own
lack of coherence if not your higher standard of luxury. Come, price
them! What do you say? Two roubles twenty-five copecks! And remember the
condition: if you wear these out, you will have another suit for
nothing! They only do business on that system at Fedyaev's; if you've
bought a thing once, you are satisfied for life, for you will never go
there again of your own free will. Now for the boots. What do you say?
You see that they are a bit worn, but they'll last a couple of months,
for it's foreign work and foreign leather; the secretary of the English
Embassy sold them last week---he had only worn them six days, but he was
very short of cash. Price---a rouble and a half. A bargain?''

``But perhaps they won't fit,'' observed Nastasya.

``Not fit? Just look!'' and he pulled out of his pocket Raskolnikov's
old, broken boot, stiffly coated with dry mud. ``I did not go
empty-handed---they took the size from this monster. We all did our
best. And as to your linen, your landlady has seen to that. Here, to
begin with are three shirts, hempen but with a fashionable
front\ldots{}. Well now then, eighty copecks the cap, two roubles
twenty-five copecks the suit---together three roubles five copecks---a
rouble and a half for the boots---for, you see, they are very good---and
that makes four roubles fifty-five copecks; five roubles for the
underclothes---they were bought in the lo---which makes exactly nine
roubles fifty-five copecks. Forty-five copecks change in coppers. Will
you take it? And so, Rodya, you are set up with a complete new rig-out,
for your overcoat will serve, and even has a style of its own. That
comes from getting one's clothes from Sharmer's! As for your socks and
other things, I leave them to you; we've twenty-five roubles left. And
as for Pashenka and paying for your lodging, don't you worry. I tell you
she'll trust you for anything. And now, brother, let me change your
linen, for I daresay you will throw off your illness with your shirt.''

``Let me be! I don't want to!'' Raskolnikov waved him off. He had
listened with disgust to Razumihin's efforts to be playful about his
purchases.

``Come, brother, don't tell me I've been trudging around for nothing,''
Razumihin insisted. ``Nastasya, don't be bashful, but help me---that's
it,'' and in spite of Raskolnikov's resistance he changed his linen. The
latter sank back on the pillows and for a minute or two said nothing.

``It will be long before I get rid of them,'' he thought. ``What money
was all that bought with?'' he asked at last, gazing at the wall.

``Money? Why, your own, what the messenger brought from Vahrushin, your
mother sent it. Have you forgotten that, too?''

``I remember now,'' said Raskolnikov after a long, sullen silence.
Razumihin looked at him, frowning and uneasy.

The door opened and a tall, stout man whose appearance seemed familiar
to Raskolnikov came in.

\mychapter{4}{Chapter IV}\label{chapter-iv-1}

Zossimov was a tall, fat man with a puffy, colourless, clean-shaven face
and straight flaxen hair. He wore spectacles, and a big gold ring on his
fat finger. He was twenty-seven. He had on a light grey fashionable
loose coat, light summer trousers, and everything about him loose,
fashionable and spick and span; his linen was irreproachable, his
watch-chain was massive. In manner he was slow and, as it were,
nonchalant, and at the same time studiously free and easy; he made
efforts to conceal his self-importance, but it was apparent at every
instant. All his acquaintances found him tedious, but said he was clever
at his work.

``I've been to you twice to-day, brother. You see, he's come to
himself,'' cried Razumihin.

``I see, I see; and how do we feel now, eh?'' said Zossimov to
Raskolnikov, watching him carefully and, sitting down at the foot of the
sofa, he settled himself as comfortably as he could.

``He is still depressed,'' Razumihin went on. ``We've just changed his
linen and he almost cried.''

``That's very natural; you might have put it off if he did not wish
it\ldots{}. His pulse is first-rate. Is your head still aching, eh?''

``I am well, I am perfectly well!'' Raskolnikov declared positively and
irritably. He raised himself on the sofa and looked at them with
glittering eyes, but sank back on to the pillow at once and turned to
the wall. Zossimov watched him intently.

``Very good\ldots{}. Going on all right,'' he said lazily. ``Has he
eaten anything?''

They told him, and asked what he might have.

``He may have anything\ldots{} soup, tea\ldots{} mushrooms and
cucumbers, of course, you must not give him; he'd better not have meat
either, and\ldots{} but no need to tell you that!'' Razumihin and he
looked at each other. ``No more medicine or anything. I'll look at him
again to-morrow. Perhaps, to-day even\ldots{} but never mind\ldots{}''

``To-morrow evening I shall take him for a walk,'' said Razumihin. ``We
are going to the Yusupov garden and then to the Palais de Crystal.''

``I would not disturb him to-morrow at all, but I don't know\ldots{} a
little, maybe\ldots{} but we'll see.''

``Ach, what a nuisance! I've got a house-warming party to-night; it's
only a step from here. Couldn't he come? He could lie on the sofa. You
are coming?'' Razumihin said to Zossimov. ``Don't forget, you
promised.''

``All right, only rather later. What are you going to do?''

``Oh, nothing---tea, vodka, herrings. There will be a pie\ldots{} just
our friends.''

``And who?''

``All neighbours here, almost all new friends, except my old uncle, and
he is new too---he only arrived in Petersburg yesterday to see to some
business of his. We meet once in five years.''

``What is he?''

``He's been stagnating all his life as a district postmaster; gets a
little pension. He is sixty-five---not worth talking about\ldots{}. But
I am fond of him. Porfiry Petrovitch, the head of the Investigation
Department here\ldots{} But you know him.''

``Is he a relation of yours, too?''

``A very distant one. But why are you scowling? Because you quarrelled
once, won't you come then?''

``I don't care a damn for him.''

``So much the better. Well, there will be some students, a teacher, a
government clerk, a musician, an officer and Zametov.''

``Do tell me, please, what you or he''---Zossimov nodded at
Raskolnikov---``can have in common with this Zametov?''

``Oh, you particular gentleman! Principles! You are worked by
principles, as it were by springs; you won't venture to turn round on
your own account. If a man is a nice fellow, that's the only principle I
go upon. Zametov is a delightful person.''

``Though he does take bribes.''

``Well, he does! and what of it? I don't care if he does take bribes,''
Razumihin cried with unnatural irritability. ``I don't praise him for
taking bribes. I only say he is a nice man in his own way! But if one
looks at men in all ways---are there many good ones left? Why, I am sure
I shouldn't be worth a baked onion myself\ldots{} perhaps with you
thrown in.''

``That's too little; I'd give two for you.''

``And I wouldn't give more than one for you. No more of your jokes!
Zametov is no more than a boy. I can pull his hair and one must draw him
not repel him. You'll never improve a man by repelling him, especially a
boy. One has to be twice as careful with a boy. Oh, you progressive
dullards! You don't understand. You harm yourselves running another man
down\ldots{}. But if you want to know, we really have something in
common.''

``I should like to know what.''

``Why, it's all about a house-painter\ldots{}. We are getting him out of
a mess! Though indeed there's nothing to fear now. The matter is
absolutely self-evident. We only put on steam.''

``A painter?''

``Why, haven't I told you about it? I only told you the beginning then
about the murder of the old pawnbroker-woman. Well, the painter is mixed
up in it\ldots{}''

``Oh, I heard about that murder before and was rather interested in
it\ldots{} partly\ldots{} for one reason\ldots{}. I read about it in the
papers, too\ldots{}.''

``Lizaveta was murdered, too,'' Nastasya blurted out, suddenly
addressing Raskolnikov. She remained in the room all the time, standing
by the door listening.

``Lizaveta,'' murmured Raskolnikov hardly audibly.

``Lizaveta, who sold old clothes. Didn't you know her? She used to come
here. She mended a shirt for you, too.''

Raskolnikov turned to the wall where in the dirty, yellow paper he
picked out one clumsy, white flower with brown lines on it and began
examining how many petals there were in it, how many scallops in the
petals and how many lines on them. He felt his arms and legs as lifeless
as though they had been cut off. He did not attempt to move, but stared
obstinately at the flower.

``But what about the painter?'' Zossimov interrupted Nastasya's chatter
with marked displeasure. She sighed and was silent.

``Why, he was accused of the murder,'' Razumihin went on hotly.

``Was there evidence against him then?''

``Evidence, indeed! Evidence that was no evidence, and that's what we
have to prove. It was just as they pitched on those fellows, Koch and
Pestryakov, at first. Foo! how stupidly it's all done, it makes one
sick, though it's not one's business! Pestryakov may be coming
to-night\ldots{}. By the way, Rodya, you've heard about the business
already; it happened before you were ill, the day before you fainted at
the police office while they were talking about it.''

Zossimov looked curiously at Raskolnikov. He did not stir.

``But I say, Razumihin, I wonder at you. What a busybody you are!''
Zossimov observed.

``Maybe I am, but we will get him off anyway,'' shouted Razumihin,
bringing his fist down on the table. ``What's the most offensive is not
their lying---one can always forgive lying---lying is a delightful
thing, for it leads to truth---what is offensive is that they lie and
worship their own lying\ldots{}. I respect Porfiry, but\ldots{} What
threw them out at first? The door was locked, and when they came back
with the porter it was open. So it followed that Koch and Pestryakov
were the murderers---that was their logic!''

``But don't excite yourself; they simply detained them, they could not
help that\ldots{}. And, by the way, I've met that man Koch. He used to
buy unredeemed pledges from the old woman? Eh?''

``Yes, he is a swindler. He buys up bad debts, too. He makes a
profession of it. But enough of him! Do you know what makes me angry?
It's their sickening rotten, petrified routine\ldots{}. And this case
might be the means of introducing a new method. One can show from the
psychological data alone how to get on the track of the real man. `We
have facts,' they say. But facts are not everything---at least half the
business lies in how you interpret them!''

``Can you interpret them, then?''

``Anyway, one can't hold one's tongue when one has a feeling, a tangible
feeling, that one might be a help if only\ldots{}. Eh! Do you know the
details of the case?''

``I am waiting to hear about the painter.''

``Oh, yes! Well, here's the story. Early on the third day after the
murder, when they were still dandling Koch and Pestryakov---though they
accounted for every step they took and it was as plain as a
pikestaff---an unexpected fact turned up. A peasant called Dushkin, who
keeps a dram-shop facing the house, brought to the police office a
jeweller's case containing some gold ear-rings, and told a long
rigamarole. `The day before yesterday, just after eight o'clock'---mark
the day and the hour!---`a journeyman house-painter, Nikolay, who had
been in to see me already that day, brought me this box of gold
ear-rings and stones, and asked me to give him two roubles for them.
When I asked him where he got them, he said that he picked them up in
the street. I did not ask him anything more.' I am telling you Dushkin's
story. `I gave him a note'---a rouble that is---`for I thought if he did
not pawn it with me he would with another. It would all come to the same
thing---he'd spend it on drink, so the thing had better be with me. The
further you hide it the quicker you will find it, and if anything turns
up, if I hear any rumours, I'll take it to the police.' Of course,
that's all taradiddle; he lies like a horse, for I know this Dushkin, he
is a pawnbroker and a receiver of stolen goods, and he did not cheat
Nikolay out of a thirty-rouble trinket in order to give it to the
police. He was simply afraid. But no matter, to return to Dushkin's
story. `I've known this peasant, Nikolay Dementyev, from a child; he
comes from the same province and district of Zaraïsk, we are both Ryazan
men. And though Nikolay is not a drunkard, he drinks, and I knew he had
a job in that house, painting work with Dmitri, who comes from the same
village, too. As soon as he got the rouble he changed it, had a couple
of glasses, took his change and went out. But I did not see Dmitri with
him then. And the next day I heard that someone had murdered Alyona
Ivanovna and her sister, Lizaveta Ivanovna, with an axe. I knew them,
and I felt suspicious about the ear-rings at once, for I knew the
murdered woman lent money on pledges. I went to the house, and began to
make careful inquiries without saying a word to anyone. First of all I
asked, ``Is Nikolay here?'' Dmitri told me that Nikolay had gone off on
the spree; he had come home at daybreak drunk, stayed in the house about
ten minutes, and went out again. Dmitri didn't see him again and is
finishing the job alone. And their job is on the same staircase as the
murder, on the second floor. When I heard all that I did not say a word
to anyone'---that's Dushkin's tale---`but I found out what I could about
the murder, and went home feeling as suspicious as ever. And at eight
o'clock this morning'---that was the third day, you understand---`I saw
Nikolay coming in, not sober, though not to say very drunk---he could
understand what was said to him. He sat down on the bench and did not
speak. There was only one stranger in the bar and a man I knew asleep on
a bench and our two boys. ``Have you seen Dmitri?'' said I. ``No, I
haven't,'' said he. ``And you've not been here either?'' ``Not since the
day before yesterday,'' said he. ``And where did you sleep last night?''
``In Peski, with the Kolomensky men.'' ``And where did you get those
ear-rings?'' I asked. ``I found them in the street,'' and the way he
said it was a bit queer; he did not look at me. ``Did you hear what
happened that very evening, at that very hour, on that same staircase?''
said I. ``No,'' said he, ``I had not heard,'' and all the while he was
listening, his eyes were staring out of his head and he turned as white
as chalk. I told him all about it and he took his hat and began getting
up. I wanted to keep him. ``Wait a bit, Nikolay,'' said I, ``won't you
have a drink?'' And I signed to the boy to hold the door, and I came out
from behind the bar; but he darted out and down the street to the
turning at a run. I have not seen him since. Then my doubts were at an
end---it was his doing, as clear as could be\ldots{}.'''

``I should think so,'' said Zossimov.

``Wait! Hear the end. Of course they sought high and low for Nikolay;
they detained Dushkin and searched his house; Dmitri, too, was arrested;
the Kolomensky men also were turned inside out. And the day before
yesterday they arrested Nikolay in a tavern at the end of the town. He
had gone there, taken the silver cross off his neck and asked for a dram
for it. They gave it to him. A few minutes afterwards the woman went to
the cowshed, and through a crack in the wall she saw in the stable
adjoining he had made a noose of his sash from the beam, stood on a
block of wood, and was trying to put his neck in the noose. The woman
screeched her hardest; people ran in. `So that's what you are up to!'
`Take me,' he says, `to such-and-such a police officer; I'll confess
everything.' Well, they took him to that police station---that is
here---with a suitable escort. So they asked him this and that, how old
he is, `twenty-two,' and so on. At the question, `When you were working
with Dmitri, didn't you see anyone on the staircase at such-and-such a
time?'---answer: `To be sure folks may have gone up and down, but I did
not notice them.' `And didn't you hear anything, any noise, and so on?'
`We heard nothing special.' `And did you hear, Nikolay, that on the same
day Widow So-and-so and her sister were murdered and robbed?' `I never
knew a thing about it. The first I heard of it was from Afanasy
Pavlovitch the day before yesterday.' `And where did you find the
ear-rings?' `I found them on the pavement.' `Why didn't you go to work
with Dmitri the other day?' `Because I was drinking.' `And where were
you drinking?' `Oh, in such-and-such a place.' `Why did you run away
from Dushkin's?' `Because I was awfully frightened.' `What were you
frightened of?' `That I should be accused.' `How could you be
frightened, if you felt free from guilt?' Now, Zossimov, you may not
believe me, that question was put literally in those words. I know it
for a fact, it was repeated to me exactly! What do you say to that?''

``Well, anyway, there's the evidence.''

``I am not talking of the evidence now, I am talking about that
question, of their own idea of themselves. Well, so they squeezed and
squeezed him and he confessed: `I did not find it in the street, but in
the flat where I was painting with Dmitri.' `And how was that?' `Why,
Dmitri and I were painting there all day, and we were just getting ready
to go, and Dmitri took a brush and painted my face, and he ran off and I
after him. I ran after him, shouting my hardest, and at the bottom of
the stairs I ran right against the porter and some gentlemen---and how
many gentlemen were there I don't remember. And the porter swore at me,
and the other porter swore, too, and the porter's wife came out, and
swore at us, too; and a gentleman came into the entry with a lady, and
he swore at us, too, for Dmitri and I lay right across the way. I got
hold of Dmitri's hair and knocked him down and began beating him. And
Dmitri, too, caught me by the hair and began beating me. But we did it
all not for temper but in a friendly way, for sport. And then Dmitri
escaped and ran into the street, and I ran after him; but I did not
catch him, and went back to the flat alone; I had to clear up my things.
I began putting them together, expecting Dmitri to come, and there in
the passage, in the corner by the door, I stepped on the box. I saw it
lying there wrapped up in paper. I took off the paper, saw some little
hooks, undid them, and in the box were the ear-rings\ldots{}.'\,''

``Behind the door? Lying behind the door? Behind the door?'' Raskolnikov
cried suddenly, staring with a blank look of terror at Razumihin, and he
slowly sat up on the sofa, leaning on his hand.

``Yes\ldots{} why? What's the matter? What's wrong?'' Razumihin, too,
got up from his seat.

``Nothing,'' Raskolnikov answered faintly, turning to the wall. All were
silent for a while.

``He must have waked from a dream,'' Razumihin said at last, looking
inquiringly at Zossimov. The latter slightly shook his head.

``Well, go on,'' said Zossimov. ``What next?''

``What next? As soon as he saw the ear-rings, forgetting Dmitri and
everything, he took up his cap and ran to Dushkin and, as we know, got a
rouble from him. He told a lie saying he found them in the street, and
went off drinking. He keeps repeating his old story about the murder: `I
know nothing of it, never heard of it till the day before yesterday.'
`And why didn't you come to the police till now?' `I was frightened.'
`And why did you try to hang yourself?' `From anxiety.' `What anxiety?'
`That I should be accused of it.' Well, that's the whole story. And now
what do you suppose they deduced from that?''

``Why, there's no supposing. There's a clue, such as it is, a fact. You
wouldn't have your painter set free?''

``Now they've simply taken him for the murderer. They haven't a shadow
of doubt.''

``That's nonsense. You are excited. But what about the ear-rings? You
must admit that, if on the very same day and hour ear-rings from the old
woman's box have come into Nikolay's hands, they must have come there
somehow. That's a good deal in such a case.''

``How did they get there? How did they get there?'' cried Razumihin.
``How can you, a doctor, whose duty it is to study man and who has more
opportunity than anyone else for studying human nature---how can you
fail to see the character of the man in the whole story? Don't you see
at once that the answers he has given in the examination are the holy
truth? They came into his hand precisely as he has told us---he stepped
on the box and picked it up.''

``The holy truth! But didn't he own himself that he told a lie at
first?''

``Listen to me, listen attentively. The porter and Koch and Pestryakov
and the other porter and the wife of the first porter and the woman who
was sitting in the porter's lodge and the man Kryukov, who had just got
out of a cab at that minute and went in at the entry with a lady on his
arm, that is eight or ten witnesses, agree that Nikolay had Dmitri on
the ground, was lying on him beating him, while Dmitri hung on to his
hair, beating him, too. They lay right across the way, blocking the
thoroughfare. They were sworn at on all sides while they `like children'
(the very words of the witnesses) were falling over one another,
squealing, fighting and laughing with the funniest faces, and, chasing
one another like children, they ran into the street. Now take careful
note. The bodies upstairs were warm, you understand, warm when they
found them! If they, or Nikolay alone, had murdered them and broken open
the boxes, or simply taken part in the robbery, allow me to ask you one
question: do their state of mind, their squeals and giggles and childish
scuffling at the gate fit in with axes, bloodshed, fiendish cunning,
robbery? They'd just killed them, not five or ten minutes before, for
the bodies were still warm, and at once, leaving the flat open, knowing
that people would go there at once, flinging away their booty, they
rolled about like children, laughing and attracting general attention.
And there are a dozen witnesses to swear to that!''

``Of course it is strange! It's impossible, indeed, but\ldots{}''

``No, brother, no \emph{buts}. And if the ear-rings being found in
Nikolay's hands at the very day and hour of the murder constitutes an
important piece of circumstantial evidence against him---although the
explanation given by him accounts for it, and therefore it does not tell
seriously against him---one must take into consideration the facts which
prove him innocent, especially as they are facts that \emph{cannot be
denied}. And do you suppose, from the character of our legal system,
that they will accept, or that they are in a position to accept, this
fact---resting simply on a psychological impossibility---as irrefutable
and conclusively breaking down the circumstantial evidence for the
prosecution? No, they won't accept it, they certainly won't, because
they found the jewel-case and the man tried to hang himself, `which he
could not have done if he hadn't felt guilty.' That's the point, that's
what excites me, you must understand!''

``Oh, I see you are excited! Wait a bit. I forgot to ask you; what proof
is there that the box came from the old woman?''

``That's been proved,'' said Razumihin with apparent reluctance,
frowning. ``Koch recognised the jewel-case and gave the name of the
owner, who proved conclusively that it was his.''

``That's bad. Now another point. Did anyone see Nikolay at the time that
Koch and Pestryakov were going upstairs at first, and is there no
evidence about that?''

``Nobody did see him,'' Razumihin answered with vexation. ``That's the
worst of it. Even Koch and Pestryakov did not notice them on their way
upstairs, though, indeed, their evidence could not have been worth much.
They said they saw the flat was open, and that there must be work going
on in it, but they took no special notice and could not remember whether
there actually were men at work in it.''

``Hm!\ldots{} So the only evidence for the defence is that they were
beating one another and laughing. That constitutes a strong presumption,
but\ldots{} How do you explain the facts yourself?''

``How do I explain them? What is there to explain? It's clear. At any
rate, the direction in which explanation is to be sought is clear, and
the jewel-case points to it. The real murderer dropped those ear-rings.
The murderer was upstairs, locked in, when Koch and Pestryakov knocked
at the door. Koch, like an ass, did not stay at the door; so the
murderer popped out and ran down, too; for he had no other way of
escape. He hid from Koch, Pestryakov and the porter in the flat when
Nikolay and Dmitri had just run out of it. He stopped there while the
porter and others were going upstairs, waited till they were out of
hearing, and then went calmly downstairs at the very minute when Dmitri
and Nikolay ran out into the street and there was no one in the entry;
possibly he was seen, but not noticed. There are lots of people going in
and out. He must have dropped the ear-rings out of his pocket when he
stood behind the door, and did not notice he dropped them, because he
had other things to think of. The jewel-case is a conclusive proof that
he did stand there\ldots{}. That's how I explain it.''

``Too clever! No, my boy, you're too clever. That beats everything.''

``But, why, why?''

``Why, because everything fits too well\ldots{} it's too melodramatic.''

``A-ach!'' Razumihin was exclaiming, but at that moment the door opened
and a personage came in who was a stranger to all present.

\protect\hypertarget{link2HCH0012}{}{}

\mychapter{5}{Chapter V}\label{chapter-v-1}

This was a gentleman no longer young, of a stiff and portly appearance,
and a cautious and sour countenance. He began by stopping short in the
doorway, staring about him with offensive and undisguised astonishment,
as though asking himself what sort of place he had come to.
Mistrustfully and with an affectation of being alarmed and almost
affronted, he scanned Raskolnikov's low and narrow ``cabin.'' With the
same amazement he stared at Raskolnikov, who lay undressed, dishevelled,
unwashed, on his miserable dirty sofa, looking fixedly at him. Then with
the same deliberation he scrutinised the uncouth, unkempt figure and
unshaven face of Razumihin, who looked him boldly and inquiringly in the
face without rising from his seat. A constrained silence lasted for a
couple of minutes, and then, as might be expected, some scene-shifting
took place. Reflecting, probably from certain fairly unmistakable signs,
that he would get nothing in this ``cabin'' by attempting to overawe
them, the gentleman softened somewhat, and civilly, though with some
severity, emphasising every syllable of his question, addressed
Zossimov:

``Rodion Romanovitch Raskolnikov, a student, or formerly a student?''

Zossimov made a slight movement, and would have answered, had not
Razumihin anticipated him.

``Here he is lying on the sofa! What do you want?''

This familiar ``what do you want'' seemed to cut the ground from the
feet of the pompous gentleman. He was turning to Razumihin, but checked
himself in time and turned to Zossimov again.

``This is Raskolnikov,'' mumbled Zossimov, nodding towards him. Then he
gave a prolonged yawn, opening his mouth as wide as possible. Then he
lazily put his hand into his waistcoat-pocket, pulled out a huge gold
watch in a round hunter's case, opened it, looked at it and as slowly
and lazily proceeded to put it back.

Raskolnikov himself lay without speaking, on his back, gazing
persistently, though without understanding, at the stranger. Now that
his face was turned away from the strange flower on the paper, it was
extremely pale and wore a look of anguish, as though he had just
undergone an agonising operation or just been taken from the rack. But
the new-comer gradually began to arouse his attention, then his wonder,
then suspicion and even alarm. When Zossimov said ``This is
Raskolnikov'' he jumped up quickly, sat on the sofa and with an almost
defiant, but weak and breaking, voice articulated:

``Yes, I am Raskolnikov! What do you want?''

The visitor scrutinised him and pronounced impressively:

``Pyotr Petrovitch Luzhin. I believe I have reason to hope that my name
is not wholly unknown to you?''

But Raskolnikov, who had expected something quite different, gazed
blankly and dreamily at him, making no reply, as though he heard the
name of Pyotr Petrovitch for the first time.

``Is it possible that you can up to the present have received no
information?'' asked Pyotr Petrovitch, somewhat disconcerted.

In reply Raskolnikov sank languidly back on the pillow, put his hands
behind his head and gazed at the ceiling. A look of dismay came into
Luzhin's face. Zossimov and Razumihin stared at him more inquisitively
than ever, and at last he showed unmistakable signs of embarrassment.

``I had presumed and calculated,'' he faltered, ``that a letter posted
more than ten days, if not a fortnight ago\ldots{}''

``I say, why are you standing in the doorway?'' Razumihin interrupted
suddenly. ``If you've something to say, sit down. Nastasya and you are
so crowded. Nastasya, make room. Here's a chair, thread your way in!''

He moved his chair back from the table, made a little space between the
table and his knees, and waited in a rather cramped position for the
visitor to ``thread his way in.'' The minute was so chosen that it was
impossible to refuse, and the visitor squeezed his way through, hurrying
and stumbling. Reaching the chair, he sat down, looking suspiciously at
Razumihin.

``No need to be nervous,'' the latter blurted out. ``Rodya has been ill
for the last five days and delirious for three, but now he is recovering
and has got an appetite. This is his doctor, who has just had a look at
him. I am a comrade of Rodya's, like him, formerly a student, and now I
am nursing him; so don't you take any notice of us, but go on with your
business.''

``Thank you. But shall I not disturb the invalid by my presence and
conversation?'' Pyotr Petrovitch asked of Zossimov.

``N-no,'' mumbled Zossimov; ``you may amuse him.'' He yawned again.

``He has been conscious a long time, since the morning,'' went on
Razumihin, whose familiarity seemed so much like unaffected good-nature
that Pyotr Petrovitch began to be more cheerful, partly, perhaps,
because this shabby and impudent person had introduced himself as a
student.

``Your mamma,'' began Luzhin.

``Hm!'' Razumihin cleared his throat loudly. Luzhin looked at him
inquiringly.

``That's all right, go on.''

Luzhin shrugged his shoulders.

``Your mamma had commenced a letter to you while I was sojourning in her
neighbourhood. On my arrival here I purposely allowed a few days to
elapse before coming to see you, in order that I might be fully assured
that you were in full possession of the tidings; but now, to my
astonishment\ldots{}''

``I know, I know!'' Raskolnikov cried suddenly with impatient vexation.
``So you are the \emph{fiancé}? I know, and that's enough!''

There was no doubt about Pyotr Petrovitch's being offended this time,
but he said nothing. He made a violent effort to understand what it all
meant. There was a moment's silence.

Meanwhile Raskolnikov, who had turned a little towards him when he
answered, began suddenly staring at him again with marked curiosity, as
though he had not had a good look at him yet, or as though something new
had struck him; he rose from his pillow on purpose to stare at him.
There certainly was something peculiar in Pyotr Petrovitch's whole
appearance, something which seemed to justify the title of ``fiancé'' so
unceremoniously applied to him. In the first place, it was evident, far
too much so indeed, that Pyotr Petrovitch had made eager use of his few
days in the capital to get himself up and rig himself out in expectation
of his betrothed---a perfectly innocent and permissible proceeding,
indeed. Even his own, perhaps too complacent, consciousness of the
agreeable improvement in his appearance might have been forgiven in such
circumstances, seeing that Pyotr Petrovitch had taken up the rôle of
fiancé. All his clothes were fresh from the tailor's and were all right,
except for being too new and too distinctly appropriate. Even the
stylish new round hat had the same significance. Pyotr Petrovitch
treated it too respectfully and held it too carefully in his hands. The
exquisite pair of lavender gloves, real Louvain, told the same tale, if
only from the fact of his not wearing them, but carrying them in his
hand for show. Light and youthful colours predominated in Pyotr
Petrovitch's attire. He wore a charming summer jacket of a fawn shade,
light thin trousers, a waistcoat of the same, new and fine linen, a
cravat of the lightest cambric with pink stripes on it, and the best of
it was, this all suited Pyotr Petrovitch. His very fresh and even
handsome face looked younger than his forty-five years at all times. His
dark, mutton-chop whiskers made an agreeable setting on both sides,
growing thickly upon his shining, clean-shaven chin. Even his hair,
touched here and there with grey, though it had been combed and curled
at a hairdresser's, did not give him a stupid appearance, as curled hair
usually does, by inevitably suggesting a German on his wedding-day. If
there really was something unpleasing and repulsive in his rather
good-looking and imposing countenance, it was due to quite other causes.
After scanning Mr. Luzhin unceremoniously, Raskolnikov smiled
malignantly, sank back on the pillow and stared at the ceiling as
before.

But Mr. Luzhin hardened his heart and seemed to determine to take no
notice of their oddities.

``I feel the greatest regret at finding you in this situation,'' he
began, again breaking the silence with an effort. ``If I had been aware
of your illness I should have come earlier. But you know what business
is. I have, too, a very important legal affair in the Senate, not to
mention other preoccupations which you may well conjecture. I am
expecting your mamma and sister any minute.''

Raskolnikov made a movement and seemed about to speak; his face showed
some excitement. Pyotr Petrovitch paused, waited, but as nothing
followed, he went on:

``\ldots{} Any minute. I have found a lodging for them on their
arrival.''

``Where?'' asked Raskolnikov weakly.

``Very near here, in Bakaleyev's house.''

``That's in Voskresensky,'' put in Razumihin. ``There are two storeys of
rooms, let by a merchant called Yushin; I've been there.''

``Yes, rooms\ldots{}''

``A disgusting place---filthy, stinking and, what's more, of doubtful
character. Things have happened there, and there are all sorts of queer
people living there. And I went there about a scandalous business. It's
cheap, though\ldots{}''

``I could not, of course, find out so much about it, for I am a stranger
in Petersburg myself,'' Pyotr Petrovitch replied huffily. ``However, the
two rooms are exceedingly clean, and as it is for so short a
time\ldots{} I have already taken a permanent, that is, our future
flat,'' he said, addressing Raskolnikov, ``and I am having it done up.
And meanwhile I am myself cramped for room in a lodging with my friend
Andrey Semyonovitch Lebeziatnikov, in the flat of Madame Lippevechsel;
it was he who told me of Bakaleyev's house, too\ldots{}''

``Lebeziatnikov?'' said Raskolnikov slowly, as if recalling something.

``Yes, Andrey Semyonovitch Lebeziatnikov, a clerk in the Ministry. Do
you know him?''

``Yes\ldots{} no,'' Raskolnikov answered.

``Excuse me, I fancied so from your inquiry. I was once his
guardian\ldots{}. A very nice young man and advanced. I like to meet
young people: one learns new things from them.'' Luzhin looked round
hopefully at them all.

``How do you mean?'' asked Razumihin.

``In the most serious and essential matters,'' Pyotr Petrovitch replied,
as though delighted at the question. ``You see, it's ten years since I
visited Petersburg. All the novelties, reforms, ideas have reached us in
the provinces, but to see it all more clearly one must be in Petersburg.
And it's my notion that you observe and learn most by watching the
younger generation. And I confess I am delighted\ldots{}''

``At what?''

``Your question is a wide one. I may be mistaken, but I fancy I find
clearer views, more, so to say, criticism, more practicality\ldots{}''

``That's true,'' Zossimov let drop.

``Nonsense! There's no practicality.'' Razumihin flew at him.
``Practicality is a difficult thing to find; it does not drop down from
heaven. And for the last two hundred years we have been divorced from
all practical life. Ideas, if you like, are fermenting,'' he said to
Pyotr Petrovitch, ``and desire for good exists, though it's in a
childish form, and honesty you may find, although there are crowds of
brigands. Anyway, there's no practicality. Practicality goes well
shod.''

``I don't agree with you,'' Pyotr Petrovitch replied, with evident
enjoyment. ``Of course, people do get carried away and make mistakes,
but one must have indulgence; those mistakes are merely evidence of
enthusiasm for the cause and of abnormal external environment. If little
has been done, the time has been but short; of means I will not speak.
It's my personal view, if you care to know, that something has been
accomplished already. New valuable ideas, new valuable works are
circulating in the place of our old dreamy and romantic authors.
Literature is taking a maturer form, many injurious prejudices have been
rooted up and turned into ridicule\ldots{}. In a word, we have cut
ourselves off irrevocably from the past, and that, to my thinking, is a
great thing\ldots{}''

``He's learnt it by heart to show off!'' Raskolnikov pronounced
suddenly.

``What?'' asked Pyotr Petrovitch, not catching his words; but he
received no reply.

``That's all true,'' Zossimov hastened to interpose.

``Isn't it so?'' Pyotr Petrovitch went on, glancing affably at Zossimov.
``You must admit,'' he went on, addressing Razumihin with a shade of
triumph and superciliousness---he almost added ``young man''---``that
there is an advance, or, as they say now, progress in the name of
science and economic truth\ldots{}''

``A commonplace.''

``No, not a commonplace! Hitherto, for instance, if I were told, `love
thy neighbour,' what came of it?'' Pyotr Petrovitch went on, perhaps
with excessive haste. ``It came to my tearing my coat in half to share
with my neighbour and we both were left half naked. As a Russian proverb
has it, `Catch several hares and you won't catch one.' Science now tells
us, love yourself before all men, for everything in the world rests on
self-interest. You love yourself and manage your own affairs properly
and your coat remains whole. Economic truth adds that the better private
affairs are organised in society---the more whole coats, so to say---the
firmer are its foundations and the better is the common welfare
organised too. Therefore, in acquiring wealth solely and exclusively for
myself, I am acquiring, so to speak, for all, and helping to bring to
pass my neighbour's getting a little more than a torn coat; and that not
from private, personal liberality, but as a consequence of the general
advance. The idea is simple, but unhappily it has been a long time
reaching us, being hindered by idealism and sentimentality. And yet it
would seem to want very little wit to perceive it\ldots{}''

``Excuse me, I've very little wit myself,'' Razumihin cut in sharply,
``and so let us drop it. I began this discussion with an object, but
I've grown so sick during the last three years of this chattering to
amuse oneself, of this incessant flow of commonplaces, always the same,
that, by Jove, I blush even when other people talk like that. You are in
a hurry, no doubt, to exhibit your acquirements; and I don't blame you,
that's quite pardonable. I only wanted to find out what sort of man you
are, for so many unscrupulous people have got hold of the progressive
cause of late and have so distorted in their own interests everything
they touched, that the whole cause has been dragged in the mire. That's
enough!''

``Excuse me, sir,'' said Luzhin, affronted, and speaking with excessive
dignity. ``Do you mean to suggest so unceremoniously that I
too\ldots{}''

``Oh, my dear sir\ldots{} how could I?\ldots{} Come, that's enough,''
Razumihin concluded, and he turned abruptly to Zossimov to continue
their previous conversation.

Pyotr Petrovitch had the good sense to accept the disavowal. He made up
his mind to take leave in another minute or two.

``I trust our acquaintance,'' he said, addressing Raskolnikov, ``may,
upon your recovery and in view of the circumstances of which you are
aware, become closer\ldots{} Above all, I hope for your return to
health\ldots{}''

Raskolnikov did not even turn his head. Pyotr Petrovitch began getting
up from his chair.

``One of her customers must have killed her,'' Zossimov declared
positively.

``Not a doubt of it,'' replied Razumihin. ``Porfiry doesn't give his
opinion, but is examining all who have left pledges with her there.''

``Examining them?'' Raskolnikov asked aloud.

``Yes. What then?''

``Nothing.''

``How does he get hold of them?'' asked Zossimov.

``Koch has given the names of some of them, other names are on the
wrappers of the pledges and some have come forward of themselves.''

``It must have been a cunning and practised ruffian! The boldness of it!
The coolness!''

``That's just what it wasn't!'' interposed Razumihin. ``That's what
throws you all off the scent. But I maintain that he is not cunning, not
practised, and probably this was his first crime! The supposition that
it was a calculated crime and a cunning criminal doesn't work. Suppose
him to have been inexperienced, and it's clear that it was only a chance
that saved him---and chance may do anything. Why, he did not foresee
obstacles, perhaps! And how did he set to work? He took jewels worth ten
or twenty roubles, stuffing his pockets with them, ransacked the old
woman's trunks, her rags---and they found fifteen hundred roubles,
besides notes, in a box in the top drawer of the chest! He did not know
how to rob; he could only murder. It was his first crime, I assure you,
his first crime; he lost his head. And he got off more by luck than good
counsel!''

``You are talking of the murder of the old pawnbroker, I believe?''
Pyotr Petrovitch put in, addressing Zossimov. He was standing, hat and
gloves in hand, but before departing he felt disposed to throw off a few
more intellectual phrases. He was evidently anxious to make a favourable
impression and his vanity overcame his prudence.

``Yes. You've heard of it?''

``Oh, yes, being in the neighbourhood.''

``Do you know the details?''

``I can't say that; but another circumstance interests me in the
case---the whole question, so to say. Not to speak of the fact that
crime has been greatly on the increase among the lower classes during
the last five years, not to speak of the cases of robbery and arson
everywhere, what strikes me as the strangest thing is that in the higher
classes, too, crime is increasing proportionately. In one place one
hears of a student's robbing the mail on the high road; in another place
people of good social position forge false banknotes; in Moscow of late
a whole gang has been captured who used to forge lottery tickets, and
one of the ringleaders was a lecturer in universal history; then our
secretary abroad was murdered from some obscure motive of gain\ldots{}.
And if this old woman, the pawnbroker, has been murdered by someone of a
higher class in society---for peasants don't pawn gold trinkets---how
are we to explain this demoralisation of the civilised part of our
society?''

``There are many economic changes,'' put in Zossimov.

``How are we to explain it?'' Razumihin caught him up. ``It might be
explained by our inveterate impracticality.''

``How do you mean?''

``What answer had your lecturer in Moscow to make to the question why he
was forging notes? `Everybody is getting rich one way or another, so I
want to make haste to get rich too.' I don't remember the exact words,
but the upshot was that he wants money for nothing, without waiting or
working! We've grown used to having everything ready-made, to walking on
crutches, to having our food chewed for us. Then the great hour
struck,\footnote{\emph{Translator's Note}—The emancipation of the serfs in 1861 is meant.} and every man showed himself in his true colours.''

``But morality? And so to speak, principles\ldots{}''

``But why do you worry about it?'' Raskolnikov interposed suddenly.
``It's in accordance with your theory!''

``In accordance with my theory?''

``Why, carry out logically the theory you were advocating just now, and
it follows that people may be killed\ldots{}''

``Upon my word!'' cried Luzhin.

``No, that's not so,'' put in Zossimov.

Raskolnikov lay with a white face and twitching upper lip, breathing
painfully.

``There's a measure in all things,'' Luzhin went on superciliously.
``Economic ideas are not an incitement to murder, and one has but to
suppose\ldots{}''

``And is it true,'' Raskolnikov interposed once more suddenly, again in
a voice quivering with fury and delight in insulting him, ``is it true
that you told your \emph{fiancée}\ldots{} within an hour of her
acceptance, that what pleased you most\ldots{} was that she was a
beggar\ldots{} because it was better to raise a wife from poverty, so
that you may have complete control over her, and reproach her with your
being her benefactor?''

``Upon my word,'' Luzhin cried wrathfully and irritably, crimson with
confusion, ``to distort my words in this way! Excuse me, allow me to
assure you that the report which has reached you, or rather, let me say,
has been conveyed to you, has no foundation in truth, and I\ldots{}
suspect who\ldots{} in a word\ldots{} this arrow\ldots{} in a word, your
mamma\ldots{} She seemed to me in other things, with all her excellent
qualities, of a somewhat high-flown and romantic way of
thinking\ldots{}. But I was a thousand miles from supposing that she
would misunderstand and misrepresent things in so fanciful a
way\ldots{}. And indeed\ldots{} indeed\ldots{}''

``I tell you what,'' cried Raskolnikov, raising himself on his pillow
and fixing his piercing, glittering eyes upon him, ``I tell you what.''

``What?'' Luzhin stood still, waiting with a defiant and offended face.
Silence lasted for some seconds.

``Why, if ever again\ldots{} you dare to mention a single word\ldots{}
about my mother\ldots{} I shall send you flying downstairs!''

``What's the matter with you?'' cried Razumihin.

``So that's how it is?'' Luzhin turned pale and bit his lip. ``Let me
tell you, sir,'' he began deliberately, doing his utmost to restrain
himself but breathing hard, ``at the first moment I saw you you were
ill-disposed to me, but I remained here on purpose to find out more. I
could forgive a great deal in a sick man and a connection, but
you\ldots{} never after this\ldots{}''

``I am not ill,'' cried Raskolnikov.

``So much the worse\ldots{}''

``Go to hell!''

But Luzhin was already leaving without finishing his speech, squeezing
between the table and the chair; Razumihin got up this time to let him
pass. Without glancing at anyone, and not even nodding to Zossimov, who
had for some time been making signs to him to let the sick man alone, he
went out, lifting his hat to the level of his shoulders to avoid
crushing it as he stooped to go out of the door. And even the curve of
his spine was expressive of the horrible insult he had received.

``How could you---how could you!'' Razumihin said, shaking his head in
perplexity.

``Let me alone---let me alone all of you!'' Raskolnikov cried in a
frenzy. ``Will you ever leave off tormenting me? I am not afraid of you!
I am not afraid of anyone, anyone now! Get away from me! I want to be
alone, alone, alone!''

``Come along,'' said Zossimov, nodding to Razumihin.

``But we can't leave him like this!''

``Come along,'' Zossimov repeated insistently, and he went out.
Razumihin thought a minute and ran to overtake him.

``It might be worse not to obey him,'' said Zossimov on the stairs. ``He
mustn't be irritated.''

``What's the matter with him?''

``If only he could get some favourable shock, that's what would do it!
At first he was better\ldots{}. You know he has got something on his
mind! Some fixed idea weighing on him\ldots{}. I am very much afraid so;
he must have!''

``Perhaps it's that gentleman, Pyotr Petrovitch. From his conversation I
gather he is going to marry his sister, and that he had received a
letter about it just before his illness\ldots{}.''

``Yes, confound the man! he may have upset the case altogether. But have
you noticed, he takes no interest in anything, he does not respond to
anything except one point on which he seems excited---that's the
murder?''

``Yes, yes,'' Razumihin agreed, ``I noticed that, too. He is interested,
frightened. It gave him a shock on the day he was ill in the police
office; he fainted.''

``Tell me more about that this evening and I'll tell you something
afterwards. He interests me very much! In half an hour I'll go and see
him again\ldots{}. There'll be no inflammation though.''

``Thanks! And I'll wait with Pashenka meantime and will keep watch on
him through Nastasya\ldots{}.''

Raskolnikov, left alone, looked with impatience and misery at Nastasya,
but she still lingered.

``Won't you have some tea now?'' she asked.

``Later! I am sleepy! Leave me.''

He turned abruptly to the wall; Nastasya went out.

\protect\hypertarget{link2HCH0013}{}{}

\mychapter{6}{Chapter VI}\label{chapter-vi-1}

But as soon as she went out, he got up, latched the door, undid the
parcel which Razumihin had brought in that evening and had tied up again
and began dressing. Strange to say, he seemed immediately to have become
perfectly calm; not a trace of his recent delirium nor of the panic fear
that had haunted him of late. It was the first moment of a strange
sudden calm. His movements were precise and definite; a firm purpose was
evident in them. ``To-day, to-day,'' he muttered to himself. He
understood that he was still weak, but his intense spiritual
concentration gave him strength and self-confidence. He hoped, moreover,
that he would not fall down in the street. When he had dressed in
entirely new clothes, he looked at the money lying on the table, and
after a moment's thought put it in his pocket. It was twenty-five
roubles. He took also all the copper change from the ten roubles spent
by Razumihin on the clothes. Then he softly unlatched the door, went
out, slipped downstairs and glanced in at the open kitchen door.
Nastasya was standing with her back to him, blowing up the landlady's
samovar. She heard nothing. Who would have dreamed of his going out,
indeed? A minute later he was in the street.

It was nearly eight o'clock, the sun was setting. It was as stifling as
before, but he eagerly drank in the stinking, dusty town air. His head
felt rather dizzy; a sort of savage energy gleamed suddenly in his
feverish eyes and his wasted, pale and yellow face. He did not know and
did not think where he was going, he had one thought only: ``that all
\emph{this} must be ended to-day, once for all, immediately; that he
would not return home without it, because he \emph{would not go on
living like that}.'' How, with what to make an end? He had not an idea
about it, he did not even want to think of it. He drove away thought;
thought tortured him. All he knew, all he felt was that everything must
be changed ``one way or another,'' he repeated with desperate and
immovable self-confidence and determination.

From old habit he took his usual walk in the direction of the Hay
Market. A dark-haired young man with a barrel organ was standing in the
road in front of a little general shop and was grinding out a very
sentimental song. He was accompanying a girl of fifteen, who stood on
the pavement in front of him. She was dressed up in a crinoline, a
mantle and a straw hat with a flame-coloured feather in it, all very old
and shabby. In a strong and rather agreeable voice, cracked and
coarsened by street singing, she sang in hope of getting a copper from
the shop. Raskolnikov joined two or three listeners, took out a five
copeck piece and put it in the girl's hand. She broke off abruptly on a
sentimental high note, shouted sharply to the organ grinder ``Come on,''
and both moved on to the next shop.

``Do you like street music?'' said Raskolnikov, addressing a middle-aged
man standing idly by him. The man looked at him, startled and wondering.

``I love to hear singing to a street organ,'' said Raskolnikov, and his
manner seemed strangely out of keeping with the subject---``I like it on
cold, dark, damp autumn evenings---they must be damp---when all the
passers-by have pale green, sickly faces, or better still when wet snow
is falling straight down, when there's no wind---you know what I
mean?---and the street lamps shine through it\ldots{}''

``I don't know\ldots{}. Excuse me\ldots{}'' muttered the stranger,
frightened by the question and Raskolnikov's strange manner, and he
crossed over to the other side of the street.

Raskolnikov walked straight on and came out at the corner of the Hay
Market, where the huckster and his wife had talked with Lizaveta; but
they were not there now. Recognising the place, he stopped, looked round
and addressed a young fellow in a red shirt who stood gaping before a
corn chandler's shop.

``Isn't there a man who keeps a booth with his wife at this corner?''

``All sorts of people keep booths here,'' answered the young man,
glancing superciliously at Raskolnikov.

``What's his name?''

``What he was christened.''

``Aren't you a Zaraïsky man, too? Which province?''

The young man looked at Raskolnikov again.

``It's not a province, your excellency, but a district. Graciously
forgive me, your excellency!''

``Is that a tavern at the top there?''

``Yes, it's an eating-house and there's a billiard-room and you'll find
princesses there too\ldots{}. La-la!''

Raskolnikov crossed the square. In that corner there was a dense crowd
of peasants. He pushed his way into the thickest part of it, looking at
the faces. He felt an unaccountable inclination to enter into
conversation with people. But the peasants took no notice of him; they
were all shouting in groups together. He stood and thought a little and
took a turning to the right in the direction of V.

He had often crossed that little street which turns at an angle, leading
from the market-place to Sadovy Street. Of late he had often felt drawn
to wander about this district, when he felt depressed, that he might
feel more so.

Now he walked along, thinking of nothing. At that point there is a great
block of buildings, entirely let out in dram shops and eating-houses;
women were continually running in and out, bare-headed and in their
indoor clothes. Here and there they gathered in groups, on the pavement,
especially about the entrances to various festive establishments in the
lower storeys. From one of these a loud din, sounds of singing, the
tinkling of a guitar and shouts of merriment, floated into the street. A
crowd of women were thronging round the door; some were sitting on the
steps, others on the pavement, others were standing talking. A drunken
soldier, smoking a cigarette, was walking near them in the road,
swearing; he seemed to be trying to find his way somewhere, but had
forgotten where. One beggar was quarrelling with another, and a man dead
drunk was lying right across the road. Raskolnikov joined the throng of
women, who were talking in husky voices. They were bare-headed and wore
cotton dresses and goatskin shoes. There were women of forty and some
not more than seventeen; almost all had blackened eyes.

He felt strangely attracted by the singing and all the noise and uproar
in the saloon below\ldots{}. someone could be heard within dancing
frantically, marking time with his heels to the sounds of the guitar and
of a thin falsetto voice singing a jaunty air. He listened intently,
gloomily and dreamily, bending down at the entrance and peeping
inquisitively in from the pavement.

\begin{quote}
\itshape
``Oh, my handsome soldier\\
Don't beat me for nothing,''
\end{quote}

trilled the thin voice of the singer. Raskolnikov felt a great desire to
make out what he was singing, as though everything depended on that.

``Shall I go in?'' he thought. ``They are laughing. From drink. Shall I
get drunk?''

``Won't you come in?'' one of the women asked him. Her voice was still
musical and less thick than the others, she was young and not
repulsive---the only one of the group.

``Why, she's pretty,'' he said, drawing himself up and looking at her.

She smiled, much pleased at the compliment.

``You're very nice looking yourself,'' she said.

``Isn't he thin though!'' observed another woman in a deep bass. ``Have
you just come out of a hospital?''

``They're all generals' daughters, it seems, but they have all snub
noses,'' interposed a tipsy peasant with a sly smile on his face,
wearing a loose coat. ``See how jolly they are.''

``Go along with you!''

``I'll go, sweetie!''

And he darted down into the saloon below. Raskolnikov moved on.

``I say, sir,'' the girl shouted after him.

``What is it?''

She hesitated.

``I'll always be pleased to spend an hour with you, kind gentleman, but
now I feel shy. Give me six copecks for a drink, there's a nice young
man!''

Raskolnikov gave her what came first---fifteen copecks.

``Ah, what a good-natured gentleman!''

``What's your name?''

``Ask for Duclida.''

``Well, that's too much,'' one of the women observed, shaking her head
at Duclida. ``I don't know how you can ask like that. I believe I should
drop with shame\ldots{}.''

Raskolnikov looked curiously at the speaker. She was a pock-marked wench
of thirty, covered with bruises, with her upper lip swollen. She made
her criticism quietly and earnestly. ``Where is it,'' thought
Raskolnikov. ``Where is it I've read that someone condemned to death
says or thinks, an hour before his death, that if he had to live on some
high rock, on such a narrow ledge that he'd only room to stand, and the
ocean, everlasting darkness, everlasting solitude, everlasting tempest
around him, if he had to remain standing on a square yard of space all
his life, a thousand years, eternity, it were better to live so than to
die at once! Only to live, to live and live! Life, whatever it may
be!\ldots{} How true it is! Good God, how true! Man is a vile
creature!\ldots{} And vile is he who calls him vile for that,'' he added
a moment later.

He went into another street. ``Bah, the Palais de Cristal! Razumihin was
just talking of the Palais de Cristal. But what on earth was it I
wanted? Yes, the newspapers\ldots{}. Zossimov said he'd read it in the
papers. Have you the papers?'' he asked, going into a very spacious and
positively clean restaurant, consisting of several rooms, which were,
however, rather empty. Two or three people were drinking tea, and in a
room further away were sitting four men drinking champagne. Raskolnikov
fancied that Zametov was one of them, but he could not be sure at that
distance. ``What if it is?'' he thought.

``Will you have vodka?'' asked the waiter.

``Give me some tea and bring me the papers, the old ones for the last
five days, and I'll give you something.''

``Yes, sir, here's to-day's. No vodka?''

The old newspapers and the tea were brought. Raskolnikov sat down and
began to look through them.

``Oh, damn\ldots{} these are the items of intelligence. An accident on a
staircase, spontaneous combustion of a shopkeeper from alcohol, a fire
in Peski\ldots{} a fire in the Petersburg quarter\ldots{} another fire
in the Petersburg quarter\ldots{} and another fire in the Petersburg
quarter\ldots{}. Ah, here it is!'' He found at last what he was seeking
and began to read it. The lines danced before his eyes, but he read it
all and began eagerly seeking later additions in the following numbers.
His hands shook with nervous impatience as he turned the sheets.
Suddenly someone sat down beside him at his table. He looked up, it was
the head clerk Zametov, looking just the same, with the rings on his
fingers and the watch-chain, with the curly, black hair, parted and
pomaded, with the smart waistcoat, rather shabby coat and doubtful
linen. He was in a good humour, at least he was smiling very gaily and
good-humouredly. His dark face was rather flushed from the champagne he
had drunk.

``What, you here?'' he began in surprise, speaking as though he'd known
him all his life. ``Why, Razumihin told me only yesterday you were
unconscious. How strange! And do you know I've been to see you?''

Raskolnikov knew he would come up to him. He laid aside the papers and
turned to Zametov. There was a smile on his lips, and a new shade of
irritable impatience was apparent in that smile.

``I know you have,'' he answered. ``I've heard it. You looked for my
sock\ldots{}. And you know Razumihin has lost his heart to you? He says
you've been with him to Luise Ivanovna's---you know, the woman you tried
to befriend, for whom you winked to the Explosive Lieutenant and he
would not understand. Do you remember? How could he fail to
understand---it was quite clear, wasn't it?''

``What a hot head he is!''

``The explosive one?''

``No, your friend Razumihin.''

``You must have a jolly life, Mr. Zametov; entrance free to the most
agreeable places. Who's been pouring champagne into you just now?''

``We've just been\ldots{} having a drink together\ldots{}. You talk
about pouring it into me!''

``By way of a fee! You profit by everything!'' Raskolnikov laughed,
``it's all right, my dear boy,'' he added, slapping Zametov on the
shoulder. ``I am not speaking from temper, but in a friendly way, for
sport, as that workman of yours said when he was scuffling with Dmitri,
in the case of the old woman\ldots{}.''

``How do you know about it?''

``Perhaps I know more about it than you do.''

``How strange you are\ldots{}. I am sure you are still very unwell. You
oughtn't to have come out.''

``Oh, do I seem strange to you?''

``Yes. What are you doing, reading the papers?''

``Yes.''

``There's a lot about the fires.''

``No, I am not reading about the fires.'' Here he looked mysteriously at
Zametov; his lips were twisted again in a mocking smile. ``No, I am not
reading about the fires,'' he went on, winking at Zametov. ``But confess
now, my dear fellow, you're awfully anxious to know what I am reading
about?''

``I am not in the least. Mayn't I ask a question? Why do you keep
on\ldots{}?''

``Listen, you are a man of culture and education?''

``I was in the sixth class at the gymnasium,'' said Zametov with some
dignity.

``Sixth class! Ah, my cock-sparrow! With your parting and your
rings---you are a gentleman of fortune. Foo! what a charming boy!'' Here
Raskolnikov broke into a nervous laugh right in Zametov's face. The
latter drew back, more amazed than offended.

``Foo! how strange you are!'' Zametov repeated very seriously. ``I can't
help thinking you are still delirious.''

``I am delirious? You are fibbing, my cock-sparrow! So I am strange? You
find me curious, do you?''

``Yes, curious.''

``Shall I tell you what I was reading about, what I was looking for? See
what a lot of papers I've made them bring me. Suspicious, eh?''

``Well, what is it?''

``You prick up your ears?''

``How do you mean---`prick up my ears'?''

``I'll explain that afterwards, but now, my boy, I declare to
you\ldots{} no, better `I confess'\ldots{} No, that's not right either;
`I make a deposition and you take it.' I depose that I was reading, that
I was looking and searching\ldots{}.'' he screwed up his eyes and
paused. ``I was searching---and came here on purpose to do it---for news
of the murder of the old pawnbroker woman,'' he articulated at last,
almost in a whisper, bringing his face exceedingly close to the face of
Zametov. Zametov looked at him steadily, without moving or drawing his
face away. What struck Zametov afterwards as the strangest part of it
all was that silence followed for exactly a minute, and that they gazed
at one another all the while.

``What if you have been reading about it?'' he cried at last, perplexed
and impatient. ``That's no business of mine! What of it?''

``The same old woman,'' Raskolnikov went on in the same whisper, not
heeding Zametov's explanation, ``about whom you were talking in the
police-office, you remember, when I fainted. Well, do you understand
now?''

``What do you mean? Understand\ldots{} what?'' Zametov brought out,
almost alarmed.

Raskolnikov's set and earnest face was suddenly transformed, and he
suddenly went off into the same nervous laugh as before, as though
utterly unable to restrain himself. And in one flash he recalled with
extraordinary vividness of sensation a moment in the recent past, that
moment when he stood with the axe behind the door, while the latch
trembled and the men outside swore and shook it, and he had a sudden
desire to shout at them, to swear at them, to put out his tongue at
them, to mock them, to laugh, and laugh, and laugh!

``You are either mad, or\ldots{}'' began Zametov, and he broke off, as
though stunned by the idea that had suddenly flashed into his mind.

``Or? Or what? What? Come, tell me!''

``Nothing,'' said Zametov, getting angry, ``it's all nonsense!''

Both were silent. After his sudden fit of laughter Raskolnikov became
suddenly thoughtful and melancholy. He put his elbow on the table and
leaned his head on his hand. He seemed to have completely forgotten
Zametov. The silence lasted for some time.

``Why don't you drink your tea? It's getting cold,'' said Zametov.

``What! Tea? Oh, yes\ldots{}.'' Raskolnikov sipped the glass, put a
morsel of bread in his mouth and, suddenly looking at Zametov, seemed to
remember everything and pulled himself together. At the same moment his
face resumed its original mocking expression. He went on drinking tea.

``There have been a great many of these crimes lately,'' said Zametov.
``Only the other day I read in the \emph{Moscow News} that a whole gang
of false coiners had been caught in Moscow. It was a regular society.
They used to forge tickets!''

``Oh, but it was a long time ago! I read about it a month ago,''
Raskolnikov answered calmly. ``So you consider them criminals?'' he
added, smiling.

``Of course they are criminals.''

``They? They are children, simpletons, not criminals! Why, half a
hundred people meeting for such an object---what an idea! Three would be
too many, and then they want to have more faith in one another than in
themselves! One has only to blab in his cups and it all collapses.
Simpletons! They engaged untrustworthy people to change the notes---what
a thing to trust to a casual stranger! Well, let us suppose that these
simpletons succeed and each makes a million, and what follows for the
rest of their lives? Each is dependent on the others for the rest of his
life! Better hang oneself at once! And they did not know how to change
the notes either; the man who changed the notes took five thousand
roubles, and his hands trembled. He counted the first four thousand, but
did not count the fifth thousand---he was in such a hurry to get the
money into his pocket and run away. Of course he roused suspicion. And
the whole thing came to a crash through one fool! Is it possible?''

``That his hands trembled?'' observed Zametov, ``yes, that's quite
possible. That, I feel quite sure, is possible. Sometimes one can't
stand things.''

``Can't stand that?''

``Why, could you stand it then? No, I couldn't. For the sake of a
hundred roubles to face such a terrible experience? To go with false
notes into a bank where it's their business to spot that sort of thing!
No, I should not have the face to do it. Would you?''

Raskolnikov had an intense desire again ``to put his tongue out.''
Shivers kept running down his spine.

``I should do it quite differently,'' Raskolnikov began. ``This is how I
would change the notes: I'd count the first thousand three or four times
backwards and forwards, looking at every note and then I'd set to the
second thousand; I'd count that half-way through and then hold some
fifty-rouble note to the light, then turn it, then hold it to the light
again---to see whether it was a good one. `I am afraid,' I would say, `a
relation of mine lost twenty-five roubles the other day through a false
note,' and then I'd tell them the whole story. And after I began
counting the third, `No, excuse me,' I would say, `I fancy I made a
mistake in the seventh hundred in that second thousand, I am not sure.'
And so I would give up the third thousand and go back to the second and
so on to the end. And when I had finished, I'd pick out one from the
fifth and one from the second thousand and take them again to the light
and ask again, `Change them, please,' and put the clerk into such a stew
that he would not know how to get rid of me. When I'd finished and had
gone out, I'd come back, `No, excuse me,' and ask for some explanation.
That's how I'd do it.''

``Foo! what terrible things you say!'' said Zametov, laughing. ``But all
that is only talk. I dare say when it came to deeds you'd make a slip. I
believe that even a practised, desperate man cannot always reckon on
himself, much less you and I. To take an example near home---that old
woman murdered in our district. The murderer seems to have been a
desperate fellow, he risked everything in open daylight, was saved by a
miracle---but his hands shook, too. He did not succeed in robbing the
place, he couldn't stand it. That was clear from the\ldots{}''

Raskolnikov seemed offended.

``Clear? Why don't you catch him then?'' he cried, maliciously gibing at
Zametov.

``Well, they will catch him.''

``Who? You? Do you suppose you could catch him? You've a tough job! A
great point for you is whether a man is spending money or not. If he had
no money and suddenly begins spending, he must be the man. So that any
child can mislead you.''

``The fact is they always do that, though,'' answered Zametov. ``A man
will commit a clever murder at the risk of his life and then at once he
goes drinking in a tavern. They are caught spending money, they are not
all as cunning as you are. You wouldn't go to a tavern, of course?''

Raskolnikov frowned and looked steadily at Zametov.

``You seem to enjoy the subject and would like to know how I should
behave in that case, too?'' he asked with displeasure.

``I should like to,'' Zametov answered firmly and seriously. Somewhat
too much earnestness began to appear in his words and looks.

``Very much?''

``Very much!''

``All right then. This is how I should behave,'' Raskolnikov began,
again bringing his face close to Zametov's, again staring at him and
speaking in a whisper, so that the latter positively shuddered. ``This
is what I should have done. I should have taken the money and jewels, I
should have walked out of there and have gone straight to some deserted
place with fences round it and scarcely anyone to be seen, some kitchen
garden or place of that sort. I should have looked out beforehand some
stone weighing a hundredweight or more which had been lying in the
corner from the time the house was built. I would lift that
stone---there would sure to be a hollow under it, and I would put the
jewels and money in that hole. Then I'd roll the stone back so that it
would look as before, would press it down with my foot and walk away.
And for a year or two, three maybe, I would not touch it. And, well,
they could search! There'd be no trace.''

``You are a madman,'' said Zametov, and for some reason he too spoke in
a whisper, and moved away from Raskolnikov, whose eyes were glittering.
He had turned fearfully pale and his upper lip was twitching and
quivering. He bent down as close as possible to Zametov, and his lips
began to move without uttering a word. This lasted for half a minute; he
knew what he was doing, but could not restrain himself. The terrible
word trembled on his lips, like the latch on that door; in another
moment it will break out, in another moment he will let it go, he will
speak out.

``And what if it was I who murdered the old woman and Lizaveta?'' he
said suddenly and---realised what he had done.

Zametov looked wildly at him and turned white as the tablecloth. His
face wore a contorted smile.

``But is it possible?'' he brought out faintly. Raskolnikov looked
wrathfully at him.

``Own up that you believed it, yes, you did?''

``Not a bit of it, I believe it less than ever now,'' Zametov cried
hastily.

``I've caught my cock-sparrow! So you did believe it before, if now you
believe less than ever?''

``Not at all,'' cried Zametov, obviously embarrassed. ``Have you been
frightening me so as to lead up to this?''

``You don't believe it then? What were you talking about behind my back
when I went out of the police-office? And why did the explosive
lieutenant question me after I fainted? Hey, there,'' he shouted to the
waiter, getting up and taking his cap, ``how much?''

``Thirty copecks,'' the latter replied, running up.

``And there is twenty copecks for vodka. See what a lot of money!'' he
held out his shaking hand to Zametov with notes in it. ``Red notes and
blue, twenty-five roubles. Where did I get them? And where did my new
clothes come from? You know I had not a copeck. You've cross-examined my
landlady, I'll be bound\ldots{}. Well, that's enough! \emph{Assez
causé!} Till we meet again!''

He went out, trembling all over from a sort of wild hysterical
sensation, in which there was an element of insufferable rapture. Yet he
was gloomy and terribly tired. His face was twisted as after a fit. His
fatigue increased rapidly. Any shock, any irritating sensation
stimulated and revived his energies at once, but his strength failed as
quickly when the stimulus was removed.

Zametov, left alone, sat for a long time in the same place, plunged in
thought. Raskolnikov had unwittingly worked a revolution in his brain on
a certain point and had made up his mind for him conclusively.

``Ilya Petrovitch is a blockhead,'' he decided.

Raskolnikov had hardly opened the door of the restaurant when he
stumbled against Razumihin on the steps. They did not see each other
till they almost knocked against each other. For a moment they stood
looking each other up and down. Razumihin was greatly astounded, then
anger, real anger gleamed fiercely in his eyes.

``So here you are!'' he shouted at the top of his voice---``you ran away
from your bed! And here I've been looking for you under the sofa! We
went up to the garret. I almost beat Nastasya on your account. And here
he is after all. Rodya! What is the meaning of it? Tell me the whole
truth! Confess! Do you hear?''

``It means that I'm sick to death of you all and I want to be alone,''
Raskolnikov answered calmly.

``Alone? When you are not able to walk, when your face is as white as a
sheet and you are gasping for breath! Idiot!\ldots{} What have you been
doing in the Palais de Cristal? Own up at once!''

``Let me go!'' said Raskolnikov and tried to pass him. This was too much
for Razumihin; he gripped him firmly by the shoulder.

``Let you go? You dare tell me to let you go? Do you know what I'll do
with you directly? I'll pick you up, tie you up in a bundle, carry you
home under my arm and lock you up!''

``Listen, Razumihin,'' Raskolnikov began quietly, apparently
calm---``can't you see that I don't want your benevolence? A strange
desire you have to shower benefits on a man who\ldots{} curses them, who
feels them a burden in fact! Why did you seek me out at the beginning of
my illness? Maybe I was very glad to die. Didn't I tell you plainly
enough to-day that you were torturing me, that I was\ldots{} sick of
you! You seem to want to torture people! I assure you that all that is
seriously hindering my recovery, because it's continually irritating me.
You saw Zossimov went away just now to avoid irritating me. You leave me
alone too, for goodness' sake! What right have you, indeed, to keep me
by force? Don't you see that I am in possession of all my faculties now?
How, how can I persuade you not to persecute me with your kindness? I
may be ungrateful, I may be mean, only let me be, for God's sake, let me
be! Let me be, let me be!''

He began calmly, gloating beforehand over the venomous phrases he was
about to utter, but finished, panting for breath, in a frenzy, as he had
been with Luzhin.

Razumihin stood a moment, thought and let his hand drop.

``Well, go to hell then,'' he said gently and thoughtfully. ``Stay,'' he
roared, as Raskolnikov was about to move. ``Listen to me. Let me tell
you, that you are all a set of babbling, posing idiots! If you've any
little trouble you brood over it like a hen over an egg. And you are
plagiarists even in that! There isn't a sign of independent life in you!
You are made of spermaceti ointment and you've lymph in your veins
instead of blood. I don't believe in anyone of you! In any circumstances
the first thing for all of you is to be unlike a human being! Stop!'' he
cried with redoubled fury, noticing that Raskolnikov was again making a
movement---``hear me out! You know I'm having a house-warming this
evening, I dare say they've arrived by now, but I left my uncle
there---I just ran in---to receive the guests. And if you weren't a
fool, a common fool, a perfect fool, if you were an original instead of
a translation\ldots{} you see, Rodya, I recognise you're a clever
fellow, but you're a fool!---and if you weren't a fool you'd come round
to me this evening instead of wearing out your boots in the street!
Since you have gone out, there's no help for it! I'd give you a snug
easy chair, my landlady has one\ldots{} a cup of tea, company\ldots{}.
Or you could lie on the sofa---any way you would be with us\ldots{}.
Zossimov will be there too. Will you come?''

``No.''

``R-rubbish!'' Razumihin shouted, out of patience. ``How do you know?
You can't answer for yourself! You don't know anything about it\ldots{}.
Thousands of times I've fought tooth and nail with people and run back
to them afterwards\ldots{}. One feels ashamed and goes back to a man! So
remember, Potchinkov's house on the third storey\ldots{}.''

``Why, Mr. Razumihin, I do believe you'd let anybody beat you from sheer
benevolence.''

``Beat? Whom? Me? I'd twist his nose off at the mere idea! Potchinkov's
house, 47, Babushkin's flat\ldots{}.''

``I shall not come, Razumihin.'' Raskolnikov turned and walked away.

``I bet you will,'' Razumihin shouted after him. ``I refuse to know you
if you don't! Stay, hey, is Zametov in there?''

``Yes.''

``Did you see him?''

``Yes.''

``Talked to him?''

``Yes.''

``What about? Confound you, don't tell me then. Potchinkov's house, 47,
Babushkin's flat, remember!''

Raskolnikov walked on and turned the corner into Sadovy Street.
Razumihin looked after him thoughtfully. Then with a wave of his hand he
went into the house but stopped short of the stairs.

``Confound it,'' he went on almost aloud. ``He talked sensibly but
yet\ldots{} I am a fool! As if madmen didn't talk sensibly! And this was
just what Zossimov seemed afraid of.'' He struck his finger on his
forehead. ``What if\ldots{} how could I let him go off alone? He may
drown himself\ldots{}. Ach, what a blunder! I can't.'' And he ran back
to overtake Raskolnikov, but there was no trace of him. With a curse he
returned with rapid steps to the Palais de Cristal to question Zametov.

Raskolnikov walked straight to X------ Bridge, stood in the middle, and
leaning both elbows on the rail stared into the distance. On parting
with Razumihin, he felt so much weaker that he could scarcely reach this
place. He longed to sit or lie down somewhere in the street. Bending
over the water, he gazed mechanically at the last pink flush of the
sunset, at the row of houses growing dark in the gathering twilight, at
one distant attic window on the left bank, flashing as though on fire in
the last rays of the setting sun, at the darkening water of the canal,
and the water seemed to catch his attention. At last red circles flashed
before his eyes, the houses seemed moving, the passers-by, the canal
banks, the carriages, all danced before his eyes. Suddenly he started,
saved again perhaps from swooning by an uncanny and hideous sight. He
became aware of someone standing on the right side of him; he looked and
saw a tall woman with a kerchief on her head, with a long, yellow,
wasted face and red sunken eyes. She was looking straight at him, but
obviously she saw nothing and recognised no one. Suddenly she leaned her
right hand on the parapet, lifted her right leg over the railing, then
her left and threw herself into the canal. The filthy water parted and
swallowed up its victim for a moment, but an instant later the drowning
woman floated to the surface, moving slowly with the current, her head
and legs in the water, her skirt inflated like a balloon over her back.

``A woman drowning! A woman drowning!'' shouted dozens of voices; people
ran up, both banks were thronged with spectators, on the bridge people
crowded about Raskolnikov, pressing up behind him.

``Mercy on it! it's our Afrosinya!'' a woman cried tearfully close by.
``Mercy! save her! kind people, pull her out!''

``A boat, a boat'' was shouted in the crowd. But there was no need of a
boat; a policeman ran down the steps to the canal, threw off his great
coat and his boots and rushed into the water. It was easy to reach her:
she floated within a couple of yards from the steps, he caught hold of
her clothes with his right hand and with his left seized a pole which a
comrade held out to him; the drowning woman was pulled out at once. They
laid her on the granite pavement of the embankment. She soon recovered
consciousness, raised her head, sat up and began sneezing and coughing,
stupidly wiping her wet dress with her hands. She said nothing.

``She's drunk herself out of her senses,'' the same woman's voice wailed
at her side. ``Out of her senses. The other day she tried to hang
herself, we cut her down. I ran out to the shop just now, left my little
girl to look after her---and here she's in trouble again! A neighbour,
gentleman, a neighbour, we live close by, the second house from the end,
see yonder\ldots{}.''

The crowd broke up. The police still remained round the woman, someone
mentioned the police station\ldots{}. Raskolnikov looked on with a
strange sensation of indifference and apathy. He felt disgusted. ``No,
that's loathsome\ldots{} water\ldots{} it's not good enough,'' he
muttered to himself. ``Nothing will come of it,'' he added, ``no use to
wait. What about the police office\ldots{}? And why isn't Zametov at the
police office? The police office is open till ten o'clock\ldots{}.'' He
turned his back to the railing and looked about him.

``Very well then!'' he said resolutely; he moved from the bridge and
walked in the direction of the police office. His heart felt hollow and
empty. He did not want to think. Even his depression had passed, there
was not a trace now of the energy with which he had set out ``to make an
end of it all.'' Complete apathy had succeeded to it.

``Well, it's a way out of it,'' he thought, walking slowly and
listlessly along the canal bank. ``Anyway I'll make an end, for I want
to\ldots{}. But is it a way out? What does it matter! There'll be the
square yard of space---ha! But what an end! Is it really the end? Shall
I tell them or not? Ah\ldots{} damn! How tired I am! If I could find
somewhere to sit or lie down soon! What I am most ashamed of is its
being so stupid. But I don't care about that either! What idiotic ideas
come into one's head.''

To reach the police office he had to go straight forward and take the
second turning to the left. It was only a few paces away. But at the
first turning he stopped and, after a minute's thought, turned into a
side street and went two streets out of his way, possibly without any
object, or possibly to delay a minute and gain time. He walked, looking
at the ground; suddenly someone seemed to whisper in his ear; he lifted
his head and saw that he was standing at the very gate of \emph{the}
house. He had not passed it, he had not been near it since \emph{that}
evening. An overwhelming, unaccountable prompting drew him on. He went
into the house, passed through the gateway, then into the first entrance
on the right, and began mounting the familiar staircase to the fourth
storey. The narrow, steep staircase was very dark. He stopped at each
landing and looked round him with curiosity; on the first landing the
framework of the window had been taken out. ``That wasn't so then,'' he
thought. Here was the flat on the second storey where Nikolay and Dmitri
had been working. ``It's shut up and the door newly painted. So it's to
let.'' Then the third storey and the fourth. ``Here!'' He was perplexed
to find the door of the flat wide open. There were men there, he could
hear voices; he had not expected that. After brief hesitation he mounted
the last stairs and went into the flat. It, too, was being done up;
there were workmen in it. This seemed to amaze him; he somehow fancied
that he would find everything as he left it, even perhaps the corpses in
the same places on the floor. And now, bare walls, no furniture; it
seemed strange. He walked to the window and sat down on the window-sill.
There were two workmen, both young fellows, but one much younger than
the other. They were papering the walls with a new white paper covered
with lilac flowers, instead of the old, dirty, yellow one. Raskolnikov
for some reason felt horribly annoyed by this. He looked at the new
paper with dislike, as though he felt sorry to have it all so changed.
The workmen had obviously stayed beyond their time and now they were
hurriedly rolling up their paper and getting ready to go home. They took
no notice of Raskolnikov's coming in; they were talking. Raskolnikov
folded his arms and listened.

``She comes to me in the morning,'' said the elder to the younger,
``very early, all dressed up. `Why are you preening and prinking?' says
I. `I am ready to do anything to please you, Tit Vassilitch!' That's a
way of going on! And she dressed up like a regular fashion book!''

``And what is a fashion book?'' the younger one asked. He obviously
regarded the other as an authority.

``A fashion book is a lot of pictures, coloured, and they come to the
tailors here every Saturday, by post from abroad, to show folks how to
dress, the male sex as well as the female. They're pictures. The
gentlemen are generally wearing fur coats and for the ladies' fluffles,
they're beyond anything you can fancy.''

``There's nothing you can't find in Petersburg,'' the younger cried
enthusiastically, ``except father and mother, there's everything!''

``Except them, there's everything to be found, my boy,'' the elder
declared sententiously.

Raskolnikov got up and walked into the other room where the strong box,
the bed, and the chest of drawers had been; the room seemed to him very
tiny without furniture in it. The paper was the same; the paper in the
corner showed where the case of ikons had stood. He looked at it and
went to the window. The elder workman looked at him askance.

``What do you want?'' he asked suddenly.

Instead of answering Raskolnikov went into the passage and pulled the
bell. The same bell, the same cracked note. He rang it a second and a
third time; he listened and remembered. The hideous and agonisingly
fearful sensation he had felt then began to come back more and more
vividly. He shuddered at every ring and it gave him more and more
satisfaction.

``Well, what do you want? Who are you?'' the workman shouted, going out
to him. Raskolnikov went inside again.

``I want to take a flat,'' he said. ``I am looking round.''

``It's not the time to look at rooms at night! and you ought to come up
with the porter.''

``The floors have been washed, will they be painted?'' Raskolnikov went
on. ``Is there no blood?''

``What blood?''

``Why, the old woman and her sister were murdered here. There was a
perfect pool there.''

``But who are you?'' the workman cried, uneasy.

``Who am I?''

``Yes.''

``You want to know? Come to the police station, I'll tell you.''

The workmen looked at him in amazement.

``It's time for us to go, we are late. Come along, Alyoshka. We must
lock up,'' said the elder workman.

``Very well, come along,'' said Raskolnikov indifferently, and going out
first, he went slowly downstairs. ``Hey, porter,'' he cried in the
gateway.

At the entrance several people were standing, staring at the passers-by;
the two porters, a peasant woman, a man in a long coat and a few others.
Raskolnikov went straight up to them.

``What do you want?'' asked one of the porters.

``Have you been to the police office?''

``I've just been there. What do you want?''

``Is it open?''

``Of course.''

``Is the assistant there?''

``He was there for a time. What do you want?''

Raskolnikov made no reply, but stood beside them lost in thought.

``He's been to look at the flat,'' said the elder workman, coming
forward.

``Which flat?''

``Where we are at work. `Why have you washed away the blood?' says he.
`There has been a murder here,' says he, `and I've come to take it.' And
he began ringing at the bell, all but broke it. `Come to the police
station,' says he. `I'll tell you everything there.' He wouldn't leave
us.''

The porter looked at Raskolnikov, frowning and perplexed.

``Who are you?'' he shouted as impressively as he could.

``I am Rodion Romanovitch Raskolnikov, formerly a student, I live in
Shil's house, not far from here, flat Number 14, ask the porter, he
knows me.'' Raskolnikov said all this in a lazy, dreamy voice, not
turning round, but looking intently into the darkening street.

``Why have you been to the flat?''

``To look at it.''

``What is there to look at?''

``Take him straight to the police station,'' the man in the long coat
jerked in abruptly.

Raskolnikov looked intently at him over his shoulder and said in the
same slow, lazy tones:

``Come along.''

``Yes, take him,'' the man went on more confidently. ``Why was he going
into \emph{that}, what's in his mind, eh?''

``He's not drunk, but God knows what's the matter with him,'' muttered
the workman.

``But what do you want?'' the porter shouted again, beginning to get
angry in earnest---``Why are you hanging about?''

``You funk the police station then?'' said Raskolnikov jeeringly.

``How funk it? Why are you hanging about?''

``He's a rogue!'' shouted the peasant woman.

``Why waste time talking to him?'' cried the other porter, a huge
peasant in a full open coat and with keys on his belt. ``Get along! He
is a rogue and no mistake. Get along!''

And seizing Raskolnikov by the shoulder he flung him into the street. He
lurched forward, but recovered his footing, looked at the spectators in
silence and walked away.

``Strange man!'' observed the workman.

``There are strange folks about nowadays,'' said the woman.

``You should have taken him to the police station all the same,'' said
the man in the long coat.

``Better have nothing to do with him,'' decided the big porter. ``A
regular rogue! Just what he wants, you may be sure, but once take him
up, you won't get rid of him\ldots{}. We know the sort!''

``Shall I go there or not?'' thought Raskolnikov, standing in the middle
of the thoroughfare at the cross-roads, and he looked about him, as
though expecting from someone a decisive word. But no sound came, all
was dead and silent like the stones on which he walked, dead to him, to
him alone\ldots{}. All at once at the end of the street, two hundred
yards away, in the gathering dusk he saw a crowd and heard talk and
shouts. In the middle of the crowd stood a carriage\ldots{}. A light
gleamed in the middle of the street. ``What is it?'' Raskolnikov turned
to the right and went up to the crowd. He seemed to clutch at everything
and smiled coldly when he recognised it, for he had fully made up his
mind to go to the police station and knew that it would all soon be
over.

\mychapter{7}{Chapter VII}\label{chapter-vii-1}

An elegant carriage stood in the middle of the road with a pair of
spirited grey horses; there was no one in it, and the coachman had got
off his box and stood by; the horses were being held by the
bridle\ldots{}. A mass of people had gathered round, the police standing
in front. One of them held a lighted lantern which he was turning on
something lying close to the wheels. Everyone was talking, shouting,
exclaiming; the coachman seemed at a loss and kept repeating:

``What a misfortune! Good Lord, what a misfortune!''

Raskolnikov pushed his way in as far as he could, and succeeded at last
in seeing the object of the commotion and interest. On the ground a man
who had been run over lay apparently unconscious, and covered with
blood; he was very badly dressed, but not like a workman. Blood was
flowing from his head and face; his face was crushed, mutilated and
disfigured. He was evidently badly injured.

``Merciful heaven!'' wailed the coachman, ``what more could I do? If I'd
been driving fast or had not shouted to him, but I was going quietly,
not in a hurry. Everyone could see I was going along just like everybody
else. A drunken man can't walk straight, we all know\ldots{}. I saw him
crossing the street, staggering and almost falling. I shouted again and
a second and a third time, then I held the horses in, but he fell
straight under their feet! Either he did it on purpose or he was very
tipsy\ldots{}. The horses are young and ready to take fright\ldots{}
they started, he screamed\ldots{} that made them worse. That's how it
happened!''

``That's just how it was,'' a voice in the crowd confirmed.

``He shouted, that's true, he shouted three times,'' another voice
declared.

``Three times it was, we all heard it,'' shouted a third.

But the coachman was not very much distressed and frightened. It was
evident that the carriage belonged to a rich and important person who
was awaiting it somewhere; the police, of course, were in no little
anxiety to avoid upsetting his arrangements. All they had to do was to
take the injured man to the police station and the hospital. No one knew
his name.

Meanwhile Raskolnikov had squeezed in and stooped closer over him. The
lantern suddenly lighted up the unfortunate man's face. He recognised
him.

``I know him! I know him!'' he shouted, pushing to the front. ``It's a
government clerk retired from the service, Marmeladov. He lives close by
in Kozel's house\ldots{}. Make haste for a doctor! I will pay, see?'' He
pulled money out of his pocket and showed it to the policeman. He was in
violent agitation.

The police were glad that they had found out who the man was.
Raskolnikov gave his own name and address, and, as earnestly as if it
had been his father, he besought the police to carry the unconscious
Marmeladov to his lodging at once.

``Just here, three houses away,'' he said eagerly, ``the house belongs
to Kozel, a rich German. He was going home, no doubt drunk. I know him,
he is a drunkard. He has a family there, a wife, children, he has one
daughter\ldots{}. It will take time to take him to the hospital, and
there is sure to be a doctor in the house. I'll pay, I'll pay! At least
he will be looked after at home\ldots{} they will help him at once. But
he'll die before you get him to the hospital.'' He managed to slip
something unseen into the policeman's hand. But the thing was
straightforward and legitimate, and in any case help was closer here.
They raised the injured man; people volunteered to help.

Kozel's house was thirty yards away. Raskolnikov walked behind,
carefully holding Marmeladov's head and showing the way.

``This way, this way! We must take him upstairs head foremost. Turn
round! I'll pay, I'll make it worth your while,'' he muttered.

Katerina Ivanovna had just begun, as she always did at every free
moment, walking to and fro in her little room from window to stove and
back again, with her arms folded across her chest, talking to herself
and coughing. Of late she had begun to talk more than ever to her eldest
girl, Polenka, a child of ten, who, though there was much she did not
understand, understood very well that her mother needed her, and so
always watched her with her big clever eyes and strove her utmost to
appear to understand. This time Polenka was undressing her little
brother, who had been unwell all day and was going to bed. The boy was
waiting for her to take off his shirt, which had to be washed at night.
He was sitting straight and motionless on a chair, with a silent,
serious face, with his legs stretched out straight before him---heels
together and toes turned out.

He was listening to what his mother was saying to his sister, sitting
perfectly still with pouting lips and wide-open eyes, just as all good
little boys have to sit when they are undressed to go to bed. A little
girl, still younger, dressed literally in rags, stood at the screen,
waiting for her turn. The door on to the stairs was open to relieve them
a little from the clouds of tobacco smoke which floated in from the
other rooms and brought on long terrible fits of coughing in the poor,
consumptive woman. Katerina Ivanovna seemed to have grown even thinner
during that week and the hectic flush on her face was brighter than
ever.

``You wouldn't believe, you can't imagine, Polenka,'' she said, walking
about the room, ``what a happy luxurious life we had in my papa's house
and how this drunkard has brought me, and will bring you all, to ruin!
Papa was a civil colonel and only a step from being a governor; so that
everyone who came to see him said, `We look upon you, Ivan Mihailovitch,
as our governor!' When I\ldots{} when\ldots{}'' she coughed violently,
``oh, cursed life,'' she cried, clearing her throat and pressing her
hands to her breast, ``when I\ldots{} when at the last ball\ldots{} at
the marshal's\ldots{} Princess Bezzemelny saw me---who gave me the
blessing when your father and I were married, Polenka---she asked at
once `Isn't that the pretty girl who danced the shawl dance at the
breaking-up?' (You must mend that tear, you must take your needle and
darn it as I showed you, or to-morrow---cough, cough, cough---he will
make the hole bigger,'' she articulated with effort.) ``Prince
Schegolskoy, a kammerjunker, had just come from Petersburg then\ldots{}
he danced the mazurka with me and wanted to make me an offer next day;
but I thanked him in flattering expressions and told him that my heart
had long been another's. That other was your father, Polya; papa was
fearfully angry\ldots{}. Is the water ready? Give me the shirt, and the
stockings! Lida,'' said she to the youngest one, ``you must manage
without your chemise to-night\ldots{} and lay your stockings out with
it\ldots{} I'll wash them together\ldots{}. How is it that drunken
vagabond doesn't come in? He has worn his shirt till it looks like a
dish-clout, he has torn it to rags! I'd do it all together, so as not to
have to work two nights running! Oh, dear! (Cough, cough, cough, cough!)
Again! What's this?'' she cried, noticing a crowd in the passage and the
men, who were pushing into her room, carrying a burden. ``What is it?
What are they bringing? Mercy on us!''

``Where are we to put him?'' asked the policeman, looking round when
Marmeladov, unconscious and covered with blood, had been carried in.

``On the sofa! Put him straight on the sofa, with his head this way,''
Raskolnikov showed him.

``Run over in the road! Drunk!'' someone shouted in the passage.

Katerina Ivanovna stood, turning white and gasping for breath. The
children were terrified. Little Lida screamed, rushed to Polenka and
clutched at her, trembling all over.

Having laid Marmeladov down, Raskolnikov flew to Katerina Ivanovna.

``For God's sake be calm, don't be frightened!'' he said, speaking
quickly, ``he was crossing the road and was run over by a carriage,
don't be frightened, he will come to, I told them bring him here\ldots{}
I've been here already, you remember? He will come to; I'll pay!''

``He's done it this time!'' Katerina Ivanovna cried despairingly and she
rushed to her husband.

Raskolnikov noticed at once that she was not one of those women who
swoon easily. She instantly placed under the luckless man's head a
pillow, which no one had thought of and began undressing and examining
him. She kept her head, forgetting herself, biting her trembling lips
and stifling the screams which were ready to break from her.

Raskolnikov meanwhile induced someone to run for a doctor. There was a
doctor, it appeared, next door but one.

``I've sent for a doctor,'' he kept assuring Katerina Ivanovna, ``don't
be uneasy, I'll pay. Haven't you water?\ldots{} and give me a napkin or
a towel, anything, as quick as you can\ldots{}. He is injured, but not
killed, believe me\ldots{}. We shall see what the doctor says!''

Katerina Ivanovna ran to the window; there, on a broken chair in the
corner, a large earthenware basin full of water had been stood, in
readiness for washing her children's and husband's linen that night.
This washing was done by Katerina Ivanovna at night at least twice a
week, if not oftener. For the family had come to such a pass that they
were practically without change of linen, and Katerina Ivanovna could
not endure uncleanliness and, rather than see dirt in the house, she
preferred to wear herself out at night, working beyond her strength when
the rest were asleep, so as to get the wet linen hung on a line and dry
by the morning. She took up the basin of water at Raskolnikov's request,
but almost fell down with her burden. But the latter had already
succeeded in finding a towel, wetted it and began washing the blood off
Marmeladov's face.

Katerina Ivanovna stood by, breathing painfully and pressing her hands
to her breast. She was in need of attention herself. Raskolnikov began
to realise that he might have made a mistake in having the injured man
brought here. The policeman, too, stood in hesitation.

``Polenka,'' cried Katerina Ivanovna, ``run to Sonia, make haste. If you
don't find her at home, leave word that her father has been run over and
that she is to come here at once\ldots{} when she comes in. Run,
Polenka! there, put on the shawl.''

``Run your fastest!'' cried the little boy on the chair suddenly, after
which he relapsed into the same dumb rigidity, with round eyes, his
heels thrust forward and his toes spread out.

Meanwhile the room had become so full of people that you couldn't have
dropped a pin. The policemen left, all except one, who remained for a
time, trying to drive out the people who came in from the stairs. Almost
all Madame Lippevechsel's lodgers had streamed in from the inner rooms
of the flat; at first they were squeezed together in the doorway, but
afterwards they overflowed into the room. Katerina Ivanovna flew into a
fury.

``You might let him die in peace, at least,'' she shouted at the crowd,
``is it a spectacle for you to gape at? With cigarettes! (Cough, cough,
cough!) You might as well keep your hats on\ldots{}. And there is one in
his hat!\ldots{} Get away! You should respect the dead, at least!''

Her cough choked her---but her reproaches were not without result. They
evidently stood in some awe of Katerina Ivanovna. The lodgers, one after
another, squeezed back into the doorway with that strange inner feeling
of satisfaction which may be observed in the presence of a sudden
accident, even in those nearest and dearest to the victim, from which no
living man is exempt, even in spite of the sincerest sympathy and
compassion.

Voices outside were heard, however, speaking of the hospital and saying
that they'd no business to make a disturbance here.

``No business to die!'' cried Katerina Ivanovna, and she was rushing to
the door to vent her wrath upon them, but in the doorway came face to
face with Madame Lippevechsel who had only just heard of the accident
and ran in to restore order. She was a particularly quarrelsome and
irresponsible German.

``Ah, my God!'' she cried, clasping her hands, ``your husband drunken
horses have trampled! To the hospital with him! I am the landlady!''

``Amalia Ludwigovna, I beg you to recollect what you are saying,''
Katerina Ivanovna began haughtily (she always took a haughty tone with
the landlady that she might ``remember her place'' and even now could
not deny herself this satisfaction). ``Amalia Ludwigovna\ldots{}''

``I have you once before told that you to call me Amalia Ludwigovna may
not dare; I am Amalia Ivanovna.''

``You are not Amalia Ivanovna, but Amalia Ludwigovna, and as I am not
one of your despicable flatterers like Mr. Lebeziatnikov, who's laughing
behind the door at this moment (a laugh and a cry of `they are at it
again' was in fact audible at the door) so I shall always call you
Amalia Ludwigovna, though I fail to understand why you dislike that
name. You can see for yourself what has happened to Semyon Zaharovitch;
he is dying. I beg you to close that door at once and to admit no one.
Let him at least die in peace! Or I warn you the Governor-General,
himself, shall be informed of your conduct to-morrow. The prince knew me
as a girl; he remembers Semyon Zaharovitch well and has often been a
benefactor to him. Everyone knows that Semyon Zaharovitch had many
friends and protectors, whom he abandoned himself from an honourable
pride, knowing his unhappy weakness, but now (she pointed to
Raskolnikov) a generous young man has come to our assistance, who has
wealth and connections and whom Semyon Zaharovitch has known from a
child. You may rest assured, Amalia Ludwigovna\ldots{}''

All this was uttered with extreme rapidity, getting quicker and quicker,
but a cough suddenly cut short Katerina Ivanovna's eloquence. At that
instant the dying man recovered consciousness and uttered a groan; she
ran to him. The injured man opened his eyes and without recognition or
understanding gazed at Raskolnikov who was bending over him. He drew
deep, slow, painful breaths; blood oozed at the corners of his mouth and
drops of perspiration came out on his forehead. Not recognising
Raskolnikov, he began looking round uneasily. Katerina Ivanovna looked
at him with a sad but stern face, and tears trickled from her eyes.

``My God! His whole chest is crushed! How he is bleeding,'' she said in
despair. ``We must take off his clothes. Turn a little, Semyon
Zaharovitch, if you can,'' she cried to him.

Marmeladov recognised her.

``A priest,'' he articulated huskily.

Katerina Ivanovna walked to the window, laid her head against the window
frame and exclaimed in despair:

``Oh, cursed life!''

``A priest,'' the dying man said again after a moment's silence.

``They've gone for him,'' Katerina Ivanovna shouted to him, he obeyed
her shout and was silent. With sad and timid eyes he looked for her; she
returned and stood by his pillow. He seemed a little easier but not for
long.

Soon his eyes rested on little Lida, his favourite, who was shaking in
the corner, as though she were in a fit, and staring at him with her
wondering childish eyes.

``A-ah,'' he signed towards her uneasily. He wanted to say something.

``What now?'' cried Katerina Ivanovna.

``Barefoot, barefoot!'' he muttered, indicating with frenzied eyes the
child's bare feet.

``Be silent,'' Katerina Ivanovna cried irritably, ``you know why she is
barefooted.''

``Thank God, the doctor,'' exclaimed Raskolnikov, relieved.

The doctor came in, a precise little old man, a German, looking about
him mistrustfully; he went up to the sick man, took his pulse, carefully
felt his head and with the help of Katerina Ivanovna he unbuttoned the
blood-stained shirt, and bared the injured man's chest. It was gashed,
crushed and fractured, several ribs on the right side were broken. On
the left side, just over the heart, was a large, sinister-looking
yellowish-black bruise---a cruel kick from the horse's hoof. The doctor
frowned. The policeman told him that he was caught in the wheel and
turned round with it for thirty yards on the road.

``It's wonderful that he has recovered consciousness,'' the doctor
whispered softly to Raskolnikov.

``What do you think of him?'' he asked.

``He will die immediately.''

``Is there really no hope?''

``Not the faintest! He is at the last gasp\ldots{}. His head is badly
injured, too\ldots{} Hm\ldots{} I could bleed him if you like,
but\ldots{} it would be useless. He is bound to die within the next five
or ten minutes.''

``Better bleed him then.''

``If you like\ldots{}. But I warn you it will be perfectly useless.''

At that moment other steps were heard; the crowd in the passage parted,
and the priest, a little, grey old man, appeared in the doorway bearing
the sacrament. A policeman had gone for him at the time of the accident.
The doctor changed places with him, exchanging glances with him.
Raskolnikov begged the doctor to remain a little while. He shrugged his
shoulders and remained.

All stepped back. The confession was soon over. The dying man probably
understood little; he could only utter indistinct broken sounds.
Katerina Ivanovna took little Lida, lifted the boy from the chair, knelt
down in the corner by the stove and made the children kneel in front of
her. The little girl was still trembling; but the boy, kneeling on his
little bare knees, lifted his hand rhythmically, crossing himself with
precision and bowed down, touching the floor with his forehead, which
seemed to afford him especial satisfaction. Katerina Ivanovna bit her
lips and held back her tears; she prayed, too, now and then pulling
straight the boy's shirt, and managed to cover the girl's bare shoulders
with a kerchief, which she took from the chest without rising from her
knees or ceasing to pray. Meanwhile the door from the inner rooms was
opened inquisitively again. In the passage the crowd of spectators from
all the flats on the staircase grew denser and denser, but they did not
venture beyond the threshold. A single candle-end lighted up the scene.

At that moment Polenka forced her way through the crowd at the door. She
came in panting from running so fast, took off her kerchief, looked for
her mother, went up to her and said, ``She's coming, I met her in the
street.'' Her mother made her kneel beside her.

Timidly and noiselessly a young girl made her way through the crowd, and
strange was her appearance in that room, in the midst of want, rags,
death and despair. She, too, was in rags, her attire was all of the
cheapest, but decked out in gutter finery of a special stamp,
unmistakably betraying its shameful purpose. Sonia stopped short in the
doorway and looked about her bewildered, unconscious of everything. She
forgot her fourth-hand, gaudy silk dress, so unseemly here with its
ridiculous long train, and her immense crinoline that filled up the
whole doorway, and her light-coloured shoes, and the parasol she brought
with her, though it was no use at night, and the absurd round straw hat
with its flaring flame-coloured feather. Under this rakishly-tilted hat
was a pale, frightened little face with lips parted and eyes staring in
terror. Sonia was a small thin girl of eighteen with fair hair, rather
pretty, with wonderful blue eyes. She looked intently at the bed and the
priest; she too was out of breath with running. At last whispers, some
words in the crowd probably, reached her. She looked down and took a
step forward into the room, still keeping close to the door.

The service was over. Katerina Ivanovna went up to her husband again.
The priest stepped back and turned to say a few words of admonition and
consolation to Katerina Ivanovna on leaving.

``What am I to do with these?'' she interrupted sharply and irritably,
pointing to the little ones.

``God is merciful; look to the Most High for succour,'' the priest
began.

``Ach! He is merciful, but not to us.''

``That's a sin, a sin, madam,'' observed the priest, shaking his head.

``And isn't that a sin?'' cried Katerina Ivanovna, pointing to the dying
man.

``Perhaps those who have involuntarily caused the accident will agree to
compensate you, at least for the loss of his earnings.''

``You don't understand!'' cried Katerina Ivanovna angrily waving her
hand. ``And why should they compensate me? Why, he was drunk and threw
himself under the horses! What earnings? He brought us in nothing but
misery. He drank everything away, the drunkard! He robbed us to get
drink, he wasted their lives and mine for drink! And thank God he's
dying! One less to keep!''

``You must forgive in the hour of death, that's a sin, madam, such
feelings are a great sin.''

Katerina Ivanovna was busy with the dying man; she was giving him water,
wiping the blood and sweat from his head, setting his pillow straight,
and had only turned now and then for a moment to address the priest. Now
she flew at him almost in a frenzy.

``Ah, father! That's words and only words! Forgive! If he'd not been run
over, he'd have come home to-day drunk and his only shirt dirty and in
rags and he'd have fallen asleep like a log, and I should have been
sousing and rinsing till daybreak, washing his rags and the children's
and then drying them by the window and as soon as it was daylight I
should have been darning them. That's how I spend my nights!\ldots{}
What's the use of talking of forgiveness! I have forgiven as it is!''

A terrible hollow cough interrupted her words. She put her handkerchief
to her lips and showed it to the priest, pressing her other hand to her
aching chest. The handkerchief was covered with blood. The priest bowed
his head and said nothing.

Marmeladov was in the last agony; he did not take his eyes off the face
of Katerina Ivanovna, who was bending over him again. He kept trying to
say something to her; he began moving his tongue with difficulty and
articulating indistinctly, but Katerina Ivanovna, understanding that he
wanted to ask her forgiveness, called peremptorily to him:

``Be silent! No need! I know what you want to say!'' And the sick man
was silent, but at the same instant his wandering eyes strayed to the
doorway and he saw Sonia.

Till then he had not noticed her: she was standing in the shadow in a
corner.

``Who's that? Who's that?'' he said suddenly in a thick gasping voice,
in agitation, turning his eyes in horror towards the door where his
daughter was standing, and trying to sit up.

``Lie down! Lie do-own!'' cried Katerina Ivanovna.

With unnatural strength he had succeeded in propping himself on his
elbow. He looked wildly and fixedly for some time on his daughter, as
though not recognising her. He had never seen her before in such attire.
Suddenly he recognised her, crushed and ashamed in her humiliation and
gaudy finery, meekly awaiting her turn to say good-bye to her dying
father. His face showed intense suffering.

``Sonia! Daughter! Forgive!'' he cried, and he tried to hold out his
hand to her, but losing his balance, he fell off the sofa, face
downwards on the floor. They rushed to pick him up, they put him on the
sofa; but he was dying. Sonia with a faint cry ran up, embraced him and
remained so without moving. He died in her arms.

``He's got what he wanted,'' Katerina Ivanovna cried, seeing her
husband's dead body. ``Well, what's to be done now? How am I to bury
him! What can I give them to-morrow to eat?''

Raskolnikov went up to Katerina Ivanovna.

``Katerina Ivanovna,'' he began, ``last week your husband told me all
his life and circumstances\ldots{}. Believe me, he spoke of you with
passionate reverence. From that evening, when I learnt how devoted he
was to you all and how he loved and respected you especially, Katerina
Ivanovna, in spite of his unfortunate weakness, from that evening we
became friends\ldots{}. Allow me now\ldots{} to do something\ldots{} to
repay my debt to my dead friend. Here are twenty roubles, I think---and
if that can be of any assistance to you, then\ldots{} I\ldots{} in
short, I will come again, I will be sure to come again\ldots{} I shall,
perhaps, come again to-morrow\ldots{}. Good-bye!''

And he went quickly out of the room, squeezing his way through the crowd
to the stairs. But in the crowd he suddenly jostled against Nikodim
Fomitch, who had heard of the accident and had come to give instructions
in person. They had not met since the scene at the police station, but
Nikodim Fomitch knew him instantly.

``Ah, is that you?'' he asked him.

``He's dead,'' answered Raskolnikov. ``The doctor and the priest have
been, all as it should have been. Don't worry the poor woman too much,
she is in consumption as it is. Try and cheer her up, if
possible\ldots{} you are a kind-hearted man, I know\ldots{}'' he added
with a smile, looking straight in his face.

``But you are spattered with blood,'' observed Nikodim Fomitch, noticing
in the lamplight some fresh stains on Raskolnikov's waistcoat.

``Yes\ldots{} I'm covered with blood,'' Raskolnikov said with a peculiar
air; then he smiled, nodded and went downstairs.

He walked down slowly and deliberately, feverish but not conscious of
it, entirely absorbed in a new overwhelming sensation of life and
strength that surged up suddenly within him. This sensation might be
compared to that of a man condemned to death who has suddenly been
pardoned. Halfway down the staircase he was overtaken by the priest on
his way home; Raskolnikov let him pass, exchanging a silent greeting
with him. He was just descending the last steps when he heard rapid
footsteps behind him. Someone overtook him; it was Polenka. She was
running after him, calling ``Wait! wait!''

He turned round. She was at the bottom of the staircase and stopped
short a step above him. A dim light came in from the yard. Raskolnikov
could distinguish the child's thin but pretty little face, looking at
him with a bright childish smile. She had run after him with a message
which she was evidently glad to give.

``Tell me, what is your name?\ldots{} and where do you live?'' she said
hurriedly in a breathless voice.

He laid both hands on her shoulders and looked at her with a sort of
rapture. It was such a joy to him to look at her, he could not have said
why.

``Who sent you?''

``Sister Sonia sent me,'' answered the girl, smiling still more
brightly.

``I knew it was sister Sonia sent you.''

``Mamma sent me, too\ldots{} when sister Sonia was sending me, mamma
came up, too, and said `Run fast, Polenka.'\,''

``Do you love sister Sonia?''

``I love her more than anyone,'' Polenka answered with a peculiar
earnestness, and her smile became graver.

``And will you love me?''

By way of answer he saw the little girl's face approaching him, her full
lips naïvely held out to kiss him. Suddenly her arms as thin as sticks
held him tightly, her head rested on his shoulder and the little girl
wept softly, pressing her face against him.

``I am sorry for father,'' she said a moment later, raising her
tear-stained face and brushing away the tears with her hands. ``It's
nothing but misfortunes now,'' she added suddenly with that peculiarly
sedate air which children try hard to assume when they want to speak
like grown-up people.

``Did your father love you?''

``He loved Lida most,'' she went on very seriously without a smile,
exactly like grown-up people, ``he loved her because she is little and
because she is ill, too. And he always used to bring her presents. But
he taught us to read and me grammar and scripture, too,'' she added with
dignity. ``And mother never used to say anything, but we knew that she
liked it and father knew it, too. And mother wants to teach me French,
for it's time my education began.''

``And do you know your prayers?''

``Of course, we do! We knew them long ago. I say my prayers to myself as
I am a big girl now, but Kolya and Lida say them aloud with mother.
First they repeat the `Ave Maria' and then another prayer: `Lord,
forgive and bless sister Sonia,' and then another, `Lord, forgive and
bless our second father.' For our elder father is dead and this is
another one, but we do pray for the other as well.''

``Polenka, my name is Rodion. Pray sometimes for me, too. `And Thy
servant Rodion,' nothing more.''

``I'll pray for you all the rest of my life,'' the little girl declared
hotly, and suddenly smiling again she rushed at him and hugged him
warmly once more.

Raskolnikov told her his name and address and promised to be sure to
come next day. The child went away quite enchanted with him. It was past
ten when he came out into the street. In five minutes he was standing on
the bridge at the spot where the woman had jumped in.

``Enough,'' he pronounced resolutely and triumphantly. ``I've done with
fancies, imaginary terrors and phantoms! Life is real! haven't I lived
just now? My life has not yet died with that old woman! The Kingdom of
Heaven to her---and now enough, madam, leave me in peace! Now for the
reign of reason and light\ldots{} and of will, and of strength\ldots{}
and now we will see! We will try our strength!'' he added defiantly, as
though challenging some power of darkness. ``And I was ready to consent
to live in a square of space!

``I am very weak at this moment, but\ldots{} I believe my illness is all
over. I knew it would be over when I went out. By the way, Potchinkov's
house is only a few steps away. I certainly must go to Razumihin even if
it were not close by\ldots{} let him win his bet! Let us give him some
satisfaction, too---no matter! Strength, strength is what one wants, you
can get nothing without it, and strength must be won by
strength---that's what they don't know,'' he added proudly and
self-confidently and he walked with flagging footsteps from the bridge.
Pride and self-confidence grew continually stronger in him; he was
becoming a different man every moment. What was it had happened to work
this revolution in him? He did not know himself; like a man catching at
a straw, he suddenly felt that he, too, `could live, that there was
still life for him, that his life had not died with the old woman.'
Perhaps he was in too great a hurry with his conclusions, but he did not
think of that.

``But I did ask her to remember `Thy servant Rodion' in her prayers,''
the idea struck him. ``Well, that was\ldots{} in case of emergency,'' he
added and laughed himself at his boyish sally. He was in the best of
spirits.

He easily found Razumihin; the new lodger was already known at
Potchinkov's and the porter at once showed him the way. Half-way
upstairs he could hear the noise and animated conversation of a big
gathering of people. The door was wide open on the stairs; he could hear
exclamations and discussion. Razumihin's room was fairly large; the
company consisted of fifteen people. Raskolnikov stopped in the entry,
where two of the landlady's servants were busy behind a screen with two
samovars, bottles, plates and dishes of pie and savouries, brought up
from the landlady's kitchen. Raskolnikov sent in for Razumihin. He ran
out delighted. At the first glance it was apparent that he had had a
great deal to drink and, though no amount of liquor made Razumihin quite
drunk, this time he was perceptibly affected by it.

``Listen,'' Raskolnikov hastened to say, ``I've only just come to tell
you you've won your bet and that no one really knows what may not happen
to him. I can't come in; I am so weak that I shall fall down directly.
And so good evening and good-bye! Come and see me to-morrow.''

``Do you know what? I'll see you home. If you say you're weak yourself,
you must\ldots{}''

``And your visitors? Who is the curly-headed one who has just peeped
out?''

``He? Goodness only knows! Some friend of uncle's, I expect, or perhaps
he has come without being invited\ldots{} I'll leave uncle with them, he
is an invaluable person, pity I can't introduce you to him now. But
confound them all now! They won't notice me, and I need a little fresh
air, for you've come just in the nick of time---another two minutes and
I should have come to blows! They are talking such a lot of wild
stuff\ldots{} you simply can't imagine what men will say! Though why
shouldn't you imagine? Don't we talk nonsense ourselves? And let
them\ldots{} that's the way to learn not to!\ldots{} Wait a minute, I'll
fetch Zossimov.''

Zossimov pounced upon Raskolnikov almost greedily; he showed a special
interest in him; soon his face brightened.

``You must go to bed at once,'' he pronounced, examining the patient as
far as he could, ``and take something for the night. Will you take it? I
got it ready some time ago\ldots{} a powder.''

``Two, if you like,'' answered Raskolnikov. The powder was taken at
once.

``It's a good thing you are taking him home,'' observed Zossimov to
Razumihin---``we shall see how he is to-morrow, to-day he's not at all
amiss---a considerable change since the afternoon. Live and
learn\ldots{}''

``Do you know what Zossimov whispered to me when we were coming out?''
Razumihin blurted out, as soon as they were in the street. ``I won't
tell you everything, brother, because they are such fools. Zossimov told
me to talk freely to you on the way and get you to talk freely to me,
and afterwards I am to tell him about it, for he's got a notion in his
head that you are\ldots{} mad or close on it. Only fancy! In the first
place, you've three times the brains he has; in the second, if you are
not mad, you needn't care a hang that he has got such a wild idea; and
thirdly, that piece of beef whose specialty is surgery has gone mad on
mental diseases, and what's brought him to this conclusion about you was
your conversation to-day with Zametov.''

``Zametov told you all about it?''

``Yes, and he did well. Now I understand what it all means and so does
Zametov\ldots{}. Well, the fact is, Rodya\ldots{} the point is\ldots{} I
am a little drunk now\ldots{}. But that's\ldots{} no matter\ldots{} the
point is that this idea\ldots{} you understand? was just being hatched
in their brains\ldots{} you understand? That is, no one ventured to say
it aloud, because the idea is too absurd and especially since the arrest
of that painter, that bubble's burst and gone for ever. But why are they
such fools? I gave Zametov a bit of a thrashing at the time---that's
between ourselves, brother; please don't let out a hint that you know of
it; I've noticed he is a ticklish subject; it was at Luise Ivanovna's.
But to-day, to-day it's all cleared up. That Ilya Petrovitch is at the
bottom of it! He took advantage of your fainting at the police station,
but he is ashamed of it himself now; I know that\ldots{}''

Raskolnikov listened greedily. Razumihin was drunk enough to talk too
freely.

``I fainted then because it was so close and the smell of paint,'' said
Raskolnikov.

``No need to explain that! And it wasn't the paint only: the fever had
been coming on for a month; Zossimov testifies to that! But how crushed
that boy is now, you wouldn't believe! `I am not worth his little
finger,' he says. Yours, he means. He has good feelings at times,
brother. But the lesson, the lesson you gave him to-day in the Palais de
Cristal, that was too good for anything! You frightened him at first,
you know, he nearly went into convulsions! You almost convinced him
again of the truth of all that hideous nonsense, and then you
suddenly---put out your tongue at him: `There now, what do you make of
it?' It was perfect! He is crushed, annihilated now! It was masterly, by
Jove, it's what they deserve! Ah, that I wasn't there! He was hoping to
see you awfully. Porfiry, too, wants to make your acquaintance\ldots{}''

``Ah!\ldots{} he too\ldots{} but why did they put me down as mad?''

``Oh, not mad. I must have said too much, brother\ldots{}. What struck
him, you see, was that only that subject seemed to interest you; now
it's clear why it did interest you; knowing all the
circumstances\ldots{} and how that irritated you and worked in with your
illness\ldots{} I am a little drunk, brother, only, confound him, he has
some idea of his own\ldots{} I tell you, he's mad on mental diseases.
But don't you mind him\ldots{}''

For half a minute both were silent.

``Listen, Razumihin,'' began Raskolnikov, ``I want to tell you plainly:
I've just been at a death-bed, a clerk who died\ldots{} I gave them all
my money\ldots{} and besides I've just been kissed by someone who, if I
had killed anyone, would just the same\ldots{} in fact I saw someone
else there\ldots{} with a flame-coloured feather\ldots{} but I am
talking nonsense; I am very weak, support me\ldots{} we shall be at the
stairs directly\ldots{}''

``What's the matter? What's the matter with you?'' Razumihin asked
anxiously.

``I am a little giddy, but that's not the point, I am so sad, so
sad\ldots{} like a woman. Look, what's that? Look, look!''

``What is it?''

``Don't you see? A light in my room, you see? Through the
crack\ldots{}''

They were already at the foot of the last flight of stairs, at the level
of the landlady's door, and they could, as a fact, see from below that
there was a light in Raskolnikov's garret.

``Queer! Nastasya, perhaps,'' observed Razumihin.

``She is never in my room at this time and she must be in bed long ago,
but\ldots{} I don't care! Good-bye!''

``What do you mean? I am coming with you, we'll come in together!''

``I know we are going in together, but I want to shake hands here and
say good-bye to you here. So give me your hand, good-bye!''

``What's the matter with you, Rodya?''

``Nothing\ldots{} come along\ldots{} you shall be witness.''

They began mounting the stairs, and the idea struck Razumihin that
perhaps Zossimov might be right after all. ``Ah, I've upset him with my
chatter!'' he muttered to himself.

When they reached the door they heard voices in the room.

``What is it?'' cried Razumihin. Raskolnikov was the first to open the
door; he flung it wide and stood still in the doorway, dumbfoundered.

His mother and sister were sitting on his sofa and had been waiting an
hour and a half for him. Why had he never expected, never thought of
them, though the news that they had started, were on their way and would
arrive immediately, had been repeated to him only that day? They had
spent that hour and a half plying Nastasya with questions. She was
standing before them and had told them everything by now. They were
beside themselves with alarm when they heard of his ``running away''
to-day, ill and, as they understood from her story, delirious! ``Good
Heavens, what had become of him?'' Both had been weeping, both had been
in anguish for that hour and a half.

A cry of joy, of ecstasy, greeted Raskolnikov's entrance. Both rushed to
him. But he stood like one dead; a sudden intolerable sensation struck
him like a thunderbolt. He did not lift his arms to embrace them, he
could not. His mother and sister clasped him in their arms, kissed him,
laughed and cried. He took a step, tottered and fell to the ground,
fainting.

Anxiety, cries of horror, moans\ldots{} Razumihin who was standing in
the doorway flew into the room, seized the sick man in his strong arms
and in a moment had him on the sofa.

``It's nothing, nothing!'' he cried to the mother and sister---``it's
only a faint, a mere trifle! Only just now the doctor said he was much
better, that he is perfectly well! Water! See, he is coming to himself,
he is all right again!''

And seizing Dounia by the arm so that he almost dislocated it, he made
her bend down to see that ``he is all right again.'' The mother and
sister looked on him with emotion and gratitude, as their Providence.
They had heard already from Nastasya all that had been done for their
Rodya during his illness, by this ``very competent young man,'' as
Pulcheria Alexandrovna Raskolnikov called him that evening in
conversation with Dounia.

\part*{PART III}
\addcontentsline{toc}{part}{PART III}

\mychapter{1}{Chapter I}\label{chapter-i-2}

Raskolnikov got up, and sat down on the sofa. He waved his hand weakly
to Razumihin to cut short the flow of warm and incoherent consolations
he was addressing to his mother and sister, took them both by the hand
and for a minute or two gazed from one to the other without speaking.
His mother was alarmed by his expression. It revealed an emotion
agonisingly poignant, and at the same time something immovable, almost
insane. Pulcheria Alexandrovna began to cry.

Avdotya Romanovna was pale; her hand trembled in her brother's.

``Go home\ldots{} with him,'' he said in a broken voice, pointing to
Razumihin, ``good-bye till to-morrow; to-morrow everything\ldots{} Is it
long since you arrived?''

``This evening, Rodya,'' answered Pulcheria Alexandrovna, ``the train
was awfully late. But, Rodya, nothing would induce me to leave you now!
I will spend the night here, near you\ldots{}''

``Don't torture me!'' he said with a gesture of irritation.

``I will stay with him,'' cried Razumihin, ``I won't leave him for a
moment. Bother all my visitors! Let them rage to their hearts' content!
My uncle is presiding there.''

``How, how can I thank you!'' Pulcheria Alexandrovna was beginning, once
more pressing Razumihin's hands, but Raskolnikov interrupted her again.

``I can't have it! I can't have it!'' he repeated irritably, ``don't
worry me! Enough, go away\ldots{} I can't stand it!''

``Come, mamma, come out of the room at least for a minute,'' Dounia
whispered in dismay; ``we are distressing him, that's evident.''

``Mayn't I look at him after three years?'' wept Pulcheria Alexandrovna.

``Stay,'' he stopped them again, ``you keep interrupting me, and my
ideas get muddled\ldots{}. Have you seen Luzhin?''

``No, Rodya, but he knows already of our arrival. We have heard, Rodya,
that Pyotr Petrovitch was so kind as to visit you today,'' Pulcheria
Alexandrovna added somewhat timidly.

``Yes\ldots{} he was so kind\ldots{} Dounia, I promised Luzhin I'd throw
him downstairs and told him to go to hell\ldots{}.''

``Rodya, what are you saying! Surely, you don't mean to tell
us\ldots{}'' Pulcheria Alexandrovna began in alarm, but she stopped,
looking at Dounia.

Avdotya Romanovna was looking attentively at her brother, waiting for
what would come next. Both of them had heard of the quarrel from
Nastasya, so far as she had succeeded in understanding and reporting it,
and were in painful perplexity and suspense.

``Dounia,'' Raskolnikov continued with an effort, ``I don't want that
marriage, so at the first opportunity to-morrow you must refuse Luzhin,
so that we may never hear his name again.''

``Good Heavens!'' cried Pulcheria Alexandrovna.

``Brother, think what you are saying!'' Avdotya Romanovna began
impetuously, but immediately checked herself. ``You are not fit to talk
now, perhaps; you are tired,'' she added gently.

``You think I am delirious? No\ldots{} You are marrying Luzhin for
\emph{my} sake. But I won't accept the sacrifice. And so write a letter
before to-morrow, to refuse him\ldots{} Let me read it in the morning
and that will be the end of it!''

``That I can't do!'' the girl cried, offended, ``what right have
you\ldots{}''

``Dounia, you are hasty, too, be quiet, to-morrow\ldots{} Don't you
see\ldots{}'' the mother interposed in dismay. ``Better come away!''

``He is raving,'' Razumihin cried tipsily, ``or how would he dare!
To-morrow all this nonsense will be over\ldots{} to-day he certainly did
drive him away. That was so. And Luzhin got angry, too\ldots{}. He made
speeches here, wanted to show off his learning and he went out
crest-fallen\ldots{}.''

``Then it's true?'' cried Pulcheria Alexandrovna.

``Good-bye till to-morrow, brother,'' said Dounia
compassionately---``let us go, mother\ldots{} Good-bye, Rodya.''

``Do you hear, sister,'' he repeated after them, making a last effort,
``I am not delirious; this marriage is---an infamy. Let me act like a
scoundrel, but you mustn't\ldots{} one is enough\ldots{} and though I am
a scoundrel, I wouldn't own such a sister. It's me or Luzhin! Go
now\ldots{}.''

``But you're out of your mind! Despot!'' roared Razumihin; but
Raskolnikov did not and perhaps could not answer. He lay down on the
sofa, and turned to the wall, utterly exhausted. Avdotya Romanovna
looked with interest at Razumihin; her black eyes flashed; Razumihin
positively started at her glance.

Pulcheria Alexandrovna stood overwhelmed.

``Nothing would induce me to go,'' she whispered in despair to
Razumihin. ``I will stay somewhere here\ldots{} escort Dounia home.''

``You'll spoil everything,'' Razumihin answered in the same whisper,
losing patience---``come out on to the stairs, anyway. Nastasya, show a
light! I assure you,'' he went on in a half whisper on the
stairs---``that he was almost beating the doctor and me this afternoon!
Do you understand? The doctor himself! Even he gave way and left him, so
as not to irritate him. I remained downstairs on guard, but he dressed
at once and slipped off. And he will slip off again if you irritate him,
at this time of night, and will do himself some mischief\ldots{}.''

``What are you saying?''

``And Avdotya Romanovna can't possibly be left in those lodgings without
you. Just think where you are staying! That blackguard Pyotr Petrovitch
couldn't find you better lodgings\ldots{} But you know I've had a little
to drink, and that's what makes me\ldots{} swear; don't mind
it\ldots{}.''

``But I'll go to the landlady here,'' Pulcheria Alexandrovna insisted,
``I'll beseech her to find some corner for Dounia and me for the night.
I can't leave him like that, I cannot!''

This conversation took place on the landing just before the landlady's
door. Nastasya lighted them from a step below. Razumihin was in
extraordinary excitement. Half an hour earlier, while he was bringing
Raskolnikov home, he had indeed talked too freely, but he was aware of
it himself, and his head was clear in spite of the vast quantities he
had imbibed. Now he was in a state bordering on ecstasy, and all that he
had drunk seemed to fly to his head with redoubled effect. He stood with
the two ladies, seizing both by their hands, persuading them, and giving
them reasons with astonishing plainness of speech, and at almost every
word he uttered, probably to emphasise his arguments, he squeezed their
hands painfully as in a vise. He stared at Avdotya Romanovna without the
least regard for good manners. They sometimes pulled their hands out of
his huge bony paws, but far from noticing what was the matter, he drew
them all the closer to him. If they'd told him to jump head foremost
from the staircase, he would have done it without thought or hesitation
in their service. Though Pulcheria Alexandrovna felt that the young man
was really too eccentric and pinched her hand too much, in her anxiety
over her Rodya she looked on his presence as providential, and was
unwilling to notice all his peculiarities. But though Avdotya Romanovna
shared her anxiety, and was not of timorous disposition, she could not
see the glowing light in his eyes without wonder and almost alarm. It
was only the unbounded confidence inspired by Nastasya's account of her
brother's queer friend, which prevented her from trying to run away from
him, and to persuade her mother to do the same. She realised, too, that
even running away was perhaps impossible now. Ten minutes later,
however, she was considerably reassured; it was characteristic of
Razumihin that he showed his true nature at once, whatever mood he might
be in, so that people quickly saw the sort of man they had to deal with.

``You can't go to the landlady, that's perfect nonsense!'' he cried.
``If you stay, though you are his mother, you'll drive him to a frenzy,
and then goodness knows what will happen! Listen, I'll tell you what
I'll do: Nastasya will stay with him now, and I'll conduct you both
home, you can't be in the streets alone; Petersburg is an awful place in
that way\ldots{}. But no matter! Then I'll run straight back here and a
quarter of an hour later, on my word of honour, I'll bring you news how
he is, whether he is asleep, and all that. Then, listen! Then I'll run
home in a twinkling---I've a lot of friends there, all drunk---I'll
fetch Zossimov---that's the doctor who is looking after him, he is
there, too, but he is not drunk; he is not drunk, he is never drunk!
I'll drag him to Rodya, and then to you, so that you'll get two reports
in the hour---from the doctor, you understand, from the doctor himself,
that's a very different thing from my account of him! If there's
anything wrong, I swear I'll bring you here myself, but, if it's all
right, you go to bed. And I'll spend the night here, in the passage, he
won't hear me, and I'll tell Zossimov to sleep at the landlady's, to be
at hand. Which is better for him: you or the doctor? So come home then!
But the landlady is out of the question; it's all right for me, but it's
out of the question for you: she wouldn't take you, for she's\ldots{}
for she's a fool\ldots{} She'd be jealous on my account of Avdotya
Romanovna and of you, too, if you want to know\ldots{} of Avdotya
Romanovna certainly. She is an absolutely, absolutely unaccountable
character! But I am a fool, too!\ldots{} No matter! Come along! Do you
trust me? Come, do you trust me or not?''

``Let us go, mother,'' said Avdotya Romanovna, ``he will certainly do
what he has promised. He has saved Rodya already, and if the doctor
really will consent to spend the night here, what could be better?''

``You see, you\ldots{} you\ldots{} understand me, because you are an
angel!'' Razumihin cried in ecstasy, ``let us go! Nastasya! Fly upstairs
and sit with him with a light; I'll come in a quarter of an hour.''

Though Pulcheria Alexandrovna was not perfectly convinced, she made no
further resistance. Razumihin gave an arm to each and drew them down the
stairs. He still made her uneasy, as though he was competent and
good-natured, was he capable of carrying out his promise? He seemed in
such a condition\ldots{}.

``Ah, I see you think I am in such a condition!'' Razumihin broke in
upon her thoughts, guessing them, as he strolled along the pavement with
huge steps, so that the two ladies could hardly keep up with him, a fact
he did not observe, however. ``Nonsense! That is\ldots{} I am drunk like
a fool, but that's not it; I am not drunk from wine. It's seeing you has
turned my head\ldots{} But don't mind me! Don't take any notice: I am
talking nonsense, I am not worthy of you\ldots{}. I am utterly unworthy
of you! The minute I've taken you home, I'll pour a couple of pailfuls
of water over my head in the gutter here, and then I shall be all
right\ldots{}. If only you knew how I love you both! Don't laugh, and
don't be angry! You may be angry with anyone, but not with me! I am his
friend, and therefore I am your friend, too, I want to be\ldots{} I had
a presentiment\ldots{} Last year there was a moment\ldots{} though it
wasn't a presentiment really, for you seem to have fallen from heaven.
And I expect I shan't sleep all night\ldots{} Zossimov was afraid a
little time ago that he would go mad\ldots{} that's why he mustn't be
irritated.''

``What do you say?'' cried the mother.

``Did the doctor really say that?'' asked Avdotya Romanovna, alarmed.

``Yes, but it's not so, not a bit of it. He gave him some medicine, a
powder, I saw it, and then your coming here\ldots{}. Ah! It would have
been better if you had come to-morrow. It's a good thing we went away.
And in an hour Zossimov himself will report to you about everything. He
is not drunk! And I shan't be drunk\ldots{}. And what made me get so
tight? Because they got me into an argument, damn them! I've sworn never
to argue! They talk such trash! I almost came to blows! I've left my
uncle to preside. Would you believe, they insist on complete absence of
individualism and that's just what they relish! Not to be themselves, to
be as unlike themselves as they can. That's what they regard as the
highest point of progress. If only their nonsense were their own, but as
it is\ldots{}''

``Listen!'' Pulcheria Alexandrovna interrupted timidly, but it only
added fuel to the flames.

``What do you think?'' shouted Razumihin, louder than ever, ``you think
I am attacking them for talking nonsense? Not a bit! I like them to talk
nonsense. That's man's one privilege over all creation. Through error
you come to the truth! I am a man because I err! You never reach any
truth without making fourteen mistakes and very likely a hundred and
fourteen. And a fine thing, too, in its way; but we can't even make
mistakes on our own account! Talk nonsense, but talk your own nonsense,
and I'll kiss you for it. To go wrong in one's own way is better than to
go right in someone else's. In the first case you are a man, in the
second you're no better than a bird. Truth won't escape you, but life
can be cramped. There have been examples. And what are we doing now? In
science, development, thought, invention, ideals, aims, liberalism,
judgment, experience and everything, everything, everything, we are
still in the preparatory class at school. We prefer to live on other
people's ideas, it's what we are used to! Am I right, am I right?''
cried Razumihin, pressing and shaking the two ladies' hands.

``Oh, mercy, I do not know,'' cried poor Pulcheria Alexandrovna.

``Yes, yes\ldots{} though I don't agree with you in everything,'' added
Avdotya Romanovna earnestly and at once uttered a cry, for he squeezed
her hand so painfully.

``Yes, you say yes\ldots{} well after that you\ldots{} you\ldots{}'' he
cried in a transport, ``you are a fount of goodness, purity,
sense\ldots{} and perfection. Give me your hand\ldots{} you give me
yours, too! I want to kiss your hands here at once, on my
knees\ldots{}'' and he fell on his knees on the pavement, fortunately at
that time deserted.

``Leave off, I entreat you, what are you doing?'' Pulcheria Alexandrovna
cried, greatly distressed.

``Get up, get up!'' said Dounia laughing, though she, too, was upset.

``Not for anything till you let me kiss your hands! That's it! Enough! I
get up and we'll go on! I am a luckless fool, I am unworthy of you and
drunk\ldots{} and I am ashamed\ldots{}. I am not worthy to love you, but
to do homage to you is the duty of every man who is not a perfect beast!
And I've done homage\ldots{}. Here are your lodgings, and for that alone
Rodya was right in driving your Pyotr Petrovitch away\ldots{}. How dare
he! how dare he put you in such lodgings! It's a scandal! Do you know
the sort of people they take in here? And you his betrothed! You are his
betrothed? Yes? Well, then, I'll tell you, your \emph{fiancé} is a
scoundrel.''

``Excuse me, Mr. Razumihin, you are forgetting\ldots{}'' Pulcheria
Alexandrovna was beginning.

``Yes, yes, you are right, I did forget myself, I am ashamed of it,''
Razumihin made haste to apologise. ``But\ldots{} but you can't be angry
with me for speaking so! For I speak sincerely and not because\ldots{}
hm, hm! That would be disgraceful; in fact not because I'm in\ldots{}
hm! Well, anyway, I won't say why, I daren't\ldots{}. But we all saw
to-day when he came in that that man is not of our sort. Not because he
had his hair curled at the barber's, not because he was in such a hurry
to show his wit, but because he is a spy, a speculator, because he is a
skin-flint and a buffoon. That's evident. Do you think him clever? No,
he is a fool, a fool. And is he a match for you? Good heavens! Do you
see, ladies?'' he stopped suddenly on the way upstairs to their rooms,
``though all my friends there are drunk, yet they are all honest, and
though we do talk a lot of trash, and I do, too, yet we shall talk our
way to the truth at last, for we are on the right path, while Pyotr
Petrovitch\ldots{} is not on the right path. Though I've been calling
them all sorts of names just now, I do respect them all\ldots{} though I
don't respect Zametov, I like him, for he is a puppy, and that bullock
Zossimov, because he is an honest man and knows his work. But enough,
it's all said and forgiven. Is it forgiven? Well, then, let's go on. I
know this corridor, I've been here, there was a scandal here at Number
3\ldots{}. Where are you here? Which number? eight? Well, lock
yourselves in for the night, then. Don't let anybody in. In a quarter of
an hour I'll come back with news, and half an hour later I'll bring
Zossimov, you'll see! Good-bye, I'll run.''

``Good heavens, Dounia, what is going to happen?'' said Pulcheria
Alexandrovna, addressing her daughter with anxiety and dismay.

``Don't worry yourself, mother,'' said Dounia, taking off her hat and
cape. ``God has sent this gentleman to our aid, though he has come from
a drinking party. We can depend on him, I assure you. And all that he
has done for Rodya\ldots{}.''

``Ah. Dounia, goodness knows whether he will come! How could I bring
myself to leave Rodya?\ldots{} And how different, how different I had
fancied our meeting! How sullen he was, as though not pleased to see
us\ldots{}.''

Tears came into her eyes.

``No, it's not that, mother. You didn't see, you were crying all the
time. He is quite unhinged by serious illness---that's the reason.''

``Ah, that illness! What will happen, what will happen? And how he
talked to you, Dounia!'' said the mother, looking timidly at her
daughter, trying to read her thoughts and, already half consoled by
Dounia's standing up for her brother, which meant that she had already
forgiven him. ``I am sure he will think better of it to-morrow,'' she
added, probing her further.

``And I am sure that he will say the same to-morrow\ldots{} about
that,'' Avdotya Romanovna said finally. And, of course, there was no
going beyond that, for this was a point which Pulcheria Alexandrovna was
afraid to discuss. Dounia went up and kissed her mother. The latter
warmly embraced her without speaking. Then she sat down to wait
anxiously for Razumihin's return, timidly watching her daughter who
walked up and down the room with her arms folded, lost in thought. This
walking up and down when she was thinking was a habit of Avdotya
Romanovna's and the mother was always afraid to break in on her
daughter's mood at such moments.

Razumihin, of course, was ridiculous in his sudden drunken infatuation
for Avdotya Romanovna. Yet apart from his eccentric condition, many
people would have thought it justified if they had seen Avdotya
Romanovna, especially at that moment when she was walking to and fro
with folded arms, pensive and melancholy. Avdotya Romanovna was
remarkably good-looking; she was tall, strikingly well-proportioned,
strong and self-reliant---the latter quality was apparent in every
gesture, though it did not in the least detract from the grace and
softness of her movements. In face she resembled her brother, but she
might be described as really beautiful. Her hair was dark brown, a
little lighter than her brother's; there was a proud light in her almost
black eyes and yet at times a look of extraordinary kindness. She was
pale, but it was a healthy pallor; her face was radiant with freshness
and vigour. Her mouth was rather small; the full red lower lip projected
a little as did her chin; it was the only irregularity in her beautiful
face, but it gave it a peculiarly individual and almost haughty
expression. Her face was always more serious and thoughtful than gay;
but how well smiles, how well youthful, lighthearted, irresponsible,
laughter suited her face! It was natural enough that a warm, open,
simple-hearted, honest giant like Razumihin, who had never seen anyone
like her and was not quite sober at the time, should lose his head
immediately. Besides, as chance would have it, he saw Dounia for the
first time transfigured by her love for her brother and her joy at
meeting him. Afterwards he saw her lower lip quiver with indignation at
her brother's insolent, cruel and ungrateful words---and his fate was
sealed.

He had spoken the truth, moreover, when he blurted out in his drunken
talk on the stairs that Praskovya Pavlovna, Raskolnikov's eccentric
landlady, would be jealous of Pulcheria Alexandrovna as well as of
Avdotya Romanovna on his account. Although Pulcheria Alexandrovna was
forty-three, her face still retained traces of her former beauty; she
looked much younger than her age, indeed, which is almost always the
case with women who retain serenity of spirit, sensitiveness and pure
sincere warmth of heart to old age. We may add in parenthesis that to
preserve all this is the only means of retaining beauty to old age. Her
hair had begun to grow grey and thin, there had long been little crow's
foot wrinkles round her eyes, her cheeks were hollow and sunken from
anxiety and grief, and yet it was a handsome face. She was Dounia over
again, twenty years older, but without the projecting underlip.
Pulcheria Alexandrovna was emotional, but not sentimental, timid and
yielding, but only to a certain point. She could give way and accept a
great deal even of what was contrary to her convictions, but there was a
certain barrier fixed by honesty, principle and the deepest convictions
which nothing would induce her to cross.

Exactly twenty minutes after Razumihin's departure, there came two
subdued but hurried knocks at the door: he had come back.

``I won't come in, I haven't time,'' he hastened to say when the door
was opened. ``He sleeps like a top, soundly, quietly, and God grant he
may sleep ten hours. Nastasya's with him; I told her not to leave till I
came. Now I am fetching Zossimov, he will report to you and then you'd
better turn in; I can see you are too tired to do anything\ldots{}.''

And he ran off down the corridor.

``What a very competent and\ldots{} devoted young man!'' cried Pulcheria
Alexandrovna exceedingly delighted.

``He seems a splendid person!'' Avdotya Romanovna replied with some
warmth, resuming her walk up and down the room.

It was nearly an hour later when they heard footsteps in the corridor
and another knock at the door. Both women waited this time completely
relying on Razumihin's promise; he actually had succeeded in bringing
Zossimov. Zossimov had agreed at once to desert the drinking party to go
to Raskolnikov's, but he came reluctantly and with the greatest
suspicion to see the ladies, mistrusting Razumihin in his exhilarated
condition. But his vanity was at once reassured and flattered; he saw
that they were really expecting him as an oracle. He stayed just ten
minutes and succeeded in completely convincing and comforting Pulcheria
Alexandrovna. He spoke with marked sympathy, but with the reserve and
extreme seriousness of a young doctor at an important consultation. He
did not utter a word on any other subject and did not display the
slightest desire to enter into more personal relations with the two
ladies. Remarking at his first entrance the dazzling beauty of Avdotya
Romanovna, he endeavoured not to notice her at all during his visit and
addressed himself solely to Pulcheria Alexandrovna. All this gave him
extraordinary inward satisfaction. He declared that he thought the
invalid at this moment going on very satisfactorily. According to his
observations the patient's illness was due partly to his unfortunate
material surroundings during the last few months, but it had partly also
a moral origin, ``was, so to speak, the product of several material and
moral influences, anxieties, apprehensions, troubles, certain
ideas\ldots{} and so on.'' Noticing stealthily that Avdotya Romanovna
was following his words with close attention, Zossimov allowed himself
to enlarge on this theme. On Pulcheria Alexandrovna's anxiously and
timidly inquiring as to ``some suspicion of insanity,'' he replied with
a composed and candid smile that his words had been exaggerated; that
certainly the patient had some fixed idea, something approaching a
monomania---he, Zossimov, was now particularly studying this interesting
branch of medicine---but that it must be recollected that until to-day
the patient had been in delirium and\ldots{} and that no doubt the
presence of his family would have a favourable effect on his recovery
and distract his mind, ``if only all fresh shocks can be avoided,'' he
added significantly. Then he got up, took leave with an impressive and
affable bow, while blessings, warm gratitude, and entreaties were
showered upon him, and Avdotya Romanovna spontaneously offered her hand
to him. He went out exceedingly pleased with his visit and still more so
with himself.

``We'll talk to-morrow; go to bed at once!'' Razumihin said in
conclusion, following Zossimov out. ``I'll be with you to-morrow morning
as early as possible with my report.''

``That's a fetching little girl, Avdotya Romanovna,'' remarked Zossimov,
almost licking his lips as they both came out into the street.

``Fetching? You said fetching?'' roared Razumihin and he flew at
Zossimov and seized him by the throat. ``If you ever dare\ldots{}. Do
you understand? Do you understand?'' he shouted, shaking him by the
collar and squeezing him against the wall. ``Do you hear?''

``Let me go, you drunken devil,'' said Zossimov, struggling and when he
had let him go, he stared at him and went off into a sudden guffaw.
Razumihin stood facing him in gloomy and earnest reflection.

``Of course, I am an ass,'' he observed, sombre as a storm cloud, ``but
still\ldots{} you are another.''

``No, brother, not at all such another. I am not dreaming of any
folly.''

They walked along in silence and only when they were close to
Raskolnikov's lodgings, Razumihin broke the silence in considerable
anxiety.

``Listen,'' he said, ``you're a first-rate fellow, but among your other
failings, you're a loose fish, that I know, and a dirty one, too. You
are a feeble, nervous wretch, and a mass of whims, you're getting fat
and lazy and can't deny yourself anything---and I call that dirty
because it leads one straight into the dirt. You've let yourself get so
slack that I don't know how it is you are still a good, even a devoted
doctor. You---a doctor---sleep on a feather bed and get up at night to
your patients! In another three or four years you won't get up for your
patients\ldots{} But hang it all, that's not the point!\ldots{} You are
going to spend to-night in the landlady's flat here. (Hard work I've had
to persuade her!) And I'll be in the kitchen. So here's a chance for you
to get to know her better\ldots{}. It's not as you think! There's not a
trace of anything of the sort, brother\ldots{}!''

``But I don't think!''

``Here you have modesty, brother, silence, bashfulness, a savage
virtue\ldots{} and yet she's sighing and melting like wax, simply
melting! Save me from her, by all that's unholy! She's most
prepossessing\ldots{} I'll repay you, I'll do anything\ldots{}.''

Zossimov laughed more violently than ever.

``Well, you are smitten! But what am I to do with her?''

``It won't be much trouble, I assure you. Talk any rot you like to her,
as long as you sit by her and talk. You're a doctor, too; try curing her
of something. I swear you won't regret it. She has a piano, and you
know, I strum a little. I have a song there, a genuine Russian one: `I
shed hot tears.' She likes the genuine article---and well, it all began
with that song; Now you're a regular performer, a \emph{maître}, a
Rubinstein\ldots{}. I assure you, you won't regret it!''

``But have you made her some promise? Something signed? A promise of
marriage, perhaps?''

``Nothing, nothing, absolutely nothing of the kind! Besides she is not
that sort at all\ldots{}. Tchebarov tried that\ldots{}.''

``Well then, drop her!''

``But I can't drop her like that!''

``Why can't you?''

``Well, I can't, that's all about it! There's an element of attraction
here, brother.''

``Then why have you fascinated her?''

``I haven't fascinated her; perhaps I was fascinated myself in my folly.
But she won't care a straw whether it's you or I, so long as somebody
sits beside her, sighing\ldots{}. I can't explain the position,
brother\ldots{} look here, you are good at mathematics, and working at
it now\ldots{} begin teaching her the integral calculus; upon my soul,
I'm not joking, I'm in earnest, it'll be just the same to her. She will
gaze at you and sigh for a whole year together. I talked to her once for
two days at a time about the Prussian House of Lords (for one must talk
of something)---she just sighed and perspired! And you mustn't talk of
love---she's bashful to hysterics---but just let her see you can't tear
yourself away---that's enough. It's fearfully comfortable; you're quite
at home, you can read, sit, lie about, write. You may even venture on a
kiss, if you're careful.''

``But what do I want with her?''

``Ach, I can't make you understand! You see, you are made for each
other! I have often been reminded of you!\ldots{} You'll come to it in
the end! So does it matter whether it's sooner or later? There's the
feather-bed element here, brother---ach! and not only that! There's an
attraction here---here you have the end of the world, an anchorage, a
quiet haven, the navel of the earth, the three fishes that are the
foundation of the world, the essence of pancakes, of savoury fish-pies,
of the evening samovar, of soft sighs and warm shawls, and hot stoves to
sleep on---as snug as though you were dead, and yet you're alive---the
advantages of both at once! Well, hang it, brother, what stuff I'm
talking, it's bedtime! Listen. I sometimes wake up at night; so I'll go
in and look at him. But there's no need, it's all right. Don't you worry
yourself, yet if you like, you might just look in once, too. But if you
notice anything---delirium or fever---wake me at once. But there can't
be\ldots{}.''

\protect\hypertarget{link2HCH0016}{}{}

\mychapter{2}{Chapter II}\label{chapter-ii-2}

Razumihin waked up next morning at eight o'clock, troubled and serious.
He found himself confronted with many new and unlooked-for perplexities.
He had never expected that he would ever wake up feeling like that. He
remembered every detail of the previous day and he knew that a perfectly
novel experience had befallen him, that he had received an impression
unlike anything he had known before. At the same time he recognised
clearly that the dream which had fired his imagination was hopelessly
unattainable---so unattainable that he felt positively ashamed of it,
and he hastened to pass to the other more practical cares and
difficulties bequeathed him by that ``thrice accursed yesterday.''

The most awful recollection of the previous day was the way he had shown
himself ``base and mean,'' not only because he had been drunk, but
because he had taken advantage of the young girl's position to abuse her
\emph{fiancé} in his stupid jealousy, knowing nothing of their mutual
relations and obligations and next to nothing of the man himself. And
what right had he to criticise him in that hasty and unguarded manner?
Who had asked for his opinion? Was it thinkable that such a creature as
Avdotya Romanovna would be marrying an unworthy man for money? So there
must be something in him. The lodgings? But after all how could he know
the character of the lodgings? He was furnishing a flat\ldots{} Foo! how
despicable it all was! And what justification was it that he was drunk?
Such a stupid excuse was even more degrading! In wine is truth, and the
truth had all come out, ``that is, all the uncleanness of his coarse and
envious heart''! And would such a dream ever be permissible to him,
Razumihin? What was he beside such a girl---he, the drunken noisy
braggart of last night? Was it possible to imagine so absurd and cynical
a juxtaposition? Razumihin blushed desperately at the very idea and
suddenly the recollection forced itself vividly upon him of how he had
said last night on the stairs that the landlady would be jealous of
Avdotya Romanovna\ldots{} that was simply intolerable. He brought his
fist down heavily on the kitchen stove, hurt his hand and sent one of
the bricks flying.

``Of course,'' he muttered to himself a minute later with a feeling of
self-abasement, ``of course, all these infamies can never be wiped out
or smoothed over\ldots{} and so it's useless even to think of it, and I
must go to them in silence and do my duty\ldots{} in silence,
too\ldots{} and not ask forgiveness, and say nothing\ldots{} for all is
lost now!''

And yet as he dressed he examined his attire more carefully than usual.
He hadn't another suit---if he had had, perhaps he wouldn't have put it
on. ``I would have made a point of not putting it on.'' But in any case
he could not remain a cynic and a dirty sloven; he had no right to
offend the feelings of others, especially when they were in need of his
assistance and asking him to see them. He brushed his clothes carefully.
His linen was always decent; in that respect he was especially clean.

He washed that morning scrupulously---he got some soap from
Nastasya---he washed his hair, his neck and especially his hands. When
it came to the question whether to shave his stubbly chin or not
(Praskovya Pavlovna had capital razors that had been left by her late
husband), the question was angrily answered in the negative. ``Let it
stay as it is! What if they think that I shaved on purpose to\ldots{}?
They certainly would think so! Not on any account!''

``And\ldots{} the worst of it was he was so coarse, so dirty, he had the
manners of a pothouse; and\ldots{} and even admitting that he knew he
had some of the essentials of a gentleman\ldots{} what was there in that
to be proud of? Everyone ought to be a gentleman and more than
that\ldots{} and all the same (he remembered) he, too, had done little
things\ldots{} not exactly dishonest, and yet\ldots{}. And what thoughts
he sometimes had; hm\ldots{} and to set all that beside Avdotya
Romanovna! Confound it! So be it! Well, he'd make a point then of being
dirty, greasy, pothouse in his manners and he wouldn't care! He'd be
worse!''

He was engaged in such monologues when Zossimov, who had spent the night
in Praskovya Pavlovna's parlour, came in.

He was going home and was in a hurry to look at the invalid first.
Razumihin informed him that Raskolnikov was sleeping like a dormouse.
Zossimov gave orders that they shouldn't wake him and promised to see
him again about eleven.

``If he is still at home,'' he added. ``Damn it all! If one can't
control one's patients, how is one to cure them? Do you know whether
\emph{he} will go to them, or whether \emph{they} are coming here?''

``They are coming, I think,'' said Razumihin, understanding the object
of the question, ``and they will discuss their family affairs, no doubt.
I'll be off. You, as the doctor, have more right to be here than I.''

``But I am not a father confessor; I shall come and go away; I've plenty
to do besides looking after them.''

``One thing worries me,'' interposed Razumihin, frowning. ``On the way
home I talked a lot of drunken nonsense to him\ldots{} all sorts of
things\ldots{} and amongst them that you were afraid that he\ldots{}
might become insane.''

``You told the ladies so, too.''

``I know it was stupid! You may beat me if you like! Did you think so
seriously?''

``That's nonsense, I tell you, how could I think it seriously? You,
yourself, described him as a monomaniac when you fetched me to
him\ldots{} and we added fuel to the fire yesterday, you did, that is,
with your story about the painter; it was a nice conversation, when he
was, perhaps, mad on that very point! If only I'd known what happened
then at the police station and that some wretch\ldots{} had insulted him
with this suspicion! Hm\ldots{} I would not have allowed that
conversation yesterday. These monomaniacs will make a mountain out of a
mole-hill\ldots{} and see their fancies as solid realities\ldots{}. As
far as I remember, it was Zametov's story that cleared up half the
mystery, to my mind. Why, I know one case in which a hypochondriac, a
man of forty, cut the throat of a little boy of eight, because he
couldn't endure the jokes he made every day at table! And in this case
his rags, the insolent police officer, the fever and this suspicion! All
that working upon a man half frantic with hypochondria, and with his
morbid exceptional vanity! That may well have been the starting-point of
illness. Well, bother it all!\ldots{} And, by the way, that Zametov
certainly is a nice fellow, but hm\ldots{} he shouldn't have told all
that last night. He is an awful chatterbox!''

``But whom did he tell it to? You and me?''

``And Porfiry.''

``What does that matter?''

``And, by the way, have you any influence on them, his mother and
sister? Tell them to be more careful with him to-day\ldots{}.''

``They'll get on all right!'' Razumihin answered reluctantly.

``Why is he so set against this Luzhin? A man with money and she doesn't
seem to dislike him\ldots{} and they haven't a farthing, I suppose?
eh?''

``But what business is it of yours?'' Razumihin cried with annoyance.
``How can I tell whether they've a farthing? Ask them yourself and
perhaps you'll find out\ldots{}.''

``Foo! what an ass you are sometimes! Last night's wine has not gone off
yet\ldots{}. Good-bye; thank your Praskovya Pavlovna from me for my
night's lodging. She locked herself in, made no reply to my
\emph{bonjour} through the door; she was up at seven o'clock, the
samovar was taken into her from the kitchen. I was not vouchsafed a
personal interview\ldots{}.''

At nine o'clock precisely Razumihin reached the lodgings at Bakaleyev's
house. Both ladies were waiting for him with nervous impatience. They
had risen at seven o'clock or earlier. He entered looking as black as
night, bowed awkwardly and was at once furious with himself for it. He
had reckoned without his host: Pulcheria Alexandrovna fairly rushed at
him, seized him by both hands and was almost kissing them. He glanced
timidly at Avdotya Romanovna, but her proud countenance wore at that
moment an expression of such gratitude and friendliness, such complete
and unlooked-for respect (in place of the sneering looks and
ill-disguised contempt he had expected), that it threw him into greater
confusion than if he had been met with abuse. Fortunately there was a
subject for conversation, and he made haste to snatch at it.

Hearing that everything was going well and that Rodya had not yet waked,
Pulcheria Alexandrovna declared that she was glad to hear it, because
``she had something which it was very, very necessary to talk over
beforehand.'' Then followed an inquiry about breakfast and an invitation
to have it with them; they had waited to have it with him. Avdotya
Romanovna rang the bell: it was answered by a ragged dirty waiter, and
they asked him to bring tea which was served at last, but in such a
dirty and disorderly way that the ladies were ashamed. Razumihin
vigorously attacked the lodgings, but, remembering Luzhin, stopped in
embarrassment and was greatly relieved by Pulcheria Alexandrovna's
questions, which showered in a continual stream upon him.

He talked for three quarters of an hour, being constantly interrupted by
their questions, and succeeded in describing to them all the most
important facts he knew of the last year of Raskolnikov's life,
concluding with a circumstantial account of his illness. He omitted,
however, many things, which were better omitted, including the scene at
the police station with all its consequences. They listened eagerly to
his story, and, when he thought he had finished and satisfied his
listeners, he found that they considered he had hardly begun.

``Tell me, tell me! What do you think\ldots{}? Excuse me, I still don't
know your name!'' Pulcheria Alexandrovna put in hastily.

``Dmitri Prokofitch.''

``I should like very, very much to know, Dmitri Prokofitch\ldots{} how
he looks\ldots{} on things in general now, that is, how can I explain,
what are his likes and dislikes? Is he always so irritable? Tell me, if
you can, what are his hopes and, so to say, his dreams? Under what
influences is he now? In a word, I should like\ldots{}''

``Ah, mother, how can he answer all that at once?'' observed Dounia.

``Good heavens, I had not expected to find him in the least like this,
Dmitri Prokofitch!''

``Naturally,'' answered Razumihin. ``I have no mother, but my uncle
comes every year and almost every time he can scarcely recognise me,
even in appearance, though he is a clever man; and your three years'
separation means a great deal. What am I to tell you? I have known
Rodion for a year and a half; he is morose, gloomy, proud and haughty,
and of late---and perhaps for a long time before---he has been
suspicious and fanciful. He has a noble nature and a kind heart. He does
not like showing his feelings and would rather do a cruel thing than
open his heart freely. Sometimes, though, he is not at all morbid, but
simply cold and inhumanly callous; it's as though he were alternating
between two characters. Sometimes he is fearfully reserved! He says he
is so busy that everything is a hindrance, and yet he lies in bed doing
nothing. He doesn't jeer at things, not because he hasn't the wit, but
as though he hadn't time to waste on such trifles. He never listens to
what is said to him. He is never interested in what interests other
people at any given moment. He thinks very highly of himself and perhaps
he is right. Well, what more? I think your arrival will have a most
beneficial influence upon him.''

``God grant it may,'' cried Pulcheria Alexandrovna, distressed by
Razumihin's account of her Rodya.

And Razumihin ventured to look more boldly at Avdotya Romanovna at last.
He glanced at her often while he was talking, but only for a moment and
looked away again at once. Avdotya Romanovna sat at the table, listening
attentively, then got up again and began walking to and fro with her
arms folded and her lips compressed, occasionally putting in a question,
without stopping her walk. She had the same habit of not listening to
what was said. She was wearing a dress of thin dark stuff and she had a
white transparent scarf round her neck. Razumihin soon detected signs of
extreme poverty in their belongings. Had Avdotya Romanovna been dressed
like a queen, he felt that he would not be afraid of her, but perhaps
just because she was poorly dressed and that he noticed all the misery
of her surroundings, his heart was filled with dread and he began to be
afraid of every word he uttered, every gesture he made, which was very
trying for a man who already felt diffident.

``You've told us a great deal that is interesting about my brother's
character\ldots{} and have told it impartially. I am glad. I thought
that you were too uncritically devoted to him,'' observed Avdotya
Romanovna with a smile. ``I think you are right that he needs a woman's
care,'' she added thoughtfully.

``I didn't say so; but I daresay you are right, only\ldots{}''

``What?''

``He loves no one and perhaps he never will,'' Razumihin declared
decisively.

``You mean he is not capable of love?''

``Do you know, Avdotya Romanovna, you are awfully like your brother, in
everything, indeed!'' he blurted out suddenly to his own surprise, but
remembering at once what he had just before said of her brother, he
turned as red as a crab and was overcome with confusion. Avdotya
Romanovna couldn't help laughing when she looked at him.

``You may both be mistaken about Rodya,'' Pulcheria Alexandrovna
remarked, slightly piqued. ``I am not talking of our present difficulty,
Dounia. What Pyotr Petrovitch writes in this letter and what you and I
have supposed may be mistaken, but you can't imagine, Dmitri Prokofitch,
how moody and, so to say, capricious he is. I never could depend on what
he would do when he was only fifteen. And I am sure that he might do
something now that nobody else would think of doing\ldots{} Well, for
instance, do you know how a year and a half ago he astounded me and gave
me a shock that nearly killed me, when he had the idea of marrying that
girl---what was her name---his landlady's daughter?''

``Did you hear about that affair?'' asked Avdotya Romanovna.

``Do you suppose------'' Pulcheria Alexandrovna continued warmly. ``Do
you suppose that my tears, my entreaties, my illness, my possible death
from grief, our poverty would have made him pause? No, he would calmly
have disregarded all obstacles. And yet it isn't that he doesn't love
us!''

``He has never spoken a word of that affair to me,'' Razumihin answered
cautiously. ``But I did hear something from Praskovya Pavlovna herself,
though she is by no means a gossip. And what I heard certainly was
rather strange.''

``And what did you hear?'' both the ladies asked at once.

``Well, nothing very special. I only learned that the marriage, which
only failed to take place through the girl's death, was not at all to
Praskovya Pavlovna's liking. They say, too, the girl was not at all
pretty, in fact I am told positively ugly\ldots{} and such an
invalid\ldots{} and queer. But she seems to have had some good
qualities. She must have had some good qualities or it's quite
inexplicable\ldots{}. She had no money either and he wouldn't have
considered her money\ldots{}. But it's always difficult to judge in such
matters.''

``I am sure she was a good girl,'' Avdotya Romanovna observed briefly.

``God forgive me, I simply rejoiced at her death. Though I don't know
which of them would have caused most misery to the other---he to her or
she to him,'' Pulcheria Alexandrovna concluded. Then she began
tentatively questioning him about the scene on the previous day with
Luzhin, hesitating and continually glancing at Dounia, obviously to the
latter's annoyance. This incident more than all the rest evidently
caused her uneasiness, even consternation. Razumihin described it in
detail again, but this time he added his own conclusions: he openly
blamed Raskolnikov for intentionally insulting Pyotr Petrovitch, not
seeking to excuse him on the score of his illness.

``He had planned it before his illness,'' he added.

``I think so, too,'' Pulcheria Alexandrovna agreed with a dejected air.
But she was very much surprised at hearing Razumihin express himself so
carefully and even with a certain respect about Pyotr Petrovitch.
Avdotya Romanovna, too, was struck by it.

``So this is your opinion of Pyotr Petrovitch?'' Pulcheria Alexandrovna
could not resist asking.

``I can have no other opinion of your daughter's future husband,''
Razumihin answered firmly and with warmth, ``and I don't say it simply
from vulgar politeness, but because\ldots{} simply because Avdotya
Romanovna has of her own free will deigned to accept this man. If I
spoke so rudely of him last night, it was because I was disgustingly
drunk and\ldots{} mad besides; yes, mad, crazy, I lost my head
completely\ldots{} and this morning I am ashamed of it.''

He crimsoned and ceased speaking. Avdotya Romanovna flushed, but did not
break the silence. She had not uttered a word from the moment they began
to speak of Luzhin.

Without her support Pulcheria Alexandrovna obviously did not know what
to do. At last, faltering and continually glancing at her daughter, she
confessed that she was exceedingly worried by one circumstance.

``You see, Dmitri Prokofitch,'' she began. ``I'll be perfectly open with
Dmitri Prokofitch, Dounia?''

``Of course, mother,'' said Avdotya Romanovna emphatically.

``This is what it is,'' she began in haste, as though the permission to
speak of her trouble lifted a weight off her mind. ``Very early this
morning we got a note from Pyotr Petrovitch in reply to our letter
announcing our arrival. He promised to meet us at the station, you know;
instead of that he sent a servant to bring us the address of these
lodgings and to show us the way; and he sent a message that he would be
here himself this morning. But this morning this note came from him.
You'd better read it yourself; there is one point in it which worries me
very much\ldots{} you will soon see what that is, and\ldots{} tell me
your candid opinion, Dmitri Prokofitch! You know Rodya's character
better than anyone and no one can advise us better than you can. Dounia,
I must tell you, made her decision at once, but I still don't feel sure
how to act and I\ldots{} I've been waiting for your opinion.''

Razumihin opened the note which was dated the previous evening and read
as follows:

``Dear Madam, Pulcheria Alexandrovna, I have the honour to inform you
that owing to unforeseen obstacles I was rendered unable to meet you at
the railway station; I sent a very competent person with the same object
in view. I likewise shall be deprived of the honour of an interview with
you to-morrow morning by business in the Senate that does not admit of
delay, and also that I may not intrude on your family circle while you
are meeting your son, and Avdotya Romanovna her brother. I shall have
the honour of visiting you and paying you my respects at your lodgings
not later than to-morrow evening at eight o'clock precisely, and
herewith I venture to present my earnest and, I may add, imperative
request that Rodion Romanovitch may not be present at our interview---as
he offered me a gross and unprecedented affront on the occasion of my
visit to him in his illness yesterday, and, moreover, since I desire
from you personally an indispensable and circumstantial explanation upon
a certain point, in regard to which I wish to learn your own
interpretation. I have the honour to inform you, in anticipation, that
if, in spite of my request, I meet Rodion Romanovitch, I shall be
compelled to withdraw immediately and then you have only yourself to
blame. I write on the assumption that Rodion Romanovitch who appeared so
ill at my visit, suddenly recovered two hours later and so, being able
to leave the house, may visit you also. I was confirmed in that belief
by the testimony of my own eyes in the lodging of a drunken man who was
run over and has since died, to whose daughter, a young woman of
notorious behaviour, he gave twenty-five roubles on the pretext of the
funeral, which gravely surprised me knowing what pains you were at to
raise that sum. Herewith expressing my special respect to your estimable
daughter, Avdotya Romanovna, I beg you to accept the respectful homage
of

``Your humble servant,

``P. LUZHIN.''

``What am I to do now, Dmitri Prokofitch?'' began Pulcheria
Alexandrovna, almost weeping. ``How can I ask Rodya not to come?
Yesterday he insisted so earnestly on our refusing Pyotr Petrovitch and
now we are ordered not to receive Rodya! He will come on purpose if he
knows, and\ldots{} what will happen then?''

``Act on Avdotya Romanovna's decision,'' Razumihin answered calmly at
once.

``Oh, dear me! She says\ldots{} goodness knows what she says, she
doesn't explain her object! She says that it would be best, at least,
not that it would be best, but that it's absolutely necessary that Rodya
should make a point of being here at eight o'clock and that they must
meet\ldots{}. I didn't want even to show him the letter, but to prevent
him from coming by some stratagem with your help\ldots{} because he is
so irritable\ldots{}. Besides I don't understand about that drunkard who
died and that daughter, and how he could have given the daughter all the
money\ldots{} which\ldots{}''

``Which cost you such sacrifice, mother,'' put in Avdotya Romanovna.

``He was not himself yesterday,'' Razumihin said thoughtfully, ``if you
only knew what he was up to in a restaurant yesterday, though there was
sense in it too\ldots{}. Hm! He did say something, as we were going home
yesterday evening, about a dead man and a girl, but I didn't understand
a word\ldots{}. But last night, I myself\ldots{}''

``The best thing, mother, will be for us to go to him ourselves and
there I assure you we shall see at once what's to be done. Besides, it's
getting late---good heavens, it's past ten,'' she cried looking at a
splendid gold enamelled watch which hung round her neck on a thin
Venetian chain, and looked entirely out of keeping with the rest of her
dress. ``A present from her \emph{fiancé},'' thought Razumihin.

``We must start, Dounia, we must start,'' her mother cried in a flutter.
``He will be thinking we are still angry after yesterday, from our
coming so late. Merciful heavens!''

While she said this she was hurriedly putting on her hat and mantle;
Dounia, too, put on her things. Her gloves, as Razumihin noticed, were
not merely shabby but had holes in them, and yet this evident poverty
gave the two ladies an air of special dignity, which is always found in
people who know how to wear poor clothes. Razumihin looked reverently at
Dounia and felt proud of escorting her. ``The queen who mended her
stockings in prison,'' he thought, ``must have looked then every inch a
queen and even more a queen than at sumptuous banquets and levées.''

``My God!'' exclaimed Pulcheria Alexandrovna, ``little did I think that
I should ever fear seeing my son, my darling, darling Rodya! I am
afraid, Dmitri Prokofitch,'' she added, glancing at him timidly.

``Don't be afraid, mother,'' said Dounia, kissing her, ``better have
faith in him.''

``Oh, dear, I have faith in him, but I haven't slept all night,''
exclaimed the poor woman.

They came out into the street.

``Do you know, Dounia, when I dozed a little this morning I dreamed of
Marfa Petrovna\ldots{} she was all in white\ldots{} she came up to me,
took my hand, and shook her head at me, but so sternly as though she
were blaming me\ldots{}. Is that a good omen? Oh, dear me! You don't
know, Dmitri Prokofitch, that Marfa Petrovna's dead!''

``No, I didn't know; who is Marfa Petrovna?''

``She died suddenly; and only fancy\ldots{}''

``Afterwards, mamma,'' put in Dounia. ``He doesn't know who Marfa
Petrovna is.''

``Ah, you don't know? And I was thinking that you knew all about us.
Forgive me, Dmitri Prokofitch, I don't know what I am thinking about
these last few days. I look upon you really as a providence for us, and
so I took it for granted that you knew all about us. I look on you as a
relation\ldots{}. Don't be angry with me for saying so. Dear me, what's
the matter with your right hand? Have you knocked it?''

``Yes, I bruised it,'' muttered Razumihin overjoyed.

``I sometimes speak too much from the heart, so that Dounia finds fault
with me\ldots{}. But, dear me, what a cupboard he lives in! I wonder
whether he is awake? Does this woman, his landlady, consider it a room?
Listen, you say he does not like to show his feelings, so perhaps I
shall annoy him with my\ldots{} weaknesses? Do advise me, Dmitri
Prokofitch, how am I to treat him? I feel quite distracted, you know.''

``Don't question him too much about anything if you see him frown; don't
ask him too much about his health; he doesn't like that.''

``Ah, Dmitri Prokofitch, how hard it is to be a mother! But here are the
stairs\ldots{}. What an awful staircase!''

``Mother, you are quite pale, don't distress yourself, darling,'' said
Dounia caressing her, then with flashing eyes she added: ``He ought to
be happy at seeing you, and you are tormenting yourself so.''

``Wait, I'll peep in and see whether he has waked up.''

The ladies slowly followed Razumihin, who went on before, and when they
reached the landlady's door on the fourth storey, they noticed that her
door was a tiny crack open and that two keen black eyes were watching
them from the darkness within. When their eyes met, the door was
suddenly shut with such a slam that Pulcheria Alexandrovna almost cried
out.

\protect\hypertarget{link2HCH0017}{}{}

\mychapter{3}{Chapter III}\label{chapter-iii-2}

``He is well, quite well!'' Zossimov cried cheerfully as they entered.

He had come in ten minutes earlier and was sitting in the same place as
before, on the sofa. Raskolnikov was sitting in the opposite corner,
fully dressed and carefully washed and combed, as he had not been for
some time past. The room was immediately crowded, yet Nastasya managed
to follow the visitors in and stayed to listen.

Raskolnikov really was almost well, as compared with his condition the
day before, but he was still pale, listless, and sombre. He looked like
a wounded man or one who has undergone some terrible physical suffering.
His brows were knitted, his lips compressed, his eyes feverish. He spoke
little and reluctantly, as though performing a duty, and there was a
restlessness in his movements.

He only wanted a sling on his arm or a bandage on his finger to complete
the impression of a man with a painful abscess or a broken arm. The
pale, sombre face lighted up for a moment when his mother and sister
entered, but this only gave it a look of more intense suffering, in
place of its listless dejection. The light soon died away, but the look
of suffering remained, and Zossimov, watching and studying his patient
with all the zest of a young doctor beginning to practise, noticed in
him no joy at the arrival of his mother and sister, but a sort of
bitter, hidden determination to bear another hour or two of inevitable
torture. He saw later that almost every word of the following
conversation seemed to touch on some sore place and irritate it. But at
the same time he marvelled at the power of controlling himself and
hiding his feelings in a patient who the previous day had, like a
monomaniac, fallen into a frenzy at the slightest word.

``Yes, I see myself now that I am almost well,'' said Raskolnikov,
giving his mother and sister a kiss of welcome which made Pulcheria
Alexandrovna radiant at once. ``And I don't say this \emph{as I did
yesterday},'' he said, addressing Razumihin, with a friendly pressure of
his hand.

``Yes, indeed, I am quite surprised at him to-day,'' began Zossimov,
much delighted at the ladies' entrance, for he had not succeeded in
keeping up a conversation with his patient for ten minutes. ``In another
three or four days, if he goes on like this, he will be just as before,
that is, as he was a month ago, or two\ldots{} or perhaps even three.
This has been coming on for a long while\ldots{}. eh? Confess, now, that
it has been perhaps your own fault?'' he added, with a tentative smile,
as though still afraid of irritating him.

``It is very possible,'' answered Raskolnikov coldly.

``I should say, too,'' continued Zossimov with zest, ``that your
complete recovery depends solely on yourself. Now that one can talk to
you, I should like to impress upon you that it is essential to avoid the
elementary, so to speak, fundamental causes tending to produce your
morbid condition: in that case you will be cured, if not, it will go
from bad to worse. These fundamental causes I don't know, but they must
be known to you. You are an intelligent man, and must have observed
yourself, of course. I fancy the first stage of your derangement
coincides with your leaving the university. You must not be left without
occupation, and so, work and a definite aim set before you might, I
fancy, be very beneficial.''

``Yes, yes; you are perfectly right\ldots{}. I will make haste and
return to the university: and then everything will go
smoothly\ldots{}.''

Zossimov, who had begun his sage advice partly to make an effect before
the ladies, was certainly somewhat mystified, when, glancing at his
patient, he observed unmistakable mockery on his face. This lasted an
instant, however. Pulcheria Alexandrovna began at once thanking
Zossimov, especially for his visit to their lodging the previous night.

``What! he saw you last night?'' Raskolnikov asked, as though startled.
``Then you have not slept either after your journey.''

``Ach, Rodya, that was only till two o'clock. Dounia and I never go to
bed before two at home.''

``I don't know how to thank him either,'' Raskolnikov went on, suddenly
frowning and looking down. ``Setting aside the question of
payment---forgive me for referring to it (he turned to Zossimov)---I
really don't know what I have done to deserve such special attention
from you! I simply don't understand it\ldots{} and\ldots{} and\ldots{}
it weighs upon me, indeed, because I don't understand it. I tell you so
candidly.''

``Don't be irritated.'' Zossimov forced himself to laugh. ``Assume that
you are my first patient---well---we fellows just beginning to practise
love our first patients as if they were our children, and some almost
fall in love with them. And, of course, I am not rich in patients.''

``I say nothing about him,'' added Raskolnikov, pointing to Razumihin,
``though he has had nothing from me either but insult and trouble.''

``What nonsense he is talking! Why, you are in a sentimental mood
to-day, are you?'' shouted Razumihin.

If he had had more penetration he would have seen that there was no
trace of sentimentality in him, but something indeed quite the opposite.
But Avdotya Romanovna noticed it. She was intently and uneasily watching
her brother.

``As for you, mother, I don't dare to speak,'' he went on, as though
repeating a lesson learned by heart. ``It is only to-day that I have
been able to realise a little how distressed you must have been here
yesterday, waiting for me to come back.''

When he had said this, he suddenly held out his hand to his sister,
smiling without a word. But in this smile there was a flash of real
unfeigned feeling. Dounia caught it at once, and warmly pressed his
hand, overjoyed and thankful. It was the first time he had addressed her
since their dispute the previous day. The mother's face lighted up with
ecstatic happiness at the sight of this conclusive unspoken
reconciliation. ``Yes, that is what I love him for,'' Razumihin,
exaggerating it all, muttered to himself, with a vigorous turn in his
chair. ``He has these movements.''

``And how well he does it all,'' the mother was thinking to herself.
``What generous impulses he has, and how simply, how delicately he put
an end to all the misunderstanding with his sister---simply by holding
out his hand at the right minute and looking at her like that\ldots{}.
And what fine eyes he has, and how fine his whole face is!\ldots{} He is
even better looking than Dounia\ldots{}. But, good heavens, what a
suit---how terribly he's dressed!\ldots{} Vasya, the messenger boy in
Afanasy Ivanitch's shop, is better dressed! I could rush at him and hug
him\ldots{} weep over him---but I am afraid\ldots{}. Oh, dear, he's so
strange! He's talking kindly, but I'm afraid! Why, what am I afraid
of?\ldots{}''

``Oh, Rodya, you wouldn't believe,'' she began suddenly, in haste to
answer his words to her, ``how unhappy Dounia and I were yesterday! Now
that it's all over and done with and we are quite happy again---I can
tell you. Fancy, we ran here almost straight from the train to embrace
you and that woman---ah, here she is! Good morning, Nastasya!\ldots{}
She told us at once that you were lying in a high fever and had just run
away from the doctor in delirium, and they were looking for you in the
streets. You can't imagine how we felt! I couldn't help thinking of the
tragic end of Lieutenant Potanchikov, a friend of your father's---you
can't remember him, Rodya---who ran out in the same way in a high fever
and fell into the well in the court-yard and they couldn't pull him out
till next day. Of course, we exaggerated things. We were on the point of
rushing to find Pyotr Petrovitch to ask him to help\ldots{}. Because we
were alone, utterly alone,'' she said plaintively and stopped short,
suddenly, recollecting it was still somewhat dangerous to speak of Pyotr
Petrovitch, although ``we are quite happy again.''

``Yes, yes\ldots{}. Of course it's very annoying\ldots{}.'' Raskolnikov
muttered in reply, but with such a preoccupied and inattentive air that
Dounia gazed at him in perplexity.

``What else was it I wanted to say?'' He went on trying to recollect.
``Oh, yes; mother, and you too, Dounia, please don't think that I didn't
mean to come and see you to-day and was waiting for you to come first.''

``What are you saying, Rodya?'' cried Pulcheria Alexandrovna. She, too,
was surprised.

``Is he answering us as a duty?'' Dounia wondered. ``Is he being
reconciled and asking forgiveness as though he were performing a rite or
repeating a lesson?''

``I've only just waked up, and wanted to go to you, but was delayed
owing to my clothes; I forgot yesterday to ask her\ldots{}
Nastasya\ldots{} to wash out the blood\ldots{} I've only just dressed.''

``Blood! What blood?'' Pulcheria Alexandrovna asked in alarm.

``Oh, nothing---don't be uneasy. It was when I was wandering about
yesterday, rather delirious, I chanced upon a man who had been run
over\ldots{} a clerk\ldots{}''

``Delirious? But you remember everything!'' Razumihin interrupted.

``That's true,'' Raskolnikov answered with special carefulness. ``I
remember everything even to the slightest detail, and yet---why I did
that and went there and said that, I can't clearly explain now.''

``A familiar phenomenon,'' interposed Zossimov, ``actions are sometimes
performed in a masterly and most cunning way, while the direction of the
actions is deranged and dependent on various morbid impressions---it's
like a dream.''

``Perhaps it's a good thing really that he should think me almost a
madman,'' thought Raskolnikov.

``Why, people in perfect health act in the same way too,'' observed
Dounia, looking uneasily at Zossimov.

``There is some truth in your observation,'' the latter replied. ``In
that sense we are certainly all not infrequently like madmen, but with
the slight difference that the deranged are somewhat madder, for we must
draw a line. A normal man, it is true, hardly exists. Among
dozens---perhaps hundreds of thousands---hardly one is to be met with.''

At the word ``madman,'' carelessly dropped by Zossimov in his chatter on
his favourite subject, everyone frowned.

Raskolnikov sat seeming not to pay attention, plunged in thought with a
strange smile on his pale lips. He was still meditating on something.

``Well, what about the man who was run over? I interrupted you!''
Razumihin cried hastily.

``What?'' Raskolnikov seemed to wake up. ``Oh\ldots{} I got spattered
with blood helping to carry him to his lodging. By the way, mamma, I did
an unpardonable thing yesterday. I was literally out of my mind. I gave
away all the money you sent me\ldots{} to his wife for the funeral.
She's a widow now, in consumption, a poor creature\ldots{} three little
children, starving\ldots{} nothing in the house\ldots{} there's a
daughter, too\ldots{} perhaps you'd have given it yourself if you'd seen
them. But I had no right to do it I admit, especially as I knew how you
needed the money yourself. To help others one must have the right to do
it, or else \emph{Crevez, chiens, si vous n'êtes pas contents}.'' He
laughed, ``That's right, isn't it, Dounia?''

``No, it's not,'' answered Dounia firmly.

``Bah! you, too, have ideals,'' he muttered, looking at her almost with
hatred, and smiling sarcastically. ``I ought to have considered
that\ldots{}. Well, that's praiseworthy, and it's better for you\ldots{}
and if you reach a line you won't overstep, you will be unhappy\ldots{}
and if you overstep it, maybe you will be still unhappier\ldots{}. But
all that's nonsense,'' he added irritably, vexed at being carried away.
``I only meant to say that I beg your forgiveness, mother,'' he
concluded, shortly and abruptly.

``That's enough, Rodya, I am sure that everything you do is very good,''
said his mother, delighted.

``Don't be too sure,'' he answered, twisting his mouth into a smile.

A silence followed. There was a certain constraint in all this
conversation, and in the silence, and in the reconciliation, and in the
forgiveness, and all were feeling it.

``It is as though they were afraid of me,'' Raskolnikov was thinking to
himself, looking askance at his mother and sister. Pulcheria
Alexandrovna was indeed growing more timid the longer she kept silent.

``Yet in their absence I seemed to love them so much,'' flashed through
his mind.

``Do you know, Rodya, Marfa Petrovna is dead,'' Pulcheria Alexandrovna
suddenly blurted out.

``What Marfa Petrovna?''

``Oh, mercy on us---Marfa Petrovna Svidrigaïlov. I wrote you so much
about her.''

``A-a-h! Yes, I remember\ldots{}. So she's dead! Oh, really?'' he roused
himself suddenly, as if waking up. ``What did she die of?''

``Only imagine, quite suddenly,'' Pulcheria Alexandrovna answered
hurriedly, encouraged by his curiosity. ``On the very day I was sending
you that letter! Would you believe it, that awful man seems to have been
the cause of her death. They say he beat her dreadfully.''

``Why, were they on such bad terms?'' he asked, addressing his sister.

``Not at all. Quite the contrary indeed. With her, he was always very
patient, considerate even. In fact, all those seven years of their
married life he gave way to her, too much so indeed, in many cases. All
of a sudden he seems to have lost patience.''

``Then he could not have been so awful if he controlled himself for
seven years? You seem to be defending him, Dounia?''

``No, no, he's an awful man! I can imagine nothing more awful!'' Dounia
answered, almost with a shudder, knitting her brows, and sinking into
thought.

``That had happened in the morning,'' Pulcheria Alexandrovna went on
hurriedly. ``And directly afterwards she ordered the horses to be
harnessed to drive to the town immediately after dinner. She always used
to drive to the town in such cases. She ate a very good dinner, I am
told\ldots{}.''

``After the beating?''

``That was always her\ldots{} habit; and immediately after dinner, so as
not to be late in starting, she went to the bath-house\ldots{}. You see,
she was undergoing some treatment with baths. They have a cold spring
there, and she used to bathe in it regularly every day, and no sooner
had she got into the water when she suddenly had a stroke!''

``I should think so,'' said Zossimov.

``And did he beat her badly?''

``What does that matter!'' put in Dounia.

``H'm! But I don't know why you want to tell us such gossip, mother,''
said Raskolnikov irritably, as it were in spite of himself.

``Ah, my dear, I don't know what to talk about,'' broke from Pulcheria
Alexandrovna.

``Why, are you all afraid of me?'' he asked, with a constrained smile.

``That's certainly true,'' said Dounia, looking directly and sternly at
her brother. ``Mother was crossing herself with terror as she came up
the stairs.''

His face worked, as though in convulsion.

``Ach, what are you saying, Dounia! Don't be angry, please,
Rodya\ldots{}. Why did you say that, Dounia?'' Pulcheria Alexandrovna
began, overwhelmed---``You see, coming here, I was dreaming all the way,
in the train, how we should meet, how we should talk over everything
together\ldots{}. And I was so happy, I did not notice the journey! But
what am I saying? I am happy now\ldots{}. You should not,
Dounia\ldots{}. I am happy now---simply in seeing you, Rodya\ldots{}.''

``Hush, mother,'' he muttered in confusion, not looking at her, but
pressing her hand. ``We shall have time to speak freely of everything!''

As he said this, he was suddenly overwhelmed with confusion and turned
pale. Again that awful sensation he had known of late passed with deadly
chill over his soul. Again it became suddenly plain and perceptible to
him that he had just told a fearful lie---that he would never now be
able to speak freely of everything---that he would never again be able
to \emph{speak} of anything to anyone. The anguish of this thought was
such that for a moment he almost forgot himself. He got up from his
seat, and not looking at anyone walked towards the door.

``What are you about?'' cried Razumihin, clutching him by the arm.

He sat down again, and began looking about him, in silence. They were
all looking at him in perplexity.

``But what are you all so dull for?'' he shouted, suddenly and quite
unexpectedly. ``Do say something! What's the use of sitting like this?
Come, do speak. Let us talk\ldots{}. We meet together and sit in
silence\ldots{}. Come, anything!''

``Thank God; I was afraid the same thing as yesterday was beginning
again,'' said Pulcheria Alexandrovna, crossing herself.

``What is the matter, Rodya?'' asked Avdotya Romanovna, distrustfully.

``Oh, nothing! I remembered something,'' he answered, and suddenly
laughed.

``Well, if you remembered something; that's all right!\ldots{} I was
beginning to think\ldots{}'' muttered Zossimov, getting up from the
sofa. ``It is time for me to be off. I will look in again
perhaps\ldots{} if I can\ldots{}'' He made his bows, and went out.

``What an excellent man!'' observed Pulcheria Alexandrovna.

``Yes, excellent, splendid, well-educated, intelligent,'' Raskolnikov
began, suddenly speaking with surprising rapidity, and a liveliness he
had not shown till then. ``I can't remember where I met him before my
illness\ldots{}. I believe I have met him somewhere------\ldots{} And
this is a good man, too,'' he nodded at Razumihin. ``Do you like him,
Dounia?'' he asked her; and suddenly, for some unknown reason, laughed.

``Very much,'' answered Dounia.

``Foo!---what a pig you are!'' Razumihin protested, blushing in terrible
confusion, and he got up from his chair. Pulcheria Alexandrovna smiled
faintly, but Raskolnikov laughed aloud.

``Where are you off to?''

``I must go.''

``You need not at all. Stay. Zossimov has gone, so you must. Don't go.
What's the time? Is it twelve o'clock? What a pretty watch you have got,
Dounia. But why are you all silent again? I do all the talking.''

``It was a present from Marfa Petrovna,'' answered Dounia.

``And a very expensive one!'' added Pulcheria Alexandrovna.

``A-ah! What a big one! Hardly like a lady's.''

``I like that sort,'' said Dounia.

``So it is not a present from her \emph{fiancé},'' thought Razumihin,
and was unreasonably delighted.

``I thought it was Luzhin's present,'' observed Raskolnikov.

``No, he has not made Dounia any presents yet.''

``A-ah! And do you remember, mother, I was in love and wanted to get
married?'' he said suddenly, looking at his mother, who was disconcerted
by the sudden change of subject and the way he spoke of it.

``Oh, yes, my dear.''

Pulcheria Alexandrovna exchanged glances with Dounia and Razumihin.

``H'm, yes. What shall I tell you? I don't remember much indeed. She was
such a sickly girl,'' he went on, growing dreamy and looking down again.
``Quite an invalid. She was fond of giving alms to the poor, and was
always dreaming of a nunnery, and once she burst into tears when she
began talking to me about it. Yes, yes, I remember. I remember very
well. She was an ugly little thing. I really don't know what drew me to
her then---I think it was because she was always ill. If she had been
lame or hunchback, I believe I should have liked her better still,'' he
smiled dreamily. ``Yes, it was a sort of spring delirium.''

``No, it was not only spring delirium,'' said Dounia, with warm feeling.

He fixed a strained intent look on his sister, but did not hear or did
not understand her words. Then, completely lost in thought, he got up,
went up to his mother, kissed her, went back to his place and sat down.

``You love her even now?'' said Pulcheria Alexandrovna, touched.

``Her? Now? Oh, yes\ldots{}. You ask about her? No\ldots{} that's all
now, as it were, in another world\ldots{} and so long ago. And indeed
everything happening here seems somehow far away.'' He looked
attentively at them. ``You, now\ldots{} I seem to be looking at you from
a thousand miles away\ldots{} but, goodness knows why we are talking of
that! And what's the use of asking about it?'' he added with annoyance,
and biting his nails, fell into dreamy silence again.

``What a wretched lodging you have, Rodya! It's like a tomb,'' said
Pulcheria Alexandrovna, suddenly breaking the oppressive silence. ``I am
sure it's quite half through your lodging you have become so
melancholy.''

``My lodging,'' he answered, listlessly. ``Yes, the lodging had a great
deal to do with it\ldots{}. I thought that, too\ldots{}. If only you
knew, though, what a strange thing you said just now, mother,'' he said,
laughing strangely.

A little more, and their companionship, this mother and this sister,
with him after three years' absence, this intimate tone of conversation,
in face of the utter impossibility of really speaking about anything,
would have been beyond his power of endurance. But there was one urgent
matter which must be settled one way or the other that day---so he had
decided when he woke. Now he was glad to remember it, as a means of
escape.

``Listen, Dounia,'' he began, gravely and drily, ``of course I beg your
pardon for yesterday, but I consider it my duty to tell you again that I
do not withdraw from my chief point. It is me or Luzhin. If I am a
scoundrel, you must not be. One is enough. If you marry Luzhin, I cease
at once to look on you as a sister.''

``Rodya, Rodya! It is the same as yesterday again,'' Pulcheria
Alexandrovna cried, mournfully. ``And why do you call yourself a
scoundrel? I can't bear it. You said the same yesterday.''

``Brother,'' Dounia answered firmly and with the same dryness. ``In all
this there is a mistake on your part. I thought it over at night, and
found out the mistake. It is all because you seem to fancy I am
sacrificing myself to someone and for someone. That is not the case at
all. I am simply marrying for my own sake, because things are hard for
me. Though, of course, I shall be glad if I succeed in being useful to
my family. But that is not the chief motive for my decision\ldots{}.''

``She is lying,'' he thought to himself, biting his nails vindictively.
``Proud creature! She won't admit she wants to do it out of charity! Too
haughty! Oh, base characters! They even love as though they
hate\ldots{}. Oh, how I\ldots{} hate them all!''

``In fact,'' continued Dounia, ``I am marrying Pyotr Petrovitch because
of two evils I choose the less. I intend to do honestly all he expects
of me, so I am not deceiving him\ldots{}. Why did you smile just now?''
She, too, flushed, and there was a gleam of anger in her eyes.

``All?'' he asked, with a malignant grin.

``Within certain limits. Both the manner and form of Pyotr Petrovitch's
courtship showed me at once what he wanted. He may, of course, think too
well of himself, but I hope he esteems me, too\ldots{}. Why are you
laughing again?''

``And why are you blushing again? You are lying, sister. You are
intentionally lying, simply from feminine obstinacy, simply to hold your
own against me\ldots{}. You cannot respect Luzhin. I have seen him and
talked with him. So you are selling yourself for money, and so in any
case you are acting basely, and I am glad at least that you can blush
for it.''

``It is not true. I am not lying,'' cried Dounia, losing her composure.
``I would not marry him if I were not convinced that he esteems me and
thinks highly of me. I would not marry him if I were not firmly
convinced that I can respect him. Fortunately, I can have convincing
proof of it this very day\ldots{} and such a marriage is not a vileness,
as you say! And even if you were right, if I really had determined on a
vile action, is it not merciless on your part to speak to me like that?
Why do you demand of me a heroism that perhaps you have not either? It
is despotism; it is tyranny. If I ruin anyone, it is only
myself\ldots{}. I am not committing a murder. Why do you look at me like
that? Why are you so pale? Rodya, darling, what's the matter?''

``Good heavens! You have made him faint,'' cried Pulcheria Alexandrovna.

``No, no, nonsense! It's nothing. A little giddiness---not fainting. You
have fainting on the brain. H'm, yes, what was I saying? Oh, yes. In
what way will you get convincing proof to-day that you can respect him,
and that he\ldots{} esteems you, as you said. I think you said to-day?''

``Mother, show Rodya Pyotr Petrovitch's letter,'' said Dounia.

With trembling hands, Pulcheria Alexandrovna gave him the letter. He
took it with great interest, but, before opening it, he suddenly looked
with a sort of wonder at Dounia.

``It is strange,'' he said, slowly, as though struck by a new idea.
``What am I making such a fuss for? What is it all about? Marry whom you
like!''

He said this as though to himself, but said it aloud, and looked for
some time at his sister, as though puzzled. He opened the letter at
last, still with the same look of strange wonder on his face. Then,
slowly and attentively, he began reading, and read it through twice.
Pulcheria Alexandrovna showed marked anxiety, and all indeed expected
something particular.

``What surprises me,'' he began, after a short pause, handing the letter
to his mother, but not addressing anyone in particular, ``is that he is
a business man, a lawyer, and his conversation is pretentious indeed,
and yet he writes such an uneducated letter.''

They all started. They had expected something quite different.

``But they all write like that, you know,'' Razumihin observed,
abruptly.

``Have you read it?''

``Yes.''

``We showed him, Rodya. We\ldots{} consulted him just now,'' Pulcheria
Alexandrovna began, embarrassed.

``That's just the jargon of the courts,'' Razumihin put in. ``Legal
documents are written like that to this day.''

``Legal? Yes, it's just legal---business language---not so very
uneducated, and not quite educated---business language!''

``Pyotr Petrovitch makes no secret of the fact that he had a cheap
education, he is proud indeed of having made his own way,'' Avdotya
Romanovna observed, somewhat offended by her brother's tone.

``Well, if he's proud of it, he has reason, I don't deny it. You seem to
be offended, sister, at my making only such a frivolous criticism on the
letter, and to think that I speak of such trifling matters on purpose to
annoy you. It is quite the contrary, an observation apropos of the style
occurred to me that is by no means irrelevant as things stand. There is
one expression, `blame yourselves' put in very significantly and
plainly, and there is besides a threat that he will go away at once if I
am present. That threat to go away is equivalent to a threat to abandon
you both if you are disobedient, and to abandon you now after summoning
you to Petersburg. Well, what do you think? Can one resent such an
expression from Luzhin, as we should if he (he pointed to Razumihin) had
written it, or Zossimov, or one of us?''

``N-no,'' answered Dounia, with more animation. ``I saw clearly that it
was too naïvely expressed, and that perhaps he simply has no skill in
writing\ldots{} that is a true criticism, brother. I did not expect,
indeed\ldots{}''

``It is expressed in legal style, and sounds coarser than perhaps he
intended. But I must disillusion you a little. There is one expression
in the letter, one slander about me, and rather a contemptible one. I
gave the money last night to the widow, a woman in consumption, crushed
with trouble, and not `on the pretext of the funeral,' but simply to pay
for the funeral, and not to the daughter---a young woman, as he writes,
of notorious behaviour (whom I saw last night for the first time in my
life)---but to the widow. In all this I see a too hasty desire to
slander me and to raise dissension between us. It is expressed again in
legal jargon, that is to say, with a too obvious display of the aim, and
with a very naïve eagerness. He is a man of intelligence, but to act
sensibly, intelligence is not enough. It all shows the man and\ldots{} I
don't think he has a great esteem for you. I tell you this simply to
warn you, because I sincerely wish for your good\ldots{}''

Dounia did not reply. Her resolution had been taken. She was only
awaiting the evening.

``Then what is your decision, Rodya?'' asked Pulcheria Alexandrovna, who
was more uneasy than ever at the sudden, new businesslike tone of his
talk.

``What decision?''

``You see Pyotr Petrovitch writes that you are not to be with us this
evening, and that he will go away if you come. So will you\ldots{}
come?''

``That, of course, is not for me to decide, but for you first, if you
are not offended by such a request; and secondly, by Dounia, if she,
too, is not offended. I will do what you think best,'' he added, drily.

``Dounia has already decided, and I fully agree with her,'' Pulcheria
Alexandrovna hastened to declare.

``I decided to ask you, Rodya, to urge you not to fail to be with us at
this interview,'' said Dounia. ``Will you come?''

``Yes.''

``I will ask you, too, to be with us at eight o'clock,'' she said,
addressing Razumihin. ``Mother, I am inviting him, too.''

``Quite right, Dounia. Well, since you have decided,'' added Pulcheria
Alexandrovna, ``so be it. I shall feel easier myself. I do not like
concealment and deception. Better let us have the whole truth\ldots{}.
Pyotr Petrovitch may be angry or not, now!''

\protect\hypertarget{link2HCH0018}{}{}

\mychapter{4}{Chapter IV}\label{chapter-iv-2}

At that moment the door was softly opened, and a young girl walked into
the room, looking timidly about her. Everyone turned towards her with
surprise and curiosity. At first sight, Raskolnikov did not recognise
her. It was Sofya Semyonovna Marmeladov. He had seen her yesterday for
the first time, but at such a moment, in such surroundings and in such a
dress, that his memory retained a very different image of her. Now she
was a modestly and poorly-dressed young girl, very young, indeed, almost
like a child, with a modest and refined manner, with a candid but
somewhat frightened-looking face. She was wearing a very plain indoor
dress, and had on a shabby old-fashioned hat, but she still carried a
parasol. Unexpectedly finding the room full of people, she was not so
much embarrassed as completely overwhelmed with shyness, like a little
child. She was even about to retreat. ``Oh\ldots{} it's you!'' said
Raskolnikov, extremely astonished, and he, too, was confused. He at once
recollected that his mother and sister knew through Luzhin's letter of
``some young woman of notorious behaviour.'' He had only just been
protesting against Luzhin's calumny and declaring that he had seen the
girl last night for the first time, and suddenly she had walked in. He
remembered, too, that he had not protested against the expression ``of
notorious behaviour.'' All this passed vaguely and fleetingly through
his brain, but looking at her more intently, he saw that the humiliated
creature was so humiliated that he felt suddenly sorry for her. When she
made a movement to retreat in terror, it sent a pang to his heart.

``I did not expect you,'' he said, hurriedly, with a look that made her
stop. ``Please sit down. You come, no doubt, from Katerina Ivanovna.
Allow me---not there. Sit here\ldots{}.''

At Sonia's entrance, Razumihin, who had been sitting on one of
Raskolnikov's three chairs, close to the door, got up to allow her to
enter. Raskolnikov had at first shown her the place on the sofa where
Zossimov had been sitting, but feeling that the sofa which served him as
a bed, was too \emph{familiar} a place, he hurriedly motioned her to
Razumihin's chair.

``You sit here,'' he said to Razumihin, putting him on the sofa.

Sonia sat down, almost shaking with terror, and looked timidly at the
two ladies. It was evidently almost inconceivable to herself that she
could sit down beside them. At the thought of it, she was so frightened
that she hurriedly got up again, and in utter confusion addressed
Raskolnikov.

``I\ldots{} I\ldots{} have come for one minute. Forgive me for
disturbing you,'' she began falteringly. ``I come from Katerina
Ivanovna, and she had no one to send. Katerina Ivanovna told me to beg
you\ldots{} to be at the service\ldots{} in the morning\ldots{} at
Mitrofanievsky\ldots{} and then\ldots{} to us\ldots{} to her\ldots{} to
do her the honour\ldots{} she told me to beg you\ldots{}'' Sonia
stammered and ceased speaking.

``I will try, certainly, most certainly,'' answered Raskolnikov. He,
too, stood up, and he, too, faltered and could not finish his sentence.
``Please sit down,'' he said, suddenly. ``I want to talk to you. You are
perhaps in a hurry, but please, be so kind, spare me two minutes,'' and
he drew up a chair for her.

Sonia sat down again, and again timidly she took a hurried, frightened
look at the two ladies, and dropped her eyes. Raskolnikov's pale face
flushed, a shudder passed over him, his eyes glowed.

``Mother,'' he said, firmly and insistently, ``this is Sofya Semyonovna
Marmeladov, the daughter of that unfortunate Mr. Marmeladov, who was run
over yesterday before my eyes, and of whom I was just telling you.''

Pulcheria Alexandrovna glanced at Sonia, and slightly screwed up her
eyes. In spite of her embarrassment before Rodya's urgent and
challenging look, she could not deny herself that satisfaction. Dounia
gazed gravely and intently into the poor girl's face, and scrutinised
her with perplexity. Sonia, hearing herself introduced, tried to raise
her eyes again, but was more embarrassed than ever.

``I wanted to ask you,'' said Raskolnikov, hastily, ``how things were
arranged yesterday. You were not worried by the police, for instance?''

``No, that was all right\ldots{} it was too evident, the cause of
death\ldots{} they did not worry us\ldots{} only the lodgers are
angry.''

``Why?''

``At the body's remaining so long. You see it is hot now. So that,
to-day, they will carry it to the cemetery, into the chapel, until
to-morrow. At first Katerina Ivanovna was unwilling, but now she sees
herself that it's necessary\ldots{}''

``To-day, then?''

``She begs you to do us the honour to be in the church to-morrow for the
service, and then to be present at the funeral lunch.''

``She is giving a funeral lunch?''

``Yes\ldots{} just a little\ldots{}. She told me to thank you very much
for helping us yesterday. But for you, we should have had nothing for
the funeral.''

All at once her lips and chin began trembling, but, with an effort, she
controlled herself, looking down again.

During the conversation, Raskolnikov watched her carefully. She had a
thin, very thin, pale little face, rather irregular and angular, with a
sharp little nose and chin. She could not have been called pretty, but
her blue eyes were so clear, and when they lighted up, there was such a
kindliness and simplicity in her expression that one could not help
being attracted. Her face, and her whole figure indeed, had another
peculiar characteristic. In spite of her eighteen years, she looked
almost a little girl---almost a child. And in some of her gestures, this
childishness seemed almost absurd.

``But has Katerina Ivanovna been able to manage with such small means?
Does she even mean to have a funeral lunch?'' Raskolnikov asked,
persistently keeping up the conversation.

``The coffin will be plain, of course\ldots{} and everything will be
plain, so it won't cost much. Katerina Ivanovna and I have reckoned it
all out, so that there will be enough left\ldots{} and Katerina Ivanovna
was very anxious it should be so. You know one can't\ldots{} it's a
comfort to her\ldots{} she is like that, you know\ldots{}.''

``I understand, I understand\ldots{} of course\ldots{} why do you look
at my room like that? My mother has just said it is like a tomb.''

``You gave us everything yesterday,'' Sonia said suddenly, in reply, in
a loud rapid whisper; and again she looked down in confusion. Her lips
and chin were trembling once more. She had been struck at once by
Raskolnikov's poor surroundings, and now these words broke out
spontaneously. A silence followed. There was a light in Dounia's eyes,
and even Pulcheria Alexandrovna looked kindly at Sonia.

``Rodya,'' she said, getting up, ``we shall have dinner together, of
course. Come, Dounia\ldots{}. And you, Rodya, had better go for a little
walk, and then rest and lie down before you come to see us\ldots{}. I am
afraid we have exhausted you\ldots{}.''

``Yes, yes, I'll come,'' he answered, getting up fussily. ``But I have
something to see to.''

``But surely you will have dinner together?'' cried Razumihin, looking
in surprise at Raskolnikov. ``What do you mean?''

``Yes, yes, I am coming\ldots{} of course, of course! And you stay a
minute. You do not want him just now, do you, mother? Or perhaps I am
taking him from you?''

``Oh, no, no. And will you, Dmitri Prokofitch, do us the favour of
dining with us?''

``Please do,'' added Dounia.

Razumihin bowed, positively radiant. For one moment, they were all
strangely embarrassed.

``Good-bye, Rodya, that is till we meet. I do not like saying good-bye.
Good-bye, Nastasya. Ah, I have said good-bye again.''

Pulcheria Alexandrovna meant to greet Sonia, too; but it somehow failed
to come off, and she went in a flutter out of the room.

But Avdotya Romanovna seemed to await her turn, and following her mother
out, gave Sonia an attentive, courteous bow. Sonia, in confusion, gave a
hurried, frightened curtsy. There was a look of poignant discomfort in
her face, as though Avdotya Romanovna's courtesy and attention were
oppressive and painful to her.

``Dounia, good-bye,'' called Raskolnikov, in the passage. ``Give me your
hand.''

``Why, I did give it to you. Have you forgotten?'' said Dounia, turning
warmly and awkwardly to him.

``Never mind, give it to me again.'' And he squeezed her fingers warmly.

Dounia smiled, flushed, pulled her hand away, and went off quite happy.

``Come, that's capital,'' he said to Sonia, going back and looking
brightly at her. ``God give peace to the dead, the living have still to
live. That is right, isn't it?''

Sonia looked surprised at the sudden brightness of his face. He looked
at her for some moments in silence. The whole history of the dead father
floated before his memory in those moments\ldots{}.

\begin{center}\rule{0.5\linewidth}{\linethickness}\end{center}

``Heavens, Dounia,'' Pulcheria Alexandrovna began, as soon as they were
in the street, ``I really feel relieved myself at coming away---more at
ease. How little did I think yesterday in the train that I could ever be
glad of that.''

``I tell you again, mother, he is still very ill. Don't you see it?
Perhaps worrying about us upset him. We must be patient, and much, much
can be forgiven.''

``Well, you were not very patient!'' Pulcheria Alexandrovna caught her
up, hotly and jealously. ``Do you know, Dounia, I was looking at you
two. You are the very portrait of him, and not so much in face as in
soul. You are both melancholy, both morose and hot-tempered, both
haughty and both generous\ldots{}. Surely he can't be an egoist, Dounia.
Eh? When I think of what is in store for us this evening, my heart
sinks!''

``Don't be uneasy, mother. What must be, will be.''

``Dounia, only think what a position we are in! What if Pyotr Petrovitch
breaks it off?'' poor Pulcheria Alexandrovna blurted out, incautiously.

``He won't be worth much if he does,'' answered Dounia, sharply and
contemptuously.

``We did well to come away,'' Pulcheria Alexandrovna hurriedly broke in.
``He was in a hurry about some business or other. If he gets out and has
a breath of air\ldots{} it is fearfully close in his room\ldots{}. But
where is one to get a breath of air here? The very streets here feel
like shut-up rooms. Good heavens! what a town!\ldots{} stay\ldots{} this
side\ldots{} they will crush you---carrying something. Why, it is a
piano they have got, I declare\ldots{} how they push!\ldots{} I am very
much afraid of that young woman, too.''

``What young woman, mother?

``Why, that Sofya Semyonovna, who was there just now.''

``Why?''

``I have a presentiment, Dounia. Well, you may believe it or not, but as
soon as she came in, that very minute, I felt that she was the chief
cause of the trouble\ldots{}.''

``Nothing of the sort!'' cried Dounia, in vexation. ``What nonsense,
with your presentiments, mother! He only made her acquaintance the
evening before, and he did not know her when she came in.''

``Well, you will see\ldots{}. She worries me; but you will see, you will
see! I was so frightened. She was gazing at me with those eyes. I could
scarcely sit still in my chair when he began introducing her, do you
remember? It seems so strange, but Pyotr Petrovitch writes like that
about her, and he introduces her to us---to you! So he must think a
great deal of her.''

``People will write anything. We were talked about and written about,
too. Have you forgotten? I am sure that she is a good girl, and that it
is all nonsense.''

``God grant it may be!''

``And Pyotr Petrovitch is a contemptible slanderer,'' Dounia snapped
out, suddenly.

Pulcheria Alexandrovna was crushed; the conversation was not resumed.

\begin{center}\rule{0.5\linewidth}{\linethickness}\end{center}

``I will tell you what I want with you,'' said Raskolnikov, drawing
Razumihin to the window.

``Then I will tell Katerina Ivanovna that you are coming,'' Sonia said
hurriedly, preparing to depart.

``One minute, Sofya Semyonovna. We have no secrets. You are not in our
way. I want to have another word or two with you. Listen!'' he turned
suddenly to Razumihin again. ``You know that\ldots{} what's his
name\ldots{} Porfiry Petrovitch?''

``I should think so! He is a relation. Why?'' added the latter, with
interest.

``Is not he managing that case\ldots{} you know, about that
murder?\ldots{} You were speaking about it yesterday.''

``Yes\ldots{} well?'' Razumihin's eyes opened wide.

``He was inquiring for people who had pawned things, and I have some
pledges there, too---trifles---a ring my sister gave me as a keepsake
when I left home, and my father's silver watch---they are only worth
five or six roubles altogether\ldots{} but I value them. So what am I to
do now? I do not want to lose the things, especially the watch. I was
quaking just now, for fear mother would ask to look at it, when we spoke
of Dounia's watch. It is the only thing of father's left us. She would
be ill if it were lost. You know what women are. So tell me what to do.
I know I ought to have given notice at the police station, but would it
not be better to go straight to Porfiry? Eh? What do you think? The
matter might be settled more quickly. You see, mother may ask for it
before dinner.''

``Certainly not to the police station. Certainly to Porfiry,'' Razumihin
shouted in extraordinary excitement. ``Well, how glad I am. Let us go at
once. It is a couple of steps. We shall be sure to find him.''

``Very well, let us go.''

``And he will be very, very glad to make your acquaintance. I have often
talked to him of you at different times. I was speaking of you
yesterday. Let us go. So you knew the old woman? So that's it! It is all
turning out splendidly\ldots{}. Oh, yes, Sofya Ivanovna\ldots{}''

``Sofya Semyonovna,'' corrected Raskolnikov. ``Sofya Semyonovna, this is
my friend Razumihin, and he is a good man.''

``If you have to go now,'' Sonia was beginning, not looking at Razumihin
at all, and still more embarrassed.

``Let us go,'' decided Raskolnikov. ``I will come to you to-day, Sofya
Semyonovna. Only tell me where you live.''

He was not exactly ill at ease, but seemed hurried, and avoided her
eyes. Sonia gave her address, and flushed as she did so. They all went
out together.

``Don't you lock up?'' asked Razumihin, following him on to the stairs.

``Never,'' answered Raskolnikov. ``I have been meaning to buy a lock for
these two years. People are happy who have no need of locks,'' he said,
laughing, to Sonia. They stood still in the gateway.

``Do you go to the right, Sofya Semyonovna? How did you find me, by the
way?'' he added, as though he wanted to say something quite different.
He wanted to look at her soft clear eyes, but this was not easy.

``Why, you gave your address to Polenka yesterday.''

``Polenka? Oh, yes; Polenka, that is the little girl. She is your
sister? Did I give her the address?''

``Why, had you forgotten?''

``No, I remember.''

``I had heard my father speak of you\ldots{} only I did not know your
name, and he did not know it. And now I came\ldots{} and as I had learnt
your name, I asked to-day, `Where does Mr. Raskolnikov live?' I did not
know you had only a room too\ldots{}. Good-bye, I will tell Katerina
Ivanovna.''

She was extremely glad to escape at last; she went away looking down,
hurrying to get out of sight as soon as possible, to walk the twenty
steps to the turning on the right and to be at last alone, and then
moving rapidly along, looking at no one, noticing nothing, to think, to
remember, to meditate on every word, every detail. Never, never had she
felt anything like this. Dimly and unconsciously a whole new world was
opening before her. She remembered suddenly that Raskolnikov meant to
come to her that day, perhaps at once!

``Only not to-day, please, not to-day!'' she kept muttering with a
sinking heart, as though entreating someone, like a frightened child.
``Mercy! to me\ldots{} to that room\ldots{} he will see\ldots{} oh,
dear!''

She was not capable at that instant of noticing an unknown gentleman who
was watching her and following at her heels. He had accompanied her from
the gateway. At the moment when Razumihin, Raskolnikov, and she stood
still at parting on the pavement, this gentleman, who was just passing,
started on hearing Sonia's words: ``and I asked where Mr. Raskolnikov
lived?'' He turned a rapid but attentive look upon all three, especially
upon Raskolnikov, to whom Sonia was speaking; then looked back and noted
the house. All this was done in an instant as he passed, and trying not
to betray his interest, he walked on more slowly as though waiting for
something. He was waiting for Sonia; he saw that they were parting, and
that Sonia was going home.

``Home? Where? I've seen that face somewhere,'' he thought. ``I must
find out.''

At the turning he crossed over, looked round, and saw Sonia coming the
same way, noticing nothing. She turned the corner. He followed her on
the other side. After about fifty paces he crossed over again, overtook
her and kept two or three yards behind her.

He was a man about fifty, rather tall and thickly set, with broad high
shoulders which made him look as though he stooped a little. He wore
good and fashionable clothes, and looked like a gentleman of position.
He carried a handsome cane, which he tapped on the pavement at each
step; his gloves were spotless. He had a broad, rather pleasant face
with high cheek-bones and a fresh colour, not often seen in Petersburg.
His flaxen hair was still abundant, and only touched here and there with
grey, and his thick square beard was even lighter than his hair. His
eyes were blue and had a cold and thoughtful look; his lips were
crimson. He was a remarkedly well-preserved man and looked much younger
than his years.

When Sonia came out on the canal bank, they were the only two persons on
the pavement. He observed her dreaminess and preoccupation. On reaching
the house where she lodged, Sonia turned in at the gate; he followed
her, seeming rather surprised. In the courtyard she turned to the right
corner. ``Bah!'' muttered the unknown gentleman, and mounted the stairs
behind her. Only then Sonia noticed him. She reached the third storey,
turned down the passage, and rang at No. 9. On the door was inscribed in
chalk, ``Kapernaumov, Tailor.'' ``Bah!'' the stranger repeated again,
wondering at the strange coincidence, and he rang next door, at No. 8.
The doors were two or three yards apart.

``You lodge at Kapernaumov's,'' he said, looking at Sonia and laughing.
``He altered a waistcoat for me yesterday. I am staying close here at
Madame Resslich's. How odd!'' Sonia looked at him attentively.

``We are neighbours,'' he went on gaily. ``I only came to town the day
before yesterday. Good-bye for the present.''

Sonia made no reply; the door opened and she slipped in. She felt for
some reason ashamed and uneasy.

\begin{center}\rule{0.5\linewidth}{\linethickness}\end{center}

On the way to Porfiry's, Razumihin was obviously excited.

``That's capital, brother,'' he repeated several times, ``and I am glad!
I am glad!''

``What are you glad about?'' Raskolnikov thought to himself.

``I didn't know that you pledged things at the old woman's, too.
And\ldots{} was it long ago? I mean, was it long since you were there?''

``What a simple-hearted fool he is!''

``When was it?'' Raskolnikov stopped still to recollect. ``Two or three
days before her death it must have been. But I am not going to redeem
the things now,'' he put in with a sort of hurried and conspicuous
solicitude about the things. ``I've not more than a silver rouble
left\ldots{} after last night's accursed delirium!''

He laid special emphasis on the delirium.

``Yes, yes,'' Razumihin hastened to agree---with what was not clear.
``Then that's why you\ldots{} were stuck\ldots{} partly\ldots{} you know
in your delirium you were continually mentioning some rings or chains!
Yes, yes\ldots{} that's clear, it's all clear now.''

``Hullo! How that idea must have got about among them. Here this man
will go to the stake for me, and I find him delighted at having it
\emph{cleared up} why I spoke of rings in my delirium! What a hold the
idea must have on all of them!''

``Shall we find him?'' he asked suddenly.

``Oh, yes,'' Razumihin answered quickly. ``He is a nice fellow, you will
see, brother. Rather clumsy, that is to say, he is a man of polished
manners, but I mean clumsy in a different sense. He is an intelligent
fellow, very much so indeed, but he has his own range of ideas\ldots{}.
He is incredulous, sceptical, cynical\ldots{} he likes to impose on
people, or rather to make fun of them. His is the old, circumstantial
method\ldots{}. But he understands his work\ldots{} thoroughly\ldots{}.
Last year he cleared up a case of murder in which the police had hardly
a clue. He is very, very anxious to make your acquaintance!''

``On what grounds is he so anxious?''

``Oh, it's not exactly\ldots{} you see, since you've been ill I happen
to have mentioned you several times\ldots{}. So, when he heard about
you\ldots{} about your being a law student and not able to finish your
studies, he said, `What a pity!' And so I concluded\ldots{} from
everything together, not only that; yesterday Zametov\ldots{} you know,
Rodya, I talked some nonsense on the way home to you yesterday, when I
was drunk\ldots{} I am afraid, brother, of your exaggerating it, you
see.''

``What? That they think I am a madman? Maybe they are right,'' he said
with a constrained smile.

``Yes, yes\ldots{}. That is, pooh, no!\ldots{} But all that I said (and
there was something else too) it was all nonsense, drunken nonsense.''

``But why are you apologising? I am so sick of it all!'' Raskolnikov
cried with exaggerated irritability. It was partly assumed, however.

``I know, I know, I understand. Believe me, I understand. One's ashamed
to speak of it.''

``If you are ashamed, then don't speak of it.''

Both were silent. Razumihin was more than ecstatic and Raskolnikov
perceived it with repulsion. He was alarmed, too, by what Razumihin had
just said about Porfiry.

``I shall have to pull a long face with him too,'' he thought, with a
beating heart, and he turned white, ``and do it naturally, too. But the
most natural thing would be to do nothing at all. Carefully do nothing
at all! No, \emph{carefully} would not be natural again\ldots{}. Oh,
well, we shall see how it turns out\ldots{}. We shall see\ldots{}
directly. Is it a good thing to go or not? The butterfly flies to the
light. My heart is beating, that's what's bad!''

``In this grey house,'' said Razumihin.

``The most important thing, does Porfiry know that I was at the old
hag's flat yesterday\ldots{} and asked about the blood? I must find that
out instantly, as soon as I go in, find out from his face;
otherwise\ldots{} I'll find out, if it's my ruin.''

``I say, brother,'' he said suddenly, addressing Razumihin, with a sly
smile, ``I have been noticing all day that you seem to be curiously
excited. Isn't it so?''

``Excited? Not a bit of it,'' said Razumihin, stung to the quick.

``Yes, brother, I assure you it's noticeable. Why, you sat on your chair
in a way you never do sit, on the edge somehow, and you seemed to be
writhing all the time. You kept jumping up for nothing. One moment you
were angry, and the next your face looked like a sweetmeat. You even
blushed; especially when you were invited to dinner, you blushed
awfully.''

``Nothing of the sort, nonsense! What do you mean?''

``But why are you wriggling out of it, like a schoolboy? By Jove, there
he's blushing again.''

``What a pig you are!''

``But why are you so shamefaced about it? Romeo! Stay, I'll tell of you
to-day. Ha-ha-ha! I'll make mother laugh, and someone else,
too\ldots{}''

``Listen, listen, listen, this is serious\ldots{}. What next, you
fiend!'' Razumihin was utterly overwhelmed, turning cold with horror.
``What will you tell them? Come, brother\ldots{} foo! what a pig you
are!''

``You are like a summer rose. And if only you knew how it suits you; a
Romeo over six foot high! And how you've washed to-day---you cleaned
your nails, I declare. Eh? That's something unheard of! Why, I do
believe you've got pomatum on your hair! Bend down.''

``Pig!''

Raskolnikov laughed as though he could not restrain himself. So
laughing, they entered Porfiry Petrovitch's flat. This is what
Raskolnikov wanted: from within they could be heard laughing as they
came in, still guffawing in the passage.

``Not a word here or I'll\ldots{} brain you!'' Razumihin whispered
furiously, seizing Raskolnikov by the shoulder.

\protect\hypertarget{link2HCH0019}{}{}

\mychapter{5}{Chapter V}\label{chapter-v-2}

Raskolnikov was already entering the room. He came in looking as though
he had the utmost difficulty not to burst out laughing again. Behind him
Razumihin strode in gawky and awkward, shamefaced and red as a peony,
with an utterly crestfallen and ferocious expression. His face and whole
figure really were ridiculous at that moment and amply justified
Raskolnikov's laughter. Raskolnikov, not waiting for an introduction,
bowed to Porfiry Petrovitch, who stood in the middle of the room looking
inquiringly at them. He held out his hand and shook hands, still
apparently making desperate efforts to subdue his mirth and utter a few
words to introduce himself. But he had no sooner succeeded in assuming a
serious air and muttering something when he suddenly glanced again as
though accidentally at Razumihin, and could no longer control himself:
his stifled laughter broke out the more irresistibly the more he tried
to restrain it. The extraordinary ferocity with which Razumihin received
this ``spontaneous'' mirth gave the whole scene the appearance of most
genuine fun and naturalness. Razumihin strengthened this impression as
though on purpose.

``Fool! You fiend,'' he roared, waving his arm which at once struck a
little round table with an empty tea-glass on it. Everything was sent
flying and crashing.

``But why break chairs, gentlemen? You know it's a loss to the Crown,''
Porfiry Petrovitch quoted gaily.

Raskolnikov was still laughing, with his hand in Porfiry Petrovitch's,
but anxious not to overdo it, awaited the right moment to put a natural
end to it. Razumihin, completely put to confusion by upsetting the table
and smashing the glass, gazed gloomily at the fragments, cursed and
turned sharply to the window where he stood looking out with his back to
the company with a fiercely scowling countenance, seeing nothing.
Porfiry Petrovitch laughed and was ready to go on laughing, but
obviously looked for explanations. Zametov had been sitting in the
corner, but he rose at the visitors' entrance and was standing in
expectation with a smile on his lips, though he looked with surprise and
even it seemed incredulity at the whole scene and at Raskolnikov with a
certain embarrassment. Zametov's unexpected presence struck Raskolnikov
unpleasantly.

``I've got to think of that,'' he thought. ``Excuse me, please,'' he
began, affecting extreme embarrassment. ``Raskolnikov.''

``Not at all, very pleasant to see you\ldots{} and how pleasantly you've
come in\ldots{}. Why, won't he even say good-morning?'' Porfiry
Petrovitch nodded at Razumihin.

``Upon my honour I don't know why he is in such a rage with me. I only
told him as we came along that he was like Romeo\ldots{} and proved it.
And that was all, I think!''

``Pig!'' ejaculated Razumihin, without turning round.

``There must have been very grave grounds for it, if he is so furious at
the word,'' Porfiry laughed.

``Oh, you sharp lawyer!\ldots{} Damn you all!'' snapped Razumihin, and
suddenly bursting out laughing himself, he went up to Porfiry with a
more cheerful face as though nothing had happened. ``That'll do! We are
all fools. To come to business. This is my friend Rodion Romanovitch
Raskolnikov; in the first place he has heard of you and wants to make
your acquaintance, and secondly, he has a little matter of business with
you. Bah! Zametov, what brought you here? Have you met before? Have you
known each other long?''

``What does this mean?'' thought Raskolnikov uneasily.

Zametov seemed taken aback, but not very much so.

``Why, it was at your rooms we met yesterday,'' he said easily.

``Then I have been spared the trouble. All last week he was begging me
to introduce him to you. Porfiry and you have sniffed each other out
without me. Where is your tobacco?''

Porfiry Petrovitch was wearing a dressing-gown, very clean linen, and
trodden-down slippers. He was a man of about five and thirty, short,
stout even to corpulence, and clean shaven. He wore his hair cut short
and had a large round head, particularly prominent at the back. His
soft, round, rather snub-nosed face was of a sickly yellowish colour,
but had a vigorous and rather ironical expression. It would have been
good-natured except for a look in the eyes, which shone with a watery,
mawkish light under almost white, blinking eyelashes. The expression of
those eyes was strangely out of keeping with his somewhat womanish
figure, and gave it something far more serious than could be guessed at
first sight.

As soon as Porfiry Petrovitch heard that his visitor had a little matter
of business with him, he begged him to sit down on the sofa and sat down
himself on the other end, waiting for him to explain his business, with
that careful and over-serious attention which is at once oppressive and
embarrassing, especially to a stranger, and especially if what you are
discussing is in your opinion of far too little importance for such
exceptional solemnity. But in brief and coherent phrases Raskolnikov
explained his business clearly and exactly, and was so well satisfied
with himself that he even succeeded in taking a good look at Porfiry.
Porfiry Petrovitch did not once take his eyes off him. Razumihin,
sitting opposite at the same table, listened warmly and impatiently,
looking from one to the other every moment with rather excessive
interest.

``Fool,'' Raskolnikov swore to himself.

``You have to give information to the police,'' Porfiry replied, with a
most businesslike air, ``that having learnt of this incident, that is of
the murder, you beg to inform the lawyer in charge of the case that such
and such things belong to you, and that you desire to redeem
them\ldots{} or\ldots{} but they will write to you.''

``That's just the point, that at the present moment,'' Raskolnikov tried
his utmost to feign embarrassment, ``I am not quite in funds\ldots{} and
even this trifling sum is beyond me\ldots{} I only wanted, you see, for
the present to declare that the things are mine, and that when I have
money\ldots{}.''

``That's no matter,'' answered Porfiry Petrovitch, receiving his
explanation of his pecuniary position coldly, ``but you can, if you
prefer, write straight to me, to say, that having been informed of the
matter, and claiming such and such as your property, you beg\ldots{}''

``On an ordinary sheet of paper?'' Raskolnikov interrupted eagerly,
again interested in the financial side of the question.

``Oh, the most ordinary,'' and suddenly Porfiry Petrovitch looked with
obvious irony at him, screwing up his eyes and, as it were, winking at
him. But perhaps it was Raskolnikov's fancy, for it all lasted but a
moment. There was certainly something of the sort, Raskolnikov could
have sworn he winked at him, goodness knows why.

``He knows,'' flashed through his mind like lightning.

``Forgive my troubling you about such trifles,'' he went on, a little
disconcerted, ``the things are only worth five roubles, but I prize them
particularly for the sake of those from whom they came to me, and I must
confess that I was alarmed when I heard\ldots{}''

``That's why you were so much struck when I mentioned to Zossimov that
Porfiry was inquiring for everyone who had pledges!'' Razumihin put in
with obvious intention.

This was really unbearable. Raskolnikov could not help glancing at him
with a flash of vindictive anger in his black eyes, but immediately
recollected himself.

``You seem to be jeering at me, brother?'' he said to him, with a
well-feigned irritability. ``I dare say I do seem to you absurdly
anxious about such trash; but you mustn't think me selfish or grasping
for that, and these two things may be anything but trash in my eyes. I
told you just now that the silver watch, though it's not worth a cent,
is the only thing left us of my father's. You may laugh at me, but my
mother is here,'' he turned suddenly to Porfiry, ``and if she knew,'' he
turned again hurriedly to Razumihin, carefully making his voice tremble,
``that the watch was lost, she would be in despair! You know what women
are!''

``Not a bit of it! I didn't mean that at all! Quite the contrary!''
shouted Razumihin distressed.

``Was it right? Was it natural? Did I overdo it?'' Raskolnikov asked
himself in a tremor. ``Why did I say that about women?''

``Oh, your mother is with you?'' Porfiry Petrovitch inquired.

``Yes.''

``When did she come?''

``Last night.''

Porfiry paused as though reflecting.

``Your things would not in any case be lost,'' he went on calmly and
coldly. ``I have been expecting you here for some time.''

And as though that was a matter of no importance, he carefully offered
the ash-tray to Razumihin, who was ruthlessly scattering cigarette ash
over the carpet. Raskolnikov shuddered, but Porfiry did not seem to be
looking at him, and was still concerned with Razumihin's cigarette.

``What? Expecting him? Why, did you know that he had pledges
\emph{there}?'' cried Razumihin.

Porfiry Petrovitch addressed himself to Raskolnikov.

``Your things, the ring and the watch, were wrapped up together, and on
the paper your name was legibly written in pencil, together with the
date on which you left them with her\ldots{}''

``How observant you are!'' Raskolnikov smiled awkwardly, doing his very
utmost to look him straight in the face, but he failed, and suddenly
added:

``I say that because I suppose there were a great many pledges\ldots{}
that it must be difficult to remember them all\ldots{}. But you remember
them all so clearly, and\ldots{} and\ldots{}''

``Stupid! Feeble!'' he thought. ``Why did I add that?''

``But we know all who had pledges, and you are the only one who hasn't
come forward,'' Porfiry answered with hardly perceptible irony.

``I haven't been quite well.''

``I heard that too. I heard, indeed, that you were in great distress
about something. You look pale still.''

``I am not pale at all\ldots{}. No, I am quite well,'' Raskolnikov
snapped out rudely and angrily, completely changing his tone. His anger
was mounting, he could not repress it. ``And in my anger I shall betray
myself,'' flashed through his mind again. ``Why are they torturing me?''

``Not quite well!'' Razumihin caught him up. ``What next! He was
unconscious and delirious all yesterday. Would you believe, Porfiry, as
soon as our backs were turned, he dressed, though he could hardly stand,
and gave us the slip and went off on a spree somewhere till midnight,
delirious all the time! Would you believe it! Extraordinary!''

``Really delirious? You don't say so!'' Porfiry shook his head in a
womanish way.

``Nonsense! Don't you believe it! But you don't believe it anyway,''
Raskolnikov let slip in his anger. But Porfiry Petrovitch did not seem
to catch those strange words.

``But how could you have gone out if you hadn't been delirious?''
Razumihin got hot suddenly. ``What did you go out for? What was the
object of it? And why on the sly? Were you in your senses when you did
it? Now that all danger is over I can speak plainly.''

``I was awfully sick of them yesterday.'' Raskolnikov addressed Porfiry
suddenly with a smile of insolent defiance, ``I ran away from them to
take lodgings where they wouldn't find me, and took a lot of money with
me. Mr. Zametov there saw it. I say, Mr. Zametov, was I sensible or
delirious yesterday; settle our dispute.''

He could have strangled Zametov at that moment, so hateful were his
expression and his silence to him.

``In my opinion you talked sensibly and even artfully, but you were
extremely irritable,'' Zametov pronounced dryly.

``And Nikodim Fomitch was telling me to-day,'' put in Porfiry
Petrovitch, ``that he met you very late last night in the lodging of a
man who had been run over.''

``And there,'' said Razumihin, ``weren't you mad then? You gave your
last penny to the widow for the funeral. If you wanted to help, give
fifteen or twenty even, but keep three roubles for yourself at least,
but he flung away all the twenty-five at once!''

``Maybe I found a treasure somewhere and you know nothing of it? So
that's why I was liberal yesterday\ldots{}. Mr. Zametov knows I've found
a treasure! Excuse us, please, for disturbing you for half an hour with
such trivialities,'' he said, turning to Porfiry Petrovitch, with
trembling lips. ``We are boring you, aren't we?''

``Oh no, quite the contrary, quite the contrary! If only you knew how
you interest me! It's interesting to look on and listen\ldots{} and I am
really glad you have come forward at last.''

``But you might give us some tea! My throat's dry,'' cried Razumihin.

``Capital idea! Perhaps we will all keep you company. Wouldn't you
like\ldots{} something more essential before tea?''

``Get along with you!''

Porfiry Petrovitch went out to order tea.

Raskolnikov's thoughts were in a whirl. He was in terrible exasperation.

``The worst of it is they don't disguise it; they don't care to stand on
ceremony! And how if you didn't know me at all, did you come to talk to
Nikodim Fomitch about me? So they don't care to hide that they are
tracking me like a pack of dogs. They simply spit in my face.'' He was
shaking with rage. ``Come, strike me openly, don't play with me like a
cat with a mouse. It's hardly civil, Porfiry Petrovitch, but perhaps I
won't allow it! I shall get up and throw the whole truth in your ugly
faces, and you'll see how I despise you.'' He could hardly breathe.
``And what if it's only my fancy? What if I am mistaken, and through
inexperience I get angry and don't keep up my nasty part? Perhaps it's
all unintentional. All their phrases are the usual ones, but there is
something about them\ldots{}. It all might be said, but there is
something. Why did he say bluntly, `With her'? Why did Zametov add that
I spoke artfully? Why do they speak in that tone? Yes, the tone\ldots{}.
Razumihin is sitting here, why does he see nothing? That innocent
blockhead never does see anything! Feverish again! Did Porfiry wink at
me just now? Of course it's nonsense! What could he wink for? Are they
trying to upset my nerves or are they teasing me? Either it's ill fancy
or they know! Even Zametov is rude\ldots{}. Is Zametov rude? Zametov has
changed his mind. I foresaw he would change his mind! He is at home
here, while it's my first visit. Porfiry does not consider him a
visitor; sits with his back to him. They're as thick as thieves, no
doubt, over me! Not a doubt they were talking about me before we came.
Do they know about the flat? If only they'd make haste! When I said that
I ran away to take a flat he let it pass\ldots{}. I put that in cleverly
about a flat, it may be of use afterwards\ldots{}. Delirious,
indeed\ldots{} ha-ha-ha! He knows all about last night! He didn't know
of my mother's arrival! The hag had written the date on in pencil! You
are wrong, you won't catch me! There are no facts\ldots{} it's all
supposition! You produce facts! The flat even isn't a fact but delirium.
I know what to say to them\ldots{}. Do they know about the flat? I won't
go without finding out. What did I come for? But my being angry now,
maybe is a fact! Fool, how irritable I am! Perhaps that's right; to play
the invalid\ldots{}. He is feeling me. He will try to catch me. Why did
I come?''

All this flashed like lightning through his mind.

Porfiry Petrovitch returned quickly. He became suddenly more jovial.

``Your party yesterday, brother, has left my head rather\ldots{}. And I
am out of sorts altogether,'' he began in quite a different tone,
laughing to Razumihin.

``Was it interesting? I left you yesterday at the most interesting
point. Who got the best of it?''

``Oh, no one, of course. They got on to everlasting questions, floated
off into space.''

``Only fancy, Rodya, what we got on to yesterday. Whether there is such
a thing as crime. I told you that we talked our heads off.''

``What is there strange? It's an everyday social question,'' Raskolnikov
answered casually.

``The question wasn't put quite like that,'' observed Porfiry.

``Not quite, that's true,'' Razumihin agreed at once, getting warm and
hurried as usual. ``Listen, Rodion, and tell us your opinion, I want to
hear it. I was fighting tooth and nail with them and wanted you to help
me. I told them you were coming\ldots{}. It began with the socialist
doctrine. You know their doctrine; crime is a protest against the
abnormality of the social organisation and nothing more, and nothing
more; no other causes admitted!\ldots{}''

``You are wrong there,'' cried Porfiry Petrovitch; he was noticeably
animated and kept laughing as he looked at Razumihin, which made him
more excited than ever.

``Nothing is admitted,'' Razumihin interrupted with heat.

``I am not wrong. I'll show you their pamphlets. Everything with them is
`the influence of environment,' and nothing else. Their favourite
phrase! From which it follows that, if society is normally organised,
all crime will cease at once, since there will be nothing to protest
against and all men will become righteous in one instant. Human nature
is not taken into account, it is excluded, it's not supposed to exist!
They don't recognise that humanity, developing by a historical living
process, will become at last a normal society, but they believe that a
social system that has come out of some mathematical brain is going to
organise all humanity at once and make it just and sinless in an
instant, quicker than any living process! That's why they instinctively
dislike history, `nothing but ugliness and stupidity in it,' and they
explain it all as stupidity! That's why they so dislike the
\emph{living} process of life; they don't want a \emph{living soul}! The
living soul demands life, the soul won't obey the rules of mechanics,
the soul is an object of suspicion, the soul is retrograde! But what
they want though it smells of death and can be made of India-rubber, at
least is not alive, has no will, is servile and won't revolt! And it
comes in the end to their reducing everything to the building of walls
and the planning of rooms and passages in a phalanstery! The phalanstery
is ready, indeed, but your human nature is not ready for the
phalanstery---it wants life, it hasn't completed its vital process, it's
too soon for the graveyard! You can't skip over nature by logic. Logic
presupposes three possibilities, but there are millions! Cut away a
million, and reduce it all to the question of comfort! That's the
easiest solution of the problem! It's seductively clear and you musn't
think about it. That's the great thing, you mustn't think! The whole
secret of life in two pages of print!''

``Now he is off, beating the drum! Catch hold of him, do!'' laughed
Porfiry. ``Can you imagine,'' he turned to Raskolnikov, ``six people
holding forth like that last night, in one room, with punch as a
preliminary! No, brother, you are wrong, environment accounts for a
great deal in crime; I can assure you of that.''

``Oh, I know it does, but just tell me: a man of forty violates a child
of ten; was it environment drove him to it?''

``Well, strictly speaking, it did,'' Porfiry observed with noteworthy
gravity; ``a crime of that nature may be very well ascribed to the
influence of environment.''

Razumihin was almost in a frenzy. ``Oh, if you like,'' he roared. ``I'll
prove to you that your white eyelashes may very well be ascribed to the
Church of Ivan the Great's being two hundred and fifty feet high, and I
will prove it clearly, exactly, progressively, and even with a Liberal
tendency! I undertake to! Will you bet on it?''

``Done! Let's hear, please, how he will prove it!''

``He is always humbugging, confound him,'' cried Razumihin, jumping up
and gesticulating. ``What's the use of talking to you? He does all that
on purpose; you don't know him, Rodion! He took their side yesterday,
simply to make fools of them. And the things he said yesterday! And they
were delighted! He can keep it up for a fortnight together. Last year he
persuaded us that he was going into a monastery: he stuck to it for two
months. Not long ago he took it into his head to declare he was going to
get married, that he had everything ready for the wedding. He ordered
new clothes indeed. We all began to congratulate him. There was no
bride, nothing, all pure fantasy!''

``Ah, you are wrong! I got the clothes before. It was the new clothes in
fact that made me think of taking you in.''

``Are you such a good dissembler?'' Raskolnikov asked carelessly.

``You wouldn't have supposed it, eh? Wait a bit, I shall take you in,
too. Ha-ha-ha! No, I'll tell you the truth. All these questions about
crime, environment, children, recall to my mind an article of yours
which interested me at the time. `On Crime'\ldots{} or something of the
sort, I forget the title, I read it with pleasure two months ago in the
\emph{Periodical Review}.''

``My article? In the \emph{Periodical Review}?'' Raskolnikov asked in
astonishment. ``I certainly did write an article upon a book six months
ago when I left the university, but I sent it to the \emph{Weekly
Review}.''

``But it came out in the \emph{Periodical}.''

``And the \emph{Weekly Review} ceased to exist, so that's why it wasn't
printed at the time.''

``That's true; but when it ceased to exist, the \emph{Weekly Review} was
amalgamated with the \emph{Periodical}, and so your article appeared two
months ago in the latter. Didn't you know?''

Raskolnikov had not known.

``Why, you might get some money out of them for the article! What a
strange person you are! You lead such a solitary life that you know
nothing of matters that concern you directly. It's a fact, I assure
you.''

``Bravo, Rodya! I knew nothing about it either!'' cried Razumihin.
``I'll run to-day to the reading-room and ask for the number. Two months
ago? What was the date? It doesn't matter though, I will find it. Think
of not telling us!''

``How did you find out that the article was mine? It's only signed with
an initial.''

``I only learnt it by chance, the other day. Through the editor; I know
him\ldots{}. I was very much interested.''

``I analysed, if I remember, the psychology of a criminal before and
after the crime.''

``Yes, and you maintained that the perpetration of a crime is always
accompanied by illness. Very, very original, but\ldots{} it was not that
part of your article that interested me so much, but an idea at the end
of the article which I regret to say you merely suggested without
working it out clearly. There is, if you recollect, a suggestion that
there are certain persons who can\ldots{} that is, not precisely are
able to, but have a perfect right to commit breaches of morality and
crimes, and that the law is not for them.''

Raskolnikov smiled at the exaggerated and intentional distortion of his
idea.

``What? What do you mean? A right to crime? But not because of the
influence of environment?'' Razumihin inquired with some alarm even.

``No, not exactly because of it,'' answered Porfiry. ``In his article
all men are divided into `ordinary' and `extraordinary.' Ordinary men
have to live in submission, have no right to transgress the law,
because, don't you see, they are ordinary. But extraordinary men have a
right to commit any crime and to transgress the law in any way, just
because they are extraordinary. That was your idea, if I am not
mistaken?''

``What do you mean? That can't be right?'' Razumihin muttered in
bewilderment.

Raskolnikov smiled again. He saw the point at once, and knew where they
wanted to drive him. He decided to take up the challenge.

``That wasn't quite my contention,'' he began simply and modestly. ``Yet
I admit that you have stated it almost correctly; perhaps, if you like,
perfectly so.'' (It almost gave him pleasure to admit this.) ``The only
difference is that I don't contend that extraordinary people are always
bound to commit breaches of morals, as you call it. In fact, I doubt
whether such an argument could be published. I simply hinted that an
`extraordinary' man has the right\ldots{} that is not an official right,
but an inner right to decide in his own conscience to overstep\ldots{}
certain obstacles, and only in case it is essential for the practical
fulfilment of his idea (sometimes, perhaps, of benefit to the whole of
humanity). You say that my article isn't definite; I am ready to make it
as clear as I can. Perhaps I am right in thinking you want me to; very
well. I maintain that if the discoveries of Kepler and Newton could not
have been made known except by sacrificing the lives of one, a dozen, a
hundred, or more men, Newton would have had the right, would indeed have
been in duty-bound\ldots{} to \emph{eliminate} the dozen or the hundred
men for the sake of making his discoveries known to the whole of
humanity. But it does not follow from that that Newton had a right to
murder people right and left and to steal every day in the market. Then,
I remember, I maintain in my article that all\ldots{} well, legislators
and leaders of men, such as Lycurgus, Solon, Mahomet, Napoleon, and so
on, were all without exception criminals, from the very fact that,
making a new law, they transgressed the ancient one, handed down from
their ancestors and held sacred by the people, and they did not stop
short at bloodshed either, if that bloodshed---often of innocent persons
fighting bravely in defence of ancient law---were of use to their cause.
It's remarkable, in fact, that the majority, indeed, of these
benefactors and leaders of humanity were guilty of terrible carnage. In
short, I maintain that all great men or even men a little out of the
common, that is to say capable of giving some new word, must from their
very nature be criminals---more or less, of course. Otherwise it's hard
for them to get out of the common rut; and to remain in the common rut
is what they can't submit to, from their very nature again, and to my
mind they ought not, indeed, to submit to it. You see that there is
nothing particularly new in all that. The same thing has been printed
and read a thousand times before. As for my division of people into
ordinary and extraordinary, I acknowledge that it's somewhat arbitrary,
but I don't insist upon exact numbers. I only believe in my leading idea
that men are \emph{in general} divided by a law of nature into two
categories, inferior (ordinary), that is, so to say, material that
serves only to reproduce its kind, and men who have the gift or the
talent to utter \emph{a new word}. There are, of course, innumerable
sub-divisions, but the distinguishing features of both categories are
fairly well marked. The first category, generally speaking, are men
conservative in temperament and law-abiding; they live under control and
love to be controlled. To my thinking it is their duty to be controlled,
because that's their vocation, and there is nothing humiliating in it
for them. The second category all transgress the law; they are
destroyers or disposed to destruction according to their capacities. The
crimes of these men are of course relative and varied; for the most part
they seek in very varied ways the destruction of the present for the
sake of the better. But if such a one is forced for the sake of his idea
to step over a corpse or wade through blood, he can, I maintain, find
within himself, in his conscience, a sanction for wading through
blood---that depends on the idea and its dimensions, note that. It's
only in that sense I speak of their right to crime in my article (you
remember it began with the legal question). There's no need for such
anxiety, however; the masses will scarcely ever admit this right, they
punish them or hang them (more or less), and in doing so fulfil quite
justly their conservative vocation. But the same masses set these
criminals on a pedestal in the next generation and worship them (more or
less). The first category is always the man of the present, the second
the man of the future. The first preserve the world and people it, the
second move the world and lead it to its goal. Each class has an equal
right to exist. In fact, all have equal rights with me---and \emph{vive
la guerre éternelle}---till the New Jerusalem, of course!''

``Then you believe in the New Jerusalem, do you?''

``I do,'' Raskolnikov answered firmly; as he said these words and during
the whole preceding tirade he kept his eyes on one spot on the carpet.

``And\ldots{} and do you believe in God? Excuse my curiosity.''

``I do,'' repeated Raskolnikov, raising his eyes to Porfiry.

``And\ldots{} do you believe in Lazarus' rising from the dead?''

``I\ldots{} I do. Why do you ask all this?''

``You believe it literally?''

``Literally.''

``You don't say so\ldots{}. I asked from curiosity. Excuse me. But let
us go back to the question; they are not always executed. Some, on the
contrary\ldots{}''

``Triumph in their lifetime? Oh, yes, some attain their ends in this
life, and then\ldots{}''

``They begin executing other people?''

``If it's necessary; indeed, for the most part they do. Your remark is
very witty.''

``Thank you. But tell me this: how do you distinguish those
extraordinary people from the ordinary ones? Are there signs at their
birth? I feel there ought to be more exactitude, more external
definition. Excuse the natural anxiety of a practical law-abiding
citizen, but couldn't they adopt a special uniform, for instance,
couldn't they wear something, be branded in some way? For you know if
confusion arises and a member of one category imagines that he belongs
to the other, begins to `eliminate obstacles' as you so happily
expressed it, then\ldots{}''

``Oh, that very often happens! That remark is wittier than the other.''

``Thank you.''

``No reason to; but take note that the mistake can only arise in the
first category, that is among the ordinary people (as I perhaps
unfortunately called them). In spite of their predisposition to
obedience very many of them, through a playfulness of nature, sometimes
vouchsafed even to the cow, like to imagine themselves advanced people,
`destroyers,' and to push themselves into the `new movement,' and this
quite sincerely. Meanwhile the really \emph{new} people are very often
unobserved by them, or even despised as reactionaries of grovelling
tendencies. But I don't think there is any considerable danger here, and
you really need not be uneasy for they never go very far. Of course,
they might have a thrashing sometimes for letting their fancy run away
with them and to teach them their place, but no more; in fact, even this
isn't necessary as they castigate themselves, for they are very
conscientious: some perform this service for one another and others
chastise themselves with their own hands\ldots{}. They will impose
various public acts of penitence upon themselves with a beautiful and
edifying effect; in fact you've nothing to be uneasy about\ldots{}. It's
a law of nature.''

``Well, you have certainly set my mind more at rest on that score; but
there's another thing worries me. Tell me, please, are there many people
who have the right to kill others, these extraordinary people? I am
ready to bow down to them, of course, but you must admit it's alarming
if there are a great many of them, eh?''

``Oh, you needn't worry about that either,'' Raskolnikov went on in the
same tone. ``People with new ideas, people with the faintest capacity
for saying something \emph{new}, are extremely few in number,
extraordinarily so in fact. One thing only is clear, that the appearance
of all these grades and sub-divisions of men must follow with unfailing
regularity some law of nature. That law, of course, is unknown at
present, but I am convinced that it exists, and one day may become
known. The vast mass of mankind is mere material, and only exists in
order by some great effort, by some mysterious process, by means of some
crossing of races and stocks, to bring into the world at last perhaps
one man out of a thousand with a spark of independence. One in ten
thousand perhaps---I speak roughly, approximately---is born with some
independence, and with still greater independence one in a hundred
thousand. The man of genius is one of millions, and the great geniuses,
the crown of humanity, appear on earth perhaps one in many thousand
millions. In fact I have not peeped into the retort in which all this
takes place. But there certainly is and must be a definite law, it
cannot be a matter of chance.''

``Why, are you both joking?'' Razumihin cried at last. ``There you sit,
making fun of one another. Are you serious, Rodya?''

Raskolnikov raised his pale and almost mournful face and made no reply.
And the unconcealed, persistent, nervous, and \emph{discourteous}
sarcasm of Porfiry seemed strange to Razumihin beside that quiet and
mournful face.

``Well, brother, if you are really serious\ldots{} You are right, of
course, in saying that it's not new, that it's like what we've read and
heard a thousand times already; but what is really original in all this,
and is exclusively your own, to my horror, is that you sanction
bloodshed \emph{in the name of conscience}, and, excuse my saying so,
with such fanaticism\ldots{}. That, I take it, is the point of your
article. But that sanction of bloodshed \emph{by conscience} is to my
mind\ldots{} more terrible than the official, legal sanction of
bloodshed\ldots{}.''

``You are quite right, it is more terrible,'' Porfiry agreed.

``Yes, you must have exaggerated! There is some mistake, I shall read
it. You can't think that! I shall read it.''

``All that is not in the article, there's only a hint of it,'' said
Raskolnikov.

``Yes, yes.'' Porfiry couldn't sit still. ``Your attitude to crime is
pretty clear to me now, but\ldots{} excuse me for my impertinence (I am
really ashamed to be worrying you like this), you see, you've removed my
anxiety as to the two grades getting mixed, but\ldots{} there are
various practical possibilities that make me uneasy! What if some man or
youth imagines that he is a Lycurgus or Mahomet---a future one of
course---and suppose he begins to remove all obstacles\ldots{}. He has
some great enterprise before him and needs money for it\ldots{} and
tries to get it\ldots{} do you see?''

Zametov gave a sudden guffaw in his corner. Raskolnikov did not even
raise his eyes to him.

``I must admit,'' he went on calmly, ``that such cases certainly must
arise. The vain and foolish are particularly apt to fall into that
snare; young people especially.''

``Yes, you see. Well then?''

``What then?'' Raskolnikov smiled in reply; ``that's not my fault. So it
is and so it always will be. He said just now (he nodded at Razumihin)
that I sanction bloodshed. Society is too well protected by prisons,
banishment, criminal investigators, penal servitude. There's no need to
be uneasy. You have but to catch the thief.''

``And what if we do catch him?''

``Then he gets what he deserves.''

``You are certainly logical. But what of his conscience?''

``Why do you care about that?''

``Simply from humanity.''

``If he has a conscience he will suffer for his mistake. That will be
his punishment---as well as the prison.''

``But the real geniuses,'' asked Razumihin frowning, ``those who have
the right to murder? Oughtn't they to suffer at all even for the blood
they've shed?''

``Why the word \emph{ought}? It's not a matter of permission or
prohibition. He will suffer if he is sorry for his victim. Pain and
suffering are always inevitable for a large intelligence and a deep
heart. The really great men must, I think, have great sadness on
earth,'' he added dreamily, not in the tone of the conversation.

He raised his eyes, looked earnestly at them all, smiled, and took his
cap. He was too quiet by comparison with his manner at his entrance, and
he felt this. Everyone got up.

``Well, you may abuse me, be angry with me if you like,'' Porfiry
Petrovitch began again, ``but I can't resist. Allow me one little
question (I know I am troubling you). There is just one little notion I
want to express, simply that I may not forget it.''

``Very good, tell me your little notion,'' Raskolnikov stood waiting,
pale and grave before him.

``Well, you see\ldots{} I really don't know how to express it
properly\ldots{}. It's a playful, psychological idea\ldots{}. When you
were writing your article, surely you couldn't have helped, he-he!
fancying yourself\ldots{} just a little, an `extraordinary' man,
uttering a \emph{new word} in your sense\ldots{}. That's so, isn't it?''

``Quite possibly,'' Raskolnikov answered contemptuously.

Razumihin made a movement.

``And, if so, could you bring yourself in case of worldly difficulties
and hardship or for some service to humanity---to overstep
obstacles?\ldots{} For instance, to rob and murder?''

And again he winked with his left eye, and laughed noiselessly just as
before.

``If I did I certainly should not tell you,'' Raskolnikov answered with
defiant and haughty contempt.

``No, I was only interested on account of your article, from a literary
point of view\ldots{}''

``Foo! how obvious and insolent that is!'' Raskolnikov thought with
repulsion.

``Allow me to observe,'' he answered dryly, ``that I don't consider
myself a Mahomet or a Napoleon, nor any personage of that kind, and not
being one of them I cannot tell you how I should act.''

``Oh, come, don't we all think ourselves Napoleons now in Russia?''
Porfiry Petrovitch said with alarming familiarity.

Something peculiar betrayed itself in the very intonation of his voice.

``Perhaps it was one of these future Napoleons who did for Alyona
Ivanovna last week?'' Zametov blurted out from the corner.

Raskolnikov did not speak, but looked firmly and intently at Porfiry.
Razumihin was scowling gloomily. He seemed before this to be noticing
something. He looked angrily around. There was a minute of gloomy
silence. Raskolnikov turned to go.

``Are you going already?'' Porfiry said amiably, holding out his hand
with excessive politeness. ``Very, very glad of your acquaintance. As
for your request, have no uneasiness, write just as I told you, or,
better still, come to me there yourself in a day or two\ldots{}
to-morrow, indeed. I shall be there at eleven o'clock for certain. We'll
arrange it all; we'll have a talk. As one of the last to be
\emph{there}, you might perhaps be able to tell us something,'' he added
with a most good-natured expression.

``You want to cross-examine me officially in due form?'' Raskolnikov
asked sharply.

``Oh, why? That's not necessary for the present. You misunderstand me. I
lose no opportunity, you see, and\ldots{} I've talked with all who had
pledges\ldots{}. I obtained evidence from some of them, and you are the
last\ldots{}. Yes, by the way,'' he cried, seemingly suddenly delighted,
``I just remember, what was I thinking of?'' he turned to Razumihin,
``you were talking my ears off about that Nikolay\ldots{} of course, I
know, I know very well,'' he turned to Raskolnikov, ``that the fellow is
innocent, but what is one to do? We had to trouble Dmitri too\ldots{}.
This is the point, this is all: when you went up the stairs it was past
seven, wasn't it?''

``Yes,'' answered Raskolnikov, with an unpleasant sensation at the very
moment he spoke that he need not have said it.

``Then when you went upstairs between seven and eight, didn't you see in
a flat that stood open on a second storey, do you remember? two workmen
or at least one of them? They were painting there, didn't you notice
them? It's very, very important for them.''

``Painters? No, I didn't see them,'' Raskolnikov answered slowly, as
though ransacking his memory, while at the same instant he was racking
every nerve, almost swooning with anxiety to conjecture as quickly as
possible where the trap lay and not to overlook anything. ``No, I didn't
see them, and I don't think I noticed a flat like that open\ldots{}. But
on the fourth storey'' (he had mastered the trap now and was triumphant)
``I remember now that someone was moving out of the flat opposite Alyona
Ivanovna's\ldots{}. I remember\ldots{} I remember it clearly. Some
porters were carrying out a sofa and they squeezed me against the wall.
But painters\ldots{} no, I don't remember that there were any painters,
and I don't think that there was a flat open anywhere, no, there
wasn't.''

``What do you mean?'' Razumihin shouted suddenly, as though he had
reflected and realised. ``Why, it was on the day of the murder the
painters were at work, and he was there three days before? What are you
asking?''

``Foo! I have muddled it!'' Porfiry slapped himself on the forehead.
``Deuce take it! This business is turning my brain!'' he addressed
Raskolnikov somewhat apologetically. ``It would be such a great thing
for us to find out whether anyone had seen them between seven and eight
at the flat, so I fancied you could perhaps have told us
something\ldots{}. I quite muddled it.''

``Then you should be more careful,'' Razumihin observed grimly.

The last words were uttered in the passage. Porfiry Petrovitch saw them
to the door with excessive politeness.

They went out into the street gloomy and sullen, and for some steps they
did not say a word. Raskolnikov drew a deep breath.

\protect\hypertarget{link2HCH0020}{}{}

\mychapter{6}{Chapter VI}\label{chapter-vi-2}

``I don't believe it, I can't believe it!'' repeated Razumihin, trying
in perplexity to refute Raskolnikov's arguments.

They were by now approaching Bakaleyev's lodgings, where Pulcheria
Alexandrovna and Dounia had been expecting them a long while. Razumihin
kept stopping on the way in the heat of discussion, confused and excited
by the very fact that they were for the first time speaking openly about
\emph{it}.

``Don't believe it, then!'' answered Raskolnikov, with a cold, careless
smile. ``You were noticing nothing as usual, but I was weighing every
word.''

``You are suspicious. That is why you weighed their words\ldots{}
h'm\ldots{} certainly, I agree, Porfiry's tone was rather strange, and
still more that wretch Zametov!\ldots{} You are right, there was
something about him---but why? Why?''

``He has changed his mind since last night.''

``Quite the contrary! If they had that brainless idea, they would do
their utmost to hide it, and conceal their cards, so as to catch you
afterwards\ldots{}. But it was all impudent and careless.''

``If they had had facts---I mean, real facts---or at least grounds for
suspicion, then they would certainly have tried to hide their game, in
the hope of getting more (they would have made a search long ago
besides). But they have no facts, not one. It is all mirage---all
ambiguous. Simply a floating idea. So they try to throw me out by
impudence. And perhaps, he was irritated at having no facts, and blurted
it out in his vexation---or perhaps he has some plan\ldots{} he seems an
intelligent man. Perhaps he wanted to frighten me by pretending to know.
They have a psychology of their own, brother. But it is loathsome
explaining it all. Stop!''

``And it's insulting, insulting! I understand you. But\ldots{} since we
have spoken openly now (and it is an excellent thing that we have at
last---I am glad) I will own now frankly that I noticed it in them long
ago, this idea. Of course the merest hint only---an insinuation---but
why an insinuation even? How dare they? What foundation have they? If
only you knew how furious I have been. Think only! Simply because a poor
student, unhinged by poverty and hypochondria, on the eve of a severe
delirious illness (note that), suspicious, vain, proud, who has not seen
a soul to speak to for six months, in rags and in boots without soles,
has to face some wretched policemen and put up with their insolence; and
the unexpected debt thrust under his nose, the I.O.U. presented by
Tchebarov, the new paint, thirty degrees Reaumur and a stifling
atmosphere, a crowd of people, the talk about the murder of a person
where he had been just before, and all that on an empty stomach---he
might well have a fainting fit! And that, that is what they found it all
on! Damn them! I understand how annoying it is, but in your place,
Rodya, I would laugh at them, or better still, spit in their ugly faces,
and spit a dozen times in all directions. I'd hit out in all directions,
neatly too, and so I'd put an end to it. Damn them! Don't be
downhearted. It's a shame!''

``He really has put it well, though,'' Raskolnikov thought.

``Damn them? But the cross-examination again, to-morrow?'' he said with
bitterness. ``Must I really enter into explanations with them? I feel
vexed as it is, that I condescended to speak to Zametov yesterday in the
restaurant\ldots{}.''

``Damn it! I will go myself to Porfiry. I will squeeze it out of him, as
one of the family: he must let me know the ins and outs of it all! And
as for Zametov\ldots{}''

``At last he sees through him!'' thought Raskolnikov.

``Stay!'' cried Razumihin, seizing him by the shoulder again. ``Stay!
you were wrong. I have thought it out. You are wrong! How was that a
trap? You say that the question about the workmen was a trap. But if you
had done \emph{that}, could you have said you had seen them painting the
flat\ldots{} and the workmen? On the contrary, you would have seen
nothing, even if you had seen it. Who would own it against himself?''

``If I had done \emph{that thing}, I should certainly have said that I
had seen the workmen and the flat,'' Raskolnikov answered, with
reluctance and obvious disgust.

``But why speak against yourself?''

``Because only peasants, or the most inexperienced novices deny
everything flatly at examinations. If a man is ever so little developed
and experienced, he will certainly try to admit all the external facts
that can't be avoided, but will seek other explanations of them, will
introduce some special, unexpected turn, that will give them another
significance and put them in another light. Porfiry might well reckon
that I should be sure to answer so, and say I had seen them to give an
air of truth, and then make some explanation.''

``But he would have told you at once that the workmen could not have
been there two days before, and that therefore you must have been there
on the day of the murder at eight o'clock. And so he would have caught
you over a detail.''

``Yes, that is what he was reckoning on, that I should not have time to
reflect, and should be in a hurry to make the most likely answer, and so
would forget that the workmen could not have been there two days
before.''

``But how could you forget it?''

``Nothing easier. It is in just such stupid things clever people are
most easily caught. The more cunning a man is, the less he suspects that
he will be caught in a simple thing. The more cunning a man is, the
simpler the trap he must be caught in. Porfiry is not such a fool as you
think\ldots{}.''

``He is a knave then, if that is so!''

Raskolnikov could not help laughing. But at the very moment, he was
struck by the strangeness of his own frankness, and the eagerness with
which he had made this explanation, though he had kept up all the
preceding conversation with gloomy repulsion, obviously with a motive,
from necessity.

``I am getting a relish for certain aspects!'' he thought to himself.
But almost at the same instant he became suddenly uneasy, as though an
unexpected and alarming idea had occurred to him. His uneasiness kept on
increasing. They had just reached the entrance to Bakaleyev's.

``Go in alone!'' said Raskolnikov suddenly. ``I will be back directly.''

``Where are you going? Why, we are just here.''

``I can't help it\ldots{}. I will come in half an hour. Tell them.''

``Say what you like, I will come with you.''

``You, too, want to torture me!'' he screamed, with such bitter
irritation, such despair in his eyes that Razumihin's hands dropped. He
stood for some time on the steps, looking gloomily at Raskolnikov
striding rapidly away in the direction of his lodging. At last, gritting
his teeth and clenching his fist, he swore he would squeeze Porfiry like
a lemon that very day, and went up the stairs to reassure Pulcheria
Alexandrovna, who was by now alarmed at their long absence.

When Raskolnikov got home, his hair was soaked with sweat and he was
breathing heavily. He went rapidly up the stairs, walked into his
unlocked room and at once fastened the latch. Then in senseless terror
he rushed to the corner, to that hole under the paper where he had put
the things; put his hand in, and for some minutes felt carefully in the
hole, in every crack and fold of the paper. Finding nothing, he got up
and drew a deep breath. As he was reaching the steps of Bakaleyev's, he
suddenly fancied that something, a chain, a stud or even a bit of paper
in which they had been wrapped with the old woman's handwriting on it,
might somehow have slipped out and been lost in some crack, and then
might suddenly turn up as unexpected, conclusive evidence against him.

He stood as though lost in thought, and a strange, humiliated, half
senseless smile strayed on his lips. He took his cap at last and went
quietly out of the room. His ideas were all tangled. He went dreamily
through the gateway.

``Here he is himself,'' shouted a loud voice.

He raised his head.

The porter was standing at the door of his little room and was pointing
him out to a short man who looked like an artisan, wearing a long coat
and a waistcoat, and looking at a distance remarkably like a woman. He
stooped, and his head in a greasy cap hung forward. From his wrinkled
flabby face he looked over fifty; his little eyes were lost in fat and
they looked out grimly, sternly and discontentedly.

``What is it?'' Raskolnikov asked, going up to the porter.

The man stole a look at him from under his brows and he looked at him
attentively, deliberately; then he turned slowly and went out of the
gate into the street without saying a word.

``What is it?'' cried Raskolnikov.

``Why, he there was asking whether a student lived here, mentioned your
name and whom you lodged with. I saw you coming and pointed you out and
he went away. It's funny.''

The porter too seemed rather puzzled, but not much so, and after
wondering for a moment he turned and went back to his room.

Raskolnikov ran after the stranger, and at once caught sight of him
walking along the other side of the street with the same even,
deliberate step with his eyes fixed on the ground, as though in
meditation. He soon overtook him, but for some time walked behind him.
At last, moving on to a level with him, he looked at his face. The man
noticed him at once, looked at him quickly, but dropped his eyes again;
and so they walked for a minute side by side without uttering a word.

``You were inquiring for me\ldots{} of the porter?'' Raskolnikov said at
last, but in a curiously quiet voice.

The man made no answer; he didn't even look at him. Again they were both
silent.

``Why do you\ldots{} come and ask for me\ldots{} and say
nothing\ldots{}. What's the meaning of it?''

Raskolnikov's voice broke and he seemed unable to articulate the words
clearly.

The man raised his eyes this time and turned a gloomy sinister look at
Raskolnikov.

``Murderer!'' he said suddenly in a quiet but clear and distinct voice.

Raskolnikov went on walking beside him. His legs felt suddenly weak, a
cold shiver ran down his spine, and his heart seemed to stand still for
a moment, then suddenly began throbbing as though it were set free. So
they walked for about a hundred paces, side by side in silence.

The man did not look at him.

``What do you mean\ldots{} what is\ldots{}. Who is a murderer?''
muttered Raskolnikov hardly audibly.

``\emph{You} are a murderer,'' the man answered still more articulately
and emphatically, with a smile of triumphant hatred, and again he looked
straight into Raskolnikov's pale face and stricken eyes.

They had just reached the cross-roads. The man turned to the left
without looking behind him. Raskolnikov remained standing, gazing after
him. He saw him turn round fifty paces away and look back at him still
standing there. Raskolnikov could not see clearly, but he fancied that
he was again smiling the same smile of cold hatred and triumph.

With slow faltering steps, with shaking knees, Raskolnikov made his way
back to his little garret, feeling chilled all over. He took off his cap
and put it on the table, and for ten minutes he stood without moving.
Then he sank exhausted on the sofa and with a weak moan of pain he
stretched himself on it. So he lay for half an hour.

He thought of nothing. Some thoughts or fragments of thoughts, some
images without order or coherence floated before his mind---faces of
people he had seen in his childhood or met somewhere once, whom he would
never have recalled, the belfry of the church at V., the billiard table
in a restaurant and some officers playing billiards, the smell of cigars
in some underground tobacco shop, a tavern room, a back staircase quite
dark, all sloppy with dirty water and strewn with egg-shells, and the
Sunday bells floating in from somewhere\ldots{}. The images followed one
another, whirling like a hurricane. Some of them he liked and tried to
clutch at, but they faded and all the while there was an oppression
within him, but it was not overwhelming, sometimes it was even
pleasant\ldots{}. The slight shivering still persisted, but that too was
an almost pleasant sensation.

He heard the hurried footsteps of Razumihin; he closed his eyes and
pretended to be asleep. Razumihin opened the door and stood for some
time in the doorway as though hesitating, then he stepped softly into
the room and went cautiously to the sofa. Raskolnikov heard Nastasya's
whisper:

``Don't disturb him! Let him sleep. He can have his dinner later.''

``Quite so,'' answered Razumihin. Both withdrew carefully and closed the
door. Another half-hour passed. Raskolnikov opened his eyes, turned on
his back again, clasping his hands behind his head.

``Who is he? Who is that man who sprang out of the earth? Where was he,
what did he see? He has seen it all, that's clear. Where was he then?
And from where did he see? Why has he only now sprung out of the earth?
And how could he see? Is it possible? Hm\ldots{}'' continued
Raskolnikov, turning cold and shivering, ``and the jewel case Nikolay
found behind the door---was that possible? A clue? You miss an
infinitesimal line and you can build it into a pyramid of evidence! A
fly flew by and saw it! Is it possible?'' He felt with sudden loathing
how weak, how physically weak he had become. ``I ought to have known
it,'' he thought with a bitter smile. ``And how dared I, knowing myself,
knowing how I should be, take up an axe and shed blood! I ought to have
known beforehand\ldots{}. Ah, but I did know!'' he whispered in despair.
At times he came to a standstill at some thought.

``No, those men are not made so. The real \emph{Master} to whom all is
permitted storms Toulon, makes a massacre in Paris, \emph{forgets} an
army in Egypt, \emph{wastes} half a million men in the Moscow expedition
and gets off with a jest at Vilna. And altars are set up to him after
his death, and so \emph{all} is permitted. No, such people, it seems,
are not of flesh but of bronze!''

One sudden irrelevant idea almost made him laugh. Napoleon, the
pyramids, Waterloo, and a wretched skinny old woman, a pawnbroker with a
red trunk under her bed---it's a nice hash for Porfiry Petrovitch to
digest! How can they digest it! It's too inartistic. ``A Napoleon creep
under an old woman's bed! Ugh, how loathsome!''

At moments he felt he was raving. He sank into a state of feverish
excitement. ``The old woman is of no consequence,'' he thought, hotly
and incoherently. ``The old woman was a mistake perhaps, but she is not
what matters! The old woman was only an illness\ldots{}. I was in a
hurry to overstep\ldots{}. I didn't kill a human being, but a principle!
I killed the principle, but I didn't overstep, I stopped on this
side\ldots{}. I was only capable of killing. And it seems I wasn't even
capable of that\ldots{} Principle? Why was that fool Razumihin abusing
the socialists? They are industrious, commercial people; `the happiness
of all' is their case. No, life is only given to me once and I shall
never have it again; I don't want to wait for `the happiness of all.' I
want to live myself, or else better not live at all. I simply couldn't
pass by my mother starving, keeping my rouble in my pocket while I
waited for the `happiness of all.' I am putting my little brick into the
happiness of all and so my heart is at peace. Ha-ha! Why have you let me
slip? I only live once, I too want\ldots{}. Ech, I am an æsthetic louse
and nothing more,'' he added suddenly, laughing like a madman. ``Yes, I
am certainly a louse,'' he went on, clutching at the idea, gloating over
it and playing with it with vindictive pleasure. ``In the first place,
because I can reason that I am one, and secondly, because for a month
past I have been troubling benevolent Providence, calling it to witness
that not for my own fleshly lusts did I undertake it, but with a grand
and noble object---ha-ha! Thirdly, because I aimed at carrying it out as
justly as possible, weighing, measuring and calculating. Of all the lice
I picked out the most useless one and proposed to take from her only as
much as I needed for the first step, no more nor less (so the rest would
have gone to a monastery, according to her will, ha-ha!). And what shows
that I am utterly a louse,'' he added, grinding his teeth, ``is that I
am perhaps viler and more loathsome than the louse I killed, and \emph{I
felt beforehand} that I should tell myself so \emph{after} killing her.
Can anything be compared with the horror of that? The vulgarity! The
abjectness! I understand the `prophet' with his sabre, on his steed:
Allah commands and `trembling' creation must obey! The `prophet' is
right, he is right when he sets a battery across the street and blows up
the innocent and the guilty without deigning to explain! It's for you to
obey, trembling creation, and not \emph{to have desires}, for that's not
for you!\ldots{} I shall never, never forgive the old woman!''

His hair was soaked with sweat, his quivering lips were parched, his
eyes were fixed on the ceiling.

``Mother, sister---how I loved them! Why do I hate them now? Yes, I hate
them, I feel a physical hatred for them, I can't bear them near
me\ldots{}. I went up to my mother and kissed her, I remember\ldots{}.
To embrace her and think if she only knew\ldots{} shall I tell her then?
That's just what I might do\ldots{}. \emph{She} must be the same as I
am,'' he added, straining himself to think, as it were struggling with
delirium. ``Ah, how I hate the old woman now! I feel I should kill her
again if she came to life! Poor Lizaveta! Why did she come in?\ldots{}
It's strange though, why is it I scarcely ever think of her, as though I
hadn't killed her? Lizaveta! Sonia! Poor gentle things, with gentle
eyes\ldots{}. Dear women! Why don't they weep? Why don't they moan? They
give up everything\ldots{} their eyes are soft and gentle\ldots{}.
Sonia, Sonia! Gentle Sonia!''

He lost consciousness; it seemed strange to him that he didn't remember
how he got into the street. It was late evening. The twilight had fallen
and the full moon was shining more and more brightly; but there was a
peculiar breathlessness in the air. There were crowds of people in the
street; workmen and business people were making their way home; other
people had come out for a walk; there was a smell of mortar, dust and
stagnant water. Raskolnikov walked along, mournful and anxious; he was
distinctly aware of having come out with a purpose, of having to do
something in a hurry, but what it was he had forgotten. Suddenly he
stood still and saw a man standing on the other side of the street,
beckoning to him. He crossed over to him, but at once the man turned and
walked away with his head hanging, as though he had made no sign to him.
``Stay, did he really beckon?'' Raskolnikov wondered, but he tried to
overtake him. When he was within ten paces he recognised him and was
frightened; it was the same man with stooping shoulders in the long
coat. Raskolnikov followed him at a distance; his heart was beating;
they went down a turning; the man still did not look round. ``Does he
know I am following him?'' thought Raskolnikov. The man went into the
gateway of a big house. Raskolnikov hastened to the gate and looked in
to see whether he would look round and sign to him. In the court-yard
the man did turn round and again seemed to beckon him. Raskolnikov at
once followed him into the yard, but the man was gone. He must have gone
up the first staircase. Raskolnikov rushed after him. He heard slow
measured steps two flights above. The staircase seemed strangely
familiar. He reached the window on the first floor; the moon shone
through the panes with a melancholy and mysterious light; then he
reached the second floor. Bah! this is the flat where the painters were
at work\ldots{} but how was it he did not recognise it at once? The
steps of the man above had died away. ``So he must have stopped or
hidden somewhere.'' He reached the third storey, should he go on? There
was a stillness that was dreadful\ldots{}. But he went on. The sound of
his own footsteps scared and frightened him. How dark it was! The man
must be hiding in some corner here. Ah! the flat was standing wide open,
he hesitated and went in. It was very dark and empty in the passage, as
though everything had been removed; he crept on tiptoe into the parlour
which was flooded with moonlight. Everything there was as before, the
chairs, the looking-glass, the yellow sofa and the pictures in the
frames. A huge, round, copper-red moon looked in at the windows. ``It's
the moon that makes it so still, weaving some mystery,'' thought
Raskolnikov. He stood and waited, waited a long while, and the more
silent the moonlight, the more violently his heart beat, till it was
painful. And still the same hush. Suddenly he heard a momentary sharp
crack like the snapping of a splinter and all was still again. A fly
flew up suddenly and struck the window pane with a plaintive buzz. At
that moment he noticed in the corner between the window and the little
cupboard something like a cloak hanging on the wall. ``Why is that cloak
here?'' he thought, ``it wasn't there before\ldots{}.'' He went up to it
quietly and felt that there was someone hiding behind it. He cautiously
moved the cloak and saw, sitting on a chair in the corner, the old woman
bent double so that he couldn't see her face; but it was she. He stood
over her. ``She is afraid,'' he thought. He stealthily took the axe from
the noose and struck her one blow, then another on the skull. But
strange to say she did not stir, as though she were made of wood. He was
frightened, bent down nearer and tried to look at her; but she, too,
bent her head lower. He bent right down to the ground and peeped up into
her face from below, he peeped and turned cold with horror: the old
woman was sitting and laughing, shaking with noiseless laughter, doing
her utmost that he should not hear it. Suddenly he fancied that the door
from the bedroom was opened a little and that there was laughter and
whispering within. He was overcome with frenzy and he began hitting the
old woman on the head with all his force, but at every blow of the axe
the laughter and whispering from the bedroom grew louder and the old
woman was simply shaking with mirth. He was rushing away, but the
passage was full of people, the doors of the flats stood open and on the
landing, on the stairs and everywhere below there were people, rows of
heads, all looking, but huddled together in silence and expectation.
Something gripped his heart, his legs were rooted to the spot, they
would not move\ldots{}. He tried to scream and woke up.

He drew a deep breath---but his dream seemed strangely to persist: his
door was flung open and a man whom he had never seen stood in the
doorway watching him intently.

Raskolnikov had hardly opened his eyes and he instantly closed them
again. He lay on his back without stirring.

``Is it still a dream?'' he wondered and again raised his eyelids hardly
perceptibly; the stranger was standing in the same place, still watching
him.

He stepped cautiously into the room, carefully closing the door after
him, went up to the table, paused a moment, still keeping his eyes on
Raskolnikov, and noiselessly seated himself on the chair by the sofa; he
put his hat on the floor beside him and leaned his hands on his cane and
his chin on his hands. It was evident that he was prepared to wait
indefinitely. As far as Raskolnikov could make out from his stolen
glances, he was a man no longer young, stout, with a full, fair, almost
whitish beard.

Ten minutes passed. It was still light, but beginning to get dusk. There
was complete stillness in the room. Not a sound came from the stairs.
Only a big fly buzzed and fluttered against the window pane. It was
unbearable at last. Raskolnikov suddenly got up and sat on the sofa.

``Come, tell me what you want.''

``I knew you were not asleep, but only pretending,'' the stranger
answered oddly, laughing calmly. ``Arkady Ivanovitch Svidrigaïlov, allow
me to introduce myself\ldots{}.''

\part*{PART IV}
\addcontentsline{toc}{part}{PART IV}

\mychapter{1}{Chapter I}\label{chapter-i-3}

``Can this be still a dream?'' Raskolnikov thought once more.

He looked carefully and suspiciously at the unexpected visitor.

``Svidrigaïlov! What nonsense! It can't be!'' he said at last aloud in
bewilderment.

His visitor did not seem at all surprised at this exclamation.

``I've come to you for two reasons. In the first place, I wanted to make
your personal acquaintance, as I have already heard a great deal about
you that is interesting and flattering; secondly, I cherish the hope
that you may not refuse to assist me in a matter directly concerning the
welfare of your sister, Avdotya Romanovna. For without your support she
might not let me come near her now, for she is prejudiced against me,
but with your assistance I reckon on\ldots{}''

``You reckon wrongly,'' interrupted Raskolnikov.

``They only arrived yesterday, may I ask you?''

Raskolnikov made no reply.

``It was yesterday, I know. I only arrived myself the day before. Well,
let me tell you this, Rodion Romanovitch, I don't consider it necessary
to justify myself, but kindly tell me what was there particularly
criminal on my part in all this business, speaking without prejudice,
with common sense?''

Raskolnikov continued to look at him in silence.

``That in my own house I persecuted a defenceless girl and `insulted her
with my infamous proposals'---is that it? (I am anticipating you.) But
you've only to assume that I, too, am a man \emph{et nihil
humanum}\ldots{} in a word, that I am capable of being attracted and
falling in love (which does not depend on our will), then everything can
be explained in the most natural manner. The question is, am I a
monster, or am I myself a victim? And what if I am a victim? In
proposing to the object of my passion to elope with me to America or
Switzerland, I may have cherished the deepest respect for her and may
have thought that I was promoting our mutual happiness! Reason is the
slave of passion, you know; why, probably, I was doing more harm to
myself than anyone!''

``But that's not the point,'' Raskolnikov interrupted with disgust.
``It's simply that whether you are right or wrong, we dislike you. We
don't want to have anything to do with you. We show you the door. Go
out!''

Svidrigaïlov broke into a sudden laugh.

``But you're\ldots{} but there's no getting round you,'' he said,
laughing in the frankest way. ``I hoped to get round you, but you took
up the right line at once!''

``But you are trying to get round me still!''

``What of it? What of it?'' cried Svidrigaïlov, laughing openly. ``But
this is what the French call \emph{bonne guerre}, and the most innocent
form of deception!\ldots{} But still you have interrupted me; one way or
another, I repeat again: there would never have been any unpleasantness
except for what happened in the garden. Marfa Petrovna\ldots{}''

``You have got rid of Marfa Petrovna, too, so they say?'' Raskolnikov
interrupted rudely.

``Oh, you've heard that, too, then? You'd be sure to, though\ldots{}.
But as for your question, I really don't know what to say, though my own
conscience is quite at rest on that score. Don't suppose that I am in
any apprehension about it. All was regular and in order; the medical
inquiry diagnosed apoplexy due to bathing immediately after a heavy
dinner and a bottle of wine, and indeed it could have proved nothing
else. But I'll tell you what I have been thinking to myself of late, on
my way here in the train, especially: didn't I contribute to all
that\ldots{} calamity, morally, in a way, by irritation or something of
the sort. But I came to the conclusion that that, too, was quite out of
the question.''

Raskolnikov laughed.

``I wonder you trouble yourself about it!''

``But what are you laughing at? Only consider, I struck her just twice
with a switch---there were no marks even\ldots{} don't regard me as a
cynic, please; I am perfectly aware how atrocious it was of me and all
that; but I know for certain, too, that Marfa Petrovna was very likely
pleased at my, so to say, warmth. The story of your sister had been
wrung out to the last drop; for the last three days Marfa Petrovna had
been forced to sit at home; she had nothing to show herself with in the
town. Besides, she had bored them so with that letter (you heard about
her reading the letter). And all of a sudden those two switches fell
from heaven! Her first act was to order the carriage to be got
out\ldots{}. Not to speak of the fact that there are cases when women
are very, very glad to be insulted in spite of all their show of
indignation. There are instances of it with everyone; human beings in
general, indeed, greatly love to be insulted, have you noticed that? But
it's particularly so with women. One might even say it's their only
amusement.''

At one time Raskolnikov thought of getting up and walking out and so
finishing the interview. But some curiosity and even a sort of prudence
made him linger for a moment.

``You are fond of fighting?'' he asked carelessly.

``No, not very,'' Svidrigaïlov answered, calmly. ``And Marfa Petrovna
and I scarcely ever fought. We lived very harmoniously, and she was
always pleased with me. I only used the whip twice in all our seven
years (not counting a third occasion of a very ambiguous character). The
first time, two months after our marriage, immediately after we arrived
in the country, and the last time was that of which we are speaking. Did
you suppose I was such a monster, such a reactionary, such a slave
driver? Ha, ha! By the way, do you remember, Rodion Romanovitch, how a
few years ago, in those days of beneficent publicity, a nobleman, I've
forgotten his name, was put to shame everywhere, in all the papers, for
having thrashed a German woman in the railway train. You remember? It
was in those days, that very year I believe, the `disgraceful action of
the \emph{Age}' took place (you know, `The Egyptian Nights,' that public
reading, you remember? The dark eyes, you know! Ah, the golden days of
our youth, where are they?). Well, as for the gentleman who thrashed the
German, I feel no sympathy with him, because after all what need is
there for sympathy? But I must say that there are sometimes such
provoking `Germans' that I don't believe there is a progressive who
could quite answer for himself. No one looked at the subject from that
point of view then, but that's the truly humane point of view, I assure
you.''

After saying this, Svidrigaïlov broke into a sudden laugh again.
Raskolnikov saw clearly that this was a man with a firm purpose in his
mind and able to keep it to himself.

``I expect you've not talked to anyone for some days?'' he asked.

``Scarcely anyone. I suppose you are wondering at my being such an
adaptable man?''

``No, I am only wondering at your being too adaptable a man.''

``Because I am not offended at the rudeness of your questions? Is that
it? But why take offence? As you asked, so I answered,'' he replied,
with a surprising expression of simplicity. ``You know, there's hardly
anything I take interest in,'' he went on, as it were dreamily,
``especially now, I've nothing to do\ldots{}. You are quite at liberty
to imagine though that I am making up to you with a motive, particularly
as I told you I want to see your sister about something. But I'll
confess frankly, I am very much bored. The last three days especially,
so I am delighted to see you\ldots{}. Don't be angry, Rodion
Romanovitch, but you seem to be somehow awfully strange yourself. Say
what you like, there's something wrong with you, and now, too\ldots{}
not this very minute, I mean, but now, generally\ldots{}. Well, well, I
won't, I won't, don't scowl! I am not such a bear, you know, as you
think.''

Raskolnikov looked gloomily at him.

``You are not a bear, perhaps, at all,'' he said. ``I fancy indeed that
you are a man of very good breeding, or at least know how on occasion to
behave like one.''

``I am not particularly interested in anyone's opinion,'' Svidrigaïlov
answered, dryly and even with a shade of haughtiness, ``and therefore
why not be vulgar at times when vulgarity is such a convenient cloak for
our climate\ldots{} and especially if one has a natural propensity that
way,'' he added, laughing again.

``But I've heard you have many friends here. You are, as they say, `not
without connections.' What can you want with me, then, unless you've
some special object?''

``That's true that I have friends here,'' Svidrigaïlov admitted, not
replying to the chief point. ``I've met some already. I've been lounging
about for the last three days, and I've seen them, or they've seen me.
That's a matter of course. I am well dressed and reckoned not a poor
man; the emancipation of the serfs hasn't affected me; my property
consists chiefly of forests and water meadows. The revenue has not
fallen off; but\ldots{} I am not going to see them, I was sick of them
long ago. I've been here three days and have called on no one\ldots{}.
What a town it is! How has it come into existence among us, tell me
that? A town of officials and students of all sorts. Yes, there's a
great deal I didn't notice when I was here eight years ago, kicking up
my heels\ldots{}. My only hope now is in anatomy, by Jove, it is!''

``Anatomy?''

``But as for these clubs, Dussauts, parades, or progress, indeed,
maybe---well, all that can go on without me,'' he went on, again without
noticing the question. ``Besides, who wants to be a card-sharper?''

``Why, have you been a card-sharper then?''

``How could I help being? There was a regular set of us, men of the best
society, eight years ago; we had a fine time. And all men of breeding,
you know, poets, men of property. And indeed as a rule in our Russian
society the best manners are found among those who've been thrashed,
have you noticed that? I've deteriorated in the country. But I did get
into prison for debt, through a low Greek who came from Nezhin. Then
Marfa Petrovna turned up; she bargained with him and bought me off for
thirty thousand silver pieces (I owed seventy thousand). We were united
in lawful wedlock and she bore me off into the country like a treasure.
You know she was five years older than I. She was very fond of me. For
seven years I never left the country. And, take note, that all my life
she held a document over me, the IOU for thirty thousand roubles, so if
I were to elect to be restive about anything I should be trapped at
once! And she would have done it! Women find nothing incompatible in
that.''

``If it hadn't been for that, would you have given her the slip?''

``I don't know what to say. It was scarcely the document restrained me.
I didn't want to go anywhere else. Marfa Petrovna herself invited me to
go abroad, seeing I was bored, but I've been abroad before, and always
felt sick there. For no reason, but the sunrise, the bay of Naples, the
sea---you look at them and it makes you sad. What's most revolting is
that one is really sad! No, it's better at home. Here at least one
blames others for everything and excuses oneself. I should have gone
perhaps on an expedition to the North Pole, because \emph{j'ai le vin
mauvais} and hate drinking, and there's nothing left but wine. I have
tried it. But, I say, I've been told Berg is going up in a great balloon
next Sunday from the Yusupov Garden and will take up passengers at a
fee. Is it true?''

``Why, would you go up?''

``I\ldots{} No, oh, no,'' muttered Svidrigaïlov really seeming to be
deep in thought.

``What does he mean? Is he in earnest?'' Raskolnikov wondered.

``No, the document didn't restrain me,'' Svidrigaïlov went on,
meditatively. ``It was my own doing, not leaving the country, and nearly
a year ago Marfa Petrovna gave me back the document on my name-day and
made me a present of a considerable sum of money, too. She had a
fortune, you know. `You see how I trust you, Arkady Ivanovitch'---that
was actually her expression. You don't believe she used it? But do you
know I managed the estate quite decently, they know me in the
neighbourhood. I ordered books, too. Marfa Petrovna at first approved,
but afterwards she was afraid of my over-studying.''

``You seem to be missing Marfa Petrovna very much?''

``Missing her? Perhaps. Really, perhaps I am. And, by the way, do you
believe in ghosts?''

``What ghosts?''

``Why, ordinary ghosts.''

``Do you believe in them?''

``Perhaps not, \emph{pour vous plaire}\ldots{}. I wouldn't say no
exactly.''

``Do you see them, then?''

Svidrigaïlov looked at him rather oddly.

``Marfa Petrovna is pleased to visit me,'' he said, twisting his mouth
into a strange smile.

``How do you mean `she is pleased to visit you'?''

``She has been three times. I saw her first on the very day of the
funeral, an hour after she was buried. It was the day before I left to
come here. The second time was the day before yesterday, at daybreak, on
the journey at the station of Malaya Vishera, and the third time was two
hours ago in the room where I am staying. I was alone.''

``Were you awake?''

``Quite awake. I was wide awake every time. She comes, speaks to me for
a minute and goes out at the door---always at the door. I can almost
hear her.''

``What made me think that something of the sort must be happening to
you?'' Raskolnikov said suddenly.

At the same moment he was surprised at having said it. He was much
excited.

``What! Did you think so?'' Svidrigaïlov asked in astonishment. ``Did
you really? Didn't I say that there was something in common between us,
eh?''

``You never said so!'' Raskolnikov cried sharply and with heat.

``Didn't I?''

``No!''

``I thought I did. When I came in and saw you lying with your eyes shut,
pretending, I said to myself at once, `Here's the man.'\,''

``What do you mean by `the man?' What are you talking about?'' cried
Raskolnikov.

``What do I mean? I really don't know\ldots{}.'' Svidrigaïlov muttered
ingenuously, as though he, too, were puzzled.

For a minute they were silent. They stared in each other's faces.

``That's all nonsense!'' Raskolnikov shouted with vexation. ``What does
she say when she comes to you?''

``She! Would you believe it, she talks of the silliest trifles and---man
is a strange creature---it makes me angry. The first time she came in (I
was tired you know: the funeral service, the funeral ceremony, the lunch
afterwards. At last I was left alone in my study. I lighted a cigar and
began to think), she came in at the door. `You've been so busy to-day,
Arkady Ivanovitch, you have forgotten to wind the dining-room clock,'
she said. All those seven years I've wound that clock every week, and if
I forgot it she would always remind me. The next day I set off on my way
here. I got out at the station at daybreak; I'd been asleep, tired out,
with my eyes half open, I was drinking some coffee. I looked up and
there was suddenly Marfa Petrovna sitting beside me with a pack of cards
in her hands. `Shall I tell your fortune for the journey, Arkady
Ivanovitch?' She was a great hand at telling fortunes. I shall never
forgive myself for not asking her to. I ran away in a fright, and,
besides, the bell rang. I was sitting to-day, feeling very heavy after a
miserable dinner from a cookshop; I was sitting smoking, all of a sudden
Marfa Petrovna again. She came in very smart in a new green silk dress
with a long train. `Good day, Arkady Ivanovitch! How do you like my
dress? Aniska can't make like this.' (Aniska was a dressmaker in the
country, one of our former serf girls who had been trained in Moscow, a
pretty wench.) She stood turning round before me. I looked at the dress,
and then I looked carefully, very carefully, at her face. `I wonder you
trouble to come to me about such trifles, Marfa Petrovna.' `Good
gracious, you won't let one disturb you about anything!' To tease her I
said, `I want to get married, Marfa Petrovna.' `That's just like you,
Arkady Ivanovitch; it does you very little credit to come looking for a
bride when you've hardly buried your wife. And if you could make a good
choice, at least, but I know it won't be for your happiness or hers, you
will only be a laughing-stock to all good people.' Then she went out and
her train seemed to rustle. Isn't it nonsense, eh?''

``But perhaps you are telling lies?'' Raskolnikov put in.

``I rarely lie,'' answered Svidrigaïlov thoughtfully, apparently not
noticing the rudeness of the question.

``And in the past, have you ever seen ghosts before?''

``Y-yes, I have seen them, but only once in my life, six years ago. I
had a serf, Filka; just after his burial I called out forgetting `Filka,
my pipe!' He came in and went to the cupboard where my pipes were. I sat
still and thought `he is doing it out of revenge,' because we had a
violent quarrel just before his death. `How dare you come in with a hole
in your elbow?' I said. `Go away, you scamp!' He turned and went out,
and never came again. I didn't tell Marfa Petrovna at the time. I wanted
to have a service sung for him, but I was ashamed.''

``You should go to a doctor.''

``I know I am not well, without your telling me, though I don't know
what's wrong; I believe I am five times as strong as you are. I didn't
ask you whether you believe that ghosts are seen, but whether you
believe that they exist.''

``No, I won't believe it!'' Raskolnikov cried, with positive anger.

``What do people generally say?'' muttered Svidrigaïlov, as though
speaking to himself, looking aside and bowing his head. ``They say, `You
are ill, so what appears to you is only unreal fantasy.' But that's not
strictly logical. I agree that ghosts only appear to the sick, but that
only proves that they are unable to appear except to the sick, not that
they don't exist.''

``Nothing of the sort,'' Raskolnikov insisted irritably.

``No? You don't think so?'' Svidrigaïlov went on, looking at him
deliberately. ``But what do you say to this argument (help me with it):
ghosts are, as it were, shreds and fragments of other worlds, the
beginning of them. A man in health has, of course, no reason to see
them, because he is above all a man of this earth and is bound for the
sake of completeness and order to live only in this life. But as soon as
one is ill, as soon as the normal earthly order of the organism is
broken, one begins to realise the possibility of another world; and the
more seriously ill one is, the closer becomes one's contact with that
other world, so that as soon as the man dies he steps straight into that
world. I thought of that long ago. If you believe in a future life, you
could believe in that, too.''

``I don't believe in a future life,'' said Raskolnikov.

Svidrigaïlov sat lost in thought.

``And what if there are only spiders there, or something of that sort,''
he said suddenly.

``He is a madman,'' thought Raskolnikov.

``We always imagine eternity as something beyond our conception,
something vast, vast! But why must it be vast? Instead of all that, what
if it's one little room, like a bath house in the country, black and
grimy and spiders in every corner, and that's all eternity is? I
sometimes fancy it like that.''

``Can it be you can imagine nothing juster and more comforting than
that?'' Raskolnikov cried, with a feeling of anguish.

``Juster? And how can we tell, perhaps that is just, and do you know
it's what I would certainly have made it,'' answered Svidrigaïlov, with
a vague smile.

This horrible answer sent a cold chill through Raskolnikov. Svidrigaïlov
raised his head, looked at him, and suddenly began laughing.

``Only think,'' he cried, ``half an hour ago we had never seen each
other, we regarded each other as enemies; there is a matter unsettled
between us; we've thrown it aside, and away we've gone into the
abstract! Wasn't I right in saying that we were birds of a feather?''

``Kindly allow me,'' Raskolnikov went on irritably, ``to ask you to
explain why you have honoured me with your visit\ldots{} and\ldots{} and
I am in a hurry, I have no time to waste. I want to go out.''

``By all means, by all means. Your sister, Avdotya Romanovna, is going
to be married to Mr. Luzhin, Pyotr Petrovitch?''

``Can you refrain from any question about my sister and from mentioning
her name? I can't understand how you dare utter her name in my presence,
if you really are Svidrigaïlov.''

``Why, but I've come here to speak about her; how can I avoid mentioning
her?''

``Very good, speak, but make haste.''

``I am sure that you must have formed your own opinion of this Mr.
Luzhin, who is a connection of mine through my wife, if you have only
seen him for half an hour, or heard any facts about him. He is no match
for Avdotya Romanovna. I believe Avdotya Romanovna is sacrificing
herself generously and imprudently for the sake of\ldots{} for the sake
of her family. I fancied from all I had heard of you that you would be
very glad if the match could be broken off without the sacrifice of
worldly advantages. Now I know you personally, I am convinced of it.''

``All this is very naïve\ldots{} excuse me, I should have said impudent
on your part,'' said Raskolnikov.

``You mean to say that I am seeking my own ends. Don't be uneasy, Rodion
Romanovitch, if I were working for my own advantage, I would not have
spoken out so directly. I am not quite a fool. I will confess something
psychologically curious about that: just now, defending my love for
Avdotya Romanovna, I said I was myself the victim. Well, let me tell you
that I've no feeling of love now, not the slightest, so that I wonder
myself indeed, for I really did feel something\ldots{}''

``Through idleness and depravity,'' Raskolnikov put in.

``I certainly am idle and depraved, but your sister has such qualities
that even I could not help being impressed by them. But that's all
nonsense, as I see myself now.''

``Have you seen that long?''

``I began to be aware of it before, but was only perfectly sure of it
the day before yesterday, almost at the moment I arrived in Petersburg.
I still fancied in Moscow, though, that I was coming to try to get
Avdotya Romanovna's hand and to cut out Mr. Luzhin.''

``Excuse me for interrupting you; kindly be brief, and come to the
object of your visit. I am in a hurry, I want to go out\ldots{}''

``With the greatest pleasure. On arriving here and determining on a
certain\ldots{} journey, I should like to make some necessary
preliminary arrangements. I left my children with an aunt; they are well
provided for; and they have no need of me personally. And a nice father
I should make, too! I have taken nothing but what Marfa Petrovna gave me
a year ago. That's enough for me. Excuse me, I am just coming to the
point. Before the journey which may come off, I want to settle Mr.
Luzhin, too. It's not that I detest him so much, but it was through him
I quarrelled with Marfa Petrovna when I learned that she had dished up
this marriage. I want now to see Avdotya Romanovna through your
mediation, and if you like in your presence, to explain to her that in
the first place she will never gain anything but harm from Mr. Luzhin.
Then, begging her pardon for all past unpleasantness, to make her a
present of ten thousand roubles and so assist the rupture with Mr.
Luzhin, a rupture to which I believe she is herself not disinclined, if
she could see the way to it.''

``You are certainly mad,'' cried Raskolnikov not so much angered as
astonished. ``How dare you talk like that!''

``I knew you would scream at me; but in the first place, though I am not
rich, this ten thousand roubles is perfectly free; I have absolutely no
need for it. If Avdotya Romanovna does not accept it, I shall waste it
in some more foolish way. That's the first thing. Secondly, my
conscience is perfectly easy; I make the offer with no ulterior motive.
You may not believe it, but in the end Avdotya Romanovna and you will
know. The point is, that I did actually cause your sister, whom I
greatly respect, some trouble and unpleasantness, and so, sincerely
regretting it, I want---not to compensate, not to repay her for the
unpleasantness, but simply to do something to her advantage, to show
that I am not, after all, privileged to do nothing but harm. If there
were a millionth fraction of self-interest in my offer, I should not
have made it so openly; and I should not have offered her ten thousand
only, when five weeks ago I offered her more, Besides, I may, perhaps,
very soon marry a young lady, and that alone ought to prevent suspicion
of any design on Avdotya Romanovna. In conclusion, let me say that in
marrying Mr. Luzhin, she is taking money just the same, only from
another man. Don't be angry, Rodion Romanovitch, think it over coolly
and quietly.''

Svidrigaïlov himself was exceedingly cool and quiet as he was saying
this.

``I beg you to say no more,'' said Raskolnikov. ``In any case this is
unpardonable impertinence.''

``Not in the least. Then a man may do nothing but harm to his neighbour
in this world, and is prevented from doing the tiniest bit of good by
trivial conventional formalities. That's absurd. If I died, for
instance, and left that sum to your sister in my will, surely she
wouldn't refuse it?''

``Very likely she would.''

``Oh, no, indeed. However, if you refuse it, so be it, though ten
thousand roubles is a capital thing to have on occasion. In any case I
beg you to repeat what I have said to Avdotya Romanovna.''

``No, I won't.''

``In that case, Rodion Romanovitch, I shall be obliged to try and see
her myself and worry her by doing so.''

``And if I do tell her, will you not try to see her?''

``I don't know really what to say. I should like very much to see her
once more.''

``Don't hope for it.''

``I'm sorry. But you don't know me. Perhaps we may become better
friends.''

``You think we may become friends?''

``And why not?'' Svidrigaïlov said, smiling. He stood up and took his
hat. ``I didn't quite intend to disturb you and I came here without
reckoning on it\ldots{} though I was very much struck by your face this
morning.''

``Where did you see me this morning?'' Raskolnikov asked uneasily.

``I saw you by chance\ldots{}. I kept fancying there is something about
you like me\ldots{}. But don't be uneasy. I am not intrusive; I used to
get on all right with card-sharpers, and I never bored Prince Svirbey, a
great personage who is a distant relation of mine, and I could write
about Raphael's \emph{Madonna} in Madam Prilukov's album, and I never
left Marfa Petrovna's side for seven years, and I used to stay the night
at Viazemsky's house in the Hay Market in the old days, and I may go up
in a balloon with Berg, perhaps.''

``Oh, all right. Are you starting soon on your travels, may I ask?''

``What travels?''

``Why, on that `journey'; you spoke of it yourself.''

``A journey? Oh, yes. I did speak of a journey. Well, that's a wide
subject\ldots{}. if only you knew what you are asking,'' he added, and
gave a sudden, loud, short laugh. ``Perhaps I'll get married instead of
the journey. They're making a match for me.''

``Here?''

``Yes.''

``How have you had time for that?''

``But I am very anxious to see Avdotya Romanovna once. I earnestly beg
it. Well, good-bye for the present. Oh, yes. I have forgotten something.
Tell your sister, Rodion Romanovitch, that Marfa Petrovna remembered her
in her will and left her three thousand roubles. That's absolutely
certain. Marfa Petrovna arranged it a week before her death, and it was
done in my presence. Avdotya Romanovna will be able to receive the money
in two or three weeks.''

``Are you telling the truth?''

``Yes, tell her. Well, your servant. I am staying very near you.''

As he went out, Svidrigaïlov ran up against Razumihin in the doorway.

\protect\hypertarget{link2HCH0022}{}{}

\mychapter{2}{Chapter II}\label{chapter-ii-3}

It was nearly eight o'clock. The two young men hurried to Bakaleyev's,
to arrive before Luzhin.

``Why, who was that?'' asked Razumihin, as soon as they were in the
street.

``It was Svidrigaïlov, that landowner in whose house my sister was
insulted when she was their governess. Through his persecuting her with
his attentions, she was turned out by his wife, Marfa Petrovna. This
Marfa Petrovna begged Dounia's forgiveness afterwards, and she's just
died suddenly. It was of her we were talking this morning. I don't know
why I'm afraid of that man. He came here at once after his wife's
funeral. He is very strange, and is determined on doing
something\ldots{}. We must guard Dounia from him\ldots{} that's what I
wanted to tell you, do you hear?''

``Guard her! What can he do to harm Avdotya Romanovna? Thank you, Rodya,
for speaking to me like that\ldots{}. We will, we will guard her. Where
does he live?''

``I don't know.''

``Why didn't you ask? What a pity! I'll find out, though.''

``Did you see him?'' asked Raskolnikov after a pause.

``Yes, I noticed him, I noticed him well.''

``You did really see him? You saw him clearly?'' Raskolnikov insisted.

``Yes, I remember him perfectly, I should know him in a thousand; I have
a good memory for faces.''

They were silent again.

``Hm!\ldots{} that's all right,'' muttered Raskolnikov. ``Do you know, I
fancied\ldots{} I keep thinking that it may have been an
hallucination.''

``What do you mean? I don't understand you.''

``Well, you all say,'' Raskolnikov went on, twisting his mouth into a
smile, ``that I am mad. I thought just now that perhaps I really am mad,
and have only seen a phantom.''

``What do you mean?''

``Why, who can tell? Perhaps I am really mad, and perhaps everything
that happened all these days may be only imagination.''

``Ach, Rodya, you have been upset again!\ldots{} But what did he say,
what did he come for?''

Raskolnikov did not answer. Razumihin thought a minute.

``Now let me tell you my story,'' he began, ``I came to you, you were
asleep. Then we had dinner and then I went to Porfiry's, Zametov was
still with him. I tried to begin, but it was no use. I couldn't speak in
the right way. They don't seem to understand and can't understand, but
are not a bit ashamed. I drew Porfiry to the window, and began talking
to him, but it was still no use. He looked away and I looked away. At
last I shook my fist in his ugly face, and told him as a cousin I'd
brain him. He merely looked at me, I cursed and came away. That was all.
It was very stupid. To Zametov I didn't say a word. But, you see, I
thought I'd made a mess of it, but as I went downstairs a brilliant idea
struck me: why should we trouble? Of course if you were in any danger or
anything, but why need you care? You needn't care a hang for them. We
shall have a laugh at them afterwards, and if I were in your place I'd
mystify them more than ever. How ashamed they'll be afterwards! Hang
them! We can thrash them afterwards, but let's laugh at them now!''

``To be sure,'' answered Raskolnikov. ``But what will you say
to-morrow?'' he thought to himself. Strange to say, till that moment it
had never occurred to him to wonder what Razumihin would think when he
knew. As he thought it, Raskolnikov looked at him. Razumihin's account
of his visit to Porfiry had very little interest for him, so much had
come and gone since then.

In the corridor they came upon Luzhin; he had arrived punctually at
eight, and was looking for the number, so that all three went in
together without greeting or looking at one another. The young men
walked in first, while Pyotr Petrovitch, for good manners, lingered a
little in the passage, taking off his coat. Pulcheria Alexandrovna came
forward at once to greet him in the doorway, Dounia was welcoming her
brother. Pyotr Petrovitch walked in and quite amiably, though with
redoubled dignity, bowed to the ladies. He looked, however, as though he
were a little put out and could not yet recover himself. Pulcheria
Alexandrovna, who seemed also a little embarrassed, hastened to make
them all sit down at the round table where a samovar was boiling. Dounia
and Luzhin were facing one another on opposite sides of the table.
Razumihin and Raskolnikov were facing Pulcheria Alexandrovna, Razumihin
was next to Luzhin and Raskolnikov was beside his sister.

A moment's silence followed. Pyotr Petrovitch deliberately drew out a
cambric handkerchief reeking of scent and blew his nose with an air of a
benevolent man who felt himself slighted, and was firmly resolved to
insist on an explanation. In the passage the idea had occurred to him to
keep on his overcoat and walk away, and so give the two ladies a sharp
and emphatic lesson and make them feel the gravity of the position. But
he could not bring himself to do this. Besides, he could not endure
uncertainty, and he wanted an explanation: if his request had been so
openly disobeyed, there was something behind it, and in that case it was
better to find it out beforehand; it rested with him to punish them and
there would always be time for that.

``I trust you had a favourable journey,'' he inquired officially of
Pulcheria Alexandrovna.

``Oh, very, Pyotr Petrovitch.''

``I am gratified to hear it. And Avdotya Romanovna is not over-fatigued
either?''

``I am young and strong, I don't get tired, but it was a great strain
for mother,'' answered Dounia.

``That's unavoidable! our national railways are of terrible length.
`Mother Russia,' as they say, is a vast country\ldots{}. In spite of all
my desire to do so, I was unable to meet you yesterday. But I trust all
passed off without inconvenience?''

``Oh, no, Pyotr Petrovitch, it was all terribly disheartening,''
Pulcheria Alexandrovna hastened to declare with peculiar intonation,
``and if Dmitri Prokofitch had not been sent us, I really believe by God
Himself, we should have been utterly lost. Here, he is! Dmitri
Prokofitch Razumihin,'' she added, introducing him to Luzhin.

``I had the pleasure\ldots{} yesterday,'' muttered Pyotr Petrovitch with
a hostile glance sidelong at Razumihin; then he scowled and was silent.

Pyotr Petrovitch belonged to that class of persons, on the surface very
polite in society, who make a great point of punctiliousness, but who,
directly they are crossed in anything, are completely disconcerted, and
become more like sacks of flour than elegant and lively men of society.
Again all was silent; Raskolnikov was obstinately mute, Avdotya
Romanovna was unwilling to open the conversation too soon. Razumihin had
nothing to say, so Pulcheria Alexandrovna was anxious again.

``Marfa Petrovna is dead, have you heard?'' she began having recourse to
her leading item of conversation.

``To be sure, I heard so. I was immediately informed, and I have come to
make you acquainted with the fact that Arkady Ivanovitch Svidrigaïlov
set off in haste for Petersburg immediately after his wife's funeral. So
at least I have excellent authority for believing.''

``To Petersburg? here?'' Dounia asked in alarm and looked at her mother.

``Yes, indeed, and doubtless not without some design, having in view the
rapidity of his departure, and all the circumstances preceding it.''

``Good heavens! won't he leave Dounia in peace even here?'' cried
Pulcheria Alexandrovna.

``I imagine that neither you nor Avdotya Romanovna have any grounds for
uneasiness, unless, of course, you are yourselves desirous of getting
into communication with him. For my part I am on my guard, and am now
discovering where he is lodging.''

``Oh, Pyotr Petrovitch, you would not believe what a fright you have
given me,'' Pulcheria Alexandrovna went on: ``I've only seen him twice,
but I thought him terrible, terrible! I am convinced that he was the
cause of Marfa Petrovna's death.''

``It's impossible to be certain about that. I have precise information.
I do not dispute that he may have contributed to accelerate the course
of events by the moral influence, so to say, of the affront; but as to
the general conduct and moral characteristics of that personage, I am in
agreement with you. I do not know whether he is well off now, and
precisely what Marfa Petrovna left him; this will be known to me within
a very short period; but no doubt here in Petersburg, if he has any
pecuniary resources, he will relapse at once into his old ways. He is
the most depraved, and abjectly vicious specimen of that class of men. I
have considerable reason to believe that Marfa Petrovna, who was so
unfortunate as to fall in love with him and to pay his debts eight years
ago, was of service to him also in another way. Solely by her exertions
and sacrifices, a criminal charge, involving an element of fantastic and
homicidal brutality for which he might well have been sentenced to
Siberia, was hushed up. That's the sort of man he is, if you care to
know.''

``Good heavens!'' cried Pulcheria Alexandrovna. Raskolnikov listened
attentively.

``Are you speaking the truth when you say that you have good evidence of
this?'' Dounia asked sternly and emphatically.

``I only repeat what I was told in secret by Marfa Petrovna. I must
observe that from the legal point of view the case was far from clear.
There was, and I believe still is, living here a woman called Resslich,
a foreigner, who lent small sums of money at interest, and did other
commissions, and with this woman Svidrigaïlov had for a long while close
and mysterious relations. She had a relation, a niece I believe, living
with her, a deaf and dumb girl of fifteen, or perhaps not more than
fourteen. Resslich hated this girl, and grudged her every crust; she
used to beat her mercilessly. One day the girl was found hanging in the
garret. At the inquest the verdict was suicide. After the usual
proceedings the matter ended, but, later on, information was given that
the child had been\ldots{} cruelly outraged by Svidrigaïlov. It is true,
this was not clearly established, the information was given by another
German woman of loose character whose word could not be trusted; no
statement was actually made to the police, thanks to Marfa Petrovna's
money and exertions; it did not get beyond gossip. And yet the story is
a very significant one. You heard, no doubt, Avdotya Romanovna, when you
were with them the story of the servant Philip who died of ill treatment
he received six years ago, before the abolition of serfdom.''

``I heard, on the contrary, that this Philip hanged himself.''

``Quite so, but what drove him, or rather perhaps disposed him, to
suicide was the systematic persecution and severity of Mr.
Svidrigaïlov.''

``I don't know that,'' answered Dounia, dryly. ``I only heard a queer
story that Philip was a sort of hypochondriac, a sort of domestic
philosopher, the servants used to say, `he read himself silly,' and that
he hanged himself partly on account of Mr. Svidrigaïlov's mockery of him
and not his blows. When I was there he behaved well to the servants, and
they were actually fond of him, though they certainly did blame him for
Philip's death.''

``I perceive, Avdotya Romanovna, that you seem disposed to undertake his
defence all of a sudden,'' Luzhin observed, twisting his lips into an
ambiguous smile, ``there's no doubt that he is an astute man, and
insinuating where ladies are concerned, of which Marfa Petrovna, who has
died so strangely, is a terrible instance. My only desire has been to be
of service to you and your mother with my advice, in view of the renewed
efforts which may certainly be anticipated from him. For my part it's my
firm conviction, that he will end in a debtor's prison again. Marfa
Petrovna had not the slightest intention of settling anything
substantial on him, having regard for his children's interests, and, if
she left him anything, it would only be the merest sufficiency,
something insignificant and ephemeral, which would not last a year for a
man of his habits.''

``Pyotr Petrovitch, I beg you,'' said Dounia, ``say no more of Mr.
Svidrigaïlov. It makes me miserable.''

``He has just been to see me,'' said Raskolnikov, breaking his silence
for the first time.

There were exclamations from all, and they all turned to him. Even Pyotr
Petrovitch was roused.

``An hour and a half ago, he came in when I was asleep, waked me, and
introduced himself,'' Raskolnikov continued. ``He was fairly cheerful
and at ease, and quite hopes that we shall become friends. He is
particularly anxious, by the way, Dounia, for an interview with you, at
which he asked me to assist. He has a proposition to make to you, and he
told me about it. He told me, too, that a week before her death Marfa
Petrovna left you three thousand roubles in her will, Dounia, and that
you can receive the money very shortly.''

``Thank God!'' cried Pulcheria Alexandrovna, crossing herself. ``Pray
for her soul, Dounia!''

``It's a fact!'' broke from Luzhin.

``Tell us, what more?'' Dounia urged Raskolnikov.

``Then he said that he wasn't rich and all the estate was left to his
children who are now with an aunt, then that he was staying somewhere
not far from me, but where, I don't know, I didn't ask\ldots{}.''

``But what, what does he want to propose to Dounia?'' cried Pulcheria
Alexandrovna in a fright. ``Did he tell you?''

``Yes.''

``What was it?''

``I'll tell you afterwards.''

Raskolnikov ceased speaking and turned his attention to his tea.

Pyotr Petrovitch looked at his watch.

``I am compelled to keep a business engagement, and so I shall not be in
your way,'' he added with an air of some pique and he began getting up.

``Don't go, Pyotr Petrovitch,'' said Dounia, ``you intended to spend the
evening. Besides, you wrote yourself that you wanted to have an
explanation with mother.''

``Precisely so, Avdotya Romanovna,'' Pyotr Petrovitch answered
impressively, sitting down again, but still holding his hat. ``I
certainly desired an explanation with you and your honoured mother upon
a very important point indeed. But as your brother cannot speak openly
in my presence of some proposals of Mr. Svidrigaïlov, I, too, do not
desire and am not able to speak openly\ldots{} in the presence of
others\ldots{} of certain matters of the greatest gravity. Moreover, my
most weighty and urgent request has been disregarded\ldots{}.''

Assuming an aggrieved air, Luzhin relapsed into dignified silence.

``Your request that my brother should not be present at our meeting was
disregarded solely at my insistance,'' said Dounia. ``You wrote that you
had been insulted by my brother; I think that this must be explained at
once, and you must be reconciled. And if Rodya really has insulted you,
then he \emph{should} and \emph{will} apologise.''

Pyotr Petrovitch took a stronger line.

``There are insults, Avdotya Romanovna, which no goodwill can make us
forget. There is a line in everything which it is dangerous to overstep;
and when it has been overstepped, there is no return.''

``That wasn't what I was speaking of exactly, Pyotr Petrovitch,'' Dounia
interrupted with some impatience. ``Please understand that our whole
future depends now on whether all this is explained and set right as
soon as possible. I tell you frankly at the start that I cannot look at
it in any other light, and if you have the least regard for me, all this
business must be ended to-day, however hard that may be. I repeat that
if my brother is to blame he will ask your forgiveness.''

``I am surprised at your putting the question like that,'' said Luzhin,
getting more and more irritated. ``Esteeming, and so to say, adoring
you, I may at the same time, very well indeed, be able to dislike some
member of your family. Though I lay claim to the happiness of your hand,
I cannot accept duties incompatible with\ldots{}''

``Ah, don't be so ready to take offence, Pyotr Petrovitch,'' Dounia
interrupted with feeling, ``and be the sensible and generous man I have
always considered, and wish to consider, you to be. I've given you a
great promise, I am your betrothed. Trust me in this matter and, believe
me, I shall be capable of judging impartially. My assuming the part of
judge is as much a surprise for my brother as for you. When I insisted
on his coming to our interview to-day after your letter, I told him
nothing of what I meant to do. Understand that, if you are not
reconciled, I must choose between you---it must be either you or he.
That is how the question rests on your side and on his. I don't want to
be mistaken in my choice, and I must not be. For your sake I must break
off with my brother, for my brother's sake I must break off with you. I
can find out for certain now whether he is a brother to me, and I want
to know it; and of you, whether I am dear to you, whether you esteem me,
whether you are the husband for me.''

``Avdotya Romanovna,'' Luzhin declared huffily, ``your words are of too
much consequence to me; I will say more, they are offensive in view of
the position I have the honour to occupy in relation to you. To say
nothing of your strange and offensive setting me on a level with an
impertinent boy, you admit the possibility of breaking your promise to
me. You say `you or he,' showing thereby of how little consequence I am
in your eyes\ldots{} I cannot let this pass considering the relationship
and\ldots{} the obligations existing between us.''

``What!'' cried Dounia, flushing. ``I set your interest beside all that
has hitherto been most precious in my life, what has made up the
\emph{whole} of my life, and here you are offended at my making too
\emph{little} account of you.''

Raskolnikov smiled sarcastically, Razumihin fidgeted, but Pyotr
Petrovitch did not accept the reproof; on the contrary, at every word he
became more persistent and irritable, as though he relished it.

``Love for the future partner of your life, for your husband, ought to
outweigh your love for your brother,'' he pronounced sententiously,
``and in any case I cannot be put on the same level\ldots{}. Although I
said so emphatically that I would not speak openly in your brother's
presence, nevertheless, I intend now to ask your honoured mother for a
necessary explanation on a point of great importance closely affecting
my dignity. Your son,'' he turned to Pulcheria Alexandrovna, ``yesterday
in the presence of Mr. Razsudkin (or\ldots{} I think that's it? excuse
me I have forgotten your surname,'' he bowed politely to Razumihin)
``insulted me by misrepresenting the idea I expressed to you in a
private conversation, drinking coffee, that is, that marriage with a
poor girl who has had experience of trouble is more advantageous from
the conjugal point of view than with one who has lived in luxury, since
it is more profitable for the moral character. Your son intentionally
exaggerated the significance of my words and made them ridiculous,
accusing me of malicious intentions, and, as far as I could see, relied
upon your correspondence with him. I shall consider myself happy,
Pulcheria Alexandrovna, if it is possible for you to convince me of an
opposite conclusion, and thereby considerately reassure me. Kindly let
me know in what terms precisely you repeated my words in your letter to
Rodion Romanovitch.''

``I don't remember,'' faltered Pulcheria Alexandrovna. ``I repeated them
as I understood them. I don't know how Rodya repeated them to you,
perhaps he exaggerated.''

``He could not have exaggerated them, except at your instigation.''

``Pyotr Petrovitch,'' Pulcheria Alexandrovna declared with dignity,
``the proof that Dounia and I did not take your words in a very bad
sense is the fact that we are here.''

``Good, mother,'' said Dounia approvingly.

``Then this is my fault again,'' said Luzhin, aggrieved.

``Well, Pyotr Petrovitch, you keep blaming Rodion, but you yourself have
just written what was false about him,'' Pulcheria Alexandrovna added,
gaining courage.

``I don't remember writing anything false.''

``You wrote,'' Raskolnikov said sharply, not turning to Luzhin, ``that I
gave money yesterday not to the widow of the man who was killed, as was
the fact, but to his daughter (whom I had never seen till yesterday).
You wrote this to make dissension between me and my family, and for that
object added coarse expressions about the conduct of a girl whom you
don't know. All that is mean slander.''

``Excuse me, sir,'' said Luzhin, quivering with fury. ``I enlarged upon
your qualities and conduct in my letter solely in response to your
sister's and mother's inquiries, how I found you, and what impression
you made on me. As for what you've alluded to in my letter, be so good
as to point out one word of falsehood, show, that is, that you didn't
throw away your money, and that there are not worthless persons in that
family, however unfortunate.''

``To my thinking, you, with all your virtues, are not worth the little
finger of that unfortunate girl at whom you throw stones.''

``Would you go so far then as to let her associate with your mother and
sister?''

``I have done so already, if you care to know. I made her sit down
to-day with mother and Dounia.''

``Rodya!'' cried Pulcheria Alexandrovna. Dounia crimsoned, Razumihin
knitted his brows. Luzhin smiled with lofty sarcasm.

``You may see for yourself, Avdotya Romanovna,'' he said, ``whether it
is possible for us to agree. I hope now that this question is at an end,
once and for all. I will withdraw, that I may not hinder the pleasures
of family intimacy, and the discussion of secrets.'' He got up from his
chair and took his hat. ``But in withdrawing, I venture to request that
for the future I may be spared similar meetings, and, so to say,
compromises. I appeal particularly to you, honoured Pulcheria
Alexandrovna, on this subject, the more as my letter was addressed to
you and to no one else.''

Pulcheria Alexandrovna was a little offended.

``You seem to think we are completely under your authority, Pyotr
Petrovitch. Dounia has told you the reason your desire was disregarded,
she had the best intentions. And indeed you write as though you were
laying commands upon me. Are we to consider every desire of yours as a
command? Let me tell you on the contrary that you ought to show
particular delicacy and consideration for us now, because we have thrown
up everything, and have come here relying on you, and so we are in any
case in a sense in your hands.''

``That is not quite true, Pulcheria Alexandrovna, especially at the
present moment, when the news has come of Marfa Petrovna's legacy, which
seems indeed very apropos, judging from the new tone you take to me,''
he added sarcastically.

``Judging from that remark, we may certainly presume that you were
reckoning on our helplessness,'' Dounia observed irritably.

``But now in any case I cannot reckon on it, and I particularly desire
not to hinder your discussion of the secret proposals of Arkady
Ivanovitch Svidrigaïlov, which he has entrusted to your brother and
which have, I perceive, a great and possibly a very agreeable interest
for you.''

``Good heavens!'' cried Pulcheria Alexandrovna.

Razumihin could not sit still on his chair.

``Aren't you ashamed now, sister?'' asked Raskolnikov.

``I am ashamed, Rodya,'' said Dounia. ``Pyotr Petrovitch, go away,'' she
turned to him, white with anger.

Pyotr Petrovitch had apparently not at all expected such a conclusion.
He had too much confidence in himself, in his power and in the
helplessness of his victims. He could not believe it even now. He turned
pale, and his lips quivered.

``Avdotya Romanovna, if I go out of this door now, after such a
dismissal, then, you may reckon on it, I will never come back. Consider
what you are doing. My word is not to be shaken.''

``What insolence!'' cried Dounia, springing up from her seat. ``I don't
want you to come back again.''

``What! So that's how it stands!'' cried Luzhin, utterly unable to the
last moment to believe in the rupture and so completely thrown out of
his reckoning now. ``So that's how it stands! But do you know, Avdotya
Romanovna, that I might protest?''

``What right have you to speak to her like that?'' Pulcheria
Alexandrovna intervened hotly. ``And what can you protest about? What
rights have you? Am I to give my Dounia to a man like you? Go away,
leave us altogether! We are to blame for having agreed to a wrong
action, and I above all\ldots{}.''

``But you have bound me, Pulcheria Alexandrovna,'' Luzhin stormed in a
frenzy, ``by your promise, and now you deny it and\ldots{}
besides\ldots{} I have been led on account of that into
expenses\ldots{}.''

This last complaint was so characteristic of Pyotr Petrovitch, that
Raskolnikov, pale with anger and with the effort of restraining it,
could not help breaking into laughter. But Pulcheria Alexandrovna was
furious.

``Expenses? What expenses? Are you speaking of our trunk? But the
conductor brought it for nothing for you. Mercy on us, we have bound
you! What are you thinking about, Pyotr Petrovitch, it was you bound us,
hand and foot, not we!''

``Enough, mother, no more please,'' Avdotya Romanovna implored. ``Pyotr
Petrovitch, do be kind and go!''

``I am going, but one last word,'' he said, quite unable to control
himself. ``Your mamma seems to have entirely forgotten that I made up my
mind to take you, so to speak, after the gossip of the town had spread
all over the district in regard to your reputation. Disregarding public
opinion for your sake and reinstating your reputation, I certainly might
very well reckon on a fitting return, and might indeed look for
gratitude on your part. And my eyes have only now been opened! I see
myself that I may have acted very, very recklessly in disregarding the
universal verdict\ldots{}.''

``Does the fellow want his head smashed?'' cried Razumihin, jumping up.

``You are a mean and spiteful man!'' cried Dounia.

``Not a word! Not a movement!'' cried Raskolnikov, holding Razumihin
back; then going close up to Luzhin, ``Kindly leave the room!'' he said
quietly and distinctly, ``and not a word more or\ldots{}''

Pyotr Petrovitch gazed at him for some seconds with a pale face that
worked with anger, then he turned, went out, and rarely has any man
carried away in his heart such vindictive hatred as he felt against
Raskolnikov. Him, and him alone, he blamed for everything. It is
noteworthy that as he went downstairs he still imagined that his case
was perhaps not utterly lost, and that, so far as the ladies were
concerned, all might ``very well indeed'' be set right again.

\protect\hypertarget{link2HCH0023}{}{}

\mychapter{3}{Chapter III}\label{chapter-iii-3}

The fact was that up to the last moment he had never expected such an
ending; he had been overbearing to the last degree, never dreaming that
two destitute and defenceless women could escape from his control. This
conviction was strengthened by his vanity and conceit, a conceit to the
point of fatuity. Pyotr Petrovitch, who had made his way up from
insignificance, was morbidly given to self-admiration, had the highest
opinion of his intelligence and capacities, and sometimes even gloated
in solitude over his image in the glass. But what he loved and valued
above all was the money he had amassed by his labour, and by all sorts
of devices: that money made him the equal of all who had been his
superiors.

When he had bitterly reminded Dounia that he had decided to take her in
spite of evil report, Pyotr Petrovitch had spoken with perfect sincerity
and had, indeed, felt genuinely indignant at such ``black ingratitude.''
And yet, when he made Dounia his offer, he was fully aware of the
groundlessness of all the gossip. The story had been everywhere
contradicted by Marfa Petrovna, and was by then disbelieved by all the
townspeople, who were warm in Dounia'a defence. And he would not have
denied that he knew all that at the time. Yet he still thought highly of
his own resolution in lifting Dounia to his level and regarded it as
something heroic. In speaking of it to Dounia, he had let out the secret
feeling he cherished and admired, and he could not understand that
others should fail to admire it too. He had called on Raskolnikov with
the feelings of a benefactor who is about to reap the fruits of his good
deeds and to hear agreeable flattery. And as he went downstairs now, he
considered himself most undeservedly injured and unrecognised.

Dounia was simply essential to him; to do without her was unthinkable.
For many years he had had voluptuous dreams of marriage, but he had gone
on waiting and amassing money. He brooded with relish, in profound
secret, over the image of a girl---virtuous, poor (she must be poor),
very young, very pretty, of good birth and education, very timid, one
who had suffered much, and was completely humbled before him, one who
would all her life look on him as her saviour, worship him, admire him
and only him. How many scenes, how many amorous episodes he had imagined
on this seductive and playful theme, when his work was over! And,
behold, the dream of so many years was all but realised; the beauty and
education of Avdotya Romanovna had impressed him; her helpless position
had been a great allurement; in her he had found even more than he
dreamed of. Here was a girl of pride, character, virtue, of education
and breeding superior to his own (he felt that), and this creature would
be slavishly grateful all her life for his heroic condescension, and
would humble herself in the dust before him, and he would have absolute,
unbounded power over her!\ldots{} Not long before, he had, too, after
long reflection and hesitation, made an important change in his career
and was now entering on a wider circle of business. With this change his
cherished dreams of rising into a higher class of society seemed likely
to be realised\ldots{}. He was, in fact, determined to try his fortune
in Petersburg. He knew that women could do a very great deal. The
fascination of a charming, virtuous, highly educated woman might make
his way easier, might do wonders in attracting people to him, throwing
an aureole round him, and now everything was in ruins! This sudden
horrible rupture affected him like a clap of thunder; it was like a
hideous joke, an absurdity. He had only been a tiny bit masterful, had
not even time to speak out, had simply made a joke, been carried
away---and it had ended so seriously. And, of course, too, he did love
Dounia in his own way; he already possessed her in his dreams---and all
at once! No! The next day, the very next day, it must all be set right,
smoothed over, settled. Above all he must crush that conceited milksop
who was the cause of it all. With a sick feeling he could not help
recalling Razumihin too, but, he soon reassured himself on that score;
as though a fellow like that could be put on a level with him! The man
he really dreaded in earnest was Svidrigaïlov\ldots{}. He had, in short,
a great deal to attend to\ldots{}.

\begin{center}\rule{0.5\linewidth}{\linethickness}\end{center}

``No, I, I am more to blame than anyone!'' said Dounia, kissing and
embracing her mother. ``I was tempted by his money, but on my honour,
brother, I had no idea he was such a base man. If I had seen through him
before, nothing would have tempted me! Don't blame me, brother!''

``God has delivered us! God has delivered us!'' Pulcheria Alexandrovna
muttered, but half consciously, as though scarcely able to realise what
had happened.

They were all relieved, and in five minutes they were laughing. Only now
and then Dounia turned white and frowned, remembering what had passed.
Pulcheria Alexandrovna was surprised to find that she, too, was glad:
she had only that morning thought rupture with Luzhin a terrible
misfortune. Razumihin was delighted. He did not yet dare to express his
joy fully, but he was in a fever of excitement as though a ton-weight
had fallen off his heart. Now he had the right to devote his life to
them, to serve them\ldots{}. Anything might happen now! But he felt
afraid to think of further possibilities and dared not let his
imagination range. But Raskolnikov sat still in the same place, almost
sullen and indifferent. Though he had been the most insistent on getting
rid of Luzhin, he seemed now the least concerned at what had happened.
Dounia could not help thinking that he was still angry with her, and
Pulcheria Alexandrovna watched him timidly.

``What did Svidrigaïlov say to you?'' said Dounia, approaching him.

``Yes, yes!'' cried Pulcheria Alexandrovna.

Raskolnikov raised his head.

``He wants to make you a present of ten thousand roubles and he desires
to see you once in my presence.''

``See her! On no account!'' cried Pulcheria Alexandrovna. ``And how dare
he offer her money!''

Then Raskolnikov repeated (rather dryly) his conversation with
Svidrigaïlov, omitting his account of the ghostly visitations of Marfa
Petrovna, wishing to avoid all unnecessary talk.

``What answer did you give him?'' asked Dounia.

``At first I said I would not take any message to you. Then he said that
he would do his utmost to obtain an interview with you without my help.
He assured me that his passion for you was a passing infatuation, now he
has no feeling for you. He doesn't want you to marry Luzhin\ldots{}. His
talk was altogether rather muddled.''

``How do you explain him to yourself, Rodya? How did he strike you?''

``I must confess I don't quite understand him. He offers you ten
thousand, and yet says he is not well off. He says he is going away, and
in ten minutes he forgets he has said it. Then he says he is going to be
married and has already fixed on the girl\ldots{}. No doubt he has a
motive, and probably a bad one. But it's odd that he should be so clumsy
about it if he had any designs against you\ldots{}. Of course, I refused
this money on your account, once for all. Altogether, I thought him very
strange\ldots{}. One might almost think he was mad. But I may be
mistaken; that may only be the part he assumes. The death of Marfa
Petrovna seems to have made a great impression on him.''

``God rest her soul,'' exclaimed Pulcheria Alexandrovna. ``I shall
always, always pray for her! Where should we be now, Dounia, without
this three thousand! It's as though it had fallen from heaven! Why,
Rodya, this morning we had only three roubles in our pocket and Dounia
and I were just planning to pawn her watch, so as to avoid borrowing
from that man until he offered help.''

Dounia seemed strangely impressed by Svidrigaïlov's offer. She still
stood meditating.

``He has got some terrible plan,'' she said in a half whisper to
herself, almost shuddering.

Raskolnikov noticed this disproportionate terror.

``I fancy I shall have to see him more than once again,'' he said to
Dounia.

``We will watch him! I will track him out!'' cried Razumihin,
vigorously. ``I won't lose sight of him. Rodya has given me leave. He
said to me himself just now. `Take care of my sister.' Will you give me
leave, too, Avdotya Romanovna?''

Dounia smiled and held out her hand, but the look of anxiety did not
leave her face. Pulcheria Alexandrovna gazed at her timidly, but the
three thousand roubles had obviously a soothing effect on her.

A quarter of an hour later, they were all engaged in a lively
conversation. Even Raskolnikov listened attentively for some time,
though he did not talk. Razumihin was the speaker.

``And why, why should you go away?'' he flowed on ecstatically. ``And
what are you to do in a little town? The great thing is, you are all
here together and you need one another---you do need one another,
believe me. For a time, anyway\ldots{}. Take me into partnership, and I
assure you we'll plan a capital enterprise. Listen! I'll explain it all
in detail to you, the whole project! It all flashed into my head this
morning, before anything had happened\ldots{} I tell you what; I have an
uncle, I must introduce him to you (a most accommodating and respectable
old man). This uncle has got a capital of a thousand roubles, and he
lives on his pension and has no need of that money. For the last two
years he has been bothering me to borrow it from him and pay him six per
cent. interest. I know what that means; he simply wants to help me. Last
year I had no need of it, but this year I resolved to borrow it as soon
as he arrived. Then you lend me another thousand of your three and we
have enough for a start, so we'll go into partnership, and what are we
going to do?''

Then Razumihin began to unfold his project, and he explained at length
that almost all our publishers and booksellers know nothing at all of
what they are selling, and for that reason they are usually bad
publishers, and that any decent publications pay as a rule and give a
profit, sometimes a considerable one. Razumihin had, indeed, been
dreaming of setting up as a publisher. For the last two years he had
been working in publishers' offices, and knew three European languages
well, though he had told Raskolnikov six days before that he was
``schwach'' in German with an object of persuading him to take half his
translation and half the payment for it. He had told a lie then, and
Raskolnikov knew he was lying.

``Why, why should we let our chance slip when we have one of the chief
means of success---money of our own!'' cried Razumihin warmly. ``Of
course there will be a lot of work, but we will work, you, Avdotya
Romanovna, I, Rodion\ldots{}. You get a splendid profit on some books
nowadays! And the great point of the business is that we shall know just
what wants translating, and we shall be translating, publishing,
learning all at once. I can be of use because I have experience. For
nearly two years I've been scuttling about among the publishers, and now
I know every detail of their business. You need not be a saint to make
pots, believe me! And why, why should we let our chance slip! Why, I
know---and I kept the secret---two or three books which one might get a
hundred roubles simply for thinking of translating and publishing.
Indeed, and I would not take five hundred for the very idea of one of
them. And what do you think? If I were to tell a publisher, I dare say
he'd hesitate---they are such blockheads! And as for the business side,
printing, paper, selling, you trust to me, I know my way about. We'll
begin in a small way and go on to a large. In any case it will get us
our living and we shall get back our capital.''

Dounia's eyes shone.

``I like what you are saying, Dmitri Prokofitch!'' she said.

``I know nothing about it, of course,'' put in Pulcheria Alexandrovna,
``it may be a good idea, but again God knows. It's new and untried. Of
course, we must remain here at least for a time.'' She looked at Rodya.

``What do you think, brother?'' said Dounia.

``I think he's got a very good idea,'' he answered. ``Of course, it's
too soon to dream of a publishing firm, but we certainly might bring out
five or six books and be sure of success. I know of one book myself
which would be sure to go well. And as for his being able to manage it,
there's no doubt about that either. He knows the business\ldots{}. But
we can talk it over later\ldots{}.''

``Hurrah!'' cried Razumihin. ``Now, stay, there's a flat here in this
house, belonging to the same owner. It's a special flat apart, not
communicating with these lodgings. It's furnished, rent moderate, three
rooms. Suppose you take them to begin with. I'll pawn your watch
to-morrow and bring you the money, and everything can be arranged then.
You can all three live together, and Rodya will be with you. But where
are you off to, Rodya?''

``What, Rodya, you are going already?'' Pulcheria Alexandrovna asked in
dismay.

``At such a minute?'' cried Razumihin.

Dounia looked at her brother with incredulous wonder. He held his cap in
his hand, he was preparing to leave them.

``One would think you were burying me or saying good-bye for ever,'' he
said somewhat oddly. He attempted to smile, but it did not turn out a
smile. ``But who knows, perhaps it is the last time we shall see each
other\ldots{}'' he let slip accidentally. It was what he was thinking,
and it somehow was uttered aloud.

``What is the matter with you?'' cried his mother.

``Where are you going, Rodya?'' asked Dounia rather strangely.

``Oh, I'm quite obliged to\ldots{}'' he answered vaguely, as though
hesitating what he would say. But there was a look of sharp
determination in his white face.

``I meant to say\ldots{} as I was coming here\ldots{} I meant to tell
you, mother, and you, Dounia, that it would be better for us to part for
a time. I feel ill, I am not at peace\ldots{}. I will come afterwards, I
will come of myself\ldots{} when it's possible. I remember you and love
you\ldots{}. Leave me, leave me alone. I decided this even
before\ldots{} I'm absolutely resolved on it. Whatever may come to me,
whether I come to ruin or not, I want to be alone. Forget me altogether,
it's better. Don't inquire about me. When I can, I'll come of myself
or\ldots{} I'll send for you. Perhaps it will all come back, but now if
you love me, give me up\ldots{} else I shall begin to hate you, I feel
it\ldots{}. Good-bye!''

``Good God!'' cried Pulcheria Alexandrovna. Both his mother and his
sister were terribly alarmed. Razumihin was also.

``Rodya, Rodya, be reconciled with us! Let us be as before!'' cried his
poor mother.

He turned slowly to the door and slowly went out of the room. Dounia
overtook him.

``Brother, what are you doing to mother?'' she whispered, her eyes
flashing with indignation.

He looked dully at her.

``No matter, I shall come\ldots{}. I'm coming,'' he muttered in an
undertone, as though not fully conscious of what he was saying, and he
went out of the room.

``Wicked, heartless egoist!'' cried Dounia.

``He is insane, but not heartless. He is mad! Don't you see it? You're
heartless after that!'' Razumihin whispered in her ear, squeezing her
hand tightly. ``I shall be back directly,'' he shouted to the
horror-stricken mother, and he ran out of the room.

Raskolnikov was waiting for him at the end of the passage.

``I knew you would run after me,'' he said. ``Go back to them---be with
them\ldots{} be with them to-morrow and always\ldots{}. I\ldots{}
perhaps I shall come\ldots{} if I can. Good-bye.''

And without holding out his hand he walked away.

``But where are you going? What are you doing? What's the matter with
you? How can you go on like this?'' Razumihin muttered, at his wits'
end.

Raskolnikov stopped once more.

``Once for all, never ask me about anything. I have nothing to tell you.
Don't come to see me. Maybe I'll come here\ldots{}. Leave me, but
\emph{don't leave} them. Do you understand me?''

It was dark in the corridor, they were standing near the lamp. For a
minute they were looking at one another in silence. Razumihin remembered
that minute all his life. Raskolnikov's burning and intent eyes grew
more penetrating every moment, piercing into his soul, into his
consciousness. Suddenly Razumihin started. Something strange, as it
were, passed between them\ldots{}. Some idea, some hint, as it were,
slipped, something awful, hideous, and suddenly understood on both
sides\ldots{}. Razumihin turned pale.

``Do you understand now?'' said Raskolnikov, his face twitching
nervously. ``Go back, go to them,'' he said suddenly, and turning
quickly, he went out of the house.

I will not attempt to describe how Razumihin went back to the ladies,
how he soothed them, how he protested that Rodya needed rest in his
illness, protested that Rodya was sure to come, that he would come every
day, that he was very, very much upset, that he must not be irritated,
that he, Razumihin, would watch over him, would get him a doctor, the
best doctor, a consultation\ldots{}. In fact from that evening Razumihin
took his place with them as a son and a brother.

\protect\hypertarget{link2HCH0024}{}{}

\mychapter{4}{Chapter IV}\label{chapter-iv-3}

Raskolnikov went straight to the house on the canal bank where Sonia
lived. It was an old green house of three storeys. He found the porter
and obtained from him vague directions as to the whereabouts of
Kapernaumov, the tailor. Having found in the corner of the courtyard the
entrance to the dark and narrow staircase, he mounted to the second
floor and came out into a gallery that ran round the whole second storey
over the yard. While he was wandering in the darkness, uncertain where
to turn for Kapernaumov's door, a door opened three paces from him; he
mechanically took hold of it.

``Who is there?'' a woman's voice asked uneasily.

``It's I\ldots{} come to see you,'' answered Raskolnikov and he walked
into the tiny entry.

On a broken chair stood a candle in a battered copper candlestick.

``It's you! Good heavens!'' cried Sonia weakly, and she stood rooted to
the spot.

``Which is your room? This way?'' and Raskolnikov, trying not to look at
her, hastened in.

A minute later Sonia, too, came in with the candle, set down the
candlestick and, completely disconcerted, stood before him inexpressibly
agitated and apparently frightened by his unexpected visit. The colour
rushed suddenly to her pale face and tears came into her eyes\ldots{}
She felt sick and ashamed and happy, too\ldots{}. Raskolnikov turned
away quickly and sat on a chair by the table. He scanned the room in a
rapid glance.

It was a large but exceedingly low-pitched room, the only one let by the
Kapernaumovs, to whose rooms a closed door led in the wall on the left.
In the opposite side on the right hand wall was another door, always
kept locked. That led to the next flat, which formed a separate lodging.
Sonia's room looked like a barn; it was a very irregular quadrangle and
this gave it a grotesque appearance. A wall with three windows looking
out on to the canal ran aslant so that one corner formed a very acute
angle, and it was difficult to see in it without very strong light. The
other corner was disproportionately obtuse. There was scarcely any
furniture in the big room: in the corner on the right was a bedstead,
beside it, nearest the door, a chair. A plain, deal table covered by a
blue cloth stood against the same wall, close to the door into the other
flat. Two rush-bottom chairs stood by the table. On the opposite wall
near the acute angle stood a small plain wooden chest of drawers
looking, as it were, lost in a desert. That was all there was in the
room. The yellow, scratched and shabby wall-paper was black in the
corners. It must have been damp and full of fumes in the winter. There
was every sign of poverty; even the bedstead had no curtain.

Sonia looked in silence at her visitor, who was so attentively and
unceremoniously scrutinising her room, and even began at last to tremble
with terror, as though she was standing before her judge and the arbiter
of her destinies.

``I am late\ldots{}. It's eleven, isn't it?'' he asked, still not
lifting his eyes.

``Yes,'' muttered Sonia, ``oh yes, it is,'' she added, hastily, as
though in that lay her means of escape. ``My landlady's clock has just
struck\ldots{} I heard it myself\ldots{}.''

``I've come to you for the last time,'' Raskolnikov went on gloomily,
although this was the first time. ``I may perhaps not see you
again\ldots{}''

``Are you\ldots{} going away?''

``I don't know\ldots{} to-morrow\ldots{}.''

``Then you are not coming to Katerina Ivanovna to-morrow?'' Sonia's
voice shook.

``I don't know. I shall know to-morrow morning\ldots{}. Never mind that:
I've come to say one word\ldots{}.''

He raised his brooding eyes to her and suddenly noticed that he was
sitting down while she was all the while standing before him.

``Why are you standing? Sit down,'' he said in a changed voice, gentle
and friendly.

She sat down. He looked kindly and almost compassionately at her.

``How thin you are! What a hand! Quite transparent, like a dead hand.''

He took her hand. Sonia smiled faintly.

``I have always been like that,'' she said.

``Even when you lived at home?''

``Yes.''

``Of course, you were,'' he added abruptly and the expression of his
face and the sound of his voice changed again suddenly.

He looked round him once more.

``You rent this room from the Kapernaumovs?''

``Yes\ldots{}.''

``They live there, through that door?''

``Yes\ldots{}. They have another room like this.''

``All in one room?''

``Yes.''

``I should be afraid in your room at night,'' he observed gloomily.

``They are very good people, very kind,'' answered Sonia, who still
seemed bewildered, ``and all the furniture, everything\ldots{}
everything is theirs. And they are very kind and the children, too,
often come to see me.''

``They all stammer, don't they?''

``Yes\ldots{}. He stammers and he's lame. And his wife, too\ldots{}.
It's not exactly that she stammers, but she can't speak plainly. She is
a very kind woman. And he used to be a house serf. And there are seven
children\ldots{} and it's only the eldest one that stammers and the
others are simply ill\ldots{} but they don't stammer\ldots{}. But where
did you hear about them?'' she added with some surprise.

``Your father told me, then. He told me all about you\ldots{}. And how
you went out at six o'clock and came back at nine and how Katerina
Ivanovna knelt down by your bed.''

Sonia was confused.

``I fancied I saw him to-day,'' she whispered hesitatingly.

``Whom?''

``Father. I was walking in the street, out there at the corner, about
ten o'clock and he seemed to be walking in front. It looked just like
him. I wanted to go to Katerina Ivanovna\ldots{}.''

``You were walking in the streets?''

``Yes,'' Sonia whispered abruptly, again overcome with confusion and
looking down.

``Katerina Ivanovna used to beat you, I dare say?''

``Oh no, what are you saying? No!'' Sonia looked at him almost with
dismay.

``You love her, then?''

``Love her? Of course!'' said Sonia with plaintive emphasis, and she
clasped her hands in distress. ``Ah, you don't\ldots{}. If you only
knew! You see, she is quite like a child\ldots{}. Her mind is quite
unhinged, you see\ldots{} from sorrow. And how clever she used to
be\ldots{} how generous\ldots{} how kind! Ah, you don't understand, you
don't understand!''

Sonia said this as though in despair, wringing her hands in excitement
and distress. Her pale cheeks flushed, there was a look of anguish in
her eyes. It was clear that she was stirred to the very depths, that she
was longing to speak, to champion, to express something. A sort of
\emph{insatiable} compassion, if one may so express it, was reflected in
every feature of her face.

``Beat me! how can you? Good heavens, beat me! And if she did beat me,
what then? What of it? You know nothing, nothing about it\ldots{}. She
is so unhappy\ldots{} ah, how unhappy! And ill\ldots{}. She is seeking
righteousness, she is pure. She has such faith that there must be
righteousness everywhere and she expects it\ldots{}. And if you were to
torture her, she wouldn't do wrong. She doesn't see that it's impossible
for people to be righteous and she is angry at it. Like a child, like a
child. She is good!''

``And what will happen to you?''

Sonia looked at him inquiringly.

``They are left on your hands, you see. They were all on your hands
before, though\ldots{}. And your father came to you to beg for drink.
Well, how will it be now?''

``I don't know,'' Sonia articulated mournfully.

``Will they stay there?''

``I don't know\ldots{}. They are in debt for the lodging, but the
landlady, I hear, said to-day that she wanted to get rid of them, and
Katerina Ivanovna says that she won't stay another minute.''

``How is it she is so bold? She relies upon you?''

``Oh, no, don't talk like that\ldots{}. We are one, we live like one.''
Sonia was agitated again and even angry, as though a canary or some
other little bird were to be angry. ``And what could she do? What, what
could she do?'' she persisted, getting hot and excited. ``And how she
cried to-day! Her mind is unhinged, haven't you noticed it? At one
minute she is worrying like a child that everything should be right
to-morrow, the lunch and all that\ldots{}. Then she is wringing her
hands, spitting blood, weeping, and all at once she will begin knocking
her head against the wall, in despair. Then she will be comforted again.
She builds all her hopes on you; she says that you will help her now and
that she will borrow a little money somewhere and go to her native town
with me and set up a boarding school for the daughters of gentlemen and
take me to superintend it, and we will begin a new splendid life. And
she kisses and hugs me, comforts me, and you know she has such faith,
such faith in her fancies! One can't contradict her. And all the day
long she has been washing, cleaning, mending. She dragged the wash tub
into the room with her feeble hands and sank on the bed, gasping for
breath. We went this morning to the shops to buy shoes for Polenka and
Lida for theirs are quite worn out. Only the money we'd reckoned wasn't
enough, not nearly enough. And she picked out such dear little boots,
for she has taste, you don't know. And there in the shop she burst out
crying before the shopmen because she hadn't enough\ldots{}. Ah, it was
sad to see her\ldots{}.''

``Well, after that I can understand your living like this,'' Raskolnikov
said with a bitter smile.

``And aren't you sorry for them? Aren't you sorry?'' Sonia flew at him
again. ``Why, I know, you gave your last penny yourself, though you'd
seen nothing of it, and if you'd seen everything, oh dear! And how
often, how often I've brought her to tears! Only last week! Yes, I! Only
a week before his death. I was cruel! And how often I've done it! Ah,
I've been wretched at the thought of it all day!''

Sonia wrung her hands as she spoke at the pain of remembering it.

``You were cruel?''

``Yes, I---I. I went to see them,'' she went on, weeping, ``and father
said, `read me something, Sonia, my head aches, read to me, here's a
book.' He had a book he had got from Andrey Semyonovitch Lebeziatnikov,
he lives there, he always used to get hold of such funny books. And I
said, `I can't stay,' as I didn't want to read, and I'd gone in chiefly
to show Katerina Ivanovna some collars. Lizaveta, the pedlar, sold me
some collars and cuffs cheap, pretty, new, embroidered ones. Katerina
Ivanovna liked them very much; she put them on and looked at herself in
the glass and was delighted with them. `Make me a present of them,
Sonia,' she said, `please do.' `\emph{Please do},' she said, she wanted
them so much. And when could she wear them? They just reminded her of
her old happy days. She looked at herself in the glass, admired herself,
and she has no clothes at all, no things of her own, hasn't had all
these years! And she never asks anyone for anything; she is proud, she'd
sooner give away everything. And these she asked for, she liked them so
much. And I was sorry to give them. `What use are they to you, Katerina
Ivanovna?' I said. I spoke like that to her, I ought not to have said
that! She gave me such a look. And she was so grieved, so grieved at my
refusing her. And it was so sad to see\ldots{}. And she was not grieved
for the collars, but for my refusing, I saw that. Ah, if only I could
bring it all back, change it, take back those words! Ah, if I\ldots{}
but it's nothing to you!''

``Did you know Lizaveta, the pedlar?''

``Yes\ldots{}. Did you know her?'' Sonia asked with some surprise.

``Katerina Ivanovna is in consumption, rapid consumption; she will soon
die,'' said Raskolnikov after a pause, without answering her question.

``Oh, no, no, no!''

And Sonia unconsciously clutched both his hands, as though imploring
that she should not.

``But it will be better if she does die.''

``No, not better, not at all better!'' Sonia unconsciously repeated in
dismay.

``And the children? What can you do except take them to live with you?''

``Oh, I don't know,'' cried Sonia, almost in despair, and she put her
hands to her head.

It was evident that that idea had very often occurred to her before and
he had only roused it again.

``And, what, if even now, while Katerina Ivanovna is alive, you get ill
and are taken to the hospital, what will happen then?'' he persisted
pitilessly.

``How can you? That cannot be!''

And Sonia's face worked with awful terror.

``Cannot be?'' Raskolnikov went on with a harsh smile. ``You are not
insured against it, are you? What will happen to them then? They will be
in the street, all of them, she will cough and beg and knock her head
against some wall, as she did to-day, and the children will cry\ldots{}.
Then she will fall down, be taken to the police station and to the
hospital, she will die, and the children\ldots{}''

``Oh, no\ldots{}. God will not let it be!'' broke at last from Sonia's
overburdened bosom.

She listened, looking imploringly at him, clasping her hands in dumb
entreaty, as though it all depended upon him.

Raskolnikov got up and began to walk about the room. A minute passed.
Sonia was standing with her hands and her head hanging in terrible
dejection.

``And can't you save? Put by for a rainy day?'' he asked, stopping
suddenly before her.

``No,'' whispered Sonia.

``Of course not. Have you tried?'' he added almost ironically.

``Yes.''

``And it didn't come off! Of course not! No need to ask.''

And again he paced the room. Another minute passed.

``You don't get money every day?''

Sonia was more confused than ever and colour rushed into her face again.

``No,'' she whispered with a painful effort.

``It will be the same with Polenka, no doubt,'' he said suddenly.

``No, no! It can't be, no!'' Sonia cried aloud in desperation, as though
she had been stabbed. ``God would not allow anything so awful!''

``He lets others come to it.''

``No, no! God will protect her, God!'' she repeated beside herself.

``But, perhaps, there is no God at all,'' Raskolnikov answered with a
sort of malignance, laughed and looked at her.

Sonia's face suddenly changed; a tremor passed over it. She looked at
him with unutterable reproach, tried to say something, but could not
speak and broke into bitter, bitter sobs, hiding her face in her hands.

``You say Katerina Ivanovna's mind is unhinged; your own mind is
unhinged,'' he said after a brief silence.

Five minutes passed. He still paced up and down the room in silence, not
looking at her. At last he went up to her; his eyes glittered. He put
his two hands on her shoulders and looked straight into her tearful
face. His eyes were hard, feverish and piercing, his lips were
twitching. All at once he bent down quickly and dropping to the ground,
kissed her foot. Sonia drew back from him as from a madman. And
certainly he looked like a madman.

``What are you doing to me?'' she muttered, turning pale, and a sudden
anguish clutched at her heart.

He stood up at once.

``I did not bow down to you, I bowed down to all the suffering of
humanity,'' he said wildly and walked away to the window. ``Listen,'' he
added, turning to her a minute later. ``I said just now to an insolent
man that he was not worth your little finger\ldots{} and that I did my
sister honour making her sit beside you.''

``Ach, you said that to them! And in her presence?'' cried Sonia,
frightened. ``Sit down with me! An honour! Why, I'm\ldots{}
dishonourable\ldots{}. Ah, why did you say that?''

``It was not because of your dishonour and your sin I said that of you,
but because of your great suffering. But you are a great sinner, that's
true,'' he added almost solemnly, ``and your worst sin is that you have
destroyed and betrayed yourself \emph{for nothing}. Isn't that fearful?
Isn't it fearful that you are living in this filth which you loathe so,
and at the same time you know yourself (you've only to open your eyes)
that you are not helping anyone by it, not saving anyone from anything?
Tell me,'' he went on almost in a frenzy, ``how this shame and
degradation can exist in you side by side with other, opposite, holy
feelings? It would be better, a thousand times better and wiser to leap
into the water and end it all!''

``But what would become of them?'' Sonia asked faintly, gazing at him
with eyes of anguish, but not seeming surprised at his suggestion.

Raskolnikov looked strangely at her. He read it all in her face; so she
must have had that thought already, perhaps many times, and earnestly
she had thought out in her despair how to end it and so earnestly, that
now she scarcely wondered at his suggestion. She had not even noticed
the cruelty of his words. (The significance of his reproaches and his
peculiar attitude to her shame she had, of course, not noticed either,
and that, too, was clear to him.) But he saw how monstrously the thought
of her disgraceful, shameful position was torturing her and had long
tortured her. ``What, what,'' he thought, ``could hitherto have hindered
her from putting an end to it?'' Only then he realised what those poor
little orphan children and that pitiful half-crazy Katerina Ivanovna,
knocking her head against the wall in her consumption, meant for Sonia.

But, nevertheless, it was clear to him again that with her character and
the amount of education she had after all received, she could not in any
case remain so. He was still confronted by the question, how could she
have remained so long in that position without going out of her mind,
since she could not bring herself to jump into the water? Of course he
knew that Sonia's position was an exceptional case, though unhappily not
unique and not infrequent, indeed; but that very exceptionalness, her
tinge of education, her previous life might, one would have thought,
have killed her at the first step on that revolting path. What held her
up---surely not depravity? All that infamy had obviously only touched
her mechanically, not one drop of real depravity had penetrated to her
heart; he saw that. He saw through her as she stood before him\ldots{}.

``There are three ways before her,'' he thought, ``the canal, the
madhouse, or\ldots{} at last to sink into depravity which obscures the
mind and turns the heart to stone.''

The last idea was the most revolting, but he was a sceptic, he was
young, abstract, and therefore cruel, and so he could not help believing
that the last end was the most likely.

``But can that be true?'' he cried to himself. ``Can that creature who
has still preserved the purity of her spirit be consciously drawn at
last into that sink of filth and iniquity? Can the process already have
begun? Can it be that she has only been able to bear it till now,
because vice has begun to be less loathsome to her? No, no, that cannot
be!'' he cried, as Sonia had just before. ``No, what has kept her from
the canal till now is the idea of sin and they, the children\ldots{}.
And if she has not gone out of her mind\ldots{} but who says she has not
gone out of her mind? Is she in her senses? Can one talk, can one reason
as she does? How can she sit on the edge of the abyss of loathsomeness
into which she is slipping and refuse to listen when she is told of
danger? Does she expect a miracle? No doubt she does. Doesn't that all
mean madness?''

He stayed obstinately at that thought. He liked that explanation indeed
better than any other. He began looking more intently at her.

``So you pray to God a great deal, Sonia?'' he asked her.

Sonia did not speak; he stood beside her waiting for an answer.

``What should I be without God?'' she whispered rapidly, forcibly,
glancing at him with suddenly flashing eyes, and squeezing his hand.

``Ah, so that is it!'' he thought.

``And what does God do for you?'' he asked, probing her further.

Sonia was silent a long while, as though she could not answer. Her weak
chest kept heaving with emotion.

``Be silent! Don't ask! You don't deserve!'' she cried suddenly, looking
sternly and wrathfully at him.

``That's it, that's it,'' he repeated to himself.

``He does everything,'' she whispered quickly, looking down again.

``That's the way out! That's the explanation,'' he decided, scrutinising
her with eager curiosity, with a new, strange, almost morbid feeling. He
gazed at that pale, thin, irregular, angular little face, those soft
blue eyes, which could flash with such fire, such stern energy, that
little body still shaking with indignation and anger---and it all seemed
to him more and more strange, almost impossible. ``She is a religious
maniac!'' he repeated to himself.

There was a book lying on the chest of drawers. He had noticed it every
time he paced up and down the room. Now he took it up and looked at it.
It was the New Testament in the Russian translation. It was bound in
leather, old and worn.

``Where did you get that?'' he called to her across the room.

She was still standing in the same place, three steps from the table.

``It was brought me,'' she answered, as it were unwillingly, not looking
at him.

``Who brought it?''

``Lizaveta, I asked her for it.''

``Lizaveta! strange!'' he thought.

Everything about Sonia seemed to him stranger and more wonderful every
moment. He carried the book to the candle and began to turn over the
pages.

``Where is the story of Lazarus?'' he asked suddenly.

Sonia looked obstinately at the ground and would not answer. She was
standing sideways to the table.

``Where is the raising of Lazarus? Find it for me, Sonia.''

She stole a glance at him.

``You are not looking in the right place\ldots{}. It's in the fourth
gospel,'' she whispered sternly, without looking at him.

``Find it and read it to me,'' he said. He sat down with his elbow on
the table, leaned his head on his hand and looked away sullenly,
prepared to listen.

``In three weeks' time they'll welcome me in the madhouse! I shall be
there if I am not in a worse place,'' he muttered to himself.

Sonia heard Raskolnikov's request distrustfully and moved hesitatingly
to the table. She took the book however.

``Haven't you read it?'' she asked, looking up at him across the table.

Her voice became sterner and sterner.

``Long ago\ldots{}. When I was at school. Read!''

``And haven't you heard it in church?''

``I\ldots{} haven't been. Do you often go?''

``N-no,'' whispered Sonia.

Raskolnikov smiled.

``I understand\ldots{}. And you won't go to your father's funeral
to-morrow?''

``Yes, I shall. I was at church last week, too\ldots{} I had a requiem
service.''

``For whom?''

``For Lizaveta. She was killed with an axe.''

His nerves were more and more strained. His head began to go round.

``Were you friends with Lizaveta?''

``Yes\ldots{}. She was good\ldots{} she used to come\ldots{} not
often\ldots{} she couldn't\ldots{}. We used to read together and\ldots{}
talk. She will see God.''

The last phrase sounded strange in his ears. And here was something new
again: the mysterious meetings with Lizaveta and both of
them---religious maniacs.

``I shall be a religious maniac myself soon! It's infectious!''

``Read!'' he cried irritably and insistently.

Sonia still hesitated. Her heart was throbbing. She hardly dared to read
to him. He looked almost with exasperation at the ``unhappy lunatic.''

``What for? You don't believe?\ldots{}'' she whispered softly and as it
were breathlessly.

``Read! I want you to,'' he persisted. ``You used to read to Lizaveta.''

Sonia opened the book and found the place. Her hands were shaking, her
voice failed her. Twice she tried to begin and could not bring out the
first syllable.

``Now a certain man was sick named Lazarus of Bethany\ldots{}'' she
forced herself at last to read, but at the third word her voice broke
like an overstrained string. There was a catch in her breath.

Raskolnikov saw in part why Sonia could not bring herself to read to him
and the more he saw this, the more roughly and irritably he insisted on
her doing so. He understood only too well how painful it was for her to
betray and unveil all that was her \emph{own}. He understood that these
feelings really were her \emph{secret treasure}, which she had kept
perhaps for years, perhaps from childhood, while she lived with an
unhappy father and a distracted stepmother crazed by grief, in the midst
of starving children and unseemly abuse and reproaches. But at the same
time he knew now and knew for certain that, although it filled her with
dread and suffering, yet she had a tormenting desire to read and to read
to \emph{him} that he might hear it, and to read \emph{now} whatever
might come of it!\ldots{} He read this in her eyes, he could see it in
her intense emotion. She mastered herself, controlled the spasm in her
throat and went on reading the eleventh chapter of St. John. She went on
to the nineteenth verse:

``And many of the Jews came to Martha and Mary to comfort them
concerning their brother.

``Then Martha as soon as she heard that Jesus was coming went and met
Him: but Mary sat still in the house.

``Then said Martha unto Jesus, Lord, if Thou hadst been here, my brother
had not died.

``But I know that even now whatsoever Thou wilt ask of God, God will
give it Thee\ldots{}.''

Then she stopped again with a shamefaced feeling that her voice would
quiver and break again.

``Jesus said unto her, thy brother shall rise again.

``Martha saith unto Him, I know that he shall rise again in the
resurrection, at the last day.

``Jesus said unto her, I am the resurrection and the life: he that
believeth in Me though he were dead, yet shall he live.

``And whosoever liveth and believeth in Me shall never die. Believest
thou this?

``She saith unto Him,''

(And drawing a painful breath, Sonia read distinctly and forcibly as
though she were making a public confession of faith.)

``Yea, Lord: I believe that Thou art the Christ, the Son of God Which
should come into the world.''

She stopped and looked up quickly at him, but controlling herself went
on reading. Raskolnikov sat without moving, his elbows on the table and
his eyes turned away. She read to the thirty-second verse.

``Then when Mary was come where Jesus was and saw Him, she fell down at
His feet, saying unto Him, Lord if Thou hadst been here, my brother had
not died.

``When Jesus therefore saw her weeping, and the Jews also weeping which
came with her, He groaned in the spirit and was troubled,

``And said, Where have ye laid him? They said unto Him, Lord, come and
see.

``Jesus wept.

``Then said the Jews, behold how He loved him!

``And some of them said, could not this Man which opened the eyes of the
blind, have caused that even this man should not have died?''

Raskolnikov turned and looked at her with emotion. Yes, he had known it!
She was trembling in a real physical fever. He had expected it. She was
getting near the story of the greatest miracle and a feeling of immense
triumph came over her. Her voice rang out like a bell; triumph and joy
gave it power. The lines danced before her eyes, but she knew what she
was reading by heart. At the last verse ``Could not this Man which
opened the eyes of the blind\ldots{}'' dropping her voice she
passionately reproduced the doubt, the reproach and censure of the blind
disbelieving Jews, who in another moment would fall at His feet as
though struck by thunder, sobbing and believing\ldots{}. ``And \emph{he,
he}---too, is blinded and unbelieving, he, too, will hear, he, too, will
believe, yes, yes! At once, now,'' was what she was dreaming, and she
was quivering with happy anticipation.

``Jesus therefore again groaning in Himself cometh to the grave. It was
a cave, and a stone lay upon it.

``Jesus said, Take ye away the stone. Martha, the sister of him that was
dead, saith unto Him, Lord by this time he stinketh: for he hath been
dead four days.''

She laid emphasis on the word \emph{four}.

``Jesus saith unto her, Said I not unto thee that if thou wouldest
believe, thou shouldest see the glory of God?

``Then they took away the stone from the place where the dead was laid.
And Jesus lifted up His eyes and said, Father, I thank Thee that Thou
hast heard Me.

``And I knew that Thou hearest Me always; but because of the people
which stand by I said it, that they may believe that Thou hast sent Me.

``And when He thus had spoken, He cried with a loud voice, Lazarus, come
forth.

``And he that was dead came forth.''

(She read loudly, cold and trembling with ecstasy, as though she were
seeing it before her eyes.)

``Bound hand and foot with graveclothes; and his face was bound about
with a napkin. Jesus saith unto them, Loose him and let him go.

``Then many of the Jews which came to Mary and had seen the things which
Jesus did believed on Him.''

She could read no more, closed the book and got up from her chair
quickly.

``That is all about the raising of Lazarus,'' she whispered severely and
abruptly, and turning away she stood motionless, not daring to raise her
eyes to him. She still trembled feverishly. The candle-end was
flickering out in the battered candlestick, dimly lighting up in the
poverty-stricken room the murderer and the harlot who had so strangely
been reading together the eternal book. Five minutes or more passed.

``I came to speak of something,'' Raskolnikov said aloud, frowning. He
got up and went to Sonia. She lifted her eyes to him in silence. His
face was particularly stern and there was a sort of savage determination
in it.

``I have abandoned my family to-day,'' he said, ``my mother and sister.
I am not going to see them. I've broken with them completely.''

``What for?'' asked Sonia amazed. Her recent meeting with his mother and
sister had left a great impression which she could not analyse. She
heard his news almost with horror.

``I have only you now,'' he added. ``Let us go together\ldots{}. I've
come to you, we are both accursed, let us go our way together!''

His eyes glittered ``as though he were mad,'' Sonia thought, in her
turn.

``Go where?'' she asked in alarm and she involuntarily stepped back.

``How do I know? I only know it's the same road, I know that and nothing
more. It's the same goal!''

She looked at him and understood nothing. She knew only that he was
terribly, infinitely unhappy.

``No one of them will understand, if you tell them, but I have
understood. I need you, that is why I have come to you.''

``I don't understand,'' whispered Sonia.

``You'll understand later. Haven't you done the same? You, too, have
transgressed\ldots{} have had the strength to transgress. You have laid
hands on yourself, you have destroyed a life\ldots{} \emph{your own}
(it's all the same!). You might have lived in spirit and understanding,
but you'll end in the Hay Market\ldots{}. But you won't be able to stand
it, and if you remain alone you'll go out of your mind like me. You are
like a mad creature already. So we must go together on the same road!
Let us go!''

``What for? What's all this for?'' said Sonia, strangely and violently
agitated by his words.

``What for? Because you can't remain like this, that's why! You must
look things straight in the face at last, and not weep like a child and
cry that God won't allow it. What will happen, if you should really be
taken to the hospital to-morrow? She is mad and in consumption, she'll
soon die and the children? Do you mean to tell me Polenka won't come to
grief? Haven't you seen children here at the street corners sent out by
their mothers to beg? I've found out where those mothers live and in
what surroundings. Children can't remain children there! At seven the
child is vicious and a thief. Yet children, you know, are the image of
Christ: `theirs is the kingdom of Heaven.' He bade us honour and love
them, they are the humanity of the future\ldots{}.''

``What's to be done, what's to be done?'' repeated Sonia, weeping
hysterically and wringing her hands.

``What's to be done? Break what must be broken, once for all, that's
all, and take the suffering on oneself. What, you don't understand?
You'll understand later\ldots{}. Freedom and power, and above all,
power! Over all trembling creation and all the ant-heap!\ldots{} That's
the goal, remember that! That's my farewell message. Perhaps it's the
last time I shall speak to you. If I don't come to-morrow, you'll hear
of it all, and then remember these words. And some day later on, in
years to come, you'll understand perhaps what they meant. If I come
to-morrow, I'll tell you who killed Lizaveta\ldots{}. Good-bye.''

Sonia started with terror.

``Why, do you know who killed her?'' she asked, chilled with horror,
looking wildly at him.

``I know and will tell\ldots{} you, only you. I have chosen you out. I'm
not coming to you to ask forgiveness, but simply to tell you. I chose
you out long ago to hear this, when your father talked of you and when
Lizaveta was alive, I thought of it. Good-bye, don't shake hands.
To-morrow!''

He went out. Sonia gazed at him as at a madman. But she herself was like
one insane and felt it. Her head was going round.

``Good heavens, how does he know who killed Lizaveta? What did those
words mean? It's awful!'' But at the same time \emph{the idea} did not
enter her head, not for a moment! ``Oh, he must be terribly
unhappy!\ldots{} He has abandoned his mother and sister\ldots{}. What
for? What has happened? And what had he in his mind? What did he say to
her? He had kissed her foot and said\ldots{} said (yes, he had said it
clearly) that he could not live without her\ldots{}. Oh, merciful
heavens!''

Sonia spent the whole night feverish and delirious. She jumped up from
time to time, wept and wrung her hands, then sank again into feverish
sleep and dreamt of Polenka, Katerina Ivanovna and Lizaveta, of reading
the gospel and him\ldots{} him with pale face, with burning eyes\ldots{}
kissing her feet, weeping.

On the other side of the door on the right, which divided Sonia's room
from Madame Resslich's flat, was a room which had long stood empty. A
card was fixed on the gate and a notice stuck in the windows over the
canal advertising it to let. Sonia had long been accustomed to the
room's being uninhabited. But all that time Mr. Svidrigaïlov had been
standing, listening at the door of the empty room. When Raskolnikov went
out he stood still, thought a moment, went on tiptoe to his own room
which adjoined the empty one, brought a chair and noiselessly carried it
to the door that led to Sonia's room. The conversation had struck him as
interesting and remarkable, and he had greatly enjoyed it---so much so
that he brought a chair that he might not in the future, to-morrow, for
instance, have to endure the inconvenience of standing a whole hour, but
might listen in comfort.

\protect\hypertarget{link2HCH0025}{}{}

\mychapter{5}{Chapter V}\label{chapter-v-3}

When next morning at eleven o'clock punctually Raskolnikov went into the
department of the investigation of criminal causes and sent his name in
to Porfiry Petrovitch, he was surprised at being kept waiting so long:
it was at least ten minutes before he was summoned. He had expected that
they would pounce upon him. But he stood in the waiting-room, and
people, who apparently had nothing to do with him, were continually
passing to and fro before him. In the next room which looked like an
office, several clerks were sitting writing and obviously they had no
notion who or what Raskolnikov might be. He looked uneasily and
suspiciously about him to see whether there was not some guard, some
mysterious watch being kept on him to prevent his escape. But there was
nothing of the sort: he saw only the faces of clerks absorbed in petty
details, then other people, no one seemed to have any concern with him.
He might go where he liked for them. The conviction grew stronger in him
that if that enigmatic man of yesterday, that phantom sprung out of the
earth, had seen everything, they would not have let him stand and wait
like that. And would they have waited till he elected to appear at
eleven? Either the man had not yet given information, or\ldots{} or
simply he knew nothing, had seen nothing (and how could he have seen
anything?) and so all that had happened to him the day before was again
a phantom exaggerated by his sick and overstrained imagination. This
conjecture had begun to grow strong the day before, in the midst of all
his alarm and despair. Thinking it all over now and preparing for a
fresh conflict, he was suddenly aware that he was trembling---and he
felt a rush of indignation at the thought that he was trembling with
fear at facing that hateful Porfiry Petrovitch. What he dreaded above
all was meeting that man again; he hated him with an intense,
unmitigated hatred and was afraid his hatred might betray him. His
indignation was such that he ceased trembling at once; he made ready to
go in with a cold and arrogant bearing and vowed to himself to keep as
silent as possible, to watch and listen and for once at least to control
his overstrained nerves. At that moment he was summoned to Porfiry
Petrovitch.

He found Porfiry Petrovitch alone in his study. His study was a room
neither large nor small, furnished with a large writing-table, that
stood before a sofa, upholstered in checked material, a bureau, a
bookcase in the corner and several chairs---all government furniture, of
polished yellow wood. In the further wall there was a closed door,
beyond it there were no doubt other rooms. On Raskolnikov's entrance
Porfiry Petrovitch had at once closed the door by which he had come in
and they remained alone. He met his visitor with an apparently genial
and good-tempered air, and it was only after a few minutes that
Raskolnikov saw signs of a certain awkwardness in him, as though he had
been thrown out of his reckoning or caught in something very secret.

``Ah, my dear fellow! Here you are\ldots{} in our domain''\ldots{} began
Porfiry, holding out both hands to him. ``Come, sit down, old
man\ldots{} or perhaps you don't like to be called `my dear fellow' and
`old man!'---\emph{tout court}? Please don't think it too
familiar\ldots{}. Here, on the sofa.''

Raskolnikov sat down, keeping his eyes fixed on him. ``In our domain,''
the apologies for familiarity, the French phrase \emph{tout court}, were
all characteristic signs.

``He held out both hands to me, but he did not give me one---he drew it
back in time,'' struck him suspiciously. Both were watching each other,
but when their eyes met, quick as lightning they looked away.

``I brought you this paper\ldots{} about the watch. Here it is. Is it
all right or shall I copy it again?''

``What? A paper? Yes, yes, don't be uneasy, it's all right,'' Porfiry
Petrovitch said as though in haste, and after he had said it he took the
paper and looked at it. ``Yes, it's all right. Nothing more is needed,''
he declared with the same rapidity and he laid the paper on the table.

A minute later when he was talking of something else he took it from the
table and put it on his bureau.

``I believe you said yesterday you would like to question me\ldots{}
formally\ldots{} about my acquaintance with the murdered woman?''
Raskolnikov was beginning again. ``Why did I put in `I believe'\,''
passed through his mind in a flash. ``Why am I so uneasy at having put
in that `\emph{I believe}'?'' came in a second flash. And he suddenly
felt that his uneasiness at the mere contact with Porfiry, at the first
words, at the first looks, had grown in an instant to monstrous
proportions, and that this was fearfully dangerous. His nerves were
quivering, his emotion was increasing. ``It's bad, it's bad! I shall say
too much again.''

``Yes, yes, yes! There's no hurry, there's no hurry,'' muttered Porfiry
Petrovitch, moving to and fro about the table without any apparent aim,
as it were making dashes towards the window, the bureau and the table,
at one moment avoiding Raskolnikov's suspicious glance, then again
standing still and looking him straight in the face.

His fat round little figure looked very strange, like a ball rolling
from one side to the other and rebounding back.

``We've plenty of time. Do you smoke? have you your own? Here, a
cigarette!'' he went on, offering his visitor a cigarette. ``You know I
am receiving you here, but my own quarters are through there, you know,
my government quarters. But I am living outside for the time, I had to
have some repairs done here. It's almost finished now\ldots{}.
Government quarters, you know, are a capital thing. Eh, what do you
think?''

``Yes, a capital thing,'' answered Raskolnikov, looking at him almost
ironically.

``A capital thing, a capital thing,'' repeated Porfiry Petrovitch, as
though he had just thought of something quite different. ``Yes, a
capital thing,'' he almost shouted at last, suddenly staring at
Raskolnikov and stopping short two steps from him.

This stupid repetition was too incongruous in its ineptitude with the
serious, brooding and enigmatic glance he turned upon his visitor.

But this stirred Raskolnikov's spleen more than ever and he could not
resist an ironical and rather incautious challenge.

``Tell me, please,'' he asked suddenly, looking almost insolently at him
and taking a kind of pleasure in his own insolence. ``I believe it's a
sort of legal rule, a sort of legal tradition---for all investigating
lawyers---to begin their attack from afar, with a trivial, or at least
an irrelevant subject, so as to encourage, or rather, to divert the man
they are cross-examining, to disarm his caution and then all at once to
give him an unexpected knock-down blow with some fatal question. Isn't
that so? It's a sacred tradition, mentioned, I fancy, in all the manuals
of the art?''

``Yes, yes\ldots{}. Why, do you imagine that was why I spoke about
government quarters\ldots{} eh?''

And as he said this Porfiry Petrovitch screwed up his eyes and winked; a
good-humoured, crafty look passed over his face. The wrinkles on his
forehead were smoothed out, his eyes contracted, his features broadened
and he suddenly went off into a nervous prolonged laugh, shaking all
over and looking Raskolnikov straight in the face. The latter forced
himself to laugh, too, but when Porfiry, seeing that he was laughing,
broke into such a guffaw that he turned almost crimson, Raskolnikov's
repulsion overcame all precaution; he left off laughing, scowled and
stared with hatred at Porfiry, keeping his eyes fixed on him while his
intentionally prolonged laughter lasted. There was lack of precaution on
both sides, however, for Porfiry Petrovitch seemed to be laughing in his
visitor's face and to be very little disturbed at the annoyance with
which the visitor received it. The latter fact was very significant in
Raskolnikov's eyes: he saw that Porfiry Petrovitch had not been
embarrassed just before either, but that he, Raskolnikov, had perhaps
fallen into a trap; that there must be something, some motive here
unknown to him; that, perhaps, everything was in readiness and in
another moment would break upon him\ldots{}

He went straight to the point at once, rose from his seat and took his
cap.

``Porfiry Petrovitch,'' he began resolutely, though with considerable
irritation, ``yesterday you expressed a desire that I should come to you
for some inquiries'' (he laid special stress on the word ``inquiries'').
``I have come and if you have anything to ask me, ask it, and if not,
allow me to withdraw. I have no time to spare\ldots{}. I have to be at
the funeral of that man who was run over, of whom you\ldots{} know
also,'' he added, feeling angry at once at having made this addition and
more irritated at his anger. ``I am sick of it all, do you hear? and
have long been. It's partly what made me ill. In short,'' he shouted,
feeling that the phrase about his illness was still more out of place,
``in short, kindly examine me or let me go, at once. And if you must
examine me, do so in the proper form! I will not allow you to do so
otherwise, and so meanwhile, good-bye, as we have evidently nothing to
keep us now.''

``Good heavens! What do you mean? What shall I question you about?''
cackled Porfiry Petrovitch with a change of tone, instantly leaving off
laughing. ``Please don't disturb yourself,'' he began fidgeting from
place to place and fussily making Raskolnikov sit down. ``There's no
hurry, there's no hurry, it's all nonsense. Oh, no, I'm very glad you've
come to see me at last\ldots{} I look upon you simply as a visitor. And
as for my confounded laughter, please excuse it, Rodion Romanovitch.
Rodion Romanovitch? That is your name?\ldots{} It's my nerves, you
tickled me so with your witty observation; I assure you, sometimes I
shake with laughter like an india-rubber ball for half an hour at a
time\ldots{}. I'm often afraid of an attack of paralysis. Do sit down.
Please do, or I shall think you are angry\ldots{}''

Raskolnikov did not speak; he listened, watching him, still frowning
angrily. He did sit down, but still held his cap.

``I must tell you one thing about myself, my dear Rodion Romanovitch,''
Porfiry Petrovitch continued, moving about the room and again avoiding
his visitor's eyes. ``You see, I'm a bachelor, a man of no consequence
and not used to society; besides, I have nothing before me, I'm set, I'm
running to seed and\ldots{} and have you noticed, Rodion Romanovitch,
that in our Petersburg circles, if two clever men meet who are not
intimate, but respect each other, like you and me, it takes them half an
hour before they can find a subject for conversation---they are dumb,
they sit opposite each other and feel awkward. Everyone has subjects of
conversation, ladies for instance\ldots{} people in high society always
have their subjects of conversation, \emph{c'est de rigueur}, but people
of the middle sort like us, thinking people that is, are always
tongue-tied and awkward. What is the reason of it? Whether it is the
lack of public interest, or whether it is we are so honest we don't want
to deceive one another, I don't know. What do you think? Do put down
your cap, it looks as if you were just going, it makes me
uncomfortable\ldots{} I am so delighted\ldots{}''

Raskolnikov put down his cap and continued listening in silence with a
serious frowning face to the vague and empty chatter of Porfiry
Petrovitch. ``Does he really want to distract my attention with his
silly babble?''

``I can't offer you coffee here; but why not spend five minutes with a
friend?'' Porfiry pattered on, ``and you know all these official
duties\ldots{} please don't mind my running up and down, excuse it, my
dear fellow, I am very much afraid of offending you, but exercise is
absolutely indispensable for me. I'm always sitting and so glad to be
moving about for five minutes\ldots{} I suffer from my sedentary
life\ldots{} I always intend to join a gymnasium; they say that
officials of all ranks, even Privy Councillors, may be seen skipping
gaily there; there you have it, modern science\ldots{} yes, yes\ldots{}.
But as for my duties here, inquiries and all such formalities\ldots{}
you mentioned inquiries yourself just now\ldots{} I assure you these
interrogations are sometimes more embarrassing for the interrogator than
for the interrogated\ldots{}. You made the observation yourself just now
very aptly and wittily.'' (Raskolnikov had made no observation of the
kind.) ``One gets into a muddle! A regular muddle! One keeps harping on
the same note, like a drum! There is to be a reform and we shall be
called by a different name, at least, he-he-he! And as for our legal
tradition, as you so wittily called it, I thoroughly agree with you.
Every prisoner on trial, even the rudest peasant, knows that they begin
by disarming him with irrelevant questions (as you so happily put it)
and then deal him a knock-down blow, he-he-he!---your felicitous
comparison, he-he! So you really imagined that I meant by `government
quarters'\ldots{} he-he! You are an ironical person. Come. I won't go
on! Ah, by the way, yes! One word leads to another. You spoke of
formality just now, apropos of the inquiry, you know. But what's the use
of formality? In many cases it's nonsense. Sometimes one has a friendly
chat and gets a good deal more out of it. One can always fall back on
formality, allow me to assure you. And after all, what does it amount
to? An examining lawyer cannot be bounded by formality at every step.
The work of investigation is, so to speak, a free art in its own way,
he-he-he!''

Porfiry Petrovitch took breath a moment. He had simply babbled on
uttering empty phrases, letting slip a few enigmatic words and again
reverting to incoherence. He was almost running about the room, moving
his fat little legs quicker and quicker, looking at the ground, with his
right hand behind his back, while with his left making gesticulations
that were extraordinarily incongruous with his words. Raskolnikov
suddenly noticed that as he ran about the room he seemed twice to stop
for a moment near the door, as though he were listening.

``Is he expecting anything?''

``You are certainly quite right about it,'' Porfiry began gaily, looking
with extraordinary simplicity at Raskolnikov (which startled him and
instantly put him on his guard); ``certainly quite right in laughing so
wittily at our legal forms, he-he! Some of these elaborate psychological
methods are exceedingly ridiculous and perhaps useless, if one adheres
too closely to the forms. Yes\ldots{} I am talking of forms again. Well,
if I recognise, or more strictly speaking, if I suspect someone or other
to be a criminal in any case entrusted to me\ldots{} you're reading for
the law, of course, Rodion Romanovitch?''

``Yes, I was\ldots{}''

``Well, then it is a precedent for you for the future---though don't
suppose I should venture to instruct you after the articles you publish
about crime! No, I simply make bold to state it by way of fact, if I
took this man or that for a criminal, why, I ask, should I worry him
prematurely, even though I had evidence against him? In one case I may
be bound, for instance, to arrest a man at once, but another may be in
quite a different position, you know, so why shouldn't I let him walk
about the town a bit? he-he-he! But I see you don't quite understand, so
I'll give you a clearer example. If I put him in prison too soon, I may
very likely give him, so to speak, moral support, he-he! You're
laughing?''

Raskolnikov had no idea of laughing. He was sitting with compressed
lips, his feverish eyes fixed on Porfiry Petrovitch's.

``Yet that is the case, with some types especially, for men are so
different. You say `evidence'. Well, there may be evidence. But
evidence, you know, can generally be taken two ways. I am an examining
lawyer and a weak man, I confess it. I should like to make a proof, so
to say, mathematically clear. I should like to make a chain of evidence
such as twice two are four, it ought to be a direct, irrefutable proof!
And if I shut him up too soon---even though I might be convinced
\emph{he} was the man, I should very likely be depriving myself of the
means of getting further evidence against him. And how? By giving him,
so to speak, a definite position, I shall put him out of suspense and
set his mind at rest, so that he will retreat into his shell. They say
that at Sevastopol, soon after Alma, the clever people were in a
terrible fright that the enemy would attack openly and take Sevastopol
at once. But when they saw that the enemy preferred a regular siege,
they were delighted, I am told and reassured, for the thing would drag
on for two months at least. You're laughing, you don't believe me again?
Of course, you're right, too. You're right, you're right. These are
special cases, I admit. But you must observe this, my dear Rodion
Romanovitch, the general case, the case for which all legal forms and
rules are intended, for which they are calculated and laid down in
books, does not exist at all, for the reason that every case, every
crime, for instance, so soon as it actually occurs, at once becomes a
thoroughly special case and sometimes a case unlike any that's gone
before. Very comic cases of that sort sometimes occur. If I leave one
man quite alone, if I don't touch him and don't worry him, but let him
know or at least suspect every moment that I know all about it and am
watching him day and night, and if he is in continual suspicion and
terror, he'll be bound to lose his head. He'll come of himself, or maybe
do something which will make it as plain as twice two are four---it's
delightful. It may be so with a simple peasant, but with one of our
sort, an intelligent man cultivated on a certain side, it's a dead
certainty. For, my dear fellow, it's a very important matter to know on
what side a man is cultivated. And then there are nerves, there are
nerves, you have overlooked them! Why, they are all sick, nervous and
irritable!\ldots{} And then how they all suffer from spleen! That I
assure you is a regular gold-mine for us. And it's no anxiety to me, his
running about the town free! Let him, let him walk about for a bit! I
know well enough that I've caught him and that he won't escape me. Where
could he escape to, he-he? Abroad, perhaps? A Pole will escape abroad,
but not here, especially as I am watching and have taken measures. Will
he escape into the depths of the country perhaps? But you know, peasants
live there, real rude Russian peasants. A modern cultivated man would
prefer prison to living with such strangers as our peasants. He-he! But
that's all nonsense, and on the surface. It's not merely that he has
nowhere to run to, he is \emph{psychologically} unable to escape me,
he-he! What an expression! Through a law of nature he can't escape me if
he had anywhere to go. Have you seen a butterfly round a candle? That's
how he will keep circling and circling round me. Freedom will lose its
attractions. He'll begin to brood, he'll weave a tangle round himself,
he'll worry himself to death! What's more he will provide me with a
mathematical proof---if I only give him long enough interval\ldots{}.
And he'll keep circling round me, getting nearer and nearer and
then---flop! He'll fly straight into my mouth and I'll swallow him, and
that will be very amusing, he-he-he! You don't believe me?''

Raskolnikov made no reply; he sat pale and motionless, still gazing with
the same intensity into Porfiry's face.

``It's a lesson,'' he thought, turning cold. ``This is beyond the cat
playing with a mouse, like yesterday. He can't be showing off his power
with no motive\ldots{} prompting me; he is far too clever for
that\ldots{} he must have another object. What is it? It's all nonsense,
my friend, you are pretending, to scare me! You've no proofs and the man
I saw had no real existence. You simply want to make me lose my head, to
work me up beforehand and so to crush me. But you are wrong, you won't
do it! But why give me such a hint? Is he reckoning on my shattered
nerves? No, my friend, you are wrong, you won't do it even though you
have some trap for me\ldots{} let us see what you have in store for
me.''

And he braced himself to face a terrible and unknown ordeal. At times he
longed to fall on Porfiry and strangle him. This anger was what he
dreaded from the beginning. He felt that his parched lips were flecked
with foam, his heart was throbbing. But he was still determined not to
speak till the right moment. He realised that this was the best policy
in his position, because instead of saying too much he would be
irritating his enemy by his silence and provoking him into speaking too
freely. Anyhow, this was what he hoped for.

``No, I see you don't believe me, you think I am playing a harmless joke
on you,'' Porfiry began again, getting more and more lively, chuckling
at every instant and again pacing round the room. ``And to be sure
you're right: God has given me a figure that can awaken none but comic
ideas in other people; a buffoon; but let me tell you, and I repeat it,
excuse an old man, my dear Rodion Romanovitch, you are a man still
young, so to say, in your first youth and so you put intellect above
everything, like all young people. Playful wit and abstract arguments
fascinate you and that's for all the world like the old Austrian
\emph{Hof-kriegsrath}, as far as I can judge of military matters, that
is: on paper they'd beaten Napoleon and taken him prisoner, and there in
their study they worked it all out in the cleverest fashion, but look
you, General Mack surrendered with all his army, he-he-he! I see, I see,
Rodion Romanovitch, you are laughing at a civilian like me, taking
examples out of military history! But I can't help it, it's my weakness.
I am fond of military science. And I'm ever so fond of reading all
military histories. I've certainly missed my proper career. I ought to
have been in the army, upon my word I ought. I shouldn't have been a
Napoleon, but I might have been a major, he-he! Well, I'll tell you the
whole truth, my dear fellow, about this \emph{special case}, I mean:
actual fact and a man's temperament, my dear sir, are weighty matters
and it's astonishing how they sometimes deceive the sharpest
calculation! I---listen to an old man---am speaking seriously, Rodion
Romanovitch'' (as he said this Porfiry Petrovitch, who was scarcely
five-and-thirty, actually seemed to have grown old; even his voice
changed and he seemed to shrink together) ``Moreover, I'm a candid
man\ldots{} am I a candid man or not? What do you say? I fancy I really
am: I tell you these things for nothing and don't even expect a reward
for it, he-he! Well, to proceed, wit in my opinion is a splendid thing,
it is, so to say, an adornment of nature and a consolation of life, and
what tricks it can play! So that it sometimes is hard for a poor
examining lawyer to know where he is, especially when he's liable to be
carried away by his own fancy, too, for you know he is a man after all!
But the poor fellow is saved by the criminal's temperament, worse luck
for him! But young people carried away by their own wit don't think of
that `when they overstep all obstacles,' as you wittily and cleverly
expressed it yesterday. He will lie---that is, the man who is a
\emph{special case}, the incognito, and he will lie well, in the
cleverest fashion; you might think he would triumph and enjoy the fruits
of his wit, but at the most interesting, the most flagrant moment he
will faint. Of course there may be illness and a stuffy room as well,
but anyway! Anyway he's given us the idea! He lied incomparably, but he
didn't reckon on his temperament. That's what betrays him! Another time
he will be carried away by his playful wit into making fun of the man
who suspects him, he will turn pale as it were on purpose to mislead,
but his paleness will be \emph{too natural}, too much like the real
thing, again he has given us an idea! Though his questioner may be
deceived at first, he will think differently next day if he is not a
fool, and, of course, it is like that at every step! He puts himself
forward where he is not wanted, speaks continually when he ought to keep
silent, brings in all sorts of allegorical allusions, he-he! Comes and
asks why didn't you take me long ago? he-he-he! And that can happen, you
know, with the cleverest man, the psychologist, the literary man. The
temperament reflects everything like a mirror! Gaze into it and admire
what you see! But why are you so pale, Rodion Romanovitch? Is the room
stuffy? Shall I open the window?''

``Oh, don't trouble, please,'' cried Raskolnikov and he suddenly broke
into a laugh. ``Please don't trouble.''

Porfiry stood facing him, paused a moment and suddenly he too laughed.
Raskolnikov got up from the sofa, abruptly checking his hysterical
laughter.

``Porfiry Petrovitch,'' he began, speaking loudly and distinctly, though
his legs trembled and he could scarcely stand. ``I see clearly at last
that you actually suspect me of murdering that old woman and her sister
Lizaveta. Let me tell you for my part that I am sick of this. If you
find that you have a right to prosecute me legally, to arrest me, then
prosecute me, arrest me. But I will not let myself be jeered at to my
face and worried\ldots{}''

His lips trembled, his eyes glowed with fury and he could not restrain
his voice.

``I won't allow it!'' he shouted, bringing his fist down on the table.
``Do you hear that, Porfiry Petrovitch? I won't allow it.''

``Good heavens! What does it mean?'' cried Porfiry Petrovitch,
apparently quite frightened. ``Rodion Romanovitch, my dear fellow, what
is the matter with you?''

``I won't allow it,'' Raskolnikov shouted again.

``Hush, my dear man! They'll hear and come in. Just think, what could we
say to them?'' Porfiry Petrovitch whispered in horror, bringing his face
close to Raskolnikov's.

``I won't allow it, I won't allow it,'' Raskolnikov repeated
mechanically, but he too spoke in a sudden whisper.

Porfiry turned quickly and ran to open the window.

``Some fresh air! And you must have some water, my dear fellow. You're
ill!'' and he was running to the door to call for some when he found a
decanter of water in the corner. ``Come, drink a little,'' he whispered,
rushing up to him with the decanter. ``It will be sure to do you good.''

Porfiry Petrovitch's alarm and sympathy were so natural that Raskolnikov
was silent and began looking at him with wild curiosity. He did not take
the water, however.

``Rodion Romanovitch, my dear fellow, you'll drive yourself out of your
mind, I assure you, ach, ach! Have some water, do drink a little.''

He forced him to take the glass. Raskolnikov raised it mechanically to
his lips, but set it on the table again with disgust.

``Yes, you've had a little attack! You'll bring back your illness again,
my dear fellow,'' Porfiry Petrovitch cackled with friendly sympathy,
though he still looked rather disconcerted. ``Good heavens, you must
take more care of yourself! Dmitri Prokofitch was here, came to see me
yesterday---I know, I know, I've a nasty, ironical temper, but what they
made of it!\ldots{} Good heavens, he came yesterday after you'd been. We
dined and he talked and talked away, and I could only throw up my hands
in despair! Did he come from you? But do sit down, for mercy's sake, sit
down!''

``No, not from me, but I knew he went to you and why he went,''
Raskolnikov answered sharply.

``You knew?''

``I knew. What of it?''

``Why this, Rodion Romanovitch, that I know more than that about you; I
know about everything. I know how you went \emph{to take a flat} at
night when it was dark and how you rang the bell and asked about the
blood, so that the workmen and the porter did not know what to make of
it. Yes, I understand your state of mind at that time\ldots{} but you'll
drive yourself mad like that, upon my word! You'll lose your head!
You're full of generous indignation at the wrongs you've received, first
from destiny, and then from the police officers, and so you rush from
one thing to another to force them to speak out and make an end of it
all, because you are sick of all this suspicion and foolishness. That's
so, isn't it? I have guessed how you feel, haven't I? Only in that way
you'll lose your head and Razumihin's, too; he's too \emph{good} a man
for such a position, you must know that. You are ill and he is good and
your illness is infectious for him\ldots{} I'll tell you about it when
you are more yourself\ldots{}. But do sit down, for goodness' sake.
Please rest, you look shocking, do sit down.''

Raskolnikov sat down; he no longer shivered, he was hot all over. In
amazement he listened with strained attention to Porfiry Petrovitch who
still seemed frightened as he looked after him with friendly solicitude.
But he did not believe a word he said, though he felt a strange
inclination to believe. Porfiry's unexpected words about the flat had
utterly overwhelmed him. ``How can it be, he knows about the flat
then,'' he thought suddenly, ``and he tells it me himself!''

``Yes, in our legal practice there was a case almost exactly similar, a
case of morbid psychology,'' Porfiry went on quickly. ``A man confessed
to murder and how he kept it up! It was a regular hallucination; he
brought forward facts, he imposed upon everyone and why? He had been
partly, but only partly, unintentionally the cause of a murder and when
he knew that he had given the murderers the opportunity, he sank into
dejection, it got on his mind and turned his brain, he began imagining
things and he persuaded himself that he was the murderer. But at last
the High Court of Appeal went into it and the poor fellow was acquitted
and put under proper care. Thanks to the Court of Appeal! Tut-tut-tut!
Why, my dear fellow, you may drive yourself into delirium if you have
the impulse to work upon your nerves, to go ringing bells at night and
asking about blood! I've studied all this morbid psychology in my
practice. A man is sometimes tempted to jump out of a window or from a
belfry. Just the same with bell-ringing\ldots{}. It's all illness,
Rodion Romanovitch! You have begun to neglect your illness. You should
consult an experienced doctor, what's the good of that fat fellow? You
are lightheaded! You were delirious when you did all this!''

For a moment Raskolnikov felt everything going round.

``Is it possible, is it possible,'' flashed through his mind, ``that he
is still lying? He can't be, he can't be.'' He rejected that idea,
feeling to what a degree of fury it might drive him, feeling that that
fury might drive him mad.

``I was not delirious. I knew what I was doing,'' he cried, straining
every faculty to penetrate Porfiry's game, ``I was quite myself, do you
hear?''

``Yes, I hear and understand. You said yesterday you were not delirious,
you were particularly emphatic about it! I understand all you can tell
me! A-ach!\ldots{} Listen, Rodion Romanovitch, my dear fellow. If you
were actually a criminal, or were somehow mixed up in this damnable
business, would you insist that you were not delirious but in full
possession of your faculties? And so emphatically and persistently?
Would it be possible? Quite impossible, to my thinking. If you had
anything on your conscience, you certainly ought to insist that you were
delirious. That's so, isn't it?''

There was a note of slyness in this inquiry. Raskolnikov drew back on
the sofa as Porfiry bent over him and stared in silent perplexity at
him.

``Another thing about Razumihin---you certainly ought to have said that
he came of his own accord, to have concealed your part in it! But you
don't conceal it! You lay stress on his coming at your instigation.''

Raskolnikov had not done so. A chill went down his back.

``You keep telling lies,'' he said slowly and weakly, twisting his lips
into a sickly smile, ``you are trying again to show that you know all my
game, that you know all I shall say beforehand,'' he said, conscious
himself that he was not weighing his words as he ought. ``You want to
frighten me\ldots{} or you are simply laughing at me\ldots{}''

He still stared at him as he said this and again there was a light of
intense hatred in his eyes.

``You keep lying,'' he said. ``You know perfectly well that the best
policy for the criminal is to tell the truth as nearly as
possible\ldots{} to conceal as little as possible. I don't believe
you!''

``What a wily person you are!'' Porfiry tittered, ``there's no catching
you; you've a perfect monomania. So you don't believe me? But still you
do believe me, you believe a quarter; I'll soon make you believe the
whole, because I have a sincere liking for you and genuinely wish you
good.''

Raskolnikov's lips trembled.

``Yes, I do,'' went on Porfiry, touching Raskolnikov's arm genially,
``you must take care of your illness. Besides, your mother and sister
are here now; you must think of them. You must soothe and comfort them
and you do nothing but frighten them\ldots{}''

``What has that to do with you? How do you know it? What concern is it
of yours? You are keeping watch on me and want to let me know it?''

``Good heavens! Why, I learnt it all from you yourself! You don't notice
that in your excitement you tell me and others everything. From
Razumihin, too, I learnt a number of interesting details yesterday. No,
you interrupted me, but I must tell you that, for all your wit, your
suspiciousness makes you lose the common-sense view of things. To return
to bell-ringing, for instance. I, an examining lawyer, have betrayed a
precious thing like that, a real fact (for it is a fact worth having),
and you see nothing in it! Why, if I had the slightest suspicion of you,
should I have acted like that? No, I should first have disarmed your
suspicions and not let you see I knew of that fact, should have diverted
your attention and suddenly have dealt you a knock-down blow (your
expression) saying: `And what were you doing, sir, pray, at ten or
nearly eleven at the murdered woman's flat and why did you ring the bell
and why did you ask about blood? And why did you invite the porters to
go with you to the police station, to the lieutenant?' That's how I
ought to have acted if I had a grain of suspicion of you. I ought to
have taken your evidence in due form, searched your lodging and perhaps
have arrested you, too\ldots{} so I have no suspicion of you, since I
have not done that! But you can't look at it normally and you see
nothing, I say again.''

Raskolnikov started so that Porfiry Petrovitch could not fail to
perceive it.

``You are lying all the while,'' he cried, ``I don't know your object,
but you are lying. You did not speak like that just now and I cannot be
mistaken!''

``I am lying?'' Porfiry repeated, apparently incensed, but preserving a
good-humoured and ironical face, as though he were not in the least
concerned at Raskolnikov's opinion of him. ``I am lying\ldots{} but how
did I treat you just now, I, the examining lawyer? Prompting you and
giving you every means for your defence; illness, I said, delirium,
injury, melancholy and the police officers and all the rest of it? Ah!
He-he-he! Though, indeed, all those psychological means of defence are
not very reliable and cut both ways: illness, delirium, I don't
remember---that's all right, but why, my good sir, in your illness and
in your delirium were you haunted by just those delusions and not by any
others? There may have been others, eh? He-he-he!''

Raskolnikov looked haughtily and contemptuously at him.

``Briefly,'' he said loudly and imperiously, rising to his feet and in
so doing pushing Porfiry back a little, ``briefly, I want to know, do
you acknowledge me perfectly free from suspicion or not? Tell me,
Porfiry Petrovitch, tell me once for all and make haste!''

``What a business I'm having with you!'' cried Porfiry with a perfectly
good-humoured, sly and composed face. ``And why do you want to know, why
do you want to know so much, since they haven't begun to worry you? Why,
you are like a child asking for matches! And why are you so uneasy? Why
do you force yourself upon us, eh? He-he-he!''

``I repeat,'' Raskolnikov cried furiously, ``that I can't put up with
it!''

``With what? Uncertainty?'' interrupted Porfiry.

``Don't jeer at me! I won't have it! I tell you I won't have it. I can't
and I won't, do you hear, do you hear?'' he shouted, bringing his fist
down on the table again.

``Hush! Hush! They'll overhear! I warn you seriously, take care of
yourself. I am not joking,'' Porfiry whispered, but this time there was
not the look of old womanish good nature and alarm in his face. Now he
was peremptory, stern, frowning and for once laying aside all
mystification.

But this was only for an instant. Raskolnikov, bewildered, suddenly fell
into actual frenzy, but, strange to say, he again obeyed the command to
speak quietly, though he was in a perfect paroxysm of fury.

``I will not allow myself to be tortured,'' he whispered, instantly
recognising with hatred that he could not help obeying the command and
driven to even greater fury by the thought. ``Arrest me, search me, but
kindly act in due form and don't play with me! Don't dare!''

``Don't worry about the form,'' Porfiry interrupted with the same sly
smile, as it were, gloating with enjoyment over Raskolnikov. ``I invited
you to see me quite in a friendly way.''

``I don't want your friendship and I spit on it! Do you hear? And, here,
I take my cap and go. What will you say now if you mean to arrest me?''

He took up his cap and went to the door.

``And won't you see my little surprise?'' chuckled Porfiry, again taking
him by the arm and stopping him at the door.

He seemed to become more playful and good-humoured which maddened
Raskolnikov.

``What surprise?'' he asked, standing still and looking at Porfiry in
alarm.

``My little surprise, it's sitting there behind the door, he-he-he!''
(He pointed to the locked door.) ``I locked him in that he should not
escape.''

``What is it? Where? What?\ldots{}''

Raskolnikov walked to the door and would have opened it, but it was
locked.

``It's locked, here is the key!''

And he brought a key out of his pocket.

``You are lying,'' roared Raskolnikov without restraint, ``you lie, you
damned punchinello!'' and he rushed at Porfiry who retreated to the
other door, not at all alarmed.

``I understand it all! You are lying and mocking so that I may betray
myself to you\ldots{}''

``Why, you could not betray yourself any further, my dear Rodion
Romanovitch. You are in a passion. Don't shout, I shall call the
clerks.''

``You are lying! Call the clerks! You knew I was ill and tried to work
me into a frenzy to make me betray myself, that was your object! Produce
your facts! I understand it all. You've no evidence, you have only
wretched rubbishly suspicions like Zametov's! You knew my character, you
wanted to drive me to fury and then to knock me down with priests and
deputies\ldots{}. Are you waiting for them? eh! What are you waiting
for? Where are they? Produce them?''

``Why deputies, my good man? What things people will imagine! And to do
so would not be acting in form as you say, you don't know the business,
my dear fellow\ldots{}. And there's no escaping form, as you see,''
Porfiry muttered, listening at the door through which a noise could be
heard.

``Ah, they're coming,'' cried Raskolnikov. ``You've sent for them! You
expected them! Well, produce them all: your deputies, your witnesses,
what you like!\ldots{} I am ready!''

But at this moment a strange incident occurred, something so unexpected
that neither Raskolnikov nor Porfiry Petrovitch could have looked for
such a conclusion to their interview.

\protect\hypertarget{link2HCH0026}{}{}

\mychapter{6}{Chapter VI}\label{chapter-vi-3}

When he remembered the scene afterwards, this is how Raskolnikov saw it.

The noise behind the door increased, and suddenly the door was opened a
little.

``What is it?'' cried Porfiry Petrovitch, annoyed. ``Why, I gave
orders\ldots{}''

For an instant there was no answer, but it was evident that there were
several persons at the door, and that they were apparently pushing
somebody back.

``What is it?'' Porfiry Petrovitch repeated, uneasily.

``The prisoner Nikolay has been brought,'' someone answered.

``He is not wanted! Take him away! Let him wait! What's he doing here?
How irregular!'' cried Porfiry, rushing to the door.

``But he\ldots{}'' began the same voice, and suddenly ceased.

Two seconds, not more, were spent in actual struggle, then someone gave
a violent shove, and then a man, very pale, strode into the room.

This man's appearance was at first sight very strange. He stared
straight before him, as though seeing nothing. There was a determined
gleam in his eyes; at the same time there was a deathly pallor in his
face, as though he were being led to the scaffold. His white lips were
faintly twitching.

He was dressed like a workman and was of medium height, very young,
slim, his hair cut in round crop, with thin spare features. The man whom
he had thrust back followed him into the room and succeeded in seizing
him by the shoulder; he was a warder; but Nikolay pulled his arm away.

Several persons crowded inquisitively into the doorway. Some of them
tried to get in. All this took place almost instantaneously.

``Go away, it's too soon! Wait till you are sent for!\ldots{} Why have
you brought him so soon?'' Porfiry Petrovitch muttered, extremely
annoyed, and as it were thrown out of his reckoning.

But Nikolay suddenly knelt down.

``What's the matter?'' cried Porfiry, surprised.

``I am guilty! Mine is the sin! I am the murderer,'' Nikolay articulated
suddenly, rather breathless, but speaking fairly loudly.

For ten seconds there was silence as though all had been struck dumb;
even the warder stepped back, mechanically retreated to the door, and
stood immovable.

``What is it?'' cried Porfiry Petrovitch, recovering from his momentary
stupefaction.

``I\ldots{} am the murderer,'' repeated Nikolay, after a brief pause.

``What\ldots{} you\ldots{} what\ldots{} whom did you kill?'' Porfiry
Petrovitch was obviously bewildered.

Nikolay again was silent for a moment.

``Alyona Ivanovna and her sister Lizaveta Ivanovna, I\ldots{}
killed\ldots{} with an axe. Darkness came over me,'' he added suddenly,
and was again silent.

He still remained on his knees. Porfiry Petrovitch stood for some
moments as though meditating, but suddenly roused himself and waved back
the uninvited spectators. They instantly vanished and closed the door.
Then he looked towards Raskolnikov, who was standing in the corner,
staring wildly at Nikolay and moved towards him, but stopped short,
looked from Nikolay to Raskolnikov and then again at Nikolay, and
seeming unable to restrain himself darted at the latter.

``You're in too great a hurry,'' he shouted at him, almost angrily. ``I
didn't ask you what came over you\ldots{}. Speak, did you kill them?''

``I am the murderer\ldots{}. I want to give evidence,'' Nikolay
pronounced.

``Ach! What did you kill them with?''

``An axe. I had it ready.''

``Ach, he is in a hurry! Alone?''

Nikolay did not understand the question.

``Did you do it alone?''

``Yes, alone. And Mitka is not guilty and had no share in it.''

``Don't be in a hurry about Mitka! A-ach! How was it you ran downstairs
like that at the time? The porters met you both!''

``It was to put them off the scent\ldots{} I ran after Mitka,'' Nikolay
replied hurriedly, as though he had prepared the answer.

``I knew it!'' cried Porfiry, with vexation. ``It's not his own tale he
is telling,'' he muttered as though to himself, and suddenly his eyes
rested on Raskolnikov again.

He was apparently so taken up with Nikolay that for a moment he had
forgotten Raskolnikov. He was a little taken aback.

``My dear Rodion Romanovitch, excuse me!'' he flew up to him, ``this
won't do; I'm afraid you must go\ldots{} it's no good your
staying\ldots{} I will\ldots{} you see, what a surprise!\ldots{}
Good-bye!''

And taking him by the arm, he showed him to the door.

``I suppose you didn't expect it?'' said Raskolnikov who, though he had
not yet fully grasped the situation, had regained his courage.

``You did not expect it either, my friend. See how your hand is
trembling! He-he!''

``You're trembling, too, Porfiry Petrovitch!''

``Yes, I am; I didn't expect it.''

They were already at the door; Porfiry was impatient for Raskolnikov to
be gone.

``And your little surprise, aren't you going to show it to me?''
Raskolnikov said, sarcastically.

``Why, his teeth are chattering as he asks, he-he! You are an ironical
person! Come, till we meet!''

``I believe we can say \emph{good-bye}!''

``That's in God's hands,'' muttered Porfiry, with an unnatural smile.

As he walked through the office, Raskolnikov noticed that many people
were looking at him. Among them he saw the two porters from \emph{the}
house, whom he had invited that night to the police station. They stood
there waiting. But he was no sooner on the stairs than he heard the
voice of Porfiry Petrovitch behind him. Turning round, he saw the latter
running after him, out of breath.

``One word, Rodion Romanovitch; as to all the rest, it's in God's hands,
but as a matter of form there are some questions I shall have to ask
you\ldots{} so we shall meet again, shan't we?''

And Porfiry stood still, facing him with a smile.

``Shan't we?'' he added again.

He seemed to want to say something more, but could not speak out.

``You must forgive me, Porfiry Petrovitch, for what has just
passed\ldots{} I lost my temper,'' began Raskolnikov, who had so far
regained his courage that he felt irresistibly inclined to display his
coolness.

``Don't mention it, don't mention it,'' Porfiry replied, almost
gleefully. ``I myself, too\ldots{} I have a wicked temper, I admit it!
But we shall meet again. If it's God's will, we may see a great deal of
one another.''

``And will get to know each other through and through?'' added
Raskolnikov.

``Yes; know each other through and through,'' assented Porfiry
Petrovitch, and he screwed up his eyes, looking earnestly at
Raskolnikov. ``Now you're going to a birthday party?''

``To a funeral.''

``Of course, the funeral! Take care of yourself, and get well.''

``I don't know what to wish you,'' said Raskolnikov, who had begun to
descend the stairs, but looked back again. ``I should like to wish you
success, but your office is such a comical one.''

``Why comical?'' Porfiry Petrovitch had turned to go, but he seemed to
prick up his ears at this.

``Why, how you must have been torturing and harassing that poor Nikolay
psychologically, after your fashion, till he confessed! You must have
been at him day and night, proving to him that he was the murderer, and
now that he has confessed, you'll begin vivisecting him again. `You are
lying,' you'll say. `You are not the murderer! You can't be! It's not
your own tale you are telling!' You must admit it's a comical
business!''

``He-he-he! You noticed then that I said to Nikolay just now that it was
not his own tale he was telling?''

``How could I help noticing it!''

``He-he! You are quick-witted. You notice everything! You've really a
playful mind! And you always fasten on the comic side\ldots{} he-he!
They say that was the marked characteristic of Gogol, among the
writers.''

``Yes, of Gogol.''

``Yes, of Gogol\ldots{}. I shall look forward to meeting you.''

``So shall I.''

Raskolnikov walked straight home. He was so muddled and bewildered that
on getting home he sat for a quarter of an hour on the sofa, trying to
collect his thoughts. He did not attempt to think about Nikolay; he was
stupefied; he felt that his confession was something inexplicable,
amazing---something beyond his understanding. But Nikolay's confession
was an actual fact. The consequences of this fact were clear to him at
once, its falsehood could not fail to be discovered, and then they would
be after him again. Till then, at least, he was free and must do
something for himself, for the danger was imminent.

But how imminent? His position gradually became clear to him.
Remembering, sketchily, the main outlines of his recent scene with
Porfiry, he could not help shuddering again with horror. Of course, he
did not yet know all Porfiry's aims, he could not see into all his
calculations. But he had already partly shown his hand, and no one knew
better than Raskolnikov how terrible Porfiry's ``lead'' had been for
him. A little more and he \emph{might} have given himself away
completely, circumstantially. Knowing his nervous temperament and from
the first glance seeing through him, Porfiry, though playing a bold
game, was bound to win. There's no denying that Raskolnikov had
compromised himself seriously, but no \emph{facts} had come to light as
yet; there was nothing positive. But was he taking a true view of the
position? Wasn't he mistaken? What had Porfiry been trying to get at?
Had he really some surprise prepared for him? And what was it? Had he
really been expecting something or not? How would they have parted if it
had not been for the unexpected appearance of Nikolay?

Porfiry had shown almost all his cards---of course, he had risked
something in showing them---and if he had really had anything up his
sleeve (Raskolnikov reflected), he would have shown that, too. What was
that ``surprise''? Was it a joke? Had it meant anything? Could it have
concealed anything like a fact, a piece of positive evidence? His
yesterday's visitor? What had become of him? Where was he to-day? If
Porfiry really had any evidence, it must be connected with him\ldots{}.

He sat on the sofa with his elbows on his knees and his face hidden in
his hands. He was still shivering nervously. At last he got up, took his
cap, thought a minute, and went to the door.

He had a sort of presentiment that for to-day, at least, he might
consider himself out of danger. He had a sudden sense almost of joy; he
wanted to make haste to Katerina Ivanovna's. He would be too late for
the funeral, of course, but he would be in time for the memorial dinner,
and there at once he would see Sonia.

He stood still, thought a moment, and a suffering smile came for a
moment on to his lips.

``To-day! To-day,'' he repeated to himself. ``Yes, to-day! So it must
be\ldots{}.''

But as he was about to open the door, it began opening of itself. He
started and moved back. The door opened gently and slowly, and there
suddenly appeared a figure---yesterday's visitor \emph{from
underground}.

The man stood in the doorway, looked at Raskolnikov without speaking,
and took a step forward into the room. He was exactly the same as
yesterday; the same figure, the same dress, but there was a great change
in his face; he looked dejected and sighed deeply. If he had only put
his hand up to his cheek and leaned his head on one side he would have
looked exactly like a peasant woman.

``What do you want?'' asked Raskolnikov, numb with terror. The man was
still silent, but suddenly he bowed down almost to the ground, touching
it with his finger.

``What is it?'' cried Raskolnikov.

``I have sinned,'' the man articulated softly.

``How?''

``By evil thoughts.''

They looked at one another.

``I was vexed. When you came, perhaps in drink, and bade the porters go
to the police station and asked about the blood, I was vexed that they
let you go and took you for drunken. I was so vexed that I lost my
sleep. And remembering the address we came here yesterday and asked for
you\ldots{}.''

``Who came?'' Raskolnikov interrupted, instantly beginning to recollect.

``I did, I've wronged you.''

``Then you come from that house?''

``I was standing at the gate with them\ldots{} don't you remember? We
have carried on our trade in that house for years past. We cure and
prepare hides, we take work home\ldots{} most of all I was
vexed\ldots{}.''

And the whole scene of the day before yesterday in the gateway came
clearly before Raskolnikov's mind; he recollected that there had been
several people there besides the porters, women among them. He
remembered one voice had suggested taking him straight to the
police-station. He could not recall the face of the speaker, and even
now he did not recognise it, but he remembered that he had turned round
and made him some answer\ldots{}.

So this was the solution of yesterday's horror. The most awful thought
was that he had been actually almost lost, had almost done for himself
on account of such a \emph{trivial} circumstance. So this man could tell
nothing except his asking about the flat and the blood stains. So
Porfiry, too, had nothing but that \emph{delirium}, no facts but this
\emph{psychology} which \emph{cuts both ways}, nothing positive. So if
no more facts come to light (and they must not, they must not!)
then\ldots{} then what can they do to him? How can they convict him,
even if they arrest him? And Porfiry then had only just heard about the
flat and had not known about it before.

``Was it you who told Porfiry\ldots{} that I'd been there?'' he cried,
struck by a sudden idea.

``What Porfiry?''

``The head of the detective department?''

``Yes. The porters did not go there, but I went.''

``To-day?''

``I got there two minutes before you. And I heard, I heard it all, how
he worried you.''

``Where? What? When?''

``Why, in the next room. I was sitting there all the time.''

``What? Why, then you were the surprise? But how could it happen? Upon
my word!''

``I saw that the porters did not want to do what I said,'' began the
man; ``for it's too late, said they, and maybe he'll be angry that we
did not come at the time. I was vexed and I lost my sleep, and I began
making inquiries. And finding out yesterday where to go, I went to-day.
The first time I went he wasn't there, when I came an hour later he
couldn't see me. I went the third time, and they showed me in. I
informed him of everything, just as it happened, and he began skipping
about the room and punching himself on the chest. `What do you
scoundrels mean by it? If I'd known about it I should have arrested
him!' Then he ran out, called somebody and began talking to him in the
corner, then he turned to me, scolding and questioning me. He scolded me
a great deal; and I told him everything, and I told him that you didn't
dare to say a word in answer to me yesterday and that you didn't
recognise me. And he fell to running about again and kept hitting
himself on the chest, and getting angry and running about, and when you
were announced he told me to go into the next room. `Sit there a bit,'
he said. `Don't move, whatever you may hear.' And he set a chair there
for me and locked me in. `Perhaps,' he said, `I may call you.' And when
Nikolay'd been brought he let me out as soon as you were gone. `I shall
send for you again and question you,' he said.''

``And did he question Nikolay while you were there?''

``He got rid of me as he did of you, before he spoke to Nikolay.''

The man stood still, and again suddenly bowed down, touching the ground
with his finger.

``Forgive me for my evil thoughts, and my slander.''

``May God forgive you,'' answered Raskolnikov.

And as he said this, the man bowed down again, but not to the ground,
turned slowly and went out of the room.

``It all cuts both ways, now it all cuts both ways,'' repeated
Raskolnikov, and he went out more confident than ever.

``Now we'll make a fight for it,'' he said, with a malicious smile, as
he went down the stairs. His malice was aimed at himself; with shame and
contempt he recollected his ``cowardice.''

\part*{PART V}
\addcontentsline{toc}{part}{PART V}

\mychapter{1}{Chapter I}\label{chapter-i-4}

The morning that followed the fateful interview with Dounia and her
mother brought sobering influences to bear on Pyotr Petrovitch.
Intensely unpleasant as it was, he was forced little by little to accept
as a fact beyond recall what had seemed to him only the day before
fantastic and incredible. The black snake of wounded vanity had been
gnawing at his heart all night. When he got out of bed, Pyotr Petrovitch
immediately looked in the looking-glass. He was afraid that he had
jaundice. However his health seemed unimpaired so far, and looking at
his noble, clear-skinned countenance which had grown fattish of late,
Pyotr Petrovitch for an instant was positively comforted in the
conviction that he would find another bride and, perhaps, even a better
one. But coming back to the sense of his present position, he turned
aside and spat vigorously, which excited a sarcastic smile in Andrey
Semyonovitch Lebeziatnikov, the young friend with whom he was staying.
That smile Pyotr Petrovitch noticed, and at once set it down against his
young friend's account. He had set down a good many points against him
of late. His anger was redoubled when he reflected that he ought not to
have told Andrey Semyonovitch about the result of yesterday's interview.
That was the second mistake he had made in temper, through impulsiveness
and irritability\ldots{}. Moreover, all that morning one unpleasantness
followed another. He even found a hitch awaiting him in his legal case
in the senate. He was particularly irritated by the owner of the flat
which had been taken in view of his approaching marriage and was being
redecorated at his own expense; the owner, a rich German tradesman,
would not entertain the idea of breaking the contract which had just
been signed and insisted on the full forfeit money, though Pyotr
Petrovitch would be giving him back the flat practically redecorated. In
the same way the upholsterers refused to return a single rouble of the
instalment paid for the furniture purchased but not yet removed to the
flat.

``Am I to get married simply for the sake of the furniture?'' Pyotr
Petrovitch ground his teeth and at the same time once more he had a
gleam of desperate hope. ``Can all that be really so irrevocably over?
Is it no use to make another effort?'' The thought of Dounia sent a
voluptuous pang through his heart. He endured anguish at that moment,
and if it had been possible to slay Raskolnikov instantly by wishing it,
Pyotr Petrovitch would promptly have uttered the wish.

``It was my mistake, too, not to have given them money,'' he thought, as
he returned dejectedly to Lebeziatnikov's room, ``and why on earth was I
such a Jew? It was false economy! I meant to keep them without a penny
so that they should turn to me as their providence, and look at them!
foo! If I'd spent some fifteen hundred roubles on them for the trousseau
and presents, on knick-knacks, dressing-cases, jewellery, materials, and
all that sort of trash from Knopp's and the English shop, my position
would have been better and\ldots{} stronger! They could not have refused
me so easily! They are the sort of people that would feel bound to
return money and presents if they broke it off; and they would find it
hard to do it! And their conscience would prick them: how can we dismiss
a man who has hitherto been so generous and delicate?\ldots{}. H'm! I've
made a blunder.''

And grinding his teeth again, Pyotr Petrovitch called himself a
fool---but not aloud, of course.

He returned home, twice as irritated and angry as before. The
preparations for the funeral dinner at Katerina Ivanovna's excited his
curiosity as he passed. He had heard about it the day before; he
fancied, indeed, that he had been invited, but absorbed in his own cares
he had paid no attention. Inquiring of Madame Lippevechsel who was busy
laying the table while Katerina Ivanovna was away at the cemetery, he
heard that the entertainment was to be a great affair, that all the
lodgers had been invited, among them some who had not known the dead
man, that even Andrey Semyonovitch Lebeziatnikov was invited in spite of
his previous quarrel with Katerina Ivanovna, that he, Pyotr Petrovitch,
was not only invited, but was eagerly expected as he was the most
important of the lodgers. Amalia Ivanovna herself had been invited with
great ceremony in spite of the recent unpleasantness, and so she was
very busy with preparations and was taking a positive pleasure in them;
she was moreover dressed up to the nines, all in new black silk, and she
was proud of it. All this suggested an idea to Pyotr Petrovitch and he
went into his room, or rather Lebeziatnikov's, somewhat thoughtful. He
had learnt that Raskolnikov was to be one of the guests.

Andrey Semyonovitch had been at home all the morning. The attitude of
Pyotr Petrovitch to this gentleman was strange, though perhaps natural.
Pyotr Petrovitch had despised and hated him from the day he came to stay
with him and at the same time he seemed somewhat afraid of him. He had
not come to stay with him on his arrival in Petersburg simply from
parsimony, though that had been perhaps his chief object. He had heard
of Andrey Semyonovitch, who had once been his ward, as a leading young
progressive who was taking an important part in certain interesting
circles, the doings of which were a legend in the provinces. It had
impressed Pyotr Petrovitch. These powerful omniscient circles who
despised everyone and showed everyone up had long inspired in him a
peculiar but quite vague alarm. He had not, of course, been able to form
even an approximate notion of what they meant. He, like everyone, had
heard that there were, especially in Petersburg, progressives of some
sort, nihilists and so on, and, like many people, he exaggerated and
distorted the significance of those words to an absurd degree. What for
many years past he had feared more than anything was \emph{being shown
up} and this was the chief ground for his continual uneasiness at the
thought of transferring his business to Petersburg. He was afraid of
this as little children are sometimes panic-stricken. Some years before,
when he was just entering on his own career, he had come upon two cases
in which rather important personages in the province, patrons of his,
had been cruelly shown up. One instance had ended in great scandal for
the person attacked and the other had very nearly ended in serious
trouble. For this reason Pyotr Petrovitch intended to go into the
subject as soon as he reached Petersburg and, if necessary, to
anticipate contingencies by seeking the favour of ``our younger
generation.'' He relied on Andrey Semyonovitch for this and before his
visit to Raskolnikov he had succeeded in picking up some current
phrases. He soon discovered that Andrey Semyonovitch was a commonplace
simpleton, but that by no means reassured Pyotr Petrovitch. Even if he
had been certain that all the progressives were fools like him, it would
not have allayed his uneasiness. All the doctrines, the ideas, the
systems, with which Andrey Semyonovitch pestered him had no interest for
him. He had his own object---he simply wanted to find out at once what
was happening \emph{here}. Had these people any power or not? Had he
anything to fear from them? Would they expose any enterprise of his? And
what precisely was now the object of their attacks? Could he somehow
make up to them and get round them if they really were powerful? Was
this the thing to do or not? Couldn't he gain something through them? In
fact hundreds of questions presented themselves.

Andrey Semyonovitch was an anæmic, scrofulous little man, with strangely
flaxen mutton-chop whiskers of which he was very proud. He was a clerk
and had almost always something wrong with his eyes. He was rather
soft-hearted, but self-confident and sometimes extremely conceited in
speech, which had an absurd effect, incongruous with his little figure.
He was one of the lodgers most respected by Amalia Ivanovna, for he did
not get drunk and paid regularly for his lodgings. Andrey Semyonovitch
really was rather stupid; he attached himself to the cause of progress
and ``our younger generation'' from enthusiasm. He was one of the
numerous and varied legion of dullards, of half-animate abortions,
conceited, half-educated coxcombs, who attach themselves to the idea
most in fashion only to vulgarise it and who caricature every cause they
serve, however sincerely.

Though Lebeziatnikov was so good-natured, he, too, was beginning to
dislike Pyotr Petrovitch. This happened on both sides unconsciously.
However simple Andrey Semyonovitch might be, he began to see that Pyotr
Petrovitch was duping him and secretly despising him, and that ``he was
not the right sort of man.'' He had tried expounding to him the system
of Fourier and the Darwinian theory, but of late Pyotr Petrovitch began
to listen too sarcastically and even to be rude. The fact was he had
begun instinctively to guess that Lebeziatnikov was not merely a
commonplace simpleton, but, perhaps, a liar, too, and that he had no
connections of any consequence even in his own circle, but had simply
picked things up third-hand; and that very likely he did not even know
much about his own work of propaganda, for he was in too great a muddle.
A fine person he would be to show anyone up! It must be noted, by the
way, that Pyotr Petrovitch had during those ten days eagerly accepted
the strangest praise from Andrey Semyonovitch; he had not protested, for
instance, when Andrey Semyonovitch belauded him for being ready to
contribute to the establishment of the new ``commune,'' or to abstain
from christening his future children, or to acquiesce if Dounia were to
take a lover a month after marriage, and so on. Pyotr Petrovitch so
enjoyed hearing his own praises that he did not disdain even such
virtues when they were attributed to him.

Pyotr Petrovitch had had occasion that morning to realise some
five-per-cent bonds and now he sat down to the table and counted over
bundles of notes. Andrey Semyonovitch who hardly ever had any money
walked about the room pretending to himself to look at all those bank
notes with indifference and even contempt. Nothing would have convinced
Pyotr Petrovitch that Andrey Semyonovitch could really look on the money
unmoved, and the latter, on his side, kept thinking bitterly that Pyotr
Petrovitch was capable of entertaining such an idea about him and was,
perhaps, glad of the opportunity of teasing his young friend by
reminding him of his inferiority and the great difference between them.

He found him incredibly inattentive and irritable, though he, Andrey
Semyonovitch, began enlarging on his favourite subject, the foundation
of a new special ``commune.'' The brief remarks that dropped from Pyotr
Petrovitch between the clicking of the beads on the reckoning frame
betrayed unmistakable and discourteous irony. But the ``humane'' Andrey
Semyonovitch ascribed Pyotr Petrovitch's ill-humour to his recent breach
with Dounia and he was burning with impatience to discourse on that
theme. He had something progressive to say on the subject which might
console his worthy friend and ``could not fail'' to promote his
development.

``There is some sort of festivity being prepared at that\ldots{} at the
widow's, isn't there?'' Pyotr Petrovitch asked suddenly, interrupting
Andrey Semyonovitch at the most interesting passage.

``Why, don't you know? Why, I was telling you last night what I think
about all such ceremonies. And she invited you too, I heard. You were
talking to her yesterday\ldots{}''

``I should never have expected that beggarly fool would have spent on
this feast all the money she got from that other fool, Raskolnikov. I
was surprised just now as I came through at the preparations there, the
wines! Several people are invited. It's beyond everything!'' continued
Pyotr Petrovitch, who seemed to have some object in pursuing the
conversation. ``What? You say I am asked too? When was that? I don't
remember. But I shan't go. Why should I? I only said a word to her in
passing yesterday of the possibility of her obtaining a year's salary as
a destitute widow of a government clerk. I suppose she has invited me on
that account, hasn't she? He-he-he!''

``I don't intend to go either,'' said Lebeziatnikov.

``I should think not, after giving her a thrashing! You might well
hesitate, he-he!''

``Who thrashed? Whom?'' cried Lebeziatnikov, flustered and blushing.

``Why, you thrashed Katerina Ivanovna a month ago. I heard so
yesterday\ldots{} so that's what your convictions amount to\ldots{} and
the woman question, too, wasn't quite sound, he-he-he!'' and Pyotr
Petrovitch, as though comforted, went back to clicking his beads.

``It's all slander and nonsense!'' cried Lebeziatnikov, who was always
afraid of allusions to the subject. ``It was not like that at all, it
was quite different. You've heard it wrong; it's a libel. I was simply
defending myself. She rushed at me first with her nails, she pulled out
all my whiskers\ldots{}. It's permissable for anyone, I should hope, to
defend himself and I never allow anyone to use violence to me on
principle, for it's an act of despotism. What was I to do? I simply
pushed her back.''

``He-he-he!'' Luzhin went on laughing maliciously.

``You keep on like that because you are out of humour yourself\ldots{}.
But that's nonsense and it has nothing, nothing whatever to do with the
woman question! You don't understand; I used to think, indeed, that if
women are equal to men in all respects, even in strength (as is
maintained now) there ought to be equality in that, too. Of course, I
reflected afterwards that such a question ought not really to arise, for
there ought not to be fighting and in the future society fighting is
unthinkable\ldots{} and that it would be a queer thing to seek for
equality in fighting. I am not so stupid\ldots{} though, of course,
there is fighting\ldots{} there won't be later, but at present there
is\ldots{} confound it! How muddled one gets with you! It's not on that
account that I am not going. I am not going on principle, not to take
part in the revolting convention of memorial dinners, that's why!
Though, of course, one might go to laugh at it\ldots{}. I am sorry there
won't be any priests at it. I should certainly go if there were.''

``Then you would sit down at another man's table and insult it and those
who invited you. Eh?''

``Certainly not insult, but protest. I should do it with a good object.
I might indirectly assist the cause of enlightenment and propaganda.
It's a duty of every man to work for enlightenment and propaganda and
the more harshly, perhaps, the better. I might drop a seed, an
idea\ldots{}. And something might grow up from that seed. How should I
be insulting them? They might be offended at first, but afterwards
they'd see I'd done them a service. You know, Terebyeva (who is in the
community now) was blamed because when she left her family and\ldots{}
devoted\ldots{} herself, she wrote to her father and mother that she
wouldn't go on living conventionally and was entering on a free marriage
and it was said that that was too harsh, that she might have spared them
and have written more kindly. I think that's all nonsense and there's no
need of softness; on the contrary, what's wanted is protest. Varents had
been married seven years, she abandoned her two children, she told her
husband straight out in a letter: `I have realised that I cannot be
happy with you. I can never forgive you that you have deceived me by
concealing from me that there is another organisation of society by
means of the communities. I have only lately learned it from a
great-hearted man to whom I have given myself and with whom I am
establishing a community. I speak plainly because I consider it
dishonest to deceive you. Do as you think best. Do not hope to get me
back, you are too late. I hope you will be happy.' That's how letters
like that ought to be written!''

``Is that Terebyeva the one you said had made a third free marriage?''

``No, it's only the second, really! But what if it were the fourth, what
if it were the fifteenth, that's all nonsense! And if ever I regretted
the death of my father and mother, it is now, and I sometimes think if
my parents were living what a protest I would have aimed at them! I
would have done something on purpose\ldots{} I would have shown them! I
would have astonished them! I am really sorry there is no one!''

``To surprise! He-he! Well, be that as you will,'' Pyotr Petrovitch
interrupted, ``but tell me this; do you know the dead man's daughter,
the delicate-looking little thing? It's true what they say about her,
isn't it?''

``What of it? I think, that is, it is my own personal conviction that
this is the normal condition of women. Why not? I mean,
\emph{distinguons}. In our present society it is not altogether normal,
because it is compulsory, but in the future society it will be perfectly
normal, because it will be voluntary. Even as it is, she was quite
right: she was suffering and that was her asset, so to speak, her
capital which she had a perfect right to dispose of. Of course, in the
future society there will be no need of assets, but her part will have
another significance, rational and in harmony with her environment. As
to Sofya Semyonovna personally, I regard her action as a vigorous
protest against the organisation of society, and I respect her deeply
for it; I rejoice indeed when I look at her!''

``I was told that you got her turned out of these lodgings.''

Lebeziatnikov was enraged.

``That's another slander,'' he yelled. ``It was not so at all! That was
all Katerina Ivanovna's invention, for she did not understand! And I
never made love to Sofya Semyonovna! I was simply developing her,
entirely disinterestedly, trying to rouse her to protest\ldots{}. All I
wanted was her protest and Sofya Semyonovna could not have remained here
anyway!''

``Have you asked her to join your community?''

``You keep on laughing and very inappropriately, allow me to tell you.
You don't understand! There is no such rôle in a community. The
community is established that there should be no such rôles. In a
community, such a rôle is essentially transformed and what is stupid
here is sensible there, what, under present conditions, is unnatural
becomes perfectly natural in the community. It all depends on the
environment. It's all the environment and man himself is nothing. And I
am on good terms with Sofya Semyonovna to this day, which is a proof
that she never regarded me as having wronged her. I am trying now to
attract her to the community, but on quite, quite a different footing.
What are you laughing at? We are trying to establish a community of our
own, a special one, on a broader basis. We have gone further in our
convictions. We reject more! And meanwhile I'm still developing Sofya
Semyonovna. She has a beautiful, beautiful character!''

``And you take advantage of her fine character, eh? He-he!''

``No, no! Oh, no! On the contrary.''

``Oh, on the contrary! He-he-he! A queer thing to say!''

``Believe me! Why should I disguise it? In fact, I feel it strange
myself how timid, chaste and modern she is with me!''

``And you, of course, are developing her\ldots{} he-he! trying to prove
to her that all that modesty is nonsense?''

``Not at all, not at all! How coarsely, how stupidly---excuse me saying
so---you misunderstand the word development! Good heavens, how\ldots{}
crude you still are! We are striving for the freedom of women and you
have only one idea in your head\ldots{}. Setting aside the general
question of chastity and feminine modesty as useless in themselves and
indeed prejudices, I fully accept her chastity with me, because that's
for her to decide. Of course if she were to tell me herself that she
wanted me, I should think myself very lucky, because I like the girl
very much; but as it is, no one has ever treated her more courteously
than I, with more respect for her dignity\ldots{} I wait in hopes,
that's all!''

``You had much better make her a present of something. I bet you never
thought of that.''

``You don't understand, as I've told you already! Of course, she is in
such a position, but it's another question. Quite another question! You
simply despise her. Seeing a fact which you mistakenly consider
deserving of contempt, you refuse to take a humane view of a fellow
creature. You don't know what a character she is! I am only sorry that
of late she has quite given up reading and borrowing books. I used to
lend them to her. I am sorry, too, that with all the energy and
resolution in protesting---which she has already shown once---she has
little self-reliance, little, so to say, independence, so as to break
free from certain prejudices and certain foolish ideas. Yet she
thoroughly understands some questions, for instance about kissing of
hands, that is, that it's an insult to a woman for a man to kiss her
hand, because it's a sign of inequality. We had a debate about it and I
described it to her. She listened attentively to an account of the
workmen's associations in France, too. Now I am explaining the question
of coming into the room in the future society.''

``And what's that, pray?''

``We had a debate lately on the question: Has a member of the community
the right to enter another member's room, whether man or woman, at any
time\ldots{} and we decided that he has!''

``It might be at an inconvenient moment, he-he!''

Lebeziatnikov was really angry.

``You are always thinking of something unpleasant,'' he cried with
aversion. ``Tfoo! How vexed I am that when I was expounding our system,
I referred prematurely to the question of personal privacy! It's always
a stumbling-block to people like you, they turn it into ridicule before
they understand it. And how proud they are of it, too! Tfoo! I've often
maintained that that question should not be approached by a novice till
he has a firm faith in the system. And tell me, please, what do you find
so shameful even in cesspools? I should be the first to be ready to
clean out any cesspool you like. And it's not a question of
self-sacrifice, it's simply work, honourable, useful work which is as
good as any other and much better than the work of a Raphael and a
Pushkin, because it is more useful.''

``And more honourable, more honourable, he-he-he!''

``What do you mean by `more honourable'? I don't understand such
expressions to describe human activity. `More honourable,'
`nobler'---all those are old-fashioned prejudices which I reject.
Everything which is \emph{of use} to mankind is honourable. I only
understand one word: \emph{useful}! You can snigger as much as you like,
but that's so!''

Pyotr Petrovitch laughed heartily. He had finished counting the money
and was putting it away. But some of the notes he left on the table. The
``cesspool question'' had already been a subject of dispute between
them. What was absurd was that it made Lebeziatnikov really angry, while
it amused Luzhin and at that moment he particularly wanted to anger his
young friend.

``It's your ill-luck yesterday that makes you so ill-humoured and
annoying,'' blurted out Lebeziatnikov, who in spite of his
``independence'' and his ``protests'' did not venture to oppose Pyotr
Petrovitch and still behaved to him with some of the respect habitual in
earlier years.

``You'd better tell me this,'' Pyotr Petrovitch interrupted with haughty
displeasure, ``can you\ldots{} or rather are you really friendly enough
with that young person to ask her to step in here for a minute? I think
they've all come back from the cemetery\ldots{} I heard the sound of
steps\ldots{} I want to see her, that young person.''

``What for?'' Lebeziatnikov asked with surprise.

``Oh, I want to. I am leaving here to-day or to-morrow and therefore I
wanted to speak to her about\ldots{} However, you may be present during
the interview. It's better you should be, indeed. For there's no knowing
what you might imagine.''

``I shan't imagine anything. I only asked and, if you've anything to say
to her, nothing is easier than to call her in. I'll go directly and you
may be sure I won't be in your way.''

Five minutes later Lebeziatnikov came in with Sonia. She came in very
much surprised and overcome with shyness as usual. She was always shy in
such circumstances and was always afraid of new people, she had been as
a child and was even more so now\ldots{}. Pyotr Petrovitch met her
``politely and affably,'' but with a certain shade of bantering
familiarity which in his opinion was suitable for a man of his
respectability and weight in dealing with a creature so young and so
\emph{interesting} as she. He hastened to ``reassure'' her and made her
sit down facing him at the table. Sonia sat down, looked about her---at
Lebeziatnikov, at the notes lying on the table and then again at Pyotr
Petrovitch and her eyes remained riveted on him. Lebeziatnikov was
moving to the door. Pyotr Petrovitch signed to Sonia to remain seated
and stopped Lebeziatnikov.

``Is Raskolnikov in there? Has he come?'' he asked him in a whisper.

``Raskolnikov? Yes. Why? Yes, he is there. I saw him just come
in\ldots{}. Why?''

``Well, I particularly beg you to remain here with us and not to leave
me alone with this\ldots{} young woman. I only want a few words with
her, but God knows what they may make of it. I shouldn't like
Raskolnikov to repeat anything\ldots{}. You understand what I mean?''

``I understand!'' Lebeziatnikov saw the point. ``Yes, you are
right\ldots{}. Of course, I am convinced personally that you have no
reason to be uneasy, but\ldots{} still, you are right. Certainly I'll
stay. I'll stand here at the window and not be in your way\ldots{} I
think you are right\ldots{}''

Pyotr Petrovitch returned to the sofa, sat down opposite Sonia, looked
attentively at her and assumed an extremely dignified, even severe
expression, as much as to say, ``don't you make any mistake, madam.''
Sonia was overwhelmed with embarrassment.

``In the first place, Sofya Semyonovna, will you make my excuses to your
respected mamma\ldots{}. That's right, isn't it? Katerina Ivanovna
stands in the place of a mother to you?'' Pyotr Petrovitch began with
great dignity, though affably.

It was evident that his intentions were friendly.

``Quite so, yes; the place of a mother,'' Sonia answered, timidly and
hurriedly.

``Then will you make my apologies to her? Through inevitable
circumstances I am forced to be absent and shall not be at the dinner in
spite of your mamma's kind invitation.''

``Yes\ldots{} I'll tell her\ldots{} at once.''

And Sonia hastily jumped up from her seat.

``Wait, that's not all,'' Pyotr Petrovitch detained her, smiling at her
simplicity and ignorance of good manners, ``and you know me little, my
dear Sofya Semyonovna, if you suppose I would have ventured to trouble a
person like you for a matter of so little consequence affecting myself
only. I have another object.''

Sonia sat down hurriedly. Her eyes rested again for an instant on the
grey-and-rainbow-coloured notes that remained on the table, but she
quickly looked away and fixed her eyes on Pyotr Petrovitch. She felt it
horribly indecorous, especially for \emph{her}, to look at another
person's money. She stared at the gold eye-glass which Pyotr Petrovitch
held in his left hand and at the massive and extremely handsome ring
with a yellow stone on his middle finger. But suddenly she looked away
and, not knowing where to turn, ended by staring Pyotr Petrovitch again
straight in the face. After a pause of still greater dignity he
continued.

``I chanced yesterday in passing to exchange a couple of words with
Katerina Ivanovna, poor woman. That was sufficient to enable me to
ascertain that she is in a position---preternatural, if one may so
express it.''

``Yes\ldots{} preternatural\ldots{}'' Sonia hurriedly assented.

``Or it would be simpler and more comprehensible to say, ill.''

``Yes, simpler and more comprehen\ldots{} yes, ill.''

``Quite so. So then from a feeling of humanity and so to speak
compassion, I should be glad to be of service to her in any way,
foreseeing her unfortunate position. I believe the whole of this
poverty-stricken family depends now entirely on you?''

``Allow me to ask,'' Sonia rose to her feet, ``did you say something to
her yesterday of the possibility of a pension? Because she told me you
had undertaken to get her one. Was that true?''

``Not in the slightest, and indeed it's an absurdity! I merely hinted at
her obtaining temporary assistance as the widow of an official who had
died in the service---if only she has patronage\ldots{} but apparently
your late parent had not served his full term and had not indeed been in
the service at all of late. In fact, if there could be any hope, it
would be very ephemeral, because there would be no claim for assistance
in that case, far from it\ldots{}. And she is dreaming of a pension
already, he-he-he!\ldots{} A go-ahead lady!''

``Yes, she is. For she is credulous and good-hearted, and she believes
everything from the goodness of her heart and\ldots{} and\ldots{} and
she is like that\ldots{} yes\ldots{} You must excuse her,'' said Sonia,
and again she got up to go.

``But you haven't heard what I have to say.''

``No, I haven't heard,'' muttered Sonia.

``Then sit down.'' She was terribly confused; she sat down again a third
time.

``Seeing her position with her unfortunate little ones, I should be
glad, as I have said before, so far as lies in my power, to be of
service, that is, so far as is in my power, not more. One might for
instance get up a subscription for her, or a lottery, something of the
sort, such as is always arranged in such cases by friends or even
outsiders desirous of assisting people. It was of that I intended to
speak to you; it might be done.''

``Yes, yes\ldots{} God will repay you for it,'' faltered Sonia, gazing
intently at Pyotr Petrovitch.

``It might be, but we will talk of it later. We might begin it to-day,
we will talk it over this evening and lay the foundation so to speak.
Come to me at seven o'clock. Mr. Lebeziatnikov, I hope, will assist us.
But there is one circumstance of which I ought to warn you beforehand
and for which I venture to trouble you, Sofya Semyonovna, to come here.
In my opinion money cannot be, indeed it's unsafe to put it into
Katerina Ivanovna's own hands. The dinner to-day is a proof of that.
Though she has not, so to speak, a crust of bread for to-morrow
and\ldots{} well, boots or shoes, or anything; she has bought to-day
Jamaica rum, and even, I believe, Madeira and\ldots{} and coffee. I saw
it as I passed through. To-morrow it will all fall upon you again, they
won't have a crust of bread. It's absurd, really, and so, to my
thinking, a subscription ought to be raised so that the unhappy widow
should not know of the money, but only you, for instance. Am I right?''

``I don't know\ldots{} this is only to-day, once in her life\ldots{}.
She was so anxious to do honour, to celebrate the memory\ldots{}. And
she is very sensible\ldots{} but just as you think and I shall be very,
very\ldots{} they will all be\ldots{} and God will reward\ldots{} and
the orphans\ldots{}''

Sonia burst into tears.

``Very well, then, keep it in mind; and now will you accept for the
benefit of your relation the small sum that I am able to spare, from me
personally. I am very anxious that my name should not be mentioned in
connection with it. Here\ldots{} having so to speak anxieties of my own,
I cannot do more\ldots{}''

And Pyotr Petrovitch held out to Sonia a ten-rouble note carefully
unfolded. Sonia took it, flushed crimson, jumped up, muttered something
and began taking leave. Pyotr Petrovitch accompanied her ceremoniously
to the door. She got out of the room at last, agitated and distressed,
and returned to Katerina Ivanovna, overwhelmed with confusion.

All this time Lebeziatnikov had stood at the window or walked about the
room, anxious not to interrupt the conversation; when Sonia had gone he
walked up to Pyotr Petrovitch and solemnly held out his hand.

``I heard and \emph{saw} everything,'' he said, laying stress on the
last verb. ``That is honourable, I mean to say, it's humane! You wanted
to avoid gratitude, I saw! And although I cannot, I confess, in
principle sympathise with private charity, for it not only fails to
eradicate the evil but even promotes it, yet I must admit that I saw
your action with pleasure---yes, yes, I like it.''

``That's all nonsense,'' muttered Pyotr Petrovitch, somewhat
disconcerted, looking carefully at Lebeziatnikov.

``No, it's not nonsense! A man who has suffered distress and annoyance
as you did yesterday and who yet can sympathise with the misery of
others, such a man\ldots{} even though he is making a social
mistake---is still deserving of respect! I did not expect it indeed of
you, Pyotr Petrovitch, especially as according to your ideas\ldots{} oh,
what a drawback your ideas are to you! How distressed you are for
instance by your ill-luck yesterday,'' cried the simple-hearted
Lebeziatnikov, who felt a return of affection for Pyotr Petrovitch.
``And, what do you want with marriage, with \emph{legal} marriage, my
dear, noble Pyotr Petrovitch? Why do you cling to this \emph{legality}
of marriage? Well, you may beat me if you like, but I am glad,
positively glad it hasn't come off, that you are free, that you are not
quite lost for humanity\ldots{}. you see, I've spoken my mind!''

``Because I don't want in your free marriage to be made a fool of and to
bring up another man's children, that's why I want legal marriage,''
Luzhin replied in order to make some answer.

He seemed preoccupied by something.

``Children? You referred to children,'' Lebeziatnikov started off like a
warhorse at the trumpet call. ``Children are a social question and a
question of first importance, I agree; but the question of children has
another solution. Some refuse to have children altogether, because they
suggest the institution of the family. We'll speak of children later,
but now as to the question of honour, I confess that's my weak point.
That horrid, military, Pushkin expression is unthinkable in the
dictionary of the future. What does it mean indeed? It's nonsense, there
will be no deception in a free marriage! That is only the natural
consequence of a legal marriage, so to say, its corrective, a protest.
So that indeed it's not humiliating\ldots{} and if I ever, to suppose an
absurdity, were to be legally married, I should be positively glad of
it. I should say to my wife: `My dear, hitherto I have loved you, now I
respect you, for you've shown you can protest!' You laugh! That's
because you are incapable of getting away from prejudices. Confound it
all! I understand now where the unpleasantness is of being deceived in a
legal marriage, but it's simply a despicable consequence of a despicable
position in which both are humiliated. When the deception is open, as in
a free marriage, then it does not exist, it's unthinkable. Your wife
will only prove how she respects you by considering you incapable of
opposing her happiness and avenging yourself on her for her new husband.
Damn it all! I sometimes dream if I were to be married, pfoo! I mean if
I were to marry, legally or not, it's just the same, I should present my
wife with a lover if she had not found one for herself. `My dear,' I
should say, `I love you, but even more than that I desire you to respect
me. See!' Am I not right?''

Pyotr Petrovitch sniggered as he listened, but without much merriment.
He hardly heard it indeed. He was preoccupied with something else and
even Lebeziatnikov at last noticed it. Pyotr Petrovitch seemed excited
and rubbed his hands. Lebeziatnikov remembered all this and reflected
upon it afterwards.

\protect\hypertarget{link2HCH0028}{}{}

\mychapter{2}{Chapter II}\label{chapter-ii-4}

It would be difficult to explain exactly what could have originated the
idea of that senseless dinner in Katerina Ivanovna's disordered brain.
Nearly ten of the twenty roubles, given by Raskolnikov for Marmeladov's
funeral, were wasted upon it. Possibly Katerina Ivanovna felt obliged to
honour the memory of the deceased ``suitably,'' that all the lodgers,
and still more Amalia Ivanovna, might know ``that he was in no way their
inferior, and perhaps very much their superior,'' and that no one had
the right ``to turn up his nose at him.'' Perhaps the chief element was
that peculiar ``poor man's pride,'' which compels many poor people to
spend their last savings on some traditional social ceremony, simply in
order to do ``like other people,'' and not to ``be looked down upon.''
It is very probable, too, that Katerina Ivanovna longed on this
occasion, at the moment when she seemed to be abandoned by everyone, to
show those ``wretched contemptible lodgers'' that she knew ``how to do
things, how to entertain'' and that she had been brought up ``in a
genteel, she might almost say aristocratic colonel's family'' and had
not been meant for sweeping floors and washing the children's rags at
night. Even the poorest and most broken-spirited people are sometimes
liable to these paroxysms of pride and vanity which take the form of an
irresistible nervous craving. And Katerina Ivanovna was not
broken-spirited; she might have been killed by circumstance, but her
spirit could not have been broken, that is, she could not have been
intimidated, her will could not be crushed. Moreover Sonia had said with
good reason that her mind was unhinged. She could not be said to be
insane, but for a year past she had been so harassed that her mind might
well be overstrained. The later stages of consumption are apt, doctors
tell us, to affect the intellect.

There was no great variety of wines, nor was there Madeira; but wine
there was. There was vodka, rum and Lisbon wine, all of the poorest
quality but in sufficient quantity. Besides the traditional rice and
honey, there were three or four dishes, one of which consisted of
pancakes, all prepared in Amalia Ivanovna's kitchen. Two samovars were
boiling, that tea and punch might be offered after dinner. Katerina
Ivanovna had herself seen to purchasing the provisions, with the help of
one of the lodgers, an unfortunate little Pole who had somehow been
stranded at Madame Lippevechsel's. He promptly put himself at Katerina
Ivanovna's disposal and had been all that morning and all the day before
running about as fast as his legs could carry him, and very anxious that
everyone should be aware of it. For every trifle he ran to Katerina
Ivanovna, even hunting her out at the bazaar, at every instant called
her ``\emph{Pani}.'' She was heartily sick of him before the end, though
she had declared at first that she could not have got on without this
``serviceable and magnanimous man.'' It was one of Katerina Ivanovna's
characteristics to paint everyone she met in the most glowing colours.
Her praises were so exaggerated as sometimes to be embarrassing; she
would invent various circumstances to the credit of her new acquaintance
and quite genuinely believe in their reality. Then all of a sudden she
would be disillusioned and would rudely and contemptuously repulse the
person she had only a few hours before been literally adoring. She was
naturally of a gay, lively and peace-loving disposition, but from
continual failures and misfortunes she had come to desire so
\emph{keenly} that all should live in peace and joy and should not
\emph{dare} to break the peace, that the slightest jar, the smallest
disaster reduced her almost to frenzy, and she would pass in an instant
from the brightest hopes and fancies to cursing her fate and raving, and
knocking her head against the wall.

Amalia Ivanovna, too, suddenly acquired extraordinary importance in
Katerina Ivanovna's eyes and was treated by her with extraordinary
respect, probably only because Amalia Ivanovna had thrown herself heart
and soul into the preparations. She had undertaken to lay the table, to
provide the linen, crockery, etc., and to cook the dishes in her
kitchen, and Katerina Ivanovna had left it all in her hands and gone
herself to the cemetery. Everything had been well done. Even the
table-cloth was nearly clean; the crockery, knives, forks and glasses
were, of course, of all shapes and patterns, lent by different lodgers,
but the table was properly laid at the time fixed, and Amalia Ivanovna,
feeling she had done her work well, had put on a black silk dress and a
cap with new mourning ribbons and met the returning party with some
pride. This pride, though justifiable, displeased Katerina Ivanovna for
some reason: ``as though the table could not have been laid except by
Amalia Ivanovna!'' She disliked the cap with new ribbons, too. ``Could
she be stuck up, the stupid German, because she was mistress of the
house, and had consented as a favour to help her poor lodgers! As a
favour! Fancy that! Katerina Ivanovna's father who had been a colonel
and almost a governor had sometimes had the table set for forty persons,
and then anyone like Amalia Ivanovna, or rather Ludwigovna, would not
have been allowed into the kitchen.''

Katerina Ivanovna, however, put off expressing her feelings for the time
and contented herself with treating her coldly, though she decided
inwardly that she would certainly have to put Amalia Ivanovna down and
set her in her proper place, for goodness only knew what she was
fancying herself. Katerina Ivanovna was irritated too by the fact that
hardly any of the lodgers invited had come to the funeral, except the
Pole who had just managed to run into the cemetery, while to the
memorial dinner the poorest and most insignificant of them had turned
up, the wretched creatures, many of them not quite sober. The older and
more respectable of them all, as if by common consent, stayed away.
Pyotr Petrovitch Luzhin, for instance, who might be said to be the most
respectable of all the lodgers, did not appear, though Katerina Ivanovna
had the evening before told all the world, that is Amalia Ivanovna,
Polenka, Sonia and the Pole, that he was the most generous,
noble-hearted man with a large property and vast connections, who had
been a friend of her first husband's, and a guest in her father's house,
and that he had promised to use all his influence to secure her a
considerable pension. It must be noted that when Katerina Ivanovna
exalted anyone's connections and fortune, it was without any ulterior
motive, quite disinterestedly, for the mere pleasure of adding to the
consequence of the person praised. Probably ``taking his cue'' from
Luzhin, ``that contemptible wretch Lebeziatnikov had not turned up
either. What did he fancy himself? He was only asked out of kindness and
because he was sharing the same room with Pyotr Petrovitch and was a
friend of his, so that it would have been awkward not to invite him.''

Among those who failed to appear were ``the genteel lady and her
old-maidish daughter,'' who had only been lodgers in the house for the
last fortnight, but had several times complained of the noise and uproar
in Katerina Ivanovna's room, especially when Marmeladov had come back
drunk. Katerina Ivanovna heard this from Amalia Ivanovna who,
quarrelling with Katerina Ivanovna, and threatening to turn the whole
family out of doors, had shouted at her that they ``were not worth the
foot'' of the honourable lodgers whom they were disturbing. Katerina
Ivanovna determined now to invite this lady and her daughter, ``whose
foot she was not worth,'' and who had turned away haughtily when she
casually met them, so that they might know that ``she was more noble in
her thoughts and feelings and did not harbour malice,'' and might see
that she was not accustomed to her way of living. She had proposed to
make this clear to them at dinner with allusions to her late father's
governorship, and also at the same time to hint that it was exceedingly
stupid of them to turn away on meeting her. The fat colonel-major (he
was really a discharged officer of low rank) was also absent, but it
appeared that he had been ``not himself'' for the last two days. The
party consisted of the Pole, a wretched looking clerk with a spotty face
and a greasy coat, who had not a word to say for himself, and smelt
abominably, a deaf and almost blind old man who had once been in the
post office and who had been from immemorial ages maintained by someone
at Amalia Ivanovna's.

A retired clerk of the commissariat department came, too; he was drunk,
had a loud and most unseemly laugh and only fancy---was without a
waistcoat! One of the visitors sat straight down to the table without
even greeting Katerina Ivanovna. Finally one person having no suit
appeared in his dressing-gown, but this was too much, and the efforts of
Amalia Ivanovna and the Pole succeeded in removing him. The Pole brought
with him, however, two other Poles who did not live at Amalia Ivanovna's
and whom no one had seen here before. All this irritated Katerina
Ivanovna intensely. ``For whom had they made all these preparations
then?'' To make room for the visitors the children had not even been
laid for at the table; but the two little ones were sitting on a bench
in the furthest corner with their dinner laid on a box, while Polenka as
a big girl had to look after them, feed them, and keep their noses wiped
like well-bred children's.

Katerina Ivanovna, in fact, could hardly help meeting her guests with
increased dignity, and even haughtiness. She stared at some of them with
special severity, and loftily invited them to take their seats. Rushing
to the conclusion that Amalia Ivanovna must be responsible for those who
were absent, she began treating her with extreme nonchalance, which the
latter promptly observed and resented. Such a beginning was no good omen
for the end. All were seated at last.

Raskolnikov came in almost at the moment of their return from the
cemetery. Katerina Ivanovna was greatly delighted to see him, in the
first place, because he was the one ``educated visitor, and, as everyone
knew, was in two years to take a professorship in the university,'' and
secondly because he immediately and respectfully apologised for having
been unable to be at the funeral. She positively pounced upon him, and
made him sit on her left hand (Amalia Ivanovna was on her right). In
spite of her continual anxiety that the dishes should be passed round
correctly and that everyone should taste them, in spite of the agonising
cough which interrupted her every minute and seemed to have grown worse
during the last few days, she hastened to pour out in a half whisper to
Raskolnikov all her suppressed feelings and her just indignation at the
failure of the dinner, interspersing her remarks with lively and
uncontrollable laughter at the expense of her visitors and especially of
her landlady.

``It's all that cuckoo's fault! You know whom I mean? Her, her!''
Katerina Ivanovna nodded towards the landlady. ``Look at her, she's
making round eyes, she feels that we are talking about her and can't
understand. Pfoo, the owl! Ha-ha! (Cough-cough-cough.) And what does she
put on that cap for? (Cough-cough-cough.) Have you noticed that she
wants everyone to consider that she is patronising me and doing me an
honour by being here? I asked her like a sensible woman to invite
people, especially those who knew my late husband, and look at the set
of fools she has brought! The sweeps! Look at that one with the spotty
face. And those wretched Poles, ha-ha-ha! (Cough-cough-cough.) Not one
of them has ever poked his nose in here, I've never set eyes on them.
What have they come here for, I ask you? There they sit in a row. Hey,
\emph{pan}!'' she cried suddenly to one of them, ``have you tasted the
pancakes? Take some more! Have some beer! Won't you have some vodka?
Look, he's jumped up and is making his bows, they must be quite starved,
poor things. Never mind, let them eat! They don't make a noise, anyway,
though I'm really afraid for our landlady's silver spoons\ldots{} Amalia
Ivanovna!'' she addressed her suddenly, almost aloud, ``if your spoons
should happen to be stolen, I won't be responsible, I warn you!
Ha-ha-ha!'' She laughed turning to Raskolnikov, and again nodding
towards the landlady, in high glee at her sally. ``She didn't
understand, she didn't understand again! Look how she sits with her
mouth open! An owl, a real owl! An owl in new ribbons, ha-ha-ha!''

Here her laugh turned again to an insufferable fit of coughing that
lasted five minutes. Drops of perspiration stood out on her forehead and
her handkerchief was stained with blood. She showed Raskolnikov the
blood in silence, and as soon as she could get her breath began
whispering to him again with extreme animation and a hectic flush on her
cheeks.

``Do you know, I gave her the most delicate instructions, so to speak,
for inviting that lady and her daughter, you understand of whom I am
speaking? It needed the utmost delicacy, the greatest nicety, but she
has managed things so that that fool, that conceited baggage, that
provincial nonentity, simply because she is the widow of a major, and
has come to try and get a pension and to fray out her skirts in the
government offices, because at fifty she paints her face (everybody
knows it)\ldots{} a creature like that did not think fit to come, and
has not even answered the invitation, which the most ordinary good
manners required! I can't understand why Pyotr Petrovitch has not come?
But where's Sonia? Where has she gone? Ah, there she is at last! what is
it, Sonia, where have you been? It's odd that even at your father's
funeral you should be so unpunctual. Rodion Romanovitch, make room for
her beside you. That's your place, Sonia\ldots{} take what you like.
Have some of the cold entrée with jelly, that's the best. They'll bring
the pancakes directly. Have they given the children some? Polenka, have
you got everything? (Cough-cough-cough.) That's all right. Be a good
girl, Lida, and, Kolya, don't fidget with your feet; sit like a little
gentleman. What are you saying, Sonia?''

Sonia hastened to give her Pyotr Petrovitch's apologies, trying to speak
loud enough for everyone to hear and carefully choosing the most
respectful phrases which she attributed to Pyotr Petrovitch. She added
that Pyotr Petrovitch had particularly told her to say that, as soon as
he possibly could, he would come immediately to discuss \emph{business}
alone with her and to consider what could be done for her, etc., etc.

Sonia knew that this would comfort Katerina Ivanovna, would flatter her
and gratify her pride. She sat down beside Raskolnikov; she made him a
hurried bow, glancing curiously at him. But for the rest of the time she
seemed to avoid looking at him or speaking to him. She seemed
absent-minded, though she kept looking at Katerina Ivanovna, trying to
please her. Neither she nor Katerina Ivanovna had been able to get
mourning; Sonia was wearing dark brown, and Katerina Ivanovna had on her
only dress, a dark striped cotton one.

The message from Pyotr Petrovitch was very successful. Listening to
Sonia with dignity, Katerina Ivanovna inquired with equal dignity how
Pyotr Petrovitch was, then at once whispered almost aloud to Raskolnikov
that it certainly would have been strange for a man of Pyotr
Petrovitch's position and standing to find himself in such
``extraordinary company,'' in spite of his devotion to her family and
his old friendship with her father.

``That's why I am so grateful to you, Rodion Romanovitch, that you have
not disdained my hospitality, even in such surroundings,'' she added
almost aloud. ``But I am sure that it was only your special affection
for my poor husband that has made you keep your promise.''

Then once more with pride and dignity she scanned her visitors, and
suddenly inquired aloud across the table of the deaf man: ``Wouldn't he
have some more meat, and had he been given some wine?'' The old man made
no answer and for a long while could not understand what he was asked,
though his neighbours amused themselves by poking and shaking him. He
simply gazed about him with his mouth open, which only increased the
general mirth.

``What an imbecile! Look, look! Why was he brought? But as to Pyotr
Petrovitch, I always had confidence in him,'' Katerina Ivanovna
continued, ``and, of course, he is not like\ldots{}'' with an extremely
stern face she addressed Amalia Ivanovna so sharply and loudly that the
latter was quite disconcerted, ``not like your dressed up draggletails
whom my father would not have taken as cooks into his kitchen, and my
late husband would have done them honour if he had invited them in the
goodness of his heart.''

``Yes, he was fond of drink, he was fond of it, he did drink!'' cried
the commissariat clerk, gulping down his twelfth glass of vodka.

``My late husband certainly had that weakness, and everyone knows it,''
Katerina Ivanovna attacked him at once, ``but he was a kind and
honourable man, who loved and respected his family. The worst of it was
his good nature made him trust all sorts of disreputable people, and he
drank with fellows who were not worth the sole of his shoe. Would you
believe it, Rodion Romanovitch, they found a gingerbread cock in his
pocket; he was dead drunk, but he did not forget the children!''

``A cock? Did you say a cock?'' shouted the commissariat clerk.

Katerina Ivanovna did not vouchsafe a reply. She sighed, lost in
thought.

``No doubt you think, like everyone, that I was too severe with him,''
she went on, addressing Raskolnikov. ``But that's not so! He respected
me, he respected me very much! He was a kind-hearted man! And how sorry
I was for him sometimes! He would sit in a corner and look at me, I used
to feel so sorry for him, I used to want to be kind to him and then
would think to myself: `Be kind to him and he will drink again,' it was
only by severity that you could keep him within bounds.''

``Yes, he used to get his hair pulled pretty often,'' roared the
commissariat clerk again, swallowing another glass of vodka.

``Some fools would be the better for a good drubbing, as well as having
their hair pulled. I am not talking of my late husband now!'' Katerina
Ivanovna snapped at him.

The flush on her cheeks grew more and more marked, her chest heaved. In
another minute she would have been ready to make a scene. Many of the
visitors were sniggering, evidently delighted. They began poking the
commissariat clerk and whispering something to him. They were evidently
trying to egg him on.

``Allow me to ask what are you alluding to,'' began the clerk, ``that is
to say, whose\ldots{} about whom\ldots{} did you say just now\ldots{}
But I don't care! That's nonsense! Widow! I forgive you\ldots{}. Pass!''

And he took another drink of vodka.

Raskolnikov sat in silence, listening with disgust. He only ate from
politeness, just tasting the food that Katerina Ivanovna was continually
putting on his plate, to avoid hurting her feelings. He watched Sonia
intently. But Sonia became more and more anxious and distressed; she,
too, foresaw that the dinner would not end peaceably, and saw with
terror Katerina Ivanovna's growing irritation. She knew that she, Sonia,
was the chief reason for the `genteel' ladies' contemptuous treatment of
Katerina Ivanovna's invitation. She had heard from Amalia Ivanovna that
the mother was positively offended at the invitation and had asked the
question: ``How could she let her daughter sit down beside \emph{that
young person}?'' Sonia had a feeling that Katerina Ivanovna had already
heard this and an insult to Sonia meant more to Katerina Ivanovna than
an insult to herself, her children, or her father, Sonia knew that
Katerina Ivanovna would not be satisfied now, ``till she had shown those
draggletails that they were both\ldots{}'' To make matters worse someone
passed Sonia, from the other end of the table, a plate with two hearts
pierced with an arrow, cut out of black bread. Katerina Ivanovna flushed
crimson and at once said aloud across the table that the man who sent it
was ``a drunken ass!''

Amalia Ivanovna was foreseeing something amiss, and at the same time
deeply wounded by Katerina Ivanovna's haughtiness, and to restore the
good-humour of the company and raise herself in their esteem she began,
apropos of nothing, telling a story about an acquaintance of hers ``Karl
from the chemist's,'' who was driving one night in a cab, and that ``the
cabman wanted him to kill, and Karl very much begged him not to kill,
and wept and clasped hands, and frightened and from fear pierced his
heart.'' Though Katerina Ivanovna smiled, she observed at once that
Amalia Ivanovna ought not to tell anecdotes in Russian; the latter was
still more offended, and she retorted that her ``\emph{Vater aus Berlin}
was a very important man, and always went with his hands in pockets.''
Katerina Ivanovna could not restrain herself and laughed so much that
Amalia Ivanovna lost patience and could scarcely control herself.

``Listen to the owl!'' Katerina Ivanovna whispered at once, her
good-humour almost restored, ``she meant to say he kept his hands in his
pockets, but she said he put his hands in people's pockets.
(Cough-cough.) And have you noticed, Rodion Romanovitch, that all these
Petersburg foreigners, the Germans especially, are all stupider than we!
Can you fancy anyone of us telling how `Karl from the chemist's'
`pierced his heart from fear' and that the idiot, instead of punishing
the cabman, `clasped his hands and wept, and much begged.' Ah, the fool!
And you know she fancies it's very touching and does not suspect how
stupid she is! To my thinking that drunken commissariat clerk is a great
deal cleverer, anyway one can see that he has addled his brains with
drink, but you know, these foreigners are always so well behaved and
serious\ldots{}. Look how she sits glaring! She is angry, ha-ha!
(Cough-cough-cough.)''

Regaining her good-humour, Katerina Ivanovna began at once telling
Raskolnikov that when she had obtained her pension, she intended to open
a school for the daughters of gentlemen in her native town T------. This
was the first time she had spoken to him of the project, and she
launched out into the most alluring details. It suddenly appeared that
Katerina Ivanovna had in her hands the very certificate of honour of
which Marmeladov had spoken to Raskolnikov in the tavern, when he told
him that Katerina Ivanovna, his wife, had danced the shawl dance before
the governor and other great personages on leaving school. This
certificate of honour was obviously intended now to prove Katerina
Ivanovna's right to open a boarding-school; but she had armed herself
with it chiefly with the object of overwhelming ``those two stuck-up
draggletails'' if they came to the dinner, and proving incontestably
that Katerina Ivanovna was of the most noble, ``she might even say
aristocratic family, a colonel's daughter and was far superior to
certain adventuresses who have been so much to the fore of late.'' The
certificate of honour immediately passed into the hands of the drunken
guests, and Katerina Ivanovna did not try to retain it, for it actually
contained the statement \emph{en toutes lettres}, that her father was of
the rank of a major, and also a companion of an order, so that she
really was almost the daughter of a colonel.

Warming up, Katerina Ivanovna proceeded to enlarge on the peaceful and
happy life they would lead in T------, on the gymnasium teachers whom
she would engage to give lessons in her boarding-school, one a most
respectable old Frenchman, one Mangot, who had taught Katerina Ivanovna
herself in old days and was still living in T------, and would no doubt
teach in her school on moderate terms. Next she spoke of Sonia who would
go with her to T------ and help her in all her plans. At this someone at
the further end of the table gave a sudden guffaw.

Though Katerina Ivanovna tried to appear to be disdainfully unaware of
it, she raised her voice and began at once speaking with conviction of
Sonia's undoubted ability to assist her, of ``her gentleness, patience,
devotion, generosity and good education,'' tapping Sonia on the cheek
and kissing her warmly twice. Sonia flushed crimson, and Katerina
Ivanovna suddenly burst into tears, immediately observing that she was
``nervous and silly, that she was too much upset, that it was time to
finish, and as the dinner was over, it was time to hand round the tea.''

At that moment, Amalia Ivanovna, deeply aggrieved at taking no part in
the conversation, and not being listened to, made one last effort, and
with secret misgivings ventured on an exceedingly deep and weighty
observation, that ``in the future boarding-school she would have to pay
particular attention to \emph{die Wäsche}, and that there certainly must
be a good \emph{dame} to look after the linen, and secondly that the
young ladies must not novels at night read.''

Katerina Ivanovna, who certainly was upset and very tired, as well as
heartily sick of the dinner, at once cut short Amalia Ivanovna, saying
``she knew nothing about it and was talking nonsense, that it was the
business of the laundry maid, and not of the directress of a high-class
boarding-school to look after \emph{die Wäsche}, and as for
novel-reading, that was simply rudeness, and she begged her to be
silent.'' Amalia Ivanovna fired up and getting angry observed that she
only ``meant her good,'' and that ``she had meant her very good,'' and
that ``it was long since she had paid her \emph{gold} for the
lodgings.''

Katerina Ivanovna at once ``set her down,'' saying that it was a lie to
say she wished her good, because only yesterday when her dead husband
was lying on the table, she had worried her about the lodgings. To this
Amalia Ivanovna very appropriately observed that she had invited those
ladies, but ``those ladies had not come, because those ladies \emph{are}
ladies and cannot come to a lady who is not a lady.'' Katerina Ivanovna
at once pointed out to her, that as she was a slut she could not judge
what made one really a lady. Amalia Ivanovna at once declared that her
``\emph{Vater aus Berlin} was a very, very important man, and both hands
in pockets went, and always used to say: `Poof! poof!''' and she leapt
up from the table to represent her father, sticking her hands in her
pockets, puffing her cheeks, and uttering vague sounds resembling
``poof! poof!'' amid loud laughter from all the lodgers, who purposely
encouraged Amalia Ivanovna, hoping for a fight.

But this was too much for Katerina Ivanovna, and she at once declared,
so that all could hear, that Amalia Ivanovna probably never had a
father, but was simply a drunken Petersburg Finn, and had certainly once
been a cook and probably something worse. Amalia Ivanovna turned as red
as a lobster and squealed that perhaps Katerina Ivanovna never had a
father, ``but she had a \emph{Vater aus Berlin} and that he wore a long
coat and always said poof-poof-poof!''

Katerina Ivanovna observed contemptuously that all knew what her family
was and that on that very certificate of honour it was stated in print
that her father was a colonel, while Amalia Ivanovna's father---if she
really had one---was probably some Finnish milkman, but that probably
she never had a father at all, since it was still uncertain whether her
name was Amalia Ivanovna or Amalia Ludwigovna.

At this Amalia Ivanovna, lashed to fury, struck the table with her fist,
and shrieked that she was Amalia Ivanovna, and not Ludwigovna, ``that
her \emph{Vater} was named Johann and that he was a burgomeister, and
that Katerina Ivanovna's \emph{Vater} was quite never a burgomeister.''
Katerina Ivanovna rose from her chair, and with a stern and apparently
calm voice (though she was pale and her chest was heaving) observed that
``if she dared for one moment to set her contemptible wretch of a father
on a level with her papa, she, Katerina Ivanovna, would tear her cap off
her head and trample it under foot.'' Amalia Ivanovna ran about the
room, shouting at the top of her voice, that she was mistress of the
house and that Katerina Ivanovna should leave the lodgings that minute;
then she rushed for some reason to collect the silver spoons from the
table. There was a great outcry and uproar, the children began crying.
Sonia ran to restrain Katerina Ivanovna, but when Amalia Ivanovna
shouted something about ``the yellow ticket,'' Katerina Ivanovna pushed
Sonia away, and rushed at the landlady to carry out her threat.

At that minute the door opened, and Pyotr Petrovitch Luzhin appeared on
the threshold. He stood scanning the party with severe and vigilant
eyes. Katerina Ivanovna rushed to him.

\protect\hypertarget{link2HCH0029}{}{}

\mychapter{3}{Chapter III}\label{chapter-iii-4}

``Pyotr Petrovitch,'' she cried, ``protect me\ldots{} you at least! Make
this foolish woman understand that she can't behave like this to a lady
in misfortune\ldots{} that there is a law for such things\ldots{}. I'll
go to the governor-general himself\ldots{}. She shall answer for
it\ldots{}. Remembering my father's hospitality protect these orphans.''

``Allow me, madam\ldots{}. Allow me.'' Pyotr Petrovitch waved her off.
``Your papa as you are well aware I had not the honour of knowing''
(someone laughed aloud) ``and I do not intend to take part in your
everlasting squabbles with Amalia Ivanovna\ldots{}. I have come here to
speak of my own affairs\ldots{} and I want to have a word with your
stepdaughter, Sofya\ldots{} Ivanovna, I think it is? Allow me to pass.''

Pyotr Petrovitch, edging by her, went to the opposite corner where Sonia
was.

Katerina Ivanovna remained standing where she was, as though
thunderstruck. She could not understand how Pyotr Petrovitch could deny
having enjoyed her father's hospitality. Though she had invented it
herself, she believed in it firmly by this time. She was struck too by
the businesslike, dry and even contemptuous menacing tone of Pyotr
Petrovitch. All the clamour gradually died away at his entrance. Not
only was this ``serious business man'' strikingly incongruous with the
rest of the party, but it was evident, too, that he had come upon some
matter of consequence, that some exceptional cause must have brought him
and that therefore something was going to happen. Raskolnikov, standing
beside Sonia, moved aside to let him pass; Pyotr Petrovitch did not seem
to notice him. A minute later Lebeziatnikov, too, appeared in the
doorway; he did not come in, but stood still, listening with marked
interest, almost wonder, and seemed for a time perplexed.

``Excuse me for possibly interrupting you, but it's a matter of some
importance,'' Pyotr Petrovitch observed, addressing the company
generally. ``I am glad indeed to find other persons present. Amalia
Ivanovna, I humbly beg you as mistress of the house to pay careful
attention to what I have to say to Sofya Ivanovna. Sofya Ivanovna,'' he
went on, addressing Sonia, who was very much surprised and already
alarmed, ``immediately after your visit I found that a hundred-rouble
note was missing from my table, in the room of my friend Mr.
Lebeziatnikov. If in any way whatever you know and will tell us where it
is now, I assure you on my word of honour and call all present to
witness that the matter shall end there. In the opposite case I shall be
compelled to have recourse to very serious measures and then\ldots{} you
must blame yourself.''

Complete silence reigned in the room. Even the crying children were
still. Sonia stood deadly pale, staring at Luzhin and unable to say a
word. She seemed not to understand. Some seconds passed.

``Well, how is it to be then?'' asked Luzhin, looking intently at her.

``I don't know\ldots{}. I know nothing about it,'' Sonia articulated
faintly at last.

``No, you know nothing?'' Luzhin repeated and again he paused for some
seconds. ``Think a moment, mademoiselle,'' he began severely, but still,
as it were, admonishing her. ``Reflect, I am prepared to give you time
for consideration. Kindly observe this: if I were not so entirely
convinced I should not, you may be sure, with my experience venture to
accuse you so directly. Seeing that for such direct accusation before
witnesses, if false or even mistaken, I should myself in a certain sense
be made responsible, I am aware of that. This morning I changed for my
own purposes several five-per-cent securities for the sum of
approximately three thousand roubles. The account is noted down in my
pocket-book. On my return home I proceeded to count the money---as Mr.
Lebeziatnikov will bear witness---and after counting two thousand three
hundred roubles I put the rest in my pocket-book in my coat pocket.
About five hundred roubles remained on the table and among them three
notes of a hundred roubles each. At that moment you entered (at my
invitation)---and all the time you were present you were exceedingly
embarrassed; so that three times you jumped up in the middle of the
conversation and tried to make off. Mr. Lebeziatnikov can bear witness
to this. You yourself, mademoiselle, probably will not refuse to confirm
my statement that I invited you through Mr. Lebeziatnikov, solely in
order to discuss with you the hopeless and destitute position of your
relative, Katerina Ivanovna (whose dinner I was unable to attend), and
the advisability of getting up something of the nature of a
subscription, lottery or the like, for her benefit. You thanked me and
even shed tears. I describe all this as it took place, primarily to
recall it to your mind and secondly to show you that not the slightest
detail has escaped my recollection. Then I took a ten-rouble note from
the table and handed it to you by way of first instalment on my part for
the benefit of your relative. Mr. Lebeziatnikov saw all this. Then I
accompanied you to the door---you being still in the same state of
embarrassment---after which, being left alone with Mr. Lebeziatnikov I
talked to him for ten minutes---then Mr. Lebeziatnikov went out and I
returned to the table with the money lying on it, intending to count it
and to put it aside, as I proposed doing before. To my surprise one
hundred-rouble note had disappeared. Kindly consider the position. Mr.
Lebeziatnikov I cannot suspect. I am ashamed to allude to such a
supposition. I cannot have made a mistake in my reckoning, for the
minute before your entrance I had finished my accounts and found the
total correct. You will admit that recollecting your embarrassment, your
eagerness to get away and the fact that you kept your hands for some
time on the table, and taking into consideration your social position
and the habits associated with it, I was, so to say, with horror and
positively against my will, \emph{compelled} to entertain a
suspicion---a cruel, but justifiable suspicion! I will add further and
repeat that in spite of my positive conviction, I realise that I run a
certain risk in making this accusation, but as you see, I could not let
it pass. I have taken action and I will tell you why: solely, madam,
solely, owing to your black ingratitude! Why! I invite you for the
benefit of your destitute relative, I present you with my donation of
ten roubles and you, on the spot, repay me for all that with such an
action. It is too bad! You need a lesson. Reflect! Moreover, like a true
friend I beg you---and you could have no better friend at this
moment---think what you are doing, otherwise I shall be immovable! Well,
what do you say?''

``I have taken nothing,'' Sonia whispered in terror, ``you gave me ten
roubles, here it is, take it.''

Sonia pulled her handkerchief out of her pocket, untied a corner of it,
took out the ten-rouble note and gave it to Luzhin.

``And the hundred roubles you do not confess to taking?'' he insisted
reproachfully, not taking the note.

Sonia looked about her. All were looking at her with such awful, stern,
ironical, hostile eyes. She looked at Raskolnikov\ldots{} he stood
against the wall, with his arms crossed, looking at her with glowing
eyes.

``Good God!'' broke from Sonia.

``Amalia Ivanovna, we shall have to send word to the police and
therefore I humbly beg you meanwhile to send for the house porter,''
Luzhin said softly and even kindly.

``\emph{Gott der Barmherzige}! I knew she was the thief,'' cried Amalia
Ivanovna, throwing up her hands.

``You knew it?'' Luzhin caught her up, ``then I suppose you had some
reason before this for thinking so. I beg you, worthy Amalia Ivanovna,
to remember your words which have been uttered before witnesses.''

There was a buzz of loud conversation on all sides. All were in
movement.

``What!'' cried Katerina Ivanovna, suddenly realising the position, and
she rushed at Luzhin. ``What! You accuse her of stealing? Sonia? Ah, the
wretches, the wretches!''

And running to Sonia she flung her wasted arms round her and held her as
in a vise.

``Sonia! how dared you take ten roubles from him? Foolish girl! Give it
to me! Give me the ten roubles at once---here!''

And snatching the note from Sonia, Katerina Ivanovna crumpled it up and
flung it straight into Luzhin's face. It hit him in the eye and fell on
the ground. Amalia Ivanovna hastened to pick it up. Pyotr Petrovitch
lost his temper.

``Hold that mad woman!'' he shouted.

At that moment several other persons, besides Lebeziatnikov, appeared in
the doorway, among them the two ladies.

``What! Mad? Am I mad? Idiot!'' shrieked Katerina Ivanovna. ``You are an
idiot yourself, pettifogging lawyer, base man! Sonia, Sonia take his
money! Sonia a thief! Why, she'd give away her last penny!'' and
Katerina Ivanovna broke into hysterical laughter. ``Did you ever see
such an idiot?'' she turned from side to side. ``And you too?'' she
suddenly saw the landlady, ``and you too, sausage eater, you declare
that she is a thief, you trashy Prussian hen's leg in a crinoline! She
hasn't been out of this room: she came straight from you, you wretch,
and sat down beside me, everyone saw her. She sat here, by Rodion
Romanovitch. Search her! Since she's not left the room, the money would
have to be on her! Search her, search her! But if you don't find it,
then excuse me, my dear fellow, you'll answer for it! I'll go to our
Sovereign, to our Sovereign, to our gracious Tsar himself, and throw
myself at his feet, to-day, this minute! I am alone in the world! They
would let me in! Do you think they wouldn't? You're wrong, I will get
in! I will get in! You reckoned on her meekness! You relied upon that!
But I am not so submissive, let me tell you! You've gone too far
yourself. Search her, search her!''

And Katerina Ivanovna in a frenzy shook Luzhin and dragged him towards
Sonia.

``I am ready, I'll be responsible\ldots{} but calm yourself, madam, calm
yourself. I see that you are not so submissive!\ldots{} Well, well, but
as to that\ldots{}'' Luzhin muttered, ``that ought to be before the
police\ldots{} though indeed there are witnesses enough as it
is\ldots{}. I am ready\ldots{}. But in any case it's difficult for a
man\ldots{} on account of her sex\ldots{}. But with the help of Amalia
Ivanovna\ldots{} though, of course, it's not the way to do
things\ldots{}. How is it to be done?''

``As you will! Let anyone who likes search her!'' cried Katerina
Ivanovna. ``Sonia, turn out your pockets! See! Look, monster, the pocket
is empty, here was her handkerchief! Here is the other pocket, look!
D'you see, d'you see?''

And Katerina Ivanovna turned---or rather snatched---both pockets inside
out. But from the right pocket a piece of paper flew out and describing
a parabola in the air fell at Luzhin's feet. Everyone saw it, several
cried out. Pyotr Petrovitch stooped down, picked up the paper in two
fingers, lifted it where all could see it and opened it. It was a
hundred-rouble note folded in eight. Pyotr Petrovitch held up the note
showing it to everyone.

``Thief! Out of my lodging. Police, police!'' yelled Amalia Ivanovna.
``They must to Siberia be sent! Away!''

Exclamations arose on all sides. Raskolnikov was silent, keeping his
eyes fixed on Sonia, except for an occasional rapid glance at Luzhin.
Sonia stood still, as though unconscious. She was hardly able to feel
surprise. Suddenly the colour rushed to her cheeks; she uttered a cry
and hid her face in her hands.

``No, it wasn't I! I didn't take it! I know nothing about it,'' she
cried with a heartrending wail, and she ran to Katerina Ivanovna, who
clasped her tightly in her arms, as though she would shelter her from
all the world.

``Sonia! Sonia! I don't believe it! You see, I don't believe it!'' she
cried in the face of the obvious fact, swaying her to and fro in her
arms like a baby, kissing her face continually, then snatching at her
hands and kissing them, too, ``you took it! How stupid these people are!
Oh dear! You are fools, fools,'' she cried, addressing the whole room,
``you don't know, you don't know what a heart she has, what a girl she
is! She take it, she? She'd sell her last rag, she'd go barefoot to help
you if you needed it, that's what she is! She has the yellow passport
because my children were starving, she sold herself for us! Ah, husband,
husband! Do you see? Do you see? What a memorial dinner for you!
Merciful heavens! Defend her, why are you all standing still? Rodion
Romanovitch, why don't you stand up for her? Do you believe it, too? You
are not worth her little finger, all of you together! Good God! Defend
her now, at least!''

The wail of the poor, consumptive, helpless woman seemed to produce a
great effect on her audience. The agonised, wasted, consumptive face,
the parched blood-stained lips, the hoarse voice, the tears unrestrained
as a child's, the trustful, childish and yet despairing prayer for help
were so piteous that everyone seemed to feel for her. Pyotr Petrovitch
at any rate was at once moved to \emph{compassion}.

``Madam, madam, this incident does not reflect upon you!'' he cried
impressively, ``no one would take upon himself to accuse you of being an
instigator or even an accomplice in it, especially as you have proved
her guilt by turning out her pockets, showing that you had no previous
idea of it. I am most ready, most ready to show compassion, if poverty,
so to speak, drove Sofya Semyonovna to it, but why did you refuse to
confess, mademoiselle? Were you afraid of the disgrace? The first step?
You lost your head, perhaps? One can quite understand it\ldots{}. But
how could you have lowered yourself to such an action? Gentlemen,'' he
addressed the whole company, ``gentlemen! Compassionate and, so to say,
commiserating these people, I am ready to overlook it even now in spite
of the personal insult lavished upon me! And may this disgrace be a
lesson to you for the future,'' he said, addressing Sonia, ``and I will
carry the matter no further. Enough!''

Pyotr Petrovitch stole a glance at Raskolnikov. Their eyes met, and the
fire in Raskolnikov's seemed ready to reduce him to ashes. Meanwhile
Katerina Ivanovna apparently heard nothing. She was kissing and hugging
Sonia like a madwoman. The children, too, were embracing Sonia on all
sides, and Polenka---though she did not fully understand what was
wrong---was drowned in tears and shaking with sobs, as she hid her
pretty little face, swollen with weeping, on Sonia's shoulder.

``How vile!'' a loud voice cried suddenly in the doorway.

Pyotr Petrovitch looked round quickly.

``What vileness!'' Lebeziatnikov repeated, staring him straight in the
face.

Pyotr Petrovitch gave a positive start---all noticed it and recalled it
afterwards. Lebeziatnikov strode into the room.

``And you dared to call me as witness?'' he said, going up to Pyotr
Petrovitch.

``What do you mean? What are you talking about?'' muttered Luzhin.

``I mean that you\ldots{} are a slanderer, that's what my words mean!''
Lebeziatnikov said hotly, looking sternly at him with his short-sighted
eyes.

He was extremely angry. Raskolnikov gazed intently at him, as though
seizing and weighing each word. Again there was a silence. Pyotr
Petrovitch indeed seemed almost dumbfounded for the first moment.

``If you mean that for me,\ldots{}'' he began, stammering. ``But what's
the matter with you? Are you out of your mind?''

``I'm in my mind, but you are a scoundrel! Ah, how vile! I have heard
everything. I kept waiting on purpose to understand it, for I must own
even now it is not quite logical\ldots{}. What you have done it all for
I can't understand.''

``Why, what have I done then? Give over talking in your nonsensical
riddles! Or maybe you are drunk!''

``You may be a drunkard, perhaps, vile man, but I am not! I never touch
vodka, for it's against my convictions. Would you believe it, he, he
himself, with his own hands gave Sofya Semyonovna that hundred-rouble
note---I saw it, I was a witness, I'll take my oath! He did it, he!''
repeated Lebeziatnikov, addressing all.

``Are you crazy, milksop?'' squealed Luzhin. ``She is herself before
you---she herself here declared just now before everyone that I gave her
only ten roubles. How could I have given it to her?''

``I saw it, I saw it,'' Lebeziatnikov repeated, ``and though it is
against my principles, I am ready this very minute to take any oath you
like before the court, for I saw how you slipped it in her pocket. Only
like a fool I thought you did it out of kindness! When you were saying
good-bye to her at the door, while you held her hand in one hand, with
the other, the left, you slipped the note into her pocket. I saw it, I
saw it!''

Luzhin turned pale.

``What lies!'' he cried impudently, ``why, how could you, standing by
the window, see the note? You fancied it with your short-sighted eyes.
You are raving!''

``No, I didn't fancy it. And though I was standing some way off, I saw
it all. And though it certainly would be hard to distinguish a note from
the window---that's true---I knew for certain that it was a
hundred-rouble note, because, when you were going to give Sofya
Semyonovna ten roubles, you took up from the table a hundred-rouble note
(I saw it because I was standing near then, and an idea struck me at
once, so that I did not forget you had it in your hand). You folded it
and kept it in your hand all the time. I didn't think of it again until,
when you were getting up, you changed it from your right hand to your
left and nearly dropped it! I noticed it because the same idea struck me
again, that you meant to do her a kindness without my seeing. You can
fancy how I watched you and I saw how you succeeded in slipping it into
her pocket. I saw it, I saw it, I'll take my oath.''

Lebeziatnikov was almost breathless. Exclamations arose on all hands
chiefly expressive of wonder, but some were menacing in tone. They all
crowded round Pyotr Petrovitch. Katerina Ivanovna flew to Lebeziatnikov.

``I was mistaken in you! Protect her! You are the only one to take her
part! She is an orphan. God has sent you!''

Katerina Ivanovna, hardly knowing what she was doing, sank on her knees
before him.

``A pack of nonsense!'' yelled Luzhin, roused to fury, ``it's all
nonsense you've been talking! `An idea struck you, you didn't think, you
noticed'---what does it amount to? So I gave it to her on the sly on
purpose? What for? With what object? What have I to do with
this\ldots{}?''

``What for? That's what I can't understand, but that what I am telling
you is the fact, that's certain! So far from my being mistaken, you
infamous criminal man, I remember how, on account of it, a question
occurred to me at once, just when I was thanking you and pressing your
hand. What made you put it secretly in her pocket? Why you did it
secretly, I mean? Could it be simply to conceal it from me, knowing that
my convictions are opposed to yours and that I do not approve of private
benevolence, which effects no radical cure? Well, I decided that you
really were ashamed of giving such a large sum before me. Perhaps, too,
I thought, he wants to give her a surprise, when she finds a whole
hundred-rouble note in her pocket. (For I know, some benevolent people
are very fond of decking out their charitable actions in that way.) Then
the idea struck me, too, that you wanted to test her, to see whether,
when she found it, she would come to thank you. Then, too, that you
wanted to avoid thanks and that, as the saying is, your right hand
should not know\ldots{} something of that sort, in fact. I thought of so
many possibilities that I put off considering it, but still thought it
indelicate to show you that I knew your secret. But another idea struck
me again that Sofya Semyonovna might easily lose the money before she
noticed it, that was why I decided to come in here to call her out of
the room and to tell her that you put a hundred roubles in her pocket.
But on my way I went first to Madame Kobilatnikov's to take them the
`General Treatise on the Positive Method' and especially to recommend
Piderit's article (and also Wagner's); then I come on here and what a
state of things I find! Now could I, could I, have all these ideas and
reflections if I had not seen you put the hundred-rouble note in her
pocket?''

When Lebeziatnikov finished his long-winded harangue with the logical
deduction at the end, he was quite tired, and the perspiration streamed
from his face. He could not, alas, even express himself correctly in
Russian, though he knew no other language, so that he was quite
exhausted, almost emaciated after this heroic exploit. But his speech
produced a powerful effect. He had spoken with such vehemence, with such
conviction that everyone obviously believed him. Pyotr Petrovitch felt
that things were going badly with him.

``What is it to do with me if silly ideas did occur to you?'' he
shouted, ``that's no evidence. You may have dreamt it, that's all! And I
tell you, you are lying, sir. You are lying and slandering from some
spite against me, simply from pique, because I did not agree with your
free-thinking, godless, social propositions!''

But this retort did not benefit Pyotr Petrovitch. Murmurs of disapproval
were heard on all sides.

``Ah, that's your line now, is it!'' cried Lebeziatnikov, ``that's
nonsense! Call the police and I'll take my oath! There's only one thing
I can't understand: what made him risk such a contemptible action. Oh,
pitiful, despicable man!''

``I can explain why he risked such an action, and if necessary, I, too,
will swear to it,'' Raskolnikov said at last in a firm voice, and he
stepped forward.

He appeared to be firm and composed. Everyone felt clearly, from the
very look of him that he really knew about it and that the mystery would
be solved.

``Now I can explain it all to myself,'' said Raskolnikov, addressing
Lebeziatnikov. ``From the very beginning of the business, I suspected
that there was some scoundrelly intrigue at the bottom of it. I began to
suspect it from some special circumstances known to me only, which I
will explain at once to everyone: they account for everything. Your
valuable evidence has finally made everything clear to me. I beg all,
all to listen. This gentleman (he pointed to Luzhin) was recently
engaged to be married to a young lady---my sister, Avdotya Romanovna
Raskolnikov. But coming to Petersburg he quarrelled with me, the day
before yesterday, at our first meeting and I drove him out of my
room---I have two witnesses to prove it. He is a very spiteful
man\ldots{}. The day before yesterday I did not know that he was staying
here, in your room, and that consequently on the very day we
quarrelled---the day before yesterday---he saw me give Katerina Ivanovna
some money for the funeral, as a friend of the late Mr. Marmeladov. He
at once wrote a note to my mother and informed her that I had given away
all my money, not to Katerina Ivanovna but to Sofya Semyonovna, and
referred in a most contemptible way to the\ldots{} character of Sofya
Semyonovna, that is, hinted at the character of my attitude to Sofya
Semyonovna. All this you understand was with the object of dividing me
from my mother and sister, by insinuating that I was squandering on
unworthy objects the money which they had sent me and which was all they
had. Yesterday evening, before my mother and sister and in his presence,
I declared that I had given the money to Katerina Ivanovna for the
funeral and not to Sofya Semyonovna and that I had no acquaintance with
Sofya Semyonovna and had never seen her before, indeed. At the same time
I added that he, Pyotr Petrovitch Luzhin, with all his virtues, was not
worth Sofya Semyonovna's little finger, though he spoke so ill of her.
To his question---would I let Sofya Semyonovna sit down beside my
sister, I answered that I had already done so that day. Irritated that
my mother and sister were unwilling to quarrel with me at his
insinuations, he gradually began being unpardonably rude to them. A
final rupture took place and he was turned out of the house. All this
happened yesterday evening. Now I beg your special attention: consider:
if he had now succeeded in proving that Sofya Semyonovna was a thief, he
would have shown to my mother and sister that he was almost right in his
suspicions, that he had reason to be angry at my putting my sister on a
level with Sofya Semyonovna, that, in attacking me, he was protecting
and preserving the honour of my sister, his betrothed. In fact he might
even, through all this, have been able to estrange me from my family,
and no doubt he hoped to be restored to favour with them; to say nothing
of revenging himself on me personally, for he has grounds for supposing
that the honour and happiness of Sofya Semyonovna are very precious to
me. That was what he was working for! That's how I understand it. That's
the whole reason for it and there can be no other!''

It was like this, or somewhat like this, that Raskolnikov wound up his
speech which was followed very attentively, though often interrupted by
exclamations from his audience. But in spite of interruptions he spoke
clearly, calmly, exactly, firmly. His decisive voice, his tone of
conviction and his stern face made a great impression on everyone.

``Yes, yes, that's it,'' Lebeziatnikov assented gleefully, ``that must
be it, for he asked me, as soon as Sofya Semyonovna came into our room,
whether you were here, whether I had seen you among Katerina Ivanovna's
guests. He called me aside to the window and asked me in secret. It was
essential for him that you should be here! That's it, that's it!''

Luzhin smiled contemptuously and did not speak. But he was very pale. He
seemed to be deliberating on some means of escape. Perhaps he would have
been glad to give up everything and get away, but at the moment this was
scarcely possible. It would have implied admitting the truth of the
accusations brought against him. Moreover, the company, which had
already been excited by drink, was now too much stirred to allow it. The
commissariat clerk, though indeed he had not grasped the whole position,
was shouting louder than anyone and was making some suggestions very
unpleasant to Luzhin. But not all those present were drunk; lodgers came
in from all the rooms. The three Poles were tremendously excited and
were continually shouting at him: ``The \emph{pan} is a \emph{lajdak}!''
and muttering threats in Polish. Sonia had been listening with strained
attention, though she too seemed unable to grasp it all; she seemed as
though she had just returned to consciousness. She did not take her eyes
off Raskolnikov, feeling that all her safety lay in him. Katerina
Ivanovna breathed hard and painfully and seemed fearfully exhausted.
Amalia Ivanovna stood looking more stupid than anyone, with her mouth
wide open, unable to make out what had happened. She only saw that Pyotr
Petrovitch had somehow come to grief.

Raskolnikov was attempting to speak again, but they did not let him.
Everyone was crowding round Luzhin with threats and shouts of abuse. But
Pyotr Petrovitch was not intimidated. Seeing that his accusation of
Sonia had completely failed, he had recourse to insolence:

``Allow me, gentlemen, allow me! Don't squeeze, let me pass!'' he said,
making his way through the crowd. ``And no threats, if you please! I
assure you it will be useless, you will gain nothing by it. On the
contrary, you'll have to answer, gentlemen, for violently obstructing
the course of justice. The thief has been more than unmasked, and I
shall prosecute. Our judges are not so blind and\ldots{} not so drunk,
and will not believe the testimony of two notorious infidels, agitators,
and atheists, who accuse me from motives of personal revenge which they
are foolish enough to admit\ldots{}. Yes, allow me to pass!''

``Don't let me find a trace of you in my room! Kindly leave at once, and
everything is at an end between us! When I think of the trouble I've
been taking, the way I've been expounding\ldots{} all this fortnight!''

``I told you myself to-day that I was going, when you tried to keep me;
now I will simply add that you are a fool. I advise you to see a doctor
for your brains and your short sight. Let me pass, gentlemen!''

He forced his way through. But the commissariat clerk was unwilling to
let him off so easily: he picked up a glass from the table, brandished
it in the air and flung it at Pyotr Petrovitch; but the glass flew
straight at Amalia Ivanovna. She screamed, and the clerk, overbalancing,
fell heavily under the table. Pyotr Petrovitch made his way to his room
and half an hour later had left the house. Sonia, timid by nature, had
felt before that day that she could be ill-treated more easily than
anyone, and that she could be wronged with impunity. Yet till that
moment she had fancied that she might escape misfortune by care,
gentleness and submissiveness before everyone. Her disappointment was
too great. She could, of course, bear with patience and almost without
murmur anything, even this. But for the first minute she felt it too
bitter. In spite of her triumph and her justification---when her first
terror and stupefaction had passed and she could understand it all
clearly---the feeling of her helplessness and of the wrong done to her
made her heart throb with anguish and she was overcome with hysterical
weeping. At last, unable to bear any more, she rushed out of the room
and ran home, almost immediately after Luzhin's departure. When amidst
loud laughter the glass flew at Amalia Ivanovna, it was more than the
landlady could endure. With a shriek she rushed like a fury at Katerina
Ivanovna, considering her to blame for everything.

``Out of my lodgings! At once! Quick march!''

And with these words she began snatching up everything she could lay her
hands on that belonged to Katerina Ivanovna, and throwing it on the
floor. Katerina Ivanovna, pale, almost fainting, and gasping for breath,
jumped up from the bed where she had sunk in exhaustion and darted at
Amalia Ivanovna. But the battle was too unequal: the landlady waved her
away like a feather.

``What! As though that godless calumny was not enough---this vile
creature attacks me! What! On the day of my husband's funeral I am
turned out of my lodging! After eating my bread and salt she turns me
into the street, with my orphans! Where am I to go?'' wailed the poor
woman, sobbing and gasping. ``Good God!'' she cried with flashing eyes,
``is there no justice upon earth? Whom should you protect if not us
orphans? We shall see! There is law and justice on earth, there is, I
will find it! Wait a bit, godless creature! Polenka, stay with the
children, I'll come back. Wait for me, if you have to wait in the
street. We will see whether there is justice on earth!''

And throwing over her head that green shawl which Marmeladov had
mentioned to Raskolnikov, Katerina Ivanovna squeezed her way through the
disorderly and drunken crowd of lodgers who still filled the room, and,
wailing and tearful, she ran into the street---with a vague intention of
going at once somewhere to find justice. Polenka with the two little
ones in her arms crouched, terrified, on the trunk in the corner of the
room, where she waited trembling for her mother to come back. Amalia
Ivanovna raged about the room, shrieking, lamenting and throwing
everything she came across on the floor. The lodgers talked
incoherently, some commented to the best of their ability on what had
happened, others quarrelled and swore at one another, while others
struck up a song\ldots{}.

``Now it's time for me to go,'' thought Raskolnikov. ``Well, Sofya
Semyonovna, we shall see what you'll say now!''

And he set off in the direction of Sonia's lodgings.

\protect\hypertarget{link2HCH0030}{}{}

\mychapter{4}{Chapter IV}\label{chapter-iv-4}

Raskolnikov had been a vigorous and active champion of Sonia against
Luzhin, although he had such a load of horror and anguish in his own
heart. But having gone through so much in the morning, he found a sort
of relief in a change of sensations, apart from the strong personal
feeling which impelled him to defend Sonia. He was agitated too,
especially at some moments, by the thought of his approaching interview
with Sonia: he \emph{had} to tell her who had killed Lizaveta. He knew
the terrible suffering it would be to him and, as it were, brushed away
the thought of it. So when he cried as he left Katerina Ivanovna's,
``Well, Sofya Semyonovna, we shall see what you'll say now!'' he was
still superficially excited, still vigorous and defiant from his triumph
over Luzhin. But, strange to say, by the time he reached Sonia's
lodging, he felt a sudden impotence and fear. He stood still in
hesitation at the door, asking himself the strange question: ``Must he
tell her who killed Lizaveta?'' It was a strange question because he
felt at the very time not only that he could not help telling her, but
also that he could not put off the telling. He did not yet know why it
must be so, he only \emph{felt} it, and the agonising sense of his
impotence before the inevitable almost crushed him. To cut short his
hesitation and suffering, he quickly opened the door and looked at Sonia
from the doorway. She was sitting with her elbows on the table and her
face in her hands, but seeing Raskolnikov she got up at once and came to
meet him as though she were expecting him.

``What would have become of me but for you?'' she said quickly, meeting
him in the middle of the room.

Evidently she was in haste to say this to him. It was what she had been
waiting for.

Raskolnikov went to the table and sat down on the chair from which she
had only just risen. She stood facing him, two steps away, just as she
had done the day before.

``Well, Sonia?'' he said, and felt that his voice was trembling, ``it
was all due to `your social position and the habits associated with it.'
Did you understand that just now?''

Her face showed her distress.

``Only don't talk to me as you did yesterday,'' she interrupted him.
``Please don't begin it. There is misery enough without that.''

She made haste to smile, afraid that he might not like the reproach.

``I was silly to come away from there. What is happening there now? I
wanted to go back directly, but I kept thinking that\ldots{} you would
come.''

He told her that Amalia Ivanovna was turning them out of their lodging
and that Katerina Ivanovna had run off somewhere ``to seek justice.''

``My God!'' cried Sonia, ``let's go at once\ldots{}.''

And she snatched up her cape.

``It's everlastingly the same thing!'' said Raskolnikov, irritably.
``You've no thought except for them! Stay a little with me.''

``But\ldots{} Katerina Ivanovna?''

``You won't lose Katerina Ivanovna, you may be sure, she'll come to you
herself since she has run out,'' he added peevishly. ``If she doesn't
find you here, you'll be blamed for it\ldots{}.''

Sonia sat down in painful suspense. Raskolnikov was silent, gazing at
the floor and deliberating.

``This time Luzhin did not want to prosecute you,'' he began, not
looking at Sonia, ``but if he had wanted to, if it had suited his plans,
he would have sent you to prison if it had not been for Lebeziatnikov
and me. Ah?''

``Yes,'' she assented in a faint voice. ``Yes,'' she repeated,
preoccupied and distressed.

``But I might easily not have been there. And it was quite an accident
Lebeziatnikov's turning up.''

Sonia was silent.

``And if you'd gone to prison, what then? Do you remember what I said
yesterday?''

Again she did not answer. He waited.

``I thought you would cry out again `don't speak of it, leave off.'\,''
Raskolnikov gave a laugh, but rather a forced one. ``What, silence
again?'' he asked a minute later. ``We must talk about something, you
know. It would be interesting for me to know how you would decide a
certain `problem' as Lebeziatnikov would say.'' (He was beginning to
lose the thread.) ``No, really, I am serious. Imagine, Sonia, that you
had known all Luzhin's intentions beforehand. Known, that is, for a
fact, that they would be the ruin of Katerina Ivanovna and the children
and yourself thrown in---since you don't count yourself for
anything---Polenka too\ldots{} for she'll go the same way. Well, if
suddenly it all depended on your decision whether he or they should go
on living, that is whether Luzhin should go on living and doing wicked
things, or Katerina Ivanovna should die? How would you decide which of
them was to die? I ask you?''

Sonia looked uneasily at him. There was something peculiar in this
hesitating question, which seemed approaching something in a roundabout
way.

``I felt that you were going to ask some question like that,'' she said,
looking inquisitively at him.

``I dare say you did. But how is it to be answered?''

``Why do you ask about what could not happen?'' said Sonia reluctantly.

``Then it would be better for Luzhin to go on living and doing wicked
things? You haven't dared to decide even that!''

``But I can't know the Divine Providence\ldots{}. And why do you ask
what can't be answered? What's the use of such foolish questions? How
could it happen that it should depend on my decision---who has made me a
judge to decide who is to live and who is not to live?''

``Oh, if the Divine Providence is to be mixed up in it, there is no
doing anything,'' Raskolnikov grumbled morosely.

``You'd better say straight out what you want!'' Sonia cried in
distress. ``You are leading up to something again\ldots{}. Can you have
come simply to torture me?''

She could not control herself and began crying bitterly. He looked at
her in gloomy misery. Five minutes passed.

``Of course you're right, Sonia,'' he said softly at last. He was
suddenly changed. His tone of assumed arrogance and helpless defiance
was gone. Even his voice was suddenly weak. ``I told you yesterday that
I was not coming to ask forgiveness and almost the first thing I've said
is to ask forgiveness\ldots{}. I said that about Luzhin and Providence
for my own sake. I was asking forgiveness, Sonia\ldots{}.''

He tried to smile, but there was something helpless and incomplete in
his pale smile. He bowed his head and hid his face in his hands.

And suddenly a strange, surprising sensation of a sort of bitter hatred
for Sonia passed through his heart. As it were wondering and frightened
of this sensation, he raised his head and looked intently at her; but he
met her uneasy and painfully anxious eyes fixed on him; there was love
in them; his hatred vanished like a phantom. It was not the real
feeling; he had taken the one feeling for the other. It only meant that
\emph{that} minute had come.

He hid his face in his hands again and bowed his head. Suddenly he
turned pale, got up from his chair, looked at Sonia, and without
uttering a word sat down mechanically on her bed.

His sensations that moment were terribly like the moment when he had
stood over the old woman with the axe in his hand and felt that ``he
must not lose another minute.''

``What's the matter?'' asked Sonia, dreadfully frightened.

He could not utter a word. This was not at all, not at all the way he
had intended to ``tell'' and he did not understand what was happening to
him now. She went up to him, softly, sat down on the bed beside him and
waited, not taking her eyes off him. Her heart throbbed and sank. It was
unendurable; he turned his deadly pale face to her. His lips worked,
helplessly struggling to utter something. A pang of terror passed
through Sonia's heart.

``What's the matter?'' she repeated, drawing a little away from him.

``Nothing, Sonia, don't be frightened\ldots{}. It's nonsense. It really
is nonsense, if you think of it,'' he muttered, like a man in delirium.
``Why have I come to torture you?'' he added suddenly, looking at her.
``Why, really? I keep asking myself that question, Sonia\ldots{}.''

He had perhaps been asking himself that question a quarter of an hour
before, but now he spoke helplessly, hardly knowing what he said and
feeling a continual tremor all over.

``Oh, how you are suffering!'' she muttered in distress, looking
intently at him.

``It's all nonsense\ldots{}. Listen, Sonia.'' He suddenly smiled, a pale
helpless smile for two seconds. ``You remember what I meant to tell you
yesterday?''

Sonia waited uneasily.

``I said as I went away that perhaps I was saying good-bye for ever, but
that if I came to-day I would tell you who\ldots{} who killed
Lizaveta.''

She began trembling all over.

``Well, here I've come to tell you.''

``Then you really meant it yesterday?'' she whispered with difficulty.
``How do you know?'' she asked quickly, as though suddenly regaining her
reason.

Sonia's face grew paler and paler, and she breathed painfully.

``I know.''

She paused a minute.

``Have they found him?'' she asked timidly.

``No.''

``Then how do you know about \emph{it}?'' she asked again, hardly
audibly and again after a minute's pause.

He turned to her and looked very intently at her.

``Guess,'' he said, with the same distorted helpless smile.

A shudder passed over her.

``But you\ldots{} why do you frighten me like this?'' she said, smiling
like a child.

``I must be a great friend of \emph{his}\ldots{} since I know,''
Raskolnikov went on, still gazing into her face, as though he could not
turn his eyes away. ``He\ldots{} did not mean to kill that
Lizaveta\ldots{} he\ldots{} killed her accidentally\ldots{}. He meant to
kill the old woman when she was alone and he went there\ldots{} and then
Lizaveta came in\ldots{} he killed her too.''

Another awful moment passed. Both still gazed at one another.

``You can't guess, then?'' he asked suddenly, feeling as though he were
flinging himself down from a steeple.

``N-no\ldots{}'' whispered Sonia.

``Take a good look.''

As soon as he had said this again, the same familiar sensation froze his
heart. He looked at her and all at once seemed to see in her face the
face of Lizaveta. He remembered clearly the expression in Lizaveta's
face, when he approached her with the axe and she stepped back to the
wall, putting out her hand, with childish terror in her face, looking as
little children do when they begin to be frightened of something,
looking intently and uneasily at what frightens them, shrinking back and
holding out their little hands on the point of crying. Almost the same
thing happened now to Sonia. With the same helplessness and the same
terror, she looked at him for a while and, suddenly putting out her left
hand, pressed her fingers faintly against his breast and slowly began to
get up from the bed, moving further from him and keeping her eyes fixed
even more immovably on him. Her terror infected him. The same fear
showed itself on his face. In the same way he stared at her and almost
with the same \emph{childish} smile.

``Have you guessed?'' he whispered at last.

``Good God!'' broke in an awful wail from her bosom.

She sank helplessly on the bed with her face in the pillows, but a
moment later she got up, moved quickly to him, seized both his hands
and, gripping them tight in her thin fingers, began looking into his
face again with the same intent stare. In this last desperate look she
tried to look into him and catch some last hope. But there was no hope;
there was no doubt remaining; it was all true! Later on, indeed, when
she recalled that moment, she thought it strange and wondered why she
had seen at once that there was no doubt. She could not have said, for
instance, that she had foreseen something of the sort---and yet now, as
soon as he told her, she suddenly fancied that she had really foreseen
this very thing.

``Stop, Sonia, enough! don't torture me,'' he begged her miserably.

It was not at all, not at all like this he had thought of telling her,
but this is how it happened.

She jumped up, seeming not to know what she was doing, and, wringing her
hands, walked into the middle of the room; but quickly went back and sat
down again beside him, her shoulder almost touching his. All of a sudden
she started as though she had been stabbed, uttered a cry and fell on
her knees before him, she did not know why.

``What have you done---what have you done to yourself?'' she said in
despair, and, jumping up, she flung herself on his neck, threw her arms
round him, and held him tightly.

Raskolnikov drew back and looked at her with a mournful smile.

``You are a strange girl, Sonia---you kiss me and hug me when I tell you
about that\ldots{}. You don't think what you are doing.''

``There is no one---no one in the whole world now so unhappy as you!''
she cried in a frenzy, not hearing what he said, and she suddenly broke
into violent hysterical weeping.

A feeling long unfamiliar to him flooded his heart and softened it at
once. He did not struggle against it. Two tears started into his eyes
and hung on his eyelashes.

``Then you won't leave me, Sonia?'' he said, looking at her almost with
hope.

``No, no, never, nowhere!'' cried Sonia. ``I will follow you, I will
follow you everywhere. Oh, my God! Oh, how miserable I am!\ldots{} Why,
why didn't I know you before! Why didn't you come before? Oh, dear!''

``Here I have come.''

``Yes, now! What's to be done now?\ldots{} Together, together!'' she
repeated as it were unconsciously, and she hugged him again. ``I'll
follow you to Siberia!''

He recoiled at this, and the same hostile, almost haughty smile came to
his lips.

``Perhaps I don't want to go to Siberia yet, Sonia,'' he said.

Sonia looked at him quickly.

Again after her first passionate, agonising sympathy for the unhappy man
the terrible idea of the murder overwhelmed her. In his changed tone she
seemed to hear the murderer speaking. She looked at him bewildered. She
knew nothing as yet, why, how, with what object it had been. Now all
these questions rushed at once into her mind. And again she could not
believe it: ``He, he is a murderer! Could it be true?''

``What's the meaning of it? Where am I?'' she said in complete
bewilderment, as though still unable to recover herself. ``How could
you, you, a man like you\ldots{}. How could you bring yourself to
it?\ldots{} What does it mean?''

``Oh, well---to plunder. Leave off, Sonia,'' he answered wearily, almost
with vexation.

Sonia stood as though struck dumb, but suddenly she cried:

``You were hungry! It was\ldots{} to help your mother? Yes?''

``No, Sonia, no,'' he muttered, turning away and hanging his head. ``I
was not so hungry\ldots{}. I certainly did want to help my mother,
but\ldots{} that's not the real thing either\ldots{}. Don't torture me,
Sonia.''

Sonia clasped her hands.

``Could it, could it all be true? Good God, what a truth! Who could
believe it? And how could you give away your last farthing and yet rob
and murder! Ah,'' she cried suddenly, ``that money you gave Katerina
Ivanovna\ldots{} that money\ldots{}. Can that money\ldots{}''

``No, Sonia,'' he broke in hurriedly, ``that money was not it. Don't
worry yourself! That money my mother sent me and it came when I was ill,
the day I gave it to you\ldots{}. Razumihin saw it\ldots{} he received
it for me\ldots{}. That money was mine---my own.''

Sonia listened to him in bewilderment and did her utmost to comprehend.

``And \emph{that} money\ldots{}. I don't even know really whether there
was any money,'' he added softly, as though reflecting. ``I took a purse
off her neck, made of chamois leather\ldots{} a purse stuffed full of
something\ldots{} but I didn't look in it; I suppose I hadn't
time\ldots{}. And the things---chains and trinkets---I buried under a
stone with the purse next morning in a yard off the V------ Prospect.
They are all there now\ldots{}.''

Sonia strained every nerve to listen.

``Then why\ldots{} why, you said you did it to rob, but you took
nothing?'' she asked quickly, catching at a straw.

``I don't know\ldots{}. I haven't yet decided whether to take that money
or not,'' he said, musing again; and, seeming to wake up with a start,
he gave a brief ironical smile. ``Ach, what silly stuff I am talking,
eh?''

The thought flashed through Sonia's mind, wasn't he mad? But she
dismissed it at once. ``No, it was something else.'' She could make
nothing of it, nothing.

``Do you know, Sonia,'' he said suddenly with conviction, ``let me tell
you: if I'd simply killed because I was hungry,'' laying stress on every
word and looking enigmatically but sincerely at her, ``I should be
\emph{happy} now. You must believe that! What would it matter to you,''
he cried a moment later with a sort of despair, ``what would it matter
to you if I were to confess that I did wrong? What do you gain by such a
stupid triumph over me? Ah, Sonia, was it for that I've come to you
to-day?''

Again Sonia tried to say something, but did not speak.

``I asked you to go with me yesterday because you are all I have left.''

``Go where?'' asked Sonia timidly.

``Not to steal and not to murder, don't be anxious,'' he smiled
bitterly. ``We are so different\ldots{}. And you know, Sonia, it's only
now, only this moment that I understand \emph{where} I asked you to go
with me yesterday! Yesterday when I said it I did not know where. I
asked you for one thing, I came to you for one thing---not to leave me.
You won't leave me, Sonia?''

She squeezed his hand.

``And why, why did I tell her? Why did I let her know?'' he cried a
minute later in despair, looking with infinite anguish at her. ``Here
you expect an explanation from me, Sonia; you are sitting and waiting
for it, I see that. But what can I tell you? You won't understand and
will only suffer misery\ldots{} on my account! Well, you are crying and
embracing me again. Why do you do it? Because I couldn't bear my burden
and have come to throw it on another: you suffer too, and I shall feel
better! And can you love such a mean wretch?''

``But aren't you suffering, too?'' cried Sonia.

Again a wave of the same feeling surged into his heart, and again for an
instant softened it.

``Sonia, I have a bad heart, take note of that. It may explain a great
deal. I have come because I am bad. There are men who wouldn't have
come. But I am a coward and\ldots{} a mean wretch. But\ldots{} never
mind! That's not the point. I must speak now, but I don't know how to
begin.''

He paused and sank into thought.

``Ach, we are so different,'' he cried again, ``we are not alike. And
why, why did I come? I shall never forgive myself that.''

``No, no, it was a good thing you came,'' cried Sonia. ``It's better I
should know, far better!''

He looked at her with anguish.

``What if it were really that?'' he said, as though reaching a
conclusion. ``Yes, that's what it was! I wanted to become a Napoleon,
that is why I killed her\ldots{}. Do you understand now?''

``N-no,'' Sonia whispered naïvely and timidly. ``Only speak, speak, I
shall understand, I shall understand \emph{in myself}!'' she kept
begging him.

``You'll understand? Very well, we shall see!'' He paused and was for
some time lost in meditation.

``It was like this: I asked myself one day this question---what if
Napoleon, for instance, had happened to be in my place, and if he had
not had Toulon nor Egypt nor the passage of Mont Blanc to begin his
career with, but instead of all those picturesque and monumental things,
there had simply been some ridiculous old hag, a pawnbroker, who had to
be murdered too to get money from her trunk (for his career, you
understand). Well, would he have brought himself to that if there had
been no other means? Wouldn't he have felt a pang at its being so far
from monumental and\ldots{} and sinful, too? Well, I must tell you that
I worried myself fearfully over that `question' so that I was awfully
ashamed when I guessed at last (all of a sudden, somehow) that it would
not have given him the least pang, that it would not even have struck
him that it was not monumental\ldots{} that he would not have seen that
there was anything in it to pause over, and that, if he had had no other
way, he would have strangled her in a minute without thinking about it!
Well, I too\ldots{} left off thinking about it\ldots{} murdered her,
following his example. And that's exactly how it was! Do you think it
funny? Yes, Sonia, the funniest thing of all is that perhaps that's just
how it was.''

Sonia did not think it at all funny.

``You had better tell me straight out\ldots{} without examples,'' she
begged, still more timidly and scarcely audibly.

He turned to her, looked sadly at her and took her hands.

``You are right again, Sonia. Of course that's all nonsense, it's almost
all talk! You see, you know of course that my mother has scarcely
anything, my sister happened to have a good education and was condemned
to drudge as a governess. All their hopes were centered on me. I was a
student, but I couldn't keep myself at the university and was forced for
a time to leave it. Even if I had lingered on like that, in ten or
twelve years I might (with luck) hope to be some sort of teacher or
clerk with a salary of a thousand roubles'' (he repeated it as though it
were a lesson) ``and by that time my mother would be worn out with grief
and anxiety and I could not succeed in keeping her in comfort while my
sister\ldots{} well, my sister might well have fared worse! And it's a
hard thing to pass everything by all one's life, to turn one's back upon
everything, to forget one's mother and decorously accept the insults
inflicted on one's sister. Why should one? When one has buried them to
burden oneself with others---wife and children---and to leave them again
without a farthing? So I resolved to gain possession of the old woman's
money and to use it for my first years without worrying my mother, to
keep myself at the university and for a little while after leaving
it---and to do this all on a broad, thorough scale, so as to build up a
completely new career and enter upon a new life of independence\ldots{}.
Well\ldots{} that's all\ldots{}. Well, of course in killing the old
woman I did wrong\ldots{}. Well, that's enough.''

He struggled to the end of his speech in exhaustion and let his head
sink.

``Oh, that's not it, that's not it,'' Sonia cried in distress. ``How
could one\ldots{} no, that's not right, not right.''

``You see yourself that it's not right. But I've spoken truly, it's the
truth.''

``As though that could be the truth! Good God!''

``I've only killed a louse, Sonia, a useless, loathsome, harmful
creature.''

``A human being---a louse!''

``I too know it wasn't a louse,'' he answered, looking strangely at her.
``But I am talking nonsense, Sonia,'' he added. ``I've been talking
nonsense a long time\ldots{}. That's not it, you are right there. There
were quite, quite other causes for it! I haven't talked to anyone for so
long, Sonia\ldots{}. My head aches dreadfully now.''

His eyes shone with feverish brilliance. He was almost delirious; an
uneasy smile strayed on his lips. His terrible exhaustion could be seen
through his excitement. Sonia saw how he was suffering. She too was
growing dizzy. And he talked so strangely; it seemed somehow
comprehensible, but yet\ldots{} ``But how, how! Good God!'' And she
wrung her hands in despair.

``No, Sonia, that's not it,'' he began again suddenly, raising his head,
as though a new and sudden train of thought had struck and as it were
roused him---``that's not it! Better\ldots{} imagine---yes, it's
certainly better---imagine that I am vain, envious, malicious, base,
vindictive and\ldots{} well, perhaps with a tendency to insanity. (Let's
have it all out at once! They've talked of madness already, I noticed.)
I told you just now I could not keep myself at the university. But do
you know that perhaps I might have done? My mother would have sent me
what I needed for the fees and I could have earned enough for clothes,
boots and food, no doubt. Lessons had turned up at half a rouble.
Razumihin works! But I turned sulky and wouldn't. (Yes, sulkiness,
that's the right word for it!) I sat in my room like a spider. You've
been in my den, you've seen it\ldots{}. And do you know, Sonia, that low
ceilings and tiny rooms cramp the soul and the mind? Ah, how I hated
that garret! And yet I wouldn't go out of it! I wouldn't on purpose! I
didn't go out for days together, and I wouldn't work, I wouldn't even
eat, I just lay there doing nothing. If Nastasya brought me anything, I
ate it, if she didn't, I went all day without; I wouldn't ask, on
purpose, from sulkiness! At night I had no light, I lay in the dark and
I wouldn't earn money for candles. I ought to have studied, but I sold
my books; and the dust lies an inch thick on the notebooks on my table.
I preferred lying still and thinking. And I kept thinking\ldots{}. And I
had dreams all the time, strange dreams of all sorts, no need to
describe! Only then I began to fancy that\ldots{} No, that's not it!
Again I am telling you wrong! You see I kept asking myself then: why am
I so stupid that if others are stupid---and I know they are---yet I
won't be wiser? Then I saw, Sonia, that if one waits for everyone to get
wiser it will take too long\ldots{}. Afterwards I understood that that
would never come to pass, that men won't change and that nobody can
alter it and that it's not worth wasting effort over it. Yes, that's so.
That's the law of their nature, Sonia,\ldots{} that's so!\ldots{} And I
know now, Sonia, that whoever is strong in mind and spirit will have
power over them. Anyone who is greatly daring is right in their eyes. He
who despises most things will be a lawgiver among them and he who dares
most of all will be most in the right! So it has been till now and so it
will always be. A man must be blind not to see it!''

Though Raskolnikov looked at Sonia as he said this, he no longer cared
whether she understood or not. The fever had complete hold of him; he
was in a sort of gloomy ecstasy (he certainly had been too long without
talking to anyone). Sonia felt that his gloomy creed had become his
faith and code.

``I divined then, Sonia,'' he went on eagerly, ``that power is only
vouchsafed to the man who dares to stoop and pick it up. There is only
one thing, one thing needful: one has only to dare! Then for the first
time in my life an idea took shape in my mind which no one had ever
thought of before me, no one! I saw clear as daylight how strange it is
that not a single person living in this mad world has had the daring to
go straight for it all and send it flying to the devil! I\ldots{} I
wanted \emph{to have the daring}\ldots{} and I killed her. I only wanted
to have the daring, Sonia! That was the whole cause of it!''

``Oh hush, hush,'' cried Sonia, clasping her hands. ``You turned away
from God and God has smitten you, has given you over to the devil!''

``Then Sonia, when I used to lie there in the dark and all this became
clear to me, was it a temptation of the devil, eh?''

``Hush, don't laugh, blasphemer! You don't understand, you don't
understand! Oh God! He won't understand!''

``Hush, Sonia! I am not laughing. I know myself that it was the devil
leading me. Hush, Sonia, hush!'' he repeated with gloomy insistence. ``I
know it all, I have thought it all over and over and whispered it all
over to myself, lying there in the dark\ldots{}. I've argued it all over
with myself, every point of it, and I know it all, all! And how sick,
how sick I was then of going over it all! I have kept wanting to forget
it and make a new beginning, Sonia, and leave off thinking. And you
don't suppose that I went into it headlong like a fool? I went into it
like a wise man, and that was just my destruction. And you mustn't
suppose that I didn't know, for instance, that if I began to question
myself whether I had the right to gain power---I certainly hadn't the
right---or that if I asked myself whether a human being is a louse it
proved that it wasn't so for me, though it might be for a man who would
go straight to his goal without asking questions\ldots{}. If I worried
myself all those days, wondering whether Napoleon would have done it or
not, I felt clearly of course that I wasn't Napoleon. I had to endure
all the agony of that battle of ideas, Sonia, and I longed to throw it
off: I wanted to murder without casuistry, to murder for my own sake,
for myself alone! I didn't want to lie about it even to myself. It
wasn't to help my mother I did the murder---that's nonsense---I didn't
do the murder to gain wealth and power and to become a benefactor of
mankind. Nonsense! I simply did it; I did the murder for myself, for
myself alone, and whether I became a benefactor to others, or spent my
life like a spider catching men in my web and sucking the life out of
men, I couldn't have cared at that moment\ldots{}. And it was not the
money I wanted, Sonia, when I did it. It was not so much the money I
wanted, but something else\ldots{}. I know it all now\ldots{}.
Understand me! Perhaps I should never have committed a murder again. I
wanted to find out something else; it was something else led me on. I
wanted to find out then and quickly whether I was a louse like everybody
else or a man. Whether I can step over barriers or not, whether I dare
stoop to pick up or not, whether I am a trembling creature or whether I
have the \emph{right}\ldots{}''

``To kill? Have the right to kill?'' Sonia clasped her hands.

``Ach, Sonia!'' he cried irritably and seemed about to make some retort,
but was contemptuously silent. ``Don't interrupt me, Sonia. I want to
prove one thing only, that the devil led me on then and he has shown me
since that I had not the right to take that path, because I am just such
a louse as all the rest. He was mocking me and here I've come to you
now! Welcome your guest! If I were not a louse, should I have come to
you? Listen: when I went then to the old woman's I only went to
\emph{try}\ldots{}. You may be sure of that!''

``And you murdered her!''

``But how did I murder her? Is that how men do murders? Do men go to
commit a murder as I went then? I will tell you some day how I went! Did
I murder the old woman? I murdered myself, not her! I crushed myself
once for all, for ever\ldots{}. But it was the devil that killed that
old woman, not I. Enough, enough, Sonia, enough! Let me be!'' he cried
in a sudden spasm of agony, ``let me be!''

He leaned his elbows on his knees and squeezed his head in his hands as
in a vise.

``What suffering!'' A wail of anguish broke from Sonia.

``Well, what am I to do now?'' he asked, suddenly raising his head and
looking at her with a face hideously distorted by despair.

``What are you to do?'' she cried, jumping up, and her eyes that had
been full of tears suddenly began to shine. ``Stand up!'' (She seized
him by the shoulder, he got up, looking at her almost bewildered.) ``Go
at once, this very minute, stand at the cross-roads, bow down, first
kiss the earth which you have defiled and then bow down to all the world
and say to all men aloud, `I am a murderer!' Then God will send you life
again. Will you go, will you go?'' she asked him, trembling all over,
snatching his two hands, squeezing them tight in hers and gazing at him
with eyes full of fire.

He was amazed at her sudden ecstasy.

``You mean Siberia, Sonia? I must give myself up?'' he asked gloomily.

``Suffer and expiate your sin by it, that's what you must do.''

``No! I am not going to them, Sonia!''

``But how will you go on living? What will you live for?'' cried Sonia,
``how is it possible now? Why, how can you talk to your mother? (Oh,
what will become of them now?) But what am I saying? You have abandoned
your mother and your sister already. He has abandoned them already! Oh,
God!'' she cried, ``why, he knows it all himself. How, how can he live
by himself! What will become of you now?''

``Don't be a child, Sonia,'' he said softly. ``What wrong have I done
them? Why should I go to them? What should I say to them? That's only a
phantom\ldots{}. They destroy men by millions themselves and look on it
as a virtue. They are knaves and scoundrels, Sonia! I am not going to
them. And what should I say to them---that I murdered her, but did not
dare to take the money and hid it under a stone?'' he added with a
bitter smile. ``Why, they would laugh at me, and would call me a fool
for not getting it. A coward and a fool! They wouldn't understand and
they don't deserve to understand. Why should I go to them? I won't.
Don't be a child, Sonia\ldots{}.''

``It will be too much for you to bear, too much!'' she repeated, holding
out her hands in despairing supplication.

``Perhaps I've been unfair to myself,'' he observed gloomily, pondering,
``perhaps after all I am a man and not a louse and I've been in too
great a hurry to condemn myself. I'll make another fight for it.''

A haughty smile appeared on his lips.

``What a burden to bear! And your whole life, your whole life!''

``I shall get used to it,'' he said grimly and thoughtfully. ``Listen,''
he began a minute later, ``stop crying, it's time to talk of the facts:
I've come to tell you that the police are after me, on my
track\ldots{}.''

``Ach!'' Sonia cried in terror.

``Well, why do you cry out? You want me to go to Siberia and now you are
frightened? But let me tell you: I shall not give myself up. I shall
make a struggle for it and they won't do anything to me. They've no real
evidence. Yesterday I was in great danger and believed I was lost; but
to-day things are going better. All the facts they know can be explained
two ways, that's to say I can turn their accusations to my credit, do
you understand? And I shall, for I've learnt my lesson. But they will
certainly arrest me. If it had not been for something that happened,
they would have done so to-day for certain; perhaps even now they will
arrest me to-day\ldots{}. But that's no matter, Sonia; they'll let me
out again\ldots{} for there isn't any real proof against me, and there
won't be, I give you my word for it. And they can't convict a man on
what they have against me. Enough\ldots{}. I only tell you that you may
know\ldots{}. I will try to manage somehow to put it to my mother and
sister so that they won't be frightened\ldots{}. My sister's future is
secure, however, now, I believe\ldots{} and my mother's must be
too\ldots{}. Well, that's all. Be careful, though. Will you come and see
me in prison when I am there?''

``Oh, I will, I will.''

They sat side by side, both mournful and dejected, as though they had
been cast up by the tempest alone on some deserted shore. He looked at
Sonia and felt how great was her love for him, and strange to say he
felt it suddenly burdensome and painful to be so loved. Yes, it was a
strange and awful sensation! On his way to see Sonia he had felt that
all his hopes rested on her; he expected to be rid of at least part of
his suffering, and now, when all her heart turned towards him, he
suddenly felt that he was immeasurably unhappier than before.

``Sonia,'' he said, ``you'd better not come and see me when I am in
prison.''

Sonia did not answer, she was crying. Several minutes passed.

``Have you a cross on you?'' she asked, as though suddenly thinking of
it.

He did not at first understand the question.

``No, of course not. Here, take this one, of cypress wood. I have
another, a copper one that belonged to Lizaveta. I changed with
Lizaveta: she gave me her cross and I gave her my little ikon. I will
wear Lizaveta's now and give you this. Take it\ldots{} it's mine! It's
mine, you know,'' she begged him. ``We will go to suffer together, and
together we will bear our cross!''

``Give it me,'' said Raskolnikov.

He did not want to hurt her feelings. But immediately he drew back the
hand he held out for the cross.

``Not now, Sonia. Better later,'' he added to comfort her.

``Yes, yes, better,'' she repeated with conviction, ``when you go to
meet your suffering, then put it on. You will come to me, I'll put it on
you, we will pray and go together.''

At that moment someone knocked three times at the door.

``Sofya Semyonovna, may I come in?'' they heard in a very familiar and
polite voice.

Sonia rushed to the door in a fright. The flaxen head of Mr.
Lebeziatnikov appeared at the door.

\protect\hypertarget{link2HCH0031}{}{}

\mychapter{5}{Chapter V}\label{chapter-v-4}

Lebeziatnikov looked perturbed.

``I've come to you, Sofya Semyonovna,'' he began. ``Excuse me\ldots{} I
thought I should find you,'' he said, addressing Raskolnikov suddenly,
``that is, I didn't mean anything\ldots{} of that sort\ldots{} But I
just thought\ldots{} Katerina Ivanovna has gone out of her mind,'' he
blurted out suddenly, turning from Raskolnikov to Sonia.

Sonia screamed.

``At least it seems so. But\ldots{} we don't know what to do, you see!
She came back---she seems to have been turned out somewhere, perhaps
beaten\ldots{}. So it seems at least,\ldots{} She had run to your
father's former chief, she didn't find him at home: he was dining at
some other general's\ldots{}. Only fancy, she rushed off there, to the
other general's, and, imagine, she was so persistent that she managed to
get the chief to see her, had him fetched out from dinner, it seems. You
can imagine what happened. She was turned out, of course; but, according
to her own story, she abused him and threw something at him. One may
well believe it\ldots{}. How it is she wasn't taken up, I can't
understand! Now she is telling everyone, including Amalia Ivanovna; but
it's difficult to understand her, she is screaming and flinging herself
about\ldots{}. Oh yes, she shouts that since everyone has abandoned her,
she will take the children and go into the street with a barrel-organ,
and the children will sing and dance, and she too, and collect money,
and will go every day under the general's window\ldots{} `to let
everyone see well-born children, whose father was an official, begging
in the street.' She keeps beating the children and they are all crying.
She is teaching Lida to sing `My Village,' the boy to dance, Polenka the
same. She is tearing up all the clothes, and making them little caps
like actors; she means to carry a tin basin and make it tinkle, instead
of music\ldots{}. She won't listen to anything\ldots{}. Imagine the
state of things! It's beyond anything!''

Lebeziatnikov would have gone on, but Sonia, who had heard him almost
breathless, snatched up her cloak and hat, and ran out of the room,
putting on her things as she went. Raskolnikov followed her and
Lebeziatnikov came after him.

``She has certainly gone mad!'' he said to Raskolnikov, as they went out
into the street. ``I didn't want to frighten Sofya Semyonovna, so I said
`it seemed like it,' but there isn't a doubt of it. They say that in
consumption the tubercles sometimes occur in the brain; it's a pity I
know nothing of medicine. I did try to persuade her, but she wouldn't
listen.''

``Did you talk to her about the tubercles?''

``Not precisely of the tubercles. Besides, she wouldn't have understood!
But what I say is, that if you convince a person logically that he has
nothing to cry about, he'll stop crying. That's clear. Is it your
conviction that he won't?''

``Life would be too easy if it were so,'' answered Raskolnikov.

``Excuse me, excuse me; of course it would be rather difficult for
Katerina Ivanovna to understand, but do you know that in Paris they have
been conducting serious experiments as to the possibility of curing the
insane, simply by logical argument? One professor there, a scientific
man of standing, lately dead, believed in the possibility of such
treatment. His idea was that there's nothing really wrong with the
physical organism of the insane, and that insanity is, so to say, a
logical mistake, an error of judgment, an incorrect view of things. He
gradually showed the madman his error and, would you believe it, they
say he was successful? But as he made use of douches too, how far
success was due to that treatment remains uncertain\ldots{}. So it seems
at least.''

Raskolnikov had long ceased to listen. Reaching the house where he
lived, he nodded to Lebeziatnikov and went in at the gate. Lebeziatnikov
woke up with a start, looked about him and hurried on.

Raskolnikov went into his little room and stood still in the middle of
it. Why had he come back here? He looked at the yellow and tattered
paper, at the dust, at his sofa\ldots{}. From the yard came a loud
continuous knocking; someone seemed to be hammering\ldots{} He went to
the window, rose on tiptoe and looked out into the yard for a long time
with an air of absorbed attention. But the yard was empty and he could
not see who was hammering. In the house on the left he saw some open
windows; on the window-sills were pots of sickly-looking geraniums.
Linen was hung out of the windows\ldots{} He knew it all by heart. He
turned away and sat down on the sofa.

Never, never had he felt himself so fearfully alone!

Yes, he felt once more that he would perhaps come to hate Sonia, now
that he had made her more miserable.

``Why had he gone to her to beg for her tears? What need had he to
poison her life? Oh, the meanness of it!''

``I will remain alone,'' he said resolutely, ``and she shall not come to
the prison!''

Five minutes later he raised his head with a strange smile. That was a
strange thought.

``Perhaps it really would be better in Siberia,'' he thought suddenly.

He could not have said how long he sat there with vague thoughts surging
through his mind. All at once the door opened and Dounia came in. At
first she stood still and looked at him from the doorway, just as he had
done at Sonia; then she came in and sat down in the same place as
yesterday, on the chair facing him. He looked silently and almost
vacantly at her.

``Don't be angry, brother; I've only come for one minute,'' said Dounia.

Her face looked thoughtful but not stern. Her eyes were bright and soft.
He saw that she too had come to him with love.

``Brother, now I know all, \emph{all}. Dmitri Prokofitch has explained
and told me everything. They are worrying and persecuting you through a
stupid and contemptible suspicion\ldots{}. Dmitri Prokofitch told me
that there is no danger, and that you are wrong in looking upon it with
such horror. I don't think so, and I fully understand how indignant you
must be, and that that indignation may have a permanent effect on you.
That's what I am afraid of. As for your cutting yourself off from us, I
don't judge you, I don't venture to judge you, and forgive me for having
blamed you for it. I feel that I too, if I had so great a trouble,
should keep away from everyone. I shall tell mother nothing \emph{of
this}, but I shall talk about you continually and shall tell her from
you that you will come very soon. Don't worry about her; \emph{I} will
set her mind at rest; but don't you try her too much---come once at
least; remember that she is your mother. And now I have come simply to
say'' (Dounia began to get up) ``that if you should need me or should
need\ldots{} all my life or anything\ldots{} call me, and I'll come.
Good-bye!''

She turned abruptly and went towards the door.

``Dounia!'' Raskolnikov stopped her and went towards her. ``That
Razumihin, Dmitri Prokofitch, is a very good fellow.''

Dounia flushed slightly.

``Well?'' she asked, waiting a moment.

``He is competent, hardworking, honest and capable of real love\ldots{}.
Good-bye, Dounia.''

Dounia flushed crimson, then suddenly she took alarm.

``But what does it mean, brother? Are we really parting for ever that
you\ldots{} give me such a parting message?''

``Never mind\ldots{}. Good-bye.''

He turned away, and walked to the window. She stood a moment, looked at
him uneasily, and went out troubled.

No, he was not cold to her. There was an instant (the very last one)
when he had longed to take her in his arms and \emph{say good-bye} to
her, and even \emph{to tell} her, but he had not dared even to touch her
hand.

``Afterwards she may shudder when she remembers that I embraced her, and
will feel that I stole her kiss.''

``And would \emph{she} stand that test?'' he went on a few minutes later
to himself. ``No, she wouldn't; girls like that can't stand things! They
never do.''

And he thought of Sonia.

There was a breath of fresh air from the window. The daylight was
fading. He took up his cap and went out.

He could not, of course, and would not consider how ill he was. But all
this continual anxiety and agony of mind could not but affect him. And
if he were not lying in high fever it was perhaps just because this
continual inner strain helped to keep him on his legs and in possession
of his faculties. But this artificial excitement could not last long.

He wandered aimlessly. The sun was setting. A special form of misery had
begun to oppress him of late. There was nothing poignant, nothing acute
about it; but there was a feeling of permanence, of eternity about it;
it brought a foretaste of hopeless years of this cold leaden misery, a
foretaste of an eternity ``on a square yard of space.'' Towards evening
this sensation usually began to weigh on him more heavily.

``With this idiotic, purely physical weakness, depending on the sunset
or something, one can't help doing something stupid! You'll go to
Dounia, as well as to Sonia,'' he muttered bitterly.

He heard his name called. He looked round. Lebeziatnikov rushed up to
him.

``Only fancy, I've been to your room looking for you. Only fancy, she's
carried out her plan, and taken away the children. Sofya Semyonovna and
I have had a job to find them. She is rapping on a frying-pan and making
the children dance. The children are crying. They keep stopping at the
cross-roads and in front of shops; there's a crowd of fools running
after them. Come along!''

``And Sonia?'' Raskolnikov asked anxiously, hurrying after
Lebeziatnikov.

``Simply frantic. That is, it's not Sofya Semyonovna's frantic, but
Katerina Ivanovna, though Sofya Semyonova's frantic too. But Katerina
Ivanovna is absolutely frantic. I tell you she is quite mad. They'll be
taken to the police. You can fancy what an effect that will
have\ldots{}. They are on the canal bank, near the bridge now, not far
from Sofya Semyonovna's, quite close.''

On the canal bank near the bridge and not two houses away from the one
where Sonia lodged, there was a crowd of people, consisting principally
of gutter children. The hoarse broken voice of Katerina Ivanovna could
be heard from the bridge, and it certainly was a strange spectacle
likely to attract a street crowd. Katerina Ivanovna in her old dress
with the green shawl, wearing a torn straw hat, crushed in a hideous way
on one side, was really frantic. She was exhausted and breathless. Her
wasted consumptive face looked more suffering than ever, and indeed out
of doors in the sunshine a consumptive always looks worse than at home.
But her excitement did not flag, and every moment her irritation grew
more intense. She rushed at the children, shouted at them, coaxed them,
told them before the crowd how to dance and what to sing, began
explaining to them why it was necessary, and driven to desperation by
their not understanding, beat them\ldots{}. Then she would make a rush
at the crowd; if she noticed any decently dressed person stopping to
look, she immediately appealed to him to see what these children ``from
a genteel, one may say aristocratic, house'' had been brought to. If she
heard laughter or jeering in the crowd, she would rush at once at the
scoffers and begin squabbling with them. Some people laughed, others
shook their heads, but everyone felt curious at the sight of the
madwoman with the frightened children. The frying-pan of which
Lebeziatnikov had spoken was not there, at least Raskolnikov did not see
it. But instead of rapping on the pan, Katerina Ivanovna began clapping
her wasted hands, when she made Lida and Kolya dance and Polenka sing.
She too joined in the singing, but broke down at the second note with a
fearful cough, which made her curse in despair and even shed tears. What
made her most furious was the weeping and terror of Kolya and Lida. Some
effort had been made to dress the children up as street singers are
dressed. The boy had on a turban made of something red and white to look
like a Turk. There had been no costume for Lida; she simply had a red
knitted cap, or rather a night cap that had belonged to Marmeladov,
decorated with a broken piece of white ostrich feather, which had been
Katerina Ivanovna's grandmother's and had been preserved as a family
possession. Polenka was in her everyday dress; she looked in timid
perplexity at her mother, and kept at her side, hiding her tears. She
dimly realised her mother's condition, and looked uneasily about her.
She was terribly frightened of the street and the crowd. Sonia followed
Katerina Ivanovna, weeping and beseeching her to return home, but
Katerina Ivanovna was not to be persuaded.

``Leave off, Sonia, leave off,'' she shouted, speaking fast, panting and
coughing. ``You don't know what you ask; you are like a child! I've told
you before that I am not coming back to that drunken German. Let
everyone, let all Petersburg see the children begging in the streets,
though their father was an honourable man who served all his life in
truth and fidelity, and one may say died in the service.'' (Katerina
Ivanovna had by now invented this fantastic story and thoroughly
believed it.) ``Let that wretch of a general see it! And you are silly,
Sonia: what have we to eat? Tell me that. We have worried you enough, I
won't go on so! Ah, Rodion Romanovitch, is that you?'' she cried, seeing
Raskolnikov and rushing up to him. ``Explain to this silly girl, please,
that nothing better could be done! Even organ-grinders earn their
living, and everyone will see at once that we are different, that we are
an honourable and bereaved family reduced to beggary. And that general
will lose his post, you'll see! We shall perform under his windows every
day, and if the Tsar drives by, I'll fall on my knees, put the children
before me, show them to him, and say `Defend us father.' He is the
father of the fatherless, he is merciful, he'll protect us, you'll see,
and that wretch of a general\ldots{}. Lida, \emph{tenez vous droite}!
Kolya, you'll dance again. Why are you whimpering? Whimpering again!
What are you afraid of, stupid? Goodness, what am I to do with them,
Rodion Romanovitch? If you only knew how stupid they are! What's one to
do with such children?''

And she, almost crying herself---which did not stop her uninterrupted,
rapid flow of talk---pointed to the crying children. Raskolnikov tried
to persuade her to go home, and even said, hoping to work on her vanity,
that it was unseemly for her to be wandering about the streets like an
organ-grinder, as she was intending to become the principal of a
boarding-school.

``A boarding-school, ha-ha-ha! A castle in the air,'' cried Katerina
Ivanovna, her laugh ending in a cough. ``No, Rodion Romanovitch, that
dream is over! All have forsaken us!\ldots{} And that general\ldots{}.
You know, Rodion Romanovitch, I threw an inkpot at him---it happened to
be standing in the waiting-room by the paper where you sign your name. I
wrote my name, threw it at him and ran away. Oh, the scoundrels, the
scoundrels! But enough of them, now I'll provide for the children
myself, I won't bow down to anybody! She has had to bear enough for
us!'' she pointed to Sonia. ``Polenka, how much have you got? Show me!
What, only two farthings! Oh, the mean wretches! They give us nothing,
only run after us, putting their tongues out. There, what is that
blockhead laughing at?'' (She pointed to a man in the crowd.) ``It's all
because Kolya here is so stupid; I have such a bother with him. What do
you want, Polenka? Tell me in French, \emph{parlez-moi français}. Why,
I've taught you, you know some phrases. Else how are you to show that
you are of good family, well brought-up children, and not at all like
other organ-grinders? We aren't going to have a Punch and Judy show in
the street, but to sing a genteel song\ldots{}. Ah, yes,\ldots{} What
are we to sing? You keep putting me out, but we\ldots{} you see, we are
standing here, Rodion Romanovitch, to find something to sing and get
money, something Kolya can dance to\ldots{}. For, as you can fancy, our
performance is all impromptu\ldots{}. We must talk it over and rehearse
it all thoroughly, and then we shall go to Nevsky, where there are far
more people of good society, and we shall be noticed at once. Lida knows
`My Village' only, nothing but `My Village,' and everyone sings that. We
must sing something far more genteel\ldots{}. Well, have you thought of
anything, Polenka? If only you'd help your mother! My memory's quite
gone, or I should have thought of something. We really can't sing `An
Hussar.' Ah, let us sing in French, `Cinq sous,' I have taught it you, I
have taught it you. And as it is in French, people will see at once that
you are children of good family, and that will be much more
touching\ldots{}. You might sing `Marlborough s'en va-t-en guerre,' for
that's quite a child's song and is sung as a lullaby in all the
aristocratic houses.

``\emph{Marlborough s'en va-t-en guerre Ne sait quand
reviendra}\ldots{}'' she began singing. ``But no, better sing `Cinq
sous.' Now, Kolya, your hands on your hips, make haste, and you, Lida,
keep turning the other way, and Polenka and I will sing and clap our
hands!

``\emph{Cinq sous, cinq sous Pour monter notre menage}.''

(Cough-cough-cough!) ``Set your dress straight, Polenka, it's slipped
down on your shoulders,'' she observed, panting from coughing. ``Now
it's particularly necessary to behave nicely and genteelly, that all may
see that you are well-born children. I said at the time that the bodice
should be cut longer, and made of two widths. It was your fault, Sonia,
with your advice to make it shorter, and now you see the child is quite
deformed by it\ldots{}. Why, you're all crying again! What's the matter,
stupids? Come, Kolya, begin. Make haste, make haste! Oh, what an
unbearable child!

``Cinq sous, cinq sous.

``A policeman again! What do you want?''

A policeman was indeed forcing his way through the crowd. But at that
moment a gentleman in civilian uniform and an overcoat---a solid-looking
official of about fifty with a decoration on his neck (which delighted
Katerina Ivanovna and had its effect on the policeman)---approached and
without a word handed her a green three-rouble note. His face wore a
look of genuine sympathy. Katerina Ivanovna took it and gave him a
polite, even ceremonious, bow.

``I thank you, honoured sir,'' she began loftily. ``The causes that have
induced us (take the money, Polenka: you see there are generous and
honourable people who are ready to help a poor gentlewoman in distress).
You see, honoured sir, these orphans of good family---I might even say
of aristocratic connections---and that wretch of a general sat eating
grouse\ldots{} and stamped at my disturbing him. `Your excellency,' I
said, `protect the orphans, for you knew my late husband, Semyon
Zaharovitch, and on the very day of his death the basest of scoundrels
slandered his only daughter.'\ldots{} That policeman again! Protect
me,'' she cried to the official. ``Why is that policeman edging up to
me? We have only just run away from one of them. What do you want,
fool?''

``It's forbidden in the streets. You mustn't make a disturbance.''

``It's you're making a disturbance. It's just the same as if I were
grinding an organ. What business is it of yours?''

``You have to get a licence for an organ, and you haven't got one, and
in that way you collect a crowd. Where do you lodge?''

``What, a license?'' wailed Katerina Ivanovna. ``I buried my husband
to-day. What need of a license?''

``Calm yourself, madam, calm yourself,'' began the official. ``Come
along; I will escort you\ldots{}. This is no place for you in the crowd.
You are ill.''

``Honoured sir, honoured sir, you don't know,'' screamed Katerina
Ivanovna. ``We are going to the Nevsky\ldots{}. Sonia, Sonia! Where is
she? She is crying too! What's the matter with you all? Kolya, Lida,
where are you going?'' she cried suddenly in alarm. ``Oh, silly
children! Kolya, Lida, where are they off to?\ldots{}''

Kolya and Lida, scared out of their wits by the crowd, and their
mother's mad pranks, suddenly seized each other by the hand, and ran off
at the sight of the policeman who wanted to take them away somewhere.
Weeping and wailing, poor Katerina Ivanovna ran after them. She was a
piteous and unseemly spectacle, as she ran, weeping and panting for
breath. Sonia and Polenka rushed after them.

``Bring them back, bring them back, Sonia! Oh stupid, ungrateful
children!\ldots{} Polenka! catch them\ldots{}. It's for your sakes
I\ldots{}''

She stumbled as she ran and fell down.

``She's cut herself, she's bleeding! Oh, dear!'' cried Sonia, bending
over her.

All ran up and crowded around. Raskolnikov and Lebeziatnikov were the
first at her side, the official too hastened up, and behind him the
policeman who muttered, ``Bother!'' with a gesture of impatience,
feeling that the job was going to be a troublesome one.

``Pass on! Pass on!'' he said to the crowd that pressed forward.

``She's dying,'' someone shouted.

``She's gone out of her mind,'' said another.

``Lord have mercy upon us,'' said a woman, crossing herself. ``Have they
caught the little girl and the boy? They're being brought back, the
elder one's got them\ldots{}. Ah, the naughty imps!''

When they examined Katerina Ivanovna carefully, they saw that she had
not cut herself against a stone, as Sonia thought, but that the blood
that stained the pavement red was from her chest.

``I've seen that before,'' muttered the official to Raskolnikov and
Lebeziatnikov; ``that's consumption; the blood flows and chokes the
patient. I saw the same thing with a relative of my own not long
ago\ldots{} nearly a pint of blood, all in a minute\ldots{}. What's to
be done though? She is dying.''

``This way, this way, to my room!'' Sonia implored. ``I live
here!\ldots{} See, that house, the second from here\ldots{}. Come to me,
make haste,'' she turned from one to the other. ``Send for the doctor!
Oh, dear!''

Thanks to the official's efforts, this plan was adopted, the policeman
even helping to carry Katerina Ivanovna. She was carried to Sonia's
room, almost unconscious, and laid on the bed. The blood was still
flowing, but she seemed to be coming to herself. Raskolnikov,
Lebeziatnikov, and the official accompanied Sonia into the room and were
followed by the policeman, who first drove back the crowd which followed
to the very door. Polenka came in holding Kolya and Lida, who were
trembling and weeping. Several persons came in too from the
Kapernaumovs' room; the landlord, a lame one-eyed man of strange
appearance with whiskers and hair that stood up like a brush, his wife,
a woman with an everlastingly scared expression, and several
open-mouthed children with wonder-struck faces. Among these,
Svidrigaïlov suddenly made his appearance. Raskolnikov looked at him
with surprise, not understanding where he had come from and not having
noticed him in the crowd. A doctor and priest wore spoken of. The
official whispered to Raskolnikov that he thought it was too late now
for the doctor, but he ordered him to be sent for. Kapernaumov ran
himself.

Meanwhile Katerina Ivanovna had regained her breath. The bleeding ceased
for a time. She looked with sick but intent and penetrating eyes at
Sonia, who stood pale and trembling, wiping the sweat from her brow with
a handkerchief. At last she asked to be raised. They sat her up on the
bed, supporting her on both sides.

``Where are the children?'' she said in a faint voice. ``You've brought
them, Polenka? Oh the sillies! Why did you run away\ldots{}. Och!''

Once more her parched lips were covered with blood. She moved her eyes,
looking about her.

``So that's how you live, Sonia! Never once have I been in your room.''

She looked at her with a face of suffering.

``We have been your ruin, Sonia. Polenka, Lida, Kolya, come here! Well,
here they are, Sonia, take them all! I hand them over to you, I've had
enough! The ball is over.'' (Cough!) ``Lay me down, let me die in
peace.''

They laid her back on the pillow.

``What, the priest? I don't want him. You haven't got a rouble to spare.
I have no sins. God must forgive me without that. He knows how I have
suffered\ldots{}. And if He won't forgive me, I don't care!''

She sank more and more into uneasy delirium. At times she shuddered,
turned her eyes from side to side, recognised everyone for a minute, but
at once sank into delirium again. Her breathing was hoarse and
difficult, there was a sort of rattle in her throat.

``I said to him, your excellency,'' she ejaculated, gasping after each
word. ``That Amalia Ludwigovna, ah! Lida, Kolya, hands on your hips,
make haste! \emph{Glissez, glissez! pas de basque!} Tap with your heels,
be a graceful child!

``\emph{Du hast Diamanten und Perlen}

``What next? That's the thing to sing.

``\emph{Du hast die schonsten Augen Madchen, was willst du mehr?}

``What an idea! \emph{Was willst du mehr?} What things the fool invents!
Ah, yes!

``In the heat of midday in the vale of Dagestan.

``Ah, how I loved it! I loved that song to distraction, Polenka! Your
father, you know, used to sing it when we were engaged\ldots{}. Oh those
days! Oh that's the thing for us to sing! How does it go? I've
forgotten. Remind me! How was it?''

She was violently excited and tried to sit up. At last, in a horribly
hoarse, broken voice, she began, shrieking and gasping at every word,
with a look of growing terror.

``In the heat of midday!\ldots{} in the vale!\ldots{} of
Dagestan!\ldots{} With lead in my breast!\ldots{}''

``Your excellency!'' she wailed suddenly with a heart-rending scream and
a flood of tears, ``protect the orphans! You have been their father's
guest\ldots{} one may say aristocratic\ldots{}.'' She started, regaining
consciousness, and gazed at all with a sort of terror, but at once
recognised Sonia.

``Sonia, Sonia!'' she articulated softly and caressingly, as though
surprised to find her there. ``Sonia darling, are you here, too?''

They lifted her up again.

``Enough! It's over! Farewell, poor thing! I am done for! I am broken!''
she cried with vindictive despair, and her head fell heavily back on the
pillow.

She sank into unconsciousness again, but this time it did not last long.
Her pale, yellow, wasted face dropped back, her mouth fell open, her leg
moved convulsively, she gave a deep, deep sigh and died.

Sonia fell upon her, flung her arms about her, and remained motionless
with her head pressed to the dead woman's wasted bosom. Polenka threw
herself at her mother's feet, kissing them and weeping violently. Though
Kolya and Lida did not understand what had happened, they had a feeling
that it was something terrible; they put their hands on each other's
little shoulders, stared straight at one another and both at once opened
their mouths and began screaming. They were both still in their fancy
dress; one in a turban, the other in the cap with the ostrich feather.

And how did ``the certificate of merit'' come to be on the bed beside
Katerina Ivanovna? It lay there by the pillow; Raskolnikov saw it.

He walked away to the window. Lebeziatnikov skipped up to him.

``She is dead,'' he said.

``Rodion Romanovitch, I must have two words with you,'' said
Svidrigaïlov, coming up to them.

Lebeziatnikov at once made room for him and delicately withdrew.
Svidrigaïlov drew Raskolnikov further away.

``I will undertake all the arrangements, the funeral and that. You know
it's a question of money and, as I told you, I have plenty to spare. I
will put those two little ones and Polenka into some good orphan asylum,
and I will settle fifteen hundred roubles to be paid to each on coming
of age, so that Sofya Semyonovna need have no anxiety about them. And I
will pull her out of the mud too, for she is a good girl, isn't she? So
tell Avdotya Romanovna that that is how I am spending her ten
thousand.''

``What is your motive for such benevolence?'' asked Raskolnikov.

``Ah! you sceptical person!'' laughed Svidrigaïlov. ``I told you I had
no need of that money. Won't you admit that it's simply done from
humanity? She wasn't `a louse,' you know'' (he pointed to the corner
where the dead woman lay), ``was she, like some old pawnbroker woman?
Come, you'll agree, is Luzhin to go on living, and doing wicked things
or is she to die? And if I didn't help them, Polenka would go the same
way.''

He said this with an air of a sort of gay winking slyness, keeping his
eyes fixed on Raskolnikov, who turned white and cold, hearing his own
phrases, spoken to Sonia. He quickly stepped back and looked wildly at
Svidrigaïlov.

``How do you know?'' he whispered, hardly able to breathe.

``Why, I lodge here at Madame Resslich's, the other side of the wall.
Here is Kapernaumov, and there lives Madame Resslich, an old and devoted
friend of mine. I am a neighbour.''

``You?''

``Yes,'' continued Svidrigaïlov, shaking with laughter. ``I assure you
on my honour, dear Rodion Romanovitch, that you have interested me
enormously. I told you we should become friends, I foretold it. Well,
here we have. And you will see what an accommodating person I am. You'll
see that you can get on with me!''

\part*{PART VI}
\addcontentsline{toc}{part}{PART VI}

\mychapter{1}{Chapter I}\label{chapter-i-5}

A strange period began for Raskolnikov: it was as though a fog had
fallen upon him and wrapped him in a dreary solitude from which there
was no escape. Recalling that period long after, he believed that his
mind had been clouded at times, and that it had continued so, with
intervals, till the final catastrophe. He was convinced that he had been
mistaken about many things at that time, for instance as to the date of
certain events. Anyway, when he tried later on to piece his
recollections together, he learnt a great deal about himself from what
other people told him. He had mixed up incidents and had explained
events as due to circumstances which existed only in his imagination. At
times he was a prey to agonies of morbid uneasiness, amounting sometimes
to panic. But he remembered, too, moments, hours, perhaps whole days, of
complete apathy, which came upon him as a reaction from his previous
terror and might be compared with the abnormal insensibility, sometimes
seen in the dying. He seemed to be trying in that latter stage to escape
from a full and clear understanding of his position. Certain essential
facts which required immediate consideration were particularly irksome
to him. How glad he would have been to be free from some cares, the
neglect of which would have threatened him with complete, inevitable
ruin.

He was particularly worried about Svidrigaïlov, he might be said to be
permanently thinking of Svidrigaïlov. From the time of Svidrigaïlov's
too menacing and unmistakable words in Sonia's room at the moment of
Katerina Ivanovna's death, the normal working of his mind seemed to
break down. But although this new fact caused him extreme uneasiness,
Raskolnikov was in no hurry for an explanation of it. At times, finding
himself in a solitary and remote part of the town, in some wretched
eating-house, sitting alone lost in thought, hardly knowing how he had
come there, he suddenly thought of Svidrigaïlov. He recognised suddenly,
clearly, and with dismay that he ought at once to come to an
understanding with that man and to make what terms he could. Walking
outside the city gates one day, he positively fancied that they had
fixed a meeting there, that he was waiting for Svidrigaïlov. Another
time he woke up before daybreak lying on the ground under some bushes
and could not at first understand how he had come there.

But during the two or three days after Katerina Ivanovna's death, he had
two or three times met Svidrigaïlov at Sonia's lodging, where he had
gone aimlessly for a moment. They exchanged a few words and made no
reference to the vital subject, as though they were tacitly agreed not
to speak of it for a time.

Katerina Ivanovna's body was still lying in the coffin, Svidrigaïlov was
busy making arrangements for the funeral. Sonia too was very busy. At
their last meeting Svidrigaïlov informed Raskolnikov that he had made an
arrangement, and a very satisfactory one, for Katerina Ivanovna's
children; that he had, through certain connections, succeeded in getting
hold of certain personages by whose help the three orphans could be at
once placed in very suitable institutions; that the money he had settled
on them had been of great assistance, as it is much easier to place
orphans with some property than destitute ones. He said something too
about Sonia and promised to come himself in a day or two to see
Raskolnikov, mentioning that ``he would like to consult with him, that
there were things they must talk over\ldots{}.''

This conversation took place in the passage on the stairs. Svidrigaïlov
looked intently at Raskolnikov and suddenly, after a brief pause,
dropping his voice, asked: ``But how is it, Rodion Romanovitch; you
don't seem yourself? You look and you listen, but you don't seem to
understand. Cheer up! We'll talk things over; I am only sorry, I've so
much to do of my own business and other people's. Ah, Rodion
Romanovitch,'' he added suddenly, ``what all men need is fresh air,
fresh air\ldots{} more than anything!''

He moved to one side to make way for the priest and server, who were
coming up the stairs. They had come for the requiem service. By
Svidrigaïlov's orders it was sung twice a day punctually. Svidrigaïlov
went his way. Raskolnikov stood still a moment, thought, and followed
the priest into Sonia's room. He stood at the door. They began quietly,
slowly and mournfully singing the service. From his childhood the
thought of death and the presence of death had something oppressive and
mysteriously awful; and it was long since he had heard the requiem
service. And there was something else here as well, too awful and
disturbing. He looked at the children: they were all kneeling by the
coffin; Polenka was weeping. Behind them Sonia prayed, softly and, as it
were, timidly weeping.

``These last two days she hasn't said a word to me, she hasn't glanced
at me,'' Raskolnikov thought suddenly. The sunlight was bright in the
room; the incense rose in clouds; the priest read, ``Give rest, oh
Lord\ldots{}.'' Raskolnikov stayed all through the service. As he
blessed them and took his leave, the priest looked round strangely.
After the service, Raskolnikov went up to Sonia. She took both his hands
and let her head sink on his shoulder. This slight friendly gesture
bewildered Raskolnikov. It seemed strange to him that there was no trace
of repugnance, no trace of disgust, no tremor in her hand. It was the
furthest limit of self-abnegation, at least so he interpreted it.

Sonia said nothing. Raskolnikov pressed her hand and went out. He felt
very miserable. If it had been possible to escape to some solitude, he
would have thought himself lucky, even if he had to spend his whole life
there. But although he had almost always been by himself of late, he had
never been able to feel alone. Sometimes he walked out of the town on to
the high road, once he had even reached a little wood, but the lonelier
the place was, the more he seemed to be aware of an uneasy presence near
him. It did not frighten him, but greatly annoyed him, so that he made
haste to return to the town, to mingle with the crowd, to enter
restaurants and taverns, to walk in busy thoroughfares. There he felt
easier and even more solitary. One day at dusk he sat for an hour
listening to songs in a tavern and he remembered that he positively
enjoyed it. But at last he had suddenly felt the same uneasiness again,
as though his conscience smote him. ``Here I sit listening to singing,
is that what I ought to be doing?'' he thought. Yet he felt at once that
that was not the only cause of his uneasiness; there was something
requiring immediate decision, but it was something he could not clearly
understand or put into words. It was a hopeless tangle. ``No, better the
struggle again! Better Porfiry again\ldots{} or Svidrigaïlov\ldots{}.
Better some challenge again\ldots{} some attack. Yes, yes!'' he thought.
He went out of the tavern and rushed away almost at a run. The thought
of Dounia and his mother suddenly reduced him almost to a panic. That
night he woke up before morning among some bushes in Krestovsky Island,
trembling all over with fever; he walked home, and it was early morning
when he arrived. After some hours' sleep the fever left him, but he woke
up late, two o'clock in the afternoon.

He remembered that Katerina Ivanovna's funeral had been fixed for that
day, and was glad that he was not present at it. Nastasya brought him
some food; he ate and drank with appetite, almost with greediness. His
head was fresher and he was calmer than he had been for the last three
days. He even felt a passing wonder at his previous attacks of panic.

The door opened and Razumihin came in.

``Ah, he's eating, then he's not ill,'' said Razumihin. He took a chair
and sat down at the table opposite Raskolnikov.

He was troubled and did not attempt to conceal it. He spoke with evident
annoyance, but without hurry or raising his voice. He looked as though
he had some special fixed determination.

``Listen,'' he began resolutely. ``As far as I am concerned, you may all
go to hell, but from what I see, it's clear to me that I can't make head
or tail of it; please don't think I've come to ask you questions. I
don't want to know, hang it! If you begin telling me your secrets, I
dare say I shouldn't stay to listen, I should go away cursing. I have
only come to find out once for all whether it's a fact that you are mad?
There is a conviction in the air that you are mad or very nearly so. I
admit I've been disposed to that opinion myself, judging from your
stupid, repulsive and quite inexplicable actions, and from your recent
behavior to your mother and sister. Only a monster or a madman could
treat them as you have; so you must be mad.''

``When did you see them last?''

``Just now. Haven't you seen them since then? What have you been doing
with yourself? Tell me, please. I've been to you three times already.
Your mother has been seriously ill since yesterday. She had made up her
mind to come to you; Avdotya Romanovna tried to prevent her; she
wouldn't hear a word. `If he is ill, if his mind is giving way, who can
look after him like his mother?' she said. We all came here together, we
couldn't let her come alone all the way. We kept begging her to be calm.
We came in, you weren't here; she sat down, and stayed ten minutes,
while we stood waiting in silence. She got up and said: `If he's gone
out, that is, if he is well, and has forgotten his mother, it's
humiliating and unseemly for his mother to stand at his door begging for
kindness.' She returned home and took to her bed; now she is in a fever.
`I see,' she said, `that he has time for \emph{his girl}.' She means by
\emph{your girl} Sofya Semyonovna, your betrothed or your mistress, I
don't know. I went at once to Sofya Semyonovna's, for I wanted to know
what was going on. I looked round, I saw the coffin, the children
crying, and Sofya Semyonovna trying them on mourning dresses. No sign of
you. I apologised, came away, and reported to Avdotya Romanovna. So
that's all nonsense and you haven't got a girl; the most likely thing is
that you are mad. But here you sit, guzzling boiled beef as though you'd
not had a bite for three days. Though as far as that goes, madmen eat
too, but though you have not said a word to me yet\ldots{} you are not
mad! That I'd swear! Above all, you are not mad! So you may go to hell,
all of you, for there's some mystery, some secret about it, and I don't
intend to worry my brains over your secrets. So I've simply come to
swear at you,'' he finished, getting up, ``to relieve my mind. And I
know what to do now.''

``What do you mean to do now?''

``What business is it of yours what I mean to do?''

``You are going in for a drinking bout.''

``How\ldots{} how did you know?''

``Why, it's pretty plain.''

Razumihin paused for a minute.

``You always have been a very rational person and you've never been mad,
never,'' he observed suddenly with warmth. ``You're right: I shall
drink. Good-bye!''

And he moved to go out.

``I was talking with my sister---the day before yesterday, I think it
was---about you, Razumihin.''

``About me! But\ldots{} where can you have seen her the day before
yesterday?'' Razumihin stopped short and even turned a little pale.

One could see that his heart was throbbing slowly and violently.

``She came here by herself, sat there and talked to me.''

``She did!''

``Yes.''

``What did you say to her\ldots{} I mean, about me?''

``I told her you were a very good, honest, and industrious man. I didn't
tell her you love her, because she knows that herself.''

``She knows that herself?''

``Well, it's pretty plain. Wherever I might go, whatever happened to me,
you would remain to look after them. I, so to speak, give them into your
keeping, Razumihin. I say this because I know quite well how you love
her, and am convinced of the purity of your heart. I know that she too
may love you and perhaps does love you already. Now decide for yourself,
as you know best, whether you need go in for a drinking bout or not.''

``Rodya! You see\ldots{} well\ldots{}. Ach, damn it! But where do you
mean to go? Of course, if it's all a secret, never mind\ldots{}. But
I\ldots{} I shall find out the secret\ldots{} and I am sure that it must
be some ridiculous nonsense and that you've made it all up. Anyway you
are a capital fellow, a capital fellow!\ldots{}''

``That was just what I wanted to add, only you interrupted, that that
was a very good decision of yours not to find out these secrets. Leave
it to time, don't worry about it. You'll know it all in time when it
must be. Yesterday a man said to me that what a man needs is fresh air,
fresh air, fresh air. I mean to go to him directly to find out what he
meant by that.''

Razumihin stood lost in thought and excitement, making a silent
conclusion.

``He's a political conspirator! He must be. And he's on the eve of some
desperate step, that's certain. It can only be that! And\ldots{} and
Dounia knows,'' he thought suddenly.

``So Avdotya Romanovna comes to see you,'' he said, weighing each
syllable, ``and you're going to see a man who says we need more air, and
so of course that letter\ldots{} that too must have something to do with
it,'' he concluded to himself.

``What letter?''

``She got a letter to-day. It upset her very much---very much indeed.
Too much so. I began speaking of you, she begged me not to. Then\ldots{}
then she said that perhaps we should very soon have to part\ldots{} then
she began warmly thanking me for something; then she went to her room
and locked herself in.''

``She got a letter?'' Raskolnikov asked thoughtfully.

``Yes, and you didn't know? hm\ldots{}''

They were both silent.

``Good-bye, Rodion. There was a time, brother, when I\ldots{}. Never
mind, good-bye. You see, there was a time\ldots{}. Well, good-bye! I
must be off too. I am not going to drink. There's no need now\ldots{}.
That's all stuff!''

He hurried out; but when he had almost closed the door behind him, he
suddenly opened it again, and said, looking away:

``Oh, by the way, do you remember that murder, you know Porfiry's, that
old woman? Do you know the murderer has been found, he has confessed and
given the proofs. It's one of those very workmen, the painter, only
fancy! Do you remember I defended them here? Would you believe it, all
that scene of fighting and laughing with his companions on the stairs
while the porter and the two witnesses were going up, he got up on
purpose to disarm suspicion. The cunning, the presence of mind of the
young dog! One can hardly credit it; but it's his own explanation, he
has confessed it all. And what a fool I was about it! Well, he's simply
a genius of hypocrisy and resourcefulness in disarming the suspicions of
the lawyers---so there's nothing much to wonder at, I suppose! Of course
people like that are always possible. And the fact that he couldn't keep
up the character, but confessed, makes him easier to believe in. But
what a fool I was! I was frantic on their side!''

``Tell me, please, from whom did you hear that, and why does it interest
you so?'' Raskolnikov asked with unmistakable agitation.

``What next? You ask me why it interests me!\ldots{} Well, I heard it
from Porfiry, among others\ldots{} It was from him I heard almost all
about it.''

``From Porfiry?''

``From Porfiry.''

``What\ldots{} what did he say?'' Raskolnikov asked in dismay.

``He gave me a capital explanation of it. Psychologically, after his
fashion.''

``He explained it? Explained it himself?''

``Yes, yes; good-bye. I'll tell you all about it another time, but now
I'm busy. There was a time when I fancied\ldots{} But no matter, another
time!\ldots{} What need is there for me to drink now? You have made me
drunk without wine. I am drunk, Rodya! Good-bye, I'm going. I'll come
again very soon.''

He went out.

``He's a political conspirator, there's not a doubt about it,''
Razumihin decided, as he slowly descended the stairs. ``And he's drawn
his sister in; that's quite, quite in keeping with Avdotya Romanovna's
character. There are interviews between them!\ldots{} She hinted at it
too\ldots{} So many of her words\ldots{}. and hints\ldots{} bear that
meaning! And how else can all this tangle be explained? Hm! And I was
almost thinking\ldots{} Good heavens, what I thought! Yes, I took leave
of my senses and I wronged him! It was his doing, under the lamp in the
corridor that day. Pfoo! What a crude, nasty, vile idea on my part!
Nikolay is a brick, for confessing\ldots{}. And how clear it all is now!
His illness then, all his strange actions\ldots{} before this, in the
university, how morose he used to be, how gloomy\ldots{}. But what's the
meaning now of that letter? There's something in that, too, perhaps.
Whom was it from? I suspect\ldots{}! No, I must find out!''

He thought of Dounia, realising all he had heard and his heart throbbed,
and he suddenly broke into a run.

As soon as Razumihin went out, Raskolnikov got up, turned to the window,
walked into one corner and then into another, as though forgetting the
smallness of his room, and sat down again on the sofa. He felt, so to
speak, renewed; again the struggle, so a means of escape had come.

``Yes, a means of escape had come! It had been too stifling, too
cramping, the burden had been too agonising. A lethargy had come upon
him at times. From the moment of the scene with Nikolay at Porfiry's he
had been suffocating, penned in without hope of escape. After Nikolay's
confession, on that very day had come the scene with Sonia; his
behaviour and his last words had been utterly unlike anything he could
have imagined beforehand; he had grown feebler, instantly and
fundamentally! And he had agreed at the time with Sonia, he had agreed
in his heart he could not go on living alone with such a thing on his
mind!

``And Svidrigaïlov was a riddle\ldots{} He worried him, that was true,
but somehow not on the same point. He might still have a struggle to
come with Svidrigaïlov. Svidrigaïlov, too, might be a means of escape;
but Porfiry was a different matter.

``And so Porfiry himself had explained it to Razumihin, had explained it
\emph{psychologically}. He had begun bringing in his damned psychology
again! Porfiry? But to think that Porfiry should for one moment believe
that Nikolay was guilty, after what had passed between them before
Nikolay's appearance, after that tête-à-tête interview, which could have
only \emph{one} explanation? (During those days Raskolnikov had often
recalled passages in that scene with Porfiry; he could not bear to let
his mind rest on it.) Such words, such gestures had passed between them,
they had exchanged such glances, things had been said in such a tone and
had reached such a pass, that Nikolay, whom Porfiry had seen through at
the first word, at the first gesture, could not have shaken his
conviction.

``And to think that even Razumihin had begun to suspect! The scene in
the corridor under the lamp had produced its effect then. He had rushed
to Porfiry\ldots{}. But what had induced the latter to receive him like
that? What had been his object in putting Razumihin off with Nikolay? He
must have some plan; there was some design, but what was it? It was true
that a long time had passed since that morning---too long a time---and
no sight nor sound of Porfiry. Well, that was a bad sign\ldots{}.''

Raskolnikov took his cap and went out of the room, still pondering. It
was the first time for a long while that he had felt clear in his mind,
at least. ``I must settle Svidrigaïlov,'' he thought, ``and as soon as
possible; he, too, seems to be waiting for me to come to him of my own
accord.'' And at that moment there was such a rush of hate in his weary
heart that he might have killed either of those two---Porfiry or
Svidrigaïlov. At least he felt that he would be capable of doing it
later, if not now.

``We shall see, we shall see,'' he repeated to himself.

But no sooner had he opened the door than he stumbled upon Porfiry
himself in the passage. He was coming in to see him. Raskolnikov was
dumbfounded for a minute, but only for one minute. Strange to say, he
was not very much astonished at seeing Porfiry and scarcely afraid of
him. He was simply startled, but was quickly, instantly, on his guard.
``Perhaps this will mean the end? But how could Porfiry have approached
so quietly, like a cat, so that he had heard nothing? Could he have been
listening at the door?''

``You didn't expect a visitor, Rodion Romanovitch,'' Porfiry explained,
laughing. ``I've been meaning to look in a long time; I was passing by
and thought why not go in for five minutes. Are you going out? I won't
keep you long. Just let me have one cigarette.''

``Sit down, Porfiry Petrovitch, sit down.'' Raskolnikov gave his visitor
a seat with so pleased and friendly an expression that he would have
marvelled at himself, if he could have seen it.

The last moment had come, the last drops had to be drained! So a man
will sometimes go through half an hour of mortal terror with a brigand,
yet when the knife is at his throat at last, he feels no fear.

Raskolnikov seated himself directly facing Porfiry, and looked at him
without flinching. Porfiry screwed up his eyes and began lighting a
cigarette.

``Speak, speak,'' seemed as though it would burst from Raskolnikov's
heart. ``Come, why don't you speak?''

\protect\hypertarget{link2HCH0033}{}{}

\mychapter{2}{Chapter II}\label{chapter-ii-5}

``Ah these cigarettes!'' Porfiry Petrovitch ejaculated at last, having
lighted one. ``They are pernicious, positively pernicious, and yet I
can't give them up! I cough, I begin to have tickling in my throat and a
difficulty in breathing. You know I am a coward, I went lately to Dr.
B------n; he always gives at least half an hour to each patient. He
positively laughed looking at me; he sounded me: `Tobacco's bad for
you,' he said, `your lungs are affected.' But how am I to give it up?
What is there to take its place? I don't drink, that's the mischief,
he-he-he, that I don't. Everything is relative, Rodion Romanovitch,
everything is relative!''

``Why, he's playing his professional tricks again,'' Raskolnikov thought
with disgust. All the circumstances of their last interview suddenly
came back to him, and he felt a rush of the feeling that had come upon
him then.

``I came to see you the day before yesterday, in the evening; you didn't
know?'' Porfiry Petrovitch went on, looking round the room. ``I came
into this very room. I was passing by, just as I did to-day, and I
thought I'd return your call. I walked in as your door was wide open, I
looked round, waited and went out without leaving my name with your
servant. Don't you lock your door?''

Raskolnikov's face grew more and more gloomy. Porfiry seemed to guess
his state of mind.

``I've come to have it out with you, Rodion Romanovitch, my dear fellow!
I owe you an explanation and must give it to you,'' he continued with a
slight smile, just patting Raskolnikov's knee.

But almost at the same instant a serious and careworn look came into his
face; to his surprise Raskolnikov saw a touch of sadness in it. He had
never seen and never suspected such an expression in his face.

``A strange scene passed between us last time we met, Rodion
Romanovitch. Our first interview, too, was a strange one; but
then\ldots{} and one thing after another! This is the point: I have
perhaps acted unfairly to you; I feel it. Do you remember how we parted?
Your nerves were unhinged and your knees were shaking and so were mine.
And, you know, our behaviour was unseemly, even ungentlemanly. And yet
we are gentlemen, above all, in any case, gentlemen; that must be
understood. Do you remember what we came to?\ldots{} and it was quite
indecorous.''

``What is he up to, what does he take me for?'' Raskolnikov asked
himself in amazement, raising his head and looking with open eyes on
Porfiry.

``I've decided openness is better between us,'' Porfiry Petrovitch went
on, turning his head away and dropping his eyes, as though unwilling to
disconcert his former victim and as though disdaining his former wiles.
``Yes, such suspicions and such scenes cannot continue for long. Nikolay
put a stop to it, or I don't know what we might not have come to. That
damned workman was sitting at the time in the next room---can you
realise that? You know that, of course; and I am aware that he came to
you afterwards. But what you supposed then was not true: I had not sent
for anyone, I had made no kind of arrangements. You ask why I hadn't?
What shall I say to you? it had all come upon me so suddenly. I had
scarcely sent for the porters (you noticed them as you went out, I dare
say). An idea flashed upon me; I was firmly convinced at the time, you
see, Rodion Romanovitch. Come, I thought---even if I let one thing slip
for a time, I shall get hold of something else---I shan't lose what I
want, anyway. You are nervously irritable, Rodion Romanovitch, by
temperament; it's out of proportion with other qualities of your heart
and character, which I flatter myself I have to some extent divined. Of
course I did reflect even then that it does not always happen that a man
gets up and blurts out his whole story. It does happen sometimes, if you
make a man lose all patience, though even then it's rare. I was capable
of realising that. If I only had a fact, I thought, the least little
fact to go upon, something I could lay hold of, something tangible, not
merely psychological. For if a man is guilty, you must be able to get
something substantial out of him; one may reckon upon most surprising
results indeed. I was reckoning on your temperament, Rodion Romanovitch,
on your temperament above all things! I had great hopes of you at that
time.''

``But what are you driving at now?'' Raskolnikov muttered at last,
asking the question without thinking.

``What is he talking about?'' he wondered distractedly, ``does he really
take me to be innocent?''

``What am I driving at? I've come to explain myself, I consider it my
duty, so to speak. I want to make clear to you how the whole business,
the whole misunderstanding arose. I've caused you a great deal of
suffering, Rodion Romanovitch. I am not a monster. I understand what it
must mean for a man who has been unfortunate, but who is proud,
imperious and above all, impatient, to have to bear such treatment! I
regard you in any case as a man of noble character and not without
elements of magnanimity, though I don't agree with all your convictions.
I wanted to tell you this first, frankly and quite sincerely, for above
all I don't want to deceive you. When I made your acquaintance, I felt
attracted by you. Perhaps you will laugh at my saying so. You have a
right to. I know you disliked me from the first and indeed you've no
reason to like me. You may think what you like, but I desire now to do
all I can to efface that impression and to show that I am a man of heart
and conscience. I speak sincerely.''

Porfiry Petrovitch made a dignified pause. Raskolnikov felt a rush of
renewed alarm. The thought that Porfiry believed him to be innocent
began to make him uneasy.

``It's scarcely necessary to go over everything in detail,'' Porfiry
Petrovitch went on. ``Indeed, I could scarcely attempt it. To begin with
there were rumours. Through whom, how, and when those rumours came to
me\ldots{} and how they affected you, I need not go into. My suspicions
were aroused by a complete accident, which might just as easily not have
happened. What was it? Hm! I believe there is no need to go into that
either. Those rumours and that accident led to one idea in my mind. I
admit it openly---for one may as well make a clean breast of it---I was
the first to pitch on you. The old woman's notes on the pledges and the
rest of it---that all came to nothing. Yours was one of a hundred. I
happened, too, to hear of the scene at the office, from a man who
described it capitally, unconsciously reproducing the scene with great
vividness. It was just one thing after another, Rodion Romanovitch, my
dear fellow! How could I avoid being brought to certain ideas? From a
hundred rabbits you can't make a horse, a hundred suspicions don't make
a proof, as the English proverb says, but that's only from the rational
point of view---you can't help being partial, for after all a lawyer is
only human. I thought, too, of your article in that journal, do you
remember, on your first visit we talked of it? I jeered at you at the
time, but that was only to lead you on. I repeat, Rodion Romanovitch,
you are ill and impatient. That you were bold, headstrong, in earnest
and\ldots{} had felt a great deal I recognised long before. I, too, have
felt the same, so that your article seemed familiar to me. It was
conceived on sleepless nights, with a throbbing heart, in ecstasy and
suppressed enthusiasm. And that proud suppressed enthusiasm in young
people is dangerous! I jeered at you then, but let me tell you that, as
a literary amateur, I am awfully fond of such first essays, full of the
heat of youth. There is a mistiness and a chord vibrating in the mist.
Your article is absurd and fantastic, but there's a transparent
sincerity, a youthful incorruptible pride and the daring of despair in
it. It's a gloomy article, but that's what's fine in it. I read your
article and put it aside, thinking as I did so `that man won't go the
common way.' Well, I ask you, after that as a preliminary, how could I
help being carried away by what followed? Oh, dear, I am not saying
anything, I am not making any statement now. I simply noted it at the
time. What is there in it? I reflected. There's nothing in it, that is
really nothing and perhaps absolutely nothing. And it's not at all the
thing for the prosecutor to let himself be carried away by notions: here
I have Nikolay on my hands with actual evidence against him---you may
think what you like of it, but it's evidence. He brings in his
psychology, too; one has to consider him, too, for it's a matter of life
and death. Why am I explaining this to you? That you may understand, and
not blame my malicious behaviour on that occasion. It was not malicious,
I assure you, he-he! Do you suppose I didn't come to search your room at
the time? I did, I did, he-he! I was here when you were lying ill in
bed, not officially, not in my own person, but I was here. Your room was
searched to the last thread at the first suspicion; but \emph{umsonst}!
I thought to myself, now that man will come, will come of himself and
quickly, too; if he's guilty, he's sure to come. Another man wouldn't,
but he will. And you remember how Mr. Razumihin began discussing the
subject with you? We arranged that to excite you, so we purposely spread
rumours, that he might discuss the case with you, and Razumihin is not a
man to restrain his indignation. Mr. Zametov was tremendously struck by
your anger and your open daring. Think of blurting out in a restaurant
`I killed her.' It was too daring, too reckless. I thought so myself, if
he is guilty he will be a formidable opponent. That was what I thought
at the time. I was expecting you. But you simply bowled Zametov over
and\ldots{} well, you see, it all lies in this---that this damnable
psychology can be taken two ways! Well, I kept expecting you, and so it
was, you came! My heart was fairly throbbing. Ach!

``Now, why need you have come? Your laughter, too, as you came in, do
you remember? I saw it all plain as daylight, but if I hadn't expected
you so specially, I should not have noticed anything in your laughter.
You see what influence a mood has! Mr. Razumihin then---ah, that stone,
that stone under which the things were hidden! I seem to see it
somewhere in a kitchen garden. It was in a kitchen garden, you told
Zametov and afterwards you repeated that in my office? And when we began
picking your article to pieces, how you explained it! One could take
every word of yours in two senses, as though there were another meaning
hidden.

``So in this way, Rodion Romanovitch, I reached the furthest limit, and
knocking my head against a post, I pulled myself up, asking myself what
I was about. After all, I said, you can take it all in another sense if
you like, and it's more natural so, indeed. I couldn't help admitting it
was more natural. I was bothered! `No, I'd better get hold of some
little fact' I said. So when I heard of the bell-ringing, I held my
breath and was all in a tremor. `Here is my little fact,' thought I, and
I didn't think it over, I simply wouldn't. I would have given a thousand
roubles at that minute to have seen you with my own eyes, when you
walked a hundred paces beside that workman, after he had called you
murderer to your face, and you did not dare to ask him a question all
the way. And then what about your trembling, what about your
bell-ringing in your illness, in semi-delirium?

``And so, Rodion Romanovitch, can you wonder that I played such pranks
on you? And what made you come at that very minute? Someone seemed to
have sent you, by Jove! And if Nikolay had not parted us\ldots{} and do
you remember Nikolay at the time? Do you remember him clearly? It was a
thunderbolt, a regular thunderbolt! And how I met him! I didn't believe
in the thunderbolt, not for a minute. You could see it for yourself; and
how could I? Even afterwards, when you had gone and he began making
very, very plausible answers on certain points, so that I was surprised
at him myself, even then I didn't believe his story! You see what it is
to be as firm as a rock! No, thought I, \emph{Morgenfrüh}. What has
Nikolay got to do with it!''

``Razumihin told me just now that you think Nikolay guilty and had
yourself assured him of it\ldots{}.''

His voice failed him, and he broke off. He had been listening in
indescribable agitation, as this man who had seen through and through
him, went back upon himself. He was afraid of believing it and did not
believe it. In those still ambiguous words he kept eagerly looking for
something more definite and conclusive.

``Mr. Razumihin!'' cried Porfiry Petrovitch, seeming glad of a question
from Raskolnikov, who had till then been silent. ``He-he-he! But I had
to put Mr. Razumihin off; two is company, three is none. Mr. Razumihin
is not the right man, besides he is an outsider. He came running to me
with a pale face\ldots{}. But never mind him, why bring him in? To
return to Nikolay, would you like to know what sort of a type he is, how
I understand him, that is? To begin with, he is still a child and not
exactly a coward, but something by way of an artist. Really, don't laugh
at my describing him so. He is innocent and responsive to influence. He
has a heart, and is a fantastic fellow. He sings and dances, he tells
stories, they say, so that people come from other villages to hear him.
He attends school too, and laughs till he cries if you hold up a finger
to him; he will drink himself senseless---not as a regular vice, but at
times, when people treat him, like a child. And he stole, too, then,
without knowing it himself, for `How can it be stealing, if one picks it
up?' And do you know he is an Old Believer, or rather a dissenter? There
have been Wanderers\footnote{\emph{Translator's Note}—A religious sect.} in his family, and he was for two years in
his village under the spiritual guidance of a certain elder. I learnt
all this from Nikolay and from his fellow villagers. And what's more, he
wanted to run into the wilderness! He was full of fervour, prayed at
night, read the old books, `the true' ones, and read himself crazy.

``Petersburg had a great effect upon him, especially the women and the
wine. He responds to everything and he forgot the elder and all that. I
learnt that an artist here took a fancy to him, and used to go and see
him, and now this business came upon him.

``Well, he was frightened, he tried to hang himself! He ran away! How
can one get over the idea the people have of Russian legal proceedings?
The very word `trial' frightens some of them. Whose fault is it? We
shall see what the new juries will do. God grant they do good! Well, in
prison, it seems, he remembered the venerable elder; the Bible, too,
made its appearance again. Do you know, Rodion Romanovitch, the force of
the word `suffering' among some of these people! It's not a question of
suffering for someone's benefit, but simply, `one must suffer.' If they
suffer at the hands of the authorities, so much the better. In my time
there was a very meek and mild prisoner who spent a whole year in prison
always reading his Bible on the stove at night and he read himself
crazy, and so crazy, do you know, that one day, apropos of nothing, he
seized a brick and flung it at the governor; though he had done him no
harm. And the way he threw it too: aimed it a yard on one side on
purpose, for fear of hurting him. Well, we know what happens to a
prisoner who assaults an officer with a weapon. So `he took his
suffering.'

``So I suspect now that Nikolay wants to take his suffering or something
of the sort. I know it for certain from facts, indeed. Only he doesn't
know that I know. What, you don't admit that there are such fantastic
people among the peasants? Lots of them. The elder now has begun
influencing him, especially since he tried to hang himself. But he'll
come and tell me all himself. You think he'll hold out? Wait a bit,
he'll take his words back. I am waiting from hour to hour for him to
come and abjure his evidence. I have come to like that Nikolay and am
studying him in detail. And what do you think? He-he! He answered me
very plausibly on some points, he obviously had collected some evidence
and prepared himself cleverly. But on other points he is simply at sea,
knows nothing and doesn't even suspect that he doesn't know!

``No, Rodion Romanovitch, Nikolay doesn't come in! This is a fantastic,
gloomy business, a modern case, an incident of to-day when the heart of
man is troubled, when the phrase is quoted that blood `renews,' when
comfort is preached as the aim of life. Here we have bookish dreams, a
heart unhinged by theories. Here we see resolution in the first stage,
but resolution of a special kind: he resolved to do it like jumping over
a precipice or from a bell tower and his legs shook as he went to the
crime. He forgot to shut the door after him, and murdered two people for
a theory. He committed the murder and couldn't take the money, and what
he did manage to snatch up he hid under a stone. It wasn't enough for
him to suffer agony behind the door while they battered at the door and
rung the bell, no, he had to go to the empty lodging, half delirious, to
recall the bell-ringing, he wanted to feel the cold shiver over
again\ldots{}. Well, that we grant, was through illness, but consider
this: he is a murderer, but looks upon himself as an honest man,
despises others, poses as injured innocence. No, that's not the work of
a Nikolay, my dear Rodion Romanovitch!''

All that had been said before had sounded so like a recantation that
these words were too great a shock. Raskolnikov shuddered as though he
had been stabbed.

``Then\ldots{} who then\ldots{} is the murderer?'' he asked in a
breathless voice, unable to restrain himself.

Porfiry Petrovitch sank back in his chair, as though he were amazed at
the question.

``Who is the murderer?'' he repeated, as though unable to believe his
ears. ``Why, \emph{you}, Rodion Romanovitch! You are the murderer,'' he
added, almost in a whisper, in a voice of genuine conviction.

Raskolnikov leapt from the sofa, stood up for a few seconds and sat down
again without uttering a word. His face twitched convulsively.

``Your lip is twitching just as it did before,'' Porfiry Petrovitch
observed almost sympathetically. ``You've been misunderstanding me, I
think, Rodion Romanovitch,'' he added after a brief pause, ``that's why
you are so surprised. I came on purpose to tell you everything and deal
openly with you.''

``It was not I murdered her,'' Raskolnikov whispered like a frightened
child caught in the act.

``No, it was you, you Rodion Romanovitch, and no one else,'' Porfiry
whispered sternly, with conviction.

They were both silent and the silence lasted strangely long, about ten
minutes. Raskolnikov put his elbow on the table and passed his fingers
through his hair. Porfiry Petrovitch sat quietly waiting. Suddenly
Raskolnikov looked scornfully at Porfiry.

``You are at your old tricks again, Porfiry Petrovitch! Your old method
again. I wonder you don't get sick of it!''

``Oh, stop that, what does that matter now? It would be a different
matter if there were witnesses present, but we are whispering alone. You
see yourself that I have not come to chase and capture you like a hare.
Whether you confess it or not is nothing to me now; for myself, I am
convinced without it.''

``If so, what did you come for?'' Raskolnikov asked irritably. ``I ask
you the same question again: if you consider me guilty, why don't you
take me to prison?''

``Oh, that's your question! I will answer you, point for point. In the
first place, to arrest you so directly is not to my interest.''

``How so? If you are convinced you ought\ldots{}.''

``Ach, what if I am convinced? That's only my dream for the time. Why
should I put you in safety? You know that's it, since you ask me to do
it. If I confront you with that workman for instance and you say to him
`were you drunk or not? Who saw me with you? I simply took you to be
drunk, and you were drunk, too.' Well, what could I answer, especially
as your story is a more likely one than his? for there's nothing but
psychology to support his evidence---that's almost unseemly with his
ugly mug, while you hit the mark exactly, for the rascal is an
inveterate drunkard and notoriously so. And I have myself admitted
candidly several times already that that psychology can be taken in two
ways and that the second way is stronger and looks far more probable,
and that apart from that I have as yet nothing against you. And though I
shall put you in prison and indeed have come---quite contrary to
etiquette---to inform you of it beforehand, yet I tell you frankly, also
contrary to etiquette, that it won't be to my advantage. Well, secondly,
I've come to you because\ldots{}''

``Yes, yes, secondly?'' Raskolnikov was listening breathless.

``Because, as I told you just now, I consider I owe you an explanation.
I don't want you to look upon me as a monster, as I have a genuine
liking for you, you may believe me or not. And in the third place I've
come to you with a direct and open proposition---that you should
surrender and confess. It will be infinitely more to your advantage and
to my advantage too, for my task will be done. Well, is this open on my
part or not?''

Raskolnikov thought a minute.

``Listen, Porfiry Petrovitch. You said just now you have nothing but
psychology to go on, yet now you've gone on mathematics. Well, what if
you are mistaken yourself, now?''

``No, Rodion Romanovitch, I am not mistaken. I have a little fact even
then, Providence sent it me.''

``What little fact?''

``I won't tell you what, Rodion Romanovitch. And in any case, I haven't
the right to put it off any longer, I must arrest you. So think it over:
it makes no difference to me \emph{now} and so I speak only for your
sake. Believe me, it will be better, Rodion Romanovitch.''

Raskolnikov smiled malignantly.

``That's not simply ridiculous, it's positively shameless. Why, even if
I were guilty, which I don't admit, what reason should I have to
confess, when you tell me yourself that I shall be in greater safety in
prison?''

``Ah, Rodion Romanovitch, don't put too much faith in words, perhaps
prison will not be altogether a restful place. That's only theory and my
theory, and what authority am I for you? Perhaps, too, even now I am
hiding something from you? I can't lay bare everything, he-he! And how
can you ask what advantage? Don't you know how it would lessen your
sentence? You would be confessing at a moment when another man has taken
the crime on himself and so has muddled the whole case. Consider that! I
swear before God that I will so arrange that your confession shall come
as a complete surprise. We will make a clean sweep of all these
psychological points, of a suspicion against you, so that your crime
will appear to have been something like an aberration, for in truth it
was an aberration. I am an honest man, Rodion Romanovitch, and will keep
my word.''

Raskolnikov maintained a mournful silence and let his head sink
dejectedly. He pondered a long while and at last smiled again, but his
smile was sad and gentle.

``No!'' he said, apparently abandoning all attempt to keep up
appearances with Porfiry, ``it's not worth it, I don't care about
lessening the sentence!''

``That's just what I was afraid of!'' Porfiry cried warmly and, as it
seemed, involuntarily. ``That's just what I feared, that you wouldn't
care about the mitigation of sentence.''

Raskolnikov looked sadly and expressively at him.

``Ah, don't disdain life!'' Porfiry went on. ``You have a great deal of
it still before you. How can you say you don't want a mitigation of
sentence? You are an impatient fellow!''

``A great deal of what lies before me?''

``Of life. What sort of prophet are you, do you know much about it? Seek
and ye shall find. This may be God's means for bringing you to Him. And
it's not for ever, the bondage\ldots{}.''

``The time will be shortened,'' laughed Raskolnikov.

``Why, is it the bourgeois disgrace you are afraid of? It may be that
you are afraid of it without knowing it, because you are young! But
anyway \emph{you} shouldn't be afraid of giving yourself up and
confessing.''

``Ach, hang it!'' Raskolnikov whispered with loathing and contempt, as
though he did not want to speak aloud.

He got up again as though he meant to go away, but sat down again in
evident despair.

``Hang it, if you like! You've lost faith and you think that I am
grossly flattering you; but how long has your life been? How much do you
understand? You made up a theory and then were ashamed that it broke
down and turned out to be not at all original! It turned out something
base, that's true, but you are not hopelessly base. By no means so base!
At least you didn't deceive yourself for long, you went straight to the
furthest point at one bound. How do I regard you? I regard you as one of
those men who would stand and smile at their torturer while he cuts
their entrails out, if only they have found faith or God. Find it and
you will live. You have long needed a change of air. Suffering, too, is
a good thing. Suffer! Maybe Nikolay is right in wanting to suffer. I
know you don't believe in it---but don't be over-wise; fling yourself
straight into life, without deliberation; don't be afraid---the flood
will bear you to the bank and set you safe on your feet again. What
bank? How can I tell? I only believe that you have long life before you.
I know that you take all my words now for a set speech prepared
beforehand, but maybe you will remember them after. They may be of use
some time. That's why I speak. It's as well that you only killed the old
woman. If you'd invented another theory you might perhaps have done
something a thousand times more hideous. You ought to thank God,
perhaps. How do you know? Perhaps God is saving you for something. But
keep a good heart and have less fear! Are you afraid of the great
expiation before you? No, it would be shameful to be afraid of it. Since
you have taken such a step, you must harden your heart. There is justice
in it. You must fulfil the demands of justice. I know that you don't
believe it, but indeed, life will bring you through. You will live it
down in time. What you need now is fresh air, fresh air, fresh air!''

Raskolnikov positively started.

``But who are you? what prophet are you? From the height of what
majestic calm do you proclaim these words of wisdom?''

``Who am I? I am a man with nothing to hope for, that's all. A man
perhaps of feeling and sympathy, maybe of some knowledge too, but my day
is over. But you are a different matter, there is life waiting for you.
Though, who knows? maybe your life, too, will pass off in smoke and come
to nothing. Come, what does it matter, that you will pass into another
class of men? It's not comfort you regret, with your heart! What of it
that perhaps no one will see you for so long? It's not time, but
yourself that will decide that. Be the sun and all will see you. The sun
has before all to be the sun. Why are you smiling again? At my being
such a Schiller? I bet you're imagining that I am trying to get round
you by flattery. Well, perhaps I am, he-he-he! Perhaps you'd better not
believe my word, perhaps you'd better never believe it altogether---I'm
made that way, I confess it. But let me add, you can judge for yourself,
I think, how far I am a base sort of man and how far I am honest.''

``When do you mean to arrest me?''

``Well, I can let you walk about another day or two. Think it over, my
dear fellow, and pray to God. It's more in your interest, believe me.''

``And what if I run away?'' asked Raskolnikov with a strange smile.

``No, you won't run away. A peasant would run away, a fashionable
dissenter would run away, the flunkey of another man's thought, for
you've only to show him the end of your little finger and he'll be ready
to believe in anything for the rest of his life. But you've ceased to
believe in your theory already, what will you run away with? And what
would you do in hiding? It would be hateful and difficult for you, and
what you need more than anything in life is a definite position, an
atmosphere to suit you. And what sort of atmosphere would you have? If
you ran away, you'd come back to yourself. \emph{You can't get on
without us.} And if I put you in prison---say you've been there a month,
or two, or three---remember my word, you'll confess of yourself and
perhaps to your own surprise. You won't know an hour beforehand that you
are coming with a confession. I am convinced that you will decide, `to
take your suffering.' You don't believe my words now, but you'll come to
it of yourself. For suffering, Rodion Romanovitch, is a great thing.
Never mind my having grown fat, I know all the same. Don't laugh at it,
there's an idea in suffering, Nokolay is right. No, you won't run away,
Rodion Romanovitch.''

Raskolnikov got up and took his cap. Porfiry Petrovitch also rose.

``Are you going for a walk? The evening will be fine, if only we don't
have a storm. Though it would be a good thing to freshen the air.''

He, too, took his cap.

``Porfiry Petrovitch, please don't take up the notion that I have
confessed to you to-day,'' Raskolnikov pronounced with sullen
insistence. ``You're a strange man and I have listened to you from
simple curiosity. But I have admitted nothing, remember that!''

``Oh, I know that, I'll remember. Look at him, he's trembling! Don't be
uneasy, my dear fellow, have it your own way. Walk about a bit, you
won't be able to walk too far. If anything happens, I have one request
to make of you,'' he added, dropping his voice. ``It's an awkward one,
but important. If anything were to happen (though indeed I don't believe
in it and think you quite incapable of it), yet in case you were taken
during these forty or fifty hours with the notion of putting an end to
the business in some other way, in some fantastic fashion---laying hands
on yourself---(it's an absurd proposition, but you must forgive me for
it) do leave a brief but precise note, only two lines, and mention the
stone. It will be more generous. Come, till we meet! Good thoughts and
sound decisions to you!''

Porfiry went out, stooping and avoiding looking at Raskolnikov. The
latter went to the window and waited with irritable impatience till he
calculated that Porfiry had reached the street and moved away. Then he
too went hurriedly out of the room.

\protect\hypertarget{link2HCH0034}{}{}

\mychapter{3}{Chapter III}\label{chapter-iii-5}

He hurried to Svidrigaïlov's. What he had to hope from that man he did
not know. But that man had some hidden power over him. Having once
recognised this, he could not rest, and now the time had come.

On the way, one question particularly worried him: had Svidrigaïlov been
to Porfiry's?

As far as he could judge, he would swear to it, that he had not. He
pondered again and again, went over Porfiry's visit; no, he hadn't been,
of course he hadn't.

But if he had not been yet, would he go? Meanwhile, for the present he
fancied he couldn't. Why? He could not have explained, but if he could,
he would not have wasted much thought over it at the moment. It all
worried him and at the same time he could not attend to it. Strange to
say, none would have believed it perhaps, but he only felt a faint vague
anxiety about his immediate future. Another, much more important anxiety
tormented him---it concerned himself, but in a different, more vital
way. Moreover, he was conscious of immense moral fatigue, though his
mind was working better that morning than it had done of late.

And was it worth while, after all that had happened, to contend with
these new trivial difficulties? Was it worth while, for instance, to
manoeuvre that Svidrigaïlov should not go to Porfiry's? Was it worth
while to investigate, to ascertain the facts, to waste time over anyone
like Svidrigaïlov?

Oh, how sick he was of it all!

And yet he was hastening to Svidrigaïlov; could he be expecting
something \emph{new} from him, information, or means of escape? Men will
catch at straws! Was it destiny or some instinct bringing them together?
Perhaps it was only fatigue, despair; perhaps it was not Svidrigaïlov
but some other whom he needed, and Svidrigaïlov had simply presented
himself by chance. Sonia? But what should he go to Sonia for now? To beg
her tears again? He was afraid of Sonia, too. Sonia stood before him as
an irrevocable sentence. He must go his own way or hers. At that moment
especially he did not feel equal to seeing her. No, would it not be
better to try Svidrigaïlov? And he could not help inwardly owning that
he had long felt that he must see him for some reason.

But what could they have in common? Their very evil-doing could not be
of the same kind. The man, moreover, was very unpleasant, evidently
depraved, undoubtedly cunning and deceitful, possibly malignant. Such
stories were told about him. It is true he was befriending Katerina
Ivanovna's children, but who could tell with what motive and what it
meant? The man always had some design, some project.

There was another thought which had been continually hovering of late
about Raskolnikov's mind, and causing him great uneasiness. It was so
painful that he made distinct efforts to get rid of it. He sometimes
thought that Svidrigaïlov was dogging his footsteps. Svidrigaïlov had
found out his secret and had had designs on Dounia. What if he had them
still? Wasn't it practically certain that he had? And what if, having
learnt his secret and so having gained power over him, he were to use it
as a weapon against Dounia?

This idea sometimes even tormented his dreams, but it had never
presented itself so vividly to him as on his way to Svidrigaïlov. The
very thought moved him to gloomy rage. To begin with, this would
transform everything, even his own position; he would have at once to
confess his secret to Dounia. Would he have to give himself up perhaps
to prevent Dounia from taking some rash step? The letter? This morning
Dounia had received a letter. From whom could she get letters in
Petersburg? Luzhin, perhaps? It's true Razumihin was there to protect
her, but Razumihin knew nothing of the position. Perhaps it was his duty
to tell Razumihin? He thought of it with repugnance.

In any case he must see Svidrigaïlov as soon as possible, he decided
finally. Thank God, the details of the interview were of little
consequence, if only he could get at the root of the matter; but if
Svidrigaïlov were capable\ldots{} if he were intriguing against
Dounia---then\ldots{}

Raskolnikov was so exhausted by what he had passed through that month
that he could only decide such questions in one way; ``then I shall kill
him,'' he thought in cold despair.

A sudden anguish oppressed his heart, he stood still in the middle of
the street and began looking about to see where he was and which way he
was going. He found himself in X. Prospect, thirty or forty paces from
the Hay Market, through which he had come. The whole second storey of
the house on the left was used as a tavern. All the windows were wide
open; judging from the figures moving at the windows, the rooms were
full to overflowing. There were sounds of singing, of clarionet and
violin, and the boom of a Turkish drum. He could hear women shrieking.
He was about to turn back wondering why he had come to the X. Prospect,
when suddenly at one of the end windows he saw Svidrigaïlov, sitting at
a tea-table right in the open window with a pipe in his mouth.
Raskolnikov was dreadfully taken aback, almost terrified. Svidrigaïlov
was silently watching and scrutinising him and, what struck Raskolnikov
at once, seemed to be meaning to get up and slip away unobserved.
Raskolnikov at once pretended not to have seen him, but to be looking
absent-mindedly away, while he watched him out of the corner of his eye.
His heart was beating violently. Yet, it was evident that Svidrigaïlov
did not want to be seen. He took the pipe out of his mouth and was on
the point of concealing himself, but as he got up and moved back his
chair, he seemed to have become suddenly aware that Raskolnikov had seen
him, and was watching him. What had passed between them was much the
same as what happened at their first meeting in Raskolnikov's room. A
sly smile came into Svidrigaïlov's face and grew broader and broader.
Each knew that he was seen and watched by the other. At last
Svidrigaïlov broke into a loud laugh.

``Well, well, come in if you want me; I am here!'' he shouted from the
window.

Raskolnikov went up into the tavern. He found Svidrigaïlov in a tiny
back room, adjoining the saloon in which merchants, clerks and numbers
of people of all sorts were drinking tea at twenty little tables to the
desperate bawling of a chorus of singers. The click of billiard balls
could be heard in the distance. On the table before Svidrigaïlov stood
an open bottle and a glass half full of champagne. In the room he found
also a boy with a little hand organ, a healthy-looking red-cheeked girl
of eighteen, wearing a tucked-up striped skirt, and a Tyrolese hat with
ribbons. In spite of the chorus in the other room, she was singing some
servants' hall song in a rather husky contralto, to the accompaniment of
the organ.

``Come, that's enough,'' Svidrigaïlov stopped her at Raskolnikov's
entrance. The girl at once broke off and stood waiting respectfully. She
had sung her guttural rhymes, too, with a serious and respectful
expression in her face.

``Hey, Philip, a glass!'' shouted Svidrigaïlov.

``I won't drink anything,'' said Raskolnikov.

``As you like, I didn't mean it for you. Drink, Katia! I don't want
anything more to-day, you can go.'' He poured her out a full glass, and
laid down a yellow note.

Katia drank off her glass of wine, as women do, without putting it down,
in twenty gulps, took the note and kissed Svidrigaïlov's hand, which he
allowed quite seriously. She went out of the room and the boy trailed
after her with the organ. Both had been brought in from the street.
Svidrigaïlov had not been a week in Petersburg, but everything about him
was already, so to speak, on a patriarchal footing; the waiter, Philip,
was by now an old friend and very obsequious.

The door leading to the saloon had a lock on it. Svidrigaïlov was at
home in this room and perhaps spent whole days in it. The tavern was
dirty and wretched, not even second-rate.

``I was going to see you and looking for you,'' Raskolnikov began, ``but
I don't know what made me turn from the Hay Market into the X. Prospect
just now. I never take this turning. I turn to the right from the Hay
Market. And this isn't the way to you. I simply turned and here you are.
It is strange!''

``Why don't you say at once `it's a miracle'?''

``Because it may be only chance.''

``Oh, that's the way with all you folk,'' laughed Svidrigaïlov. ``You
won't admit it, even if you do inwardly believe it a miracle! Here you
say that it may be only chance. And what cowards they all are here,
about having an opinion of their own, you can't fancy, Rodion
Romanovitch. I don't mean you, you have an opinion of your own and are
not afraid to have it. That's how it was you attracted my curiosity.''

``Nothing else?''

``Well, that's enough, you know,'' Svidrigaïlov was obviously
exhilarated, but only slightly so, he had not had more than half a glass
of wine.

``I fancy you came to see me before you knew that I was capable of
having what you call an opinion of my own,'' observed Raskolnikov.

``Oh, well, it was a different matter. Everyone has his own plans. And
apropos of the miracle let me tell you that I think you have been asleep
for the last two or three days. I told you of this tavern myself, there
is no miracle in your coming straight here. I explained the way myself,
told you where it was, and the hours you could find me here. Do you
remember?''

``I don't remember,'' answered Raskolnikov with surprise.

``I believe you. I told you twice. The address has been stamped
mechanically on your memory. You turned this way mechanically and yet
precisely according to the direction, though you are not aware of it.
When I told you then, I hardly hoped you understood me. You give
yourself away too much, Rodion Romanovitch. And another thing, I'm
convinced there are lots of people in Petersburg who talk to themselves
as they walk. This is a town of crazy people. If only we had scientific
men, doctors, lawyers and philosophers might make most valuable
investigations in Petersburg each in his own line. There are few places
where there are so many gloomy, strong and queer influences on the soul
of man as in Petersburg. The mere influences of climate mean so much.
And it's the administrative centre of all Russia and its character must
be reflected on the whole country. But that is neither here nor there
now. The point is that I have several times watched you. You walk out of
your house---holding your head high---twenty paces from home you let it
sink, and fold your hands behind your back. You look and evidently see
nothing before nor beside you. At last you begin moving your lips and
talking to yourself, and sometimes you wave one hand and declaim, and at
last stand still in the middle of the road. That's not at all the thing.
Someone may be watching you besides me, and it won't do you any good.
It's nothing really to do with me and I can't cure you, but, of course,
you understand me.''

``Do you know that I am being followed?'' asked Raskolnikov, looking
inquisitively at him.

``No, I know nothing about it,'' said Svidrigaïlov, seeming surprised.

``Well, then, let us leave me alone,'' Raskolnikov muttered, frowning.

``Very good, let us leave you alone.''

``You had better tell me, if you come here to drink, and directed me
twice to come here to you, why did you hide, and try to get away just
now when I looked at the window from the street? I saw it.''

``He-he! And why was it you lay on your sofa with closed eyes and
pretended to be asleep, though you were wide awake while I stood in your
doorway? I saw it.''

``I may have had\ldots{} reasons. You know that yourself.''

``And I may have had my reasons, though you don't know them.''

Raskolnikov dropped his right elbow on the table, leaned his chin in the
fingers of his right hand, and stared intently at Svidrigaïlov. For a
full minute he scrutinised his face, which had impressed him before. It
was a strange face, like a mask; white and red, with bright red lips,
with a flaxen beard, and still thick flaxen hair. His eyes were somehow
too blue and their expression somehow too heavy and fixed. There was
something awfully unpleasant in that handsome face, which looked so
wonderfully young for his age. Svidrigaïlov was smartly dressed in light
summer clothes and was particularly dainty in his linen. He wore a huge
ring with a precious stone in it.

``Have I got to bother myself about you, too, now?'' said Raskolnikov
suddenly, coming with nervous impatience straight to the point. ``Even
though perhaps you are the most dangerous man if you care to injure me,
I don't want to put myself out any more. I will show you at once that I
don't prize myself as you probably think I do. I've come to tell you at
once that if you keep to your former intentions with regard to my sister
and if you think to derive any benefit in that direction from what has
been discovered of late, I will kill you before you get me locked up.
You can reckon on my word. You know that I can keep it. And in the
second place if you want to tell me anything---for I keep fancying all
this time that you have something to tell me---make haste and tell it,
for time is precious and very likely it will soon be too late.''

``Why in such haste?'' asked Svidrigaïlov, looking at him curiously.

``Everyone has his plans,'' Raskolnikov answered gloomily and
impatiently.

``You urged me yourself to frankness just now, and at the first question
you refuse to answer,'' Svidrigaïlov observed with a smile. ``You keep
fancying that I have aims of my own and so you look at me with
suspicion. Of course it's perfectly natural in your position. But though
I should like to be friends with you, I shan't trouble myself to
convince you of the contrary. The game isn't worth the candle and I
wasn't intending to talk to you about anything special.''

``What did you want me, for, then? It was you who came hanging about
me.''

``Why, simply as an interesting subject for observation. I liked the
fantastic nature of your position---that's what it was! Besides you are
the brother of a person who greatly interested me, and from that person
I had in the past heard a very great deal about you, from which I
gathered that you had a great influence over her; isn't that enough?
Ha-ha-ha! Still I must admit that your question is rather complex, and
is difficult for me to answer. Here, you, for instance, have come to me
not only for a definite object, but for the sake of hearing something
new. Isn't that so? Isn't that so?'' persisted Svidrigaïlov with a sly
smile. ``Well, can't you fancy then that I, too, on my way here in the
train was reckoning on you, on your telling me something new, and on my
making some profit out of you! You see what rich men we are!''

``What profit could you make?''

``How can I tell you? How do I know? You see in what a tavern I spend
all my time and it's my enjoyment, that's to say it's no great
enjoyment, but one must sit somewhere; that poor Katia now---you saw
her?\ldots{} If only I had been a glutton now, a club gourmand, but you
see I can eat this.''

He pointed to a little table in the corner where the remnants of a
terrible-looking beef-steak and potatoes lay on a tin dish.

``Have you dined, by the way? I've had something and want nothing more.
I don't drink, for instance, at all. Except for champagne I never touch
anything, and not more than a glass of that all the evening, and even
that is enough to make my head ache. I ordered it just now to wind
myself up, for I am just going off somewhere and you see me in a
peculiar state of mind. That was why I hid myself just now like a
schoolboy, for I was afraid you would hinder me. But I believe,'' he
pulled out his watch, ``I can spend an hour with you. It's half-past
four now. If only I'd been something, a landowner, a father, a cavalry
officer, a photographer, a journalist\ldots{} I am nothing, no
specialty, and sometimes I am positively bored. I really thought you
would tell me something new.''

``But what are you, and why have you come here?''

``What am I? You know, a gentleman, I served for two years in the
cavalry, then I knocked about here in Petersburg, then I married Marfa
Petrovna and lived in the country. There you have my biography!''

``You are a gambler, I believe?''

``No, a poor sort of gambler. A card-sharper---not a gambler.''

``You have been a card-sharper then?''

``Yes, I've been a card-sharper too.''

``Didn't you get thrashed sometimes?''

``It did happen. Why?''

``Why, you might have challenged them\ldots{} altogether it must have
been lively.''

``I won't contradict you, and besides I am no hand at philosophy. I
confess that I hastened here for the sake of the women.''

``As soon as you buried Marfa Petrovna?''

``Quite so,'' Svidrigaïlov smiled with engaging candour. ``What of it?
You seem to find something wrong in my speaking like that about women?''

``You ask whether I find anything wrong in vice?''

``Vice! Oh, that's what you are after! But I'll answer you in order,
first about women in general; you know I am fond of talking. Tell me,
what should I restrain myself for? Why should I give up women, since I
have a passion for them? It's an occupation, anyway.''

``So you hope for nothing here but vice?''

``Oh, very well, for vice then. You insist on its being vice. But anyway
I like a direct question. In this vice at least there is something
permanent, founded indeed upon nature and not dependent on fantasy,
something present in the blood like an ever-burning ember, for ever
setting one on fire and, maybe, not to be quickly extinguished, even
with years. You'll agree it's an occupation of a sort.''

``That's nothing to rejoice at, it's a disease and a dangerous one.''

``Oh, that's what you think, is it! I agree, that it is a disease like
everything that exceeds moderation. And, of course, in this one must
exceed moderation. But in the first place, everybody does so in one way
or another, and in the second place, of course, one ought to be moderate
and prudent, however mean it may be, but what am I to do? If I hadn't
this, I might have to shoot myself. I am ready to admit that a decent
man ought to put up with being bored, but yet\ldots{}''

``And could you shoot yourself?''

``Oh, come!'' Svidrigaïlov parried with disgust. ``Please don't speak of
it,'' he added hurriedly and with none of the bragging tone he had shown
in all the previous conversation. His face quite changed. ``I admit it's
an unpardonable weakness, but I can't help it. I am afraid of death and
I dislike its being talked of. Do you know that I am to a certain extent
a mystic?''

``Ah, the apparitions of Marfa Petrovna! Do they still go on visiting
you?''

``Oh, don't talk of them; there have been no more in Petersburg,
confound them!'' he cried with an air of irritation. ``Let's rather talk
of that\ldots{} though\ldots{} H'm! I have not much time, and can't stay
long with you, it's a pity! I should have found plenty to tell you.''

``What's your engagement, a woman?''

``Yes, a woman, a casual incident\ldots{}. No, that's not what I want to
talk of.''

``And the hideousness, the filthiness of all your surroundings, doesn't
that affect you? Have you lost the strength to stop yourself?''

``And do you pretend to strength, too? He-he-he! You surprised me just
now, Rodion Romanovitch, though I knew beforehand it would be so. You
preach to me about vice and æsthetics! You---a Schiller, you---an
idealist! Of course that's all as it should be and it would be
surprising if it were not so, yet it is strange in reality\ldots{}. Ah,
what a pity I have no time, for you're a most interesting type! And,
by-the-way, are you fond of Schiller? I am awfully fond of him.''

``But what a braggart you are,'' Raskolnikov said with some disgust.

``Upon my word, I am not,'' answered Svidrigaïlov laughing. ``However, I
won't dispute it, let me be a braggart, why not brag, if it hurts no
one? I spent seven years in the country with Marfa Petrovna, so now when
I come across an intelligent person like you---intelligent and highly
interesting---I am simply glad to talk and, besides, I've drunk that
half-glass of champagne and it's gone to my head a little. And besides,
there's a certain fact that has wound me up tremendously, but about that
I\ldots{} will keep quiet. Where are you off to?'' he asked in alarm.

Raskolnikov had begun getting up. He felt oppressed and stifled and, as
it were, ill at ease at having come here. He felt convinced that
Svidrigaïlov was the most worthless scoundrel on the face of the earth.

``A-ach! Sit down, stay a little!'' Svidrigaïlov begged. ``Let them
bring you some tea, anyway. Stay a little, I won't talk nonsense, about
myself, I mean. I'll tell you something. If you like I'll tell you how a
woman tried `to save' me, as you would call it? It will be an answer to
your first question indeed, for the woman was your sister. May I tell
you? It will help to spend the time.''

``Tell me, but I trust that you\ldots{}''

``Oh, don't be uneasy. Besides, even in a worthless low fellow like me,
Avdotya Romanovna can only excite the deepest respect.''

\protect\hypertarget{link2HCH0035}{}{}

\mychapter{4}{Chapter IV}\label{chapter-iv-5}

``You know perhaps---yes, I told you myself,'' began Svidrigaïlov,
``that I was in the debtors' prison here, for an immense sum, and had
not any expectation of being able to pay it. There's no need to go into
particulars how Marfa Petrovna bought me out; do you know to what a
point of insanity a woman can sometimes love? She was an honest woman,
and very sensible, although completely uneducated. Would you believe
that this honest and jealous woman, after many scenes of hysterics and
reproaches, condescended to enter into a kind of contract with me which
she kept throughout our married life? She was considerably older than I,
and besides, she always kept a clove or something in her mouth. There
was so much swinishness in my soul and honesty too, of a sort, as to
tell her straight out that I couldn't be absolutely faithful to her.
This confession drove her to frenzy, but yet she seems in a way to have
liked my brutal frankness. She thought it showed I was unwilling to
deceive her if I warned her like this beforehand and for a jealous
woman, you know, that's the first consideration. After many tears an
unwritten contract was drawn up between us: first, that I would never
leave Marfa Petrovna and would always be her husband; secondly, that I
would never absent myself without her permission; thirdly, that I would
never set up a permanent mistress; fourthly, in return for this, Marfa
Petrovna gave me a free hand with the maidservants, but only with her
secret knowledge; fifthly, God forbid my falling in love with a woman of
our class; sixthly, in case I---which God forbid---should be visited by
a great serious passion I was bound to reveal it to Marfa Petrovna. On
this last score, however, Marfa Petrovna was fairly at ease. She was a
sensible woman and so she could not help looking upon me as a dissolute
profligate incapable of real love. But a sensible woman and a jealous
woman are two very different things, and that's where the trouble came
in. But to judge some people impartially we must renounce certain
preconceived opinions and our habitual attitude to the ordinary people
about us. I have reason to have faith in your judgment rather than in
anyone's. Perhaps you have already heard a great deal that was
ridiculous and absurd about Marfa Petrovna. She certainly had some very
ridiculous ways, but I tell you frankly that I feel really sorry for the
innumerable woes of which I was the cause. Well, and that's enough, I
think, by way of a decorous \emph{oraison funèbre} for the most tender
wife of a most tender husband. When we quarrelled, I usually held my
tongue and did not irritate her and that gentlemanly conduct rarely
failed to attain its object, it influenced her, it pleased her, indeed.
These were times when she was positively proud of me. But your sister
she couldn't put up with, anyway. And however she came to risk taking
such a beautiful creature into her house as a governess. My explanation
is that Marfa Petrovna was an ardent and impressionable woman and simply
fell in love herself---literally fell in love---with your sister. Well,
little wonder---look at Avdotya Romanovna! I saw the danger at the first
glance and what do you think, I resolved not to look at her even. But
Avdotya Romanovna herself made the first step, would you believe it?
Would you believe it too that Marfa Petrovna was positively angry with
me at first for my persistent silence about your sister, for my careless
reception of her continual adoring praises of Avdotya Romanovna. I don't
know what it was she wanted! Well, of course, Marfa Petrovna told
Avdotya Romanovna every detail about me. She had the unfortunate habit
of telling literally everyone all our family secrets and continually
complaining of me; how could she fail to confide in such a delightful
new friend? I expect they talked of nothing else but me and no doubt
Avdotya Romanovna heard all those dark mysterious rumours that were
current about me\ldots{}. I don't mind betting that you too have heard
something of the sort already?''

``I have. Luzhin charged you with having caused the death of a child. Is
that true?''

``Don't refer to those vulgar tales, I beg,'' said Svidrigaïlov with
disgust and annoyance. ``If you insist on wanting to know about all that
idiocy, I will tell you one day, but now\ldots{}''

``I was told too about some footman of yours in the country whom you
treated badly.''

``I beg you to drop the subject,'' Svidrigaïlov interrupted again with
obvious impatience.

``Was that the footman who came to you after death to fill your
pipe?\ldots{} you told me about it yourself.'' Raskolnikov felt more and
more irritated.

Svidrigaïlov looked at him attentively and Raskolnikov fancied he caught
a flash of spiteful mockery in that look. But Svidrigaïlov restrained
himself and answered very civilly:

``Yes, it was. I see that you, too, are extremely interested and shall
feel it my duty to satisfy your curiosity at the first opportunity. Upon
my soul! I see that I really might pass for a romantic figure with some
people. Judge how grateful I must be to Marfa Petrovna for having
repeated to Avdotya Romanovna such mysterious and interesting gossip
about me. I dare not guess what impression it made on her, but in any
case it worked in my interests. With all Avdotya Romanovna's natural
aversion and in spite of my invariably gloomy and repellent aspect---she
did at least feel pity for me, pity for a lost soul. And if once a
girl's heart is moved to \emph{pity}, it's more dangerous than anything.
She is bound to want to `save him,' to bring him to his senses, and lift
him up and draw him to nobler aims, and restore him to new life and
usefulness---well, we all know how far such dreams can go. I saw at once
that the bird was flying into the cage of herself. And I too made ready.
I think you are frowning, Rodion Romanovitch? There's no need. As you
know, it all ended in smoke. (Hang it all, what a lot I am drinking!) Do
you know, I always, from the very beginning, regretted that it wasn't
your sister's fate to be born in the second or third century A.D., as
the daughter of a reigning prince or some governor or pro-consul in Asia
Minor. She would undoubtedly have been one of those who would endure
martyrdom and would have smiled when they branded her bosom with hot
pincers. And she would have gone to it of herself. And in the fourth or
fifth century she would have walked away into the Egyptian desert and
would have stayed there thirty years living on roots and ecstasies and
visions. She is simply thirsting to face some torture for someone, and
if she can't get her torture, she'll throw herself out of a window. I've
heard something of a Mr. Razumihin---he's said to be a sensible fellow;
his surname suggests it, indeed. He's probably a divinity student. Well,
he'd better look after your sister! I believe I understand her, and I am
proud of it. But at the beginning of an acquaintance, as you know, one
is apt to be more heedless and stupid. One doesn't see clearly. Hang it
all, why is she so handsome? It's not my fault. In fact, it began on my
side with a most irresistible physical desire. Avdotya Romanovna is
awfully chaste, incredibly and phenomenally so. Take note, I tell you
this about your sister as a fact. She is almost morbidly chaste, in
spite of her broad intelligence, and it will stand in her way. There
happened to be a girl in the house then, Parasha, a black-eyed wench,
whom I had never seen before---she had just come from another
village---very pretty, but incredibly stupid: she burst into tears,
wailed so that she could be heard all over the place and caused scandal.
One day after dinner Avdotya Romanovna followed me into an avenue in the
garden and with flashing eyes \emph{insisted} on my leaving poor Parasha
alone. It was almost our first conversation by ourselves. I, of course,
was only too pleased to obey her wishes, tried to appear disconcerted,
embarrassed, in fact played my part not badly. Then came interviews,
mysterious conversations, exhortations, entreaties, supplications, even
tears---would you believe it, even tears? Think what the passion for
propaganda will bring some girls to! I, of course, threw it all on my
destiny, posed as hungering and thirsting for light, and finally
resorted to the most powerful weapon in the subjection of the female
heart, a weapon which never fails one. It's the well-known
resource---flattery. Nothing in the world is harder than speaking the
truth and nothing easier than flattery. If there's the hundredth part of
a false note in speaking the truth, it leads to a discord, and that
leads to trouble. But if all, to the last note, is false in flattery, it
is just as agreeable, and is heard not without satisfaction. It may be a
coarse satisfaction, but still a satisfaction. And however coarse the
flattery, at least half will be sure to seem true. That's so for all
stages of development and classes of society. A vestal virgin might be
seduced by flattery. I can never remember without laughter how I once
seduced a lady who was devoted to her husband, her children, and her
principles. What fun it was and how little trouble! And the lady really
had principles---of her own, anyway. All my tactics lay in simply being
utterly annihilated and prostrate before her purity. I flattered her
shamelessly, and as soon as I succeeded in getting a pressure of the
hand, even a glance from her, I would reproach myself for having
snatched it by force, and would declare that she had resisted, so that I
could never have gained anything but for my being so unprincipled. I
maintained that she was so innocent that she could not foresee my
treachery, and yielded to me unconsciously, unawares, and so on. In
fact, I triumphed, while my lady remained firmly convinced that she was
innocent, chaste, and faithful to all her duties and obligations and had
succumbed quite by accident. And how angry she was with me when I
explained to her at last that it was my sincere conviction that she was
just as eager as I. Poor Marfa Petrovna was awfully weak on the side of
flattery, and if I had only cared to, I might have had all her property
settled on me during her lifetime. (I am drinking an awful lot of wine
now and talking too much.) I hope you won't be angry if I mention now
that I was beginning to produce the same effect on Avdotya Romanovna.
But I was stupid and impatient and spoiled it all. Avdotya Romanovna had
several times---and one time in particular---been greatly displeased by
the expression of my eyes, would you believe it? There was sometimes a
light in them which frightened her and grew stronger and stronger and
more unguarded till it was hateful to her. No need to go into detail,
but we parted. There I acted stupidly again. I fell to jeering in the
coarsest way at all such propaganda and efforts to convert me; Parasha
came on to the scene again, and not she alone; in fact there was a
tremendous to-do. Ah, Rodion Romanovitch, if you could only see how your
sister's eyes can flash sometimes! Never mind my being drunk at this
moment and having had a whole glass of wine. I am speaking the truth. I
assure you that this glance has haunted my dreams; the very rustle of
her dress was more than I could stand at last. I really began to think
that I might become epileptic. I could never have believed that I could
be moved to such a frenzy. It was essential, indeed, to be reconciled,
but by then it was impossible. And imagine what I did then! To what a
pitch of stupidity a man can be brought by frenzy! Never undertake
anything in a frenzy, Rodion Romanovitch. I reflected that Avdotya
Romanovna was after all a beggar (ach, excuse me, that's not the
word\ldots{} but does it matter if it expresses the meaning?), that she
lived by her work, that she had her mother and you to keep (ach, hang
it, you are frowning again), and I resolved to offer her all my
money---thirty thousand roubles I could have realised then---if she
would run away with me here, to Petersburg. Of course I should have
vowed eternal love, rapture, and so on. Do you know, I was so wild about
her at that time that if she had told me to poison Marfa Petrovna or to
cut her throat and to marry herself, it would have been done at once!
But it ended in the catastrophe of which you know already. You can fancy
how frantic I was when I heard that Marfa Petrovna had got hold of that
scoundrelly attorney, Luzhin, and had almost made a match between
them---which would really have been just the same thing as I was
proposing. Wouldn't it? Wouldn't it? I notice that you've begun to be
very attentive\ldots{} you interesting young man\ldots{}.''

Svidrigaïlov struck the table with his fist impatiently. He was flushed.
Raskolnikov saw clearly that the glass or glass and a half of champagne
that he had sipped almost unconsciously was affecting him---and he
resolved to take advantage of the opportunity. He felt very suspicious
of Svidrigaïlov.

``Well, after what you have said, I am fully convinced that you have
come to Petersburg with designs on my sister,'' he said directly to
Svidrigaïlov, in order to irritate him further.

``Oh, nonsense,'' said Svidrigaïlov, seeming to rouse himself. ``Why, I
told you\ldots{} besides your sister can't endure me.''

``Yes, I am certain that she can't, but that's not the point.''

``Are you so sure that she can't?'' Svidrigaïlov screwed up his eyes and
smiled mockingly. ``You are right, she doesn't love me, but you can
never be sure of what has passed between husband and wife or lover and
mistress. There's always a little corner which remains a secret to the
world and is only known to those two. Will you answer for it that
Avdotya Romanovna regarded me with aversion?''

``From some words you've dropped, I notice that you still have
designs---and of course evil ones---on Dounia and mean to carry them out
promptly.''

``What, have I dropped words like that?'' Svidrigaïlov asked in naïve
dismay, taking not the slightest notice of the epithet bestowed on his
designs.

``Why, you are dropping them even now. Why are you so frightened? What
are you so afraid of now?''

``Me---afraid? Afraid of you? You have rather to be afraid of me,
\emph{cher ami}. But what nonsense\ldots{}. I've drunk too much though,
I see that. I was almost saying too much again. Damn the wine! Hi!
there, water!''

He snatched up the champagne bottle and flung it without ceremony out of
the window. Philip brought the water.

``That's all nonsense!'' said Svidrigaïlov, wetting a towel and putting
it to his head. ``But I can answer you in one word and annihilate all
your suspicions. Do you know that I am going to get married?''

``You told me so before.''

``Did I? I've forgotten. But I couldn't have told you so for certain for
I had not even seen my betrothed; I only meant to. But now I really have
a betrothed and it's a settled thing, and if it weren't that I have
business that can't be put off, I would have taken you to see them at
once, for I should like to ask your advice. Ach, hang it, only ten
minutes left! See, look at the watch. But I must tell you, for it's an
interesting story, my marriage, in its own way. Where are you off to?
Going again?''

``No, I'm not going away now.''

``Not at all? We shall see. I'll take you there, I'll show you my
betrothed, only not now. For you'll soon have to be off. You have to go
to the right and I to the left. Do you know that Madame Resslich, the
woman I am lodging with now, eh? I know what you're thinking, that she's
the woman whose girl they say drowned herself in the winter. Come, are
you listening? She arranged it all for me. You're bored, she said, you
want something to fill up your time. For, you know, I am a gloomy,
depressed person. Do you think I'm light-hearted? No, I'm gloomy. I do
no harm, but sit in a corner without speaking a word for three days at a
time. And that Resslich is a sly hussy, I tell you. I know what she has
got in her mind; she thinks I shall get sick of it, abandon my wife and
depart, and she'll get hold of her and make a profit out of her---in our
class, of course, or higher. She told me the father was a broken-down
retired official, who has been sitting in a chair for the last three
years with his legs paralysed. The mamma, she said, was a sensible
woman. There is a son serving in the provinces, but he doesn't help;
there is a daughter, who is married, but she doesn't visit them. And
they've two little nephews on their hands, as though their own children
were not enough, and they've taken from school their youngest daughter,
a girl who'll be sixteen in another month, so that then she can be
married. She was for me. We went there. How funny it was! I present
myself---a landowner, a widower, of a well-known name, with connections,
with a fortune. What if I am fifty and she is not sixteen? Who thinks of
that? But it's fascinating, isn't it? It is fascinating, ha-ha! You
should have seen how I talked to the papa and mamma. It was worth paying
to have seen me at that moment. She comes in, curtseys, you can fancy,
still in a short frock---an unopened bud! Flushing like a sunset---she
had been told, no doubt. I don't know how you feel about female faces,
but to my mind these sixteen years, these childish eyes, shyness and
tears of bashfulness are better than beauty; and she is a perfect little
picture, too. Fair hair in little curls, like a lamb's, full little rosy
lips, tiny feet, a charmer!\ldots{} Well, we made friends. I told them I
was in a hurry owing to domestic circumstances, and the next day, that
is the day before yesterday, we were betrothed. When I go now I take her
on my knee at once and keep her there\ldots{}. Well, she flushes like a
sunset and I kiss her every minute. Her mamma of course impresses on her
that this is her husband and that this must be so. It's simply
delicious! The present betrothed condition is perhaps better than
marriage. Here you have what is called \emph{la nature et la vérité},
ha-ha! I've talked to her twice, she is far from a fool. Sometimes she
steals a look at me that positively scorches me. Her face is like
Raphael's Madonna. You know, the Sistine Madonna's face has something
fantastic in it, the face of mournful religious ecstasy. Haven't you
noticed it? Well, she's something in that line. The day after we'd been
betrothed, I bought her presents to the value of fifteen hundred
roubles---a set of diamonds and another of pearls and a silver
dressing-case as large as this, with all sorts of things in it, so that
even my Madonna's face glowed. I sat her on my knee, yesterday, and I
suppose rather too unceremoniously---she flushed crimson and the tears
started, but she didn't want to show it. We were left alone, she
suddenly flung herself on my neck (for the first time of her own
accord), put her little arms round me, kissed me, and vowed that she
would be an obedient, faithful, and good wife, would make me happy,
would devote all her life, every minute of her life, would sacrifice
everything, everything, and that all she asks in return is my
\emph{respect}, and that she wants `nothing, nothing more from me, no
presents.' You'll admit that to hear such a confession, alone, from an
angel of sixteen in a muslin frock, with little curls, with a flush of
maiden shyness in her cheeks and tears of enthusiasm in her eyes is
rather fascinating! Isn't it fascinating? It's worth paying for, isn't
it? Well\ldots{} listen, we'll go to see my betrothed, only not just
now!''

``The fact is this monstrous difference in age and development excites
your sensuality! Will you really make such a marriage?''

``Why, of course. Everyone thinks of himself, and he lives most gaily
who knows best how to deceive himself. Ha-ha! But why are you so keen
about virtue? Have mercy on me, my good friend. I am a sinful man.
Ha-ha-ha!''

``But you have provided for the children of Katerina Ivanovna.
Though\ldots{} though you had your own reasons\ldots{}. I understand it
all now.''

``I am always fond of children, very fond of them,'' laughed
Svidrigaïlov. ``I can tell you one curious instance of it. The first day
I came here I visited various haunts, after seven years I simply rushed
at them. You probably notice that I am not in a hurry to renew
acquaintance with my old friends. I shall do without them as long as I
can. Do you know, when I was with Marfa Petrovna in the country, I was
haunted by the thought of these places where anyone who knows his way
about can find a great deal. Yes, upon my soul! The peasants have vodka,
the educated young people, shut out from activity, waste themselves in
impossible dreams and visions and are crippled by theories; Jews have
sprung up and are amassing money, and all the rest give themselves up to
debauchery. From the first hour the town reeked of its familiar odours.
I chanced to be in a frightful den---I like my dens dirty---it was a
dance, so called, and there was a \emph{cancan} such as I never saw in
my day. Yes, there you have progress. All of a sudden I saw a little
girl of thirteen, nicely dressed, dancing with a specialist in that
line, with another one \emph{vis-à-vis}. Her mother was sitting on a
chair by the wall. You can't fancy what a \emph{cancan} that was! The
girl was ashamed, blushed, at last felt insulted, and began to cry. Her
partner seized her and began whirling her round and performing before
her; everyone laughed and---I like your public, even the \emph{cancan}
public---they laughed and shouted, `Serves her right---serves her right!
Shouldn't bring children!' Well, it's not my business whether that
consoling reflection was logical or not. I at once fixed on my plan, sat
down by the mother, and began by saying that I too was a stranger and
that people here were ill-bred and that they couldn't distinguish decent
folks and treat them with respect, gave her to understand that I had
plenty of money, offered to take them home in my carriage. I took them
home and got to know them. They were lodging in a miserable little hole
and had only just arrived from the country. She told me that she and her
daughter could only regard my acquaintance as an honour. I found out
that they had nothing of their own and had come to town upon some legal
business. I proffered my services and money. I learnt that they had gone
to the dancing saloon by mistake, believing that it was a genuine
dancing class. I offered to assist in the young girl's education in
French and dancing. My offer was accepted with enthusiasm as an
honour---and we are still friendly\ldots{}. If you like, we'll go and
see them, only not just now.''

``Stop! Enough of your vile, nasty anecdotes, depraved vile, sensual
man!''

``Schiller, you are a regular Schiller! \emph{O la vertu va-t-elle se
nicher?} But you know I shall tell you these things on purpose, for the
pleasure of hearing your outcries!''

``I dare say. I can see I am ridiculous myself,'' muttered Raskolnikov
angrily.

Svidrigaïlov laughed heartily; finally he called Philip, paid his bill,
and began getting up.

``I say, but I am drunk, \emph{assez causé},'' he said. ``It's been a
pleasure.''

``I should rather think it must be a pleasure!'' cried Raskolnikov,
getting up. ``No doubt it is a pleasure for a worn-out profligate to
describe such adventures with a monstrous project of the same sort in
his mind---especially under such circumstances and to such a man as
me\ldots{}. It's stimulating!''

``Well, if you come to that,'' Svidrigaïlov answered, scrutinising
Raskolnikov with some surprise, ``if you come to that, you are a
thorough cynic yourself. You've plenty to make you so, anyway. You can
understand a great deal\ldots{} and you can do a great deal too. But
enough. I sincerely regret not having had more talk with you, but I
shan't lose sight of you\ldots{}. Only wait a bit.''

Svidrigaïlov walked out of the restaurant. Raskolnikov walked out after
him. Svidrigaïlov was not however very drunk, the wine had affected him
for a moment, but it was passing off every minute. He was preoccupied
with something of importance and was frowning. He was apparently excited
and uneasy in anticipation of something. His manner to Raskolnikov had
changed during the last few minutes, and he was ruder and more sneering
every moment. Raskolnikov noticed all this, and he too was uneasy. He
became very suspicious of Svidrigaïlov and resolved to follow him.

They came out on to the pavement.

``You go to the right, and I to the left, or if you like, the other way.
Only \emph{adieu, mon plaisir}, may we meet again.''

And he walked to the right towards the Hay Market.

\protect\hypertarget{link2HCH0036}{}{}

\mychapter{5}{Chapter V}\label{chapter-v-5}

Raskolnikov walked after him.

``What's this?'' cried Svidrigaïlov turning round, ``I thought I
said\ldots{}''

``It means that I am not going to lose sight of you now.''

``What?''

Both stood still and gazed at one another, as though measuring their
strength.

``From all your half tipsy stories,'' Raskolnikov observed harshly, ``I
am \emph{positive} that you have not given up your designs on my sister,
but are pursuing them more actively than ever. I have learnt that my
sister received a letter this morning. You have hardly been able to sit
still all this time\ldots{}. You may have unearthed a wife on the way,
but that means nothing. I should like to make certain myself.''

Raskolnikov could hardly have said himself what he wanted and of what he
wished to make certain.

``Upon my word! I'll call the police!''

``Call away!''

Again they stood for a minute facing each other. At last Svidrigaïlov's
face changed. Having satisfied himself that Raskolnikov was not
frightened at his threat, he assumed a mirthful and friendly air.

``What a fellow! I purposely refrained from referring to your affair,
though I am devoured by curiosity. It's a fantastic affair. I've put it
off till another time, but you're enough to rouse the dead\ldots{}.
Well, let us go, only I warn you beforehand I am only going home for a
moment, to get some money; then I shall lock up the flat, take a cab and
go to spend the evening at the Islands. Now, now are you going to follow
me?''

``I'm coming to your lodgings, not to see you but Sofya Semyonovna, to
say I'm sorry not to have been at the funeral.''

``That's as you like, but Sofya Semyonovna is not at home. She has taken
the three children to an old lady of high rank, the patroness of some
orphan asylums, whom I used to know years ago. I charmed the old lady by
depositing a sum of money with her to provide for the three children of
Katerina Ivanovna and subscribing to the institution as well. I told her
too the story of Sofya Semyonovna in full detail, suppressing nothing.
It produced an indescribable effect on her. That's why Sofya Semyonovna
has been invited to call to-day at the X. Hotel where the lady is
staying for the time.''

``No matter, I'll come all the same.''

``As you like, it's nothing to me, but I won't come with you; here we
are at home. By the way, I am convinced that you regard me with
suspicion just because I have shown such delicacy and have not so far
troubled you with questions\ldots{} you understand? It struck you as
extraordinary; I don't mind betting it's that. Well, it teaches one to
show delicacy!''

``And to listen at doors!''

``Ah, that's it, is it?'' laughed Svidrigaïlov. ``Yes, I should have
been surprised if you had let that pass after all that has happened.
Ha-ha! Though I did understand something of the pranks you had been up
to and were telling Sofya Semyonovna about, what was the meaning of it?
Perhaps I am quite behind the times and can't understand. For goodness'
sake, explain it, my dear boy. Expound the latest theories!''

``You couldn't have heard anything. You're making it all up!''

``But I'm not talking about that (though I did hear something). No, I'm
talking of the way you keep sighing and groaning now. The Schiller in
you is in revolt every moment, and now you tell me not to listen at
doors. If that's how you feel, go and inform the police that you had
this mischance: you made a little mistake in your theory. But if you are
convinced that one mustn't listen at doors, but one may murder old women
at one's pleasure, you'd better be off to America and make haste. Run,
young man! There may still be time. I'm speaking sincerely. Haven't you
the money? I'll give you the fare.''

``I'm not thinking of that at all,'' Raskolnikov interrupted with
disgust.

``I understand (but don't put yourself out, don't discuss it if you
don't want to). I understand the questions you are worrying over---moral
ones, aren't they? Duties of citizen and man? Lay them all aside. They
are nothing to you now, ha-ha! You'll say you are still a man and a
citizen. If so you ought not to have got into this coil. It's no use
taking up a job you are not fit for. Well, you'd better shoot yourself,
or don't you want to?''

``You seem trying to enrage me, to make me leave you.''

``What a queer fellow! But here we are. Welcome to the staircase. You
see, that's the way to Sofya Semyonovna. Look, there is no one at home.
Don't you believe me? Ask Kapernaumov. She leaves the key with him. Here
is Madame de Kapernaumov herself. Hey, what? She is rather deaf. Has she
gone out? Where? Did you hear? She is not in and won't be till late in
the evening probably. Well, come to my room; you wanted to come and see
me, didn't you? Here we are. Madame Resslich's not at home. She is a
woman who is always busy, an excellent woman I assure you\ldots{}. She
might have been of use to you if you had been a little more sensible.
Now, see! I take this five-per-cent bond out of the bureau---see what a
lot I've got of them still---this one will be turned into cash to-day. I
mustn't waste any more time. The bureau is locked, the flat is locked,
and here we are again on the stairs. Shall we take a cab? I'm going to
the Islands. Would you like a lift? I'll take this carriage. Ah, you
refuse? You are tired of it! Come for a drive! I believe it will come on
to rain. Never mind, we'll put down the hood\ldots{}.''

Svidrigaïlov was already in the carriage. Raskolnikov decided that his
suspicions were at least for that moment unjust. Without answering a
word he turned and walked back towards the Hay Market. If he had only
turned round on his way he might have seen Svidrigaïlov get out not a
hundred paces off, dismiss the cab and walk along the pavement. But he
had turned the corner and could see nothing. Intense disgust drew him
away from Svidrigaïlov.

``To think that I could for one instant have looked for help from that
coarse brute, that depraved sensualist and blackguard!'' he cried.

Raskolnikov's judgment was uttered too lightly and hastily: there was
something about Svidrigaïlov which gave him a certain original, even a
mysterious character. As concerned his sister, Raskolnikov was convinced
that Svidrigaïlov would not leave her in peace. But it was too tiresome
and unbearable to go on thinking and thinking about this.

When he was alone, he had not gone twenty paces before he sank, as
usual, into deep thought. On the bridge he stood by the railing and
began gazing at the water. And his sister was standing close by him.

He met her at the entrance to the bridge, but passed by without seeing
her. Dounia had never met him like this in the street before and was
struck with dismay. She stood still and did not know whether to call to
him or not. Suddenly she saw Svidrigaïlov coming quickly from the
direction of the Hay Market.

He seemed to be approaching cautiously. He did not go on to the bridge,
but stood aside on the pavement, doing all he could to avoid
Raskolnikov's seeing him. He had observed Dounia for some time and had
been making signs to her. She fancied he was signalling to beg her not
to speak to her brother, but to come to him.

That was what Dounia did. She stole by her brother and went up to
Svidrigaïlov.

``Let us make haste away,'' Svidrigaïlov whispered to her, ``I don't
want Rodion Romanovitch to know of our meeting. I must tell you I've
been sitting with him in the restaurant close by, where he looked me up
and I had great difficulty in getting rid of him. He has somehow heard
of my letter to you and suspects something. It wasn't you who told him,
of course, but if not you, who then?''

``Well, we've turned the corner now,'' Dounia interrupted, ``and my
brother won't see us. I have to tell you that I am going no further with
you. Speak to me here. You can tell it all in the street.''

``In the first place, I can't say it in the street; secondly, you must
hear Sofya Semyonovna too; and, thirdly, I will show you some
papers\ldots{}. Oh well, if you won't agree to come with me, I shall
refuse to give any explanation and go away at once. But I beg you not to
forget that a very curious secret of your beloved brother's is entirely
in my keeping.''

Dounia stood still, hesitating, and looked at Svidrigaïlov with
searching eyes.

``What are you afraid of?'' he observed quietly. ``The town is not the
country. And even in the country you did me more harm than I did you.''

``Have you prepared Sofya Semyonovna?''

``No, I have not said a word to her and am not quite certain whether she
is at home now. But most likely she is. She has buried her stepmother
to-day: she is not likely to go visiting on such a day. For the time I
don't want to speak to anyone about it and I half regret having spoken
to you. The slightest indiscretion is as bad as betrayal in a thing like
this. I live there in that house, we are coming to it. That's the porter
of our house---he knows me very well; you see, he's bowing; he sees I'm
coming with a lady and no doubt he has noticed your face already and you
will be glad of that if you are afraid of me and suspicious. Excuse my
putting things so coarsely. I haven't a flat to myself; Sofya
Semyonovna's room is next to mine---she lodges in the next flat. The
whole floor is let out in lodgings. Why are you frightened like a child?
Am I really so terrible?''

Svidrigaïlov's lips were twisted in a condescending smile; but he was in
no smiling mood. His heart was throbbing and he could scarcely breathe.
He spoke rather loud to cover his growing excitement. But Dounia did not
notice this peculiar excitement, she was so irritated by his remark that
she was frightened of him like a child and that he was so terrible to
her.

``Though I know that you are not a man\ldots{} of honour, I am not in
the least afraid of you. Lead the way,'' she said with apparent
composure, but her face was very pale.

Svidrigaïlov stopped at Sonia's room.

``Allow me to inquire whether she is at home\ldots{}. She is not. How
unfortunate! But I know she may come quite soon. If she's gone out, it
can only be to see a lady about the orphans. Their mother is
dead\ldots{}. I've been meddling and making arrangements for them. If
Sofya Semyonovna does not come back in ten minutes, I will send her to
you, to-day if you like. This is my flat. These are my two rooms. Madame
Resslich, my landlady, has the next room. Now, look this way. I will
show you my chief piece of evidence: this door from my bedroom leads
into two perfectly empty rooms, which are to let. Here they are\ldots{}
You must look into them with some attention.''

Svidrigaïlov occupied two fairly large furnished rooms. Dounia was
looking about her mistrustfully, but saw nothing special in the
furniture or position of the rooms. Yet there was something to observe,
for instance, that Svidrigaïlov's flat was exactly between two sets of
almost uninhabited apartments. His rooms were not entered directly from
the passage, but through the landlady's two almost empty rooms.
Unlocking a door leading out of his bedroom, Svidrigaïlov showed Dounia
the two empty rooms that were to let. Dounia stopped in the doorway, not
knowing what she was called to look upon, but Svidrigaïlov hastened to
explain.

``Look here, at this second large room. Notice that door, it's locked.
By the door stands a chair, the only one in the two rooms. I brought it
from my rooms so as to listen more conveniently. Just the other side of
the door is Sofya Semyonovna's table; she sat there talking to Rodion
Romanovitch. And I sat here listening on two successive evenings, for
two hours each time---and of course I was able to learn something, what
do you think?''

``You listened?''

``Yes, I did. Now come back to my room; we can't sit down here.''

He brought Avdotya Romanovna back into his sitting-room and offered her
a chair. He sat down at the opposite side of the table, at least seven
feet from her, but probably there was the same glow in his eyes which
had once frightened Dounia so much. She shuddered and once more looked
about her distrustfully. It was an involuntary gesture; she evidently
did not wish to betray her uneasiness. But the secluded position of
Svidrigaïlov's lodging had suddenly struck her. She wanted to ask
whether his landlady at least were at home, but pride kept her from
asking. Moreover, she had another trouble in her heart incomparably
greater than fear for herself. She was in great distress.

``Here is your letter,'' she said, laying it on the table. ``Can it be
true what you write? You hint at a crime committed, you say, by my
brother. You hint at it too clearly; you daren't deny it now. I must
tell you that I'd heard of this stupid story before you wrote and don't
believe a word of it. It's a disgusting and ridiculous suspicion. I know
the story and why and how it was invented. You can have no proofs. You
promised to prove it. Speak! But let me warn you that I don't believe
you! I don't believe you!''

Dounia said this, speaking hurriedly, and for an instant the colour
rushed to her face.

``If you didn't believe it, how could you risk coming alone to my rooms?
Why have you come? Simply from curiosity?''

``Don't torment me. Speak, speak!''

``There's no denying that you are a brave girl. Upon my word, I thought
you would have asked Mr. Razumihin to escort you here. But he was not
with you nor anywhere near. I was on the look-out. It's spirited of you,
it proves you wanted to spare Rodion Romanovitch. But everything is
divine in you\ldots{}. About your brother, what am I to say to you?
You've just seen him yourself. What did you think of him?''

``Surely that's not the only thing you are building on?''

``No, not on that, but on his own words. He came here on two successive
evenings to see Sofya Semyonovna. I've shown you where they sat. He made
a full confession to her. He is a murderer. He killed an old woman, a
pawnbroker, with whom he had pawned things himself. He killed her sister
too, a pedlar woman called Lizaveta, who happened to come in while he
was murdering her sister. He killed them with an axe he brought with
him. He murdered them to rob them and he did rob them. He took money and
various things\ldots{}. He told all this, word for word, to Sofya
Semyonovna, the only person who knows his secret. But she has had no
share by word or deed in the murder; she was as horrified at it as you
are now. Don't be anxious, she won't betray him.''

``It cannot be,'' muttered Dounia, with white lips. She gasped for
breath. ``It cannot be. There was not the slightest cause, no sort of
ground\ldots{}. It's a lie, a lie!''

``He robbed her, that was the cause, he took money and things. It's true
that by his own admission he made no use of the money or things, but hid
them under a stone, where they are now. But that was because he dared
not make use of them.''

``But how could he steal, rob? How could he dream of it?'' cried Dounia,
and she jumped up from the chair. ``Why, you know him, and you've seen
him, can he be a thief?''

She seemed to be imploring Svidrigaïlov; she had entirely forgotten her
fear.

``There are thousands and millions of combinations and possibilities,
Avdotya Romanovna. A thief steals and knows he is a scoundrel, but I've
heard of a gentleman who broke open the mail. Who knows, very likely he
thought he was doing a gentlemanly thing! Of course I should not have
believed it myself if I'd been told of it as you have, but I believe my
own ears. He explained all the causes of it to Sofya Semyonovna too, but
she did not believe her ears at first, yet she believed her own eyes at
last.''

``What\ldots{} were the causes?''

``It's a long story, Avdotya Romanovna. Here's\ldots{} how shall I tell
you?---A theory of a sort, the same one by which I for instance consider
that a single misdeed is permissible if the principal aim is right, a
solitary wrongdoing and hundreds of good deeds! It's galling too, of
course, for a young man of gifts and overweening pride to know that if
he had, for instance, a paltry three thousand, his whole career, his
whole future would be differently shaped and yet not to have that three
thousand. Add to that, nervous irritability from hunger, from lodging in
a hole, from rags, from a vivid sense of the charm of his social
position and his sister's and mother's position too. Above all, vanity,
pride and vanity, though goodness knows he may have good qualities
too\ldots{}. I am not blaming him, please don't think it; besides, it's
not my business. A special little theory came in too---a theory of a
sort---dividing mankind, you see, into material and superior persons,
that is persons to whom the law does not apply owing to their
superiority, who make laws for the rest of mankind, the material, that
is. It's all right as a theory, \emph{une théorie comme une autre}.
Napoleon attracted him tremendously, that is, what affected him was that
a great many men of genius have not hesitated at wrongdoing, but have
overstepped the law without thinking about it. He seems to have fancied
that he was a genius too---that is, he was convinced of it for a time.
He has suffered a great deal and is still suffering from the idea that
he could make a theory, but was incapable of boldly overstepping the
law, and so he is not a man of genius. And that's humiliating for a
young man of any pride, in our day especially\ldots{}.''

``But remorse? You deny him any moral feeling then? Is he like that?''

``Ah, Avdotya Romanovna, everything is in a muddle now; not that it was
ever in very good order. Russians in general are broad in their ideas,
Avdotya Romanovna, broad like their land and exceedingly disposed to the
fantastic, the chaotic. But it's a misfortune to be broad without a
special genius. Do you remember what a lot of talk we had together on
this subject, sitting in the evenings on the terrace after supper? Why,
you used to reproach me with breadth! Who knows, perhaps we were talking
at the very time when he was lying here thinking over his plan. There
are no sacred traditions amongst us, especially in the educated class,
Avdotya Romanovna. At the best someone will make them up somehow for
himself out of books or from some old chronicle. But those are for the
most part the learned and all old fogeys, so that it would be almost
ill-bred in a man of society. You know my opinions in general, though. I
never blame anyone. I do nothing at all, I persevere in that. But we've
talked of this more than once before. I was so happy indeed as to
interest you in my opinions\ldots{}. You are very pale, Avdotya
Romanovna.''

``I know his theory. I read that article of his about men to whom all is
permitted. Razumihin brought it to me.''

``Mr. Razumihin? Your brother's article? In a magazine? Is there such an
article? I didn't know. It must be interesting. But where are you going,
Avdotya Romanovna?''

``I want to see Sofya Semyonovna,'' Dounia articulated faintly. ``How do
I go to her? She has come in, perhaps. I must see her at once. Perhaps
she\ldots{}''

Avdotya Romanovna could not finish. Her breath literally failed her.

``Sofya Semyonovna will not be back till night, at least I believe not.
She was to have been back at once, but if not, then she will not be in
till quite late.''

``Ah, then you are lying! I see\ldots{} you were lying\ldots{} lying all
the time\ldots{}. I don't believe you! I don't believe you!'' cried
Dounia, completely losing her head.

Almost fainting, she sank on to a chair which Svidrigaïlov made haste to
give her.

``Avdotya Romanovna, what is it? Control yourself! Here is some water.
Drink a little\ldots{}.''

He sprinkled some water over her. Dounia shuddered and came to herself.

``It has acted violently,'' Svidrigaïlov muttered to himself, frowning.
``Avdotya Romanovna, calm yourself! Believe me, he has friends. We will
save him. Would you like me to take him abroad? I have money, I can get
a ticket in three days. And as for the murder, he will do all sorts of
good deeds yet, to atone for it. Calm yourself. He may become a great
man yet. Well, how are you? How do you feel?''

``Cruel man! To be able to jeer at it! Let me go\ldots{}''

``Where are you going?''

``To him. Where is he? Do you know? Why is this door locked? We came in
at that door and now it is locked. When did you manage to lock it?''

``We couldn't be shouting all over the flat on such a subject. I am far
from jeering; it's simply that I'm sick of talking like this. But how
can you go in such a state? Do you want to betray him? You will drive
him to fury, and he will give himself up. Let me tell you, he is already
being watched; they are already on his track. You will simply be giving
him away. Wait a little: I saw him and was talking to him just now. He
can still be saved. Wait a bit, sit down; let us think it over together.
I asked you to come in order to discuss it alone with you and to
consider it thoroughly. But do sit down!''

``How can you save him? Can he really be saved?''

Dounia sat down. Svidrigaïlov sat down beside her.

``It all depends on you, on you, on you alone,'' he began with glowing
eyes, almost in a whisper and hardly able to utter the words for
emotion.

Dounia drew back from him in alarm. He too was trembling all over.

``You\ldots{} one word from you, and he is saved. I\ldots{} I'll save
him. I have money and friends. I'll send him away at once. I'll get a
passport, two passports, one for him and one for me. I have
friends\ldots{} capable people\ldots{}. If you like, I'll take a
passport for you\ldots{} for your mother\ldots{}. What do you want with
Razumihin? I love you too\ldots{}. I love you beyond everything\ldots{}.
Let me kiss the hem of your dress, let me, let me\ldots{}. The very
rustle of it is too much for me. Tell me, `do that,' and I'll do it.
I'll do everything. I will do the impossible. What you believe, I will
believe. I'll do anything---anything! Don't, don't look at me like that.
Do you know that you are killing me?\ldots{}''

He was almost beginning to rave\ldots{}. Something seemed suddenly to go
to his head. Dounia jumped up and rushed to the door.

``Open it! Open it!'' she called, shaking the door. ``Open it! Is there
no one there?''

Svidrigaïlov got up and came to himself. His still trembling lips slowly
broke into an angry mocking smile.

``There is no one at home,'' he said quietly and emphatically. ``The
landlady has gone out, and it's waste of time to shout like that. You
are only exciting yourself uselessly.''

``Where is the key? Open the door at once, at once, base man!''

``I have lost the key and cannot find it.''

``This is an outrage,'' cried Dounia, turning pale as death. She rushed
to the furthest corner, where she made haste to barricade herself with a
little table.

She did not scream, but she fixed her eyes on her tormentor and watched
every movement he made.

Svidrigaïlov remained standing at the other end of the room facing her.
He was positively composed, at least in appearance, but his face was
pale as before. The mocking smile did not leave his face.

``You spoke of outrage just now, Avdotya Romanovna. In that case you may
be sure I've taken measures. Sofya Semyonovna is not at home. The
Kapernaumovs are far away---there are five locked rooms between. I am at
least twice as strong as you are and I have nothing to fear, besides.
For you could not complain afterwards. You surely would not be willing
actually to betray your brother? Besides, no one would believe you. How
should a girl have come alone to visit a solitary man in his lodgings?
So that even if you do sacrifice your brother, you could prove nothing.
It is very difficult to prove an assault, Avdotya Romanovna.''

``Scoundrel!'' whispered Dounia indignantly.

``As you like, but observe I was only speaking by way of a general
proposition. It's my personal conviction that you are perfectly
right---violence is hateful. I only spoke to show you that you need have
no remorse even if\ldots{} you were willing to save your brother of your
own accord, as I suggest to you. You would be simply submitting to
circumstances, to violence, in fact, if we must use that word. Think
about it. Your brother's and your mother's fate are in your hands. I
will be your slave\ldots{} all my life\ldots{} I will wait here.''

Svidrigaïlov sat down on the sofa about eight steps from Dounia. She had
not the slightest doubt now of his unbending determination. Besides, she
knew him. Suddenly she pulled out of her pocket a revolver, cocked it
and laid it in her hand on the table. Svidrigaïlov jumped up.

``Aha! So that's it, is it?'' he cried, surprised but smiling
maliciously. ``Well, that completely alters the aspect of affairs.
You've made things wonderfully easier for me, Avdotya Romanovna. But
where did you get the revolver? Was it Mr. Razumihin? Why, it's my
revolver, an old friend! And how I've hunted for it! The shooting
lessons I've given you in the country have not been thrown away.''

``It's not your revolver, it belonged to Marfa Petrovna, whom you
killed, wretch! There was nothing of yours in her house. I took it when
I began to suspect what you were capable of. If you dare to advance one
step, I swear I'll kill you.'' She was frantic.

``But your brother? I ask from curiosity,'' said Svidrigaïlov, still
standing where he was.

``Inform, if you want to! Don't stir! Don't come nearer! I'll shoot! You
poisoned your wife, I know; you are a murderer yourself!'' She held the
revolver ready.

``Are you so positive I poisoned Marfa Petrovna?''

``You did! You hinted it yourself; you talked to me of poison\ldots{}. I
know you went to get it\ldots{} you had it in readiness\ldots{}. It was
your doing\ldots{}. It must have been your doing\ldots{}. Scoundrel!''

``Even if that were true, it would have been for your sake\ldots{} you
would have been the cause.''

``You are lying! I hated you always, always\ldots{}.''

``Oho, Avdotya Romanovna! You seem to have forgotten how you softened to
me in the heat of propaganda. I saw it in your eyes. Do you remember
that moonlight night, when the nightingale was singing?''

``That's a lie,'' there was a flash of fury in Dounia's eyes, ``that's a
lie and a libel!''

``A lie? Well, if you like, it's a lie. I made it up. Women ought not to
be reminded of such things,'' he smiled. ``I know you will shoot, you
pretty wild creature. Well, shoot away!''

Dounia raised the revolver, and deadly pale, gazed at him, measuring the
distance and awaiting the first movement on his part. Her lower lip was
white and quivering and her big black eyes flashed like fire. He had
never seen her so handsome. The fire glowing in her eyes at the moment
she raised the revolver seemed to kindle him and there was a pang of
anguish in his heart. He took a step forward and a shot rang out. The
bullet grazed his hair and flew into the wall behind. He stood still and
laughed softly.

``The wasp has stung me. She aimed straight at my head. What's this?
Blood?'' he pulled out his handkerchief to wipe the blood, which flowed
in a thin stream down his right temple. The bullet seemed to have just
grazed the skin.

Dounia lowered the revolver and looked at Svidrigaïlov not so much in
terror as in a sort of wild amazement. She seemed not to understand what
she was doing and what was going on.

``Well, you missed! Fire again, I'll wait,'' said Svidrigaïlov softly,
still smiling, but gloomily. ``If you go on like that, I shall have time
to seize you before you cock again.''

Dounia started, quickly cocked the pistol and again raised it.

``Let me be,'' she cried in despair. ``I swear I'll shoot again.
I\ldots{} I'll kill you.''

``Well\ldots{} at three paces you can hardly help it. But if you
don't\ldots{} then.'' His eyes flashed and he took two steps forward.
Dounia shot again: it missed fire.

``You haven't loaded it properly. Never mind, you have another charge
there. Get it ready, I'll wait.''

He stood facing her, two paces away, waiting and gazing at her with wild
determination, with feverishly passionate, stubborn, set eyes. Dounia
saw that he would sooner die than let her go. ``And\ldots{} now, of
course she would kill him, at two paces!'' Suddenly she flung away the
revolver.

``She's dropped it!'' said Svidrigaïlov with surprise, and he drew a
deep breath. A weight seemed to have rolled from his heart---perhaps not
only the fear of death; indeed he may scarcely have felt it at that
moment. It was the deliverance from another feeling, darker and more
bitter, which he could not himself have defined.

He went to Dounia and gently put his arm round her waist. She did not
resist, but, trembling like a leaf, looked at him with suppliant eyes.
He tried to say something, but his lips moved without being able to
utter a sound.

``Let me go,'' Dounia implored. Svidrigaïlov shuddered. Her voice now
was quite different.

``Then you don't love me?'' he asked softly. Dounia shook her head.

``And\ldots{} and you can't? Never?'' he whispered in despair.

``Never!''

There followed a moment of terrible, dumb struggle in the heart of
Svidrigaïlov. He looked at her with an indescribable gaze. Suddenly he
withdrew his arm, turned quickly to the window and stood facing it.
Another moment passed.

``Here's the key.''

He took it out of the left pocket of his coat and laid it on the table
behind him, without turning or looking at Dounia.

``Take it! Make haste!''

He looked stubbornly out of the window. Dounia went up to the table to
take the key.

``Make haste! Make haste!'' repeated Svidrigaïlov, still without turning
or moving. But there seemed a terrible significance in the tone of that
``make haste.''

Dounia understood it, snatched up the key, flew to the door, unlocked it
quickly and rushed out of the room. A minute later, beside herself, she
ran out on to the canal bank in the direction of X. Bridge.

Svidrigaïlov remained three minutes standing at the window. At last he
slowly turned, looked about him and passed his hand over his forehead. A
strange smile contorted his face, a pitiful, sad, weak smile, a smile of
despair. The blood, which was already getting dry, smeared his hand. He
looked angrily at it, then wetted a towel and washed his temple. The
revolver which Dounia had flung away lay near the door and suddenly
caught his eye. He picked it up and examined it. It was a little pocket
three-barrel revolver of old-fashioned construction. There were still
two charges and one capsule left in it. It could be fired again. He
thought a little, put the revolver in his pocket, took his hat and went
out.

\protect\hypertarget{link2HCH0037}{}{}

\mychapter{6}{Chapter VI}\label{chapter-vi-4}

He spent that evening till ten o'clock going from one low haunt to
another. Katia too turned up and sang another gutter song, how a certain

``villain and tyrant,''

``began kissing Katia.''

Svidrigaïlov treated Katia and the organ-grinder and some singers and
the waiters and two little clerks. He was particularly drawn to these
clerks by the fact that they both had crooked noses, one bent to the
left and the other to the right. They took him finally to a pleasure
garden, where he paid for their entrance. There was one lanky
three-year-old pine-tree and three bushes in the garden, besides a
``Vauxhall,'' which was in reality a drinking-bar where tea too was
served, and there were a few green tables and chairs standing round it.
A chorus of wretched singers and a drunken but exceedingly depressed
German clown from Munich with a red nose entertained the public. The
clerks quarrelled with some other clerks and a fight seemed imminent.
Svidrigaïlov was chosen to decide the dispute. He listened to them for a
quarter of an hour, but they shouted so loud that there was no
possibility of understanding them. The only fact that seemed certain was
that one of them had stolen something and had even succeeded in selling
it on the spot to a Jew, but would not share the spoil with his
companion. Finally it appeared that the stolen object was a teaspoon
belonging to the Vauxhall. It was missed and the affair began to seem
troublesome. Svidrigaïlov paid for the spoon, got up, and walked out of
the garden. It was about six o'clock. He had not drunk a drop of wine
all this time and had ordered tea more for the sake of appearances than
anything.

It was a dark and stifling evening. Threatening storm-clouds came over
the sky about ten o'clock. There was a clap of thunder, and the rain
came down like a waterfall. The water fell not in drops, but beat on the
earth in streams. There were flashes of lightning every minute and each
flash lasted while one could count five.

Drenched to the skin, he went home, locked himself in, opened the
bureau, took out all his money and tore up two or three papers. Then,
putting the money in his pocket, he was about to change his clothes,
but, looking out of the window and listening to the thunder and the
rain, he gave up the idea, took up his hat and went out of the room
without locking the door. He went straight to Sonia. She was at home.

She was not alone: the four Kapernaumov children were with her. She was
giving them tea. She received Svidrigaïlov in respectful silence,
looking wonderingly at his soaking clothes. The children all ran away at
once in indescribable terror.

Svidrigaïlov sat down at the table and asked Sonia to sit beside him.
She timidly prepared to listen.

``I may be going to America, Sofya Semyonovna,'' said Svidrigaïlov,
``and as I am probably seeing you for the last time, I have come to make
some arrangements. Well, did you see the lady to-day? I know what she
said to you, you need not tell me.'' (Sonia made a movement and
blushed.) ``Those people have their own way of doing things. As to your
sisters and your brother, they are really provided for and the money
assigned to them I've put into safe keeping and have received
acknowledgments. You had better take charge of the receipts, in case
anything happens. Here, take them! Well now, that's settled. Here are
three 5-per-cent bonds to the value of three thousand roubles. Take
those for yourself, entirely for yourself, and let that be strictly
between ourselves, so that no one knows of it, whatever you hear. You
will need the money, for to go on living in the old way, Sofya
Semyonovna, is bad, and besides there is no need for it now.''

``I am so much indebted to you, and so are the children and my
stepmother,'' said Sonia hurriedly, ``and if I've said so little\ldots{}
please don't consider\ldots{}''

``That's enough! that's enough!''

``But as for the money, Arkady Ivanovitch, I am very grateful to you,
but I don't need it now. I can always earn my own living. Don't think me
ungrateful. If you are so charitable, that money\ldots{}.''

``It's for you, for you, Sofya Semyonovna, and please don't waste words
over it. I haven't time for it. You will want it. Rodion Romanovitch has
two alternatives: a bullet in the brain or Siberia.'' (Sonia looked
wildly at him, and started.) ``Don't be uneasy, I know all about it from
himself and I am not a gossip; I won't tell anyone. It was good advice
when you told him to give himself up and confess. It would be much
better for him. Well, if it turns out to be Siberia, he will go and you
will follow him. That's so, isn't it? And if so, you'll need money.
You'll need it for him, do you understand? Giving it to you is the same
as my giving it to him. Besides, you promised Amalia Ivanovna to pay
what's owing. I heard you. How can you undertake such obligations so
heedlessly, Sofya Semyonovna? It was Katerina Ivanovna's debt and not
yours, so you ought not to have taken any notice of the German woman.
You can't get through the world like that. If you are ever questioned
about me---to-morrow or the day after you will be asked---don't say
anything about my coming to see you now and don't show the money to
anyone or say a word about it. Well, now good-bye.'' (He got up.) ``My
greetings to Rodion Romanovitch. By the way, you'd better put the money
for the present in Mr. Razumihin's keeping. You know Mr. Razumihin? Of
course you do. He's not a bad fellow. Take it to him to-morrow
or\ldots{} when the time comes. And till then, hide it carefully.''

Sonia too jumped up from her chair and looked in dismay at Svidrigaïlov.
She longed to speak, to ask a question, but for the first moments she
did not dare and did not know how to begin.

``How can you\ldots{} how can you be going now, in such rain?''

``Why, be starting for America, and be stopped by rain! Ha, ha!
Good-bye, Sofya Semyonovna, my dear! Live and live long, you will be of
use to others. By the way\ldots{} tell Mr. Razumihin I send my greetings
to him. Tell him Arkady Ivanovitch Svidrigaïlov sends his greetings. Be
sure to.''

He went out, leaving Sonia in a state of wondering anxiety and vague
apprehension.

It appeared afterwards that on the same evening, at twenty past eleven,
he made another very eccentric and unexpected visit. The rain still
persisted. Drenched to the skin, he walked into the little flat where
the parents of his betrothed lived, in Third Street in Vassilyevsky
Island. He knocked some time before he was admitted, and his visit at
first caused great perturbation; but Svidrigaïlov could be very
fascinating when he liked, so that the first, and indeed very
intelligent surmise of the sensible parents that Svidrigaïlov had
probably had so much to drink that he did not know what he was doing
vanished immediately. The decrepit father was wheeled in to see
Svidrigaïlov by the tender and sensible mother, who as usual began the
conversation with various irrelevant questions. She never asked a direct
question, but began by smiling and rubbing her hands and then, if she
were obliged to ascertain something---for instance, when Svidrigaïlov
would like to have the wedding---she would begin by interested and
almost eager questions about Paris and the court life there, and only by
degrees brought the conversation round to Third Street. On other
occasions this had of course been very impressive, but this time Arkady
Ivanovitch seemed particularly impatient, and insisted on seeing his
betrothed at once, though he had been informed, to begin with, that she
had already gone to bed. The girl of course appeared.

Svidrigaïlov informed her at once that he was obliged by very important
affairs to leave Petersburg for a time, and therefore brought her
fifteen thousand roubles and begged her accept them as a present from
him, as he had long been intending to make her this trifling present
before their wedding. The logical connection of the present with his
immediate departure and the absolute necessity of visiting them for that
purpose in pouring rain at midnight was not made clear. But it all went
off very well; even the inevitable ejaculations of wonder and regret,
the inevitable questions were extraordinarily few and restrained. On the
other hand, the gratitude expressed was most glowing and was reinforced
by tears from the most sensible of mothers. Svidrigaïlov got up,
laughed, kissed his betrothed, patted her cheek, declared he would soon
come back, and noticing in her eyes, together with childish curiosity, a
sort of earnest dumb inquiry, reflected and kissed her again, though he
felt sincere anger inwardly at the thought that his present would be
immediately locked up in the keeping of the most sensible of mothers. He
went away, leaving them all in a state of extraordinary excitement, but
the tender mamma, speaking quietly in a half whisper, settled some of
the most important of their doubts, concluding that Svidrigaïlov was a
great man, a man of great affairs and connections and of great
wealth---there was no knowing what he had in his mind. He would start
off on a journey and give away money just as the fancy took him, so that
there was nothing surprising about it. Of course it was strange that he
was wet through, but Englishmen, for instance, are even more eccentric,
and all these people of high society didn't think of what was said of
them and didn't stand on ceremony. Possibly, indeed, he came like that
on purpose to show that he was not afraid of anyone. Above all, not a
word should be said about it, for God knows what might come of it, and
the money must be locked up, and it was most fortunate that Fedosya, the
cook, had not left the kitchen. And above all not a word must be said to
that old cat, Madame Resslich, and so on and so on. They sat up
whispering till two o'clock, but the girl went to bed much earlier,
amazed and rather sorrowful.

Svidrigaïlov meanwhile, exactly at midnight, crossed the bridge on the
way back to the mainland. The rain had ceased and there was a roaring
wind. He began shivering, and for one moment he gazed at the black
waters of the Little Neva with a look of special interest, even inquiry.
But he soon felt it very cold, standing by the water; he turned and went
towards Y. Prospect. He walked along that endless street for a long
time, almost half an hour, more than once stumbling in the dark on the
wooden pavement, but continually looking for something on the right side
of the street. He had noticed passing through this street lately that
there was a hotel somewhere towards the end, built of wood, but fairly
large, and its name he remembered was something like Adrianople. He was
not mistaken: the hotel was so conspicuous in that God-forsaken place
that he could not fail to see it even in the dark. It was a long,
blackened wooden building, and in spite of the late hour there were
lights in the windows and signs of life within. He went in and asked a
ragged fellow who met him in the corridor for a room. The latter,
scanning Svidrigaïlov, pulled himself together and led him at once to a
close and tiny room in the distance, at the end of the corridor, under
the stairs. There was no other, all were occupied. The ragged fellow
looked inquiringly.

``Is there tea?'' asked Svidrigaïlov.

``Yes, sir.''

``What else is there?''

``Veal, vodka, savouries.''

``Bring me tea and veal.''

``And you want nothing else?'' he asked with apparent surprise.

``Nothing, nothing.''

The ragged man went away, completely disillusioned.

``It must be a nice place,'' thought Svidrigaïlov. ``How was it I didn't
know it? I expect I look as if I came from a café chantant and have had
some adventure on the way. It would be interesting to know who stayed
here?''

He lighted the candle and looked at the room more carefully. It was a
room so low-pitched that Svidrigaïlov could only just stand up in it; it
had one window; the bed, which was very dirty, and the plain-stained
chair and table almost filled it up. The walls looked as though they
were made of planks, covered with shabby paper, so torn and dusty that
the pattern was indistinguishable, though the general
colour---yellow---could still be made out. One of the walls was cut
short by the sloping ceiling, though the room was not an attic but just
under the stairs.

Svidrigaïlov set down the candle, sat down on the bed and sank into
thought. But a strange persistent murmur which sometimes rose to a shout
in the next room attracted his attention. The murmur had not ceased from
the moment he entered the room. He listened: someone was upbraiding and
almost tearfully scolding, but he heard only one voice.

Svidrigaïlov got up, shaded the light with his hand and at once he saw
light through a crack in the wall; he went up and peeped through. The
room, which was somewhat larger than his, had two occupants. One of
them, a very curly-headed man with a red inflamed face, was standing in
the pose of an orator, without his coat, with his legs wide apart to
preserve his balance, and smiting himself on the breast. He reproached
the other with being a beggar, with having no standing whatever. He
declared that he had taken the other out of the gutter and he could turn
him out when he liked, and that only the finger of Providence sees it
all. The object of his reproaches was sitting in a chair, and had the
air of a man who wants dreadfully to sneeze, but can't. He sometimes
turned sheepish and befogged eyes on the speaker, but obviously had not
the slightest idea what he was talking about and scarcely heard it. A
candle was burning down on the table; there were wine-glasses, a nearly
empty bottle of vodka, bread and cucumber, and glasses with the dregs of
stale tea. After gazing attentively at this, Svidrigaïlov turned away
indifferently and sat down on the bed.

The ragged attendant, returning with the tea, could not resist asking
him again whether he didn't want anything more, and again receiving a
negative reply, finally withdrew. Svidrigaïlov made haste to drink a
glass of tea to warm himself, but could not eat anything. He began to
feel feverish. He took off his coat and, wrapping himself in the
blanket, lay down on the bed. He was annoyed. ``It would have been
better to be well for the occasion,'' he thought with a smile. The room
was close, the candle burnt dimly, the wind was roaring outside, he
heard a mouse scratching in the corner and the room smelt of mice and of
leather. He lay in a sort of reverie: one thought followed another. He
felt a longing to fix his imagination on something. ``It must be a
garden under the window,'' he thought. ``There's a sound of trees. How I
dislike the sound of trees on a stormy night, in the dark! They give one
a horrid feeling.'' He remembered how he had disliked it when he passed
Petrovsky Park just now. This reminded him of the bridge over the Little
Neva and he felt cold again as he had when standing there. ``I never
have liked water,'' he thought, ``even in a landscape,'' and he suddenly
smiled again at a strange idea: ``Surely now all these questions of
taste and comfort ought not to matter, but I've become more particular,
like an animal that picks out a special place\ldots{} for such an
occasion. I ought to have gone into the Petrovsky Park! I suppose it
seemed dark, cold, ha-ha! As though I were seeking pleasant
sensations!\ldots{} By the way, why haven't I put out the candle?'' he
blew it out. ``They've gone to bed next door,'' he thought, not seeing
the light at the crack. ``Well, now, Marfa Petrovna, now is the time for
you to turn up; it's dark, and the very time and place for you. But now
you won't come!''

He suddenly recalled how, an hour before carrying out his design on
Dounia, he had recommended Raskolnikov to trust her to Razumihin's
keeping. ``I suppose I really did say it, as Raskolnikov guessed, to
tease myself. But what a rogue that Raskolnikov is! He's gone through a
good deal. He may be a successful rogue in time when he's got over his
nonsense. But now he's \emph{too} eager for life. These young men are
contemptible on that point. But, hang the fellow! Let him please
himself, it's nothing to do with me.''

He could not get to sleep. By degrees Dounia's image rose before him,
and a shudder ran over him. ``No, I must give up all that now,'' he
thought, rousing himself. ``I must think of something else. It's queer
and funny. I never had a great hatred for anyone, I never particularly
desired to avenge myself even, and that's a bad sign, a bad sign, a bad
sign. I never liked quarrelling either, and never lost my
temper---that's a bad sign too. And the promises I made her just now,
too---Damnation! But---who knows?---perhaps she would have made a new
man of me somehow\ldots{}.''

He ground his teeth and sank into silence again. Again Dounia's image
rose before him, just as she was when, after shooting the first time,
she had lowered the revolver in terror and gazed blankly at him, so that
he might have seized her twice over and she would not have lifted a hand
to defend herself if he had not reminded her. He recalled how at that
instant he felt almost sorry for her, how he had felt a pang at his
heart\ldots{}

``Aïe! Damnation, these thoughts again! I must put it away!''

He was dozing off; the feverish shiver had ceased, when suddenly
something seemed to run over his arm and leg under the bedclothes. He
started. ``Ugh! hang it! I believe it's a mouse,'' he thought, ``that's
the veal I left on the table.'' He felt fearfully disinclined to pull
off the blanket, get up, get cold, but all at once something unpleasant
ran over his leg again. He pulled off the blanket and lighted the
candle. Shaking with feverish chill he bent down to examine the bed:
there was nothing. He shook the blanket and suddenly a mouse jumped out
on the sheet. He tried to catch it, but the mouse ran to and fro in
zigzags without leaving the bed, slipped between his fingers, ran over
his hand and suddenly darted under the pillow. He threw down the pillow,
but in one instant felt something leap on his chest and dart over his
body and down his back under his shirt. He trembled nervously and woke
up.

The room was dark. He was lying on the bed and wrapped up in the blanket
as before. The wind was howling under the window. ``How disgusting,'' he
thought with annoyance.

He got up and sat on the edge of the bedstead with his back to the
window. ``It's better not to sleep at all,'' he decided. There was a
cold damp draught from the window, however; without getting up he drew
the blanket over him and wrapped himself in it. He was not thinking of
anything and did not want to think. But one image rose after another,
incoherent scraps of thought without beginning or end passed through his
mind. He sank into drowsiness. Perhaps the cold, or the dampness, or the
dark, or the wind that howled under the window and tossed the trees
roused a sort of persistent craving for the fantastic. He kept dwelling
on images of flowers, he fancied a charming flower garden, a bright,
warm, almost hot day, a holiday---Trinity day. A fine, sumptuous country
cottage in the English taste overgrown with fragrant flowers, with
flower beds going round the house; the porch, wreathed in climbers, was
surrounded with beds of roses. A light, cool staircase, carpeted with
rich rugs, was decorated with rare plants in china pots. He noticed
particularly in the windows nosegays of tender, white, heavily fragrant
narcissus bending over their bright, green, thick long stalks. He was
reluctant to move away from them, but he went up the stairs and came
into a large, high drawing-room and again everywhere---at the windows,
the doors on to the balcony, and on the balcony itself---were flowers.
The floors were strewn with freshly-cut fragrant hay, the windows were
open, a fresh, cool, light air came into the room. The birds were
chirruping under the window, and in the middle of the room, on a table
covered with a white satin shroud, stood a coffin. The coffin was
covered with white silk and edged with a thick white frill; wreaths of
flowers surrounded it on all sides. Among the flowers lay a girl in a
white muslin dress, with her arms crossed and pressed on her bosom, as
though carved out of marble. But her loose fair hair was wet; there was
a wreath of roses on her head. The stern and already rigid profile of
her face looked as though chiselled of marble too, and the smile on her
pale lips was full of an immense unchildish misery and sorrowful appeal.
Svidrigaïlov knew that girl; there was no holy image, no burning candle
beside the coffin; no sound of prayers: the girl had drowned herself.
She was only fourteen, but her heart was broken. And she had destroyed
herself, crushed by an insult that had appalled and amazed that childish
soul, had smirched that angel purity with unmerited disgrace and torn
from her a last scream of despair, unheeded and brutally disregarded, on
a dark night in the cold and wet while the wind howled\ldots{}.

Svidrigaïlov came to himself, got up from the bed and went to the
window. He felt for the latch and opened it. The wind lashed furiously
into the little room and stung his face and his chest, only covered with
his shirt, as though with frost. Under the window there must have been
something like a garden, and apparently a pleasure garden. There, too,
probably there were tea-tables and singing in the daytime. Now drops of
rain flew in at the window from the trees and bushes; it was dark as in
a cellar, so that he could only just make out some dark blurs of
objects. Svidrigaïlov, bending down with elbows on the window-sill,
gazed for five minutes into the darkness; the boom of a cannon, followed
by a second one, resounded in the darkness of the night. ``Ah, the
signal! The river is overflowing,'' he thought. ``By morning it will be
swirling down the street in the lower parts, flooding the basements and
cellars. The cellar rats will swim out, and men will curse in the rain
and wind as they drag their rubbish to their upper storeys. What time is
it now?'' And he had hardly thought it when, somewhere near, a clock on
the wall, ticking away hurriedly, struck three.

``Aha! It will be light in an hour! Why wait? I'll go out at once
straight to the park. I'll choose a great bush there drenched with rain,
so that as soon as one's shoulder touches it, millions of drops drip on
one's head.''

He moved away from the window, shut it, lighted the candle, put on his
waistcoat, his overcoat and his hat and went out, carrying the candle,
into the passage to look for the ragged attendant who would be asleep
somewhere in the midst of candle-ends and all sorts of rubbish, to pay
him for the room and leave the hotel. ``It's the best minute; I couldn't
choose a better.''

He walked for some time through a long narrow corridor without finding
anyone and was just going to call out, when suddenly in a dark corner
between an old cupboard and the door he caught sight of a strange object
which seemed to be alive. He bent down with the candle and saw a little
girl, not more than five years old, shivering and crying, with her
clothes as wet as a soaking house-flannel. She did not seem afraid of
Svidrigaïlov, but looked at him with blank amazement out of her big
black eyes. Now and then she sobbed as children do when they have been
crying a long time, but are beginning to be comforted. The child's face
was pale and tired, she was numb with cold. ``How can she have come
here? She must have hidden here and not slept all night.'' He began
questioning her. The child suddenly becoming animated, chattered away in
her baby language, something about ``mammy'' and that ``mammy would beat
her,'' and about some cup that she had ``bwoken.'' The child chattered
on without stopping. He could only guess from what she said that she was
a neglected child, whose mother, probably a drunken cook, in the service
of the hotel, whipped and frightened her; that the child had broken a
cup of her mother's and was so frightened that she had run away the
evening before, had hidden for a long while somewhere outside in the
rain, at last had made her way in here, hidden behind the cupboard and
spent the night there, crying and trembling from the damp, the darkness
and the fear that she would be badly beaten for it. He took her in his
arms, went back to his room, sat her on the bed, and began undressing
her. The torn shoes which she had on her stockingless feet were as wet
as if they had been standing in a puddle all night. When he had
undressed her, he put her on the bed, covered her up and wrapped her in
the blanket from her head downwards. She fell asleep at once. Then he
sank into dreary musing again.

``What folly to trouble myself,'' he decided suddenly with an oppressive
feeling of annoyance. ``What idiocy!'' In vexation he took up the candle
to go and look for the ragged attendant again and make haste to go away.
``Damn the child!'' he thought as he opened the door, but he turned
again to see whether the child was asleep. He raised the blanket
carefully. The child was sleeping soundly, she had got warm under the
blanket, and her pale cheeks were flushed. But strange to say that flush
seemed brighter and coarser than the rosy cheeks of childhood. ``It's a
flush of fever,'' thought Svidrigaïlov. It was like the flush from
drinking, as though she had been given a full glass to drink. Her
crimson lips were hot and glowing; but what was this? He suddenly
fancied that her long black eyelashes were quivering, as though the lids
were opening and a sly crafty eye peeped out with an unchildlike wink,
as though the little girl were not asleep, but pretending. Yes, it was
so. Her lips parted in a smile. The corners of her mouth quivered, as
though she were trying to control them. But now she quite gave up all
effort, now it was a grin, a broad grin; there was something shameless,
provocative in that quite unchildish face; it was depravity, it was the
face of a harlot, the shameless face of a French harlot. Now both eyes
opened wide; they turned a glowing, shameless glance upon him; they
laughed, invited him\ldots{}. There was something infinitely hideous and
shocking in that laugh, in those eyes, in such nastiness in the face of
a child. ``What, at five years old?'' Svidrigaïlov muttered in genuine
horror. ``What does it mean?'' And now she turned to him, her little
face all aglow, holding out her arms\ldots{}. ``Accursed child!''
Svidrigaïlov cried, raising his hand to strike her, but at that moment
he woke up.

He was in the same bed, still wrapped in the blanket. The candle had not
been lighted, and daylight was streaming in at the windows.

``I've had nightmare all night!'' He got up angrily, feeling utterly
shattered; his bones ached. There was a thick mist outside and he could
see nothing. It was nearly five. He had overslept himself! He got up,
put on his still damp jacket and overcoat. Feeling the revolver in his
pocket, he took it out and then he sat down, took a notebook out of his
pocket and in the most conspicuous place on the title page wrote a few
lines in large letters. Reading them over, he sank into thought with his
elbows on the table. The revolver and the notebook lay beside him. Some
flies woke up and settled on the untouched veal, which was still on the
table. He stared at them and at last with his free right hand began
trying to catch one. He tried till he was tired, but could not catch it.
At last, realising that he was engaged in this interesting pursuit, he
started, got up and walked resolutely out of the room. A minute later he
was in the street.

A thick milky mist hung over the town. Svidrigaïlov walked along the
slippery dirty wooden pavement towards the Little Neva. He was picturing
the waters of the Little Neva swollen in the night, Petrovsky Island,
the wet paths, the wet grass, the wet trees and bushes and at last the
bush\ldots{}. He began ill-humouredly staring at the houses, trying to
think of something else. There was not a cabman or a passer-by in the
street. The bright yellow, wooden, little houses looked dirty and
dejected with their closed shutters. The cold and damp penetrated his
whole body and he began to shiver. From time to time he came across shop
signs and read each carefully. At last he reached the end of the wooden
pavement and came to a big stone house. A dirty, shivering dog crossed
his path with its tail between its legs. A man in a greatcoat lay face
downwards; dead drunk, across the pavement. He looked at him and went
on. A high tower stood up on the left. ``Bah!'' he shouted, ``here is a
place. Why should it be Petrovsky? It will be in the presence of an
official witness anyway\ldots{}.''

He almost smiled at this new thought and turned into the street where
there was the big house with the tower. At the great closed gates of the
house, a little man stood with his shoulder leaning against them,
wrapped in a grey soldier's coat, with a copper Achilles helmet on his
head. He cast a drowsy and indifferent glance at Svidrigaïlov. His face
wore that perpetual look of peevish dejection, which is so sourly
printed on all faces of Jewish race without exception. They both,
Svidrigaïlov and Achilles, stared at each other for a few minutes
without speaking. At last it struck Achilles as irregular for a man not
drunk to be standing three steps from him, staring and not saying a
word.

``What do you want here?'' he said, without moving or changing his
position.

``Nothing, brother, good morning,'' answered Svidrigaïlov.

``This isn't the place.''

``I am going to foreign parts, brother.''

``To foreign parts?''

``To America.''

``America.''

Svidrigaïlov took out the revolver and cocked it. Achilles raised his
eyebrows.

``I say, this is not the place for such jokes!''

``Why shouldn't it be the place?''

``Because it isn't.''

``Well, brother, I don't mind that. It's a good place. When you are
asked, you just say he was going, he said, to America.''

He put the revolver to his right temple.

``You can't do it here, it's not the place,'' cried Achilles, rousing
himself, his eyes growing bigger and bigger.

Svidrigaïlov pulled the trigger.

\protect\hypertarget{link2HCH0038}{}{}

\mychapter{7}{Chapter VII}\label{chapter-vii-2}

The same day, about seven o'clock in the evening, Raskolnikov was on his
way to his mother's and sister's lodging---the lodging in Bakaleyev's
house which Razumihin had found for them. The stairs went up from the
street. Raskolnikov walked with lagging steps, as though still
hesitating whether to go or not. But nothing would have turned him back:
his decision was taken.

``Besides, it doesn't matter, they still know nothing,'' he thought,
``and they are used to thinking of me as eccentric.''

He was appallingly dressed: his clothes torn and dirty, soaked with a
night's rain. His face was almost distorted from fatigue, exposure, the
inward conflict that had lasted for twenty-four hours. He had spent all
the previous night alone, God knows where. But anyway he had reached a
decision.

He knocked at the door which was opened by his mother. Dounia was not at
home. Even the servant happened to be out. At first Pulcheria
Alexandrovna was speechless with joy and surprise; then she took him by
the hand and drew him into the room.

``Here you are!'' she began, faltering with joy. ``Don't be angry with
me, Rodya, for welcoming you so foolishly with tears: I am laughing not
crying. Did you think I was crying? No, I am delighted, but I've got
into such a stupid habit of shedding tears. I've been like that ever
since your father's death. I cry for anything. Sit down, dear boy, you
must be tired; I see you are. Ah, how muddy you are.''

``I was in the rain yesterday, mother\ldots{}.'' Raskolnikov began.

``No, no,'' Pulcheria Alexandrovna hurriedly interrupted, ``you thought
I was going to cross-question you in the womanish way I used to; don't
be anxious, I understand, I understand it all: now I've learned the ways
here and truly I see for myself that they are better. I've made up my
mind once for all: how could I understand your plans and expect you to
give an account of them? God knows what concerns and plans you may have,
or what ideas you are hatching; so it's not for me to keep nudging your
elbow, asking you what you are thinking about? But, my goodness! why am
I running to and fro as though I were crazy\ldots{}? I am reading your
article in the magazine for the third time, Rodya. Dmitri Prokofitch
brought it to me. Directly I saw it I cried out to myself: `There,
foolish one,' I thought, `that's what he is busy about; that's the
solution of the mystery! Learned people are always like that. He may
have some new ideas in his head just now; he is thinking them over and I
worry him and upset him.' I read it, my dear, and of course there was a
great deal I did not understand; but that's only natural---how should
I?''

``Show me, mother.''

Raskolnikov took the magazine and glanced at his article. Incongruous as
it was with his mood and his circumstances, he felt that strange and
bitter sweet sensation that every author experiences the first time he
sees himself in print; besides, he was only twenty-three. It lasted only
a moment. After reading a few lines he frowned and his heart throbbed
with anguish. He recalled all the inward conflict of the preceding
months. He flung the article on the table with disgust and anger.

``But, however foolish I may be, Rodya, I can see for myself that you
will very soon be one of the leading---if not the leading man---in the
world of Russian thought. And they dared to think you were mad! You
don't know, but they really thought that. Ah, the despicable creatures,
how could they understand genius! And Dounia, Dounia was all but
believing it---what do you say to that? Your father sent twice to
magazines---the first time poems (I've got the manuscript and will show
you) and the second time a whole novel (I begged him to let me copy it
out) and how we prayed that they should be taken---they weren't! I was
breaking my heart, Rodya, six or seven days ago over your food and your
clothes and the way you are living. But now I see again how foolish I
was, for you can attain any position you like by your intellect and
talent. No doubt you don't care about that for the present and you are
occupied with much more important matters\ldots{}.''

``Dounia's not at home, mother?''

``No, Rodya. I often don't see her; she leaves me alone. Dmitri
Prokofitch comes to see me, it's so good of him, and he always talks
about you. He loves you and respects you, my dear. I don't say that
Dounia is very wanting in consideration. I am not complaining. She has
her ways and I have mine; she seems to have got some secrets of late and
I never have any secrets from you two. Of course, I am sure that Dounia
has far too much sense, and besides she loves you and me\ldots{} but I
don't know what it will all lead to. You've made me so happy by coming
now, Rodya, but she has missed you by going out; when she comes in I'll
tell her: `Your brother came in while you were out. Where have you been
all this time?' You mustn't spoil me, Rodya, you know; come when you
can, but if you can't, it doesn't matter, I can wait. I shall know,
anyway, that you are fond of me, that will be enough for me. I shall
read what you write, I shall hear about you from everyone, and sometimes
you'll come yourself to see me. What could be better? Here you've come
now to comfort your mother, I see that.''

Here Pulcheria Alexandrovna began to cry.

``Here I am again! Don't mind my foolishness. My goodness, why am I
sitting here?'' she cried, jumping up. ``There is coffee and I don't
offer you any. Ah, that's the selfishness of old age. I'll get it at
once!''

``Mother, don't trouble, I am going at once. I haven't come for that.
Please listen to me.''

Pulcheria Alexandrovna went up to him timidly.

``Mother, whatever happens, whatever you hear about me, whatever you are
told about me, will you always love me as you do now?'' he asked
suddenly from the fullness of his heart, as though not thinking of his
words and not weighing them.

``Rodya, Rodya, what is the matter? How can you ask me such a question?
Why, who will tell me anything about you? Besides, I shouldn't believe
anyone, I should refuse to listen.''

``I've come to assure you that I've always loved you and I am glad that
we are alone, even glad Dounia is out,'' he went on with the same
impulse. ``I have come to tell you that though you will be unhappy, you
must believe that your son loves you now more than himself, and that all
you thought about me, that I was cruel and didn't care about you, was
all a mistake. I shall never cease to love you\ldots{}. Well, that's
enough: I thought I must do this and begin with this\ldots{}.''

Pulcheria Alexandrovna embraced him in silence, pressing him to her
bosom and weeping gently.

``I don't know what is wrong with you, Rodya,'' she said at last. ``I've
been thinking all this time that we were simply boring you and now I see
that there is a great sorrow in store for you, and that's why you are
miserable. I've foreseen it a long time, Rodya. Forgive me for speaking
about it. I keep thinking about it and lie awake at nights. Your sister
lay talking in her sleep all last night, talking of nothing but you. I
caught something, but I couldn't make it out. I felt all the morning as
though I were going to be hanged, waiting for something, expecting
something, and now it has come! Rodya, Rodya, where are you going? You
are going away somewhere?''

``Yes.''

``That's what I thought! I can come with you, you know, if you need me.
And Dounia, too; she loves you, she loves you dearly---and Sofya
Semyonovna may come with us if you like. You see, I am glad to look upon
her as a daughter even\ldots{} Dmitri Prokofitch will help us to go
together. But\ldots{} where\ldots{} are you going?''

``Good-bye, mother.''

``What, to-day?'' she cried, as though losing him for ever.

``I can't stay, I must go now\ldots{}.''

``And can't I come with you?''

``No, but kneel down and pray to God for me. Your prayer perhaps will
reach Him.''

``Let me bless you and sign you with the cross. That's right, that's
right. Oh, God, what are we doing?''

Yes, he was glad, he was very glad that there was no one there, that he
was alone with his mother. For the first time after all those awful
months his heart was softened. He fell down before her, he kissed her
feet and both wept, embracing. And she was not surprised and did not
question him this time. For some days she had realised that something
awful was happening to her son and that now some terrible minute had
come for him.

``Rodya, my darling, my first born,'' she said sobbing, ``now you are
just as when you were little. You would run like this to me and hug me
and kiss me. When your father was living and we were poor, you comforted
us simply by being with us and when I buried your father, how often we
wept together at his grave and embraced, as now. And if I've been crying
lately, it's that my mother's heart had a foreboding of trouble. The
first time I saw you, that evening, you remember, as soon as we arrived
here, I guessed simply from your eyes. My heart sank at once, and to-day
when I opened the door and looked at you, I thought the fatal hour had
come. Rodya, Rodya, you are not going away to-day?''

``No!''

``You'll come again?''

``Yes\ldots{} I'll come.''

``Rodya, don't be angry, I don't dare to question you. I know I mustn't.
Only say two words to me---is it far where you are going?''

``Very far.''

``What is awaiting you there? Some post or career for you?''

``What God sends\ldots{} only pray for me.'' Raskolnikov went to the
door, but she clutched him and gazed despairingly into his eyes. Her
face worked with terror.

``Enough, mother,'' said Raskolnikov, deeply regretting that he had
come.

``Not for ever, it's not yet for ever? You'll come, you'll come
to-morrow?''

``I will, I will, good-bye.'' He tore himself away at last.

It was a warm, fresh, bright evening; it had cleared up in the morning.
Raskolnikov went to his lodgings; he made haste. He wanted to finish all
before sunset. He did not want to meet anyone till then. Going up the
stairs he noticed that Nastasya rushed from the samovar to watch him
intently. ``Can anyone have come to see me?'' he wondered. He had a
disgusted vision of Porfiry. But opening his door he saw Dounia. She was
sitting alone, plunged in deep thought, and looked as though she had
been waiting a long time. He stopped short in the doorway. She rose from
the sofa in dismay and stood up facing him. Her eyes, fixed upon him,
betrayed horror and infinite grief. And from those eyes alone he saw at
once that she knew.

``Am I to come in or go away?'' he asked uncertainly.

``I've been all day with Sofya Semyonovna. We were both waiting for you.
We thought that you would be sure to come there.''

Raskolnikov went into the room and sank exhausted on a chair.

``I feel weak, Dounia, I am very tired; and I should have liked at this
moment to be able to control myself.''

He glanced at her mistrustfully.

``Where were you all night?''

``I don't remember clearly. You see, sister, I wanted to make up my mind
once for all, and several times I walked by the Neva, I remember that I
wanted to end it all there, but\ldots{} I couldn't make up my mind,'' he
whispered, looking at her mistrustfully again.

``Thank God! That was just what we were afraid of, Sofya Semyonovna and
I. Then you still have faith in life? Thank God, thank God!''

Raskolnikov smiled bitterly.

``I haven't faith, but I have just been weeping in mother's arms; I
haven't faith, but I have just asked her to pray for me. I don't know
how it is, Dounia, I don't understand it.''

``Have you been at mother's? Have you told her?'' cried Dounia,
horror-stricken. ``Surely you haven't done that?''

``No, I didn't tell her\ldots{} in words; but she understood a great
deal. She heard you talking in your sleep. I am sure she half
understands it already. Perhaps I did wrong in going to see her. I don't
know why I did go. I am a contemptible person, Dounia.''

``A contemptible person, but ready to face suffering! You are, aren't
you?''

``Yes, I am going. At once. Yes, to escape the disgrace I thought of
drowning myself, Dounia, but as I looked into the water, I thought that
if I had considered myself strong till now I'd better not be afraid of
disgrace,'' he said, hurrying on. ``It's pride, Dounia.''

``Pride, Rodya.''

There was a gleam of fire in his lustreless eyes; he seemed to be glad
to think that he was still proud.

``You don't think, sister, that I was simply afraid of the water?'' he
asked, looking into her face with a sinister smile.

``Oh, Rodya, hush!'' cried Dounia bitterly. Silence lasted for two
minutes. He sat with his eyes fixed on the floor; Dounia stood at the
other end of the table and looked at him with anguish. Suddenly he got
up.

``It's late, it's time to go! I am going at once to give myself up. But
I don't know why I am going to give myself up.''

Big tears fell down her cheeks.

``You are crying, sister, but can you hold out your hand to me?''

``You doubted it?''

She threw her arms round him.

``Aren't you half expiating your crime by facing the suffering?'' she
cried, holding him close and kissing him.

``Crime? What crime?'' he cried in sudden fury. ``That I killed a vile
noxious insect, an old pawnbroker woman, of use to no one!\ldots{}
Killing her was atonement for forty sins. She was sucking the life out
of poor people. Was that a crime? I am not thinking of it and I am not
thinking of expiating it, and why are you all rubbing it in on all
sides? `A crime! a crime!' Only now I see clearly the imbecility of my
cowardice, now that I have decided to face this superfluous disgrace.
It's simply because I am contemptible and have nothing in me that I have
decided to, perhaps too for my advantage, as that\ldots{}
Porfiry\ldots{} suggested!''

``Brother, brother, what are you saying? Why, you have shed blood?''
cried Dounia in despair.

``Which all men shed,'' he put in almost frantically, ``which flows and
has always flowed in streams, which is spilt like champagne, and for
which men are crowned in the Capitol and are called afterwards
benefactors of mankind. Look into it more carefully and understand it! I
too wanted to do good to men and would have done hundreds, thousands of
good deeds to make up for that one piece of stupidity, not stupidity
even, simply clumsiness, for the idea was by no means so stupid as it
seems now that it has failed\ldots{}. (Everything seems stupid when it
fails.) By that stupidity I only wanted to put myself into an
independent position, to take the first step, to obtain means, and then
everything would have been smoothed over by benefits immeasurable in
comparison\ldots{}. But I\ldots{} I couldn't carry out even the first
step, because I am contemptible, that's what's the matter! And yet I
won't look at it as you do. If I had succeeded I should have been
crowned with glory, but now I'm trapped.''

``But that's not so, not so! Brother, what are you saying?''

``Ah, it's not picturesque, not æsthetically attractive! I fail to
understand why bombarding people by regular siege is more honourable.
The fear of appearances is the first symptom of impotence. I've never,
never recognised this more clearly than now, and I am further than ever
from seeing that what I did was a crime. I've never, never been stronger
and more convinced than now.''

The colour had rushed into his pale exhausted face, but as he uttered
his last explanation, he happened to meet Dounia's eyes and he saw such
anguish in them that he could not help being checked. He felt that he
had, anyway, made these two poor women miserable, that he was, anyway,
the cause\ldots{}

``Dounia darling, if I am guilty forgive me (though I cannot be forgiven
if I am guilty). Good-bye! We won't dispute. It's time, high time to go.
Don't follow me, I beseech you, I have somewhere else to go\ldots{}. But
you go at once and sit with mother. I entreat you to! It's my last
request of you. Don't leave her at all; I left her in a state of
anxiety, that she is not fit to bear; she will die or go out of her
mind. Be with her! Razumihin will be with you. I've been talking to
him\ldots{}. Don't cry about me: I'll try to be honest and manly all my
life, even if I am a murderer. Perhaps I shall some day make a name. I
won't disgrace you, you will see; I'll still show\ldots{}. Now good-bye
for the present,'' he concluded hurriedly, noticing again a strange
expression in Dounia's eyes at his last words and promises. ``Why are
you crying? Don't cry, don't cry: we are not parting for ever! Ah, yes!
Wait a minute, I'd forgotten!''

He went to the table, took up a thick dusty book, opened it and took
from between the pages a little water-colour portrait on ivory. It was
the portrait of his landlady's daughter, who had died of fever, that
strange girl who had wanted to be a nun. For a minute he gazed at the
delicate expressive face of his betrothed, kissed the portrait and gave
it to Dounia.

``I used to talk a great deal about it to her, only to her,'' he said
thoughtfully. ``To her heart I confided much of what has since been so
hideously realised. Don't be uneasy,'' he returned to Dounia, ``she was
as much opposed to it as you, and I am glad that she is gone. The great
point is that everything now is going to be different, is going to be
broken in two,'' he cried, suddenly returning to his dejection.
``Everything, everything, and am I prepared for it? Do I want it myself?
They say it is necessary for me to suffer! What's the object of these
senseless sufferings? shall I know any better what they are for, when I
am crushed by hardships and idiocy, and weak as an old man after twenty
years' penal servitude? And what shall I have to live for then? Why am I
consenting to that life now? Oh, I knew I was contemptible when I stood
looking at the Neva at daybreak to-day!''

At last they both went out. It was hard for Dounia, but she loved him.
She walked away, but after going fifty paces she turned round to look at
him again. He was still in sight. At the corner he too turned and for
the last time their eyes met; but noticing that she was looking at him,
he motioned her away with impatience and even vexation, and turned the
corner abruptly.

``I am wicked, I see that,'' he thought to himself, feeling ashamed a
moment later of his angry gesture to Dounia. ``But why are they so fond
of me if I don't deserve it? Oh, if only I were alone and no one loved
me and I too had never loved anyone! \emph{Nothing of all this would
have happened.} But I wonder shall I in those fifteen or twenty years
grow so meek that I shall humble myself before people and whimper at
every word that I am a criminal? Yes, that's it, that's it, that's what
they are sending me there for, that's what they want. Look at them
running to and fro about the streets, every one of them a scoundrel and
a criminal at heart and, worse still, an idiot. But try to get me off
and they'd be wild with righteous indignation. Oh, how I hate them
all!''

He fell to musing by what process it could come to pass, that he could
be humbled before all of them, indiscriminately---humbled by conviction.
And yet why not? It must be so. Would not twenty years of continual
bondage crush him utterly? Water wears out a stone. And why, why should
he live after that? Why should he go now when he knew that it would be
so? It was the hundredth time perhaps that he had asked himself that
question since the previous evening, but still he went.

\protect\hypertarget{link2HCH0039}{}{}

\mychapter{8}{Chapter VIII}\label{chapter-viii}

When he went into Sonia's room, it was already getting dark. All day
Sonia had been waiting for him in terrible anxiety. Dounia had been
waiting with her. She had come to her that morning, remembering
Svidrigaïlov's words that Sonia knew. We will not describe the
conversation and tears of the two girls, and how friendly they became.
Dounia gained one comfort at least from that interview, that her brother
would not be alone. He had gone to her, Sonia, first with his
confession; he had gone to her for human fellowship when he needed it;
she would go with him wherever fate might send him. Dounia did not ask,
but she knew it was so. She looked at Sonia almost with reverence and at
first almost embarrassed her by it. Sonia was almost on the point of
tears. She felt herself, on the contrary, hardly worthy to look at
Dounia. Dounia's gracious image when she had bowed to her so attentively
and respectfully at their first meeting in Raskolnikov's room had
remained in her mind as one of the fairest visions of her life.

Dounia at last became impatient and, leaving Sonia, went to her
brother's room to await him there; she kept thinking that he would come
there first. When she had gone, Sonia began to be tortured by the dread
of his committing suicide, and Dounia too feared it. But they had spent
the day trying to persuade each other that that could not be, and both
were less anxious while they were together. As soon as they parted, each
thought of nothing else. Sonia remembered how Svidrigaïlov had said to
her the day before that Raskolnikov had two alternatives---Siberia
or\ldots{} Besides she knew his vanity, his pride and his lack of faith.

``Is it possible that he has nothing but cowardice and fear of death to
make him live?'' she thought at last in despair.

Meanwhile the sun was setting. Sonia was standing in dejection, looking
intently out of the window, but from it she could see nothing but the
unwhitewashed blank wall of the next house. At last when she began to
feel sure of his death---he walked into the room.

She gave a cry of joy, but looking carefully into his face she turned
pale.

``Yes,'' said Raskolnikov, smiling. ``I have come for your cross, Sonia.
It was you told me to go to the cross-roads; why is it you are
frightened now it's come to that?''

Sonia gazed at him astonished. His tone seemed strange to her; a cold
shiver ran over her, but in a moment she guessed that the tone and the
words were a mask. He spoke to her looking away, as though to avoid
meeting her eyes.

``You see, Sonia, I've decided that it will be better so. There is one
fact\ldots{}. But it's a long story and there's no need to discuss it.
But do you know what angers me? It annoys me that all those stupid
brutish faces will be gaping at me directly, pestering me with their
stupid questions, which I shall have to answer---they'll point their
fingers at me\ldots{}. Tfoo! You know I am not going to Porfiry, I am
sick of him. I'd rather go to my friend, the Explosive Lieutenant; how I
shall surprise him, what a sensation I shall make! But I must be cooler;
I've become too irritable of late. You know I was nearly shaking my fist
at my sister just now, because she turned to take a last look at me.
It's a brutal state to be in! Ah! what am I coming to! Well, where are
the crosses?''

He seemed hardly to know what he was doing. He could not stay still or
concentrate his attention on anything; his ideas seemed to gallop after
one another, he talked incoherently, his hands trembled slightly.

Without a word Sonia took out of the drawer two crosses, one of cypress
wood and one of copper. She made the sign of the cross over herself and
over him, and put the wooden cross on his neck.

``It's the symbol of my taking up the cross,'' he laughed. ``As though I
had not suffered much till now! The wooden cross, that is the peasant
one; the copper one, that is Lizaveta's---you will wear yourself, show
me! So she had it on\ldots{} at that moment? I remember two things like
these too, a silver one and a little ikon. I threw them back on the old
woman's neck. Those would be appropriate now, really, those are what I
ought to put on now\ldots{}. But I am talking nonsense and forgetting
what matters; I'm somehow forgetful\ldots{}. You see I have come to warn
you, Sonia, so that you might know\ldots{} that's all---that's all I
came for. But I thought I had more to say. You wanted me to go yourself.
Well, now I am going to prison and you'll have your wish. Well, what are
you crying for? You too? Don't. Leave off! Oh, how I hate it all!''

But his feeling was stirred; his heart ached, as he looked at her. ``Why
is she grieving too?'' he thought to himself. ``What am I to her? Why
does she weep? Why is she looking after me, like my mother or Dounia?
She'll be my nurse.''

``Cross yourself, say at least one prayer,'' Sonia begged in a timid
broken voice.

``Oh certainly, as much as you like! And sincerely, Sonia,
sincerely\ldots{}.''

But he wanted to say something quite different.

He crossed himself several times. Sonia took up her shawl and put it
over her head. It was the green \emph{drap de dames} shawl of which
Marmeladov had spoken, ``the family shawl.'' Raskolnikov thought of that
looking at it, but he did not ask. He began to feel himself that he was
certainly forgetting things and was disgustingly agitated. He was
frightened at this. He was suddenly struck too by the thought that Sonia
meant to go with him.

``What are you doing? Where are you going? Stay here, stay! I'll go
alone,'' he cried in cowardly vexation, and almost resentful, he moved
towards the door. ``What's the use of going in procession?'' he muttered
going out.

Sonia remained standing in the middle of the room. He had not even said
good-bye to her; he had forgotten her. A poignant and rebellious doubt
surged in his heart.

``Was it right, was it right, all this?'' he thought again as he went
down the stairs. ``Couldn't he stop and retract it all\ldots{} and not
go?''

But still he went. He felt suddenly once for all that he mustn't ask
himself questions. As he turned into the street he remembered that he
had not said good-bye to Sonia, that he had left her in the middle of
the room in her green shawl, not daring to stir after he had shouted at
her, and he stopped short for a moment. At the same instant, another
thought dawned upon him, as though it had been lying in wait to strike
him then.

``Why, with what object did I go to her just now? I told her---on
business; on what business? I had no sort of business! To tell her I was
\emph{going}; but where was the need? Do I love her? No, no, I drove her
away just now like a dog. Did I want her crosses? Oh, how low I've sunk!
No, I wanted her tears, I wanted to see her terror, to see how her heart
ached! I had to have something to cling to, something to delay me, some
friendly face to see! And I dared to believe in myself, to dream of what
I would do! I am a beggarly contemptible wretch, contemptible!''

He walked along the canal bank, and he had not much further to go. But
on reaching the bridge he stopped and turning out of his way along it
went to the Hay Market.

He looked eagerly to right and left, gazed intently at every object and
could not fix his attention on anything; everything slipped away. ``In
another week, another month I shall be driven in a prison van over this
bridge, how shall I look at the canal then? I should like to remember
this!'' slipped into his mind. ``Look at this sign! How shall I read
those letters then? It's written here `Campany,' that's a thing to
remember, that letter \emph{a}, and to look at it again in a month---how
shall I look at it then? What shall I be feeling and thinking
then?\ldots{} How trivial it all must be, what I am fretting about now!
Of course it must all be interesting\ldots{} in its way\ldots{}
(Ha-ha-ha! What am I thinking about?) I am becoming a baby, I am showing
off to myself; why am I ashamed? Foo! how people shove! that fat man---a
German he must be---who pushed against me, does he know whom he pushed?
There's a peasant woman with a baby, begging. It's curious that she
thinks me happier than she is. I might give her something, for the
incongruity of it. Here's a five copeck piece left in my pocket, where
did I get it? Here, here\ldots{} take it, my good woman!''

``God bless you,'' the beggar chanted in a lachrymose voice.

He went into the Hay Market. It was distasteful, very distasteful to be
in a crowd, but he walked just where he saw most people. He would have
given anything in the world to be alone; but he knew himself that he
would not have remained alone for a moment. There was a man drunk and
disorderly in the crowd; he kept trying to dance and falling down. There
was a ring round him. Raskolnikov squeezed his way through the crowd,
stared for some minutes at the drunken man and suddenly gave a short
jerky laugh. A minute later he had forgotten him and did not see him,
though he still stared. He moved away at last, not remembering where he
was; but when he got into the middle of the square an emotion suddenly
came over him, overwhelming him body and mind.

He suddenly recalled Sonia's words, ``Go to the cross-roads, bow down to
the people, kiss the earth, for you have sinned against it too, and say
aloud to the whole world, `I am a murderer.'\,'' He trembled,
remembering that. And the hopeless misery and anxiety of all that time,
especially of the last hours, had weighed so heavily upon him that he
positively clutched at the chance of this new unmixed, complete
sensation. It came over him like a fit; it was like a single spark
kindled in his soul and spreading fire through him. Everything in him
softened at once and the tears started into his eyes. He fell to the
earth on the spot\ldots{}.

He knelt down in the middle of the square, bowed down to the earth, and
kissed that filthy earth with bliss and rapture. He got up and bowed
down a second time.

``He's boozed,'' a youth near him observed.

There was a roar of laughter.

``He's going to Jerusalem, brothers, and saying good-bye to his children
and his country. He's bowing down to all the world and kissing the great
city of St. Petersburg and its pavement,'' added a workman who was a
little drunk.

``Quite a young man, too!'' observed a third.

``And a gentleman,'' someone observed soberly.

``There's no knowing who's a gentleman and who isn't nowadays.''

These exclamations and remarks checked Raskolnikov, and the words, ``I
am a murderer,'' which were perhaps on the point of dropping from his
lips, died away. He bore these remarks quietly, however, and, without
looking round, he turned down a street leading to the police office. He
had a glimpse of something on the way which did not surprise him; he had
felt that it must be so. The second time he bowed down in the Hay Market
he saw, standing fifty paces from him on the left, Sonia. She was hiding
from him behind one of the wooden shanties in the market-place. She had
followed him then on his painful way! Raskolnikov at that moment felt
and knew once for all that Sonia was with him for ever and would follow
him to the ends of the earth, wherever fate might take him. It wrung his
heart\ldots{} but he was just reaching the fatal place.

He went into the yard fairly resolutely. He had to mount to the third
storey. ``I shall be some time going up,'' he thought. He felt as though
the fateful moment was still far off, as though he had plenty of time
left for consideration.

Again the same rubbish, the same eggshells lying about on the spiral
stairs, again the open doors of the flats, again the same kitchens and
the same fumes and stench coming from them. Raskolnikov had not been
here since that day. His legs were numb and gave way under him, but
still they moved forward. He stopped for a moment to take breath, to
collect himself, so as to enter \emph{like a man}. ``But why? what
for?'' he wondered, reflecting. ``If I must drink the cup what
difference does it make? The more revolting the better.'' He imagined
for an instant the figure of the ``explosive lieutenant,'' Ilya
Petrovitch. Was he actually going to him? Couldn't he go to someone
else? To Nikodim Fomitch? Couldn't he turn back and go straight to
Nikodim Fomitch's lodgings? At least then it would be done
privately\ldots{}. No, no! To the ``explosive lieutenant''! If he must
drink it, drink it off at once.

Turning cold and hardly conscious, he opened the door of the office.
There were very few people in it this time---only a house porter and a
peasant. The doorkeeper did not even peep out from behind his screen.
Raskolnikov walked into the next room. ``Perhaps I still need not
speak,'' passed through his mind. Some sort of clerk not wearing a
uniform was settling himself at a bureau to write. In a corner another
clerk was seating himself. Zametov was not there, nor, of course,
Nikodim Fomitch.

``No one in?'' Raskolnikov asked, addressing the person at the bureau.

``Whom do you want?''

``A-ah! Not a sound was heard, not a sight was seen, but I scent the
Russian\ldots{} how does it go on in the fairy tale\ldots{} I've
forgotten! `At your service!'\,'' a familiar voice cried suddenly.

Raskolnikov shuddered. The Explosive Lieutenant stood before him. He had
just come in from the third room. ``It is the hand of fate,'' thought
Raskolnikov. ``Why is he here?''

``You've come to see us? What about?'' cried Ilya Petrovitch. He was
obviously in an exceedingly good humour and perhaps a trifle
exhilarated. ``If it's on business you are rather early.\footnote{\emph{Translator's Note}—Dostoevsky appears to have forgotten that it is after
sunset, and that the last time Raskolnikov visited the
police office at two in the afternoon he was reproached for
coming too late.} It's
only a chance that I am here\ldots{} however I'll do what I can. I must
admit, I\ldots{} what is it, what is it? Excuse me\ldots{}.''

``Raskolnikov.''

``Of course, Raskolnikov. You didn't imagine I'd forgotten? Don't think
I am like that\ldots{} Rodion Ro---Ro---Rodionovitch, that's it, isn't
it?''

``Rodion Romanovitch.''

``Yes, yes, of course, Rodion Romanovitch! I was just getting at it. I
made many inquiries about you. I assure you I've been genuinely grieved
since that\ldots{} since I behaved like that\ldots{} it was explained to
me afterwards that you were a literary man\ldots{} and a learned one
too\ldots{} and so to say the first steps\ldots{} Mercy on us! What
literary or scientific man does not begin by some originality of
conduct! My wife and I have the greatest respect for literature, in my
wife it's a genuine passion! Literature and art! If only a man is a
gentleman, all the rest can be gained by talents, learning, good sense,
genius. As for a hat---well, what does a hat matter? I can buy a hat as
easily as I can a bun; but what's under the hat, what the hat covers, I
can't buy that! I was even meaning to come and apologise to you, but
thought maybe you'd\ldots{} But I am forgetting to ask you, is there
anything you want really? I hear your family have come?''

``Yes, my mother and sister.''

``I've even had the honour and happiness of meeting your sister---a
highly cultivated and charming person. I confess I was sorry I got so
hot with you. There it is! But as for my looking suspiciously at your
fainting fit---that affair has been cleared up splendidly! Bigotry and
fanaticism! I understand your indignation. Perhaps you are changing your
lodging on account of your family's arriving?''

``No, I only looked in\ldots{} I came to ask\ldots{} I thought that I
should find Zametov here.''

``Oh, yes! Of course, you've made friends, I heard. Well, no, Zametov is
not here. Yes, we've lost Zametov. He's not been here since
yesterday\ldots{} he quarrelled with everyone on leaving\ldots{} in the
rudest way. He is a feather-headed youngster, that's all; one might have
expected something from him, but there, you know what they are, our
brilliant young men. He wanted to go in for some examination, but it's
only to talk and boast about it, it will go no further than that. Of
course it's a very different matter with you or Mr. Razumihin there,
your friend. Your career is an intellectual one and you won't be
deterred by failure. For you, one may say, all the attractions of life
\emph{nihil est}---you are an ascetic, a monk, a hermit!\ldots{} A book,
a pen behind your ear, a learned research---that's where your spirit
soars! I am the same way myself\ldots{}. Have you read Livingstone's
Travels?''

``No.''

``Oh, I have. There are a great many Nihilists about nowadays, you know,
and indeed it is not to be wondered at. What sort of days are they? I
ask you. But we thought\ldots{} you are not a Nihilist of course? Answer
me openly, openly!''

``N-no\ldots{}''

``Believe me, you can speak openly to me as you would to yourself!
Official duty is one thing but\ldots{} you are thinking I meant to say
\emph{friendship} is quite another? No, you're wrong! It's not
friendship, but the feeling of a man and a citizen, the feeling of
humanity and of love for the Almighty. I may be an official, but I am
always bound to feel myself a man and a citizen\ldots{}. You were asking
about Zametov. Zametov will make a scandal in the French style in a
house of bad reputation, over a glass of champagne\ldots{} that's all
your Zametov is good for! While I'm perhaps, so to speak, burning with
devotion and lofty feelings, and besides I have rank, consequence, a
post! I am married and have children, I fulfil the duties of a man and a
citizen, but who is he, may I ask? I appeal to you as a man ennobled by
education\ldots{} Then these midwives, too, have become extraordinarily
numerous.''

Raskolnikov raised his eyebrows inquiringly. The words of Ilya
Petrovitch, who had obviously been dining, were for the most part a
stream of empty sounds for him. But some of them he understood. He
looked at him inquiringly, not knowing how it would end.

``I mean those crop-headed wenches,'' the talkative Ilya Petrovitch
continued. ``Midwives is my name for them. I think it a very
satisfactory one, ha-ha! They go to the Academy, study anatomy. If I
fall ill, am I to send for a young lady to treat me? What do you say?
Ha-ha!'' Ilya Petrovitch laughed, quite pleased with his own wit. ``It's
an immoderate zeal for education, but once you're educated, that's
enough. Why abuse it? Why insult honourable people, as that scoundrel
Zametov does? Why did he insult me, I ask you? Look at these suicides,
too, how common they are, you can't fancy! People spend their last
halfpenny and kill themselves, boys and girls and old people. Only this
morning we heard about a gentleman who had just come to town. Nil
Pavlitch, I say, what was the name of that gentleman who shot himself?''

``Svidrigaïlov,'' someone answered from the other room with drowsy
listlessness.

Raskolnikov started.

``Svidrigaïlov! Svidrigaïlov has shot himself!'' he cried.

``What, do you know Svidrigaïlov?''

``Yes\ldots{} I knew him\ldots{}. He hadn't been here long.''

``Yes, that's so. He had lost his wife, was a man of reckless habits and
all of a sudden shot himself, and in such a shocking way\ldots{}. He
left in his notebook a few words: that he dies in full possession of his
faculties and that no one is to blame for his death. He had money, they
say. How did you come to know him?''

``I\ldots{} was acquainted\ldots{} my sister was governess in his
family.''

``Bah-bah-bah! Then no doubt you can tell us something about him. You
had no suspicion?''

``I saw him yesterday\ldots{} he\ldots{} was drinking wine; I knew
nothing.''

Raskolnikov felt as though something had fallen on him and was stifling
him.

``You've turned pale again. It's so stuffy here\ldots{}''

``Yes, I must go,'' muttered Raskolnikov. ``Excuse my troubling
you\ldots{}.''

``Oh, not at all, as often as you like. It's a pleasure to see you and I
am glad to say so.''

Ilya Petrovitch held out his hand.

``I only wanted\ldots{} I came to see Zametov.''

``I understand, I understand, and it's a pleasure to see you.''

``I\ldots{} am very glad\ldots{} good-bye,'' Raskolnikov smiled.

He went out; he reeled, he was overtaken with giddiness and did not know
what he was doing. He began going down the stairs, supporting himself
with his right hand against the wall. He fancied that a porter pushed
past him on his way upstairs to the police office, that a dog in the
lower storey kept up a shrill barking and that a woman flung a
rolling-pin at it and shouted. He went down and out into the yard.
There, not far from the entrance, stood Sonia, pale and horror-stricken.
She looked wildly at him. He stood still before her. There was a look of
poignant agony, of despair, in her face. She clasped her hands. His lips
worked in an ugly, meaningless smile. He stood still a minute, grinned
and went back to the police office.

Ilya Petrovitch had sat down and was rummaging among some papers. Before
him stood the same peasant who had pushed by on the stairs.

``Hulloa! Back again! have you left something behind? What's the
matter?''

Raskolnikov, with white lips and staring eyes, came slowly nearer. He
walked right to the table, leaned his hand on it, tried to say
something, but could not; only incoherent sounds were audible.

``You are feeling ill, a chair! Here, sit down! Some water!''

Raskolnikov dropped on to a chair, but he kept his eyes fixed on the
face of Ilya Petrovitch, which expressed unpleasant surprise. Both
looked at one another for a minute and waited. Water was brought.

``It was I\ldots{}'' began Raskolnikov.

``Drink some water.''

Raskolnikov refused the water with his hand, and softly and brokenly,
but distinctly said:

``\emph{It was I killed the old pawnbroker woman and her sister Lizaveta
with an axe and robbed them.}''

Ilya Petrovitch opened his mouth. People ran up on all sides.

Raskolnikov repeated his statement.


\part*{EPILOGUE}\label{epilogue}
\addcontentsline{toc}{part}{EPILOGUE}

\mychapter{1}{I}

Siberia. On the banks of a broad solitary river stands a town, one of
the administrative centres of Russia; in the town there is a fortress,
in the fortress there is a prison. In the prison the second-class
convict Rodion Raskolnikov has been confined for nine months. Almost a
year and a half has passed since his crime.

There had been little difficulty about his trial. The criminal adhered
exactly, firmly, and clearly to his statement. He did not confuse nor
misrepresent the facts, nor soften them in his own interest, nor omit
the smallest detail. He explained every incident of the murder, the
secret of \emph{the pledge} (the piece of wood with a strip of metal)
which was found in the murdered woman's hand. He described minutely how
he had taken her keys, what they were like, as well as the chest and its
contents; he explained the mystery of Lizaveta's murder; described how
Koch and, after him, the student knocked, and repeated all they had said
to one another; how he afterwards had run downstairs and heard Nikolay
and Dmitri shouting; how he had hidden in the empty flat and afterwards
gone home. He ended by indicating the stone in the yard off the
Voznesensky Prospect under which the purse and the trinkets were found.
The whole thing, in fact, was perfectly clear. The lawyers and the
judges were very much struck, among other things, by the fact that he
had hidden the trinkets and the purse under a stone, without making use
of them, and that, what was more, he did not now remember what the
trinkets were like, or even how many there were. The fact that he had
never opened the purse and did not even know how much was in it seemed
incredible. There turned out to be in the purse three hundred and
seventeen roubles and sixty copecks. From being so long under the stone,
some of the most valuable notes lying uppermost had suffered from the
damp. They were a long while trying to discover why the accused man
should tell a lie about this, when about everything else he had made a
truthful and straightforward confession. Finally some of the lawyers
more versed in psychology admitted that it was possible he had really
not looked into the purse, and so didn't know what was in it when he hid
it under the stone. But they immediately drew the deduction that the
crime could only have been committed through temporary mental
derangement, through homicidal mania, without object or the pursuit of
gain. This fell in with the most recent fashionable theory of temporary
insanity, so often applied in our days in criminal cases. Moreover
Raskolnikov's hypochondriacal condition was proved by many witnesses, by
Dr. Zossimov, his former fellow students, his landlady and her servant.
All this pointed strongly to the conclusion that Raskolnikov was not
quite like an ordinary murderer and robber, but that there was another
element in the case.

To the intense annoyance of those who maintained this opinion, the
criminal scarcely attempted to defend himself. To the decisive question
as to what motive impelled him to the murder and the robbery, he
answered very clearly with the coarsest frankness that the cause was his
miserable position, his poverty and helplessness, and his desire to
provide for his first steps in life by the help of the three thousand
roubles he had reckoned on finding. He had been led to the murder
through his shallow and cowardly nature, exasperated moreover by
privation and failure. To the question what led him to confess, he
answered that it was his heartfelt repentance. All this was almost
coarse\ldots{}.

The sentence however was more merciful than could have been expected,
perhaps partly because the criminal had not tried to justify himself,
but had rather shown a desire to exaggerate his guilt. All the strange
and peculiar circumstances of the crime were taken into consideration.
There could be no doubt of the abnormal and poverty-stricken condition
of the criminal at the time. The fact that he had made no use of what he
had stolen was put down partly to the effect of remorse, partly to his
abnormal mental condition at the time of the crime. Incidentally the
murder of Lizaveta served indeed to confirm the last hypothesis: a man
commits two murders and forgets that the door is open! Finally, the
confession, at the very moment when the case was hopelessly muddled by
the false evidence given by Nikolay through melancholy and fanaticism,
and when, moreover, there were no proofs against the real criminal, no
suspicions even (Porfiry Petrovitch fully kept his word)---all this did
much to soften the sentence. Other circumstances, too, in the prisoner's
favour came out quite unexpectedly. Razumihin somehow discovered and
proved that while Raskolnikov was at the university he had helped a poor
consumptive fellow student and had spent his last penny on supporting
him for six months, and when this student died, leaving a decrepit old
father whom he had maintained almost from his thirteenth year,
Raskolnikov had got the old man into a hospital and paid for his funeral
when he died. Raskolnikov's landlady bore witness, too, that when they
had lived in another house at Five Corners, Raskolnikov had rescued two
little children from a house on fire and was burnt in doing so. This was
investigated and fairly well confirmed by many witnesses. These facts
made an impression in his favour.

And in the end the criminal was, in consideration of extenuating
circumstances, condemned to penal servitude in the second class for a
term of eight years only.

At the very beginning of the trial Raskolnikov's mother fell ill. Dounia
and Razumihin found it possible to get her out of Petersburg during the
trial. Razumihin chose a town on the railway not far from Petersburg, so
as to be able to follow every step of the trial and at the same time to
see Avdotya Romanovna as often as possible. Pulcheria Alexandrovna's
illness was a strange nervous one and was accompanied by a partial
derangement of her intellect.

When Dounia returned from her last interview with her brother, she had
found her mother already ill, in feverish delirium. That evening
Razumihin and she agreed what answers they must make to her mother's
questions about Raskolnikov and made up a complete story for her
mother's benefit of his having to go away to a distant part of Russia on
a business commission, which would bring him in the end money and
reputation.

But they were struck by the fact that Pulcheria Alexandrovna never asked
them anything on the subject, neither then nor thereafter. On the
contrary, she had her own version of her son's sudden departure; she
told them with tears how he had come to say good-bye to her, hinting
that she alone knew many mysterious and important facts, and that Rodya
had many very powerful enemies, so that it was necessary for him to be
in hiding. As for his future career, she had no doubt that it would be
brilliant when certain sinister influences could be removed. She assured
Razumihin that her son would be one day a great statesman, that his
article and brilliant literary talent proved it. This article she was
continually reading, she even read it aloud, almost took it to bed with
her, but scarcely asked where Rodya was, though the subject was
obviously avoided by the others, which might have been enough to awaken
her suspicions.

They began to be frightened at last at Pulcheria Alexandrovna's strange
silence on certain subjects. She did not, for instance, complain of
getting no letters from him, though in previous years she had only lived
on the hope of letters from her beloved Rodya. This was the cause of
great uneasiness to Dounia; the idea occurred to her that her mother
suspected that there was something terrible in her son's fate and was
afraid to ask, for fear of hearing something still more awful. In any
case, Dounia saw clearly that her mother was not in full possession of
her faculties.

It happened once or twice, however, that Pulcheria Alexandrovna gave
such a turn to the conversation that it was impossible to answer her
without mentioning where Rodya was, and on receiving unsatisfactory and
suspicious answers she became at once gloomy and silent, and this mood
lasted for a long time. Dounia saw at last that it was hard to deceive
her and came to the conclusion that it was better to be absolutely
silent on certain points; but it became more and more evident that the
poor mother suspected something terrible. Dounia remembered her
brother's telling her that her mother had overheard her talking in her
sleep on the night after her interview with Svidrigaïlov and before the
fatal day of the confession: had not she made out something from that?
Sometimes days and even weeks of gloomy silence and tears would be
succeeded by a period of hysterical animation, and the invalid would
begin to talk almost incessantly of her son, of her hopes of his
future\ldots{}. Her fancies were sometimes very strange. They humoured
her, pretended to agree with her (she saw perhaps that they were
pretending), but she still went on talking.

Five months after Raskolnikov's confession, he was sentenced. Razumihin
and Sonia saw him in prison as often as it was possible. At last the
moment of separation came. Dounia swore to her brother that the
separation should not be for ever, Razumihin did the same. Razumihin, in
his youthful ardour, had firmly resolved to lay the foundations at least
of a secure livelihood during the next three or four years, and saving
up a certain sum, to emigrate to Siberia, a country rich in every
natural resource and in need of workers, active men and capital. There
they would settle in the town where Rodya was and all together would
begin a new life. They all wept at parting.

Raskolnikov had been very dreamy for a few days before. He asked a great
deal about his mother and was constantly anxious about her. He worried
so much about her that it alarmed Dounia. When he heard about his
mother's illness he became very gloomy. With Sonia he was particularly
reserved all the time. With the help of the money left to her by
Svidrigaïlov, Sonia had long ago made her preparations to follow the
party of convicts in which he was despatched to Siberia. Not a word
passed between Raskolnikov and her on the subject, but both knew it
would be so. At the final leave-taking he smiled strangely at his
sister's and Razumihin's fervent anticipations of their happy future
together when he should come out of prison. He predicted that their
mother's illness would soon have a fatal ending. Sonia and he at last
set off.

Two months later Dounia was married to Razumihin. It was a quiet and
sorrowful wedding; Porfiry Petrovitch and Zossimov were invited however.
During all this period Razumihin wore an air of resolute determination.
Dounia put implicit faith in his carrying out his plans and indeed she
could not but believe in him. He displayed a rare strength of will.
Among other things he began attending university lectures again in order
to take his degree. They were continually making plans for the future;
both counted on settling in Siberia within five years at least. Till
then they rested their hopes on Sonia.

Pulcheria Alexandrovna was delighted to give her blessing to Dounia's
marriage with Razumihin; but after the marriage she became even more
melancholy and anxious. To give her pleasure Razumihin told her how
Raskolnikov had looked after the poor student and his decrepit father
and how a year ago he had been burnt and injured in rescuing two little
children from a fire. These two pieces of news excited Pulcheria
Alexandrovna's disordered imagination almost to ecstasy. She was
continually talking about them, even entering into conversation with
strangers in the street, though Dounia always accompanied her. In public
conveyances and shops, wherever she could capture a listener, she would
begin the discourse about her son, his article, how he had helped the
student, how he had been burnt at the fire, and so on! Dounia did not
know how to restrain her. Apart from the danger of her morbid
excitement, there was the risk of someone's recalling Raskolnikov's name
and speaking of the recent trial. Pulcheria Alexandrovna found out the
address of the mother of the two children her son had saved and insisted
on going to see her.

At last her restlessness reached an extreme point. She would sometimes
begin to cry suddenly and was often ill and feverishly delirious. One
morning she declared that by her reckoning Rodya ought soon to be home,
that she remembered when he said good-bye to her he said that they must
expect him back in nine months. She began to prepare for his coming,
began to do up her room for him, to clean the furniture, to wash and put
up new hangings and so on. Dounia was anxious, but said nothing and
helped her to arrange the room. After a fatiguing day spent in continual
fancies, in joyful day-dreams and tears, Pulcheria Alexandrovna was
taken ill in the night and by morning she was feverish and delirious. It
was brain fever. She died within a fortnight. In her delirium she
dropped words which showed that she knew a great deal more about her
son's terrible fate than they had supposed.

For a long time Raskolnikov did not know of his mother's death, though a
regular correspondence had been maintained from the time he reached
Siberia. It was carried on by means of Sonia, who wrote every month to
the Razumihins and received an answer with unfailing regularity. At
first they found Sonia's letters dry and unsatisfactory, but later on
they came to the conclusion that the letters could not be better, for
from these letters they received a complete picture of their unfortunate
brother's life. Sonia's letters were full of the most matter-of-fact
detail, the simplest and clearest description of all Raskolnikov's
surroundings as a convict. There was no word of her own hopes, no
conjecture as to the future, no description of her feelings. Instead of
any attempt to interpret his state of mind and inner life, she gave the
simple facts---that is, his own words, an exact account of his health,
what he asked for at their interviews, what commission he gave her and
so on. All these facts she gave with extraordinary minuteness. The
picture of their unhappy brother stood out at last with great clearness
and precision. There could be no mistake, because nothing was given but
facts.

But Dounia and her husband could get little comfort out of the news,
especially at first. Sonia wrote that he was constantly sullen and not
ready to talk, that he scarcely seemed interested in the news she gave
him from their letters, that he sometimes asked after his mother and
that when, seeing that he had guessed the truth, she told him at last of
her death, she was surprised to find that he did not seem greatly
affected by it, not externally at any rate. She told them that, although
he seemed so wrapped up in himself and, as it were, shut himself off
from everyone---he took a very direct and simple view of his new life;
that he understood his position, expected nothing better for the time,
had no ill-founded hopes (as is so common in his position) and scarcely
seemed surprised at anything in his surroundings, so unlike anything he
had known before. She wrote that his health was satisfactory; he did his
work without shirking or seeking to do more; he was almost indifferent
about food, but except on Sundays and holidays the food was so bad that
at last he had been glad to accept some money from her, Sonia, to have
his own tea every day. He begged her not to trouble about anything else,
declaring that all this fuss about him only annoyed him. Sonia wrote
further that in prison he shared the same room with the rest, that she
had not seen the inside of their barracks, but concluded that they were
crowded, miserable and unhealthy; that he slept on a plank bed with a
rug under him and was unwilling to make any other arrangement. But that
he lived so poorly and roughly, not from any plan or design, but simply
from inattention and indifference.

Sonia wrote simply that he had at first shown no interest in her visits,
had almost been vexed with her indeed for coming, unwilling to talk and
rude to her. But that in the end these visits had become a habit and
almost a necessity for him, so that he was positively distressed when
she was ill for some days and could not visit him. She used to see him
on holidays at the prison gates or in the guard-room, to which he was
brought for a few minutes to see her. On working days she would go to
see him at work either at the workshops or at the brick kilns, or at the
sheds on the banks of the Irtish.

About herself, Sonia wrote that she had succeeded in making some
acquaintances in the town, that she did sewing, and, as there was
scarcely a dressmaker in the town, she was looked upon as an
indispensable person in many houses. But she did not mention that the
authorities were, through her, interested in Raskolnikov; that his task
was lightened and so on.

At last the news came (Dounia had indeed noticed signs of alarm and
uneasiness in the preceding letters) that he held aloof from everyone,
that his fellow prisoners did not like him, that he kept silent for days
at a time and was becoming very pale. In the last letter Sonia wrote
that he had been taken very seriously ill and was in the convict ward of
the hospital.

\mychapter{2}{II}

He was ill a long time. But it was not the horrors of prison life, not
the hard labour, the bad food, the shaven head, or the patched clothes
that crushed him. What did he care for all those trials and hardships!
he was even glad of the hard work. Physically exhausted, he could at
least reckon on a few hours of quiet sleep. And what was the food to
him---the thin cabbage soup with beetles floating in it? In the past as
a student he had often not had even that. His clothes were warm and
suited to his manner of life. He did not even feel the fetters. Was he
ashamed of his shaven head and parti-coloured coat? Before whom? Before
Sonia? Sonia was afraid of him, how could he be ashamed before her? And
yet he was ashamed even before Sonia, whom he tortured because of it
with his contemptuous rough manner. But it was not his shaven head and
his fetters he was ashamed of: his pride had been stung to the quick. It
was wounded pride that made him ill. Oh, how happy he would have been if
he could have blamed himself! He could have borne anything then, even
shame and disgrace. But he judged himself severely, and his exasperated
conscience found no particularly terrible fault in his past, except a
simple \emph{blunder} which might happen to anyone. He was ashamed just
because he, Raskolnikov, had so hopelessly, stupidly come to grief
through some decree of blind fate, and must humble himself and submit to
``the idiocy'' of a sentence, if he were anyhow to be at peace.

Vague and objectless anxiety in the present, and in the future a
continual sacrifice leading to nothing---that was all that lay before
him. And what comfort was it to him that at the end of eight years he
would only be thirty-two and able to begin a new life! What had he to
live for? What had he to look forward to? Why should he strive? To live
in order to exist? Why, he had been ready a thousand times before to
give up existence for the sake of an idea, for a hope, even for a fancy.
Mere existence had always been too little for him; he had always wanted
more. Perhaps it was just because of the strength of his desires that he
had thought himself a man to whom more was permissible than to others.

And if only fate would have sent him repentance---burning repentance
that would have torn his heart and robbed him of sleep, that repentance,
the awful agony of which brings visions of hanging or drowning! Oh, he
would have been glad of it! Tears and agonies would at least have been
life. But he did not repent of his crime.

At least he might have found relief in raging at his stupidity, as he
had raged at the grotesque blunders that had brought him to prison. But
now in prison, \emph{in freedom}, he thought over and criticised all his
actions again and by no means found them so blundering and so grotesque
as they had seemed at the fatal time.

``In what way,'' he asked himself, ``was my theory stupider than others
that have swarmed and clashed from the beginning of the world? One has
only to look at the thing quite independently, broadly, and uninfluenced
by commonplace ideas, and my idea will by no means seem so\ldots{}
strange. Oh, sceptics and halfpenny philosophers, why do you halt
half-way!

``Why does my action strike them as so horrible?'' he said to himself.
``Is it because it was a crime? What is meant by crime? My conscience is
at rest. Of course, it was a legal crime, of course, the letter of the
law was broken and blood was shed. Well, punish me for the letter of the
law\ldots{} and that's enough. Of course, in that case many of the
benefactors of mankind who snatched power for themselves instead of
inheriting it ought to have been punished at their first steps. But
those men succeeded and so \emph{they were right}, and I didn't, and so
I had no right to have taken that step.''

It was only in that that he recognised his criminality, only in the fact
that he had been unsuccessful and had confessed it.

He suffered too from the question: why had he not killed himself? Why
had he stood looking at the river and preferred to confess? Was the
desire to live so strong and was it so hard to overcome it? Had not
Svidrigaïlov overcome it, although he was afraid of death?

In misery he asked himself this question, and could not understand that,
at the very time he had been standing looking into the river, he had
perhaps been dimly conscious of the fundamental falsity in himself and
his convictions. He didn't understand that that consciousness might be
the promise of a future crisis, of a new view of life and of his future
resurrection.

He preferred to attribute it to the dead weight of instinct which he
could not step over, again through weakness and meanness. He looked at
his fellow prisoners and was amazed to see how they all loved life and
prized it. It seemed to him that they loved and valued life more in
prison than in freedom. What terrible agonies and privations some of
them, the tramps for instance, had endured! Could they care so much for
a ray of sunshine, for the primeval forest, the cold spring hidden away
in some unseen spot, which the tramp had marked three years before, and
longed to see again, as he might to see his sweetheart, dreaming of the
green grass round it and the bird singing in the bush? As he went on he
saw still more inexplicable examples.

In prison, of course, there was a great deal he did not see and did not
want to see; he lived as it were with downcast eyes. It was loathsome
and unbearable for him to look. But in the end there was much that
surprised him and he began, as it were involuntarily, to notice much
that he had not suspected before. What surprised him most of all was the
terrible impossible gulf that lay between him and all the rest. They
seemed to be a different species, and he looked at them and they at him
with distrust and hostility. He felt and knew the reasons of his
isolation, but he would never have admitted till then that those reasons
were so deep and strong. There were some Polish exiles, political
prisoners, among them. They simply looked down upon all the rest as
ignorant churls; but Raskolnikov could not look upon them like that. He
saw that these ignorant men were in many respects far wiser than the
Poles. There were some Russians who were just as contemptuous, a former
officer and two seminarists. Raskolnikov saw their mistake as clearly.
He was disliked and avoided by everyone; they even began to hate him at
last---why, he could not tell. Men who had been far more guilty despised
and laughed at his crime.

``You're a gentleman,'' they used to say. ``You shouldn't hack about
with an axe; that's not a gentleman's work.''

The second week in Lent, his turn came to take the sacrament with his
gang. He went to church and prayed with the others. A quarrel broke out
one day, he did not know how. All fell on him at once in a fury.

``You're an infidel! You don't believe in God,'' they shouted. ``You
ought to be killed.''

He had never talked to them about God nor his belief, but they wanted to
kill him as an infidel. He said nothing. One of the prisoners rushed at
him in a perfect frenzy. Raskolnikov awaited him calmly and silently;
his eyebrows did not quiver, his face did not flinch. The guard
succeeded in intervening between him and his assailant, or there would
have been bloodshed.

There was another question he could not decide: why were they all so
fond of Sonia? She did not try to win their favour; she rarely met them,
sometimes only she came to see him at work for a moment. And yet
everybody knew her, they knew that she had come out to follow
\emph{him}, knew how and where she lived. She never gave them money, did
them no particular services. Only once at Christmas she sent them all
presents of pies and rolls. But by degrees closer relations sprang up
between them and Sonia. She would write and post letters for them to
their relations. Relations of the prisoners who visited the town, at
their instructions, left with Sonia presents and money for them. Their
wives and sweethearts knew her and used to visit her. And when she
visited Raskolnikov at work, or met a party of the prisoners on the
road, they all took off their hats to her. ``Little mother Sofya
Semyonovna, you are our dear, good little mother,'' coarse branded
criminals said to that frail little creature. She would smile and bow to
them and everyone was delighted when she smiled. They even admired her
gait and turned round to watch her walking; they admired her too for
being so little, and, in fact, did not know what to admire her most for.
They even came to her for help in their illnesses.

He was in the hospital from the middle of Lent till after Easter. When
he was better, he remembered the dreams he had had while he was feverish
and delirious. He dreamt that the whole world was condemned to a
terrible new strange plague that had come to Europe from the depths of
Asia. All were to be destroyed except a very few chosen. Some new sorts
of microbes were attacking the bodies of men, but these microbes were
endowed with intelligence and will. Men attacked by them became at once
mad and furious. But never had men considered themselves so intellectual
and so completely in possession of the truth as these sufferers, never
had they considered their decisions, their scientific conclusions, their
moral convictions so infallible. Whole villages, whole towns and peoples
went mad from the infection. All were excited and did not understand one
another. Each thought that he alone had the truth and was wretched
looking at the others, beat himself on the breast, wept, and wrung his
hands. They did not know how to judge and could not agree what to
consider evil and what good; they did not know whom to blame, whom to
justify. Men killed each other in a sort of senseless spite. They
gathered together in armies against one another, but even on the march
the armies would begin attacking each other, the ranks would be broken
and the soldiers would fall on each other, stabbing and cutting, biting
and devouring each other. The alarm bell was ringing all day long in the
towns; men rushed together, but why they were summoned and who was
summoning them no one knew. The most ordinary trades were abandoned,
because everyone proposed his own ideas, his own improvements, and they
could not agree. The land too was abandoned. Men met in groups, agreed
on something, swore to keep together, but at once began on something
quite different from what they had proposed. They accused one another,
fought and killed each other. There were conflagrations and famine. All
men and all things were involved in destruction. The plague spread and
moved further and further. Only a few men could be saved in the whole
world. They were a pure chosen people, destined to found a new race and
a new life, to renew and purify the earth, but no one had seen these
men, no one had heard their words and their voices.

Raskolnikov was worried that this senseless dream haunted his memory so
miserably, the impression of this feverish delirium persisted so long.
The second week after Easter had come. There were warm bright spring
days; in the prison ward the grating windows under which the sentinel
paced were opened. Sonia had only been able to visit him twice during
his illness; each time she had to obtain permission, and it was
difficult. But she often used to come to the hospital yard, especially
in the evening, sometimes only to stand a minute and look up at the
windows of the ward.

One evening, when he was almost well again, Raskolnikov fell asleep. On
waking up he chanced to go to the window, and at once saw Sonia in the
distance at the hospital gate. She seemed to be waiting for someone.
Something stabbed him to the heart at that minute. He shuddered and
moved away from the window. Next day Sonia did not come, nor the day
after; he noticed that he was expecting her uneasily. At last he was
discharged. On reaching the prison he learnt from the convicts that
Sofya Semyonovna was lying ill at home and was unable to go out.

He was very uneasy and sent to inquire after her; he soon learnt that
her illness was not dangerous. Hearing that he was anxious about her,
Sonia sent him a pencilled note, telling him that she was much better,
that she had a slight cold and that she would soon, very soon come and
see him at his work. His heart throbbed painfully as he read it.

Again it was a warm bright day. Early in the morning, at six o'clock, he
went off to work on the river bank, where they used to pound alabaster
and where there was a kiln for baking it in a shed. There were only
three of them sent. One of the convicts went with the guard to the
fortress to fetch a tool; the other began getting the wood ready and
laying it in the kiln. Raskolnikov came out of the shed on to the river
bank, sat down on a heap of logs by the shed and began gazing at the
wide deserted river. From the high bank a broad landscape opened before
him, the sound of singing floated faintly audible from the other bank.
In the vast steppe, bathed in sunshine, he could just see, like black
specks, the nomads' tents. There there was freedom, there other men were
living, utterly unlike those here; there time itself seemed to stand
still, as though the age of Abraham and his flocks had not passed.
Raskolnikov sat gazing, his thoughts passed into day-dreams, into
contemplation; he thought of nothing, but a vague restlessness excited
and troubled him. Suddenly he found Sonia beside him; she had come up
noiselessly and sat down at his side. It was still quite early; the
morning chill was still keen. She wore her poor old burnous and the
green shawl; her face still showed signs of illness, it was thinner and
paler. She gave him a joyful smile of welcome, but held out her hand
with her usual timidity. She was always timid of holding out her hand to
him and sometimes did not offer it at all, as though afraid he would
repel it. He always took her hand as though with repugnance, always
seemed vexed to meet her and was sometimes obstinately silent throughout
her visit. Sometimes she trembled before him and went away deeply
grieved. But now their hands did not part. He stole a rapid glance at
her and dropped his eyes on the ground without speaking. They were
alone, no one had seen them. The guard had turned away for the time.

How it happened he did not know. But all at once something seemed to
seize him and fling him at her feet. He wept and threw his arms round
her knees. For the first instant she was terribly frightened and she
turned pale. She jumped up and looked at him trembling. But at the same
moment she understood, and a light of infinite happiness came into her
eyes. She knew and had no doubt that he loved her beyond everything and
that at last the moment had come\ldots{}.

They wanted to speak, but could not; tears stood in their eyes. They
were both pale and thin; but those sick pale faces were bright with the
dawn of a new future, of a full resurrection into a new life. They were
renewed by love; the heart of each held infinite sources of life for the
heart of the other.

They resolved to wait and be patient. They had another seven years to
wait, and what terrible suffering and what infinite happiness before
them! But he had risen again and he knew it and felt it in all his
being, while she---she only lived in his life.

On the evening of the same day, when the barracks were locked,
Raskolnikov lay on his plank bed and thought of her. He had even fancied
that day that all the convicts who had been his enemies looked at him
differently; he had even entered into talk with them and they answered
him in a friendly way. He remembered that now, and thought it was bound
to be so. Wasn't everything now bound to be changed?

He thought of her. He remembered how continually he had tormented her
and wounded her heart. He remembered her pale and thin little face. But
these recollections scarcely troubled him now; he knew with what
infinite love he would now repay all her sufferings. And what were all,
\emph{all} the agonies of the past! Everything, even his crime, his
sentence and imprisonment, seemed to him now in the first rush of
feeling an external, strange fact with which he had no concern. But he
could not think for long together of anything that evening, and he could
not have analysed anything consciously; he was simply feeling. Life had
stepped into the place of theory and something quite different would
work itself out in his mind.

Under his pillow lay the New Testament. He took it up mechanically. The
book belonged to Sonia; it was the one from which she had read the
raising of Lazarus to him. At first he was afraid that she would worry
him about religion, would talk about the gospel and pester him with
books. But to his great surprise she had not once approached the subject
and had not even offered him the Testament. He had asked her for it
himself not long before his illness and she brought him the book without
a word. Till now he had not opened it.

He did not open it now, but one thought passed through his mind: ``Can
her convictions not be mine now? Her feelings, her aspirations at
least\ldots{}.''

She too had been greatly agitated that day, and at night she was taken
ill again. But she was so happy---and so unexpectedly happy---that she
was almost frightened of her happiness. Seven years, \emph{only} seven
years! At the beginning of their happiness at some moments they were
both ready to look on those seven years as though they were seven days.
He did not know that the new life would not be given him for nothing,
that he would have to pay dearly for it, that it would cost him great
striving, great suffering.

But that is the beginning of a new story---the story of the gradual
renewal of a man, the story of his gradual regeneration, of his passing
from one world into another, of his initiation into a new unknown life.
That might be the subject of a new story, but our present story is
ended.



\end{document}
