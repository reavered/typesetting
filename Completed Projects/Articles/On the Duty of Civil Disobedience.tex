\documentclass[12pt]{article}
\usepackage[utf8]{inputenc}
\usepackage[english]{babel}
\usepackage{ebgaramond}
\usepackage[top=2.5cm, bottom=2.5cm, left=2.5cm, right=2.5cm]{geometry}

\usepackage{hyperref}
\hypersetup{
    colorlinks=true,   
    urlcolor=red,
}

\usepackage[autostyle, english = american]{csquotes}
\MakeOuterQuote{"}

%=======PARAGRAPH FORMATTING=========%
\setlength{\parindent}{0pt} %no paragraph indents
\setlength{\parskip}{1em}   %single space between paragraphs

\renewcommand{\thefootnote}{\fnsymbol{footnote}}
\setlength{\skip\footins}{1cm}
\usepackage[symbol*]{footmisc}
\renewcommand{\footnotemargin}{3mm} %Setting left margin
\renewcommand{\footnotelayout}{\hspace{2mm}} %spacing between the footnote number and the text of footnote

\title{\vspace{-2.5cm}On the Duty of Civil Disobedience\\ \textit{\normalsize or}\\ \textit{\large Resistance to Civil Government\footnote{Editor's Note: Original publication title}}\vspace{-3mm}}

\author{by Henry David Thoreau, 1849}

\date{\vspace{-2em}}

\begin{document}

\maketitle
I heartily accept the motto, ``That government is best which governs
least''; and I should like to see it acted up to more rapidly and
systematically. Carried out, it finally amounts to this, which also I
believe---``That government is best which governs not at all''; and when
men are prepared for it, that will be the kind of government which they
will have. Government is at best but an expedient; but most governments
are usually, and all governments are sometimes, inexpedient. The
objections which have been brought against a standing army, and they are
many and weighty, and deserve to prevail, may also at last be brought
against a standing government. The standing army is only an arm of the
standing government. The government itself, which is only the mode which
the people have chosen to execute their will, is equally liable to be
abused and perverted before the people can act through it. Witness the
present Mexican war, the work of comparatively a few individuals using
the standing government as their tool; for in the outset, the people
would not have consented to this measure.

This American government---what is it but a tradition, though a recent
one, endeavoring to transmit itself unimpaired to posterity, but each
instant losing some of its integrity? It has not the vitality and force
of a single living man; for a single man can bend it to his will. It is
a sort of wooden gun to the people themselves. But it is not the less
necessary for this; for the people must have some complicated machinery
or other, and hear its din, to satisfy that idea of government which
they have. Governments show thus how successfully men can be imposed
upon, even impose on themselves, for their own advantage. It is
excellent, we must all allow. Yet this government never of itself
furthered any enterprise, but by the alacrity with which it got out of
its way. \emph{It} does not keep the country free. \emph{It} does not
settle the West. \emph{It} does not educate. The character inherent in
the American people has done all that has been accomplished; and it
would have done somewhat more, if the government had not sometimes got
in its way. For government is an expedient, by which men would fain
succeed in letting one another alone; and, as has been said, when it is
most expedient, the governed are most let alone by it. Trade and
commerce, if they were not made of india-rubber, would never manage to
bounce over obstacles which legislators are continually putting in their
way; and if one were to judge these men wholly by the effects of their
actions and not partly by their intentions, they would deserve to be
classed and punished with those mischievious persons who put
obstructions on the railroads.

But, to speak practically and as a citizen, unlike those who call
themselves no-government men, I ask for, not \emph{at once} no
government, but at once a better government. Let every man make known
what kind of government would command his respect, and that will be one
step toward obtaining it.

After all, the practical reason why, when the power is once in the hands
of the people, a majority are permitted, and for a long period continue,
to rule is not because they are most likely to be in the right, nor
because this seems fairest to the minority, but because they are
physically the strongest. But a government in which the majority rule in
all cases can not be based on justice, even as far as men understand it.
Can there not be a government in which the majorities do not virtually
decide right and wrong, but conscience?---in which majorities decide
only those questions to which the rule of expediency is applicable? Must
the citizen ever for a moment, or in the least degree, resign his
conscience to the legislator? Why has every man a conscience then? I
think that we should be men first, and subjects afterward. It is not
desirable to cultivate a respect for the law, so much as for the right.
The only obligation which I have a right to assume is to do at any time
what I think right. It is truly enough said that a corporation has no
conscience; but a corporation of conscientious men is a corporation
\emph{with} a conscience. Law never made men a whit more just; and, by
means of their respect for it, even the well-disposed are daily made the
agents on injustice. A common and natural result of an undue respect for
the law is, that you may see a file of soldiers, colonel, captain,
corporal, privates, powder-monkeys, and all, marching in admirable order
over hill and dale to the wars, against their wills, ay, against their
common sense and consciences, which makes it very steep marching indeed,
and produces a palpitation of the heart. They have no doubt that it is a
damnable business in which they are concerned; they are all peaceably
inclined. Now, what are they? Men at all? or small movable forts and
magazines, at the service of some unscrupulous man in power? Visit the
Navy Yard, and behold a marine, such a man as an American government can
make, or such as it can make a man with its black arts---a mere shadow
and reminiscence of humanity, a man laid out alive and standing, and
already, as one may say, buried under arms with funeral accompaniment,
though it may be,
\begin{displayquote}
"Not a drum was heard, not a funeral note,\\
\hspace{+1em}		As his corpse to the rampart we hurried;\\
Not a soldier discharged his farewell shot\\
\hspace{+1em}		O'er the grave where our hero we buried."
\end{displayquote}
The mass of men serve the state thus, not as men mainly, but as
machines, with their bodies. They are the standing army, and the
militia, jailers, constables, posse comitatus, etc. In most cases there
is no free exercise whatever of the judgement or of the moral sense; but
they put themselves on a level with wood and earth and stones; and
wooden men can perhaps be manufactured that will serve the purpose as
well. Such command no more respect than men of straw or a lump of dirt.
They have the same sort of worth only as horses and dogs. Yet such as
these even are commonly esteemed good citizens. Others---as most
legislators, politicians, lawyers, ministers, and office-holders---serve
the state chiefly with their heads; and, as they rarely make any moral
distinctions, they are as likely to serve the devil, without
\emph{intending} it, as God. A very few---as heroes, patriots, martyrs,
reformers in the great sense, and \emph{men}---serve the state with
their consciences also, and so necessarily resist it for the most part;
and they are commonly treated as enemies by it. A wise man will only be
useful as a man, and will not submit to be ``clay,'' and ``stop a hole
to keep the wind away,'' but leave that office to his dust at least:
\begin{displayquote}
"I am too high born to be propertied,\\
To be a second at control,\\
Or useful serving-man and instrument\\
To any sovereign state throughout the world."
\end{displayquote}
\clearpage
He who gives himself entirely to his fellow men appears to them useless
and selfish; but he who gives himself partially to them is pronounced a
benefactor and philanthropist.

How does it become a man to behave toward the American government today?
I answer, that he cannot without disgrace be associated with it. I
cannot for an instant recognize that political organization as \emph{my}
government which is the \emph{slave's} government also.

All men recognize the right of revolution; that is, the right to refuse
allegiance to, and to resist, the government, when its tyranny or its
inefficiency are great and unendurable. But almost all say that such is
not the case now. But such was the case, they think, in the Revolution
of '75. If one were to tell me that this was a bad government because it
taxed certain foreign commodities brought to its ports, it is most
probable that I should not make an ado about it, for I can do without
them. All machines have their friction; and possibly this does enough
good to counter-balance the evil. At any rate, it is a great evil to
make a stir about it. But when the friction comes to have its machine,
and oppression and robbery are organized, I say, let us not have such a
machine any longer. In other words, when a sixth of the population of a
nation which has undertaken to be the refuge of liberty are slaves, and
a whole country is unjustly overrun and conquered by a foreign army, and
subjected to military law, I think that it is not too soon for honest
men to rebel and revolutionize. What makes this duty the more urgent is
that fact that the country so overrun is not our own, but ours is the
invading army.

Paley, a common authority with many on moral questions, in his chapter
on the ``Duty of Submission to Civil Government,'' resolves all civil
obligation into expediency; and he proceeds to say that ``so long as the
interest of the whole society requires it, that is, so long as the
established government cannot be resisted or changed without public
inconvenience, it is the will of God . . . that the established
government be obeyed---and no longer. This principle being admitted, the
justice of every particular case of resistance is reduced to a
computation of the quantity of the danger and grievance on the one side,
and of the probability and expense of redressing it on the other.'' Of
this, he says, every man shall judge for himself. But Paley appears
never to have contemplated those cases to which the rule of expediency
does not apply, in which a people, as well as an individual, must do
justice, cost what it may. If I have unjustly wrested a plank from a
drowning man, I must restore it to him though I drown myself. This,
according to Paley, would be inconvenient. But he that would save his
life, in such a case, shall lose it. This people must cease to hold
slaves, and to make war on Mexico, though it cost them their existence
as a people.

In their practice, nations agree with Paley; but does anyone think that
Massachusetts does exactly what is right at the present crisis?
\begin{displayquote}
"A drab of stat,\\
a cloth-o'-silver slut,\\
To have her train borne up,\\
and her soul trail in the dirt."
\end{displayquote}
Practically speaking, the opponents to a reform in Massachusetts are not
a hundred thousand politicians at the South, but a hundred thousand
merchants and farmers here, who are more interested in commerce and
agriculture than they are in humanity, and are not prepared to do
justice to the slave and to Mexico, \emph{cost what it may}. I quarrel
not with far-off foes, but with those who, near at home, co-operate
with, and do the bidding of, those far away, and without whom the latter
would be harmless. We are accustomed to say, that the mass of men are
unprepared; but improvement is slow, because the few are not as
materially wiser or better than the many. It is not so important that
many should be good as you, as that there be some absolute goodness
somewhere; for that will leaven the whole lump. There are thousands who
are \emph{in opinion} opposed to slavery and to the war, who yet in
effect do nothing to put an end to them; who, esteeming themselves
children of Washington and Franklin, sit down with their hands in their
pockets, and say that they know not what to do, and do nothing; who even
postpone the question of freedom to the question of free trade, and
quietly read the prices-current along with the latest advices from
Mexico, after dinner, and, it may be, fall asleep over them both. What
is the price-current of an honest man and patriot today? They hesitate,
and they regret, and sometimes they petition; but they do nothing in
earnest and with effect. They will wait, well disposed, for others to
remedy the evil, that they may no longer have it to regret. At most,
they give up only a cheap vote, and a feeble countenance and Godspeed,
to the right, as it goes by them. There are nine hundred and ninety-nine
patrons of virtue to one virtuous man. But it is easier to deal with the
real possessor of a thing than with the temporary guardian of it.

All voting is a sort of gaming, like checkers or backgammon, with a
slight moral tinge to it, a playing with right and wrong, with moral
questions; and betting naturally accompanies it. The character of the
voters is not staked. I cast my vote, perchance, as I think right; but I
am not vitally concerned that that right should prevail. I am willing to
leave it to the majority. Its obligation, therefore, never exceeds that
of expediency. Even \emph{voting for the right} is \emph{doing} nothing
for it. It is only expressing to men feebly your desire that it should
prevail. A wise man will not leave the right to the mercy of chance, nor
wish it to prevail through the power of the majority. There is but
little virtue in the action of masses of men. When the majority shall at
length vote for the abolition of slavery, it will be because they are
indifferent to slavery, or because there is but little slavery left to
be abolished by their vote. \emph{They} will then be the only slaves.
Only \emph{his} vote can hasten the abolition of slavery who asserts his
own freedom by his vote.

I hear of a convention to be held at Baltimore, or elsewhere, for the
selection of a candidate for the Presidency, made up chiefly of editors,
and men who are politicians by profession; but I think, what is it to
any independent, intelligent, and respectable man what decision they may
come to? Shall we not have the advantage of this wisdom and honesty,
nevertheless? Can we not count upon some independent votes? Are there
not many individuals in the country who do not attend conventions? But
no: I find that the respectable man, so called, has immediately drifted
from his position, and despairs of his country, when his country has
more reasons to despair of him. He forthwith adopts one of the
candidates thus selected as the only \emph{available} one, thus proving
that he is himself \emph{available} for any purposes of the demagogue.
His vote is of no more worth than that of any unprincipled foreigner or
hireling native, who may have been bought. O for a man who is a man,
and, as my neighbor says, has a bone in his back which you cannot pass
your hand through! Our statistics are at fault: the population has been
returned too large. How many \emph{men} are there to a square thousand
miles in the country? Hardly one. Does not America offer any inducement
for men to settle here? The American has dwindled into an Odd
Fellow---one who may be known by the development of his organ of
gregariousness, and a manifest lack of intellect and cheerful
self-reliance; whose first and chief concern, on coming into the world,
is to see that the almshouses are in good repair; and, before yet he has
lawfully donned the virile garb, to collect a fund to the support of the
widows and orphans that may be; who, in short, ventures to live only by
the aid of the Mutual Insurance company, which has promised to bury him
decently.

It is not a man's duty, as a matter of course, to devote himself to the
eradication of any, even to most enormous wrong; he may still properly
have other concerns to engage him; but it is his duty, at least, to wash
his hands of it, and, if he gives it no thought longer, not to give it
practically his support. If I devote myself to other pursuits and
contemplations, I must first see, at least, that I do not pursue them
sitting upon another man's shoulders. I must get off him first, that he
may pursue his contemplations too. See what gross inconsistency is
tolerated. I have heard some of my townsmen say, ``I should like to have
them order me out to help put down an insurrection of the slaves, or to
march to Mexico---see if I would go''; and yet these very men have each,
directly by their allegiance, and so indirectly, at least, by their
money, furnished a substitute. The soldier is applauded who refuses to
serve in an unjust war by those who do not refuse to sustain the unjust
government which makes the war; is applauded by those whose own act and
authority he disregards and sets at naught; as if the state were
penitent to that degree that it hired one to scourge it while it sinned,
but not to that degree that it left off sinning for a moment. Thus,
under the name of Order and Civil Government, we are all made at last to
pay homage to and support our own meanness. After the first blush of sin
comes its indifference; and from immoral it becomes, as it were,
unmoral, and not quite unnecessary to that life which we have made.

The broadest and most prevalent error requires the most disinterested
virtue to sustain it. The slight reproach to which the virtue of
patriotism is commonly liable, the noble are most likely to incur. Those
who, while they disapprove of the character and measures of a
government, yield to it their allegiance and support are undoubtedly its
most conscientious supporters, and so frequently the most serious
obstacles to reform. Some are petitioning the State to dissolve the
Union, to disregard the requisitions of the President. Why do they not
dissolve it themselves---the union between themselves and the
State---and refuse to pay their quota into its treasury? Do not they
stand in same relation to the State that the State does to the Union?
And have not the same reasons prevented the State from resisting the
Union which have prevented them from resisting the State?

How can a man be satisfied to entertain an opinion merely, and enjoy
\emph{it}? Is there any enjoyment in it, if his opinion is that he is
aggrieved? If you are cheated out of a single dollar by your neighbor,
you do not rest satisfied with knowing you are cheated, or with saying
that you are cheated, or even with petitioning him to pay you your due;
but you take effectual steps at once to obtain the full amount, and see
to it that you are never cheated again. Action from principle, the
perception and the performance of right, changes things and relations;
it is essentially revolutionary, and does not consist wholly with
anything which was. It not only divided States and churches, it divides
families; ay, it divides the \emph{individual}, separating the
diabolical in him from the divine.

Unjust laws exist: shall we be content to obey them, or shall we
endeavor to amend them, and obey them until we have succeeded, or shall
we transgress them at once? Men, generally, under such a government as
this, think that they ought to wait until they have persuaded the
majority to alter them. They think that, if they should resist, the
remedy would be worse than the evil. But it is the fault of the
government itself that the remedy is worse than the evil. \emph{It}
makes it worse. Why is it not more apt to anticipate and provide for
reform? Why does it not cherish its wise minority? Why does it cry and
resist before it is hurt? Why does it not encourage its citizens to put
out its faults, and \emph{do} better than it would have them? Why does
it always crucify Christ and excommunicate Copernicus and Luther, and
pronounce Washington and Franklin rebels?

One would think, that a deliberate and practical denial of its authority
was the only offense never contemplated by its government; else, why has
it not assigned its definite, its suitable and proportionate, penalty?
If a man who has no property refuses but once to earn nine shillings for
the State, he is put in prison for a period unlimited by any law that I
know, and determined only by the discretion of those who put him there;
but if he should steal ninety times nine shillings from the State, he is
soon permitted to go at large again.

If the injustice is part of the necessary friction of the machine of
government, let it go, let it go: perchance it will wear
smooth---certainly the machine will wear out. If the injustice has a
spring, or a pulley, or a rope, or a crank, exclusively for itself, then
perhaps you may consider whether the remedy will not be worse than the
evil; but if it is of such a nature that it requires you to be the agent
of injustice to another, then I say, break the law. Let your life be a
counter-friction to stop the machine. What I have to do is to see, at
any rate, that I do not lend myself to the wrong which I condemn.

As for adopting the ways which the State has provided for remedying the
evil, I know not of such ways. They take too much time, and a man's life
will be gone. I have other affairs to attend to. I came into this world,
not chiefly to make this a good place to live in, but to live in it, be
it good or bad. A man has not everything to do, but something; and
because he cannot do \emph{everything}, it is not necessary that he
should be doing \emph{something} wrong. It is not my business to be
petitioning the Governor or the Legislature any more than it is theirs
to petition me; and if they should not hear my petition, what should I
do then? But in this case the State has provided no way: its very
Constitution is the evil. This may seem to be harsh and stubborn and
unconcilliatory; but it is to treat with the utmost kindness and
consideration the only spirit that can appreciate or deserves it. So is
all change for the better, like birth and death, which convulse the
body.

I do not hesitate to say, that those who call themselves Abolitionists
should at once effectually withdraw their support, both in person and
property, from the government of Massachusetts, and not wait till they
constitute a majority of one, before they suffer the right to prevail
through them. I think that it is enough if they have God on their side,
without waiting for that other one. Moreover, any man more right than
his neighbors constitutes a majority of one already.

I meet this American government, or its representative, the State
government, directly, and face to face, once a year---no more---in the
person of its tax-gatherer; this is the only mode in which a man
situated as I am necessarily meets it; and it then says distinctly,
Recognize me; and the simplest, the most effectual, and, in the present
posture of affairs, the indispensablest mode of treating with it on this
head, of expressing your little satisfaction with and love for it, is to
deny it then. My civil neighbor, the tax-gatherer, is the very man I
have to deal with---for it is, after all, with men and not with
parchment that I quarrel---and he has voluntarily chosen to be an agent
of the government. How shall he ever know well that he is and does as an
officer of the government, or as a man, until he is obliged to consider
whether he will treat me, his neighbor, for whom he has respect, as a
neighbor and well-disposed man, or as a maniac and disturber of the
peace, and see if he can get over this obstruction to his neighborlines
without a ruder and more impetuous thought or speech corresponding with
his action. I know this well, that if one thousand, if one hundred, if
ten men whom I could name---if ten \emph{honest} men only---ay, if
\emph{one} HONEST man, in this State of Massachusetts, \emph{ceasing to
hold slaves}, were actually to withdraw from this co-partnership, and be
locked up in the county jail therefor, it would be the abolition of
slavery in America. For it matters not how small the beginning may seem
to be: what is once well done is done forever. But we love better to
talk about it: that we say is our mission. Reform keeps many scores of
newspapers in its service, but not one man. If my esteemed neighbor, the
State's ambassador, who will devote his days to the settlement of the
question of human rights in the Council Chamber, instead of being
threatened with the prisons of Carolina, were to sit down the prisoner
of Massachusetts, that State which is so anxious to foist the sin of
slavery upon her sister---though at present she can discover only an act
of inhospitality to be the ground of a quarrel with her---the
Legislature would not wholly waive the subject of the following winter.

Under a government which imprisons unjustly, the true place for a just
man is also a prison. The proper place today, the only place which
Massachusetts has provided for her freer and less despondent spirits, is
in her prisons, to be put out and locked out of the State by her own
act, as they have already put themselves out by their principles. It is
there that the fugitive slave, and the Mexican prisoner on parole, and
the Indian come to plead the wrongs of his race should find them; on
that separate but more free and honorable ground, where the State places
those who are not \emph{with} her, but \emph{against} her---the only
house in a slave State in which a free man can abide with honor. If any
think that their influence would be lost there, and their voices no
longer afflict the ear of the State, that they would not be as an enemy
within its walls, they do not know by how much truth is stronger than
error, nor how much more eloquently and effectively he can combat
injustice who has experienced a little in his own person. Cast your
whole vote, not a strip of paper merely, but your whole influence. A
minority is powerless while it conforms to the majority; it is not even
a minority then; but it is irresistible when it clogs by its whole
weight. If the alternative is to keep all just men in prison, or give up
war and slavery, the State will not hesitate which to choose. If a
thousand men were not to pay their tax bills this year, that would not
be a violent and bloody measure, as it would be to pay them, and enable
the State to commit violence and shed innocent blood. This is, in fact,
the definition of a peaceable revolution, if any such is possible. If
the tax-gatherer, or any other public officer, asks me, as one has done,
``But what shall I do?'' my answer is, ``If you really wish to do
anything, resign your office.'' When the subject has refused allegiance,
and the officer has resigned from office, then the revolution is
accomplished. But even suppose blood should flow. Is there not a sort of
blood shed when the conscience is wounded? Through this wound a man's
real manhood and immortality flow out, and he bleeds to an everlasting
death. I see this blood flowing now.

I have contemplated the imprisonment of the offender, rather than the
seizure of his goods---though both will serve the same purpose---because
they who assert the purest right, and consequently are most dangerous to
a corrupt State, commonly have not spent much time in accumulating
property. To such the State renders comparatively small service, and a
slight tax is wont to appear exorbitant, particularly if they are
obliged to earn it by special labor with their hands. If there were one
who lived wholly without the use of money, the State itself would
hesitate to demand it of him. But the rich man---not to make any
invidious comparison---is always sold to the institution which makes him
rich. Absolutely speaking, the more money, the less virtue; for money
comes between a man and his objects, and obtains them for him; it was
certainly no great virtue to obtain it. It puts to rest many questions
which he would otherwise be taxed to answer; while the only new question
which it puts is the hard but superfluous one, how to spend it. Thus his
moral ground is taken from under his feet. The opportunities of living
are diminished in proportion as that are called the ``means'' are
increased. The best thing a man can do for his culture when he is rich
is to endeavor to carry out those schemes which he entertained when he
was poor. Christ answered the Herodians according to their condition.
``Show me the tribute-money,'' said he---and one took a penny out of his
pocket---if you use money which has the image of Caesar on it, and which
he has made current and valuable, that is, \emph{if you are men of the
State}, and gladly enjoy the advantages of Caesar's government, then pay
him back some of his own when he demands it. ``Render therefore to
Caesar that which is Caesar's and to God those things which are
God's''---leaving them no wiser than before as to which was which; for
they did not wish to know.

When I converse with the freest of my neighbors, I perceive that,
whatever they may say about the magnitude and seriousness of the
question, and their regard for the public tranquillity, the long and the
short of the matter is, that they cannot spare the protection of the
existing government, and they dread the consequences to their property
and families of disobedience to it. For my own part, I should not like
to think that I ever rely on the protection of the State. But, if I deny
the authority of the State when it presents its tax bill, it will soon
take and waste all my property, and so harass me and my children without
end. This is hard. This makes it impossible for a man to live honestly,
and at the same time comfortably, in outward respects. It will not be
worth the while to accumulate property; that would be sure to go again.
You must hire or squat somewhere, and raise but a small crop, and eat
that soon. You must live within yourself, and depend upon yourself
always tucked up and ready for a start, and not have many affairs. A man
may grow rich in Turkey even, if he will be in all respects a good
subject of the Turkish government. Confucius said: ``If a state is
governed by the principles of reason, poverty and misery are subjects of
shame; if a state is not governed by the principles of reason, riches
and honors are subjects of shame.'' No: until I want the protection of
Massachusetts to be extended to me in some distant Southern port, where
my liberty is endangered, or until I am bent solely on building up an
estate at home by peaceful enterprise, I can afford to refuse allegiance
to Massachusetts, and her right to my property and life. It costs me
less in every sense to incur the penalty of disobedience to the State
than it would to obey. I should feel as if I were worth less in that
case.

Some years ago, the State met me in behalf of the Church, and commanded
me to pay a certain sum toward the support of a clergyman whose
preaching my father attended, but never I myself. ``Pay,'' it said, ``or
be locked up in the jail.'' I declined to pay. But, unfortunately,
another man saw fit to pay it. I did not see why the schoolmaster should
be taxed to support the priest, and not the priest the schoolmaster; for
I was not the State's schoolmaster, but I supported myself by voluntary
subscription. I did not see why the lyceum should not present its tax
bill, and have the State to back its demand, as well as the Church.
However, at the request of the selectmen, I condescended to make some
such statement as this in writing: ``Know all men by these presents,
that I, Henry Thoreau, do not wish to be regarded as a member of any
incorporated society which I have not joined.'' This I gave to the town
clerk; and he has it. The State, having thus learned that I did not wish
to be regarded as a member of that church, has never made a like demand
on me since; though it said that it must adhere to its original
presumption that time. If I had known how to name them, I should then
have signed off in detail from all the societies which I never signed on
to; but I did not know where to find such a complete list.

I have paid no poll tax for six years. I was put into a jail once on
this account, for one night; and, as I stood considering the walls of
solid stone, two or three feet thick, the door of wood and iron, a foot
thick, and the iron grating which strained the light, I could not help
being struck with the foolishness of that institution which treated me
as if I were mere flesh and blood and bones, to be locked up. I wondered
that it should have concluded at length that this was the best use it
could put me to, and had never thought to avail itself of my services in
some way. I saw that, if there was a wall of stone between me and my
townsmen, there was a still more difficult one to climb or break through
before they could get to be as free as I was. I did nor for a moment
feel confined, and the walls seemed a great waste of stone and mortar. I
felt as if I alone of all my townsmen had paid my tax. They plainly did
not know how to treat me, but behaved like persons who are underbred. In
every threat and in every compliment there was a blunder; for they
thought that my chief desire was to stand the other side of that stone
wall. I could not but smile to see how industriously they locked the
door on my meditations, which followed them out again without let or
hindrance, and \emph{they} were really all that was dangerous. As they
could not reach me, they had resolved to punish my body; just as boys,
if they cannot come at some person against whom they have a spite, will
abuse his dog. I saw that the State was half-witted, that it was timid
as a lone woman with her silver spoons, and that it did not know its
friends from its foes, and I lost all my remaining respect for it, and
pitied it.

Thus the state never intentionally confronts a man's sense, intellectual
or moral, but only his body, his senses. It is not armed with superior
wit or honesty, but with superior physical strength. I was not born to
be forced. I will breathe after my own fashion. Let us see who is the
strongest. What force has a multitude? They only can force me who obey a
higher law than I. They force me to become like themselves. I do not
hear of \emph{men} being \emph{forced} to live this way or that by
masses of men. What sort of life were that to live? When I meet a
government which says to me, ``Your money or your life,'' why should I
be in haste to give it my money? It may be in a great strait, and not
know what to do: I cannot help that. It must help itself; do as I do. It
is not worth the while to snivel about it. I am not responsible for the
successful working of the machinery of society. I am not the son of the
engineer. I perceive that, when an acorn and a chestnut fall side by
side, the one does not remain inert to make way for the other, but both
obey their own laws, and spring and grow and flourish as best they can,
till one, perchance, overshadows and destroys the other. If a plant
cannot live according to nature, it dies; and so a man.

The night in prison was novel and interesting enough. The prisoners in
their shirtsleeves were enjoying a chat and the evening air in the
doorway, when I entered. But the jailer said, ``Come, boys, it is time
to lock up''; and so they dispersed, and I heard the sound of their
steps returning into the hollow apartments. My room-mate was introduced
to me by the jailer as ``a first-rate fellow and clever man.'' When the
door was locked, he showed me where to hang my hat, and how he managed
matters there. The rooms were whitewashed once a month; and this one, at
least, was the whitest, most simply furnished, and probably neatest
apartment in town. He naturally wanted to know where I came from, and
what brought me there; and, when I had told him, I asked him in my turn
how he came there, presuming him to be an honest man, of course; and as
the world goes, I believe he was. ``Why,'' said he, ``they accuse me of
burning a barn; but I never did it.'' As near as I could discover, he
had probably gone to bed in a barn when drunk, and smoked his pipe
there; and so a barn was burnt. He had the reputation of being a clever
man, had been there some three months waiting for his trial to come on,
and would have to wait as much longer; but he was quite domesticated and
contented, since he got his board for nothing, and thought that he was
well treated.

He occupied one window, and I the other; and I saw that if one stayed
there long, his principal business would be to look out the window. I
had soon read all the tracts that were left there, and examined where
former prisoners had broken out, and where a grate had been sawed off,
and heard the history of the various occupants of that room; for I found
that even there there was a history and a gossip which never circulated
beyond the walls of the jail. Probably this is the only house in the
town where verses are composed, which are afterward printed in a
circular form, but not published. I was shown quite a long list of young
men who had been detected in an attempt to escape, who avenged
themselves by singing them.

I pumped my fellow-prisoner as dry as I could, for fear I should never
see him again; but at length he showed me which was my bed, and left me
to blow out the lamp.

It was like travelling into a far country, such as I had never expected
to behold, to lie there for one night. It seemed to me that I never had
heard the town clock strike before, not the evening sounds of the
village; for we slept with the windows open, which were inside the
grating. It was to see my native village in the light of the Middle
Ages, and our Concord was turned into a Rhine stream, and visions of
knights and castles passed before me. They were the voices of old
burghers that I heard in the streets. I was an involuntary spectator and
auditor of whatever was done and said in the kitchen of the adjacent
village inn---a wholly new and rare experience to me. It was a closer
view of my native town. I was fairly inside of it. I never had seen its
institutions before. This is one of its peculiar institutions; for it is
a shire town. I began to comprehend what its inhabitants were about.

In the morning, our breakfasts were put through the hole in the door, in
small oblong-square tin pans, made to fit, and holding a pint of
chocolate, with brown bread, and an iron spoon. When they called for the
vessels again, I was green enough to return what bread I had left, but
my comrade seized it, and said that I should lay that up for lunch or
dinner. Soon after he was let out to work at haying in a neighboring
field, whither he went every day, and would not be back till noon; so he
bade me good day, saying that he doubted if he should see me again.

When I came out of prison---for some one interfered, and paid that
tax---I did not perceive that great changes had taken place on the
common, such as he observed who went in a youth and emerged a
gray-headed man; and yet a change had come to my eyes come over the
scene---the town, and State, and country, greater than any that mere
time could effect. I saw yet more distinctly the State in which I lived.
I saw to what extent the people among whom I lived could be trusted as
good neighbors and friends; that their friendship was for summer weather
only; that they did not greatly propose to do right; that they were a
distinct race from me by their prejudices and superstitions, as the
Chinamen and Malays are; that in their sacrifices to humanity they ran
no risks, not even to their property; that after all they were not so
noble but they treated the thief as he had treated them, and hoped, by a
certain outward observance and a few prayers, and by walking in a
particular straight though useless path from time to time, to save their
souls. This may be to judge my neighbors harshly; for I believe that
many of them are not aware that they have such an institution as the
jail in their village.

It was formerly the custom in our village, when a poor debtor came out
of jail, for his acquaintances to salute him, looking through their
fingers, which were crossed to represent the jail window, ``How do ye
do?'' My neighbors did not thus salute me, but first looked at me, and
then at one another, as if I had returned from a long journey. I was put
into jail as I was going to the shoemaker's to get a shoe which was
mended. When I was let out the next morning, I proceeded to finish my
errand, and, having put on my mended shoe, joined a huckleberry party,
who were impatient to put themselves under my conduct; and in half an
hour---for the horse was soon tackled---was in the midst of a
huckleberry field, on one of our highest hills, two miles off, and then
the State was nowhere to be seen.

This is the whole history of ``My Prisons.''

I have never declined paying the highway tax, because I am as desirous
of being a good neighbor as I am of being a bad subject; and as for
supporting schools, I am doing my part to educate my fellow countrymen
now. It is for no particular item in the tax bill that I refuse to pay
it. I simply wish to refuse allegiance to the State, to withdraw and
stand aloof from it effectually. I do not care to trace the course of my
dollar, if I could, till it buys a man or a musket to shoot one
with---the dollar is innocent---but I am concerned to trace the effects
of my allegiance. In fact, I quietly declare war with the State, after
my fashion, though I will still make use and get what advantages of her
I can, as is usual in such cases.

If others pay the tax which is demanded of me, from a sympathy with the
State, they do but what they have already done in their own case, or
rather they abet injustice to a greater extent than the State requires.
If they pay the tax from a mistaken interest in the individual taxed, to
save his property, or prevent his going to jail, it is because they have
not considered wisely how far they let their private feelings interfere
with the public good.

This, then, is my position at present. But one cannot be too much on his
guard in such a case, lest his actions be biased by obstinacy or an
undue regard for the opinions of men. Let him see that he does only what
belongs to himself and to the hour.

I think sometimes, Why, this people mean well, they are only ignorant;
they would do better if they knew how: why give your neighbors this pain
to treat you as they are not inclined to? But I think again, This is no
reason why I should do as they do, or permit others to suffer much
greater pain of a different kind. Again, I sometimes say to myself, When
many millions of men, without heat, without ill will, without personal
feelings of any kind, demand of you a few shillings only, without the
possibility, such is their constitution, of retracting or altering their
present demand, and without the possibility, on your side, of appeal to
any other millions, why expose yourself to this overwhelming brute
force? You do not resist cold and hunger, the winds and the waves, thus
obstinately; you quietly submit to a thousand similar necessities. You
do not put your head into the fire. But just in proportion as I regard
this as not wholly a brute force, but partly a human force, and consider
that I have relations to those millions as to so many millions of men,
and not of mere brute or inanimate things, I see that appeal is
possible, first and instantaneously, from them to the Maker of them,
and, secondly, from them to themselves. But if I put my head
deliberately into the fire, there is no appeal to fire or to the Maker
of fire, and I have only myself to blame. If I could convince myself
that I have any right to be satisfied with men as they are, and to treat
them accordingly, and not according, in some respects, to my
requisitions and expectations of what they and I ought to be, then, like
a good Mussulman and fatalist, I should endeavor to be satisfied with
things as they are, and say it is the will of God. And, above all, there
is this difference between resisting this and a purely brute or natural
force, that I can resist this with some effect; but I cannot expect,
like Orpheus, to change the nature of the rocks and trees and beasts.

I do not wish to quarrel with any man or nation. I do not wish to split
hairs, to make fine distinctions, or set myself up as better than my
neighbors. I seek rather, I may say, even an excuse for conforming to
the laws of the land. I am but too ready to conform to them. Indeed, I
have reason to suspect myself on this head; and each year, as the
tax-gatherer comes round, I find myself disposed to review the acts and
position of the general and State governments, and the spirit of the
people to discover a pretext for conformity.
\begin{displayquote}
"We must affect our country as our parents,\\
And if at any time we alienate\\
Out love of industry from doing it honor,\\
We must respect effects and teach the soul\\
Matter of conscience and religion,\\
And not desire of rule or benefit."\\
\end{displayquote}
I believe that the State will soon be able to take all my work of this
sort out of my hands, and then I shall be no better patriot than my
fellow-countrymen. Seen from a lower point of view, the Constitution,
with all its faults, is very good; the law and the courts are very
respectable; even this State and this American government are, in many
respects, very admirable, and rare things, to be thankful for, such as a
great many have described them; seen from a higher still, and the
highest, who shall say what they are, or that they are worth looking at
or thinking of at all?

However, the government does not concern me much, and I shall bestow the
fewest possible thoughts on it. It is not many moments that I live under
a government, even in this world. If a man is thought-free, fancy-free,
imagination-free, that which \emph{is not} never for a long time
appearing \emph{to be} to him, unwise rulers or reformers cannot fatally
interrupt him.

I know that most men think differently from myself; but those whose
lives are by profession devoted to the study of these or kindred
subjects content me as little as any. Statesmen and legislators,
standing so completely within the institution, never distinctly and
nakedly behold it. They speak of moving society, but have no
resting-place without it. They may be men of a certain experience and
discrimination, and have no doubt invented ingenious and even useful
systems, for which we sincerely thank them; but all their wit and
usefulness lie within certain not very wide limits. They are wont to
forget that the world is not governed by policy and expediency. Webster
never goes behind government, and so cannot speak with authority about
it. His words are wisdom to those legislators who contemplate no
essential reform in the existing government; but for thinkers, and those
who legislate for all time, he never once glances at the subject. I know
of those whose serene and wise speculations on this theme would soon
reveal the limits of his mind's range and hospitality. Yet, compared
with the cheap professions of most reformers, and the still cheaper
wisdom an eloquence of politicians in general, his are almost the only
sensible and valuable words, and we thank Heaven for him. Comparatively,
he is always strong, original, and, above all, practical. Still, his
quality is not wisdom, but prudence. The lawyer's truth is not Truth,
but consistency or a consistent expediency. Truth is always in harmony
with herself, and is not concerned chiefly to reveal the justice that
may consist with wrong-doing. He well deserves to be called, as he has
been called, the Defender of the Constitution. There are really no blows
to be given him but defensive ones. He is not a leader, but a follower.
His leaders are the men of '87. ``I have never made an effort,'' he
says, ``and never propose to make an effort; I have never countenanced
an effort, and never mean to countenance an effort, to disturb the
arrangement as originally made, by which various States came into the
Union.'' Still thinking of the sanction which the Constitution gives to
slavery, he says, ``Because it was part of the original compact---let it
stand.'' Notwithstanding his special acuteness and ability, he is unable
to take a fact out of its merely political relations, and behold it as
it lies absolutely to be disposed of by the intellect---what, for
instance, it behooves a man to do here in America today with regard to
slavery---but ventures, or is driven, to make some such desperate answer
to the following, while professing to speak absolutely, and as a private
man---from which what new and singular of social duties might be
inferred? ``The manner,'' says he, ``in which the governments of the
States where slavery exists are to regulate it is for their own
consideration, under the responsibility to their constituents, to the
general laws of propriety, humanity, and justice, and to God.
Associations formed elsewhere, springing from a feeling of humanity, or
any other cause, have nothing whatever to do with it. They have never
received any encouragement from me and they never will.''  {[}\textit{These
extracts have been inserted since the lecture was read -HDT}]

They who know of no purer sources of truth, who have traced up its
stream no higher, stand, and wisely stand, by the Bible and the
Constitution, and drink at it there with reverence and humanity; but
they who behold where it comes trickling into this lake or that pool,
gird up their loins once more, and continue their pilgrimage toward its
fountainhead.

No man with a genius for legislation has appeared in America. They are
rare in the history of the world. There are orators, politicians, and
eloquent men, by the thousand; but the speaker has not yet opened his
mouth to speak who is capable of settling the much-vexed questions of
the day. We love eloquence for its own sake, and not for any truth which
it may utter, or any heroism it may inspire. Our legislators have not
yet learned the comparative value of free trade and of freedom, of
union, and of rectitude, to a nation. They have no genius or talent for
comparatively humble questions of taxation and finance, commerce and
manufactures and agriculture. If we were left solely to the wordy wit of
legislators in Congress for our guidance, uncorrected by the seasonable
experience and the effectual complaints of the people, America would not
long retain her rank among the nations. For eighteen hundred years,
though perchance I have no right to say it, the New Testament has been
written; yet where is the legislator who has wisdom and practical talent
enough to avail himself of the light which it sheds on the science of
legislation.

The authority of government, even such as I am willing to submit
to---for I will cheerfully obey those who know and can do better than I,
and in many things even those who neither know nor can do so well---is
still an impure one: to be strictly just, it must have the sanction and
consent of the governed. It can have no pure right over my person and
property but what I concede to it. The progress from an absolute to a
limited monarchy, from a limited monarchy to a democracy, is a progress
toward a true respect for the individual. Even the Chinese philosopher
was wise enough to regard the individual as the basis of the empire. Is
a democracy, such as we know it, the last improvement possible in
government? Is it not possible to take a step further towards
recognizing and organizing the rights of man? There will never be a
really free and enlightened State until the State comes to recognize the
individual as a higher and independent power, from which all its own
power and authority are derived, and treats him accordingly. I please
myself with imagining a State at last which can afford to be just to all
men, and to treat the individual with respect as a neighbor; which even
would not think it inconsistent with its own repose if a few were to
live aloof from it, not meddling with it, nor embraced by it, who
fulfilled all the duties of neighbors and fellow men. A State which bore
this kind of fruit, and suffered it to drop off as fast as it ripened,
would prepare the way for a still more perfect and glorious State, which
I have also imagined, but not yet anywhere seen.



\end{document}
