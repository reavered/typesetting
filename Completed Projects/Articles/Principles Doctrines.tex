\documentclass[12pt]{article}
\usepackage[12pt]{moresize}
\usepackage[utf8]{inputenc}
\usepackage[english]{babel}
\usepackage[margin=2.5cm]{geometry}
\usepackage{ebgaramond}

%=======PARAGRAPH FORMATTING=========%
\setlength{\parindent}{0pt} %no paragraph indents
\setlength{\parskip}{1em}   %single space between paragraphs
\usepackage{enumitem}

%=======TITLE PAGE=========%
\title{\vspace{-2cm}\bfseries{Principal Doctrines}\vspace{-8mm}}
\author{by Epicurus}
\date{\vspace{-1em} \normalsize Translated by Robert Drew Hicks \vspace{-1em}}

\begin{document}

\maketitle
\begin{enumerate}[leftmargin=.5cm]
	\item A happy and eternal being has no trouble himself and brings no trouble upon any other being; hence he is exempt from movements of anger and partiality, for every such movement implies weakness 

	\item Death is nothing to us; for the body, when it has been resolved into its elements, has no feeling, and that which has no feeling is nothing to us. 

	\item The magnitude of pleasure reaches its limit in the removal of all pain. When pleasure is present, so long as it is uninterrupted, there is no pain either of body or of mind or of both together. 

	\item Continuous pain does not last long in the body; on the contrary, pain, if extreme, is present a short time, and even that degree of pain which barely outweighs pleasure in the body does not last for many days together. Illnesses of long duration even permit of an excess of pleasure over pain in the body. 
	
	\item It is impossible to live a pleasant life without living wisely and well and justly, and it is impossible to live wisely and well and justly without living pleasantly. Whenever any one of these is lacking, when, for instance, the person is not able to live wisely, though he lives well and justly, it is impossible for him to live a pleasant life. 

	\item In order to obtain security from other people any means whatever of procuring this was a natural good. 

	\item Some people have sought to become famous and renowned, thinking that thus they would make themselves secure against their fellow-humans. If, then, the life of such persons really was secure, they attained natural good; if, however, it was insecure, they have not attained the end which by nature's own prompting they originally sought. 

	\item No pleasure is in itself evil, but the things which produce certain pleasures entail annoyances many times greater than the pleasures themselves. 

	\item If all pleasure had been capable of accumulation, -- if this had gone on not only be recurrences in time, but all over the frame or, at any rate, over the principal parts of human nature, there would never have been any difference between one pleasure and another, as in fact there is.

	\item If the objects which are productive of pleasures to profligate persons really freed them from fears of the mind, -- the fears, I mean, inspired by celestial and atmospheric phenomena, the fear of death, the fear of pain; if, further, they taught them to limit their desires, we should never have any fault to find with such persons, for they would then be filled with pleasures to overflowing on all sides and would be exempt from all pain, whether of body or mind, that is, from all evil. 

	\item If we had never been molested by alarms at celestial and atmospheric phenomena, nor by the misgiving that death somehow affects us, nor by neglect of the proper limits of pains and desires, we should have had no need to study natural science. 

	\item It would be impossible to banish fear on matters of the highest importance, if a person did not know the nature of the whole universe, but lived in dread of what the legends tell us. Hence without the study of nature there was no enjoyment of unmixed pleasures. 

	\item There would be no advantage in providing security against our fellow humans, so long as we were alarmed by occurrences over our heads or beneath the earth or in general by whatever happens in the boundless universe. 

	\item When tolerable security against our fellow humans is attained, then on a basis of power sufficient to afford supports and of material prosperity arises in most genuine form the security of a quiet private life withdrawn from the multitude. 

	\item Nature's wealth at once has its bounds and is easy to procure; but the wealth of vain fancies recedes to an infinite distance. 

	\item Fortune but seldom interferes with the wise person; his greatest and highest interests have been, are, and will be, directed by reason throughout the course of his life. 

	\item The just person enjoys. the greatest peace of mind, while the unjust is full of the utmost disquietude. 

	\item Pleasure in the body admits no increase when once the pain of want has been removed; after that it only admits of variation. The limit of pleasure in the mind, however, is reached when we reflect on the things themselves and their congeners which cause the mind the greatest alarms. 

	\item Unlimited time and limited time afford an equal amount of pleasure, if we measure the limits of that pleasure by reason. 

	\item The body receives as unlimited the limits of pleasure; and to provide it requires unlimited time. But the mind, grasping in thought what the end and limit of the body is, and banishing the terrors of futurity, procures a complete and perfect life, and has no longer any need of unlimited time. Nevertheless it does not shun pleasure, and even in the hour of death, when ushered out of existence by circumstances, the mind does not lack enjoyment of the best life. 

	\item He who understands the limits of life knows how easy it is to procure enough to remove the pain of want and make the whole of life complete and perfect. Hence he has no longer any need of things which are not to be won save by labor and conflict. 

	\item We must take into account as the end all that really exists and all clear evidence of sense to which we refer our opinions; for otherwise everything will be full of uncertainty and confusion. 

	\item If you fight against all your sensations, you will have no standard to which to refer, and thus no means of judging even those judgments which you pronounce false. 

	\item If you reject absolutely any single sensation without stopping to discriminate with respect to that which awaits confirmation between matter of opinion and that which is already present, whether in sensation or in feelings or in any immediate perception of the mind, you will throw into confusion even the rest of your sensations by your groundless belief and so you will be rejecting the standard of truth altogether. If in your ideas based upon opinion you hastily affirm as true all that awaits confirmation as well as that which does not, you will not escape error, as you will be maintaining complete ambiguity whenever it is a case of judging between right and wrong opinion. 

	\item If you do not on every separate occasion refer each of your actions to the end prescribed by nature, but instead of this in the act of choice or avoidance swerve aside to some other end, your acts will not be consistent with your theories. 

	\item All such desires as lead to no pain when they remain ungratified are unnecessary, and the longing is easily got rid of, when the thing desired is difficult to procure or when the desires seem likely to produce harm. 

	\item Of all the means which are procured by wisdom to ensure happiness throughout the whole of life, by far the most important is the acquisition of friends. 

	\item The same conviction which inspires confidence that nothing we have to fear is eternal or even of long duration, also enables us to see that even in our limited conditions of life nothing enhances our security so much as friendship. 

	\item Of our desires some are natural and necessary others are natural, but not necessary; others, again, are neither natural nor necessary, but are due to illusory opinion. 

	\item Those natural desires which entail no pain when not gratified, though their objects are vehemently pursued, are also due to illusory opinion; and when they are not got rid of, it is not because of their own nature, but because of the person's illusory opinion. 

	\item Natural justice is a symbol or expression of usefullness, to prevent one person from harming or being harmed by another. 

	\item Those animals which are incapable of making covenants with one another, to the end that they may neither inflict nor suffer harm, are without either justice or injustice. And those tribes which either could not or would not form mutual covenants to the same end are in like case.

	\item There never was an absolute justice, but only an agreement made in reciprocal association in whatever localities now and again from time to time, providing against the infliction or suffering of harm. 

	\item Injustice is not in itself an evil, but only in its consequence, viz. the terror which is excited by apprehension that those appointed to punish such offenses will discover the injustice. 

	\item It is impossible for the person who secretly violates any article of the social compact to feel confident that he will remain undiscovered, even if he has already escaped ten thousand times; for right on to the end of his life he is never sure he will not be detected. 
	
	\item Taken generally, justice is the same for all, to wit, something found useful in mutual association; but in its application to particular cases of locality or conditions of whatever kind, it varies under different circumstances. 

	\item Among the things accounted just by conventional law, whatever in the needs of mutual association is attested to be useful, is thereby stamped as just, whether or not it be the same for all; and in case any law is made and does not prove suitable to the usefulness of mutual association, then this is no longer just. And should the usefulness which is expressed by the law vary and only for a time correspond with the prior conception, nevertheless for the time being it was just, so long as we do not trouble ourselves about empty words, but look simply at the facts. 
	
	\item Where without any change in circumstances the conventional laws, when judged by their consequences, were seen not to correspond with the notion of justice, such laws were not really just; but wherever the laws have ceased to be useful in consequence of a change in circumstances, in that case the laws were for the time being just when they were useful for the mutual association of the citizens, and subsequently ceased to be just when they ceased to be useful. 

	\item He who best knew how to meet fear of external foes made into one family all the creatures he could; and those he could not, he at any rate did not treat as aliens; and where he found even this impossible, he avoided all association, and, so far as was useful, kept them at a distance. 

	\item Those who were best able to provide themselves with the means of security against their neighbors, being thus in possession of the surest guarantee, passed the most agreeable life in each other's society; and their enjoyment of the fullest intimacy was such that, if one of them died before his time, the survivors did not mourn his death as if it called for sympathy. 
\end{enumerate}


\end{document}