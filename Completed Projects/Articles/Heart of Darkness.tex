\documentclass[12pt]{report}
\usepackage[12pt]{moresize}
\usepackage[utf8]{inputenc}
\usepackage[english]{babel}
\usepackage[top=2.5cm, bottom=2.5cm, left=2.5cm, right=2.5cm]{geometry}
\usepackage{ebgaramond}

%=======SECTION HEADERS=========%
\usepackage{titlesec}
\usepackage{titletoc}

%========QUOTES=========%
\usepackage{epigraph}
\usepackage[autostyle, english = american]{csquotes}

%=======PARAGRAPH FORMATTING=========%
\setlength{\parindent}{0pt} %no paragraph indents
\setlength{\parskip}{1em}   %single space between paragraphs

\renewcommand{\chaptermark}[1]{\markboth{\MakeUppercase{Book \thechapter}}{}} %Book format- heading

%=======CHAPTER FORMATTING=========%
\renewcommand\thesection{{\arabic{section}}}   %section numbering style

\titleformat
{\chapter} 
[display]
{\centering\fontfamily{ppl}\Huge} 
{\thechapter} 
{\leftmargin}{}[]

\newcommand{\mychapter}[2]{
\setcounter{chapter}{#1}
    \setcounter{section}{0}
    \chapter*{#2}
    \addcontentsline{toc}{chapter}{#2}
}

%=======SECTION HEADER SPACING=========%
\titlespacing{\chapter}{0mm}{-2em}{0em}
\titlespacing{\section}{0mm}{3mm}{2mm}

%=======TITLE PAGE=========%
\title{\HUGE\bfseries{Heart of Darkness}}
\author{\Large by Joseph Conrad}
\date{\vspace{-4mm} }

%=======FOOTNOTES=========%
\renewcommand{\thefootnote}{[\arabic{footnote}]}
\setlength{\skip\footins}{1cm}
\usepackage[]{footmisc}
\renewcommand{\footnotemargin}{3mm} %Setting left margin
\renewcommand{\footnotelayout}{\hspace{2mm}} %spacing between the footnote number and the text of footnote

\usepackage{hyperref}
\hypersetup{bookmarksnumbered} %Bookmarks are numbered in the ToC when converted to PDF or EPUB

\begin{document}

\begin{titlepage}
    \maketitle
\end{titlepage}

%=======MAIN DOCUMENT=========%
\titlespacing{\chapter}{0mm}{-2em}{1em}
\mychapter{1}{I}
The Nellie, a cruising yawl, swung to her anchor without a flutter of
the sails, and was at rest. The flood had made, the wind was nearly
calm, and being bound down the river, the only thing for it was to come
to and wait for the turn of the tide.

The sea-reach of the Thames stretched before us like the beginning of an
interminable waterway. In the offing the sea and the sky were welded
together without a joint, and in the luminous space the tanned sails of
the barges drifting up with the tide seemed to stand still in red
clusters of canvas sharply peaked, with gleams of varnished sprits. A
haze rested on the low shores that ran out to sea in vanishing flatness.
The air was dark above Gravesend, and farther back still seemed
condensed into a mournful gloom, brooding motionless over the biggest,
and the greatest, town on earth.

The Director of Companies was our captain and our host. We four
affectionately watched his back as he stood in the bows looking to
seaward. On the whole river there was nothing that looked half so
nautical. He resembled a pilot, which to a seaman is trustworthiness
personified. It was difficult to realize his work was not out there in
the luminous estuary, but behind him, within the brooding gloom.

Between us there was, as I have already said somewhere, the bond of the
sea. Besides holding our hearts together through long periods of
separation, it had the effect of making us tolerant of each other's
yarns---and even convictions. The Lawyer---the best of old
fellows---had, because of his many years and many virtues, the only
cushion on deck, and was lying on the only rug. The Accountant had
brought out already a box of dominoes, and was toying architecturally
with the bones. Marlow sat cross-legged right aft, leaning against the
mizzen-mast. He had sunken cheeks, a yellow complexion, a straight back,
an ascetic aspect, and, with his arms dropped, the palms of hands
outwards, resembled an idol. The director, satisfied the anchor had good
hold, made his way aft and sat down amongst us. We exchanged a few words
lazily. Afterwards there was silence on board the yacht. For some reason
or other we did not begin that game of dominoes. We felt meditative, and
fit for nothing but placid staring. The day was ending in a serenity of
still and exquisite brilliance. The water shone pacifically; the sky,
without a speck, was a benign immensity of unstained light; the very
mist on the Essex marsh was like a gauzy and radiant fabric, hung from
the wooded rises inland, and draping the low shores in diaphanous folds.
Only the gloom to the west, brooding over the upper reaches, became more
sombre every minute, as if angered by the approach of the sun.

And at last, in its curved and imperceptible fall, the sun sank low, and
from glowing white changed to a dull red without rays and without heat,
as if about to go out suddenly, stricken to death by the touch of that
gloom brooding over a crowd of men.

Forthwith a change came over the waters, and the serenity became less
brilliant but more profound. The old river in its broad reach rested
unruffled at the decline of day, after ages of good service done to the
race that peopled its banks, spread out in the tranquil dignity of a
waterway leading to the uttermost ends of the earth. We looked at the
venerable stream not in the vivid flush of a short day that comes and
departs for ever, but in the august light of abiding memories. And
indeed nothing is easier for a man who has, as the phrase goes,
``followed the sea'' with reverence and affection, than to evoke the
great spirit of the past upon the lower reaches of the Thames. The tidal
current runs to and fro in its unceasing service, crowded with memories
of men and ships it had borne to the rest of home or to the battles of
the sea. It had known and served all the men of whom the nation is
proud, from Sir Francis Drake to Sir John Franklin, knights all, titled
and untitled---the great knights-errant of the sea. It had borne all the
ships whose names are like jewels flashing in the night of time, from
the \emph{Golden Hind} returning with her rotund flanks full of
treasure, to be visited by the Queen's Highness and thus pass out of the
gigantic tale, to the \emph{Erebus} and \emph{Terror}, bound on other
conquests---and that never returned. It had known the ships and the men.
They had sailed from Deptford, from Greenwich, from Erith---the
adventurers and the settlers; kings' ships and the ships of men on
`Change; captains, admirals, the dark ``interlopers'' of the Eastern
trade, and the commissioned ``generals'' of East India fleets. Hunters
for gold or pursuers of fame, they all had gone out on that stream,
bearing the sword, and often the torch, messengers of the might within
the land, bearers of a spark from the sacred fire. What greatness had
not floated on the ebb of that river into the mystery of an unknown
earth!\ldots{} The dreams of men, the seed of commonwealths, the germs
of empires.

The sun set; the dusk fell on the stream, and lights began to appear
along the shore. The Chapman light-house, a three-legged thing erect on
a mud-flat, shone strongly. Lights of ships moved in the fairway---a
great stir of lights going up and going down. And farther west on the
upper reaches the place of the monstrous town was still marked ominously
on the sky, a brooding gloom in sunshine, a lurid glare under the stars.

``And this also,'' said Marlow suddenly, ``has been one of the dark
places of the earth.''

He was the only man of us who still ``followed the sea.'' The worst that
could be said of him was that he did not represent his class. He was a
seaman, but he was a wanderer, too, while most seamen lead, if one may
so express it, a sedentary life. Their minds are of the stay-at-home
order, and their home is always with them---the ship; and so is their
country---the sea. One ship is very much like another, and the sea is
always the same. In the immutability of their surroundings the foreign
shores, the foreign faces, the changing immensity of life, glide past,
veiled not by a sense of mystery but by a slightly disdainful ignorance;
for there is nothing mysterious to a seaman unless it be the sea itself,
which is the mistress of his existence and as inscrutable as Destiny.
For the rest, after his hours of work, a casual stroll or a casual spree
on shore suffices to unfold for him the secret of a whole continent, and
generally he finds the secret not worth knowing. The yarns of seamen
have a direct simplicity, the whole meaning of which lies within the
shell of a cracked nut. But Marlow was not typical (if his propensity to
spin yarns be excepted), and to him the meaning of an episode was not
inside like a kernel but outside, enveloping the tale which brought it
out only as a glow brings out a haze, in the likeness of one of these
misty halos that sometimes are made visible by the spectral illumination
of moonshine.

His remark did not seem at all surprising. It was just like Marlow. It
was accepted in silence. No one took the trouble to grunt even; and
presently he said, very slow---``I was thinking of very old times, when
the Romans first came here, nineteen hundred years ago---the other day
\ldots{}. Light came out of this river since---you say Knights? Yes; but
it is like a running blaze on a plain, like a flash of lightning in the
clouds. We live in the flicker---may it last as long as the old earth
keeps rolling! But darkness was here yesterday. Imagine the feelings of
a commander of a fine---what d'ye call `em?---trireme in the
Mediterranean, ordered suddenly to the north; run overland across the
Gauls in a hurry; put in charge of one of these craft the
legionaries---a wonderful lot of handy men they must have been,
too---used to build, apparently by the hundred, in a month or two, if we
may believe what we read. Imagine him here---the very end of the world,
a sea the colour of lead, a sky the colour of smoke, a kind of ship
about as rigid as a concertina---and going up this river with stores, or
orders, or what you like. Sand-banks, marshes, forests,
savages,---precious little to eat fit for a civilized man, nothing but
Thames water to drink. No Falernian wine here, no going ashore. Here and
there a military camp lost in a wilderness, like a needle in a bundle of
hay---cold, fog, tempests, disease, exile, and death---death skulking in
the air, in the water, in the bush. They must have been dying like flies
here. Oh, yes---he did it. Did it very well, too, no doubt, and without
thinking much about it either, except afterwards to brag of what he had
gone through in his time, perhaps. They were men enough to face the
darkness. And perhaps he was cheered by keeping his eye on a chance of
promotion to the fleet at Ravenna by and by, if he had good friends in
Rome and survived the awful climate. Or think of a decent young citizen
in a toga---perhaps too much dice, you know---coming out here in the
train of some prefect, or tax-gatherer, or trader even, to mend his
fortunes. Land in a swamp, march through the woods, and in some inland
post feel the savagery, the utter savagery, had closed round him---all
that mysterious life of the wilderness that stirs in the forest, in the
jungles, in the hearts of wild men. There's no initiation either into
such mysteries. He has to live in the midst of the incomprehensible,
which is also detestable. And it has a fascination, too, that goes to
work upon him. The fascination of the abomination---you know, imagine
the growing regrets, the longing to escape, the powerless disgust, the
surrender, the hate.''

He paused.

``Mind,'' he began again, lifting one arm from the elbow, the palm of
the hand outwards, so that, with his legs folded before him, he had the
pose of a Buddha preaching in European clothes and without a
lotus-flower---``Mind, none of us would feel exactly like this. What
saves us is efficiency---the devotion to efficiency. But these chaps
were not much account, really. They were no colonists; their
administration was merely a squeeze, and nothing more, I suspect. They
were conquerors, and for that you want only brute force---nothing to
boast of, when you have it, since your strength is just an accident
arising from the weakness of others. They grabbed what they could get
for the sake of what was to be got. It was just robbery with violence,
aggravated murder on a great scale, and men going at it blind---as is
very proper for those who tackle a darkness. The conquest of the earth,
which mostly means the taking it away from those who have a different
complexion or slightly flatter noses than ourselves, is not a pretty
thing when you look into it too much. What redeems it is the idea only.
An idea at the back of it; not a sentimental pretence but an idea; and
an unselfish belief in the idea---something you can set up, and bow down
before, and offer a sacrifice to\ldots{}.''

He broke off. Flames glided in the river, small green flames, red
flames, white flames, pursuing, overtaking, joining, crossing each
other---then separating slowly or hastily. The traffic of the great city
went on in the deepening night upon the sleepless river. We looked on,
waiting patiently---there was nothing else to do till the end of the
flood; but it was only after a long silence, when he said, in a
hesitating voice, ``I suppose you fellows remember I did once turn
fresh-water sailor for a bit,'' that we knew we were fated, before the
ebb began to run, to hear about one of Marlow's inconclusive
experiences.

``I don't want to bother you much with what happened to me personally,''
he began, showing in this remark the weakness of many tellers of tales
who seem so often unaware of what their audience would like best to
hear; ``yet to understand the effect of it on me you ought to know how I
got out there, what I saw, how I went up that river to the place where I
first met the poor chap. It was the farthest point of navigation and the
culminating point of my experience. It seemed somehow to throw a kind of
light on everything about me---and into my thoughts. It was sombre
enough, too---and pitiful---not extraordinary in any way---not very
clear either. No, not very clear. And yet it seemed to throw a kind of
light.

``I had then, as you remember, just returned to London after a lot of
Indian Ocean, Pacific, China Seas---a regular dose of the East---six
years or so, and I was loafing about, hindering you fellows in your work
and invading your homes, just as though I had got a heavenly mission to
civilize you. It was very fine for a time, but after a bit I did get
tired of resting. Then I began to look for a ship---I should think the
hardest work on earth. But the ships wouldn't even look at me. And I got
tired of that game, too.

``Now when I was a little chap I had a passion for maps. I would look
for hours at South America, or Africa, or Australia, and lose myself in
all the glories of exploration. At that time there were many blank
spaces on the earth, and when I saw one that looked particularly
inviting on a map (but they all look that) I would put my finger on it
and say, `When I grow up I will go there.' The North Pole was one of
these places, I remember. Well, I haven't been there yet, and shall not
try now. The glamour's off. Other places were scattered about the
hemispheres. I have been in some of them, and\ldots{} well, we won't
talk about that. But there was one yet---the biggest, the most blank, so
to speak---that I had a hankering after.

``True, by this time it was not a blank space any more. It had got
filled since my boyhood with rivers and lakes and names. It had ceased
to be a blank space of delightful mystery---a white patch for a boy to
dream gloriously over. It had become a place of darkness. But there was
in it one river especially, a mighty big river, that you could see on
the map, resembling an immense snake uncoiled, with its head in the sea,
its body at rest curving afar over a vast country, and its tail lost in
the depths of the land. And as I looked at the map of it in a
shop-window, it fascinated me as a snake would a bird---a silly little
bird. Then I remembered there was a big concern, a Company for trade on
that river. Dash it all! I thought to myself, they can't trade without
using some kind of craft on that lot of fresh water---steamboats! Why
shouldn't I try to get charge of one? I went on along Fleet Street, but
could not shake off the idea. The snake had charmed me.

``You understand it was a Continental concern, that Trading society; but
I have a lot of relations living on the Continent, because it's cheap
and not so nasty as it looks, they say.

``I am sorry to own I began to worry them. This was already a fresh
departure for me. I was not used to get things that way, you know. I
always went my own road and on my own legs where I had a mind to go. I
wouldn't have believed it of myself; but, then---you see---I felt
somehow I must get there by hook or by crook. So I worried them. The men
said `My dear fellow,' and did nothing. Then---would you believe it?---I
tried the women. I, Charlie Marlow, set the women to work---to get a
job. Heavens! Well, you see, the notion drove me. I had an aunt, a dear
enthusiastic soul. She wrote: `It will be delightful. I am ready to do
anything, anything for you. It is a glorious idea. I know the wife of a
very high personage in the Administration, and also a man who has lots
of influence with,' etc. She was determined to make no end of fuss to
get me appointed skipper of a river steamboat, if such was my fancy.

``I got my appointment---of course; and I got it very quick. It appears
the Company had received news that one of their captains had been killed
in a scuffle with the natives. This was my chance, and it made me the
more anxious to go. It was only months and months afterwards, when I
made the attempt to recover what was left of the body, that I heard the
original quarrel arose from a misunderstanding about some hens. Yes, two
black hens. Fresleven---that was the fellow's name, a Dane---thought
himself wronged somehow in the bargain, so he went ashore and started to
hammer the chief of the village with a stick. Oh, it didn't surprise me
in the least to hear this, and at the same time to be told that
Fresleven was the gentlest, quietest creature that ever walked on two
legs. No doubt he was; but he had been a couple of years already out
there engaged in the noble cause, you know, and he probably felt the
need at last of asserting his self-respect in some way. Therefore he
whacked the old nigger mercilessly, while a big crowd of his people
watched him, thunderstruck, till some man---I was told the chief's
son---in desperation at hearing the old chap yell, made a tentative jab
with a spear at the white man---and of course it went quite easy between
the shoulder-blades. Then the whole population cleared into the forest,
expecting all kinds of calamities to happen, while, on the other hand,
the steamer Fresleven commanded left also in a bad panic, in charge of
the engineer, I believe. Afterwards nobody seemed to trouble much about
Fresleven's remains, till I got out and stepped into his shoes. I
couldn't let it rest, though; but when an opportunity offered at last to
meet my predecessor, the grass growing through his ribs was tall enough
to hide his bones. They were all there. The supernatural being had not
been touched after he fell. And the village was deserted, the huts gaped
black, rotting, all askew within the fallen enclosures. A calamity had
come to it, sure enough. The people had vanished. Mad terror had
scattered them, men, women, and children, through the bush, and they had
never returned. What became of the hens I don't know either. I should
think the cause of progress got them, anyhow. However, through this
glorious affair I got my appointment, before I had fairly begun to hope
for it.

``I flew around like mad to get ready, and before forty-eight hours I
was crossing the Channel to show myself to my employers, and sign the
contract. In a very few hours I arrived in a city that always makes me
think of a whited sepulchre. Prejudice no doubt. I had no difficulty in
finding the Company's offices. It was the biggest thing in the town, and
everybody I met was full of it. They were going to run an over-sea
empire, and make no end of coin by trade.

``A narrow and deserted street in deep shadow, high houses, innumerable
windows with venetian blinds, a dead silence, grass sprouting right and
left, immense double doors standing ponderously ajar. I slipped through
one of these cracks, went up a swept and ungarnished staircase, as arid
as a desert, and opened the first door I came to. Two women, one fat and
the other slim, sat on straw-bottomed chairs, knitting black wool. The
slim one got up and walked straight at me---still knitting with downcast
eyes---and only just as I began to think of getting out of her way, as
you would for a somnambulist, stood still, and looked up. Her dress was
as plain as an umbrella-cover, and she turned round without a word and
preceded me into a waiting-room. I gave my name, and looked about. Deal
table in the middle, plain chairs all round the walls, on one end a
large shining map, marked with all the colours of a rainbow. There was a
vast amount of red---good to see at any time, because one knows that
some real work is done in there, a deuce of a lot of blue, a little
green, smears of orange, and, on the East Coast, a purple patch, to show
where the jolly pioneers of progress drink the jolly lager-beer.
However, I wasn't going into any of these. I was going into the yellow.
Dead in the centre. And the river was
there---fascinating---deadly---like a snake. Ough! A door opened, a
white-haired secretarial head, but wearing a compassionate expression,
appeared, and a skinny forefinger beckoned me into the sanctuary. Its
light was dim, and a heavy writing-desk squatted in the middle. From
behind that structure came out an impression of pale plumpness in a
frock-coat. The great man himself. He was five feet six, I should judge,
and had his grip on the handle-end of ever so many millions. He shook
hands, I fancy, murmured vaguely, was satisfied with my French.
\emph{Bon Voyage}.

``In about forty-five seconds I found myself again in the waiting-room
with the compassionate secretary, who, full of desolation and sympathy,
made me sign some document. I believe I undertook amongst other things
not to disclose any trade secrets. Well, I am not going to.

``I began to feel slightly uneasy. You know I am not used to such
ceremonies, and there was something ominous in the atmosphere. It was
just as though I had been let into some conspiracy---I don't
know---something not quite right; and I was glad to get out. In the
outer room the two women knitted black wool feverishly. People were
arriving, and the younger one was walking back and forth introducing
them. The old one sat on her chair. Her flat cloth slippers were propped
up on a foot-warmer, and a cat reposed on her lap. She wore a starched
white affair on her head, had a wart on one cheek, and silver-rimmed
spectacles hung on the tip of her nose. She glanced at me above the
glasses. The swift and indifferent placidity of that look troubled me.
Two youths with foolish and cheery countenances were being piloted over,
and she threw at them the same quick glance of unconcerned wisdom. She
seemed to know all about them and about me, too. An eerie feeling came
over me. She seemed uncanny and fateful. Often far away there I thought
of these two, guarding the door of Darkness, knitting black wool as for
a warm pall, one introducing, introducing continuously to the unknown,
the other scrutinizing the cheery and foolish faces with unconcerned old
eyes. \emph{Ave!} Old knitter of black wool. \emph{Morituri te
salutant}. Not many of those she looked at ever saw her again---not
half, by a long way.

``There was yet a visit to the doctor. `A simple formality,' assured me
the secretary, with an air of taking an immense part in all my sorrows.
Accordingly a young chap wearing his hat over the left eyebrow, some
clerk I suppose---there must have been clerks in the business, though
the house was as still as a house in a city of the dead---came from
somewhere up-stairs, and led me forth. He was shabby and careless, with
inkstains on the sleeves of his jacket, and his cravat was large and
billowy, under a chin shaped like the toe of an old boot. It was a
little too early for the doctor, so I proposed a drink, and thereupon he
developed a vein of joviality. As we sat over our vermouths he glorified
the Company's business, and by and by I expressed casually my surprise
at him not going out there. He became very cool and collected all at
once. `I am not such a fool as I look, quoth Plato to his disciples,' he
said sententiously, emptied his glass with great resolution, and we
rose.

``The old doctor felt my pulse, evidently thinking of something else the
while. `Good, good for there,' he mumbled, and then with a certain
eagerness asked me whether I would let him measure my head. Rather
surprised, I said Yes, when he produced a thing like calipers and got
the dimensions back and front and every way, taking notes carefully. He
was an unshaven little man in a threadbare coat like a gaberdine, with
his feet in slippers, and I thought him a harmless fool. `I always ask
leave, in the interests of science, to measure the crania of those going
out there,' he said. `And when they come back, too?' I asked. `Oh, I
never see them,' he remarked; `and, moreover, the changes take place
inside, you know.' He smiled, as if at some quiet joke. `So you are
going out there. Famous. Interesting, too.' He gave me a searching
glance, and made another note. `Ever any madness in your family?' he
asked, in a matter-of-fact tone. I felt very annoyed. `Is that question
in the interests of science, too?' `It would be,' he said, without
taking notice of my irritation, `interesting for science to watch the
mental changes of individuals, on the spot, but\ldots{}' `Are you an
alienist?' I interrupted. `Every doctor should be---a little,' answered
that original, imperturbably. `I have a little theory which you
messieurs who go out there must help me to prove. This is my share in
the advantages my country shall reap from the possession of such a
magnificent dependency. The mere wealth I leave to others. Pardon my
questions, but you are the first Englishman coming under my
observation\ldots{}' I hastened to assure him I was not in the least
typical. `If I were,' said I, `I wouldn't be talking like this with
you.' `What you say is rather profound, and probably erroneous,' he
said, with a laugh. `Avoid irritation more than exposure to the sun.
\emph{Adieu}. How do you English say, eh? Good-bye. Ah! Good-bye.
\emph{Adieu}. In the tropics one must before everything keep
calm.'\ldots{} He lifted a warning forefinger\ldots{}. `\emph{Du calme,
du calme}.'

``One thing more remained to do---say good-bye to my excellent aunt. I
found her triumphant. I had a cup of tea---the last decent cup of tea
for many days---and in a room that most soothingly looked just as you
would expect a lady's drawing-room to look, we had a long quiet chat by
the fireside. In the course of these confidences it became quite plain
to me I had been represented to the wife of the high dignitary, and
goodness knows to how many more people besides, as an exceptional and
gifted creature---a piece of good fortune for the Company---a man you
don't get hold of every day. Good heavens! and I was going to take
charge of a two-penny-half-penny river-steamboat with a penny whistle
attached! It appeared, however, I was also one of the Workers, with a
capital---you know. Something like an emissary of light, something like
a lower sort of apostle. There had been a lot of such rot let loose in
print and talk just about that time, and the excellent woman, living
right in the rush of all that humbug, got carried off her feet. She
talked about `weaning those ignorant millions from their horrid ways,'
till, upon my word, she made me quite uncomfortable. I ventured to hint
that the Company was run for profit.

```You forget, dear Charlie, that the labourer is worthy of his hire,'
she said, brightly. It's queer how out of touch with truth women are.
They live in a world of their own, and there has never been anything
like it, and never can be. It is too beautiful altogether, and if they
were to set it up it would go to pieces before the first sunset. Some
confounded fact we men have been living contentedly with ever since the
day of creation would start up and knock the whole thing over.

``After this I got embraced, told to wear flannel, be sure to write
often, and so on---and I left. In the street---I don't know why---a
queer feeling came to me that I was an imposter. Odd thing that I, who
used to clear out for any part of the world at twenty-four hours'
notice, with less thought than most men give to the crossing of a
street, had a moment---I won't say of hesitation, but of startled pause,
before this commonplace affair. The best way I can explain it to you is
by saying that, for a second or two, I felt as though, instead of going
to the centre of a continent, I were about to set off for the centre of
the earth.

``I left in a French steamer, and she called in every blamed port they
have out there, for, as far as I could see, the sole purpose of landing
soldiers and custom-house officers. I watched the coast. Watching a
coast as it slips by the ship is like thinking about an enigma. There it
is before you---smiling, frowning, inviting, grand, mean, insipid, or
savage, and always mute with an air of whispering, `Come and find out.'
This one was almost featureless, as if still in the making, with an
aspect of monotonous grimness. The edge of a colossal jungle, so
dark-green as to be almost black, fringed with white surf, ran straight,
like a ruled line, far, far away along a blue sea whose glitter was
blurred by a creeping mist. The sun was fierce, the land seemed to
glisten and drip with steam. Here and there greyish-whitish specks
showed up clustered inside the white surf, with a flag flying above them
perhaps. Settlements some centuries old, and still no bigger than
pinheads on the untouched expanse of their background. We pounded along,
stopped, landed soldiers; went on, landed custom-house clerks to levy
toll in what looked like a God-forsaken wilderness, with a tin shed and
a flag-pole lost in it; landed more soldiers---to take care of the
custom-house clerks, presumably. Some, I heard, got drowned in the surf;
but whether they did or not, nobody seemed particularly to care. They
were just flung out there, and on we went. Every day the coast looked
the same, as though we had not moved; but we passed various
places---trading places---with names like Gran' Bassam, Little Popo;
names that seemed to belong to some sordid farce acted in front of a
sinister back-cloth. The idleness of a passenger, my isolation amongst
all these men with whom I had no point of contact, the oily and languid
sea, the uniform sombreness of the coast, seemed to keep me away from
the truth of things, within the toil of a mournful and senseless
delusion. The voice of the surf heard now and then was a positive
pleasure, like the speech of a brother. It was something natural, that
had its reason, that had a meaning. Now and then a boat from the shore
gave one a momentary contact with reality. It was paddled by black
fellows. You could see from afar the white of their eyeballs glistening.
They shouted, sang; their bodies streamed with perspiration; they had
faces like grotesque masks---these chaps; but they had bone, muscle, a
wild vitality, an intense energy of movement, that was as natural and
true as the surf along their coast. They wanted no excuse for being
there. They were a great comfort to look at. For a time I would feel I
belonged still to a world of straightforward facts; but the feeling
would not last long. Something would turn up to scare it away. Once, I
remember, we came upon a man-of-war anchored off the coast. There wasn't
even a shed there, and she was shelling the bush. It appears the French
had one of their wars going on thereabouts. Her ensign dropped limp like
a rag; the muzzles of the long six-inch guns stuck out all over the low
hull; the greasy, slimy swell swung her up lazily and let her down,
swaying her thin masts. In the empty immensity of earth, sky, and water,
there she was, incomprehensible, firing into a continent. Pop, would go
one of the six-inch guns; a small flame would dart and vanish, a little
white smoke would disappear, a tiny projectile would give a feeble
screech---and nothing happened. Nothing could happen. There was a touch
of insanity in the proceeding, a sense of lugubrious drollery in the
sight; and it was not dissipated by somebody on board assuring me
earnestly there was a camp of natives---he called them enemies!---hidden
out of sight somewhere.

``We gave her her letters (I heard the men in that lonely ship were
dying of fever at the rate of three a day) and went on. We called at
some more places with farcical names, where the merry dance of death and
trade goes on in a still and earthy atmosphere as of an overheated
catacomb; all along the formless coast bordered by dangerous surf, as if
Nature herself had tried to ward off intruders; in and out of rivers,
streams of death in life, whose banks were rotting into mud, whose
waters, thickened into slime, invaded the contorted mangroves, that
seemed to writhe at us in the extremity of an impotent despair. Nowhere
did we stop long enough to get a particularized impression, but the
general sense of vague and oppressive wonder grew upon me. It was like a
weary pilgrimage amongst hints for nightmares.

``It was upward of thirty days before I saw the mouth of the big river.
We anchored off the seat of the government. But my work would not begin
till some two hundred miles farther on. So as soon as I could I made a
start for a place thirty miles higher up.

``I had my passage on a little sea-going steamer. Her captain was a
Swede, and knowing me for a seaman, invited me on the bridge. He was a
young man, lean, fair, and morose, with lanky hair and a shuffling gait.
As we left the miserable little wharf, he tossed his head contemptuously
at the shore. `Been living there?' he asked. I said, `Yes.' `Fine lot
these government chaps---are they not?' he went on, speaking English
with great precision and considerable bitterness. `It is funny what some
people will do for a few francs a month. I wonder what becomes of that
kind when it goes upcountry?' I said to him I expected to see that soon.
`So-o-o!' he exclaimed. He shuffled athwart, keeping one eye ahead
vigilantly. `Don't be too sure,' he continued. `The other day I took up
a man who hanged himself on the road. He was a Swede, too.' `Hanged
himself! Why, in God's name?' I cried. He kept on looking out
watchfully. `Who knows? The sun too much for him, or the country
perhaps.'

``At last we opened a reach. A rocky cliff appeared, mounds of turned-up
earth by the shore, houses on a hill, others with iron roofs, amongst a
waste of excavations, or hanging to the declivity. A continuous noise of
the rapids above hovered over this scene of inhabited devastation. A lot
of people, mostly black and naked, moved about like ants. A jetty
projected into the river. A blinding sunlight drowned all this at times
in a sudden recrudescence of glare. `There's your Company's station,'
said the Swede, pointing to three wooden barrack-like structures on the
rocky slope. `I will send your things up. Four boxes did you say? So.
Farewell.'

``I came upon a boiler wallowing in the grass, then found a path leading
up the hill. It turned aside for the boulders, and also for an
undersized railway-truck lying there on its back with its wheels in the
air. One was off. The thing looked as dead as the carcass of some
animal. I came upon more pieces of decaying machinery, a stack of rusty
rails. To the left a clump of trees made a shady spot, where dark things
seemed to stir feebly. I blinked, the path was steep. A horn tooted to
the right, and I saw the black people run. A heavy and dull detonation
shook the ground, a puff of smoke came out of the cliff, and that was
all. No change appeared on the face of the rock. They were building a
railway. The cliff was not in the way or anything; but this objectless
blasting was all the work going on.

``A slight clinking behind me made me turn my head. Six black men
advanced in a file, toiling up the path. They walked erect and slow,
balancing small baskets full of earth on their heads, and the clink kept
time with their footsteps. Black rags were wound round their loins, and
the short ends behind waggled to and fro like tails. I could see every
rib, the joints of their limbs were like knots in a rope; each had an
iron collar on his neck, and all were connected together with a chain
whose bights swung between them, rhythmically clinking. Another report
from the cliff made me think suddenly of that ship of war I had seen
firing into a continent. It was the same kind of ominous voice; but
these men could by no stretch of imagination be called enemies. They
were called criminals, and the outraged law, like the bursting shells,
had come to them, an insoluble mystery from the sea. All their meagre
breasts panted together, the violently dilated nostrils quivered, the
eyes stared stonily uphill. They passed me within six inches, without a
glance, with that complete, deathlike indifference of unhappy savages.
Behind this raw matter one of the reclaimed, the product of the new
forces at work, strolled despondently, carrying a rifle by its middle.
He had a uniform jacket with one button off, and seeing a white man on
the path, hoisted his weapon to his shoulder with alacrity. This was
simple prudence, white men being so much alike at a distance that he
could not tell who I might be. He was speedily reassured, and with a
large, white, rascally grin, and a glance at his charge, seemed to take
me into partnership in his exalted trust. After all, I also was a part
of the great cause of these high and just proceedings.

``Instead of going up, I turned and descended to the left. My idea was
to let that chain-gang get out of sight before I climbed the hill. You
know I am not particularly tender; I've had to strike and to fend off.
I've had to resist and to attack sometimes---that's only one way of
resisting---without counting the exact cost, according to the demands of
such sort of life as I had blundered into. I've seen the devil of
violence, and the devil of greed, and the devil of hot desire; but, by
all the stars! these were strong, lusty, red-eyed devils, that swayed
and drove men---men, I tell you. But as I stood on this hillside, I
foresaw that in the blinding sunshine of that land I would become
acquainted with a flabby, pretending, weak-eyed devil of a rapacious and
pitiless folly. How insidious he could be, too, I was only to find out
several months later and a thousand miles farther. For a moment I stood
appalled, as though by a warning. Finally I descended the hill,
obliquely, towards the trees I had seen.

``I avoided a vast artificial hole somebody had been digging on the
slope, the purpose of which I found it impossible to divine. It wasn't a
quarry or a sandpit, anyhow. It was just a hole. It might have been
connected with the philanthropic desire of giving the criminals
something to do. I don't know. Then I nearly fell into a very narrow
ravine, almost no more than a scar in the hillside. I discovered that a
lot of imported drainage-pipes for the settlement had been tumbled in
there. There wasn't one that was not broken. It was a wanton smash-up.
At last I got under the trees. My purpose was to stroll into the shade
for a moment; but no sooner within than it seemed to me I had stepped
into the gloomy circle of some Inferno. The rapids were near, and an
uninterrupted, uniform, headlong, rushing noise filled the mournful
stillness of the grove, where not a breath stirred, not a leaf moved,
with a mysterious sound---as though the tearing pace of the launched
earth had suddenly become audible.

``Black shapes crouched, lay, sat between the trees leaning against the
trunks, clinging to the earth, half coming out, half effaced within the
dim light, in all the attitudes of pain, abandonment, and despair.
Another mine on the cliff went off, followed by a slight shudder of the
soil under my feet. The work was going on. The work! And this was the
place where some of the helpers had withdrawn to die.

``They were dying slowly---it was very clear. They were not enemies,
they were not criminals, they were nothing earthly now---nothing but
black shadows of disease and starvation, lying confusedly in the
greenish gloom. Brought from all the recesses of the coast in all the
legality of time contracts, lost in uncongenial surroundings, fed on
unfamiliar food, they sickened, became inefficient, and were then
allowed to crawl away and rest. These moribund shapes were free as
air---and nearly as thin. I began to distinguish the gleam of the eyes
under the trees. Then, glancing down, I saw a face near my hand. The
black bones reclined at full length with one shoulder against the tree,
and slowly the eyelids rose and the sunken eyes looked up at me,
enormous and vacant, a kind of blind, white flicker in the depths of the
orbs, which died out slowly. The man seemed young---almost a boy---but
you know with them it's hard to tell. I found nothing else to do but to
offer him one of my good Swede's ship's biscuits I had in my pocket. The
fingers closed slowly on it and held---there was no other movement and
no other glance. He had tied a bit of white worsted round his
neck---Why? Where did he get it? Was it a badge---an ornament---a
charm---a propitiatory act? Was there any idea at all connected with it?
It looked startling round his black neck, this bit of white thread from
beyond the seas.

``Near the same tree two more bundles of acute angles sat with their
legs drawn up. One, with his chin propped on his knees, stared at
nothing, in an intolerable and appalling manner: his brother phantom
rested its forehead, as if overcome with a great weariness; and all
about others were scattered in every pose of contorted collapse, as in
some picture of a massacre or a pestilence. While I stood horror-struck,
one of these creatures rose to his hands and knees, and went off on
all-fours towards the river to drink. He lapped out of his hand, then
sat up in the sunlight, crossing his shins in front of him, and after a
time let his woolly head fall on his breastbone.

``I didn't want any more loitering in the shade, and I made haste
towards the station. When near the buildings I met a white man, in such
an unexpected elegance of get-up that in the first moment I took him for
a sort of vision. I saw a high starched collar, white cuffs, a light
alpaca jacket, snowy trousers, a clean necktie, and varnished boots. No
hat. Hair parted, brushed, oiled, under a green-lined parasol held in a
big white hand. He was amazing, and had a penholder behind his ear.

``I shook hands with this miracle, and I learned he was the Company's
chief accountant, and that all the book-keeping was done at this
station. He had come out for a moment, he said, `to get a breath of
fresh air. The expression sounded wonderfully odd, with its suggestion
of sedentary desk-life. I wouldn't have mentioned the fellow to you at
all, only it was from his lips that I first heard the name of the man
who is so indissolubly connected with the memories of that time.
Moreover, I respected the fellow. Yes; I respected his collars, his vast
cuffs, his brushed hair. His appearance was certainly that of a
hairdresser's dummy; but in the great demoralization of the land he kept
up his appearance. That's backbone. His starched collars and got-up
shirt-fronts were achievements of character. He had been out nearly
three years; and, later, I could not help asking him how he managed to
sport such linen. He had just the faintest blush, and said modestly,
`I've been teaching one of the native women about the station. It was
difficult. She had a distaste for the work.' Thus this man had verily
accomplished something. And he was devoted to his books, which were in
apple-pie order.

``Everything else in the station was in a muddle---heads, things,
buildings. Strings of dusty niggers with splay feet arrived and
departed; a stream of manufactured goods, rubbishy cottons, beads, and
brass-wire sent into the depths of darkness, and in return came a
precious trickle of ivory.

``I had to wait in the station for ten days---an eternity. I lived in a
hut in the yard, but to be out of the chaos I would sometimes get into
the accountant's office. It was built of horizontal planks, and so badly
put together that, as he bent over his high desk, he was barred from
neck to heels with narrow strips of sunlight. There was no need to open
the big shutter to see. It was hot there, too; big flies buzzed
fiendishly, and did not sting, but stabbed. I sat generally on the
floor, while, of faultless appearance (and even slightly scented),
perching on a high stool, he wrote, he wrote. Sometimes he stood up for
exercise. When a truckle-bed with a sick man (some invalid agent from
upcountry) was put in there, he exhibited a gentle annoyance. `The
groans of this sick person,' he said, `distract my attention. And
without that it is extremely difficult to guard against clerical errors
in this climate.'

``One day he remarked, without lifting his head, `In the interior you
will no doubt meet Mr. Kurtz.' On my asking who Mr. Kurtz was, he said
he was a first-class agent; and seeing my disappointment at this
information, he added slowly, laying down his pen, `He is a very
remarkable person.' Further questions elicited from him that Mr. Kurtz
was at present in charge of a trading-post, a very important one, in the
true ivory-country, at `the very bottom of there. Sends in as much ivory
as all the others put together\ldots{}' He began to write again. The
sick man was too ill to groan. The flies buzzed in a great peace.

``Suddenly there was a growing murmur of voices and a great tramping of
feet. A caravan had come in. A violent babble of uncouth sounds burst
out on the other side of the planks. All the carriers were speaking
together, and in the midst of the uproar the lamentable voice of the
chief agent was heard `giving it up' tearfully for the twentieth time
that day\ldots{}. He rose slowly. `What a frightful row,' he said. He
crossed the room gently to look at the sick man, and returning, said to
me, `He does not hear.' `What! Dead?' I asked, startled. `No, not yet,'
he answered, with great composure. Then, alluding with a toss of the
head to the tumult in the station-yard, `When one has got to make
correct entries, one comes to hate those savages---hate them to the
death.' He remained thoughtful for a moment. `When you see Mr. Kurtz' he
went on, `tell him from me that everything here'---he glanced at the
deck---' is very satisfactory. I don't like to write to him---with those
messengers of ours you never know who may get hold of your letter---at
that Central Station.' He stared at me for a moment with his mild,
bulging eyes. `Oh, he will go far, very far,' he began again. `He will
be a somebody in the Administration before long. They, above---the
Council in Europe, you know---mean him to be.'

``He turned to his work. The noise outside had ceased, and presently in
going out I stopped at the door. In the steady buzz of flies the
homeward-bound agent was lying finished and insensible; the other, bent
over his books, was making correct entries of perfectly correct
transactions; and fifty feet below the doorstep I could see the still
tree-tops of the grove of death.

``Next day I left that station at last, with a caravan of sixty men, for
a two-hundred-mile tramp.

``No use telling you much about that. Paths, paths, everywhere; a
stamped-in network of paths spreading over the empty land, through the
long grass, through burnt grass, through thickets, down and up chilly
ravines, up and down stony hills ablaze with heat; and a solitude, a
solitude, nobody, not a hut. The population had cleared out a long time
ago. Well, if a lot of mysterious niggers armed with all kinds of
fearful weapons suddenly took to travelling on the road between Deal and
Gravesend, catching the yokels right and left to carry heavy loads for
them, I fancy every farm and cottage thereabouts would get empty very
soon. Only here the dwellings were gone, too. Still I passed through
several abandoned villages. There's something pathetically childish in
the ruins of grass walls. Day after day, with the stamp and shuffle of
sixty pair of bare feet behind me, each pair under a 60-lb. load. Camp,
cook, sleep, strike camp, march. Now and then a carrier dead in harness,
at rest in the long grass near the path, with an empty water-gourd and
his long staff lying by his side. A great silence around and above.
Perhaps on some quiet night the tremor of far-off drums, sinking,
swelling, a tremor vast, faint; a sound weird, appealing, suggestive,
and wild---and perhaps with as profound a meaning as the sound of bells
in a Christian country. Once a white man in an unbuttoned uniform,
camping on the path with an armed escort of lank Zanzibaris, very
hospitable and festive---not to say drunk. Was looking after the upkeep
of the road, he declared. Can't say I saw any road or any upkeep, unless
the body of a middle-aged negro, with a bullet-hole in the forehead,
upon which I absolutely stumbled three miles farther on, may be
considered as a permanent improvement. I had a white companion, too, not
a bad chap, but rather too fleshy and with the exasperating habit of
fainting on the hot hillsides, miles away from the least bit of shade
and water. Annoying, you know, to hold your own coat like a parasol over
a man's head while he is coming to. I couldn't help asking him once what
he meant by coming there at all. `To make money, of course. What do you
think?' he said, scornfully. Then he got fever, and had to be carried in
a hammock slung under a pole. As he weighed sixteen stone I had no end
of rows with the carriers. They jibbed, ran away, sneaked off with their
loads in the night---quite a mutiny. So, one evening, I made a speech in
English with gestures, not one of which was lost to the sixty pairs of
eyes before me, and the next morning I started the hammock off in front
all right. An hour afterwards I came upon the whole concern wrecked in a
bush---man, hammock, groans, blankets, horrors. The heavy pole had
skinned his poor nose. He was very anxious for me to kill somebody, but
there wasn't the shadow of a carrier near. I remembered the old
doctor---`It would be interesting for science to watch the mental
changes of individuals, on the spot.' I felt I was becoming
scientifically interesting. However, all that is to no purpose. On the
fifteenth day I came in sight of the big river again, and hobbled into
the Central Station. It was on a back water surrounded by scrub and
forest, with a pretty border of smelly mud on one side, and on the three
others enclosed by a crazy fence of rushes. A neglected gap was all the
gate it had, and the first glance at the place was enough to let you see
the flabby devil was running that show. White men with long staves in
their hands appeared languidly from amongst the buildings, strolling up
to take a look at me, and then retired out of sight somewhere. One of
them, a stout, excitable chap with black moustaches, informed me with
great volubility and many digressions, as soon as I told him who I was,
that my steamer was at the bottom of the river. I was thunderstruck.
What, how, why? Oh, it was `all right.' The `manager himself' was there.
All quite correct. `Everybody had behaved splendidly!
splendidly!'---`you must,' he said in agitation, `go and see the general
manager at once. He is waiting!'

``I did not see the real significance of that wreck at once. I fancy I
see it now, but I am not sure---not at all. Certainly the affair was too
stupid---when I think of it---to be altogether natural. Still\ldots{}
But at the moment it presented itself simply as a confounded nuisance.
The steamer was sunk. They had started two days before in a sudden hurry
up the river with the manager on board, in charge of some volunteer
skipper, and before they had been out three hours they tore the bottom
out of her on stones, and she sank near the south bank. I asked myself
what I was to do there, now my boat was lost. As a matter of fact, I had
plenty to do in fishing my command out of the river. I had to set about
it the very next day. That, and the repairs when I brought the pieces to
the station, took some months.

``My first interview with the manager was curious. He did not ask me to
sit down after my twenty-mile walk that morning. He was commonplace in
complexion, in features, in manners, and in voice. He was of middle size
and of ordinary build. His eyes, of the usual blue, were perhaps
remarkably cold, and he certainly could make his glance fall on one as
trenchant and heavy as an axe. But even at these times the rest of his
person seemed to disclaim the intention. Otherwise there was only an
indefinable, faint expression of his lips, something stealthy---a
smile---not a smile---I remember it, but I can't explain. It was
unconscious, this smile was, though just after he had said something it
got intensified for an instant. It came at the end of his speeches like
a seal applied on the words to make the meaning of the commonest phrase
appear absolutely inscrutable. He was a common trader, from his youth up
employed in these parts---nothing more. He was obeyed, yet he inspired
neither love nor fear, nor even respect. He inspired uneasiness. That
was it! Uneasiness. Not a definite mistrust---just uneasiness---nothing
more. You have no idea how effective such a\ldots{} a\ldots{} faculty
can be. He had no genius for organizing, for initiative, or for order
even. That was evident in such things as the deplorable state of the
station. He had no learning, and no intelligence. His position had come
to him---why? Perhaps because he was never ill\ldots{} He had served
three terms of three years out there\ldots{} Because triumphant health
in the general rout of constitutions is a kind of power in itself. When
he went home on leave he rioted on a large scale---pompously. Jack
ashore---with a difference---in externals only. This one could gather
from his casual talk. He originated nothing, he could keep the routine
going---that's all. But he was great. He was great by this little thing
that it was impossible to tell what could control such a man. He never
gave that secret away. Perhaps there was nothing within him. Such a
suspicion made one pause---for out there there were no external checks.
Once when various tropical diseases had laid low almost every `agent' in
the station, he was heard to say, `Men who come out here should have no
entrails.' He sealed the utterance with that smile of his, as though it
had been a door opening into a darkness he had in his keeping. You
fancied you had seen things---but the seal was on. When annoyed at
meal-times by the constant quarrels of the white men about precedence,
he ordered an immense round table to be made, for which a special house
had to be built. This was the station's mess-room. Where he sat was the
first place---the rest were nowhere. One felt this to be his unalterable
conviction. He was neither civil nor uncivil. He was quiet. He allowed
his `boy'---an overfed young negro from the coast---to treat the white
men, under his very eyes, with provoking insolence.

``He began to speak as soon as he saw me. I had been very long on the
road. He could not wait. Had to start without me. The up-river stations
had to be relieved. There had been so many delays already that he did
not know who was dead and who was alive, and how they got on---and so
on, and so on. He paid no attention to my explanations, and, playing
with a stick of sealing-wax, repeated several times that the situation
was `very grave, very grave.' There were rumours that a very important
station was in jeopardy, and its chief, Mr. Kurtz, was ill. Hoped it was
not true. Mr. Kurtz was\ldots{} I felt weary and irritable. Hang Kurtz,
I thought. I interrupted him by saying I had heard of Mr. Kurtz on the
coast. `Ah! So they talk of him down there,' he murmured to himself.
Then he began again, assuring me Mr. Kurtz was the best agent he had, an
exceptional man, of the greatest importance to the Company; therefore I
could understand his anxiety. He was, he said, `very, very uneasy.'
Certainly he fidgeted on his chair a good deal, exclaimed, `Ah, Mr.
Kurtz!' broke the stick of sealing-wax and seemed dumfounded by the
accident. Next thing he wanted to know `how long it would take
to'\ldots{} I interrupted him again. Being hungry, you know, and kept on
my feet too. I was getting savage. `How can I tell?' I said. `I haven't
even seen the wreck yet---some months, no doubt.' All this talk seemed
to me so futile. `Some months,' he said. `Well, let us say three months
before we can make a start. Yes. That ought to do the affair.' I flung
out of his hut (he lived all alone in a clay hut with a sort of
verandah) muttering to myself my opinion of him. He was a chattering
idiot. Afterwards I took it back when it was borne in upon me
startlingly with what extreme nicety he had estimated the time requisite
for the `affair.'

``I went to work the next day, turning, so to speak, my back on that
station. In that way only it seemed to me I could keep my hold on the
redeeming facts of life. Still, one must look about sometimes; and then
I saw this station, these men strolling aimlessly about in the sunshine
of the yard. I asked myself sometimes what it all meant. They wandered
here and there with their absurd long staves in their hands, like a lot
of faithless pilgrims bewitched inside a rotten fence. The word `ivory'
rang in the air, was whispered, was sighed. You would think they were
praying to it. A taint of imbecile rapacity blew through it all, like a
whiff from some corpse. By Jove! I've never seen anything so unreal in
my life. And outside, the silent wilderness surrounding this cleared
speck on the earth struck me as something great and invincible, like
evil or truth, waiting patiently for the passing away of this fantastic
invasion.

``Oh, these months! Well, never mind. Various things happened. One
evening a grass shed full of calico, cotton prints, beads, and I don't
know what else, burst into a blaze so suddenly that you would have
thought the earth had opened to let an avenging fire consume all that
trash. I was smoking my pipe quietly by my dismantled steamer, and saw
them all cutting capers in the light, with their arms lifted high, when
the stout man with moustaches came tearing down to the river, a tin pail
in his hand, assured me that everybody was `behaving splendidly,
splendidly,' dipped about a quart of water and tore back again. I
noticed there was a hole in the bottom of his pail.

``I strolled up. There was no hurry. You see the thing had gone off like
a box of matches. It had been hopeless from the very first. The flame
had leaped high, driven everybody back, lighted up everything---and
collapsed. The shed was already a heap of embers glowing fiercely. A
nigger was being beaten near by. They said he had caused the fire in
some way; be that as it may, he was screeching most horribly. I saw him,
later, for several days, sitting in a bit of shade looking very sick and
trying to recover himself; afterwards he arose and went out---and the
wilderness without a sound took him into its bosom again. As I
approached the glow from the dark I found myself at the back of two men,
talking. I heard the name of Kurtz pronounced, then the words, `take
advantage of this unfortunate accident.' One of the men was the manager.
I wished him a good evening. `Did you ever see anything like it---eh? it
is incredible,' he said, and walked off. The other man remained. He was
a first-class agent, young, gentlemanly, a bit reserved, with a forked
little beard and a hooked nose. He was stand-offish with the other
agents, and they on their side said he was the manager's spy upon them.
As to me, I had hardly ever spoken to him before. We got into talk, and
by and by we strolled away from the hissing ruins. Then he asked me to
his room, which was in the main building of the station. He struck a
match, and I perceived that this young aristocrat had not only a
silver-mounted dressing-case but also a whole candle all to himself.
Just at that time the manager was the only man supposed to have any
right to candles. Native mats covered the clay walls; a collection of
spears, assegais, shields, knives was hung up in trophies. The business
intrusted to this fellow was the making of bricks---so I had been
informed; but there wasn't a fragment of a brick anywhere in the
station, and he had been there more than a year---waiting. It seems he
could not make bricks without something, I don't know what---straw
maybe. Anyway, it could not be found there and as it was not likely to
be sent from Europe, it did not appear clear to me what he was waiting
for. An act of special creation perhaps. However, they were all
waiting---all the sixteen or twenty pilgrims of them---for something;
and upon my word it did not seem an uncongenial occupation, from the way
they took it, though the only thing that ever came to them was
disease---as far as I could see. They beguiled the time by back-biting
and intriguing against each other in a foolish kind of way. There was an
air of plotting about that station, but nothing came of it, of course.
It was as unreal as everything else---as the philanthropic pretence of
the whole concern, as their talk, as their government, as their show of
work. The only real feeling was a desire to get appointed to a
trading-post where ivory was to be had, so that they could earn
percentages. They intrigued and slandered and hated each other only on
that account---but as to effectually lifting a little finger---oh, no.
By heavens! there is something after all in the world allowing one man
to steal a horse while another must not look at a halter. Steal a horse
straight out. Very well. He has done it. Perhaps he can ride. But there
is a way of looking at a halter that would provoke the most charitable
of saints into a kick.

``I had no idea why he wanted to be sociable, but as we chatted in there
it suddenly occurred to me the fellow was trying to get at
something---in fact, pumping me. He alluded constantly to Europe, to the
people I was supposed to know there---putting leading questions as to my
acquaintances in the sepulchral city, and so on. His little eyes
glittered like mica discs---with curiosity---though he tried to keep up
a bit of superciliousness. At first I was astonished, but very soon I
became awfully curious to see what he would find out from me. I couldn't
possibly imagine what I had in me to make it worth his while. It was
very pretty to see how he baffled himself, for in truth my body was full
only of chills, and my head had nothing in it but that wretched
steamboat business. It was evident he took me for a perfectly shameless
prevaricator. At last he got angry, and, to conceal a movement of
furious annoyance, he yawned. I rose. Then I noticed a small sketch in
oils, on a panel, representing a woman, draped and blindfolded, carrying
a lighted torch. The background was sombre---almost black. The movement
of the woman was stately, and the effect of the torchlight on the face
was sinister.

``It arrested me, and he stood by civilly, holding an empty half-pint
champagne bottle (medical comforts) with the candle stuck in it. To my
question he said Mr. Kurtz had painted this---in this very station more
than a year ago---while waiting for means to go to his trading post.
`Tell me, pray,' said I, `who is this Mr. Kurtz?'

```The chief of the Inner Station,' he answered in a short tone, looking
away. `Much obliged,' I said, laughing. `And you are the brickmaker of
the Central Station. Every one knows that.' He was silent for a while.
`He is a prodigy,' he said at last. `He is an emissary of pity and
science and progress, and devil knows what else. We want,' he began to
declaim suddenly, `for the guidance of the cause intrusted to us by
Europe, so to speak, higher intelligence, wide sympathies, a singleness
of purpose.' `Who says that?' I asked. `Lots of them,' he replied. `Some
even write that; and so \emph{he} comes here, a special being, as you
ought to know.' `Why ought I to know?' I interrupted, really surprised.
He paid no attention. `Yes. Today he is chief of the best station, next
year he will be assistant-manager, two years more and\ldots{} but I
dare-say you know what he will be in two years' time. You are of the new
gang---the gang of virtue. The same people who sent him specially also
recommended you. Oh, don't say no. I've my own eyes to trust.' Light
dawned upon me. My dear aunt's influential acquaintances were producing
an unexpected effect upon that young man. I nearly burst into a laugh.
`Do you read the Company's confidential correspondence?' I asked. He
hadn't a word to say. It was great fun. `When Mr. Kurtz,' I continued,
severely, `is General Manager, you won't have the opportunity.'

``He blew the candle out suddenly, and we went outside. The moon had
risen. Black figures strolled about listlessly, pouring water on the
glow, whence proceeded a sound of hissing; steam ascended in the
moonlight, the beaten nigger groaned somewhere. `What a row the brute
makes!' said the indefatigable man with the moustaches, appearing near
us. `Serve him right. Transgression---punishment---bang! Pitiless,
pitiless. That's the only way. This will prevent all conflagrations for
the future. I was just telling the manager\ldots{}' He noticed my
companion, and became crestfallen all at once. `Not in bed yet,' he
said, with a kind of servile heartiness; `it's so natural. Ha!
Danger---agitation.' He vanished. I went on to the riverside, and the
other followed me. I heard a scathing murmur at my ear, `Heap of
muffs---go to.' The pilgrims could be seen in knots gesticulating,
discussing. Several had still their staves in their hands. I verily
believe they took these sticks to bed with them. Beyond the fence the
forest stood up spectrally in the moonlight, and through that dim stir,
through the faint sounds of that lamentable courtyard, the silence of
the land went home to one's very heart---its mystery, its greatness, the
amazing reality of its concealed life. The hurt nigger moaned feebly
somewhere near by, and then fetched a deep sigh that made me mend my
pace away from there. I felt a hand introducing itself under my arm. `My
dear sir,' said the fellow, `I don't want to be misunderstood, and
especially by you, who will see Mr. Kurtz long before I can have that
pleasure. I wouldn't like him to get a false idea of my
disposition\ldots{}.'

``I let him run on, this \emph{papier-mache} Mephistopheles, and it
seemed to me that if I tried I could poke my forefinger through him, and
would find nothing inside but a little loose dirt, maybe. He, don't you
see, had been planning to be assistant-manager by and by under the
present man, and I could see that the coming of that Kurtz had upset
them both not a little. He talked precipitately, and I did not try to
stop him. I had my shoulders against the wreck of my steamer, hauled up
on the slope like a carcass of some big river animal. The smell of mud,
of primeval mud, by Jove! was in my nostrils, the high stillness of
primeval forest was before my eyes; there were shiny patches on the
black creek. The moon had spread over everything a thin layer of
silver---over the rank grass, over the mud, upon the wall of matted
vegetation standing higher than the wall of a temple, over the great
river I could see through a sombre gap glittering, glittering, as it
flowed broadly by without a murmur. All this was great, expectant, mute,
while the man jabbered about himself. I wondered whether the stillness
on the face of the immensity looking at us two were meant as an appeal
or as a menace. What were we who had strayed in here? Could we handle
that dumb thing, or would it handle us? I felt how big, how confoundedly
big, was that thing that couldn't talk, and perhaps was deaf as well.
What was in there? I could see a little ivory coming out from there, and
I had heard Mr. Kurtz was in there. I had heard enough about it,
too---God knows! Yet somehow it didn't bring any image with it---no more
than if I had been told an angel or a fiend was in there. I believed it
in the same way one of you might believe there are inhabitants in the
planet Mars. I knew once a Scotch sailmaker who was certain, dead sure,
there were people in Mars. If you asked him for some idea how they
looked and behaved, he would get shy and mutter something about `walking
on all-fours.' If you as much as smiled, he would---though a man of
sixty---offer to fight you. I would not have gone so far as to fight for
Kurtz, but I went for him near enough to a lie. You know I hate, detest,
and can't bear a lie, not because I am straighter than the rest of us,
but simply because it appalls me. There is a taint of death, a flavour
of mortality in lies---which is exactly what I hate and detest in the
world---what I want to forget. It makes me miserable and sick, like
biting something rotten would do. Temperament, I suppose. Well, I went
near enough to it by letting the young fool there believe anything he
liked to imagine as to my influence in Europe. I became in an instant as
much of a pretence as the rest of the bewitched pilgrims. This simply
because I had a notion it somehow would be of help to that Kurtz whom at
the time I did not see---you understand. He was just a word for me. I
did not see the man in the name any more than you do. Do you see him? Do
you see the story? Do you see anything? It seems to me I am trying to
tell you a dream---making a vain attempt, because no relation of a dream
can convey the dream-sensation, that commingling of absurdity, surprise,
and bewilderment in a tremor of struggling revolt, that notion of being
captured by the incredible which is of the very essence of
dreams\ldots{}.''

He was silent for a while.

``\ldots{} No, it is impossible; it is impossible to convey the
life-sensation of any given epoch of one's existence---that which makes
its truth, its meaning---its subtle and penetrating essence. It is
impossible. We live, as we dream---alone\ldots{}.''

He paused again as if reflecting, then added:

``Of course in this you fellows see more than I could then. You see me,
whom you know\ldots{}.''

It had become so pitch dark that we listeners could hardly see one
another. For a long time already he, sitting apart, had been no more to
us than a voice. There was not a word from anybody. The others might
have been asleep, but I was awake. I listened, I listened on the watch
for the sentence, for the word, that would give me the clue to the faint
uneasiness inspired by this narrative that seemed to shape itself
without human lips in the heavy night-air of the river.

``\ldots{} Yes---I let him run on,'' Marlow began again, ``and think
what he pleased about the powers that were behind me. I did! And there
was nothing behind me! There was nothing but that wretched, old, mangled
steamboat I was leaning against, while he talked fluently about `the
necessity for every man to get on.' `And when one comes out here, you
conceive, it is not to gaze at the moon.' Mr. Kurtz was a `universal
genius,' but even a genius would find it easier to work with `adequate
tools---intelligent men.' He did not make bricks---why, there was a
physical impossibility in the way---as I was well aware; and if he did
secretarial work for the manager, it was because `no sensible man
rejects wantonly the confidence of his superiors.' Did I see it? I saw
it. What more did I want? What I really wanted was rivets, by heaven!
Rivets. To get on with the work---to stop the hole. Rivets I wanted.
There were cases of them down at the coast---cases---piled
up---burst---split! You kicked a loose rivet at every second step in
that station-yard on the hillside. Rivets had rolled into the grove of
death. You could fill your pockets with rivets for the trouble of
stooping down---and there wasn't one rivet to be found where it was
wanted. We had plates that would do, but nothing to fasten them with.
And every week the messenger, a long negro, letter-bag on shoulder and
staff in hand, left our station for the coast. And several times a week
a coast caravan came in with trade goods---ghastly glazed calico that
made you shudder only to look at it, glass beads value about a penny a
quart, confounded spotted cotton handkerchiefs. And no rivets. Three
carriers could have brought all that was wanted to set that steamboat
afloat.

``He was becoming confidential now, but I fancy my unresponsive attitude
must have exasperated him at last, for he judged it necessary to inform
me he feared neither God nor devil, let alone any mere man. I said I
could see that very well, but what I wanted was a certain quantity of
rivets---and rivets were what really Mr. Kurtz wanted, if he had only
known it. Now letters went to the coast every week\ldots{}. `My dear
sir,' he cried, `I write from dictation.' I demanded rivets. There was a
way---for an intelligent man. He changed his manner; became very cold,
and suddenly began to talk about a hippopotamus; wondered whether
sleeping on board the steamer (I stuck to my salvage night and day) I
wasn't disturbed. There was an old hippo that had the bad habit of
getting out on the bank and roaming at night over the station grounds.
The pilgrims used to turn out in a body and empty every rifle they could
lay hands on at him. Some even had sat up o' nights for him. All this
energy was wasted, though. `That animal has a charmed life,' he said;
`but you can say this only of brutes in this country. No man---you
apprehend me?---no man here bears a charmed life.' He stood there for a
moment in the moonlight with his delicate hooked nose set a little
askew, and his mica eyes glittering without a wink, then, with a curt
Good-night, he strode off. I could see he was disturbed and considerably
puzzled, which made me feel more hopeful than I had been for days. It
was a great comfort to turn from that chap to my influential friend, the
battered, twisted, ruined, tin-pot steamboat. I clambered on board. She
rang under my feet like an empty Huntley \& Palmer biscuit-tin kicked
along a gutter; she was nothing so solid in make, and rather less pretty
in shape, but I had expended enough hard work on her to make me love
her. No influential friend would have served me better. She had given me
a chance to come out a bit---to find out what I could do. No, I don't
like work. I had rather laze about and think of all the fine things that
can be done. I don't like work---no man does---but I like what is in the
work---the chance to find yourself. Your own reality---for yourself, not
for others---what no other man can ever know. They can only see the mere
show, and never can tell what it really means.

``I was not surprised to see somebody sitting aft, on the deck, with his
legs dangling over the mud. You see I rather chummed with the few
mechanics there were in that station, whom the other pilgrims naturally
despised---on account of their imperfect manners, I suppose. This was
the foreman---a boiler-maker by trade---a good worker. He was a lank,
bony, yellow-faced man, with big intense eyes. His aspect was worried,
and his head was as bald as the palm of my hand; but his hair in falling
seemed to have stuck to his chin, and had prospered in the new locality,
for his beard hung down to his waist. He was a widower with six young
children (he had left them in charge of a sister of his to come out
there), and the passion of his life was pigeon-flying. He was an
enthusiast and a connoisseur. He would rave about pigeons. After work
hours he used sometimes to come over from his hut for a talk about his
children and his pigeons; at work, when he had to crawl in the mud under
the bottom of the steamboat, he would tie up that beard of his in a kind
of white serviette he brought for the purpose. It had loops to go over
his ears. In the evening he could be seen squatted on the bank rinsing
that wrapper in the creek with great care, then spreading it solemnly on
a bush to dry.

``I slapped him on the back and shouted, `We shall have rivets!' He
scrambled to his feet exclaiming, `No! Rivets!' as though he couldn't
believe his ears. Then in a low voice, `You\ldots{} eh?' I don't know
why we behaved like lunatics. I put my finger to the side of my nose and
nodded mysteriously. `Good for you!' he cried, snapped his fingers above
his head, lifting one foot. I tried a jig. We capered on the iron deck.
A frightful clatter came out of that hulk, and the virgin forest on the
other bank of the creek sent it back in a thundering roll upon the
sleeping station. It must have made some of the pilgrims sit up in their
hovels. A dark figure obscured the lighted doorway of the manager's hut,
vanished, then, a second or so after, the doorway itself vanished, too.
We stopped, and the silence driven away by the stamping of our feet
flowed back again from the recesses of the land. The great wall of
vegetation, an exuberant and entangled mass of trunks, branches, leaves,
boughs, festoons, motionless in the moonlight, was like a rioting
invasion of soundless life, a rolling wave of plants, piled up, crested,
ready to topple over the creek, to sweep every little man of us out of
his little existence. And it moved not. A deadened burst of mighty
splashes and snorts reached us from afar, as though an icthyosaurus had
been taking a bath of glitter in the great river. `After all,' said the
boiler-maker in a reasonable tone, `why shouldn't we get the rivets?'
Why not, indeed! I did not know of any reason why we shouldn't. `They'll
come in three weeks,' I said confidently.

``But they didn't. Instead of rivets there came an invasion, an
infliction, a visitation. It came in sections during the next three
weeks, each section headed by a donkey carrying a white man in new
clothes and tan shoes, bowing from that elevation right and left to the
impressed pilgrims. A quarrelsome band of footsore sulky niggers trod on
the heels of the donkey; a lot of tents, camp-stools, tin boxes, white
cases, brown bales would be shot down in the courtyard, and the air of
mystery would deepen a little over the muddle of the station. Five such
instalments came, with their absurd air of disorderly flight with the
loot of innumerable outfit shops and provision stores, that, one would
think, they were lugging, after a raid, into the wilderness for
equitable division. It was an inextricable mess of things decent in
themselves but that human folly made look like the spoils of thieving.

``This devoted band called itself the Eldorado Exploring Expedition, and
I believe they were sworn to secrecy. Their talk, however, was the talk
of sordid buccaneers: it was reckless without hardihood, greedy without
audacity, and cruel without courage; there was not an atom of foresight
or of serious intention in the whole batch of them, and they did not
seem aware these things are wanted for the work of the world. To tear
treasure out of the bowels of the land was their desire, with no more
moral purpose at the back of it than there is in burglars breaking into
a safe. Who paid the expenses of the noble enterprise I don't know; but
the uncle of our manager was leader of that lot.

``In exterior he resembled a butcher in a poor neighbourhood, and his
eyes had a look of sleepy cunning. He carried his fat paunch with
ostentation on his short legs, and during the time his gang infested the
station spoke to no one but his nephew. You could see these two roaming
about all day long with their heads close together in an everlasting
confab.

``I had given up worrying myself about the rivets. One's capacity for
that kind of folly is more limited than you would suppose. I said
Hang!---and let things slide. I had plenty of time for meditation, and
now and then I would give some thought to Kurtz. I wasn't very
interested in him. No. Still, I was curious to see whether this man, who
had come out equipped with moral ideas of some sort, would climb to the
top after all and how he would set about his work when there.''

\mychapter{2}{II}\label{ii}

``One evening as I was lying flat on the deck of my steamboat, I heard
voices approaching---and there were the nephew and the uncle strolling
along the bank. I laid my head on my arm again, and had nearly lost
myself in a doze, when somebody said in my ear, as it were: `I am as
harmless as a little child, but I don't like to be dictated to. Am I the
manager---or am I not? I was ordered to send him there. It's
incredible.' \ldots{} I became aware that the two were standing on the
shore alongside the forepart of the steamboat, just below my head. I did
not move; it did not occur to me to move: I was sleepy. `It \emph{is}
unpleasant,' grunted the uncle. `He has asked the Administration to be
sent there,' said the other, `with the idea of showing what he could do;
and I was instructed accordingly. Look at the influence that man must
have. Is it not frightful?' They both agreed it was frightful, then made
several bizarre remarks: `Make rain and fine weather---one man---the
Council---by the nose'---bits of absurd sentences that got the better of
my drowsiness, so that I had pretty near the whole of my wits about me
when the uncle said, `The climate may do away with this difficulty for
you. Is he alone there?' `Yes,' answered the manager; `he sent his
assistant down the river with a note to me in these terms: ``Clear this
poor devil out of the country, and don't bother sending more of that
sort. I had rather be alone than have the kind of men you can dispose of
with me.'' It was more than a year ago. Can you imagine such impudence!'
`Anything since then?' asked the other hoarsely. `Ivory,' jerked the
nephew; `lots of it---prime sort---lots---most annoying, from him.' `And
with that?' questioned the heavy rumble. `Invoice,' was the reply fired
out, so to speak. Then silence. They had been talking about Kurtz.

``I was broad awake by this time, but, lying perfectly at ease, remained
still, having no inducement to change my position. `How did that ivory
come all this way?' growled the elder man, who seemed very vexed. The
other explained that it had come with a fleet of canoes in charge of an
English half-caste clerk Kurtz had with him; that Kurtz had apparently
intended to return himself, the station being by that time bare of goods
and stores, but after coming three hundred miles, had suddenly decided
to go back, which he started to do alone in a small dugout with four
paddlers, leaving the half-caste to continue down the river with the
ivory. The two fellows there seemed astounded at anybody attempting such
a thing. They were at a loss for an adequate motive. As to me, I seemed
to see Kurtz for the first time. It was a distinct glimpse: the dugout,
four paddling savages, and the lone white man turning his back suddenly
on the headquarters, on relief, on thoughts of home---perhaps; setting
his face towards the depths of the wilderness, towards his empty and
desolate station. I did not know the motive. Perhaps he was just simply
a fine fellow who stuck to his work for its own sake. His name, you
understand, had not been pronounced once. He was `that man.' The
half-caste, who, as far as I could see, had conducted a difficult trip
with great prudence and pluck, was invariably alluded to as `that
scoundrel.' The `scoundrel' had reported that the `man' had been very
ill---had recovered imperfectly\ldots{}. The two below me moved away
then a few paces, and strolled back and forth at some little distance. I
heard: `Military post---doctor---two hundred miles---quite alone
now---unavoidable delays---nine months---no news---strange rumours.'
They approached again, just as the manager was saying, `No one, as far
as I know, unless a species of wandering trader---a pestilential fellow,
snapping ivory from the natives.' Who was it they were talking about
now? I gathered in snatches that this was some man supposed to be in
Kurtz's district, and of whom the manager did not approve. `We will not
be free from unfair competition till one of these fellows is hanged for
an example,' he said. `Certainly,' grunted the other; `get him hanged!
Why not? Anything---anything can be done in this country. That's what I
say; nobody here, you understand, \emph{here}, can endanger your
position. And why? You stand the climate---you outlast them all. The
danger is in Europe; but there before I left I took care to---' They
moved off and whispered, then their voices rose again. `The
extraordinary series of delays is not my fault. I did my best.' The fat
man sighed. `Very sad.' `And the pestiferous absurdity of his talk,'
continued the other; `he bothered me enough when he was here. ``Each
station should be like a beacon on the road towards better things, a
centre for trade of course, but also for humanizing, improving,
instructing.'' Conceive you---that ass! And he wants to be manager! No,
it's---' Here he got choked by excessive indignation, and I lifted my
head the least bit. I was surprised to see how near they were---right
under me. I could have spat upon their hats. They were looking on the
ground, absorbed in thought. The manager was switching his leg with a
slender twig: his sagacious relative lifted his head. `You have been
well since you came out this time?' he asked. The other gave a start.
`Who? I? Oh! Like a charm---like a charm. But the rest---oh, my
goodness! All sick. They die so quick, too, that I haven't the time to
send them out of the country---it's incredible!' `Hm'm. Just so,'
grunted the uncle. `Ah! my boy, trust to this---I say, trust to this.' I
saw him extend his short flipper of an arm for a gesture that took in
the forest, the creek, the mud, the river---seemed to beckon with a
dishonouring flourish before the sunlit face of the land a treacherous
appeal to the lurking death, to the hidden evil, to the profound
darkness of its heart. It was so startling that I leaped to my feet and
looked back at the edge of the forest, as though I had expected an
answer of some sort to that black display of confidence. You know the
foolish notions that come to one sometimes. The high stillness
confronted these two figures with its ominous patience, waiting for the
passing away of a fantastic invasion.

``They swore aloud together---out of sheer fright, I believe---then
pretending not to know anything of my existence, turned back to the
station. The sun was low; and leaning forward side by side, they seemed
to be tugging painfully uphill their two ridiculous shadows of unequal
length, that trailed behind them slowly over the tall grass without
bending a single blade.

``In a few days the Eldorado Expedition went into the patient
wilderness, that closed upon it as the sea closes over a diver. Long
afterwards the news came that all the donkeys were dead. I know nothing
as to the fate of the less valuable animals. They, no doubt, like the
rest of us, found what they deserved. I did not inquire. I was then
rather excited at the prospect of meeting Kurtz very soon. When I say
very soon I mean it comparatively. It was just two months from the day
we left the creek when we came to the bank below Kurtz's station.

``Going up that river was like traveling back to the earliest beginnings
of the world, when vegetation rioted on the earth and the big trees were
kings. An empty stream, a great silence, an impenetrable forest. The air
was warm, thick, heavy, sluggish. There was no joy in the brilliance of
sunshine. The long stretches of the waterway ran on, deserted, into the
gloom of overshadowed distances. On silvery sand-banks hippos and
alligators sunned themselves side by side. The broadening waters flowed
through a mob of wooded islands; you lost your way on that river as you
would in a desert, and butted all day long against shoals, trying to
find the channel, till you thought yourself bewitched and cut off for
ever from everything you had known once---somewhere---far away---in
another existence perhaps. There were moments when one's past came back
to one, as it will sometimes when you have not a moment to spare for
yourself; but it came in the shape of an unrestful and noisy dream,
remembered with wonder amongst the overwhelming realities of this
strange world of plants, and water, and silence. And this stillness of
life did not in the least resemble a peace. It was the stillness of an
implacable force brooding over an inscrutable intention. It looked at
you with a vengeful aspect. I got used to it afterwards; I did not see
it any more; I had no time. I had to keep guessing at the channel; I had
to discern, mostly by inspiration, the signs of hidden banks; I watched
for sunken stones; I was learning to clap my teeth smartly before my
heart flew out, when I shaved by a fluke some infernal sly old snag that
would have ripped the life out of the tin-pot steamboat and drowned all
the pilgrims; I had to keep a lookout for the signs of dead wood we
could cut up in the night for next day's steaming. When you have to
attend to things of that sort, to the mere incidents of the surface, the
reality---the reality, I tell you---fades. The inner truth is
hidden---luckily, luckily. But I felt it all the same; I felt often its
mysterious stillness watching me at my monkey tricks, just as it watches
you fellows performing on your respective tight-ropes for---what is it?
half-a-crown a tumble---''

``Try to be civil, Marlow,'' growled a voice, and I knew there was at
least one listener awake besides myself.

``I beg your pardon. I forgot the heartache which makes up the rest of
the price. And indeed what does the price matter, if the trick be well
done? You do your tricks very well. And I didn't do badly either, since
I managed not to sink that steamboat on my first trip. It's a wonder to
me yet. Imagine a blindfolded man set to drive a van over a bad road. I
sweated and shivered over that business considerably, I can tell you.
After all, for a seaman, to scrape the bottom of the thing that's
supposed to float all the time under his care is the unpardonable sin.
No one may know of it, but you never forget the thump---eh? A blow on
the very heart. You remember it, you dream of it, you wake up at night
and think of it---years after---and go hot and cold all over. I don't
pretend to say that steamboat floated all the time. More than once she
had to wade for a bit, with twenty cannibals splashing around and
pushing. We had enlisted some of these chaps on the way for a crew. Fine
fellows---cannibals---in their place. They were men one could work with,
and I am grateful to them. And, after all, they did not eat each other
before my face: they had brought along a provision of hippo-meat which
went rotten, and made the mystery of the wilderness stink in my
nostrils. Phoo! I can sniff it now. I had the manager on board and three
or four pilgrims with their staves---all complete. Sometimes we came
upon a station close by the bank, clinging to the skirts of the unknown,
and the white men rushing out of a tumble-down hovel, with great
gestures of joy and surprise and welcome, seemed very strange---had the
appearance of being held there captive by a spell. The word ivory would
ring in the air for a while---and on we went again into the silence,
along empty reaches, round the still bends, between the high walls of
our winding way, reverberating in hollow claps the ponderous beat of the
stern-wheel. Trees, trees, millions of trees, massive, immense, running
up high; and at their foot, hugging the bank against the stream, crept
the little begrimed steamboat, like a sluggish beetle crawling on the
floor of a lofty portico. It made you feel very small, very lost, and
yet it was not altogether depressing, that feeling. After all, if you
were small, the grimy beetle crawled on---which was just what you wanted
it to do. Where the pilgrims imagined it crawled to I don't know. To
some place where they expected to get something. I bet! For me it
crawled towards Kurtz---exclusively; but when the steam-pipes started
leaking we crawled very slow. The reaches opened before us and closed
behind, as if the forest had stepped leisurely across the water to bar
the way for our return. We penetrated deeper and deeper into the heart
of darkness. It was very quiet there. At night sometimes the roll of
drums behind the curtain of trees would run up the river and remain
sustained faintly, as if hovering in the air high over our heads, till
the first break of day. Whether it meant war, peace, or prayer we could
not tell. The dawns were heralded by the descent of a chill stillness;
the wood-cutters slept, their fires burned low; the snapping of a twig
would make you start. We were wanderers on a prehistoric earth, on an
earth that wore the aspect of an unknown planet. We could have fancied
ourselves the first of men taking possession of an accursed inheritance,
to be subdued at the cost of profound anguish and of excessive toil. But
suddenly, as we struggled round a bend, there would be a glimpse of rush
walls, of peaked grass-roofs, a burst of yells, a whirl of black limbs,
a mass of hands clapping of feet stamping, of bodies swaying, of eyes
rolling, under the droop of heavy and motionless foliage. The steamer
toiled along slowly on the edge of a black and incomprehensible frenzy.
The prehistoric man was cursing us, praying to us, welcoming us---who
could tell? We were cut off from the comprehension of our surroundings;
we glided past like phantoms, wondering and secretly appalled, as sane
men would be before an enthusiastic outbreak in a madhouse. We could not
understand because we were too far and could not remember because we
were travelling in the night of first ages, of those ages that are gone,
leaving hardly a sign---and no memories.

``The earth seemed unearthly. We are accustomed to look upon the
shackled form of a conquered monster, but there---there you could look
at a thing monstrous and free. It was unearthly, and the men were---No,
they were not inhuman. Well, you know, that was the worst of it---this
suspicion of their not being inhuman. It would come slowly to one. They
howled and leaped, and spun, and made horrid faces; but what thrilled
you was just the thought of their humanity---like yours---the thought of
your remote kinship with this wild and passionate uproar. Ugly. Yes, it
was ugly enough; but if you were man enough you would admit to yourself
that there was in you just the faintest trace of a response to the
terrible frankness of that noise, a dim suspicion of there being a
meaning in it which you---you so remote from the night of first
ages---could comprehend. And why not? The mind of man is capable of
anything---because everything is in it, all the past as well as all the
future. What was there after all? Joy, fear, sorrow, devotion, valour,
rage---who can tell?---but truth---truth stripped of its cloak of time.
Let the fool gape and shudder---the man knows, and can look on without a
wink. But he must at least be as much of a man as these on the shore. He
must meet that truth with his own true stuff---with his own inborn
strength. Principles won't do. Acquisitions, clothes, pretty rags---rags
that would fly off at the first good shake. No; you want a deliberate
belief. An appeal to me in this fiendish row---is there? Very well; I
hear; I admit, but I have a voice, too, and for good or evil mine is the
speech that cannot be silenced. Of course, a fool, what with sheer
fright and fine sentiments, is always safe. Who's that grunting? You
wonder I didn't go ashore for a howl and a dance? Well, no---I didn't.
Fine sentiments, you say? Fine sentiments, be hanged! I had no time. I
had to mess about with white-lead and strips of woolen blanket helping
to put bandages on those leaky steam-pipes---I tell you. I had to watch
the steering, and circumvent those snags, and get the tin-pot along by
hook or by crook. There was surface-truth enough in these things to save
a wiser man. And between whiles I had to look after the savage who was
fireman. He was an improved specimen; he could fire up a vertical
boiler. He was there below me, and, upon my word, to look at him was as
edifying as seeing a dog in a parody of breeches and a feather hat,
walking on his hind-legs. A few months of training had done for that
really fine chap. He squinted at the steam-gauge and at the water-gauge
with an evident effort of intrepidity---and he had filed teeth, too, the
poor devil, and the wool of his pate shaved into queer patterns, and
three ornamental scars on each of his cheeks. He ought to have been
clapping his hands and stamping his feet on the bank, instead of which
he was hard at work, a thrall to strange witchcraft, full of improving
knowledge. He was useful because he had been instructed; and what he
knew was this---that should the water in that transparent thing
disappear, the evil spirit inside the boiler would get angry through the
greatness of his thirst, and take a terrible vengeance. So he sweated
and fired up and watched the glass fearfully (with an impromptu charm,
made of rags, tied to his arm, and a piece of polished bone, as big as a
watch, stuck flatways through his lower lip), while the wooded banks
slipped past us slowly, the short noise was left behind, the
interminable miles of silence---and we crept on, towards Kurtz. But the
snags were thick, the water was treacherous and shallow, the boiler
seemed indeed to have a sulky devil in it, and thus neither that fireman
nor I had any time to peer into our creepy thoughts.

``Some fifty miles below the Inner Station we came upon a hut of reeds,
an inclined and melancholy pole, with the unrecognizable tatters of what
had been a flag of some sort flying from it, and a neatly stacked
wood-pile. This was unexpected. We came to the bank, and on the stack of
firewood found a flat piece of board with some faded pencil-writing on
it. When deciphered it said: `Wood for you. Hurry up. Approach
cautiously.' There was a signature, but it was illegible---not Kurtz---a
much longer word. `Hurry up.' Where? Up the river? `Approach
cautiously.' We had not done so. But the warning could not have been
meant for the place where it could be only found after approach.
Something was wrong above. But what---and how much? That was the
question. We commented adversely upon the imbecility of that telegraphic
style. The bush around said nothing, and would not let us look very far,
either. A torn curtain of red twill hung in the doorway of the hut, and
flapped sadly in our faces. The dwelling was dismantled; but we could
see a white man had lived there not very long ago. There remained a rude
table---a plank on two posts; a heap of rubbish reposed in a dark
corner, and by the door I picked up a book. It had lost its covers, and
the pages had been thumbed into a state of extremely dirty softness; but
the back had been lovingly stitched afresh with white cotton thread,
which looked clean yet. It was an extraordinary find. Its title was,
\emph{An Inquiry into some Points of Seamanship}, by a man Towser,
Towson---some such name---Master in his Majesty's Navy. The matter
looked dreary reading enough, with illustrative diagrams and repulsive
tables of figures, and the copy was sixty years old. I handled this
amazing antiquity with the greatest possible tenderness, lest it should
dissolve in my hands. Within, Towson or Towser was inquiring earnestly
into the breaking strain of ships' chains and tackle, and other such
matters. Not a very enthralling book; but at the first glance you could
see there a singleness of intention, an honest concern for the right way
of going to work, which made these humble pages, thought out so many
years ago, luminous with another than a professional light. The simple
old sailor, with his talk of chains and purchases, made me forget the
jungle and the pilgrims in a delicious sensation of having come upon
something unmistakably real. Such a book being there was wonderful
enough; but still more astounding were the notes pencilled in the
margin, and plainly referring to the text. I couldn't believe my eyes!
They were in cipher! Yes, it looked like cipher. Fancy a man lugging
with him a book of that description into this nowhere and studying
it---and making notes---in cipher at that! It was an extravagant
mystery.

``I had been dimly aware for some time of a worrying noise, and when I
lifted my eyes I saw the wood-pile was gone, and the manager, aided by
all the pilgrims, was shouting at me from the riverside. I slipped the
book into my pocket. I assure you to leave off reading was like tearing
myself away from the shelter of an old and solid friendship.

``I started the lame engine ahead. `It must be this miserable
trader---this intruder,' exclaimed the manager, looking back
malevolently at the place we had left. `He must be English,' I said. `It
will not save him from getting into trouble if he is not careful,'
muttered the manager darkly. I observed with assumed innocence that no
man was safe from trouble in this world.

``The current was more rapid now, the steamer seemed at her last gasp,
the stern-wheel flopped languidly, and I caught myself listening on
tiptoe for the next beat of the boat, for in sober truth I expected the
wretched thing to give up every moment. It was like watching the last
flickers of a life. But still we crawled. Sometimes I would pick out a
tree a little way ahead to measure our progress towards Kurtz by, but I
lost it invariably before we got abreast. To keep the eyes so long on
one thing was too much for human patience. The manager displayed a
beautiful resignation. I fretted and fumed and took to arguing with
myself whether or no I would talk openly with Kurtz; but before I could
come to any conclusion it occurred to me that my speech or my silence,
indeed any action of mine, would be a mere futility. What did it matter
what any one knew or ignored? What did it matter who was manager? One
gets sometimes such a flash of insight. The essentials of this affair
lay deep under the surface, beyond my reach, and beyond my power of
meddling.

``Towards the evening of the second day we judged ourselves about eight
miles from Kurtz's station. I wanted to push on; but the manager looked
grave, and told me the navigation up there was so dangerous that it
would be advisable, the sun being very low already, to wait where we
were till next morning. Moreover, he pointed out that if the warning to
approach cautiously were to be followed, we must approach in
daylight---not at dusk or in the dark. This was sensible enough. Eight
miles meant nearly three hours' steaming for us, and I could also see
suspicious ripples at the upper end of the reach. Nevertheless, I was
annoyed beyond expression at the delay, and most unreasonably, too,
since one night more could not matter much after so many months. As we
had plenty of wood, and caution was the word, I brought up in the middle
of the stream. The reach was narrow, straight, with high sides like a
railway cutting. The dusk came gliding into it long before the sun had
set. The current ran smooth and swift, but a dumb immobility sat on the
banks. The living trees, lashed together by the creepers and every
living bush of the undergrowth, might have been changed into stone, even
to the slenderest twig, to the lightest leaf. It was not sleep---it
seemed unnatural, like a state of trance. Not the faintest sound of any
kind could be heard. You looked on amazed, and began to suspect yourself
of being deaf---then the night came suddenly, and struck you blind as
well. About three in the morning some large fish leaped, and the loud
splash made me jump as though a gun had been fired. When the sun rose
there was a white fog, very warm and clammy, and more blinding than the
night. It did not shift or drive; it was just there, standing all round
you like something solid. At eight or nine, perhaps, it lifted as a
shutter lifts. We had a glimpse of the towering multitude of trees, of
the immense matted jungle, with the blazing little ball of the sun
hanging over it---all perfectly still---and then the white shutter came
down again, smoothly, as if sliding in greased grooves. I ordered the
chain, which we had begun to heave in, to be paid out again. Before it
stopped running with a muffled rattle, a cry, a very loud cry, as of
infinite desolation, soared slowly in the opaque air. It ceased. A
complaining clamour, modulated in savage discords, filled our ears. The
sheer unexpectedness of it made my hair stir under my cap. I don't know
how it struck the others: to me it seemed as though the mist itself had
screamed, so suddenly, and apparently from all sides at once, did this
tumultuous and mournful uproar arise. It culminated in a hurried
outbreak of almost intolerably excessive shrieking, which stopped short,
leaving us stiffened in a variety of silly attitudes, and obstinately
listening to the nearly as appalling and excessive silence. `Good God!
What is the meaning---' stammered at my elbow one of the pilgrims---a
little fat man, with sandy hair and red whiskers, who wore sidespring
boots, and pink pyjamas tucked into his socks. Two others remained
open-mouthed a while minute, then dashed into the little cabin, to rush
out incontinently and stand darting scared glances, with Winchesters at
`ready' in their hands. What we could see was just the steamer we were
on, her outlines blurred as though she had been on the point of
dissolving, and a misty strip of water, perhaps two feet broad, around
her---and that was all. The rest of the world was nowhere, as far as our
eyes and ears were concerned. Just nowhere. Gone, disappeared; swept off
without leaving a whisper or a shadow behind.

``I went forward, and ordered the chain to be hauled in short, so as to
be ready to trip the anchor and move the steamboat at once if necessary.
`Will they attack?' whispered an awed voice. `We will be all butchered
in this fog,' murmured another. The faces twitched with the strain, the
hands trembled slightly, the eyes forgot to wink. It was very curious to
see the contrast of expressions of the white men and of the black
fellows of our crew, who were as much strangers to that part of the
river as we, though their homes were only eight hundred miles away. The
whites, of course greatly discomposed, had besides a curious look of
being painfully shocked by such an outrageous row. The others had an
alert, naturally interested expression; but their faces were essentially
quiet, even those of the one or two who grinned as they hauled at the
chain. Several exchanged short, grunting phrases, which seemed to settle
the matter to their satisfaction. Their headman, a young, broad-chested
black, severely draped in dark-blue fringed cloths, with fierce nostrils
and his hair all done up artfully in oily ringlets, stood near me.
`Aha!' I said, just for good fellowship's sake. `Catch `im,' he snapped,
with a bloodshot widening of his eyes and a flash of sharp
teeth---`catch `im. Give `im to us.' `To you, eh?' I asked; `what would
you do with them?' `Eat `im!' he said curtly, and, leaning his elbow on
the rail, looked out into the fog in a dignified and profoundly pensive
attitude. I would no doubt have been properly horrified, had it not
occurred to me that he and his chaps must be very hungry: that they must
have been growing increasingly hungry for at least this month past. They
had been engaged for six months (I don't think a single one of them had
any clear idea of time, as we at the end of countless ages have. They
still belonged to the beginnings of time---had no inherited experience
to teach them as it were), and of course, as long as there was a piece
of paper written over in accordance with some farcical law or other made
down the river, it didn't enter anybody's head to trouble how they would
live. Certainly they had brought with them some rotten hippo-meat, which
couldn't have lasted very long, anyway, even if the pilgrims hadn't, in
the midst of a shocking hullabaloo, thrown a considerable quantity of it
overboard. It looked like a high-handed proceeding; but it was really a
case of legitimate self-defence. You can't breathe dead hippo waking,
sleeping, and eating, and at the same time keep your precarious grip on
existence. Besides that, they had given them every week three pieces of
brass wire, each about nine inches long; and the theory was they were to
buy their provisions with that currency in riverside villages. You can
see how \emph{that} worked. There were either no villages, or the people
were hostile, or the director, who like the rest of us fed out of tins,
with an occasional old he-goat thrown in, didn't want to stop the
steamer for some more or less recondite reason. So, unless they
swallowed the wire itself, or made loops of it to snare the fishes with,
I don't see what good their extravagant salary could be to them. I must
say it was paid with a regularity worthy of a large and honourable
trading company. For the rest, the only thing to eat---though it didn't
look eatable in the least---I saw in their possession was a few lumps of
some stuff like half-cooked dough, of a dirty lavender colour, they kept
wrapped in leaves, and now and then swallowed a piece of, but so small
that it seemed done more for the looks of the thing than for any serious
purpose of sustenance. Why in the name of all the gnawing devils of
hunger they didn't go for us---they were thirty to five---and have a
good tuck-in for once, amazes me now when I think of it. They were big
powerful men, with not much capacity to weigh the consequences, with
courage, with strength, even yet, though their skins were no longer
glossy and their muscles no longer hard. And I saw that something
restraining, one of those human secrets that baffle probability, had
come into play there. I looked at them with a swift quickening of
interest---not because it occurred to me I might be eaten by them before
very long, though I own to you that just then I perceived---in a new
light, as it were---how unwholesome the pilgrims looked, and I hoped,
yes, I positively hoped, that my aspect was not so---what shall I
say?---so---unappetizing: a touch of fantastic vanity which fitted well
with the dream-sensation that pervaded all my days at that time. Perhaps
I had a little fever, too. One can't live with one's finger
everlastingly on one's pulse. I had often `a little fever,' or a little
touch of other things---the playful paw-strokes of the wilderness, the
preliminary trifling before the more serious onslaught which came in due
course. Yes; I looked at them as you would on any human being, with a
curiosity of their impulses, motives, capacities, weaknesses, when
brought to the test of an inexorable physical necessity. Restraint! What
possible restraint? Was it superstition, disgust, patience, fear---or
some kind of primitive honour? No fear can stand up to hunger, no
patience can wear it out, disgust simply does not exist where hunger is;
and as to superstition, beliefs, and what you may call principles, they
are less than chaff in a breeze. Don't you know the devilry of lingering
starvation, its exasperating torment, its black thoughts, its sombre and
brooding ferocity? Well, I do. It takes a man all his inborn strength to
fight hunger properly. It's really easier to face bereavement,
dishonour, and the perdition of one's soul---than this kind of prolonged
hunger. Sad, but true. And these chaps, too, had no earthly reason for
any kind of scruple. Restraint! I would just as soon have expected
restraint from a hyena prowling amongst the corpses of a battlefield.
But there was the fact facing me---the fact dazzling, to be seen, like
the foam on the depths of the sea, like a ripple on an unfathomable
enigma, a mystery greater---when I thought of it---than the curious,
inexplicable note of desperate grief in this savage clamour that had
swept by us on the river-bank, behind the blind whiteness of the fog.

``Two pilgrims were quarrelling in hurried whispers as to which bank.
`Left.' `no, no; how can you? Right, right, of course.' `It is very
serious,' said the manager's voice behind me; `I would be desolated if
anything should happen to Mr. Kurtz before we came up.' I looked at him,
and had not the slightest doubt he was sincere. He was just the kind of
man who would wish to preserve appearances. That was his restraint. But
when he muttered something about going on at once, I did not even take
the trouble to answer him. I knew, and he knew, that it was impossible.
Were we to let go our hold of the bottom, we would be absolutely in the
air---in space. We wouldn't be able to tell where we were going
to---whether up or down stream, or across---till we fetched against one
bank or the other---and then we wouldn't know at first which it was. Of
course I made no move. I had no mind for a smash-up. You couldn't
imagine a more deadly place for a shipwreck. Whether we drowned at once
or not, we were sure to perish speedily in one way or another. `I
authorize you to take all the risks,' he said, after a short silence. `I
refuse to take any,' I said shortly; which was just the answer he
expected, though its tone might have surprised him. `Well, I must defer
to your judgment. You are captain,' he said with marked civility. I
turned my shoulder to him in sign of my appreciation, and looked into
the fog. How long would it last? It was the most hopeless lookout. The
approach to this Kurtz grubbing for ivory in the wretched bush was beset
by as many dangers as though he had been an enchanted princess sleeping
in a fabulous castle. `Will they attack, do you think?' asked the
manager, in a confidential tone.

``I did not think they would attack, for several obvious reasons. The
thick fog was one. If they left the bank in their canoes they would get
lost in it, as we would be if we attempted to move. Still, I had also
judged the jungle of both banks quite impenetrable---and yet eyes were
in it, eyes that had seen us. The riverside bushes were certainly very
thick; but the undergrowth behind was evidently penetrable. However,
during the short lift I had seen no canoes anywhere in the
reach---certainly not abreast of the steamer. But what made the idea of
attack inconceivable to me was the nature of the noise---of the cries we
had heard. They had not the fierce character boding immediate hostile
intention. Unexpected, wild, and violent as they had been, they had
given me an irresistible impression of sorrow. The glimpse of the
steamboat had for some reason filled those savages with unrestrained
grief. The danger, if any, I expounded, was from our proximity to a
great human passion let loose. Even extreme grief may ultimately vent
itself in violence---but more generally takes the form of
apathy\ldots{}.

``You should have seen the pilgrims stare! They had no heart to grin, or
even to revile me: but I believe they thought me gone mad---with fright,
maybe. I delivered a regular lecture. My dear boys, it was no good
bothering. Keep a lookout? Well, you may guess I watched the fog for the
signs of lifting as a cat watches a mouse; but for anything else our
eyes were of no more use to us than if we had been buried miles deep in
a heap of cotton-wool. It felt like it, too---choking, warm, stifling.
Besides, all I said, though it sounded extravagant, was absolutely true
to fact. What we afterwards alluded to as an attack was really an
attempt at repulse. The action was very far from being aggressive---it
was not even defensive, in the usual sense: it was undertaken under the
stress of desperation, and in its essence was purely protective.

``It developed itself, I should say, two hours after the fog lifted, and
its commencement was at a spot, roughly speaking, about a mile and a
half below Kurtz's station. We had just floundered and flopped round a
bend, when I saw an islet, a mere grassy hummock of bright green, in the
middle of the stream. It was the only thing of the kind; but as we
opened the reach more, I perceived it was the head of a long sand-bank,
or rather of a chain of shallow patches stretching down the middle of
the river. They were discoloured, just awash, and the whole lot was seen
just under the water, exactly as a man's backbone is seen running down
the middle of his back under the skin. Now, as far as I did see, I could
go to the right or to the left of this. I didn't know either channel, of
course. The banks looked pretty well alike, the depth appeared the same;
but as I had been informed the station was on the west side, I naturally
headed for the western passage.

``No sooner had we fairly entered it than I became aware it was much
narrower than I had supposed. To the left of us there was the long
uninterrupted shoal, and to the right a high, steep bank heavily
overgrown with bushes. Above the bush the trees stood in serried ranks.
The twigs overhung the current thickly, and from distance to distance a
large limb of some tree projected rigidly over the stream. It was then
well on in the afternoon, the face of the forest was gloomy, and a broad
strip of shadow had already fallen on the water. In this shadow we
steamed up---very slowly, as you may imagine. I sheered her well
inshore---the water being deepest near the bank, as the sounding-pole
informed me.

``One of my hungry and forbearing friends was sounding in the bows just
below me. This steamboat was exactly like a decked scow. On the deck,
there were two little teakwood houses, with doors and windows. The
boiler was in the fore-end, and the machinery right astern. Over the
whole there was a light roof, supported on stanchions. The funnel
projected through that roof, and in front of the funnel a small cabin
built of light planks served for a pilot-house. It contained a couch,
two camp-stools, a loaded Martini-Henry leaning in one corner, a tiny
table, and the steering-wheel. It had a wide door in front and a broad
shutter at each side. All these were always thrown open, of course. I
spent my days perched up there on the extreme fore-end of that roof,
before the door. At night I slept, or tried to, on the couch. An
athletic black belonging to some coast tribe and educated by my poor
predecessor, was the helmsman. He sported a pair of brass earrings, wore
a blue cloth wrapper from the waist to the ankles, and thought all the
world of himself. He was the most unstable kind of fool I had ever seen.
He steered with no end of a swagger while you were by; but if he lost
sight of you, he became instantly the prey of an abject funk, and would
let that cripple of a steamboat get the upper hand of him in a minute.

``I was looking down at the sounding-pole, and feeling much annoyed to
see at each try a little more of it stick out of that river, when I saw
my poleman give up on the business suddenly, and stretch himself flat on
the deck, without even taking the trouble to haul his pole in. He kept
hold on it though, and it trailed in the water. At the same time the
fireman, whom I could also see below me, sat down abruptly before his
furnace and ducked his head. I was amazed. Then I had to look at the
river mighty quick, because there was a snag in the fairway. Sticks,
little sticks, were flying about---thick: they were whizzing before my
nose, dropping below me, striking behind me against my pilot-house. All
this time the river, the shore, the woods, were very quiet---perfectly
quiet. I could only hear the heavy splashing thump of the stern-wheel
and the patter of these things. We cleared the snag clumsily. Arrows, by
Jove! We were being shot at! I stepped in quickly to close the shutter
on the landside. That fool-helmsman, his hands on the spokes, was
lifting his knees high, stamping his feet, champing his mouth, like a
reined-in horse. Confound him! And we were staggering within ten feet of
the bank. I had to lean right out to swing the heavy shutter, and I saw
a face amongst the leaves on the level with my own, looking at me very
fierce and steady; and then suddenly, as though a veil had been removed
from my eyes, I made out, deep in the tangled gloom, naked breasts,
arms, legs, glaring eyes---the bush was swarming with human limbs in
movement, glistening of bronze colour. The twigs shook, swayed, and
rustled, the arrows flew out of them, and then the shutter came to.
`Steer her straight,' I said to the helmsman. He held his head rigid,
face forward; but his eyes rolled, he kept on lifting and setting down
his feet gently, his mouth foamed a little. `Keep quiet!' I said in a
fury. I might just as well have ordered a tree not to sway in the wind.
I darted out. Below me there was a great scuffle of feet on the iron
deck; confused exclamations; a voice screamed, `Can you turn back?' I
caught sight of a V-shaped ripple on the water ahead. What? Another
snag! A fusillade burst out under my feet. The pilgrims had opened with
their Winchesters, and were simply squirting lead into that bush. A
deuce of a lot of smoke came up and drove slowly forward. I swore at it.
Now I couldn't see the ripple or the snag either. I stood in the
doorway, peering, and the arrows came in swarms. They might have been
poisoned, but they looked as though they wouldn't kill a cat. The bush
began to howl. Our wood-cutters raised a warlike whoop; the report of a
rifle just at my back deafened me. I glanced over my shoulder, and the
pilot-house was yet full of noise and smoke when I made a dash at the
wheel. The fool-nigger had dropped everything, to throw the shutter open
and let off that Martini-Henry. He stood before the wide opening,
glaring, and I yelled at him to come back, while I straightened the
sudden twist out of that steamboat. There was no room to turn even if I
had wanted to, the snag was somewhere very near ahead in that confounded
smoke, there was no time to lose, so I just crowded her into the
bank---right into the bank, where I knew the water was deep.

``We tore slowly along the overhanging bushes in a whirl of broken twigs
and flying leaves. The fusillade below stopped short, as I had foreseen
it would when the squirts got empty. I threw my head back to a glinting
whizz that traversed the pilot-house, in at one shutter-hole and out at
the other. Looking past that mad helmsman, who was shaking the empty
rifle and yelling at the shore, I saw vague forms of men running bent
double, leaping, gliding, distinct, incomplete, evanescent. Something
big appeared in the air before the shutter, the rifle went overboard,
and the man stepped back swiftly, looked at me over his shoulder in an
extraordinary, profound, familiar manner, and fell upon my feet. The
side of his head hit the wheel twice, and the end of what appeared a
long cane clattered round and knocked over a little camp-stool. It
looked as though after wrenching that thing from somebody ashore he had
lost his balance in the effort. The thin smoke had blown away, we were
clear of the snag, and looking ahead I could see that in another hundred
yards or so I would be free to sheer off, away from the bank; but my
feet felt so very warm and wet that I had to look down. The man had
rolled on his back and stared straight up at me; both his hands clutched
that cane. It was the shaft of a spear that, either thrown or lunged
through the opening, had caught him in the side, just below the ribs;
the blade had gone in out of sight, after making a frightful gash; my
shoes were full; a pool of blood lay very still, gleaming dark-red under
the wheel; his eyes shone with an amazing lustre. The fusillade burst
out again. He looked at me anxiously, gripping the spear like something
precious, with an air of being afraid I would try to take it away from
him. I had to make an effort to free my eyes from his gaze and attend to
the steering. With one hand I felt above my head for the line of the
steam whistle, and jerked out screech after screech hurriedly. The
tumult of angry and warlike yells was checked instantly, and then from
the depths of the woods went out such a tremulous and prolonged wail of
mournful fear and utter despair as may be imagined to follow the flight
of the last hope from the earth. There was a great commotion in the
bush; the shower of arrows stopped, a few dropping shots rang out
sharply---then silence, in which the languid beat of the stern-wheel
came plainly to my ears. I put the helm hard a-starboard at the moment
when the pilgrim in pink pyjamas, very hot and agitated, appeared in the
doorway. `The manager sends me---' he began in an official tone, and
stopped short. `Good God!' he said, glaring at the wounded man.

``We two whites stood over him, and his lustrous and inquiring glance
enveloped us both. I declare it looked as though he would presently put
to us some questions in an understandable language; but he died without
uttering a sound, without moving a limb, without twitching a muscle.
Only in the very last moment, as though in response to some sign we
could not see, to some whisper we could not hear, he frowned heavily,
and that frown gave to his black death-mask an inconceivably sombre,
brooding, and menacing expression. The lustre of inquiring glance faded
swiftly into vacant glassiness. `Can you steer?' I asked the agent
eagerly. He looked very dubious; but I made a grab at his arm, and he
understood at once I meant him to steer whether or no. To tell you the
truth, I was morbidly anxious to change my shoes and socks. `He is
dead,' murmured the fellow, immensely impressed. `No doubt about it,'
said I, tugging like mad at the shoe-laces. `And by the way, I suppose
Mr. Kurtz is dead as well by this time.'

``For the moment that was the dominant thought. There was a sense of
extreme disappointment, as though I had found out I had been striving
after something altogether without a substance. I couldn't have been
more disgusted if I had travelled all this way for the sole purpose of
talking with Mr. Kurtz. Talking with\ldots{} I flung one shoe overboard,
and became aware that that was exactly what I had been looking forward
to---a talk with Kurtz. I made the strange discovery that I had never
imagined him as doing, you know, but as discoursing. I didn't say to
myself, `Now I will never see him,' or `Now I will never shake him by
the hand,' but, `Now I will never hear him.' The man presented himself
as a voice. Not of course that I did not connect him with some sort of
action. Hadn't I been told in all the tones of jealousy and admiration
that he had collected, bartered, swindled, or stolen more ivory than all
the other agents together? That was not the point. The point was in his
being a gifted creature, and that of all his gifts the one that stood
out preeminently, that carried with it a sense of real presence, was his
ability to talk, his words---the gift of expression, the bewildering,
the illuminating, the most exalted and the most contemptible, the
pulsating stream of light, or the deceitful flow from the heart of an
impenetrable darkness.

``The other shoe went flying unto the devil-god of that river. I
thought, `By Jove! it's all over. We are too late; he has vanished---the
gift has vanished, by means of some spear, arrow, or club. I will never
hear that chap speak after all'---and my sorrow had a startling
extravagance of emotion, even such as I had noticed in the howling
sorrow of these savages in the bush. I couldn't have felt more of lonely
desolation somehow, had I been robbed of a belief or had missed my
destiny in life\ldots{}. Why do you sigh in this beastly way, somebody?
Absurd? Well, absurd. Good Lord! mustn't a man ever---Here, give me some
tobacco.''\ldots{}

There was a pause of profound stillness, then a match flared, and
Marlow's lean face appeared, worn, hollow, with downward folds and
dropped eyelids, with an aspect of concentrated attention; and as he
took vigorous draws at his pipe, it seemed to retreat and advance out of
the night in the regular flicker of tiny flame. The match went out.

``Absurd!'' he cried. ``This is the worst of trying to tell\ldots{}.
Here you all are, each moored with two good addresses, like a hulk with
two anchors, a butcher round one corner, a policeman round another,
excellent appetites, and temperature normal---you hear---normal from
year's end to year's end. And you say, Absurd! Absurd be---exploded!
Absurd! My dear boys, what can you expect from a man who out of sheer
nervousness had just flung overboard a pair of new shoes! Now I think of
it, it is amazing I did not shed tears. I am, upon the whole, proud of
my fortitude. I was cut to the quick at the idea of having lost the
inestimable privilege of listening to the gifted Kurtz. Of course I was
wrong. The privilege was waiting for me. Oh, yes, I heard more than
enough. And I was right, too. A voice. He was very little more than a
voice. And I heard---him---it---this voice---other voices---all of them
were so little more than voices---and the memory of that time itself
lingers around me, impalpable, like a dying vibration of one immense
jabber, silly, atrocious, sordid, savage, or simply mean, without any
kind of sense. Voices, voices---even the girl herself---now---''

He was silent for a long time.

``I laid the ghost of his gifts at last with a lie,'' he began,
suddenly. ``Girl! What? Did I mention a girl? Oh, she is out of
it---completely. They---the women, I mean---are out of it---should be
out of it. We must help them to stay in that beautiful world of their
own, lest ours gets worse. Oh, she had to be out of it. You should have
heard the disinterred body of Mr. Kurtz saying, `My Intended.' You would
have perceived directly then how completely she was out of it. And the
lofty frontal bone of Mr. Kurtz! They say the hair goes on growing
sometimes, but this---ah---specimen, was impressively bald. The
wilderness had patted him on the head, and, behold, it was like a
ball---an ivory ball; it had caressed him, and---lo!---he had withered;
it had taken him, loved him, embraced him, got into his veins, consumed
his flesh, and sealed his soul to its own by the inconceivable
ceremonies of some devilish initiation. He was its spoiled and pampered
favourite. Ivory? I should think so. Heaps of it, stacks of it. The old
mud shanty was bursting with it. You would think there was not a single
tusk left either above or below the ground in the whole country. `Mostly
fossil,' the manager had remarked, disparagingly. It was no more fossil
than I am; but they call it fossil when it is dug up. It appears these
niggers do bury the tusks sometimes---but evidently they couldn't bury
this parcel deep enough to save the gifted Mr. Kurtz from his fate. We
filled the steamboat with it, and had to pile a lot on the deck. Thus he
could see and enjoy as long as he could see, because the appreciation of
this favour had remained with him to the last. You should have heard him
say, `My ivory.' Oh, yes, I heard him. `My Intended, my ivory, my
station, my river, my---' everything belonged to him. It made me hold my
breath in expectation of hearing the wilderness burst into a prodigious
peal of laughter that would shake the fixed stars in their places.
Everything belonged to him---but that was a trifle. The thing was to
know what he belonged to, how many powers of darkness claimed him for
their own. That was the reflection that made you creepy all over. It was
impossible---it was not good for one either---trying to imagine. He had
taken a high seat amongst the devils of the land---I mean literally. You
can't understand. How could you?---with solid pavement under your feet,
surrounded by kind neighbours ready to cheer you or to fall on you,
stepping delicately between the butcher and the policeman, in the holy
terror of scandal and gallows and lunatic asylums---how can you imagine
what particular region of the first ages a man's untrammelled feet may
take him into by the way of solitude---utter solitude without a
policeman---by the way of silence---utter silence, where no warning
voice of a kind neighbour can be heard whispering of public opinion?
These little things make all the great difference. When they are gone
you must fall back upon your own innate strength, upon your own capacity
for faithfulness. Of course you may be too much of a fool to go
wrong---too dull even to know you are being assaulted by the powers of
darkness. I take it, no fool ever made a bargain for his soul with the
devil; the fool is too much of a fool, or the devil too much of a
devil---I don't know which. Or you may be such a thunderingly exalted
creature as to be altogether deaf and blind to anything but heavenly
sights and sounds. Then the earth for you is only a standing place---and
whether to be like this is your loss or your gain I won't pretend to
say. But most of us are neither one nor the other. The earth for us is a
place to live in, where we must put up with sights, with sounds, with
smells, too, by Jove!---breathe dead hippo, so to speak, and not be
contaminated. And there, don't you see? Your strength comes in, the
faith in your ability for the digging of unostentatious holes to bury
the stuff in---your power of devotion, not to yourself, but to an
obscure, back-breaking business. And that's difficult enough. Mind, I am
not trying to excuse or even explain---I am trying to account to myself
for---for---Mr. Kurtz---for the shade of Mr. Kurtz. This initiated
wraith from the back of Nowhere honoured me with its amazing confidence
before it vanished altogether. This was because it could speak English
to me. The original Kurtz had been educated partly in England, and---as
he was good enough to say himself---his sympathies were in the right
place. His mother was half-English, his father was half-French. All
Europe contributed to the making of Kurtz; and by and by I learned that,
most appropriately, the International Society for the Suppression of
Savage Customs had intrusted him with the making of a report, for its
future guidance. And he had written it, too. I've seen it. I've read it.
It was eloquent, vibrating with eloquence, but too high-strung, I think.
Seventeen pages of close writing he had found time for! But this must
have been before his---let us say---nerves, went wrong, and caused him
to preside at certain midnight dances ending with unspeakable rites,
which---as far as I reluctantly gathered from what I heard at various
times---were offered up to him---do you understand?---to Mr. Kurtz
himself. But it was a beautiful piece of writing. The opening paragraph,
however, in the light of later information, strikes me now as ominous.
He began with the argument that we whites, from the point of development
we had arrived at, `must necessarily appear to them {[}savages{]} in the
nature of supernatural beings---we approach them with the might of a
deity,' and so on, and so on. `By the simple exercise of our will we can
exert a power for good practically unbounded,' etc., etc. From that
point he soared and took me with him. The peroration was magnificent,
though difficult to remember, you know. It gave me the notion of an
exotic Immensity ruled by an august Benevolence. It made me tingle with
enthusiasm. This was the unbounded power of eloquence---of words---of
burning noble words. There were no practical hints to interrupt the
magic current of phrases, unless a kind of note at the foot of the last
page, scrawled evidently much later, in an unsteady hand, may be
regarded as the exposition of a method. It was very simple, and at the
end of that moving appeal to every altruistic sentiment it blazed at
you, luminous and terrifying, like a flash of lightning in a serene sky:
`Exterminate all the brutes!' The curious part was that he had
apparently forgotten all about that valuable postscriptum, because,
later on, when he in a sense came to himself, he repeatedly entreated me
to take good care of `my pamphlet' (he called it), as it was sure to
have in the future a good influence upon his career. I had full
information about all these things, and, besides, as it turned out, I
was to have the care of his memory. I've done enough for it to give me
the indisputable right to lay it, if I choose, for an everlasting rest
in the dust-bin of progress, amongst all the sweepings and, figuratively
speaking, all the dead cats of civilization. But then, you see, I can't
choose. He won't be forgotten. Whatever he was, he was not common. He
had the power to charm or frighten rudimentary souls into an aggravated
witch-dance in his honour; he could also fill the small souls of the
pilgrims with bitter misgivings: he had one devoted friend at least, and
he had conquered one soul in the world that was neither rudimentary nor
tainted with self-seeking. No; I can't forget him, though I am not
prepared to affirm the fellow was exactly worth the life we lost in
getting to him. I missed my late helmsman awfully---I missed him even
while his body was still lying in the pilot-house. Perhaps you will
think it passing strange this regret for a savage who was no more
account than a grain of sand in a black Sahara. Well, don't you see, he
had done something, he had steered; for months I had him at my back---a
help---an instrument. It was a kind of partnership. He steered for
me---I had to look after him, I worried about his deficiencies, and thus
a subtle bond had been created, of which I only became aware when it was
suddenly broken. And the intimate profundity of that look he gave me
when he received his hurt remains to this day in my memory---like a
claim of distant kinship affirmed in a supreme moment.

``Poor fool! If he had only left that shutter alone. He had no
restraint, no restraint---just like Kurtz---a tree swayed by the wind.
As soon as I had put on a dry pair of slippers, I dragged him out, after
first jerking the spear out of his side, which operation I confess I
performed with my eyes shut tight. His heels leaped together over the
little doorstep; his shoulders were pressed to my breast; I hugged him
from behind desperately. Oh! he was heavy, heavy; heavier than any man
on earth, I should imagine. Then without more ado I tipped him
overboard. The current snatched him as though he had been a wisp of
grass, and I saw the body roll over twice before I lost sight of it for
ever. All the pilgrims and the manager were then congregated on the
awning-deck about the pilot-house, chattering at each other like a flock
of excited magpies, and there was a scandalized murmur at my heartless
promptitude. What they wanted to keep that body hanging about for I
can't guess. Embalm it, maybe. But I had also heard another, and a very
ominous, murmur on the deck below. My friends the wood-cutters were
likewise scandalized, and with a better show of reason---though I admit
that the reason itself was quite inadmissible. Oh, quite! I had made up
my mind that if my late helmsman was to be eaten, the fishes alone
should have him. He had been a very second-rate helmsman while alive,
but now he was dead he might have become a first-class temptation, and
possibly cause some startling trouble. Besides, I was anxious to take
the wheel, the man in pink pyjamas showing himself a hopeless duffer at
the business.

``This I did directly the simple funeral was over. We were going
half-speed, keeping right in the middle of the stream, and I listened to
the talk about me. They had given up Kurtz, they had given up the
station; Kurtz was dead, and the station had been burnt---and so
on---and so on. The red-haired pilgrim was beside himself with the
thought that at least this poor Kurtz had been properly avenged. `Say!
We must have made a glorious slaughter of them in the bush. Eh? What do
you think? Say?' He positively danced, the bloodthirsty little gingery
beggar. And he had nearly fainted when he saw the wounded man! I could
not help saying, `You made a glorious lot of smoke, anyhow.' I had seen,
from the way the tops of the bushes rustled and flew, that almost all
the shots had gone too high. You can't hit anything unless you take aim
and fire from the shoulder; but these chaps fired from the hip with
their eyes shut. The retreat, I maintained---and I was right---was
caused by the screeching of the steam whistle. Upon this they forgot
Kurtz, and began to howl at me with indignant protests.

``The manager stood by the wheel murmuring confidentially about the
necessity of getting well away down the river before dark at all events,
when I saw in the distance a clearing on the riverside and the outlines
of some sort of building. `What's this?' I asked. He clapped his hands
in wonder. `The station!' he cried. I edged in at once, still going
half-speed.

``Through my glasses I saw the slope of a hill interspersed with rare
trees and perfectly free from undergrowth. A long decaying building on
the summit was half buried in the high grass; the large holes in the
peaked roof gaped black from afar; the jungle and the woods made a
background. There was no enclosure or fence of any kind; but there had
been one apparently, for near the house half-a-dozen slim posts remained
in a row, roughly trimmed, and with their upper ends ornamented with
round carved balls. The rails, or whatever there had been between, had
disappeared. Of course the forest surrounded all that. The river-bank
was clear, and on the waterside I saw a white man under a hat like a
cart-wheel beckoning persistently with his whole arm. Examining the edge
of the forest above and below, I was almost certain I could see
movements---human forms gliding here and there. I steamed past
prudently, then stopped the engines and let her drift down. The man on
the shore began to shout, urging us to land. `We have been attacked,'
screamed the manager. `I know---I know. It's all right,' yelled back the
other, as cheerful as you please. `Come along. It's all right. I am
glad.'

``His aspect reminded me of something I had seen---something funny I had
seen somewhere. As I manoeuvred to get alongside, I was asking myself,
`What does this fellow look like?' Suddenly I got it. He looked like a
harlequin. His clothes had been made of some stuff that was brown
holland probably, but it was covered with patches all over, with bright
patches, blue, red, and yellow---patches on the back, patches on the
front, patches on elbows, on knees; coloured binding around his jacket,
scarlet edging at the bottom of his trousers; and the sunshine made him
look extremely gay and wonderfully neat withal, because you could see
how beautifully all this patching had been done. A beardless, boyish
face, very fair, no features to speak of, nose peeling, little blue
eyes, smiles and frowns chasing each other over that open countenance
like sunshine and shadow on a wind-swept plain. `Look out, captain!' he
cried; `there's a snag lodged in here last night.' What! Another snag? I
confess I swore shamefully. I had nearly holed my cripple, to finish off
that charming trip. The harlequin on the bank turned his little pug-nose
up to me. `You English?' he asked, all smiles. `Are you?' I shouted from
the wheel. The smiles vanished, and he shook his head as if sorry for my
disappointment. Then he brightened up. `Never mind!' he cried
encouragingly. `Are we in time?' I asked. `He is up there,' he replied,
with a toss of the head up the hill, and becoming gloomy all of a
sudden. His face was like the autumn sky, overcast one moment and bright
the next.

``When the manager, escorted by the pilgrims, all of them armed to the
teeth, had gone to the house this chap came on board. `I say, I don't
like this. These natives are in the bush,' I said. He assured me
earnestly it was all right. `They are simple people,' he added; `well, I
am glad you came. It took me all my time to keep them off.' `But you
said it was all right,' I cried. `Oh, they meant no harm,' he said; and
as I stared he corrected himself, `Not exactly.' Then vivaciously, `My
faith, your pilot-house wants a clean-up!' In the next breath he advised
me to keep enough steam on the boiler to blow the whistle in case of any
trouble. `One good screech will do more for you than all your rifles.
They are simple people,' he repeated. He rattled away at such a rate he
quite overwhelmed me. He seemed to be trying to make up for lots of
silence, and actually hinted, laughing, that such was the case. `Don't
you talk with Mr. Kurtz?' I said. `You don't talk with that man---you
listen to him,' he exclaimed with severe exaltation. `But now---' He
waved his arm, and in the twinkling of an eye was in the uttermost
depths of despondency. In a moment he came up again with a jump,
possessed himself of both my hands, shook them continuously, while he
gabbled: `Brother sailor\ldots{} honour\ldots{} pleasure\ldots{}
delight\ldots{} introduce myself\ldots{} Russian\ldots{} son of an
arch-priest\ldots{} Government of Tambov\ldots{} What? Tobacco! English
tobacco; the excellent English tobacco! Now, that's brotherly. Smoke?
Where's a sailor that does not smoke?''

``The pipe soothed him, and gradually I made out he had run away from
school, had gone to sea in a Russian ship; ran away again; served some
time in English ships; was now reconciled with the arch-priest. He made
a point of that. `But when one is young one must see things, gather
experience, ideas; enlarge the mind.' `Here!' I interrupted. `You can
never tell! Here I met Mr. Kurtz,' he said, youthfully solemn and
reproachful. I held my tongue after that. It appears he had persuaded a
Dutch trading-house on the coast to fit him out with stores and goods,
and had started for the interior with a light heart and no more idea of
what would happen to him than a baby. He had been wandering about that
river for nearly two years alone, cut off from everybody and everything.
`I am not so young as I look. I am twenty-five,' he said. `At first old
Van Shuyten would tell me to go to the devil,' he narrated with keen
enjoyment; `but I stuck to him, and talked and talked, till at last he
got afraid I would talk the hind-leg off his favourite dog, so he gave
me some cheap things and a few guns, and told me he hoped he would never
see my face again. Good old Dutchman, Van Shuyten. I've sent him one
small lot of ivory a year ago, so that he can't call me a little thief
when I get back. I hope he got it. And for the rest I don't care. I had
some wood stacked for you. That was my old house. Did you see?'

``I gave him Towson's book. He made as though he would kiss me, but
restrained himself. `The only book I had left, and I thought I had lost
it,' he said, looking at it ecstatically. `So many accidents happen to a
man going about alone, you know. Canoes get upset sometimes---and
sometimes you've got to clear out so quick when the people get angry.'
He thumbed the pages. `You made notes in Russian?' I asked. He nodded.
`I thought they were written in cipher,' I said. He laughed, then became
serious. `I had lots of trouble to keep these people off,' he said. `Did
they want to kill you?' I asked. `Oh, no!' he cried, and checked
himself. `Why did they attack us?' I pursued. He hesitated, then said
shamefacedly, `They don't want him to go.' `Don't they?' I said
curiously. He nodded a nod full of mystery and wisdom. `I tell you,' he
cried, `this man has enlarged my mind.' He opened his arms wide, staring
at me with his little blue eyes that were perfectly round.''

\mychapter{3}{III}\label{iii}

``I looked at him, lost in astonishment. There he was before me, in
motley, as though he had absconded from a troupe of mimes, enthusiastic,
fabulous. His very existence was improbable, inexplicable, and
altogether bewildering. He was an insoluble problem. It was
inconceivable how he had existed, how he had succeeded in getting so
far, how he had managed to remain---why he did not instantly disappear.
`I went a little farther,' he said, `then still a little farther---till
I had gone so far that I don't know how I'll ever get back. Never mind.
Plenty time. I can manage. You take Kurtz away quick---quick---I tell
you.' The glamour of youth enveloped his parti-coloured rags, his
destitution, his loneliness, the essential desolation of his futile
wanderings. For months---for years---his life hadn't been worth a day's
purchase; and there he was gallantly, thoughtlessly alive, to all
appearances indestructible solely by the virtue of his few years and of
his unreflecting audacity. I was seduced into something like
admiration---like envy. Glamour urged him on, glamour kept him
unscathed. He surely wanted nothing from the wilderness but space to
breathe in and to push on through. His need was to exist, and to move
onwards at the greatest possible risk, and with a maximum of privation.
If the absolutely pure, uncalculating, unpractical spirit of adventure
had ever ruled a human being, it ruled this bepatched youth. I almost
envied him the possession of this modest and clear flame. It seemed to
have consumed all thought of self so completely, that even while he was
talking to you, you forgot that it was he---the man before your
eyes---who had gone through these things. I did not envy him his
devotion to Kurtz, though. He had not meditated over it. It came to him,
and he accepted it with a sort of eager fatalism. I must say that to me
it appeared about the most dangerous thing in every way he had come upon
so far.

``They had come together unavoidably, like two ships becalmed near each
other, and lay rubbing sides at last. I suppose Kurtz wanted an
audience, because on a certain occasion, when encamped in the forest,
they had talked all night, or more probably Kurtz had talked. `We talked
of everything,' he said, quite transported at the recollection. `I
forgot there was such a thing as sleep. The night did not seem to last
an hour. Everything! Everything!\ldots{} Of love, too.' `Ah, he talked
to you of love!' I said, much amused. `It isn't what you think,' he
cried, almost passionately. `It was in general. He made me see
things---things.'

``He threw his arms up. We were on deck at the time, and the headman of
my wood-cutters, lounging near by, turned upon him his heavy and
glittering eyes. I looked around, and I don't know why, but I assure you
that never, never before, did this land, this river, this jungle, the
very arch of this blazing sky, appear to me so hopeless and so dark, so
impenetrable to human thought, so pitiless to human weakness. `And, ever
since, you have been with him, of course?' I said.

``On the contrary. It appears their intercourse had been very much
broken by various causes. He had, as he informed me proudly, managed to
nurse Kurtz through two illnesses (he alluded to it as you would to some
risky feat), but as a rule Kurtz wandered alone, far in the depths of
the forest. `Very often coming to this station, I had to wait days and
days before he would turn up,' he said. `Ah, it was worth waiting
for!---sometimes.' `What was he doing? exploring or what?' I asked. `Oh,
yes, of course'; he had discovered lots of villages, a lake, too---he
did not know exactly in what direction; it was dangerous to inquire too
much---but mostly his expeditions had been for ivory. `But he had no
goods to trade with by that time,' I objected. `There's a good lot of
cartridges left even yet,' he answered, looking away. `To speak plainly,
he raided the country,' I said. He nodded. `Not alone, surely!' He
muttered something about the villages round that lake. `Kurtz got the
tribe to follow him, did he?' I suggested. He fidgeted a little. `They
adored him,' he said. The tone of these words was so extraordinary that
I looked at him searchingly. It was curious to see his mingled eagerness
and reluctance to speak of Kurtz. The man filled his life, occupied his
thoughts, swayed his emotions. `What can you expect?' he burst out; `he
came to them with thunder and lightning, you know---and they had never
seen anything like it---and very terrible. He could be very terrible.
You can't judge Mr. Kurtz as you would an ordinary man. No, no, no!
Now---just to give you an idea---I don't mind telling you, he wanted to
shoot me, too, one day---but I don't judge him.' `Shoot you!' I cried
`What for?' `Well, I had a small lot of ivory the chief of that village
near my house gave me. You see I used to shoot game for them. Well, he
wanted it, and wouldn't hear reason. He declared he would shoot me
unless I gave him the ivory and then cleared out of the country, because
he could do so, and had a fancy for it, and there was nothing on earth
to prevent him killing whom he jolly well pleased. And it was true, too.
I gave him the ivory. What did I care! But I didn't clear out. No, no. I
couldn't leave him. I had to be careful, of course, till we got friendly
again for a time. He had his second illness then. Afterwards I had to
keep out of the way; but I didn't mind. He was living for the most part
in those villages on the lake. When he came down to the river, sometimes
he would take to me, and sometimes it was better for me to be careful.
This man suffered too much. He hated all this, and somehow he couldn't
get away. When I had a chance I begged him to try and leave while there
was time; I offered to go back with him. And he would say yes, and then
he would remain; go off on another ivory hunt; disappear for weeks;
forget himself amongst these people---forget himself---you know.' `Why!
he's mad,' I said. He protested indignantly. Mr. Kurtz couldn't be mad.
If I had heard him talk, only two days ago, I wouldn't dare hint at such
a thing\ldots{}. I had taken up my binoculars while we talked, and was
looking at the shore, sweeping the limit of the forest at each side and
at the back of the house. The consciousness of there being people in
that bush, so silent, so quiet---as silent and quiet as the ruined house
on the hill---made me uneasy. There was no sign on the face of nature of
this amazing tale that was not so much told as suggested to me in
desolate exclamations, completed by shrugs, in interrupted phrases, in
hints ending in deep sighs. The woods were unmoved, like a mask---heavy,
like the closed door of a prison---they looked with their air of hidden
knowledge, of patient expectation, of unapproachable silence. The
Russian was explaining to me that it was only lately that Mr. Kurtz had
come down to the river, bringing along with him all the fighting men of
that lake tribe. He had been absent for several months---getting himself
adored, I suppose---and had come down unexpectedly, with the intention
to all appearance of making a raid either across the river or down
stream. Evidently the appetite for more ivory had got the better of
the---what shall I say?---less material aspirations. However he had got
much worse suddenly. `I heard he was lying helpless, and so I came
up---took my chance,' said the Russian. `Oh, he is bad, very bad.' I
directed my glass to the house. There were no signs of life, but there
was the ruined roof, the long mud wall peeping above the grass, with
three little square window-holes, no two of the same size; all this
brought within reach of my hand, as it were. And then I made a brusque
movement, and one of the remaining posts of that vanished fence leaped
up in the field of my glass. You remember I told you I had been struck
at the distance by certain attempts at ornamentation, rather remarkable
in the ruinous aspect of the place. Now I had suddenly a nearer view,
and its first result was to make me throw my head back as if before a
blow. Then I went carefully from post to post with my glass, and I saw
my mistake. These round knobs were not ornamental but symbolic; they
were expressive and puzzling, striking and disturbing---food for thought
and also for vultures if there had been any looking down from the sky;
but at all events for such ants as were industrious enough to ascend the
pole. They would have been even more impressive, those heads on the
stakes, if their faces had not been turned to the house. Only one, the
first I had made out, was facing my way. I was not so shocked as you may
think. The start back I had given was really nothing but a movement of
surprise. I had expected to see a knob of wood there, you know. I
returned deliberately to the first I had seen---and there it was, black,
dried, sunken, with closed eyelids---a head that seemed to sleep at the
top of that pole, and, with the shrunken dry lips showing a narrow white
line of the teeth, was smiling, too, smiling continuously at some
endless and jocose dream of that eternal slumber.

``I am not disclosing any trade secrets. In fact, the manager said
afterwards that Mr. Kurtz's methods had ruined the district. I have no
opinion on that point, but I want you clearly to understand that there
was nothing exactly profitable in these heads being there. They only
showed that Mr. Kurtz lacked restraint in the gratification of his
various lusts, that there was something wanting in him---some small
matter which, when the pressing need arose, could not be found under his
magnificent eloquence. Whether he knew of this deficiency himself I
can't say. I think the knowledge came to him at last---only at the very
last. But the wilderness had found him out early, and had taken on him a
terrible vengeance for the fantastic invasion. I think it had whispered
to him things about himself which he did not know, things of which he
had no conception till he took counsel with this great solitude---and
the whisper had proved irresistibly fascinating. It echoed loudly within
him because he was hollow at the core\ldots{}. I put down the glass, and
the head that had appeared near enough to be spoken to seemed at once to
have leaped away from me into inaccessible distance.

``The admirer of Mr. Kurtz was a bit crestfallen. In a hurried,
indistinct voice he began to assure me he had not dared to take
these---say, symbols---down. He was not afraid of the natives; they
would not stir till Mr. Kurtz gave the word. His ascendancy was
extraordinary. The camps of these people surrounded the place, and the
chiefs came every day to see him. They would crawl\ldots{}. `I don't
want to know anything of the ceremonies used when approaching Mr.
Kurtz,' I shouted. Curious, this feeling that came over me that such
details would be more intolerable than those heads drying on the stakes
under Mr. Kurtz's windows. After all, that was only a savage sight,
while I seemed at one bound to have been transported into some lightless
region of subtle horrors, where pure, uncomplicated savagery was a
positive relief, being something that had a right to
exist---obviously---in the sunshine. The young man looked at me with
surprise. I suppose it did not occur to him that Mr. Kurtz was no idol
of mine. He forgot I hadn't heard any of these splendid monologues on,
what was it? on love, justice, conduct of life---or what not. If it had
come to crawling before Mr. Kurtz, he crawled as much as the veriest
savage of them all. I had no idea of the conditions, he said: these
heads were the heads of rebels. I shocked him excessively by laughing.
Rebels! What would be the next definition I was to hear? There had been
enemies, criminals, workers---and these were rebels. Those rebellious
heads looked very subdued to me on their sticks. `You don't know how
such a life tries a man like Kurtz,' cried Kurtz's last disciple. `Well,
and you?' I said. `I! I! I am a simple man. I have no great thoughts. I
want nothing from anybody. How can you compare me to\ldots{}?' His
feelings were too much for speech, and suddenly he broke down. `I don't
understand,' he groaned. `I've been doing my best to keep him alive, and
that's enough. I had no hand in all this. I have no abilities. There
hasn't been a drop of medicine or a mouthful of invalid food for months
here. He was shamefully abandoned. A man like this, with such ideas.
Shamefully! Shamefully! I---I---haven't slept for the last ten
nights\ldots{}'

``His voice lost itself in the calm of the evening. The long shadows of
the forest had slipped downhill while we talked, had gone far beyond the
ruined hovel, beyond the symbolic row of stakes. All this was in the
gloom, while we down there were yet in the sunshine, and the stretch of
the river abreast of the clearing glittered in a still and dazzling
splendour, with a murky and overshadowed bend above and below. Not a
living soul was seen on the shore. The bushes did not rustle.

``Suddenly round the corner of the house a group of men appeared, as
though they had come up from the ground. They waded waist-deep in the
grass, in a compact body, bearing an improvised stretcher in their
midst. Instantly, in the emptiness of the landscape, a cry arose whose
shrillness pierced the still air like a sharp arrow flying straight to
the very heart of the land; and, as if by enchantment, streams of human
beings---of naked human beings---with spears in their hands, with bows,
with shields, with wild glances and savage movements, were poured into
the clearing by the dark-faced and pensive forest. The bushes shook, the
grass swayed for a time, and then everything stood still in attentive
immobility.

```Now, if he does not say the right thing to them we are all done for,'
said the Russian at my elbow. The knot of men with the stretcher had
stopped, too, halfway to the steamer, as if petrified. I saw the man on
the stretcher sit up, lank and with an uplifted arm, above the shoulders
of the bearers. `Let us hope that the man who can talk so well of love
in general will find some particular reason to spare us this time,' I
said. I resented bitterly the absurd danger of our situation, as if to
be at the mercy of that atrocious phantom had been a dishonouring
necessity. I could not hear a sound, but through my glasses I saw the
thin arm extended commandingly, the lower jaw moving, the eyes of that
apparition shining darkly far in its bony head that nodded with
grotesque jerks. Kurtz---Kurtz---that means short in German---don't it?
Well, the name was as true as everything else in his life---and death.
He looked at least seven feet long. His covering had fallen off, and his
body emerged from it pitiful and appalling as from a winding-sheet. I
could see the cage of his ribs all astir, the bones of his arm waving.
It was as though an animated image of death carved out of old ivory had
been shaking its hand with menaces at a motionless crowd of men made of
dark and glittering bronze. I saw him open his mouth wide---it gave him
a weirdly voracious aspect, as though he had wanted to swallow all the
air, all the earth, all the men before him. A deep voice reached me
faintly. He must have been shouting. He fell back suddenly. The
stretcher shook as the bearers staggered forward again, and almost at
the same time I noticed that the crowd of savages was vanishing without
any perceptible movement of retreat, as if the forest that had ejected
these beings so suddenly had drawn them in again as the breath is drawn
in a long aspiration.

``Some of the pilgrims behind the stretcher carried his arms---two
shot-guns, a heavy rifle, and a light revolver-carbine---the
thunderbolts of that pitiful Jupiter. The manager bent over him
murmuring as he walked beside his head. They laid him down in one of the
little cabins---just a room for a bed place and a camp-stool or two, you
know. We had brought his belated correspondence, and a lot of torn
envelopes and open letters littered his bed. His hand roamed feebly
amongst these papers. I was struck by the fire of his eyes and the
composed languor of his expression. It was not so much the exhaustion of
disease. He did not seem in pain. This shadow looked satiated and calm,
as though for the moment it had had its fill of all the emotions.

``He rustled one of the letters, and looking straight in my face said,
`I am glad.' Somebody had been writing to him about me. These special
recommendations were turning up again. The volume of tone he emitted
without effort, almost without the trouble of moving his lips, amazed
me. A voice! a voice! It was grave, profound, vibrating, while the man
did not seem capable of a whisper. However, he had enough strength in
him---factitious no doubt---to very nearly make an end of us, as you
shall hear directly.

``The manager appeared silently in the doorway; I stepped out at once
and he drew the curtain after me. The Russian, eyed curiously by the
pilgrims, was staring at the shore. I followed the direction of his
glance.

``Dark human shapes could be made out in the distance, flitting
indistinctly against the gloomy border of the forest, and near the river
two bronze figures, leaning on tall spears, stood in the sunlight under
fantastic head-dresses of spotted skins, warlike and still in statuesque
repose. And from right to left along the lighted shore moved a wild and
gorgeous apparition of a woman.

``She walked with measured steps, draped in striped and fringed cloths,
treading the earth proudly, with a slight jingle and flash of barbarous
ornaments. She carried her head high; her hair was done in the shape of
a helmet; she had brass leggings to the knee, brass wire gauntlets to
the elbow, a crimson spot on her tawny cheek, innumerable necklaces of
glass beads on her neck; bizarre things, charms, gifts of witch-men,
that hung about her, glittered and trembled at every step. She must have
had the value of several elephant tusks upon her. She was savage and
superb, wild-eyed and magnificent; there was something ominous and
stately in her deliberate progress. And in the hush that had fallen
suddenly upon the whole sorrowful land, the immense wilderness, the
colossal body of the fecund and mysterious life seemed to look at her,
pensive, as though it had been looking at the image of its own tenebrous
and passionate soul.

``She came abreast of the steamer, stood still, and faced us. Her long
shadow fell to the water's edge. Her face had a tragic and fierce aspect
of wild sorrow and of dumb pain mingled with the fear of some
struggling, half-shaped resolve. She stood looking at us without a stir,
and like the wilderness itself, with an air of brooding over an
inscrutable purpose. A whole minute passed, and then she made a step
forward. There was a low jingle, a glint of yellow metal, a sway of
fringed draperies, and she stopped as if her heart had failed her. The
young fellow by my side growled. The pilgrims murmured at my back. She
looked at us all as if her life had depended upon the unswerving
steadiness of her glance. Suddenly she opened her bared arms and threw
them up rigid above her head, as though in an uncontrollable desire to
touch the sky, and at the same time the swift shadows darted out on the
earth, swept around on the river, gathering the steamer into a shadowy
embrace. A formidable silence hung over the scene.

``She turned away slowly, walked on, following the bank, and passed into
the bushes to the left. Once only her eyes gleamed back at us in the
dusk of the thickets before she disappeared.

```If she had offered to come aboard I really think I would have tried
to shoot her,' said the man of patches, nervously. `I have been risking
my life every day for the last fortnight to keep her out of the house.
She got in one day and kicked up a row about those miserable rags I
picked up in the storeroom to mend my clothes with. I wasn't decent. At
least it must have been that, for she talked like a fury to Kurtz for an
hour, pointing at me now and then. I don't understand the dialect of
this tribe. Luckily for me, I fancy Kurtz felt too ill that day to care,
or there would have been mischief. I don't understand\ldots{}. No---it's
too much for me. Ah, well, it's all over now.'

``At this moment I heard Kurtz's deep voice behind the curtain: `Save
me!---save the ivory, you mean. Don't tell me. Save \emph{me!} Why, I've
had to save you. You are interrupting my plans now. Sick! Sick! Not so
sick as you would like to believe. Never mind. I'll carry my ideas out
yet---I will return. I'll show you what can be done. You with your
little peddling notions---you are interfering with me. I will return.
I\ldots{}.'

``The manager came out. He did me the honour to take me under the arm
and lead me aside. `He is very low, very low,' he said. He considered it
necessary to sigh, but neglected to be consistently sorrowful. `We have
done all we could for him---haven't we? But there is no disguising the
fact, Mr. Kurtz has done more harm than good to the Company. He did not
see the time was not ripe for vigorous action. Cautiously,
cautiously---that's my principle. We must be cautious yet. The district
is closed to us for a time. Deplorable! Upon the whole, the trade will
suffer. I don't deny there is a remarkable quantity of ivory---mostly
fossil. We must save it, at all events---but look how precarious the
position is---and why? Because the method is unsound.' `Do you,' said I,
looking at the shore, `call it ``unsound method?''\,' `Without doubt,'
he exclaimed hotly. `Don't you?'\ldots{} `No method at all,' I murmured
after a while. `Exactly,' he exulted. `I anticipated this. Shows a
complete want of judgment. It is my duty to point it out in the proper
quarter.' `Oh,' said I, `that fellow---what's his name?---the
brickmaker, will make a readable report for you.' He appeared confounded
for a moment. It seemed to me I had never breathed an atmosphere so
vile, and I turned mentally to Kurtz for relief---positively for relief.
`Nevertheless I think Mr. Kurtz is a remarkable man,' I said with
emphasis. He started, dropped on me a heavy glance, said very quietly,
`he \emph{was},' and turned his back on me. My hour of favour was over;
I found myself lumped along with Kurtz as a partisan of methods for
which the time was not ripe: I was unsound! Ah! but it was something to
have at least a choice of nightmares.

``I had turned to the wilderness really, not to Mr. Kurtz, who, I was
ready to admit, was as good as buried. And for a moment it seemed to me
as if I also were buried in a vast grave full of unspeakable secrets. I
felt an intolerable weight oppressing my breast, the smell of the damp
earth, the unseen presence of victorious corruption, the darkness of an
impenetrable night\ldots{}. The Russian tapped me on the shoulder. I
heard him mumbling and stammering something about `brother
seaman---couldn't conceal---knowledge of matters that would affect Mr.
Kurtz's reputation.' I waited. For him evidently Mr. Kurtz was not in
his grave; I suspect that for him Mr. Kurtz was one of the immortals.
`Well!' said I at last, `speak out. As it happens, I am Mr. Kurtz's
friend---in a way.'

``He stated with a good deal of formality that had we not been `of the
same profession,' he would have kept the matter to himself without
regard to consequences. `He suspected there was an active ill-will
towards him on the part of these white men that---' `You are right,' I
said, remembering a certain conversation I had overheard. `The manager
thinks you ought to be hanged.' He showed a concern at this intelligence
which amused me at first. `I had better get out of the way quietly,' he
said earnestly. `I can do no more for Kurtz now, and they would soon
find some excuse. What's to stop them? There's a military post three
hundred miles from here.' `Well, upon my word,' said I, `perhaps you had
better go if you have any friends amongst the savages near by.'
`Plenty,' he said. `They are simple people---and I want nothing, you
know.' He stood biting his lip, then: `I don't want any harm to happen
to these whites here, but of course I was thinking of Mr. Kurtz's
reputation---but you are a brother seaman and---' `All right,' said I,
after a time. `Mr. Kurtz's reputation is safe with me.' I did not know
how truly I spoke.

``He informed me, lowering his voice, that it was Kurtz who had ordered
the attack to be made on the steamer. `He hated sometimes the idea of
being taken away---and then again\ldots{}. But I don't understand these
matters. I am a simple man. He thought it would scare you away---that
you would give it up, thinking him dead. I could not stop him. Oh, I had
an awful time of it this last month.' `Very well,' I said. `He is all
right now.' `Ye-e-es,' he muttered, not very convinced apparently.
`Thanks,' said I; `I shall keep my eyes open.' `But quiet-eh?' he urged
anxiously. `It would be awful for his reputation if anybody here---' I
promised a complete discretion with great gravity. `I have a canoe and
three black fellows waiting not very far. I am off. Could you give me a
few Martini-Henry cartridges?' I could, and did, with proper secrecy. He
helped himself, with a wink at me, to a handful of my tobacco. `Between
sailors---you know---good English tobacco.' At the door of the
pilot-house he turned round---`I say, haven't you a pair of shoes you
could spare?' He raised one leg. `Look.' The soles were tied with
knotted strings sandalwise under his bare feet. I rooted out an old
pair, at which he looked with admiration before tucking it under his
left arm. One of his pockets (bright red) was bulging with cartridges,
from the other (dark blue) peeped `Towson's Inquiry,' etc., etc. He
seemed to think himself excellently well equipped for a renewed
encounter with the wilderness. `Ah! I'll never, never meet such a man
again. You ought to have heard him recite poetry---his own, too, it was,
he told me. Poetry!' He rolled his eyes at the recollection of these
delights. `Oh, he enlarged my mind!' `Good-bye,' said I. He shook hands
and vanished in the night. Sometimes I ask myself whether I had ever
really seen him---whether it was possible to meet such a
phenomenon!\ldots{}

``When I woke up shortly after midnight his warning came to my mind with
its hint of danger that seemed, in the starred darkness, real enough to
make me get up for the purpose of having a look round. On the hill a big
fire burned, illuminating fitfully a crooked corner of the
station-house. One of the agents with a picket of a few of our blacks,
armed for the purpose, was keeping guard over the ivory; but deep within
the forest, red gleams that wavered, that seemed to sink and rise from
the ground amongst confused columnar shapes of intense blackness, showed
the exact position of the camp where Mr. Kurtz's adorers were keeping
their uneasy vigil. The monotonous beating of a big drum filled the air
with muffled shocks and a lingering vibration. A steady droning sound of
many men chanting each to himself some weird incantation came out from
the black, flat wall of the woods as the humming of bees comes out of a
hive, and had a strange narcotic effect upon my half-awake senses. I
believe I dozed off leaning over the rail, till an abrupt burst of
yells, an overwhelming outbreak of a pent-up and mysterious frenzy, woke
me up in a bewildered wonder. It was cut short all at once, and the low
droning went on with an effect of audible and soothing silence. I
glanced casually into the little cabin. A light was burning within, but
Mr. Kurtz was not there.

``I think I would have raised an outcry if I had believed my eyes. But I
didn't believe them at first---the thing seemed so impossible. The fact
is I was completely unnerved by a sheer blank fright, pure abstract
terror, unconnected with any distinct shape of physical danger. What
made this emotion so overpowering was---how shall I define it?---the
moral shock I received, as if something altogether monstrous,
intolerable to thought and odious to the soul, had been thrust upon me
unexpectedly. This lasted of course the merest fraction of a second, and
then the usual sense of commonplace, deadly danger, the possibility of a
sudden onslaught and massacre, or something of the kind, which I saw
impending, was positively welcome and composing. It pacified me, in
fact, so much that I did not raise an alarm.

``There was an agent buttoned up inside an ulster and sleeping on a
chair on deck within three feet of me. The yells had not awakened him;
he snored very slightly; I left him to his slumbers and leaped ashore. I
did not betray Mr. Kurtz---it was ordered I should never betray him---it
was written I should be loyal to the nightmare of my choice. I was
anxious to deal with this shadow by myself alone---and to this day I
don't know why I was so jealous of sharing with any one the peculiar
blackness of that experience.

``As soon as I got on the bank I saw a trail---a broad trail through the
grass. I remember the exultation with which I said to myself, `He can't
walk---he is crawling on all-fours---I've got him.' The grass was wet
with dew. I strode rapidly with clenched fists. I fancy I had some vague
notion of falling upon him and giving him a drubbing. I don't know. I
had some imbecile thoughts. The knitting old woman with the cat obtruded
herself upon my memory as a most improper person to be sitting at the
other end of such an affair. I saw a row of pilgrims squirting lead in
the air out of Winchesters held to the hip. I thought I would never get
back to the steamer, and imagined myself living alone and unarmed in the
woods to an advanced age. Such silly things---you know. And I remember I
confounded the beat of the drum with the beating of my heart, and was
pleased at its calm regularity.

``I kept to the track though---then stopped to listen. The night was
very clear; a dark blue space, sparkling with dew and starlight, in
which black things stood very still. I thought I could see a kind of
motion ahead of me. I was strangely cocksure of everything that night. I
actually left the track and ran in a wide semicircle (I verily believe
chuckling to myself) so as to get in front of that stir, of that motion
I had seen---if indeed I had seen anything. I was circumventing Kurtz as
though it had been a boyish game.

``I came upon him, and, if he had not heard me coming, I would have
fallen over him, too, but he got up in time. He rose, unsteady, long,
pale, indistinct, like a vapour exhaled by the earth, and swayed
slightly, misty and silent before me; while at my back the fires loomed
between the trees, and the murmur of many voices issued from the forest.
I had cut him off cleverly; but when actually confronting him I seemed
to come to my senses, I saw the danger in its right proportion. It was
by no means over yet. Suppose he began to shout? Though he could hardly
stand, there was still plenty of vigour in his voice. `Go away---hide
yourself,' he said, in that profound tone. It was very awful. I glanced
back. We were within thirty yards from the nearest fire. A black figure
stood up, strode on long black legs, waving long black arms, across the
glow. It had horns---antelope horns, I think---on its head. Some
sorcerer, some witch-man, no doubt: it looked fiendlike enough. `Do you
know what you are doing?' I whispered. `Perfectly,' he answered, raising
his voice for that single word: it sounded to me far off and yet loud,
like a hail through a speaking-trumpet. `If he makes a row we are lost,'
I thought to myself. This clearly was not a case for fisticuffs, even
apart from the very natural aversion I had to beat that Shadow---this
wandering and tormented thing. `You will be lost,' I said---`utterly
lost.' One gets sometimes such a flash of inspiration, you know. I did
say the right thing, though indeed he could not have been more
irretrievably lost than he was at this very moment, when the foundations
of our intimacy were being laid---to endure---to endure---even to the
end---even beyond.

```I had immense plans,' he muttered irresolutely. `Yes,' said I; `but
if you try to shout I'll smash your head with---' There was not a stick
or a stone near. `I will throttle you for good,' I corrected myself. `I
was on the threshold of great things,' he pleaded, in a voice of
longing, with a wistfulness of tone that made my blood run cold. `And
now for this stupid scoundrel---' `Your success in Europe is assured in
any case,' I affirmed steadily. I did not want to have the throttling of
him, you understand---and indeed it would have been very little use for
any practical purpose. I tried to break the spell---the heavy, mute
spell of the wilderness---that seemed to draw him to its pitiless breast
by the awakening of forgotten and brutal instincts, by the memory of
gratified and monstrous passions. This alone, I was convinced, had
driven him out to the edge of the forest, to the bush, towards the gleam
of fires, the throb of drums, the drone of weird incantations; this
alone had beguiled his unlawful soul beyond the bounds of permitted
aspirations. And, don't you see, the terror of the position was not in
being knocked on the head---though I had a very lively sense of that
danger, too---but in this, that I had to deal with a being to whom I
could not appeal in the name of anything high or low. I had, even like
the niggers, to invoke him---himself---his own exalted and incredible
degradation. There was nothing either above or below him, and I knew it.
He had kicked himself loose of the earth. Confound the man! he had
kicked the very earth to pieces. He was alone, and I before him did not
know whether I stood on the ground or floated in the air. I've been
telling you what we said---repeating the phrases we pronounced---but
what's the good? They were common everyday words---the familiar, vague
sounds exchanged on every waking day of life. But what of that? They had
behind them, to my mind, the terrific suggestiveness of words heard in
dreams, of phrases spoken in nightmares. Soul! If anybody ever struggled
with a soul, I am the man. And I wasn't arguing with a lunatic either.
Believe me or not, his intelligence was perfectly clear---concentrated,
it is true, upon himself with horrible intensity, yet clear; and therein
was my only chance---barring, of course, the killing him there and then,
which wasn't so good, on account of unavoidable noise. But his soul was
mad. Being alone in the wilderness, it had looked within itself, and, by
heavens! I tell you, it had gone mad. I had---for my sins, I
suppose---to go through the ordeal of looking into it myself. No
eloquence could have been so withering to one's belief in mankind as his
final burst of sincerity. He struggled with himself, too. I saw it---I
heard it. I saw the inconceivable mystery of a soul that knew no
restraint, no faith, and no fear, yet struggling blindly with itself. I
kept my head pretty well; but when I had him at last stretched on the
couch, I wiped my forehead, while my legs shook under me as though I had
carried half a ton on my back down that hill. And yet I had only
supported him, his bony arm clasped round my neck---and he was not much
heavier than a child.

``When next day we left at noon, the crowd, of whose presence behind the
curtain of trees I had been acutely conscious all the time, flowed out
of the woods again, filled the clearing, covered the slope with a mass
of naked, breathing, quivering, bronze bodies. I steamed up a bit, then
swung down stream, and two thousand eyes followed the evolutions of the
splashing, thumping, fierce river-demon beating the water with its
terrible tail and breathing black smoke into the air. In front of the
first rank, along the river, three men, plastered with bright red earth
from head to foot, strutted to and fro restlessly. When we came abreast
again, they faced the river, stamped their feet, nodded their horned
heads, swayed their scarlet bodies; they shook towards the fierce
river-demon a bunch of black feathers, a mangy skin with a pendent
tail---something that looked a dried gourd; they shouted periodically
together strings of amazing words that resembled no sounds of human
language; and the deep murmurs of the crowd, interrupted suddenly, were
like the responses of some satanic litany.

``We had carried Kurtz into the pilot-house: there was more air there.
Lying on the couch, he stared through the open shutter. There was an
eddy in the mass of human bodies, and the woman with helmeted head and
tawny cheeks rushed out to the very brink of the stream. She put out her
hands, shouted something, and all that wild mob took up the shout in a
roaring chorus of articulated, rapid, breathless utterance.

```Do you understand this?' I asked.

``He kept on looking out past me with fiery, longing eyes, with a
mingled expression of wistfulness and hate. He made no answer, but I saw
a smile, a smile of indefinable meaning, appear on his colourless lips
that a moment after twitched convulsively. `Do I not?' he said slowly,
gasping, as if the words had been torn out of him by a supernatural
power.

``I pulled the string of the whistle, and I did this because I saw the
pilgrims on deck getting out their rifles with an air of anticipating a
jolly lark. At the sudden screech there was a movement of abject terror
through that wedged mass of bodies. `Don't! don't you frighten them
away,' cried some one on deck disconsolately. I pulled the string time
after time. They broke and ran, they leaped, they crouched, they
swerved, they dodged the flying terror of the sound. The three red chaps
had fallen flat, face down on the shore, as though they had been shot
dead. Only the barbarous and superb woman did not so much as flinch, and
stretched tragically her bare arms after us over the sombre and
glittering river.

``And then that imbecile crowd down on the deck started their little
fun, and I could see nothing more for smoke.

``The brown current ran swiftly out of the heart of darkness, bearing us
down towards the sea with twice the speed of our upward progress; and
Kurtz's life was running swiftly, too, ebbing, ebbing out of his heart
into the sea of inexorable time. The manager was very placid, he had no
vital anxieties now, he took us both in with a comprehensive and
satisfied glance: the `affair' had come off as well as could be wished.
I saw the time approaching when I would be left alone of the party of
`unsound method.' The pilgrims looked upon me with disfavour. I was, so
to speak, numbered with the dead. It is strange how I accepted this
unforeseen partnership, this choice of nightmares forced upon me in the
tenebrous land invaded by these mean and greedy phantoms.

``Kurtz discoursed. A voice! a voice! It rang deep to the very last. It
survived his strength to hide in the magnificent folds of eloquence the
barren darkness of his heart. Oh, he struggled! he struggled! The wastes
of his weary brain were haunted by shadowy images now---images of wealth
and fame revolving obsequiously round his unextinguishable gift of noble
and lofty expression. My Intended, my station, my career, my
ideas---these were the subjects for the occasional utterances of
elevated sentiments. The shade of the original Kurtz frequented the
bedside of the hollow sham, whose fate it was to be buried presently in
the mould of primeval earth. But both the diabolic love and the
unearthly hate of the mysteries it had penetrated fought for the
possession of that soul satiated with primitive emotions, avid of lying
fame, of sham distinction, of all the appearances of success and power.

``Sometimes he was contemptibly childish. He desired to have kings meet
him at railway-stations on his return from some ghastly Nowhere, where
he intended to accomplish great things. `You show them you have in you
something that is really profitable, and then there will be no limits to
the recognition of your ability,' he would say. `Of course you must take
care of the motives---right motives---always.' The long reaches that
were like one and the same reach, monotonous bends that were exactly
alike, slipped past the steamer with their multitude of secular trees
looking patiently after this grimy fragment of another world, the
forerunner of change, of conquest, of trade, of massacres, of blessings.
I looked ahead---piloting. `Close the shutter,' said Kurtz suddenly one
day; `I can't bear to look at this.' I did so. There was a silence. `Oh,
but I will wring your heart yet!' he cried at the invisible wilderness.

``We broke down---as I had expected---and had to lie up for repairs at
the head of an island. This delay was the first thing that shook Kurtz's
confidence. One morning he gave me a packet of papers and a
photograph---the lot tied together with a shoe-string. `Keep this for
me,' he said. `This noxious fool' (meaning the manager) `is capable of
prying into my boxes when I am not looking.' In the afternoon I saw him.
He was lying on his back with closed eyes, and I withdrew quietly, but I
heard him mutter, `Live rightly, die, die\ldots{}' I listened. There was
nothing more. Was he rehearsing some speech in his sleep, or was it a
fragment of a phrase from some newspaper article? He had been writing
for the papers and meant to do so again, `for the furthering of my
ideas. It's a duty.'

``His was an impenetrable darkness. I looked at him as you peer down at
a man who is lying at the bottom of a precipice where the sun never
shines. But I had not much time to give him, because I was helping the
engine-driver to take to pieces the leaky cylinders, to straighten a
bent connecting-rod, and in other such matters. I lived in an infernal
mess of rust, filings, nuts, bolts, spanners, hammers,
ratchet-drills---things I abominate, because I don't get on with them. I
tended the little forge we fortunately had aboard; I toiled wearily in a
wretched scrap-heap---unless I had the shakes too bad to stand.

``One evening coming in with a candle I was startled to hear him say a
little tremulously, `I am lying here in the dark waiting for death.' The
light was within a foot of his eyes. I forced myself to murmur, `Oh,
nonsense!' and stood over him as if transfixed.

``Anything approaching the change that came over his features I have
never seen before, and hope never to see again. Oh, I wasn't touched. I
was fascinated. It was as though a veil had been rent. I saw on that
ivory face the expression of sombre pride, of ruthless power, of craven
terror---of an intense and hopeless despair. Did he live his life again
in every detail of desire, temptation, and surrender during that supreme
moment of complete knowledge? He cried in a whisper at some image, at
some vision---he cried out twice, a cry that was no more than a breath:

```The horror! The horror!'

``I blew the candle out and left the cabin. The pilgrims were dining in
the mess-room, and I took my place opposite the manager, who lifted his
eyes to give me a questioning glance, which I successfully ignored. He
leaned back, serene, with that peculiar smile of his sealing the
unexpressed depths of his meanness. A continuous shower of small flies
streamed upon the lamp, upon the cloth, upon our hands and faces.
Suddenly the manager's boy put his insolent black head in the doorway,
and said in a tone of scathing contempt:

```Mistah Kurtz---he dead.'

``All the pilgrims rushed out to see. I remained, and went on with my
dinner. I believe I was considered brutally callous. However, I did not
eat much. There was a lamp in there---light, don't you know---and
outside it was so beastly, beastly dark. I went no more near the
remarkable man who had pronounced a judgment upon the adventures of his
soul on this earth. The voice was gone. What else had been there? But I
am of course aware that next day the pilgrims buried something in a
muddy hole.

``And then they very nearly buried me.

``However, as you see, I did not go to join Kurtz there and then. I did
not. I remained to dream the nightmare out to the end, and to show my
loyalty to Kurtz once more. Destiny. My destiny! Droll thing life
is---that mysterious arrangement of merciless logic for a futile
purpose. The most you can hope from it is some knowledge of
yourself---that comes too late---a crop of unextinguishable regrets. I
have wrestled with death. It is the most unexciting contest you can
imagine. It takes place in an impalpable greyness, with nothing
underfoot, with nothing around, without spectators, without clamour,
without glory, without the great desire of victory, without the great
fear of defeat, in a sickly atmosphere of tepid scepticism, without much
belief in your own right, and still less in that of your adversary. If
such is the form of ultimate wisdom, then life is a greater riddle than
some of us think it to be. I was within a hair's breadth of the last
opportunity for pronouncement, and I found with humiliation that
probably I would have nothing to say. This is the reason why I affirm
that Kurtz was a remarkable man. He had something to say. He said it.
Since I had peeped over the edge myself, I understand better the meaning
of his stare, that could not see the flame of the candle, but was wide
enough to embrace the whole universe, piercing enough to penetrate all
the hearts that beat in the darkness. He had summed up---he had judged.
`The horror!' He was a remarkable man. After all, this was the
expression of some sort of belief; it had candour, it had conviction, it
had a vibrating note of revolt in its whisper, it had the appalling face
of a glimpsed truth---the strange commingling of desire and hate. And it
is not my own extremity I remember best---a vision of greyness without
form filled with physical pain, and a careless contempt for the
evanescence of all things---even of this pain itself. No! It is his
extremity that I seem to have lived through. True, he had made that last
stride, he had stepped over the edge, while I had been permitted to draw
back my hesitating foot. And perhaps in this is the whole difference;
perhaps all the wisdom, and all truth, and all sincerity, are just
compressed into that inappreciable moment of time in which we step over
the threshold of the invisible. Perhaps! I like to think my summing-up
would not have been a word of careless contempt. Better his cry---much
better. It was an affirmation, a moral victory paid for by innumerable
defeats, by abominable terrors, by abominable satisfactions. But it was
a victory! That is why I have remained loyal to Kurtz to the last, and
even beyond, when a long time after I heard once more, not his own
voice, but the echo of his magnificent eloquence thrown to me from a
soul as translucently pure as a cliff of crystal.

``No, they did not bury me, though there is a period of time which I
remember mistily, with a shuddering wonder, like a passage through some
inconceivable world that had no hope in it and no desire. I found myself
back in the sepulchral city resenting the sight of people hurrying
through the streets to filch a little money from each other, to devour
their infamous cookery, to gulp their unwholesome beer, to dream their
insignificant and silly dreams. They trespassed upon my thoughts. They
were intruders whose knowledge of life was to me an irritating pretence,
because I felt so sure they could not possibly know the things I knew.
Their bearing, which was simply the bearing of commonplace individuals
going about their business in the assurance of perfect safety, was
offensive to me like the outrageous flauntings of folly in the face of a
danger it is unable to comprehend. I had no particular desire to
enlighten them, but I had some difficulty in restraining myself from
laughing in their faces so full of stupid importance. I daresay I was
not very well at that time. I tottered about the streets---there were
various affairs to settle---grinning bitterly at perfectly respectable
persons. I admit my behaviour was inexcusable, but then my temperature
was seldom normal in these days. My dear aunt's endeavours to `nurse up
my strength' seemed altogether beside the mark. It was not my strength
that wanted nursing, it was my imagination that wanted soothing. I kept
the bundle of papers given me by Kurtz, not knowing exactly what to do
with it. His mother had died lately, watched over, as I was told, by his
Intended. A clean-shaved man, with an official manner and wearing
gold-rimmed spectacles, called on me one day and made inquiries, at
first circuitous, afterwards suavely pressing, about what he was pleased
to denominate certain `documents.' I was not surprised, because I had
had two rows with the manager on the subject out there. I had refused to
give up the smallest scrap out of that package, and I took the same
attitude with the spectacled man. He became darkly menacing at last, and
with much heat argued that the Company had the right to every bit of
information about its `territories.' And said he, `Mr. Kurtz's knowledge
of unexplored regions must have been necessarily extensive and
peculiar---owing to his great abilities and to the deplorable
circumstances in which he had been placed: therefore---' I assured him
Mr. Kurtz's knowledge, however extensive, did not bear upon the problems
of commerce or administration. He invoked then the name of science. `It
would be an incalculable loss if,' etc., etc. I offered him the report
on the `Suppression of Savage Customs,' with the postscriptum torn off.
He took it up eagerly, but ended by sniffing at it with an air of
contempt. `This is not what we had a right to expect,' he remarked.
`Expect nothing else,' I said. `There are only private letters.' He
withdrew upon some threat of legal proceedings, and I saw him no more;
but another fellow, calling himself Kurtz's cousin, appeared two days
later, and was anxious to hear all the details about his dear relative's
last moments. Incidentally he gave me to understand that Kurtz had been
essentially a great musician. `There was the making of an immense
success,' said the man, who was an organist, I believe, with lank grey
hair flowing over a greasy coat-collar. I had no reason to doubt his
statement; and to this day I am unable to say what was Kurtz's
profession, whether he ever had any---which was the greatest of his
talents. I had taken him for a painter who wrote for the papers, or else
for a journalist who could paint---but even the cousin (who took snuff
during the interview) could not tell me what he had been---exactly. He
was a universal genius---on that point I agreed with the old chap, who
thereupon blew his nose noisily into a large cotton handkerchief and
withdrew in senile agitation, bearing off some family letters and
memoranda without importance. Ultimately a journalist anxious to know
something of the fate of his `dear colleague' turned up. This visitor
informed me Kurtz's proper sphere ought to have been politics `on the
popular side.' He had furry straight eyebrows, bristly hair cropped
short, an eyeglass on a broad ribbon, and, becoming expansive, confessed
his opinion that Kurtz really couldn't write a bit---`but heavens! how
that man could talk. He electrified large meetings. He had faith---don't
you see?---he had the faith. He could get himself to believe
anything---anything. He would have been a splendid leader of an extreme
party.' `What party?' I asked. `Any party,' answered the other. `He was
an---an---extremist.' Did I not think so? I assented. Did I know, he
asked, with a sudden flash of curiosity, `what it was that had induced
him to go out there?' `Yes,' said I, and forthwith handed him the famous
Report for publication, if he thought fit. He glanced through it
hurriedly, mumbling all the time, judged `it would do,' and took himself
off with this plunder.

``Thus I was left at last with a slim packet of letters and the girl's
portrait. She struck me as beautiful---I mean she had a beautiful
expression. I know that the sunlight can be made to lie, too, yet one
felt that no manipulation of light and pose could have conveyed the
delicate shade of truthfulness upon those features. She seemed ready to
listen without mental reservation, without suspicion, without a thought
for herself. I concluded I would go and give her back her portrait and
those letters myself. Curiosity? Yes; and also some other feeling
perhaps. All that had been Kurtz's had passed out of my hands: his soul,
his body, his station, his plans, his ivory, his career. There remained
only his memory and his Intended---and I wanted to give that up, too, to
the past, in a way---to surrender personally all that remained of him
with me to that oblivion which is the last word of our common fate. I
don't defend myself. I had no clear perception of what it was I really
wanted. Perhaps it was an impulse of unconscious loyalty, or the
fulfilment of one of those ironic necessities that lurk in the facts of
human existence. I don't know. I can't tell. But I went.

``I thought his memory was like the other memories of the dead that
accumulate in every man's life---a vague impress on the brain of shadows
that had fallen on it in their swift and final passage; but before the
high and ponderous door, between the tall houses of a street as still
and decorous as a well-kept alley in a cemetery, I had a vision of him
on the stretcher, opening his mouth voraciously, as if to devour all the
earth with all its mankind. He lived then before me; he lived as much as
he had ever lived---a shadow insatiable of splendid appearances, of
frightful realities; a shadow darker than the shadow of the night, and
draped nobly in the folds of a gorgeous eloquence. The vision seemed to
enter the house with me---the stretcher, the phantom-bearers, the wild
crowd of obedient worshippers, the gloom of the forests, the glitter of
the reach between the murky bends, the beat of the drum, regular and
muffled like the beating of a heart---the heart of a conquering
darkness. It was a moment of triumph for the wilderness, an invading and
vengeful rush which, it seemed to me, I would have to keep back alone
for the salvation of another soul. And the memory of what I had heard
him say afar there, with the horned shapes stirring at my back, in the
glow of fires, within the patient woods, those broken phrases came back
to me, were heard again in their ominous and terrifying simplicity. I
remembered his abject pleading, his abject threats, the colossal scale
of his vile desires, the meanness, the torment, the tempestuous anguish
of his soul. And later on I seemed to see his collected languid manner,
when he said one day, `This lot of ivory now is really mine. The Company
did not pay for it. I collected it myself at a very great personal risk.
I am afraid they will try to claim it as theirs though. H'm. It is a
difficult case. What do you think I ought to do---resist? Eh? I want no
more than justice.'\ldots{} He wanted no more than justice---no more
than justice. I rang the bell before a mahogany door on the first floor,
and while I waited he seemed to stare at me out of the glassy
panel---stare with that wide and immense stare embracing, condemning,
loathing all the universe. I seemed to hear the whispered cry, ``The
horror! The horror!''

``The dusk was falling. I had to wait in a lofty drawing-room with three
long windows from floor to ceiling that were like three luminous and
bedraped columns. The bent gilt legs and backs of the furniture shone in
indistinct curves. The tall marble fireplace had a cold and monumental
whiteness. A grand piano stood massively in a corner; with dark gleams
on the flat surfaces like a sombre and polished sarcophagus. A high door
opened---closed. I rose.

``She came forward, all in black, with a pale head, floating towards me
in the dusk. She was in mourning. It was more than a year since his
death, more than a year since the news came; she seemed as though she
would remember and mourn forever. She took both my hands in hers and
murmured, `I had heard you were coming.' I noticed she was not very
young---I mean not girlish. She had a mature capacity for fidelity, for
belief, for suffering. The room seemed to have grown darker, as if all
the sad light of the cloudy evening had taken refuge on her forehead.
This fair hair, this pale visage, this pure brow, seemed surrounded by
an ashy halo from which the dark eyes looked out at me. Their glance was
guileless, profound, confident, and trustful. She carried her sorrowful
head as though she were proud of that sorrow, as though she would say,
`I---I alone know how to mourn for him as he deserves.' But while we
were still shaking hands, such a look of awful desolation came upon her
face that I perceived she was one of those creatures that are not the
playthings of Time. For her he had died only yesterday. And, by Jove!
the impression was so powerful that for me, too, he seemed to have died
only yesterday---nay, this very minute. I saw her and him in the same
instant of time---his death and her sorrow---I saw her sorrow in the
very moment of his death. Do you understand? I saw them together---I
heard them together. She had said, with a deep catch of the breath, `I
have survived' while my strained ears seemed to hear distinctly, mingled
with her tone of despairing regret, the summing up whisper of his
eternal condemnation. I asked myself what I was doing there, with a
sensation of panic in my heart as though I had blundered into a place of
cruel and absurd mysteries not fit for a human being to behold. She
motioned me to a chair. We sat down. I laid the packet gently on the
little table, and she put her hand over it\ldots{}. `You knew him well,'
she murmured, after a moment of mourning silence.

```Intimacy grows quickly out there,' I said. `I knew him as well as it
is possible for one man to know another.'

```And you admired him,' she said. `It was impossible to know him and
not to admire him. Was it?'

\clearpage
```He was a remarkable man,' I said, unsteadily. Then before the
appealing fixity of her gaze, that seemed to watch for more words on my
lips, I went on, `It was impossible not to---'

```Love him,' she finished eagerly, silencing me into an appalled
dumbness. `How true! how true! But when you think that no one knew him
so well as I! I had all his noble confidence. I knew him best.'

```You knew him best,' I repeated. And perhaps she did. But with every
word spoken the room was growing darker, and only her forehead, smooth
and white, remained illumined by the inextinguishable light of belief
and love.

```You were his friend,' she went on. `His friend,' she repeated, a
little louder. `You must have been, if he had given you this, and sent
you to me. I feel I can speak to you---and oh! I must speak. I want
you---you who have heard his last words---to know I have been worthy of
him\ldots{}. It is not pride\ldots{}. Yes! I am proud to know I
understood him better than any one on earth---he told me so himself. And
since his mother died I have had no one---no one---to---to---'

``I listened. The darkness deepened. I was not even sure whether he had
given me the right bundle. I rather suspect he wanted me to take care of
another batch of his papers which, after his death, I saw the manager
examining under the lamp. And the girl talked, easing her pain in the
certitude of my sympathy; she talked as thirsty men drink. I had heard
that her engagement with Kurtz had been disapproved by her people. He
wasn't rich enough or something. And indeed I don't know whether he had
not been a pauper all his life. He had given me some reason to infer
that it was his impatience of comparative poverty that drove him out
there.

```\ldots{} Who was not his friend who had heard him speak once?' she
was saying. `He drew men towards him by what was best in them.' She
looked at me with intensity. `It is the gift of the great,' she went on,
and the sound of her low voice seemed to have the accompaniment of all
the other sounds, full of mystery, desolation, and sorrow, I had ever
heard---the ripple of the river, the soughing of the trees swayed by the
wind, the murmurs of the crowds, the faint ring of incomprehensible
words cried from afar, the whisper of a voice speaking from beyond the
threshold of an eternal darkness. `But you have heard him! You know!'
she cried.

```Yes, I know,' I said with something like despair in my heart, but
bowing my head before the faith that was in her, before that great and
saving illusion that shone with an unearthly glow in the darkness, in
the triumphant darkness from which I could not have defended her---from
which I could not even defend myself.

```What a loss to me---to us!'---she corrected herself with beautiful
generosity; then added in a murmur, `To the world.' By the last gleams
of twilight I could see the glitter of her eyes, full of tears---of
tears that would not fall.

```I have been very happy---very fortunate---very proud,' she went on.
`Too fortunate. Too happy for a little while. And now I am unhappy
for---for life.'

``She stood up; her fair hair seemed to catch all the remaining light in
a glimmer of gold. I rose, too.

```And of all this,' she went on mournfully, `of all his promise, and of
all his greatness, of his generous mind, of his noble heart, nothing
remains---nothing but a memory. You and I---'

```We shall always remember him,' I said hastily.

```No!' she cried. `It is impossible that all this should be lost---that
such a life should be sacrificed to leave nothing---but sorrow. You know
what vast plans he had. I knew of them, too---I could not perhaps
understand---but others knew of them. Something must remain. His words,
at least, have not died.'

```His words will remain,' I said.

```And his example,' she whispered to herself. `Men looked up to
him---his goodness shone in every act. His example---'

```True,' I said; `his example, too. Yes, his example. I forgot that.'

```But I do not. I cannot---I cannot believe---not yet. I cannot believe
that I shall never see him again, that nobody will see him again, never,
never, never.'

``She put out her arms as if after a retreating figure, stretching them
back and with clasped pale hands across the fading and narrow sheen of
the window. Never see him! I saw him clearly enough then. I shall see
this eloquent phantom as long as I live, and I shall see her, too, a
tragic and familiar Shade, resembling in this gesture another one,
tragic also, and bedecked with powerless charms, stretching bare brown
arms over the glitter of the infernal stream, the stream of darkness.
She said suddenly very low, `He died as he lived.'

```His end,' said I, with dull anger stirring in me, `was in every way
worthy of his life.'

```And I was not with him,' she murmured. My anger subsided before a
feeling of infinite pity.

```Everything that could be done---' I mumbled.

```Ah, but I believed in him more than any one on earth---more than his
own mother, more than---himself. He needed me! Me! I would have
treasured every sigh, every word, every sign, every glance.'

``I felt like a chill grip on my chest. `Don't,' I said, in a muffled
voice.

```Forgive me. I---I have mourned so long in silence---in
silence\ldots{}. You were with him---to the last? I think of his
loneliness. Nobody near to understand him as I would have understood.
Perhaps no one to hear\ldots{}.'

```To the very end,' I said, shakily. `I heard his very last
words\ldots{}.' I stopped in a fright.

```Repeat them,' she murmured in a heart-broken tone. `I want---I
want---something---something---to---to live with.'

``I was on the point of crying at her, `Don't you hear them?' The dusk
was repeating them in a persistent whisper all around us, in a whisper
that seemed to swell menacingly like the first whisper of a rising wind.
`The horror! The horror!'

```His last word---to live with,' she insisted. `Don't you understand I
loved him---I loved him---I loved him!'

``I pulled myself together and spoke slowly.

```The last word he pronounced was---your name.'

``I heard a light sigh and then my heart stood still, stopped dead short
by an exulting and terrible cry, by the cry of inconceivable triumph and
of unspeakable pain. `I knew it---I was sure!'\ldots{} She knew. She was
sure. I heard her weeping; she had hidden her face in her hands. It
seemed to me that the house would collapse before I could escape, that
the heavens would fall upon my head. But nothing happened. The heavens
do not fall for such a trifle. Would they have fallen, I wonder, if I
had rendered Kurtz that justice which was his due? Hadn't he said he
wanted only justice? But I couldn't. I could not tell her. It would have
been too dark---too dark altogether\ldots{}.''

Marlow ceased, and sat apart, indistinct and silent, in the pose of a
meditating Buddha. Nobody moved for a time. ``We have lost the first of
the ebb,'' said the Director suddenly. I raised my head. The offing was
barred by a black bank of clouds, and the tranquil waterway leading to
the uttermost ends of the earth flowed sombre under an overcast
sky---seemed to lead into the heart of an immense darkness.






\end{document}