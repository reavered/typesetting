\documentclass[letterpaper,12pt]{article}
\usepackage[12pt]{moresize}
\usepackage[utf8x]{inputenc}
\usepackage[english]{babel}
\usepackage[margin=2.5cm, right = 3.5cm]{geometry}
\usepackage{ebgaramond}

\usepackage{dramatist}

\renewcommand{\thefootnote}{[\fontfamily{ppl}\selectfont\arabic{footnote}]}
\setlength{\skip\footins}{1cm}
\usepackage[]{footmisc}
\renewcommand{\footnotemargin}{3mm} %Setting left margin
\renewcommand{\footnotelayout}{\hspace{2mm}} %spacing between the footnote number and the text of footnote

\usepackage{marginnote}
\renewcommand{\raggedrightmarginnote}{\raggedleft}

\title{\vspace{-2.5cm} \scshape Crito \vspace{-8mm}}
\author{}
\date{
	\vspace{-1em}
		\small \fontfamily{ppl}\selectfont Written by Plato, translated by Harold North Fowler, 1966
	\begin{center}
		$\mathsection$
	\end{center}
		\vspace{-2em}
	}

\newenvironment{setting}
	{
		\setlength{\tabcolsep}{3em}
		\begin{center}
			\section*{\normalsize \fontfamily{ppl}\selectfont \itshape \bfseries Persons of the Dialogue\vspace{-1mm}}
			\par
			\begin{tabular}{cc}
	}
	{
			\end{tabular}
		\end{center}
		\par \fontfamily{ppl}\selectfont{\small \textbf{Scene:} \textit{The Prison of Socrates.}}
		
		\hrulefill
	}

\begin{document}

\Character[Socrates]{Socrates.}{socrates} % define characters
\Character[Crito]{Crito.}{crito}

\begin{minipage}{15.45cm}
\maketitle

\begin{setting}
	\textsc{Socrates.}	&	\textsc{Crito.}
\end{setting}
\end{minipage}
\begin{drama}
\setlength{\parindent}{2em}

\socratesspeaks 
\marginnote{\small \fontfamily{ppl}\selectfont \itshape 43 a}Why have you come at this time, Crito? Or isn't it still early?
\critospeaks Yes, very early.
\socratesspeaks 
About what time?
\critospeaks 
Just before dawn.
\socratesspeaks 
I am surprised that the watchman of the prison was willing to let you in.
\critospeaks 
He is used to me by this time, Socrates, because I come here so often, and besides I have done something for him.
\socratesspeaks 
Have you just come, or some time ago?
\critospeaks 
Some little time ago. \marginnote{\small \fontfamily{ppl}\selectfont \itshape b}
\socratesspeaks 
Then why did you not wake me at once, instead of sitting by me in silence?
\critospeaks 
No, no, by Zeus, Socrates, I only wish I myself were not so sleepless and sorrowful. But I have been wondering at you for some time, seeing how sweetly you sleep; and I purposely refrained from waking you, that you might pass the time as pleasantly as possible. I have often thought throughout your life hitherto that you were of a happy disposition, and I think so more than ever in this resent misfortune, since you bear it so easily and calmly.
\socratesspeaks 
Well, Crito, it would be absurd \marginnote{\small \fontfamily{ppl}\selectfont \itshape c} if at my age I were disturbed because I must die now.
\critospeaks 
Other men as old, Socrates, become involved in similar misfortunes, but their age does not in the least prevent them from being disturbed by their fate.
\socratesspeaks 
That is true. But why have you come so early?
\critospeaks 
To bring news, Socrates, sad news, though apparently not sad to you, but sad and grievous me and all your friends, and to few of them, I think, so grievous as to me.
\socratesspeaks 
What is this news? Has the ship come from Delos, \marginnote{\small \fontfamily{ppl}\selectfont \itshape d} at the arrival of which I am to die?
\critospeaks 
It has not exactly come, but I think it will come today from the reports of some men who have come from Sunium and left it there. Now it is clear from what they say that it will come today, and so tomorrow, Socrates, your life must end.
\socratesspeaks 
Well, Crito, good luck be with us! If this is the will of the gods, so be it. However, I do not think \marginnote{\small \fontfamily{ppl}\selectfont \itshape 44 a} it will come today.
\critospeaks 
What is your reason for not thinking so?
\socratesspeaks 
I will tell you. I must die on the day after the ship comes in, must I not?
\critospeaks 
So those say who have charge of these matters.
\socratesspeaks 
Well, I think it will not come in today, but tomorrow. And my reason for this is a dream which I had a little while ago in the course of this night. And perhaps you let me sleep just at the right time.
\critospeaks 
What was the dream?
\socratesspeaks 
I dreamed that a beautiful, fair woman, clothed in white raiment, came to me and called me \marginnote{\small \fontfamily{ppl}\selectfont \itshape b} and said: Socrates, `on the third day thou wouldst come to fertile Phthia.'\footnote{Hom. Il. 9.363.}
\critospeaks 
A strange dream, Socrates.
\socratesspeaks 
No, a clear one, at any rate, I think, Crito.
\critospeaks 
Too clear, apparently. But, my dear Socrates, even now listen to me and save yourself. Since, if you die, it will be no mere single misfortune to me, but I shall lose a friend such as I can never find again, and besides, many persons who do not know you and me well \marginnote{\small \fontfamily{ppl}\selectfont \itshape c} will think I could have saved you if I had been willing to spend money, but that I would not take the trouble. And yet what reputation could be more disgraceful than that of considering one's money of more importance than one's friends? For most people will not believe that we were eager to help you to go away from here, but you refused.
\socratesspeaks 
But, my dear Crito, why do we care so much for what most people think? For the most reasonable men, whose opinion is more worth considering, will think that things were done as they really will be done. \marginnote{\small \fontfamily{ppl}\selectfont \itshape d}
\critospeaks 
But you see it is necessary, Socrates, to care for the opinion of the public, for this very trouble we are in now shows that the public is able to accomplish not by any means the least, but almost the greatest of evils, if one has a bad reputation with it.
\socratesspeaks 
I only wish, Crito, the people could accomplish the greatest evils, that they might be able to accomplish also the greatest good things. Then all would be well. But now they can do neither of the two; for they are not able to make a man wise or foolish, but they do whatever occurs to them. \marginnote{\small \fontfamily{ppl}\selectfont \itshape e}
\critospeaks 
That may well be. But, Socrates, tell me this: you are not considering me and your other friends, are you, fearing that, if you escape, the informers will make trouble for us by saying that we stole you away, and we shall be forced to lose either all our property or a good deal of money, or be punished in some other way besides? \marginnote{\small \fontfamily{ppl}\selectfont \itshape 45 a} For if you are afraid of anything of that kind, let it go; since it is right for us to run this risk, and even greater risk than this, if necessary, provided we save you. Now please do as I ask.
\socratesspeaks 
I am considering this, Crito, and many other things.
\critospeaks 
Well, do not fear this! for it is not even a large sum of money which we should pay to some men who are willing to save you and get you away from here. Besides, don't you see how cheap these informers are, and that not much money would be needed to silence them? And you \marginnote{\small \fontfamily{ppl}\selectfont \itshape b} have my money at your command, which is enough, I fancy; and moreover, if because you care for me you think you ought not to spend my money, there are foreigners here willing to spend theirs; and one of them, Simmias of Thebes, has brought for this especial purpose sufficient funds; and Cebes also and very many others are ready. So, as I say, do not give up saving yourself through fear of this. And do not be troubled by what you said in the court, that if you went away you would not know what to do with yourself. For in many other places, wherever you go, \marginnote{\small \fontfamily{ppl}\selectfont \itshape c} they will welcome you; and if you wish to go to Thessaly, I have friends there who will make much of you and will protect you, so that no one in Thessaly shall annoy you. And besides, Socrates, it seems to me the thing you are undertaking to do is not even right---betraying yourself when you might save yourself. And you are eager to bring upon yourself just what your enemies would wish and just what those were eager for who wished to destroy you. And moreover, I think you are abandoning your children, too, \marginnote{\small \fontfamily{ppl}\selectfont \itshape d} for when you might bring them up and educate them, you are going to desert them and go away, and, so far as you are concerned, their fortunes in life will be whatever they happen to meet with, and they will probably meet with such treatment as generally comes to orphans in their destitution. No. Either one ought not to beget children, or one ought to stay by them and bring them up and educate them. But you seem to me to be choosing the laziest way; and you ought to choose as a good and brave man would choose, you who have been saying all your life that you cared for virtue. So I am shamed both for you and for us, \marginnote{\small \fontfamily{ppl}\selectfont \itshape e} your friends, and I am afraid people will think that this whole affair of yours has been conducted with a sort of cowardice on our part---both the fact that the case came before the court, when it might have been avoided, and the way in which the trial itself was carried on, and finally they will think, as the crowning absurdity of the whole affair, that this opportunity has escaped us through some base cowardice on our part, \marginnote{\small \fontfamily{ppl}\selectfont \itshape 46 a} since we did not save you, and you did not save yourself, though it was quite possible if we had been of any use whatever. Take care, Socrates, that these things be not disgraceful, as well as evil, both to you and to us. Just consider, or rather it is time not to consider any longer, but to have finished considering. And there is just one possible plan; for all this must be done in the coming night. And if we delay it can no longer be done. But I beg you, Socrates, do as I say and don't refuse. \marginnote{\small \fontfamily{ppl}\selectfont \itshape b}
\socratesspeaks 
My dear Crito, your eagerness is worth a great deal, if it should prove to be rightly directed; but otherwise, the greater it is, the more hard to bear. So we must examine the question whether we ought to do this or not; for I am not only now but always a man who follows nothing but the reasoning which on consideration seems to me best. And I cannot, now that this has happened to us, discard the arguments I used to advance, but they seem to me much the same as ever, \marginnote{\small \fontfamily{ppl}\selectfont \itshape c} and I revere and honor the same ones as before. And unless we can bring forward better ones in our present situation, be assured that I shall not give way to you, not even if the power of the multitude frighten us with even more terrors than at present, as children are frightened with goblins, threatening us with imprisonments and deaths and confiscations of property. Now how could we examine the matter most reasonably? By taking up first what you say about opinions and asking whether we were right when we always used to say that we ought to pay attention \marginnote{\small \fontfamily{ppl}\selectfont \itshape d} to some opinions and not to others? Or were we right before I was condemned to death, whereas it has how been made clear that we were talking merely for the sake of argument and it was really mere play and nonsense? And I wish to investigate, Crito, in common with you, and see whether our former argument seems different to me under our present conditions, or the same, and whether we shall give it up or be guided by it. But it used to be said, I think, by those who thought they were speaking sensibly, just as I was saying now, that of the opinions held by men \marginnote{\small \fontfamily{ppl}\selectfont \itshape e} some ought to be highly esteemed and others not. In God's name, Crito, do you not think this is correct? For you, humanly speaking, \marginnote{\small \fontfamily{ppl}\selectfont \itshape 47 a} are not involved in the necessity of dying tomorrow, and therefore present conditions would not lead your judgment astray. Now say, do you not think we were correct in saying that we ought not to esteem all the opinions of men, but some and not others, and not those of all men, but only of some? What do you think? Is not this true?
\critospeaks 
It is.
\socratesspeaks 
Then we ought to esteem the good opinions and not the bad ones?
\critospeaks 
Yes.
\socratesspeaks 
And the good ones are those of the wise and the bad ones those of the foolish?
\critospeaks 
Of course.
\socratesspeaks 
Come then, what used we to say about this? \marginnote{\small \fontfamily{ppl}\selectfont \itshape b} If a man is an athlete and makes that his business, does he pay attention to every man's praise and blame and opinion or to those of one man only who is a physician or a trainer?
\critospeaks 
To those of one man only.
\socratesspeaks 
Then he ought to fear the blame and welcome the praise of that one man and not of the multitude.
\critospeaks 
Obviously.
\socratesspeaks 
And he must act and exercise and eat and drink as the one man who is his director and who knows the business thinks best rather than as all the others think.
\critospeaks 
That is true. \marginnote{\small \fontfamily{ppl}\selectfont \itshape c}
\socratesspeaks 
Well then; if he disobeys the one man and disregards his opinion and his praise, but regards words of the many who have no special knowledge, will he not come to harm?
\critospeaks 
Of course he will.
\socratesspeaks 
And what is this harm? In what direction and upon what part of the one who disobeys does it act?
\critospeaks 
Evidently upon his body; for that is what it ruins.
\socratesspeaks 
Right. Then in other matters, not to enumerate them all, in questions of right and wrong and disgraceful and noble and good and bad, which we are now considering, ought we to follow and fear \marginnote{\small \fontfamily{ppl}\selectfont \itshape d} the opinion of the many or that of the one, if there is anyone who knows about them, whom we ought to revere and fear more than all the others? And if we do not follow him, we shall injure and cripple that which we used to say is benefited by the right and is ruined by the wrong. Or is there nothing in this?
\critospeaks 
I think it is true, Socrates.
\socratesspeaks 
Well then, if through yielding to the opinion of the ignorant we ruin that which is benefited by health and injured by disease, \marginnote{\small \fontfamily{ppl}\selectfont \itshape e} is life worth living for us when that is ruined? And that is the body, is it not?
\critospeaks 
Yes.
\socratesspeaks 
Then is life worth living when the body is worthless and ruined?
\critospeaks 
Certainly not.
\socratesspeaks 
But is it worth living when that is ruined which is injured by the wrong and improved by the right? Or do we think that part of us, whatever it is, which is concerned with right and wrong, \marginnote{\small \fontfamily{ppl}\selectfont \itshape 48 a} is less important than the body?
\critospeaks 
By no means.
\socratesspeaks 
But more important?
\critospeaks 
Much more.
\socratesspeaks 
Then, most excellent friend, we must not consider at all what the many will say of us, but what he who knows about right and wrong, the one man, and truth herself will say. And so you introduced the discussion wrongly in the first place, when you began by saying we ought to consider the opinion of the multitude about the right and the noble and the good and their opposites. \marginnote{\small \fontfamily{ppl}\selectfont \itshape b} But it might, of course, be said that the multitude can put us to death.
\critospeaks 
That is clear, too. It would be said, Socrates.
\socratesspeaks 
That is true. But, my friend, the argument we have just finished seems to me still much the same as before; and now see whether we still hold to this, or not, that it is not living, but living well which we ought to consider most important.
\critospeaks 
We do hold to it.
\socratesspeaks 
And that living well and living rightly are the same thing, do we hold to that, or not?
\critospeaks 
We do.
\socratesspeaks 
Then we agree that the question is whether it is right for me to try to escape from here \marginnote{\small \fontfamily{ppl}\selectfont \itshape c} without the permission of the Athenians, or not right. And if it appears to be right, let us try it, and if not, let us give it up. But the considerations you suggest, about spending money, and reputation, and bringing up my children, these are really, Crito, the reflections of those who lightly put men to death, and would bring them to life again, if they could, without any sense, I mean the multitude. But we, since our argument so constrains us, must consider only the question we just broached, whether we shall be doing right in giving money \marginnote{\small \fontfamily{ppl}\selectfont \itshape d} and thanks to these men who will help me to escape, and in escaping or aiding the escape ourselves, or shall in truth be doing wrong, if we do all these things. And if it appears that it is wrong for us to do them, it may be that we ought not to consider either whether we must die if we stay here and keep quiet or whether we must endure any thing else whatsoever, but only the question of doing wrong.
\critospeaks 
I think what you say is right, Socrates; but think what we should do.
\socratesspeaks 
Let us, my good friend, investigate in common, and if you can contradict anything I say, do so, and I will yield to your arguments; \marginnote{\small \fontfamily{ppl}\selectfont \itshape e} but if you cannot, my dear friend, stop at once saying the same thing to me over and over, that I ought to go away from here without the consent of the Athenians; for I am anxious to act in this matter with your approval, and not contrary to your wishes. Now see if the beginning of the investigation satisfies you, and try to reply \marginnote{\small \fontfamily{ppl}\selectfont \itshape 49 a} to my questions to the best of your belief.
\critospeaks 
I will try.
\socratesspeaks 
Ought we in no way to do wrong intentionally, or should we do wrong in some ways but not in others? Or, as we often agreed in former times, is it never right or honorable to do wrong? Or have all those former conclusions of ours been overturned in these few days, and have we old men, \marginnote{\small \fontfamily{ppl}\selectfont \itshape b} seriously conversing with each other, failed all along to see that we were no better than children? Or is not what we used to say most certainly true, whether the world agree or not? And whether we must endure still more grievous sufferings than these, or lighter ones, is not wrongdoing inevitably an evil and a disgrace to the wrongdoer? Do we believe this or not?
\critospeaks 
We do.
\socratesspeaks 
Then we ought not to do wrong at all.
\critospeaks 
Why, no.
\socratesspeaks 
And we ought not even to requite wrong with wrong, as the world thinks, since we must not do wrong at all. \marginnote{\small \fontfamily{ppl}\selectfont \itshape c}
\critospeaks 
Apparently not.
\socratesspeaks 
Well, Crito, ought one to do evil or not?
\critospeaks 
Certainly not, Socrates.
\socratesspeaks 
Well, then, is it right to requite evil with evil, as the world says it is, or not right?
\critospeaks 
Not right, certainly.
\socratesspeaks 
For doing evil to people is the same thing as wronging them.
\critospeaks 
That is true.
\socratesspeaks 
Then we ought neither to requite wrong with wrong nor to do evil to anyone, no matter what he may have done to us. \marginnote{\small \fontfamily{ppl}\selectfont \itshape d} And be careful, Crito, that you do not, in agreeing to this, agree to something you do not believe; for I know that there are few who believe or ever will believe this. Now those who believe this, and those who do not, have no common ground of discussion, but they must necessarily, in view of their opinions, despise one another. Do you therefore consider very carefully whether you agree and share in this opinion, and let us take as the the starting point of our discussion the assumption that it is never right to do wrong or to requite wrong with wrong, or when we suffer evil to defend ourselves by doing evil in return. Or do you disagree and refuse your assent \marginnote{\small \fontfamily{ppl}\selectfont \itshape e} to this starting point? For I have long held this belief and I hold it yet, but if you have reached any other conclusion, speak and explain it to me. If you still hold to our former opinion, hear the next point.
\critospeaks 
I do hold to it and I agree with you; so go on.
\socratesspeaks 
Now the next thing I say, or rather ask, is this: ``ought a man to do what he has agreed to do, provided it is right, or may he violate his agreements?''
\critospeaks 
He ought to do it.
\socratesspeaks 
Then consider whether, if we go away \marginnote{\small \fontfamily{ppl}\selectfont \itshape 50 a} from here without the consent of the state, we are doing harm to the very ones to whom we least ought to do harm, or not, and whether we are abiding by what we agreed was right, or not.
\critospeaks 
I cannot answer your question, Socrates, for I do not understand.
\socratesspeaks 
Consider it in this way. If, as I was on the point of running away (or whatever it should be called), the laws and the commonwealth should come to me and ask, ``Tell me, Socrates, what have you in mind to do? Are you not intending by this thing you are trying to do, to destroy us, \marginnote{\small \fontfamily{ppl}\selectfont \itshape b} the laws, and the entire state, so far as in you lies? Or do you think that state can exist and not be overturned, in which the decisions reached by the courts have no force but are made invalid and annulled by private persons?'' What shall we say, Crito, in reply to this question and others of the same kind? For one might say many things, especially if one were an orator, about the destruction of that law which provides that the decisions reached by the courts shall be valid. Or shall we say to them, \marginnote{\small \fontfamily{ppl}\selectfont \itshape c} ``The state wronged me and did not judge the case rightly''? Shall we say that, or what?
\critospeaks 
That is what we shall say, by Zeus, Socrates.
\socratesspeaks 
What then if the laws should say, ``Socrates, is this the agreement you made with us, or did you agree to abide by the verdicts pronounced by the state?'' Now if I were surprised by what they said, perhaps they would continue, ``Don't be surprised at what we say, Socrates, but answer, since you are in the habit of employing the method of question and answer. Come, \marginnote{\small \fontfamily{ppl}\selectfont \itshape d} what fault do you find with us and the state, that you are trying to destroy us? In the first place, did we not bring you forth? Is it not through us that your father married your mother and begat you? Now tell us, have you any fault to find with those of us who are the laws of marriage?'' ``I find no fault,'' I should say. ``Or with those that have to do with the nurture of the child after he is born and with his education which you, like others, received? Did those of us who are assigned to these matters not give good directions when we told your father to educate you in music and gymnastics?'' \marginnote{\small \fontfamily{ppl}\selectfont \itshape e} ``You did,'' I should say. ``Well then, when you were born and nurtured and educated, could you say to begin with that you were not our offspring and our slave, you yourself and your ancestors? And if this is so, do you think right as between you and us rests on a basis of equality, so that whatever we undertake to do to you it is right for you to retaliate? There was no such equality of right between you and your father or your master, if you had one, so that whatever treatment you received you might return it, answering them if you were reviled, \marginnote{\small \fontfamily{ppl}\selectfont \itshape 51 a} or striking back if you were struck, and the like; and do you think that it will be proper for you to act so toward your country and the laws, so that if we undertake to destroy you, thinking it is right, you will undertake in return to destroy us laws and your country, so far as you are able, and will say that in doing this you are doing right, you who really care for virtue? Or is your wisdom such that you do not see that your country is more precious and more to be revered and is holier and in higher esteem \marginnote{\small \fontfamily{ppl}\selectfont \itshape b} among the gods and among men of understanding than your mother and your father and all your ancestors, and that you ought to show to her more reverence and obedience and humility when she is angry than to your father, and ought either to convince her by persuasion or to do whatever she commands, and to suffer, if she commands you to suffer, in silence, and if she orders you to be scourged or imprisoned or if she leads you to war to be wounded or slain, her will is to be done, and this is right, and you must not give way or draw back or leave your post, but in war and in court and everywhere, \marginnote{\small \fontfamily{ppl}\selectfont \itshape c} you must do whatever the state, your country, commands, or must show her by persuasion what is really right, but that it is impious to use violence against either your father or your mother, and much more impious to use it against your country?'' What shall we reply to this, Crito, that the laws speak the truth, or not?
\critospeaks 
I think they do.
\socratesspeaks 
``Observe then, Socrates,'' perhaps the laws would say, ``that if what we say is true, what you are now undertaking to do to us is not right. For we brought you into the world, nurtured you, and gave a share of all the good things \marginnote{\small \fontfamily{ppl}\selectfont \itshape d} we could to you and all the citizens. Yet we proclaim, by having offered the opportunity to any of the Athenians who wishes to avail himself of it, that anyone who is not pleased with us when he has become a man and has seen the administration of the city and us, the laws, may take his goods and go away wherever he likes. And none of us stands in the way or forbids any of you to take his goods and go away wherever he pleases, if we and the state do not please him, whether it be to an Athenian colony or to a foreign country where he will live as an alien. But we say that \marginnote{\small \fontfamily{ppl}\selectfont \itshape e} whoever of you stays here, seeing how we administer justice and how we govern the state in other respects, has thereby entered into an agreement with us to do what we command; and we say that he who does not obey does threefold wrong, because he disobeys us who are his parents, because he disobeys us who nurtured him, and because after agreeing to obey us he neither obeys us nor convinces us that \marginnote{\small \fontfamily{ppl}\selectfont \itshape 52 a} we are wrong, though we give him the opportunity and do not roughly order him to do what we command, but when we allow him a choice of two things, either to convince us of error or to do our bidding, he does neither of these things.'' ``We say that you, Socrates, will be exposed to these reproaches, if you do what you have in mind, and you not least of the Athenians but more than most others.'' If then I should say, ``How so?'' perhaps they might retort with justice that I had made this agreement with them more emphatically than most other Athenians. For they would say, \marginnote{\small \fontfamily{ppl}\selectfont \itshape b} ``Socrates, we have strong evidence that we and the city pleased you; for you would never have stayed in it more than all other Athenians if you had not been better pleased with it than they; you never went out from the city to a festival, or anywhere else, except on military service, and you never made any other journey, as other people do, and you had no wish to know any other city or other laws, but you were contented with us and our city. So strongly did you prefer us \marginnote{\small \fontfamily{ppl}\selectfont \itshape c} and agree to live in accordance with us; and besides, you begat children in the city, showing that it pleased you. And moreover even at your trial you might have offered exile as your penalty, if you wished, and might have done with the state's consent what you are now undertaking to do without it. But you then put on airs and said you were not disturbed if you must die, and you preferred, as you said, death to exile. And now you are not ashamed to think of those words and you do not respect us, the laws, since you are trying to bring us to naught; \marginnote{\small \fontfamily{ppl}\selectfont \itshape d} and you are doing what the meanest slave would do, since you are trying to run away contrary to the compacts and agreements you made with us that you would live in accordance with us. First then, answer this question, whether we speak the truth or not when we say that you agreed, not in word, but by your acts, to live in accordance with us.'' What shall we say to this, Crito? Must we not agree that it is true?
\critospeaks 
We must, Socrates.
\socratesspeaks 
``Are you then,'' they would say, ``not breaking \marginnote{\small \fontfamily{ppl}\selectfont \itshape e} your compacts and agreements with us, though you were not led into them by compulsion or fraud, and were not forced to make up your mind in a short time, but had seventy years, in which you could have gone away, if we did not please you and if you thought the agreements were unfair? But you preferred neither Lacedaemon nor Crete, which you are always saying are well governed, nor any other of the Greek states, \marginnote{\small \fontfamily{ppl}\selectfont \itshape 53 a} or of the foreign ones, but you went away from this city less than the lame and the blind and the other cripples. So much more than the other Athenians were you satisfied with the city and evidently therefore with us, its laws; for who would be pleased with a city apart from its laws? And now will you not abide by your agreement? You will if you take our advice, Socrates; and you will not make yourself ridiculous by going away from the city.
\par
\vspace{0.3em}
``For consider. By transgressing in this way and committing these errors, what good will you do to yourself or \marginnote{\small \fontfamily{ppl}\selectfont \itshape b} any of your friends? For it is pretty clear that your friends also will be exposed to the risk of banishment and the loss of their homes in the city or of their property. And you yourself, if you go to one of the nearest cities, to Thebes or Megara---for both are well governed---will go as an enemy, Socrates, to their government, and all who care for their own cities will look askance at you, and will consider you a destroyer of the laws, \marginnote{\small \fontfamily{ppl}\selectfont \itshape c} and you will confirm the judges in their opinion, so that they will think their verdict was just. For he who is destroyer of the laws might certainly be regarded as a destroyer of young and thoughtless men. Will you then avoid the well-governed cities and the most civilized men? And if you do this will your life be worth living? Or will you go to them and have the face to carry on---what kind of conversation, Socrates? The same kind you carried on here, saying that virtue and justice and lawful things and the laws are the most precious things to men? And do you not think that the conduct of Socrates would seem most disgraceful? \marginnote{\small \fontfamily{ppl}\selectfont \itshape d} You cannot help thinking so. Or you will keep away from these places and go to Crito's friends in Thessaly; for there great disorder and lawlessness prevail, and perhaps they would be amused to hear of the ludicrous way in which you ran away from prison by putting on a disguise, a peasant's leathern cloak or some of the other things in which runaways dress themselves up, and changing your appearance. But will no one say that you, an old man, who had probably but a short time yet to live, \marginnote{\small \fontfamily{ppl}\selectfont \itshape e} clung to life with such shameless greed that you transgressed the highest laws? Perhaps not, if you do not offend anyone; but if you do, Socrates, you will have to listen to many things that would be a disgrace to you. So you will live as an inferior and a slave to everyone. And what will you do except feast in Thessaly, as if you had gone to Thessaly to attend a banquet? What will become of our conversations about justice and \marginnote{\small \fontfamily{ppl}\selectfont \itshape 54 a} virtue? But perhaps you wish to live for the sake of your children, that you may bring them up and educate them? How so? Will you take them to Thessaly to be brought up and educated, making exiles of them, that you may give them that blessing also? Or perhaps you will not do that, but if they are brought up here while you are living, will they be better brought up and educated if you are not with them than if you were dead? Oh yes! your friends will care for them. Will they care for them if you go away to Thessaly and not if you go away to the dwellings of the dead? If those who say they are your friends \marginnote{\small \fontfamily{ppl}\selectfont \itshape b} are of any use, we must believe they will care for them in both cases alike.
\par
\vspace{0.3em}
``Ah, Socrates, be guided by us who tended your infancy. Care neither for your children nor for life nor for anything else more than for the right, that when you come to the home of the dead, you may have all these things to say in your own defence. For clearly if you do this thing it will not be better for you here, or more just or holier, no, nor for any of your friends, and neither will it be better when you reach that other abode. Now, however, you will go away wronged, if you do go away, not by us, \marginnote{\small \fontfamily{ppl}\selectfont \itshape c} the laws, but by men; but if you escape after so disgracefully requiting wrong with wrong and evil with evil, breaking your compacts and agreements with us, and injuring those whom you least ought to injure---yourself, your friends, your country and us---we shall be angry with you while you live, and there our brothers, the laws in Hades' realm, will not receive you graciously; for they will know that you tried, so far as in you lay, to destroy us. Do not let Crito persuade you to do \marginnote{\small \fontfamily{ppl}\selectfont \itshape d} what he says, but take our advice.
\par
\vspace{0.3em}
Be well assured, my dear friend, Crito, that this is what I seem to hear, as the frenzied dervishes of Cybele seem to hear the flutes, and this sound of these words re-echoes within me and prevents my hearing any other words. And be assured that, so far as I now believe, if you argue against these words you will speak in vain. Nevertheless, if you think you can accomplish anything, speak.
\critospeaks 
No, Socrates, I have nothing to say. \marginnote{\small \fontfamily{ppl}\selectfont \itshape e}
\socratesspeaks 
Then, Crito, let it be; and let us act in this way, since it is in this way that God leads us.


\end{drama}
\end{document}
