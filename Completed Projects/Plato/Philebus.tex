\documentclass[letterpaper,12pt]{article}
\usepackage[12pt]{moresize}
\usepackage[utf8]{inputenc}
\usepackage[LGR, T1]{fontenc}
\usepackage[english]{babel}
\usepackage[margin=2.5cm, right = 3.5cm]{geometry}
\usepackage{ebgaramond}
\usepackage{dramatist}

\newcommand{\textgreek}[1]{\begingroup\fontencoding{LGR}\selectfont#1\endgroup}

\renewcommand{\thefootnote}{[\fontfamily{ppl}\selectfont\arabic{footnote}]}
\setlength{\skip\footins}{1cm}
\usepackage[]{footmisc}
\renewcommand{\footnotemargin}{3mm} %Setting left margin
\renewcommand{\footnotelayout}{\hspace{2mm}} %spacing between the footnote number and the text of footnote

\usepackage{marginnote}
\renewcommand{\raggedrightmarginnote}{\raggedleft}
\newcommand{\stephpag}[1]{\marginnote{\small\itshape\fontfamily{ppl}\selectfont #1}}

\title{\vspace{-2.5cm} \scshape Philebus \vspace{-8mm}}
\author{}
\date{
	\vspace{-1em}
		\small \fontfamily{ppl}\selectfont Written by Plato, translated by Harold North Fowler, 1925
	\begin{center}
		$\mathsection$
	\end{center}
		\vspace{-2em}
	}

\newenvironment{setting}
	{
		\setlength{\tabcolsep}{4em}
		\begin{center}
			\section*{\normalsize \fontfamily{ppl}\selectfont \itshape \bfseries Persons of the Dialogue\vspace{-1mm}}
			\par
			\begin{tabular}{ll}
	}
	{
			\end{tabular}
		\end{center}
		
		\hrulefill		
	}

\begin{document}

\Character[Socrates]{Socrates.}{socrates} % define characters
\Character[Protarchus]{Protarchus.}{protarchus}
\Character[Philebus]{Philebus.}{philebus}

\begin{minipage}{15.45cm}
\maketitle

\begin{setting}
	\textsc{Socrates.}	& \textsc{Philebus.} \\
	\textsc{Protarchus.}	& 
\end{setting}
\end{minipage}
\setlength{\parindent}{2em}

\begin{drama}
\socratesspeaks
\stephpag{11 a}Observe, then, Protarchus, what the doctrine is which you are now to accept from Philebus, and what our doctrine is, against which you are to argue, if you do not agree with it. \stephpag{b} Shall we make a brief statement of each of them?
\protarchusspeaks
By all means.
\socratesspeaks
Very well: Philebus says that to all living beings enjoyment and pleasure and gaiety and whatever accords with that sort of thing are a good; whereas our contention is that not these, but wisdom and thought and memory and their kindred, right opinion and true reasonings, \stephpag{c} are better and more excellent than pleasure for all who are capable of taking part in them, and that for all those now existing or to come who can partake of them they are the most advantageous of all things. Those are pretty nearly the two doctrines we maintain, are they not, Philebus?
\philebusspeaks
Yes, Socrates, exactly.
\socratesspeaks
And do you, Protarchus, accept this doctrine which is now committed to you?
\protarchusspeaks
I must accept it; for our handsome Philebus has withdrawn.
\socratesspeaks
And must the truth about these doctrines be attained by every possible means? \stephpag{d}
\protarchusspeaks
Yes, it must.
\socratesspeaks
Then let us further agree to this:
\protarchusspeaks
To what?
\socratesspeaks
That each of us will next try to prove clearly that it is a condition and disposition of the soul which can make life happy for all human beings. Is not that what we are going to do?
\protarchusspeaks
It is.
\socratesspeaks
Then you will show that it is the condition of pleasure, and I that it is that of wisdom?
\protarchusspeaks
True.
\socratesspeaks
What if some other life be found superior to these two? \stephpag{e} Then if that life is found to be more akin to pleasure, both of us are defeated, are we not, by the life which has firm possession of this superiority, \stephpag{12 a} but the life of pleasure is victor over the life of wisdom.
\protarchusspeaks
Yes.
\socratesspeaks
But if it is more akin to wisdom, then wisdom is victorious and pleasure is vanquished? Do you agree to that? Or what do you say?
\protarchusspeaks
Yes, I at least am satisfied with that.
\socratesspeaks
But how about you, Philebus? What do you say?
\philebusspeaks
I think and always shall think that pleasure is the victor. But you, Protarchus, will make your own decision.
\protarchusspeaks
Since you entrusted the argument to me, Philebus, you can no longer dictate whether to make the agreement with Socrates or not. \stephpag{b}
\philebusspeaks
True; and for that reason I wash my hands of it and now call upon the goddess\footnote{The goddess of Pleasure, \textgreek{Ἡδονή} personified.} herself to witness that I do so.
\protarchusspeaks
And we also will bear witness to these words of yours. But all the same, Socrates, Philebus may agree or do as he likes, let us try to finish our argument in due order.
\socratesspeaks
We must try, and let us begin with the very goddess who Philebus says is spoken of as Aphrodite but is most truly named Pleasure.
\protarchusspeaks
Quite right. \stephpag{c}
\socratesspeaks
My awe, Protarchus, in respect to the names of the gods is always beyond the greatest human fear. And now I call Aphrodite by that name which is agreeable to her; but pleasure I know has various aspects, and since, as I said, we are to begin with her, we must consider and examine what her nature is. For, when you just simply hear her name, she is only one thing, but surely she takes on all sorts of shapes which are even, in a way, unlike each other. For instance, we say that the man \stephpag{d} who lives without restraint has pleasure, and that the self-restrained man takes pleasure in his very self-restraint; and again that the fool who is full of foolish opinions and hopes is pleased, and also that the wise man takes pleasure in his very wisdom. And would not any person who said these two kinds of pleasure were like each other be rightly regarded as a fool?
\protarchusspeaks
No, Socrates, for though they spring from opposite sources, they are not in themselves opposed to one another; \stephpag{e} for how can pleasure help being of all things most like pleasure, that is, like itself?
\socratesspeaks
Yes, my friend, and color is like color in so far as every one of them is a color they will all be the same, yet we all recognize that black is not only different from white, but is its exact opposite. And so, too, figure is like figure; they are all one in kind but the parts of the kind are in some instances absolutely opposed to each other, \stephpag{13 a} and in other cases there is endless variety of difference; and we can find many other examples of such relations. Do not, therefore, rely upon this argument, which makes all the most absolute opposites identical. I am afraid we shall find some pleasures the opposites of other pleasures.
\protarchusspeaks
Perhaps; but why will that injure my contention?
\socratesspeaks
Because I shall say that, although they are unlike, you apply to them a different designation. For you say that all pleasant things are good. Now no argument contends \stephpag{b} that pleasant things are not pleasant; but whereas most of them are bad and only some are good, as we assert, nevertheless you call them all good, though you confess, if forced to it by argument, that they are unlike. Now what is the identical element which exists in the good and bad pleasures alike and makes you call them all a good?
\protarchusspeaks
What do you mean, Socrates? Do you suppose anyone who asserts that the good is pleasure will concede, or will endure to hear you say, that some pleasures are good \stephpag{c} and others bad?
\socratesspeaks
But you will concede that they are unlike and in some instances opposed to each other.
\protarchusspeaks
Not in so far as they are pleasures.
\socratesspeaks
Here we are again at the same old argument, Protarchus, and we shall presently assert that one pleasure is not different from another, but all pleasures are alike, and the examples just cited do not affect us at all, but we shall behave and talk just like the most worthless \stephpag{d} and inexperienced reasoners.
\protarchusspeaks
In what way do you mean?
\socratesspeaks
Why, if I have the face to imitate you and to defend myself by saying that the utterly unlike is most completely like that which is most utterly unlike it, I can say the same things you said, and we shall prove ourselves to be excessively inexperienced, and our argument will be shipwrecked and lost. Let us, then, back her out, and perhaps if we start fair again we may come to an agreement. \stephpag{e}
\protarchusspeaks
How? Tell me.
\socratesspeaks
Assume, Protarchus, that I am questioned in turn by you.
\protarchusspeaks
What question do I ask?
\socratesspeaks
Whether wisdom and knowledge and intellect and all the things which I said at first were good, when you asked me what is good, will not have the same fate as this argument of yours.
\protarchusspeaks
How is that?
\socratesspeaks
It will appear that the forms of knowledge collectively are many and some of them are unlike each other; but if some of them \stephpag{14 a} turn out to be actually opposites, should I be fit to engage in dialectics now if, through fear of just that, I should say that no form of knowledge is unlike any other, and then, as a consequence, our argument should vanish and be lost, like a tale that is told, and we ourselves should be saved by clinging to some irrational notion?
\protarchusspeaks
No, that must never be, except the part about our being saved. However, I like the equal treatment of your doctrine and mine. Let us grant that pleasures are many and unlike and that the forms of knowledge are many and different. \stephpag{b}
\socratesspeaks
With no concealment, then, Protarchus, of the difference between my good and yours, but with fair and open acknowledgement of it, let us be bold and see if perchance on examination they will tell us whether we should say that pleasure is the good, or wisdom, or some other third principle. For surely the object of our present controversy is not to gain the victory for my assertions or yours, but both of us must fight for the most perfect truth.
\protarchusspeaks
Yes, we must.
\socratesspeaks
Then let us establish this principle still more firmly \stephpag{c} by means of an agreement.
\protarchusspeaks
What principle?
\socratesspeaks
The principle which gives trouble to all men, to some of them sometimes against their will.
\protarchusspeaks
Speak more plainly.
\socratesspeaks
I mean the principle which came in our way just now; its nature is quite marvellous. For the assertions that one is many and many are one are marvellous, and it is easy to dispute with anyone who makes either of them.
\protarchusspeaks
You mean when a person says that I, Protarchus, \stephpag{d} am by nature one and that there are also many of me which are opposites of each other, asserting that I, the same Protarchus, am great and small and heavy and light and countless other things?
\socratesspeaks
Those wonders concerning the one and the many which you have mentioned, Protarchus, are common property, and almost everybody is agreed that they ought to be disregarded because they are childish and easy and great hindrances to speculation; and this sort of thing also should be disregarded, \stephpag{e} when a man in his discussion divides the members and likewise the parts of anything, acknowledges that they all collectively are that one thing, and then mockingly refutes himself because he has been compelled to declare miracles---that the one is many and infinite and the many only one.
\protarchusspeaks
But what other wonders do you mean, Socrates, in relation to this same principle, which are not yet common property and generally acknowledged? \stephpag{15 a}
\socratesspeaks
I mean, my boy, when a person postulates unity which is not the unity of one of the things which come into being and perish, as in the examples we had just now. For in cases of a unity of that sort, as I just said, it is agreed that refutation is needless. But when the assertion is made that man is one, or ox is one, or beauty is one, or the good is one, the intense interest in these and similar unities becomes disagreement and controversy.
\protarchusspeaks
How is that? \stephpag{b}
\socratesspeaks
The first question is whether we should believe that such unities really exist; the second, how these unities, each of which is one, always the same, and admitting neither generation nor destruction, can nevertheless be permanently this one unity; and the third, how in the infinite number of things which come into being this unity, whether we are to assume that it is dispersed and has become many, or that it is entirely separated from itself---which would seem to be the most impossible notion of all being the same and one, is to be at the same time in one and in many. These are the questions, Protarchus, about this kind of one and many, \stephpag{c} not those others, which cause the utmost perplexity, if ill solved, and are, if well solved, of the greatest assistance.
\protarchusspeaks
Then is it now, Socrates, our first duty to thresh this matter out?
\socratesspeaks
Yes, that is what I should say.
\protarchusspeaks
You may assume, then, that we are all willing to agree with you about that; and perhaps it is best not to ask Philebus any questions; let sleeping dogs lie. \stephpag{d}
\socratesspeaks
Very well; then where shall we begin this great and vastly complicated battle about the matters at issue? Shall we start at this point?
\protarchusspeaks
At what point?
\socratesspeaks
We say that one and many are identified by reason, and always, both now and in the past, circulate everywhere in every thought that is uttered. This is no new thing and will never cease; it is, in my opinion, a quality within us which will never die or grow old, and which belongs to reason itself as such. And any young man, when he first has an inkling of this, is delighted, \stephpag{e} thinking he has found a treasure of wisdom; his joy fills him with enthusiasm; he joyously sets every possible argument in motion, sometimes in one direction, rolling things up and kneading them into one, and sometimes again unrolling and dividing them; he gets himself into a muddle first and foremost, then anyone who happens to be near him, whether he be younger or older or of his own age; \stephpag{16 a} he spares neither father nor mother nor any other human being who can hear, and hardly even the lower animals, for he would certainly not spare a foreigner,\footnote{Apparently foreigners are considered among the lower animals.} if he could get an interpreter anywhere.
\protarchusspeaks
Socrates, do you not see how many we are and that we are all young men? Are you not afraid that we shall join with Philebus and attack you, if you revile us? However---for we understand your meaning---if there is any way or means of removing this confusion gently from our discussion \stephpag{b} and finding some better road than this to bring us towards the goal of our argument, kindly lead on, and we will do our best to follow for our present discussion, Socrates, is no trifling matter.
\socratesspeaks
No, it is not, boys, as Philebus calls you; and there certainly is no better road, nor can there ever be, than that which I have always loved, though it has often deserted me, leaving me lonely and forlorn.
\protarchusspeaks
What is the road? Only tell us. \stephpag{c}
\socratesspeaks
One which is easy to point out, but very difficult to follow for through it all the inventions of art have been brought to light. See this is the road I mean.
\protarchusspeaks
Go on what is it?
\socratesspeaks
A gift of gods to men, as I believe, was tossed down from some divine source through the agency of a Prometheus together with a gleaming fire; and the ancients, who were better than we and lived nearer the gods, handed down the tradition that all the things which are ever said to exist are sprung from one and many and have inherent in them the finite and the infinite. This being the way in which these things are arranged, \stephpag{d} we must always assume that there is in every case one idea of everything and must look for it---for we shall find that it is there---and if we get a grasp of this, we must look next for two, if there be two, and if not, for three or some other number; and again we must treat each of those units in the same way, until we can see not only that the original unit is one and many and infinite, but just how many it is. And we must not apply the idea of infinite to plurality until we have a view of its whole number \stephpag{e} between infinity and one; then, and not before, we may let each unit of everything pass on unhindered into infinity. The gods, then, as I said, handed down to us this mode of investigating, learning, and teaching one another; but the wise men of the present day make the one \stephpag{17 a} and the many too quickly or too slowly, in haphazard fashion, and they put infinity immediately after unity; they disregard all that lies between them, and this it is which distinguishes between the dialectic and the disputatious methods of discussion.
\protarchusspeaks
I think I understand you in part, Socrates, but I need a clearer statement of some things.
\socratesspeaks
Surely my meaning, Protarchus, is made clear in the letters of the alphabet, which you were taught as a child; \stephpag{b} so learn it from them.
\protarchusspeaks
How?
\socratesspeaks
Sound, which passes out through the mouth of each and all of us, is one, and yet again it is infinite in number.
\protarchusspeaks
Yes, to be sure.
\socratesspeaks
And one of us is no wiser than the other merely for knowing that it is infinite or that it is one; but that which makes each of us a grammarian is the knowledge of the number and nature of sounds.
\protarchusspeaks
Very true.
\socratesspeaks
And it is this same knowledge which makes the musician.
\protarchusspeaks
How is that? \stephpag{c}
\socratesspeaks
Sound is one in the art of music also, so far as that art is concerned.
\protarchusspeaks
Of course.
\socratesspeaks
And we may say that there are two sounds, low and high, and a third, which is the intermediate, may we not?
\protarchusspeaks
Yes.
\socratesspeaks
But knowledge of these facts would not suffice to make you a musician, although ignorance of them would make you, if I may say so, quite worthless in respect to music.
\protarchusspeaks
Certainly.
\socratesspeaks
But, my friend, when you have grasped the number and quality of the intervals of the voice in respect to high and low pitch, and the limits of the intervals, \stephpag{d} and all the combinations derived from them, which the men of former times discovered and handed down to us, their successors, with the traditional name of harmonies, and also the corresponding effects in the movements of the body, which they say are measured by numbers and must be called rhythms and measures---and they say that we must also understand that every one and many should be considered in this way--- \stephpag{e} when you have thus grasped the facts, you have become a musician, and when by considering it in this way you have obtained a grasp of any other unity of all those which exist, you have become wise in respect to that unity. But the infinite number of individuals and the infinite number in each of them makes you in every instance indefinite in thought and of no account and not to be considered among the wise, so long as you have never fixed your eye upon any definite number in anything.
\protarchusspeaks
I think, Philebus, that what Socrates has said is excellent.
\philebusspeaks
So do I; it is excellent in itself, but why has he said it now to us, \stephpag{18 a} and what purpose is there in it?
\socratesspeaks
Protarchus, that is a very proper question which Philebus has asked us.
\protarchusspeaks
Certainly it is, so please answer it.
\socratesspeaks
I will, when I have said a little more on just this subject. For if a person begins with some unity or other, he must, as I was saying, not turn immediately to infinity, but to some definite number; now just so, conversely, when he has to take the infinite first, \stephpag{b} he must not turn immediately to the one, but must think of some number which possesses in each case some plurality, and must end by passing from all to one. Let us revert to the letters of the alphabet to illustrate this.
\protarchusspeaks
How?
\socratesspeaks
When some one, whether god or godlike man,---there is an Egyptian story that his name was Theuth---observed that sound was infinite, he was the first to notice that the vowel sounds in that infinity were not one, but many, and again that there were other elements which were not vowels but did have a sonant quality, \stephpag{c} and that these also had a definite number; and he distinguished a third kind of letters which we now call mutes. Then he divided the mutes until he distinguished each individual one, and he treated the vowels and semivowels in the same way, until he knew the number of them and gave to each and all the name of letters. Perceiving, however, that none of us could learn any one of them alone by itself without learning them all, and considering that this was a common bond which made them in a way all one, \stephpag{d} he assigned to them all a single science and called it grammar.
\philebusspeaks
I understand that more clearly than the earlier statement, Protarchus, so far as the reciprocal relations of the one and the many are concerned, but I still feel the same lack as a little while ago.
\socratesspeaks
Do you mean, Philebus, that you do not see what this has to do with the question?
\philebusspeaks
Yes; that is what Protarchus and I have been trying to discover for a long time.
\socratesspeaks
Really, have you been trying, as you say, \stephpag{e} for long time to discover it, when it was close to you all the while?
\philebusspeaks
How is that?
\socratesspeaks
Was not our discussion from the beginning about wisdom and pleasure and which of them is preferable?
\philebusspeaks
Yes, of course.
\socratesspeaks
And surely we say that each of them is one.
\philebusspeaks
Certainly.
\socratesspeaks
This, then, is precisely the question which the previous discussion puts to us: How is each of them one and many, and how is it that they are not immediately infinite, but each possesses a definite number, before the individual phenomena become infinite? \stephpag{19 a}
\protarchusspeaks
Philebus, somehow or other Socrates has led us round and plunged us into a serious question. Consider which of us shall answer it. Perhaps it is ridiculous that I, after taking your place in entire charge of the argument, should ask you to come back and answer this question because I cannot do so, but I think it would be still more ridiculous if neither of us could answer. \stephpag{b} Consider, then, what we are to do. For I think Socrates is asking us whether there are or are not kinds of pleasure, how many kinds there are, and what their nature is, and the same of wisdom.
\socratesspeaks
You are quite right, son of Callias; for, as our previous discussion showed, unless we can do this in the case of every unity, every like, every same, and their opposites, none of us can ever be of any use in anything. \stephpag{c}
\protarchusspeaks
That, Socrates, seems pretty likely to be true. However, it is splendid for the wise man to know everything, but the next best thing, it seems, is not to be ignorant of himself. I will tell you why I say that at this moment. You, Socrates, have granted to all of us this conversation and your cooperation for the purpose of determining what is the best of human possessions. For when Philebus said it was pleasure and gaiety and enjoyment and all that sort of thing, you objected and said it was not those things, but another sort, \stephpag{d} and we very properly keep reminding ourselves voluntarily of this, in order that both claims may be present in our memory for examination. You, as it appears, assert that the good which is rightly to be called better than pleasure is mind, knowledge, intelligence, art, and all their kin; you say we ought to acquire these, not that other sort. When those two claims were made and an argument arose, we playfully threatened that we would not let you go home \stephpag{e} until the discussion was brought to some satisfactory conclusion. You agreed and put yourself at our disposal for that purpose. Now, we say that, as children put it, you cannot take back a gift once fairly given. So cease this way of meeting all that we say.
\socratesspeaks
What way do you mean? \stephpag{20 a}
\protarchusspeaks
I mean puzzling us and asking questions to which we cannot at the moment give a satisfactory answer. Let us not imagine that the end of our present discussion is a mere puzzling of us all, but if we cannot answer, you must do so; for you gave us a promise. Consider, therefore, whether you yourself must distinguish the kinds of pleasure and knowledge or will let that go, in case you are able and willing to make clear in some other way the matters now at issue among us. \stephpag{b}
\socratesspeaks
I need no longer anticipate anything terrible, since you put it in that way; for the words ``in case you are willing" relieve me of all fear. And besides, I think some god has given me a vague recollection.
\protarchusspeaks
How is that, and what is the recollection about?
\socratesspeaks
I remember now having heard long ago in a dream, or perhaps when I was awake, some talk about pleasure and wisdom to the effect that neither of the two is the good, but some third thing, different from them and better than both. \stephpag{c} However, if this be now clearly proved to us, pleasure is deprived of victory for the good would no longer be identical with it. Is not that true?
\protarchusspeaks
It is.
\socratesspeaks
And we shall have, in my opinion, no longer any need of distinguishing the kinds of pleasure. But the progress of the discussion will make that still clearer.
\protarchusspeaks
Excellent! Just go on as you have begun.
\socratesspeaks
First, then, let us agree on some further small points.
\protarchusspeaks
What are they?
\socratesspeaks
Is the nature of the good necessarily perfect \stephpag{d} or imperfect?
\protarchusspeaks
The most perfect of all things, surely, Socrates.
\socratesspeaks
Well, and is the good sufficient?
\protarchusspeaks
Of course; so that it surpasses all other things in sufficiency.
\socratesspeaks
And nothing, I should say, is more certain about it than that every intelligent being pursues it, desires it, wishes to catch and get possession of it, and has no interest in anything in which the good is not included.
\protarchusspeaks
There is no denying that. \stephpag{e}
\socratesspeaks
Let us, then, look at the life of pleasure and the life of wisdom separately and consider and judge them.
\protarchusspeaks
How do you mean?
\socratesspeaks
Let there be no wisdom in the life of pleasure and no pleasure in the life of wisdom. For if either of them is the good, it cannot have need of anything else, and if, either be found to need anything, \stephpag{21 a} we can no longer regard it as our true good.
\protarchusspeaks
No, of course not.
\socratesspeaks
Shall we then undertake to test them through you?
\protarchusspeaks
By all means.
\socratesspeaks
Then answer.
\protarchusspeaks
Ask.
\socratesspeaks
Would you, Protarchus, be willing to live your whole life in the enjoyment of the greatest pleasures?
\protarchusspeaks
Of course I should.
\socratesspeaks
Would you think you needed anything further, if you were in complete possession of that enjoyment?
\protarchusspeaks
Certainly not.
\socratesspeaks
But consider whether you would not have some need of wisdom and intelligence and \stephpag{b} power of calculating your wants and the like.
\protarchusspeaks
Why should I? If I have enjoyment, I have everything.
\socratesspeaks
Then living thus you would enjoy the greatest pleasures all your life?
\protarchusspeaks
Yes; why not?
\socratesspeaks
But if you did not possess mind or memory or knowledge or true opinion, in the first place, you would not know whether you were enjoying your pleasures or not. That must be true, since you are utterly devoid of intellect, must it not?
\protarchusspeaks
Yes, it must. \stephpag{c}
\socratesspeaks
And likewise, if you had no memory you could not even remember that you ever did enjoy pleasure, and no recollection whatever of present pleasure could remain with you; if you had no true opinion you could not think you were enjoying pleasure at the time when you were enjoying it, and if you were without power of calculation you would not be able to calculate that you would enjoy it in the future; your life would not be that of a man, but of a mollusc or some other shell-fish like the oyster. \stephpag{d} Is that true, or can we imagine any other result?
\protarchusspeaks
We certainly cannot.
\socratesspeaks
And can we choose such a life?
\protarchusspeaks
This argument, Socrates, has made me utterly speechless for the present.
\socratesspeaks
Well, let us not give in yet. Let us take up the life of mind and scrutinize that in turn.
\protarchusspeaks
What sort of life do you mean?
\socratesspeaks
I ask whether anyone would be willing to live possessing wisdom and mind and knowledge and perfect memory of all things, \stephpag{e} but having no share, great or small, in pleasure, or in pain, for that matter, but being utterly unaffected by everything of that sort.
\protarchusspeaks
Neither of the two lives can ever appear desirable to me, Socrates, or, I think, to anyone else. \stephpag{22 a}
\socratesspeaks
How about the combined life, Protarchus, made up by a union of the two?
\protarchusspeaks
You mean a union of pleasure with mind or wisdom?
\socratesspeaks
Yes, I mean a union of such elements.
\protarchusspeaks
Every one will prefer this life to either of the two others---yes, every single person without exception.
\socratesspeaks
Then do we understand the consequences of what we are now saying?
\protarchusspeaks
Certainly. Three lives have been proposed, \stephpag{b} and of two of them neither is sufficient or desirable for man or any other living being.
\socratesspeaks
Then is it not already clear that neither of these two contained the good for if it did contain the good, it would be sufficient and perfect, and such as to be chosen by all living creatures which would be able to live thus all their lives; and if any of us chose anything else, he would be choosing contrary to the nature of the truly desirable, not of his own free will, but from ignorance or some unfortunate necessity.
\protarchusspeaks
That seems at any rate to be true. \stephpag{c}
\socratesspeaks
And so I think we have sufficiently proved that PhilebusÕs divinity is not to be considered identical with the good.
\philebusspeaks
But neither is your ``mind" the good, Socrates; it will be open to the same objections.
\socratesspeaks
My mind, perhaps, Philebus; but not so, I believe, the true mind, which is also divine; that is different. I do not as yet claim for mind the victory over the combined life, but we must look and see what is to be done about the second place; \stephpag{d} for each of us might perhaps put forward a claim, one that mind is the cause of this combined life, the other that pleasure is the cause and thus neither of these two would be the good, but one or the other of them might be regarded as the cause of the good. On this point I might keep up the fight all the more against Philebus and contend that in this mixed life it is mind that is more akin and more similar than pleasure to that, whatever it may be, which makes it both desirable and good; and from this point of view \stephpag{e} pleasure could advance no true claim to the first or even the second place. It is farther behind than the third place, if my mind is at all to be trusted at present.
\protarchusspeaks
Certainly, Socrates, it seems to me that pleasure has fought for the victory and has fallen in this bout, knocked down by your words. \stephpag{23 a} And we can only say, as it seems, that mind was wise in not laying claim to the victory; for it would have met with the same fate. Now pleasure, if she were to lose the second prize, would be deeply humiliated in the eyes of her lovers; for she would no longer appear even to them so lovely as before.
\socratesspeaks
Well, then, is it not better to leave her now and not to pain her by testing her to the utmost and proving her in the wrong?
\protarchusspeaks
Nonsense, Socrates! \stephpag{b}
\socratesspeaks
Nonsense because I spoke of paining pleasure, and that is impossible?
\protarchusspeaks
Not only that, but because you do not understand that not one of us will let you go yet until you have finished the argument about these matters.
\socratesspeaks
Whew, Protarchus! Then we have a long discussion before us, and not an easy one, either, this time. For in going ahead to fight mindÕs battle for the second place, I think I need a new contrivance---other weapons, as it were, than those of our previous discussion, though perhaps some of the old ones will serve. Must I then go on?
\protarchusspeaks
Of course you must.
\socratesspeaks
Then let us try to be careful \stephpag{c} in making our beginning.
\protarchusspeaks
What kind of a beginning do you mean?
\socratesspeaks
Let us divide all things that now exist in the universe into two, or rather, if you please, three classes.
\protarchusspeaks
Please tell us on what principle you would divide them.
\socratesspeaks
Let us take some of the subjects of our present discussion.
\protarchusspeaks
What subjects?
\socratesspeaks
We said that God revealed in the universe two elements, the infinite and the finite, did we not?
\protarchusspeaks
Certainly.
\socratesspeaks
Let us, then, assume these as two of our classes, and a third, made by combining these two. \stephpag{d} But I cut a ridiculous figure, it seems, when I attempt a division into classes and an enumeration.
\protarchusspeaks
What do you mean, my friend?
\socratesspeaks
I think we need a fourth class besides.
\protarchusspeaks
Tell us what it is.
\socratesspeaks
Note the cause of the combination of those two and assume that as the fourth in addition to the previous three.
\protarchusspeaks
And then will you not need a fifth, which has the power of separation?
\socratesspeaks
Perhaps; but not at present, I think. However, if we do need a fifth, \stephpag{e} you will pardon me for going after it.
\protarchusspeaks
Of course.
\socratesspeaks
First, then, let us take three of the four and, as we see that two of these are split up and scattered each one into many, let us try, by collecting each of them again into one, to learn how each of them was both one and many.
\protarchusspeaks
If you could tell me more clearly about them, I might be able to follow you. \stephpag{24 a}
\socratesspeaks
I mean, then, that the two which I select are the same which I mentioned before, the infinite and the finite. I will try to show that the infinite is, in a certain sense, many; the finite can wait.
\protarchusspeaks
Yes.
\socratesspeaks
Consider then. What I ask you to consider is difficult and debatable; but consider it all the same. In the first place, take hotter and colder and see whether you can conceive any limit of them, or whether the more and less which dwell in their very nature do not, so long as they continue to dwell therein, \stephpag{b} preclude the possibility of any end; for if there were any end of them, the more and less would themselves be ended.
\protarchusspeaks
Very true.
\socratesspeaks
But always, we affirm, in the hotter and colder there is the more and less.
\protarchusspeaks
Certainly.
\socratesspeaks
Always, then, the argument shows that these two have no end; and being endless, they are of course infinite.
\protarchusspeaks
Most emphatically, Socrates.
\socratesspeaks
I am glad you responded, my dear Protarchus, \stephpag{c} and reminded me that the word ``emphatically ``which you have just used, and the word ``gently" have the same force as ``more" and ``less." For wherever they are present, they do not allow any definite quantity to exist; they always introduce in every instance a comparison---more emphatic than that which is quieter, or vice versa---and thus they create the relation of more and less, thereby doing away with fixed quantity. For, as I said just now, if they did not abolish quantity, but allowed it and measure to make their appearance in the abode of the more and less, \stephpag{d} the emphatically and gently, those latter would be banished from their own proper place. When once they had accepted definite quantity, they would no longer be hotter or colder; for hotter and colder are always progressing and never stationary; but quantity is at rest and does not progress. By this reasoning hotter and its opposite are shown to be infinite.
\protarchusspeaks
That appears to be the case, Socrates; but, as you said, these subjects are not easy to follow. Perhaps, however, \stephpag{e} continued repetition might lead to a satisfactory agreement between the questioner and him who is questioned.
\socratesspeaks
That is a good suggestion, and I must try to carry it out. However, to avoid waste of time in discussing all the individual examples, see if we can accept this as a designation of the infinite.
\protarchusspeaks
Accept what?
\socratesspeaks
All things which appear to us to become more or less, or to admit of emphatic and gentle \stephpag{25 a} and excessive and the like, are to be put in the class of the infinite as their unity, in accordance with what we said a while ago, if you remember, that we ought to collect all things that are scattered and split up and impress upon them to the best of our ability the seal of some single nature.
\protarchusspeaks
I remember.
\socratesspeaks
And the things which do not admit of more and less and the like, but do admit of all that is opposed to them---first equality and the equal, then the double, and anything which is a definite number or measure in relation to such a number or measure--- \stephpag{b} all these might properly be assigned to the class of the finite. What do you say to that?
\protarchusspeaks
Excellent, Socrates.
\socratesspeaks
Well, what shall we say is the nature of the third class, made by combining these two?
\protarchusspeaks
You will tell me, I fancy, by answering your own question.
\socratesspeaks
Nay, a god will do so, if any god will give ear to my prayers.
\protarchusspeaks
Pray, then, and watch.
\socratesspeaks
I am watching; and I think, Protarchus, one of the gods has this moment been gracious unto me. \stephpag{c}
\protarchusspeaks
What do you mean, and what evidence have you?
\socratesspeaks
I will tell you, of course. Just follow what I say.
\protarchusspeaks
Say on.
\socratesspeaks
We spoke just now of hotter and colder, did we not?
\protarchusspeaks
Yes.
\socratesspeaks
Add to them drier and wetter, more and less, quicker and slower, greater and smaller, and all that we assigned before to the class which unites more and less. \stephpag{d}
\protarchusspeaks
You mean the class of the infinite?
\socratesspeaks
Yes. Mix with that the second class, the offspring of the limit.
\protarchusspeaks
What class do you mean?
\socratesspeaks
The class of the finite, which we ought just now to have reduced to unity, as we did that of the infinite. We have not done that, but perhaps we shall even now accomplish the same end, if these two are both unified and then the third class is revealed.
\protarchusspeaks
What third class, and what do you mean?
\socratesspeaks
The class of the equal and double and everything which puts an end \stephpag{e} to the differences between opposites and makes them commensurable and harmonious by the introduction of number.
\protarchusspeaks
I understand. I think you mean that by mixture of these elements certain results are produced in each instance.
\socratesspeaks
Yes, you are right.
\protarchusspeaks
Go on.
\socratesspeaks
In cases of illness, does not the proper combination of these elements produce health? \stephpag{26 a}
\protarchusspeaks
Certainly.
\socratesspeaks
And in the acute and the grave, the quick and the slow, which are unlimited, the addition of these same elements creates a limit and establishes the whole art of music in all its perfection, does it not?
\protarchusspeaks
Excellent.
\socratesspeaks
And again in the case of cold and hot weather, the introduction of these elements removes the excess and indefiniteness and creates moderation and harmony.
\protarchusspeaks
Assuredly.
\socratesspeaks
And thence arise the seasons and all the beauties of our world, \stephpag{b} by mixture of the infinite with the finite?
\protarchusspeaks
Of course.
\socratesspeaks
There are countless other things which I pass over, such as health, beauty, and strength of the body and the many glorious beauties of the soul. For this goddess,\footnote{This goddess may be \textgreek{Μουσική} (in which case \textgreek{ἐγγενομένη} the reading of T and G, would be preferable to \textgreek{ἐγγενόμενα} above), not music in the restricted modern sense, but the spirit of numbers and measure which underlies all music, and all the beauties of the world; or the goddess may be mentioned here in reference (and opposition) to the goddess Pleasure (12 B); she is the nameless deity who makes Pleasure and all others conform to her rules.} my fair Philebus, beholding the violence and universal wickedness which prevailed, since there was no limit of pleasures or of indulgence in them, established law and order, which contain a limit. You say she did harm; \stephpag{c} I say, on the contrary, she brought salvation. What do you think, Protarchus?
\protarchusspeaks
What you say, Socrates, pleases me greatly.
\socratesspeaks
I have spoken of these three classes, you observe.
\protarchusspeaks
Yes, I believe I understand; I think you mean that the infinite is one class and the finite is another class among existing things; but what you wish to designate as the third class, I do not comprehend very well.
\socratesspeaks
No, because the multitude which springs up in the third class overpowers you and yet the infinite also comprised many classes, \stephpag{d} nevertheless, since they were sealed with the seal of the more and less, they were seen to be of one class.
\protarchusspeaks
True.
\socratesspeaks
And the finite, again, did not contain many classes, nor were we disturbed about its natural unity.
\protarchusspeaks
Of course not.
\socratesspeaks
No, not at all. And as to the third class, understand that I mean every offspring of these two which comes into being as a result of the measures created by the cooperation of the finite.
\protarchusspeaks
I understand. \stephpag{e}
\socratesspeaks
But we said there was, in addition to three classes, a fourth to be investigated. Let us do that together. See whether you think that everything which comes into being must necessarily come into being through a cause.
\protarchusspeaks
Yes, I do; for how could it come into being apart from a cause?
\socratesspeaks
Does not the nature of that which makes or creates differ only in name from the cause, and may not the creative agent and the cause be properly considered one?
\protarchusspeaks
Yes. \stephpag{27 a}
\socratesspeaks
And, again, we shall find that, on the same principle, that which is made or created differs in name only from that which comes into being, shall we not?
\protarchusspeaks
We shall.
\socratesspeaks
And the creative agent always naturally leads, and that which is created follows after it as it comes into being?
\protarchusspeaks
Certainly.
\socratesspeaks
Then the cause and that which is the servant of the cause for the purpose of generation are not the same.
\protarchusspeaks
Of course not.
\socratesspeaks
Did not the things which come into being and the things out of which they come into being furnish us all the three classes?
\protarchusspeaks
Certainly. \stephpag{b}
\socratesspeaks
And that which produces all these, the cause, we call the fourth, as it has been satisfactorily shown to be distinct from the others?
\protarchusspeaks
Yes, it is distinct.
\socratesspeaks
It is, then, proper, now that we have distinguished the four, to make sure that we remember them separately by enumerating them in order.
\protarchusspeaks
Yes, certainly.
\socratesspeaks
The first, then, I call infinite, the second limit or finite, and the third something generated by a mixture of these two. And should I be making any mistake if I called \stephpag{c} the cause of this mixture and creation the fourth?
\protarchusspeaks
Certainly not.
\socratesspeaks
Now what is the next step in our argument, and what was our purpose in coming to the point we have reached? Was it not this? We were trying to find out whether the second place belonged to pleasure or to wisdom, were we not?
\protarchusspeaks
Yes, we were.
\socratesspeaks
And may we not, perhaps, now that we have finished with these points, be better able to come to a decision about the first and second places, which was the original subject of our discussion?
\protarchusspeaks
Perhaps. \stephpag{d}
\socratesspeaks
Well then; we decided that the mixed life of pleasure and wisdom was the victor, did we not?
\protarchusspeaks
Yes.
\socratesspeaks
And do we not see what kind of life this is, and to what class it belongs?
\protarchusspeaks
Of course we do.
\socratesspeaks
We shall say that it belongs to the third class; for that class is not formed by mixture of any two things, but of all the things which belong to the infinite, bound by the finite; and therefore this victorious life would rightly be considered a part of this class.
\protarchusspeaks
Quite rightly. \stephpag{e}
\socratesspeaks
Well then, what of your life, Philebus, of unmixed pleasure? In which of the aforesaid classes may it properly be said to belong? But before you tell me, please answer this question.
\philebusspeaks
Ask your question.
\socratesspeaks
Have pleasure and pain a limit, or are they among the things which admit of more and less?
\philebusspeaks
Yes, they are among those which admit of the more, Socrates; for pleasure would not be absolute good if it were not infinite in number and degree. \stephpag{28 a}
\socratesspeaks
Nor would pain, Philebus, be absolute evil; so it is not the infinite which supplies any element of good in pleasure; we must look for something else. Well, I grant you that pleasure and pain are in the class of the infinite but to which of the aforesaid classes, Protarchus and Philebus, can we now without irreverence assign wisdom, knowledge, and mind? I think we must find the right answer to this question, for our danger is great if we fail. \stephpag{b}
\philebusspeaks
Oh Socrates, you exalt your own god.
\socratesspeaks
And you your goddess, my friend. But the question calls for an answer, all the same.
\protarchusspeaks
Socrates is right, Philebus; you ought to do as he asks.
\philebusspeaks
Did you not, Protarchus, elect to reply in my place?
\protarchusspeaks
Yes; but now I am somewhat at a loss, and I ask you, Socrates, to be our spokesman yourself, that we may not select the wrong representative and so say something improper. \stephpag{c}
\socratesspeaks
I must do as you ask, Protarchus; and it is not difficult. But did I really, as Philebus said, embarrass you by playfully exalting my god, when I asked to what class mind and knowledge should be assigned?
\protarchusspeaks
You certainly did, Socrates.
\socratesspeaks
Yet the answer is easy; for all philosophers agree---whereby they really exalt themselves---that mind is king of heaven and earth. Perhaps they are right. But let us, if you please, investigate the question of its class more at length. \stephpag{d}
\protarchusspeaks
Speak just as you like, Socrates. Do not consider length, so far as we are concerned you cannot bore us.
\socratesspeaks
Good. Then let us begin by asking a question.
\protarchusspeaks
What is the question?
\socratesspeaks
Shall we say, Protarchus, that all things and this which is called the universe are governed by an irrational and fortuitous power and mere chance, or, on the contrary, as our forefathers said, are ordered and directed by mind and a marvellous wisdom? \stephpag{e}
\protarchusspeaks
The two points of view have nothing in common, my wonderful Socrates. For what you are now saying seems to me actually impious. But the assertion that mind orders all things is worthy of the aspect of the world, of sun, moon, stars, and the whole revolving universe; I can never say or think anything else about it.
\socratesspeaks
Do you, then, think we should assent to this and agree in the doctrine of our predecessors, \stephpag{29 a} not merely intending to repeat the words of others, with no risk to ourselves, but ready to share with them in the risk and the blame, if any clever man declares that this world is not thus ordered, but is without order?
\protarchusspeaks
Yes, of course I do.
\socratesspeaks
Then observe the argument that now comes against us.
\protarchusspeaks
Go on.
\socratesspeaks
We see the elements which belong to the natures of all living beings, fire, water, air, and earth---or, as the storm-tossed mariners say, land in sight--- \stephpag{b} in the constitution of the universe.
\protarchusspeaks
Certainly and we are truly storm-tossed in the puzzling cross-currents of this discussion.
\socratesspeaks
Well, here is a point for you to consider in relation to each of these elements as it exists in us.
\protarchusspeaks
What is the point?
\socratesspeaks
Each element in us is small and poor and in no way pure at all or endowed with the power which is worthy of its nature. Take one example and apply it to all. Fire, for instance, exists in us and also in the universe.
\protarchusspeaks
Of course. \stephpag{c}
\socratesspeaks
And that which is in us is small, weak, and poor, but that which is in the universe is marvellous in quantity, beauty, and every power which belongs to fire.
\protarchusspeaks
What you say is very true.
\socratesspeaks
Well, is the fire of the universe nourished, originated, and ruled by the fire within us, or, on the contrary, does my fire, and yours, and that of all living beings derive nourishment and all that from the universal fire?
\protarchusspeaks
That question does not even deserve an answer. \stephpag{d}
\socratesspeaks
True; and you will, I fancy, say the same of the earth which is in us living creatures and that which is in the universe, and concerning all the other elements about which I asked a moment ago your answer will be the same.
\protarchusspeaks
Yes. Who could answer otherwise without being called a lunatic?
\socratesspeaks
Nobody, I fancy. Now follow the next step. When we see that all the aforesaid elements are gathered together into a unit, do we not call them a body?
\protarchusspeaks
Of course. \stephpag{e}
\socratesspeaks
Apply the same line of thought to that which we call the universe. It would likewise be a body, being composed of the same elements.
\protarchusspeaks
Quite right.
\socratesspeaks
Does our body derive, obtain, and possess from that body, or that body from ours, nourishment and everything else that we mentioned just now?
\protarchusspeaks
That, Socrates, is another question not worth asking. \stephpag{30 a}
\socratesspeaks
Well, is this next one worth asking? What will you say to it?
\protarchusspeaks
What is it?
\socratesspeaks
Shall we not say that our body has a soul?
\protarchusspeaks
Clearly we shall.
\socratesspeaks
Where did it get it, Protarchus, unless the body of the universe had a soul, since that body has the same elements as ours, only in every way superior?
\protarchusspeaks
Clearly it could get it from no other source.
\socratesspeaks
No; for we surely do not believe, Protarchus, that of those four elements, the finite, the infinite, the combination, \stephpag{b} and the element of cause which exists in all things, this last, which gives to our bodies souls and the art of physical exercise and medical treatment when the body is ill, and which is in general a composing and healing power, is called the sum of all wisdom, and yet, while these same elements exist in the entire heaven and in great parts thereof, and area moreover, fair and pure, there is no means of including among them that nature which is the fairest and most precious of all. \stephpag{c}
\protarchusspeaks
Certainly there would be no sense in that.
\socratesspeaks
Then if that is not the case, it would be better to follow the other line of thought and say, as we have often said, that there is in the universe a plentiful infinite and a sufficient limit, and in addition a by no means feeble cause which orders and arranges years and seasons and months, and may most justly be called wisdom and mind.
\protarchusspeaks
Yes, most justly.
\socratesspeaks
Surely reason and mind could never come into being without soul.
\protarchusspeaks
No, never.
\socratesspeaks
Then in the nature of Zeus you would say that a kingly soul \stephpag{d} and a kingly mind were implanted through the power of the cause, and in other deities other noble qualities from which they derive their favorite epithets.
\protarchusspeaks
Certainly.
\socratesspeaks
Now do not imagine, Protarchus, that this is mere idle talk of mine; it confirms the utterances of those who declared of old\footnote{Anaxagoras and probably some now unknown precursors.} that mind always rules the universe.
\protarchusspeaks
Yes, certainly.
\socratesspeaks
And to my question it has furnished the reply \stephpag{e} that mind belongs to that one of our four classes which was called the cause of all. Now, you see, you have at last my answer.
\protarchusspeaks
Yes, and a very sufficient one and yet you answered without my knowing it.
\socratesspeaks
Yes, Protarchus, for sometimes a joke is a restful change from serious talk.
\protarchusspeaks
You are right.
\socratesspeaks
We have now, then, my friend, pretty clearly shown to what class mind belongs \stephpag{31 a} and what power it possesses.
\protarchusspeaks
Certainly.
\socratesspeaks
And likewise the class of pleasure was made clear some time ago.
\protarchusspeaks
Yes, it was.
\socratesspeaks
Let us, then, remember concerning both of them that mind was akin to cause and belonged more or less to that class, and that pleasure was itself infinite and belonged to the class which, in and by itself, has not and never will have either beginning or middle or end. \stephpag{b}
\protarchusspeaks
We will remember that, of course.
\socratesspeaks
Our next task is to see in what and by means of what feeling each of them comes into being whenever they do come into being. We will take pleasure first and discuss these questions in relation to pleasure, as we examined its class first. But we cannot examine pleasure successfully apart from pain.
\protarchusspeaks
If that is our proper path, let us follow it.
\socratesspeaks
Do you agree with us about the origin of pleasure? \stephpag{c}
\protarchusspeaks
What do you think it is?
\socratesspeaks
I think pain and pleasure naturally originate in the combined class.
\protarchusspeaks
Please, my dear Socrates, remind us which of the aforesaid classes you mean by the combined class.
\socratesspeaks
I will do so, as well as I can, my brilliant friend.
\protarchusspeaks
Thank you.
\socratesspeaks
By combined class, then, let us understand that which we said was the third of the four.
\protarchusspeaks
The one you mentioned after the infinite and the finite, and in which you put health and also, I believe, harmony? \stephpag{d}
\socratesspeaks
You are quite right. Now please pay very close attention.
\protarchusspeaks
I will. Say on.
\socratesspeaks
I say, then, that when, in us living beings, harmony is broken up, a disruption of nature and a generation of pain also take place at the same moment.
\protarchusspeaks
What you say is very likely.
\socratesspeaks
But if harmony is recomposed and returns to its own nature, then I say that pleasure is generated, if I may speak in the fewest and briefest words about matters of the highest import. \stephpag{e}
\protarchusspeaks
I think you are right, Socrates; but let us try to be more explicit.
\socratesspeaks
It is easiest to understand common and obvious examples, is it not?
\protarchusspeaks
What examples?
\socratesspeaks
Is hunger a kind of breaking up and a pain?
\protarchusspeaks
Yes.
\socratesspeaks
And eating, which is a filling up again, is a pleasure?
\protarchusspeaks
Yes.
\socratesspeaks
Thirst again is a destruction and a pain, but the filling with moisture \stephpag{32 a} of that which was dried up is a pleasure. Then, too, the unnatural dissolution and disintegration we experience through heat are a pain, but the natural restoration and cooling are a pleasure.
\protarchusspeaks
Certainly.
\socratesspeaks
And the unnatural hardening of the moisture in an animal through cold is pain; but the natural course of the elements returning to their place and separating is a pleasure. See, in short, if you think it is a reasonable statement that whenever in the class of living beings, \stephpag{b} which, as I said before, arises out of the natural union of the infinite and the finite, that union is destroyed, the destruction is pain, and the passage and return of all things to their own nature is pleasure.
\protarchusspeaks
Let us accept that; for it seems to me to be true in its general lines.
\socratesspeaks
Then we may assume this as one kind of pain and pleasure arising severally under the conditions I have described?
\protarchusspeaks
Let that be assumed.
\socratesspeaks
Now assume within the soul itself the anticipation of these conditions, \stephpag{c} the sweet and cheering hope of pleasant things to come, the fearful and woful expectation of painful things to come.
\protarchusspeaks
Yes, indeed, this is another kind of pleasure and pain, which belongs to the soul itself, apart from the body, and arises through expectation.
\socratesspeaks
You are right. I think that in these two kinds, both of which are, in my opinion, pure, and not formed by mixture of pain and pleasure, the truth about pleasure will be made manifest, \stephpag{d} whether the entire class is to be desired or such desirability is rather to be attributed to some other class among those we have mentioned, whereas pleasure and pain, like heat, cold, and other such things, are sometimes desirable and sometimes undesirable, because they are not good themselves, though some of them sometimes admit on occasion the nature of the good.
\protarchusspeaks
You are quite right in saying that we must track our quarry on this trail.
\socratesspeaks
First, then, let us agree on this point: If it is true, \stephpag{e} as we said, that destruction is pain and restoration is pleasure, let us consider the case of living beings in which neither destruction nor restoration is going on, and what their state is under such conditions. Fix your mind on my question: Must not every living being under those conditions necessarily be devoid of any feeling of pain or pleasure, great or small?
\protarchusspeaks
Yes, necessarily.
\socratesspeaks
Have we, then, a third condition, \stephpag{33 a} besides those of feeling pleasure and pain?
\protarchusspeaks
Certainly.
\socratesspeaks
Well then, do your best to bear it in mind; for remembering or forgetting it will make a great difference in our judgement of pleasure. And I should like, if you do not object, to speak briefly about it.
\protarchusspeaks
Pray do so.
\socratesspeaks
You know that there is nothing to hinder a man from living the life of wisdom in this manner. \stephpag{b}
\protarchusspeaks
You mean without feeling pleasure or pain?
\socratesspeaks
Yes, for it was said, you know, in our comparison of the lives that he who chose the life of mind and wisdom was to have no feeling of pleasure, great or small.
\protarchusspeaks
Yes, surely, that was said.
\socratesspeaks
Such a man, then, would have such a life; and perhaps it is not unreasonable, if that is the most divine of lives.
\protarchusspeaks
Certainly it is not likely that gods feel either joy or its opposite.
\socratesspeaks
No, it is very unlikely; for either is unseemly for them. But let us reserve the discussion of that point \stephpag{c} for another time, if it is appropriate, and we will give mind credit for it in contending for the second place, if we cannot count it for the first.
\protarchusspeaks
Quite right.
\socratesspeaks
Now the other class of pleasure, which we said was an affair of the soul alone, originates entirely in memory.
\protarchusspeaks
How is that?
\socratesspeaks
We must, apparently, first take up memory, and perception even before memory, if these matters are to be made clear to us properly. \stephpag{d}
\protarchusspeaks
What do you mean?
\socratesspeaks
Assume that some of the affections of our body are extinguished in the body before they reach the soul, leaving the soul unaffected, and that other affections permeate both body and soul and cause a vibration in both conjointly and in each individually.
\protarchusspeaks
Let us assume that.
\socratesspeaks
Shall we be right in saying that the soul forgets those which do not permeate both, and does not forget those which do? \stephpag{e}
\protarchusspeaks
Yes, certainly.
\socratesspeaks
Do not in the least imagine that when I speak of forgetting I mean that forgetfulness arises in this case; for forgetfulness is the departure of memory, and in the case under consideration memory has not yet come into being; now it is absurd to speak of the loss of that which does not exist and has not yet come into being, is it not?
\protarchusspeaks
Certainly.
\socratesspeaks
Then just change the terms.
\protarchusspeaks
How?
\socratesspeaks
Instead of saying that the soul forgets, when it is unaffected by the vibrations of the body, \stephpag{34 a} apply the term want of perception to that which you are now calling forgetfulness.
\protarchusspeaks
I understand.
\socratesspeaks
And the union of soul and body in one common affection and one common motion you may properly call perception.
\protarchusspeaks
Very true.
\socratesspeaks
Then do we now understand what we mean by perception?
\protarchusspeaks
Certainly.
\socratesspeaks
I think, then, that memory may rightly be defined as the preservation of perception. \stephpag{b}
\protarchusspeaks
Quite rightly.
\socratesspeaks
But do we not say that memory differs from recollection?
\protarchusspeaks
Perhaps.
\socratesspeaks
And is this the difference?
\protarchusspeaks
What?
\socratesspeaks
When the soul alone by itself, apart from the body, recalls completely any experience it has had in company with the body, we say that it recollects do we not?
\protarchusspeaks
Certainly.
\socratesspeaks
And again when the soul has lost the memory of a perception or of something it has learned and then alone by itself regains this, \stephpag{c} we call everything of that kind recollection.
\protarchusspeaks
You are right.
\socratesspeaks
Now my reason for saying all this is---
\protarchusspeaks
What?
\socratesspeaks
That henceforth we may comprehend as completely and clearly as possible the pleasure of the soul, and likewise its desire, apart from the body; for both of these appear to be made plain by what has been said about memory and recollection.
\protarchusspeaks
Let us, then, Socrates, discuss the next point.
\socratesspeaks
We must, it seems, consider many things in relation to the origin and general aspect of pleasure; \stephpag{d} but now I think our first task is to take up the nature and origin of desire.
\protarchusspeaks
Then let us examine that; for we shall not lose anything.
\socratesspeaks
Oh yes, Protarchus, we shall lose a great deal! When we find what we are seeking we shall lose our perplexity about these very questions.
\protarchusspeaks
That is a fair counter; but let us try to take up the next point.
\socratesspeaks
Did we not say just now that hunger, thirst, \stephpag{e} and the like were desires?
\protarchusspeaks
They are, decidedly.
\socratesspeaks
What sort of identity have we in view when we call these, which are so different, by one name?
\protarchusspeaks
By Zeus, Socrates, that question may not be easy to answer, yet it must be answered.
\socratesspeaks
Let us, then, begin again at that point with the same examples.
\protarchusspeaks
At what point?
\socratesspeaks
We say of a thing on any particular occasion, ``it's thirsty," do we not?
\protarchusspeaks
Of course.
\socratesspeaks
And that means being empty?
\protarchusspeaks
Certainly.
\socratesspeaks
And is thirst, then, a desire?
\protarchusspeaks
Yes, of drink. \stephpag{35 a}
\socratesspeaks
Of drink, or of being filled with drink?
\protarchusspeaks
Of being filled, I suppose.
\socratesspeaks
The man, then, who is empty desires, as it appears, the opposite of what he feels for, being empty, he longs to be filled.
\protarchusspeaks
That is very plain.
\socratesspeaks
Well then, is there any source from which a man who is empty at first can gain a comprehension, whether by perception or by memory, of fulness, a thing which he does not feel at the time and has never felt before?
\protarchusspeaks
It cannot be done. \stephpag{b}
\socratesspeaks
And yet he who desires, desires something, we say.
\protarchusspeaks
Of course.
\socratesspeaks
And he does not desire that which he feels; for he is thirsty, and that is emptiness, but he desires fulness.
\protarchusspeaks
Yes.
\socratesspeaks
Then somehow some part of him who is thirsty can apprehend fulness.
\protarchusspeaks
Yes, obviously.
\socratesspeaks
But it cannot be the body, for that is empty.
\protarchusspeaks
True.
\socratesspeaks
The only remaining possibility is that the soul apprehends it, \stephpag{c} which it must do by means of memory; for what other means could it employ?
\protarchusspeaks
No other, I should say.
\socratesspeaks
And do we understand the consequences of this argument?
\protarchusspeaks
What are the consequences?
\socratesspeaks
This argument declares that we have no bodily desire.
\protarchusspeaks
How so?
\socratesspeaks
Because it shows that the endeavor of every living being is always towards the opposite of the actual conditions of the body.
\protarchusspeaks
Yes, certainly.
\socratesspeaks
And the impulse which leads towards the opposite of those conditions shows that there is a memory of the opposite of the conditions.
\protarchusspeaks
Certainly. \stephpag{d}
\socratesspeaks
And the argument, by showing that memory is that which leads us towards the objects of desire, has proved that all the impulse, the desire, and the ruling principle in every living being are of the soul.
\protarchusspeaks
Quite right.
\socratesspeaks
So the argument denies utterly that the body hungers or thirsts or has any such affection.
\protarchusspeaks
Very true.
\socratesspeaks
Let us consider a further point in connection with those very affections. For I think the purpose of the argument is to point out to us a state of life existing in them. \stephpag{e}
\protarchusspeaks
Of what sort of life are you speaking, and in what affections does it exist?
\socratesspeaks
In the affections of fulness and emptiness and all which pertain to the preservation and destruction of living beings, and I am thinking that if we fall into one of these we feel pain, which is followed by joy when we change to the other.
\protarchusspeaks
That is true.
\socratesspeaks
And what if a man is between the two?
\protarchusspeaks
How between them?
\socratesspeaks
Because of his condition, he is suffering, but he remembers the pleasures the coming of which would bring him an end of his pain; as yet, however, he does not possess them. Well then, shall we say that he is \stephpag{36 a} between the affections, or not?
\protarchusspeaks
Let us say so.
\socratesspeaks
Shall we say that he is wholly pained or wholly pleased?
\protarchusspeaks
No, by Zeus, but he is afflicted with a twofold pain; he suffers in body from his sensation, and in soul from expectation and longing.
\socratesspeaks
How could you, Protarchus, speak of twofold pain? Is not an empty man sometimes possessed \stephpag{b} of a sure hope of being filled, and sometimes, on the contrary, quite hopeless?
\protarchusspeaks
Certainly.
\socratesspeaks
And do you not think that when he has a hope of being filled he takes pleasure in his memory, and yet at the same time, since he is at the moment empty, suffers pain?
\protarchusspeaks
It cannot be otherwise.
\socratesspeaks
At such a time, then, a man, or any other animal, has both pain and pleasure at once.
\protarchusspeaks
Yes, I suppose so.
\socratesspeaks
And when an empty man is without hope of being filled, what then? Is not that the time when the twofold feeling of pain would arise, which you just now observed \stephpag{c} and thought the pain simply was twofold?
\protarchusspeaks
Very true, Socrates.
\socratesspeaks
Let us make use of our examination of those affections for a particular purpose.
\protarchusspeaks
For what purpose?
\socratesspeaks
Shall we say that those pleasures and pains are true or false, or that some are true and others not so?
\protarchusspeaks
But, Socrates, how can there be false pleasures or pains?
\socratesspeaks
But, Protarchus, how can there be true and false fears, or true and false expectations, or true and false opinions? \stephpag{d}
\protarchusspeaks
Opinions I would grant you, but not the rest.
\socratesspeaks
What? I am afraid we are starting a very considerable discussion.
\protarchusspeaks
You are right.
\socratesspeaks
And yet we must consider, thou son of that man,\footnote{``Son of that man'' may mean ``son of Philebus,'' in so far as Protarchus is a pupil of Philebus, or (so Bury) ``son of Gorgias,'' the orator and teacher (cf. Plato, \emph{Phaedo} 58b), or the father of Protarchus may be referred to by the pronoun, possibly because Socrates does not at the moment recall his name or because he wishes to imply that he was a man of mark.} whether the discussion is relevant to what has gone before.
\protarchusspeaks
Yes, no doubt.
\socratesspeaks
We must dismiss everything else, tedious or otherwise, that is irrelevant.
\protarchusspeaks
Right. \stephpag{e}
\socratesspeaks
Now tell me; for I am always utterly amazed by the same questions we were just proposing.
\protarchusspeaks
What do you mean?
\socratesspeaks
Are not some pleasures false and others true?
\protarchusspeaks
How could that be?
\socratesspeaks
Then, as you maintain, nobody, either sleeping or waking or insane or deranged, ever thinks he feels pleasure when he does not feel it, and never, on the other hand, thinks he suffers pain when he does not suffer it?
\protarchusspeaks
We have, Socrates, always believed that all this is as you suggest.
\socratesspeaks
But is the belief correct? Shall we consider whether it is so or not? \stephpag{37 a}
\protarchusspeaks
I should say we ought to consider that.
\socratesspeaks
Then let us analyze still more clearly what we were just now saying about pleasure and opinion. There is a faculty of having an opinion, is there not?
\protarchusspeaks
Yes.
\socratesspeaks
And of feeling pleasure?
\protarchusspeaks
Yes.
\socratesspeaks
And there is an object of opinion?
\protarchusspeaks
Of course.
\socratesspeaks
And something by which that which feels pleasure is pleased?
\protarchusspeaks
Certainly.
\socratesspeaks
And that which has opinion, whether right or wrong, never loses its function of really having opinion? \stephpag{b}
\protarchusspeaks
Of course not.
\socratesspeaks
And that which feels pleasure, whether rightly or wrongly, will clearly never lose its function of really feeling pleasure?
\protarchusspeaks
Yes, that is true, too.
\socratesspeaks
Then we must consider how it is that opinion is both true and false and pleasure only true, though the holding of opinion and the feeling of pleasure are equally real.
\protarchusspeaks
Yes, so we must.
\socratesspeaks
You mean that we must consider this question because falsehood and truth are added as attributes to opinion, \stephpag{c} and thereby it becomes not merely opinion, but opinion of a certain quality in each instance?
\protarchusspeaks
Yes.
\socratesspeaks
And furthermore, we must reach an agreement on the question whether, even if some things have qualities, pleasure and pain are not merely what they are, without qualities or attributes.
\protarchusspeaks
Evidently we must.
\socratesspeaks
But it is easy enough to see that they have qualities. For we said a long time ago that both pains and pleasures \stephpag{d} are great and small and intense.
\protarchusspeaks
Yes, certainly.
\socratesspeaks
And if badness becomes an attribute of any of these, Protarchus, shall we say that the opinion or the pleasure thereby becomes bad?
\protarchusspeaks
Why certainly, Socrates.
\socratesspeaks
And what if rightness or its opposite becomes an attribute of one of them? Shall we not say that the opinion is right, if it has rightness, and the pleasure likewise?
\protarchusspeaks
Obviously. \stephpag{e}
\socratesspeaks
And if that which is opined is mistaken, must we not agree that the opinion, since it is at the moment making a mistake, is not right or rightly opining?
\protarchusspeaks
Of course.
\socratesspeaks
And what if we see a pain or a pleasure making a mistake in respect of that by which the pain or pleasure is caused? Shall we give it the attribute of right or good or any of the words which denote excellence?
\protarchusspeaks
That is impossible if the pleasure is mistaken.
\socratesspeaks
And certainly pleasure often seems to come to us in connection with false, not true, opinion.
\protarchusspeaks
Of course it does; and in such a case, Socrates, \stephpag{38 a} we call the opinion false; but nobody would ever call the actual pleasure false.
\socratesspeaks
You are an eager advocate of the case of pleasure just now, Protarchus.
\protarchusspeaks
Oh no, I merely say what I hear.
\socratesspeaks
Is there no difference, my friend, between the pleasure which is connected with right opinion and knowledge and that which often comes to each of us with falsehood and ignorance? \stephpag{b}
\protarchusspeaks
There is likely to be a great difference.
\socratesspeaks
Then let us proceed to the contemplation of the difference between them.
\protarchusspeaks
Lead on as you think best.
\socratesspeaks
Then this is the way I lead.
\protarchusspeaks
What way?
\socratesspeaks
Do we agree that there is such a thing as false opinion and also as true opinion?
\protarchusspeaks
There is.
\socratesspeaks
And, as we were saying just now, pleasure and pain often follow them---I mean true and false opinion.
\protarchusspeaks
Certainly.
\socratesspeaks
And do not opinion and the power of forming an opinion always come to us \stephpag{c} from memory and perception?
\protarchusspeaks
Certainly.
\socratesspeaks
Do we, then, believe that our relation to these faculties is somewhat as follows?
\protarchusspeaks
How?
\socratesspeaks
Would you say that often when a man sees things at a distance and not very clearly, he wishes to distinguish between the things which he sees?
\protarchusspeaks
Yes, I should say so.
\socratesspeaks
Next, then, would he not ask himself---
\protarchusspeaks
What?
\socratesspeaks
``What is that which is visible standing \stephpag{d} beside the rock under a tree?'' Do you not think a man might ask himself such a question if he saw such objects presented to his view?
\protarchusspeaks
To be sure.
\socratesspeaks
And after that our gazer might reply to himself correctly ``It is a man"?
\protarchusspeaks
Certainly.
\socratesspeaks
Or, again, perhaps he might be misled into the belief that it was a work of some shepherds, and then he would call the thing which he saw an image.
\protarchusspeaks
Yes, indeed. \stephpag{e}
\socratesspeaks
And if some one is with him, he might repeat aloud to his companion what he had said to himself, and thus that which we called an opinion now becomes a statement?
\protarchusspeaks
Certainly.
\socratesspeaks
But if he is alone when he has this thought, he sometimes carries it about in his mind for a long time.
\protarchusspeaks
Undoubtedly.
\socratesspeaks
Well, is your view about what takes place in such cases the same as mine?
\protarchusspeaks
What is yours?
\socratesspeaks
I think the soul at such a time is like a book.
\protarchusspeaks
How is that? \stephpag{39 a}
\socratesspeaks
Memory unites with the senses, and they and the feelings which are connected with them seem to me almost to write words in our souls; and when the feeling in question writes the truth, true opinions and true statements are produced in us; but when the writer within us writes falsehoods, the resulting opinions and statements are the opposite of true. \stephpag{b}
\protarchusspeaks
That is my view completely, and I accept it as stated.
\socratesspeaks
Then accept also the presence of another workman in our souls at such a time.
\protarchusspeaks
What workman?
\socratesspeaks
A painter, who paints in our souls pictures to illustrate the words which the writer has written.
\protarchusspeaks
But how do we say he does this, and when?
\socratesspeaks
When a man receives from sight or some other sense the opinions and utterances of the moment and afterwards beholds in his own mind the images of those opinions and utterances. \stephpag{c} That happens to us often enough, does it not?
\protarchusspeaks
It certainly does.
\socratesspeaks
And the images of the true opinions are true, and those of the false are false?
\protarchusspeaks
Assuredly.
\socratesspeaks
Then if we are right about that, let us consider a further question.
\protarchusspeaks
What is it?
\socratesspeaks
Whether this is an inevitable experience in relation to the present and the past, but not in relation to the future.
\protarchusspeaks
It is in the same relation to all kinds of time. \stephpag{d}
\socratesspeaks
Was it not said a while ago that the pleasures and pains which belong to the soul alone might come before the pleasures and pains of the body, so that we have the pleasure and pain of anticipation, which relate to the future?
\protarchusspeaks
Very true.
\socratesspeaks
Do the writings and pictures, then, which we imagined a little while ago to exist within us, relate to the past and present, \stephpag{e} but not to the future?
\protarchusspeaks
To the future especially.
\socratesspeaks
Do you say ``to the future especially" because they are all hopes relating to the future and we are always filled with hopes all our lives?
\protarchusspeaks
Precisely.
\socratesspeaks
Well, here is a further question for you to answer.
\protarchusspeaks
What is it?
\socratesspeaks
A just, pious, and good man is surely a friend of the gods, is he not?
\protarchusspeaks
Certainly.
\socratesspeaks
And an unjust and thoroughly bad man \stephpag{40 a} is the reverse?
\protarchusspeaks
Of course.
\socratesspeaks
But, as we were just now saying, every man is full of many hopes?
\protarchusspeaks
Yes, to be sure.
\socratesspeaks
And there are in all of us written words which we call hopes?
\protarchusspeaks
Yes.
\socratesspeaks
And also the images painted there; and often a man sees an abundance of gold coming into his possession, and in its train many pleasures; and he even sees a picture of himself enjoying himself immensely. \stephpag{b}
\protarchusspeaks
Yes, certainly.
\socratesspeaks
Shall we or shall we not say that of these pictures those are for the most part true which are presented to the good, because they are friends of the gods, whereas those presented to the bad are for the most part false?
\protarchusspeaks
Surely we must say that.
\socratesspeaks
Then the bad also, no less than the good, have pleasures painted in their souls, but they are false pleasures.
\protarchusspeaks
Yes, surely. \stephpag{c}
\socratesspeaks
Then the bad rejoice for the most part in the false, and the good in true pleasures.
\protarchusspeaks
That is inevitably true.
\socratesspeaks
According to our present view, then, there are false pleasures in the souls of men, imitations or caricatures of the true pleasures; and pains likewise.
\protarchusspeaks
There are.
\socratesspeaks
We saw, you remember, that he who had an opinion at all always really had an opinion, but it was sometimes not based upon realities, whether present, past, or future.
\protarchusspeaks
Certainly. \stephpag{d}
\socratesspeaks
And this it was, I believe, which created false opinion and the holding of false opinions, was it not?
\protarchusspeaks
Yes.
\socratesspeaks
Very well, must we not also grant that pleasure and pain stand in the same relation to realities?
\protarchusspeaks
What do you mean?
\socratesspeaks
I mean that he who feels pleasure at all in any way or manner always really feels pleasure, but it is sometimes not based upon realities, whether present or past, and often, perhaps most frequently, upon things which will never even be realities in the future. \stephpag{e}
\protarchusspeaks
This also, Socrates, must inevitably be the case.
\socratesspeaks
And the same may be said of fear and anger and all that sort of thing---that they are all sometimes false?
\protarchusspeaks
Certainly.
\socratesspeaks
Well, can we say that opinions become bad or good except as they become false?
\protarchusspeaks
No.
\socratesspeaks
And we understand, I believe, that pleasures also \stephpag{41 a} are not bad except by being false.
\protarchusspeaks
No; you have said quite the reverse of the truth, Socrates; for no one would be at all likely to call pains and pleasures bad because they are false, but because they are involved in another great and manifold evil.
\socratesspeaks
Then of the evil pleasures which are such because of evil we will speak a little later, if we still care to do so; but of the false pleasures we must prove in another way that they exist and come into existence in us often and in great numbers; \stephpag{b} for this may help us to reach our decisions.
\protarchusspeaks
Yes, of course; that is, if such pleasures exist.
\socratesspeaks
But they do exist, Protarchus, in my opinion; however, until we have established the truth of this opinion, it cannot be unquestioned.
\protarchusspeaks
Good.
\socratesspeaks
Then let us, like athletes, approach and grapple with this new argument.
\protarchusspeaks
Let us do so.
\socratesspeaks
We said, you may remember, a little while ago, \stephpag{c} that when desires, as they are called, exist in us, the soul is apart from the body and separate from it in feelings.
\protarchusspeaks
I remember; that was said.
\socratesspeaks
And was not the soul that which desired the opposites of the conditions of the body and the body that which caused pleasure or pain because of feeling?
\protarchusspeaks
Yes, that was the case.
\socratesspeaks
Then draw the conclusion as to what takes place in these circumstances.
\protarchusspeaks
Go on. \stephpag{d}
\socratesspeaks
What takes place is this: in these circumstances pleasures and pains exist at the same time and the sensations of opposite pleasures and pains are present side by side simultaneously, as was made clear just now.
\protarchusspeaks
Yes, that is clear.
\socratesspeaks
And have we not also said and agreed and settled something further?
\protarchusspeaks
What?
\socratesspeaks
That both pleasure and pain admit of the more and less and are of the class of the infinite.
\protarchusspeaks
Yes, we have said that, certainly.
\socratesspeaks
Then what means is there of judging rightly of this? \stephpag{e}
\protarchusspeaks
How and in what way do you mean?
\socratesspeaks
I mean to ask whether the purpose of our judgement of these matters in such circumstances is to recognize in each instance which of these elements is greater or smaller or more intense, comparing pain with pleasure, pain with pain, and pleasure with pleasure.
\protarchusspeaks
Certainly there are such differences, and that is the purpose of our judgement.
\socratesspeaks
Well then, in the case of sight, seeing things from too near at hand or from too great a distance \stephpag{42 a} obscures their real sizes and causes us to have false opinions; and does not this same thing happen in the case of pains and pleasures?
\protarchusspeaks
Yes, Socrates, even much more than in the case of sight.
\socratesspeaks
Then our present conclusion is the opposite of what we said a little while ago.
\protarchusspeaks
To what do you refer?
\socratesspeaks
A while ago these opinions, being false or true, imbued the pains and pleasures with their own condition of truth or falsehood. \stephpag{b}
\protarchusspeaks
Very true.
\socratesspeaks
But now, because they are seen at various and changing distances and are compared with one another, the pleasures themselves appear greater and more intense by comparison with the pains, and the pains in turn, through comparison with the pleasures, vary inversely as they.
\protarchusspeaks
That is inevitable for the reasons you have given.
\socratesspeaks
They both, then, appear greater and less than the reality. Now if you abstract from both of them this apparent, but unreal, excess or inferiority, you cannot say that its appearance is true, \stephpag{c} nor again can you have the face to affirm that the part of pleasure or pain which corresponds to this is true or real.
\protarchusspeaks
No, I cannot.
\socratesspeaks
Next, then, we will see whether we may not in another direction come upon pleasures and pains still more false than these appearing and existing in living beings.
\protarchusspeaks
What pleasures and what method do you mean?
\socratesspeaks
It has been said many times that pains and woes and aches and everything that is called by names of that sort are caused when nature in any instance is corrupted through combinations and dissolutions, \stephpag{d} fillings and emptyings, increases and diminutions.
\protarchusspeaks
Yes, that has been said many times.
\socratesspeaks
And we agreed that when things are restored to their natural condition, that restoration is pleasure.
\protarchusspeaks
Right.
\socratesspeaks
But when neither of these changes takes place in the body, what then?
\protarchusspeaks
When could that be the case, Socrates? \stephpag{e}
\socratesspeaks
That question of yours is not to the point, Protarchus.
\protarchusspeaks
Why not?
\socratesspeaks
Because you do not prevent my asking my own question again.
\protarchusspeaks
What question?
\socratesspeaks
Why, Protarchus, I may say, granting that such a condition does not arise, what would be the necessary result if it did?
\protarchusspeaks
You mean if the body is not changed in either direction?
\socratesspeaks
Yes.
\protarchusspeaks
It is clear, Socrates, that in that case there would never be either pleasure or pain. \stephpag{43 a}
\socratesspeaks
Excellent. But you believe, I fancy, that some such change must always be taking place in us, as the philosophers\footnote{Heracleitus and his followers.} say; for all things are always flowing and shifting.
\protarchusspeaks
Yes, that is what they say, and I think their theory is important.
\socratesspeaks
Of course it is, in view of their own importance. But I should like to avoid this argument which is rushing at us. I am going to run away; come along and escape with me.
\protarchusspeaks
What is your way of escape?
\socratesspeaks
``We grant you all this" let us say to them. \stephpag{b} But answer me this, Protarchus, are we and all other living beings always conscious of everything that happens to us of our growth and all that sort of thing---or is the truth quite the reverse of that?
\protarchusspeaks
Quite the reverse, surely; for we are almost entirely unconscious of everything of that sort.
\socratesspeaks
Then we were not right in saying just now that the fluctuations and changes cause pains and pleasures.
\protarchusspeaks
No, certainly not. \stephpag{c}
\socratesspeaks
A better and more unassailable statement would be this.
\protarchusspeaks
What?
\socratesspeaks
That the great changes cause pains and pleasures in us, but the moderate and small ones cause no pains or pleasures at all.
\protarchusspeaks
That is more correct than the other statement, Socrates.
\socratesspeaks
But if that is the case, the life of which we spoke just now would come back again.
\protarchusspeaks
What life?
\socratesspeaks
The life which we said was painless and without joys.
\protarchusspeaks
Very true.
\socratesspeaks
Let us, therefore, assume three lives, \stephpag{d} one pleasant, one painful, and one neither of the two; or do you disagree?
\protarchusspeaks
No, I agree to this, that there are the three lives.
\socratesspeaks
Then freedom from pain would not be identical with pleasure?
\protarchusspeaks
Certainly not.
\socratesspeaks
When you hear anyone say that the pleasantest of all things is to live all oneÕs life without pain, what do you understand him to mean?
\protarchusspeaks
I think he means that freedom from pain is pleasure.
\socratesspeaks
Now let us assume that we have three things; no matter what they are, \stephpag{e} but let us use fine names and call one gold, another silver, and the third neither of the two.
\protarchusspeaks
Agreed.
\socratesspeaks
Now can that which is neither become either gold or silver?
\protarchusspeaks
Certainly not.
\socratesspeaks
Neither can that middle life of which we spoke ever be rightly considered in opinion or called in speech pleasant or painful, at any rate by those who reason correctly.
\protarchusspeaks
No, certainly not.
\socratesspeaks
But surely, my friend, we are aware of persons who call it \stephpag{44 a} and consider it so.
\protarchusspeaks
Certainly.
\socratesspeaks
Do they, then, think they feel pleasure whenever they are not in pain?
\protarchusspeaks
That is what they say.
\socratesspeaks
Then they do think they feel pleasure at such times; for otherwise they would not say so.
\protarchusspeaks
Most likely.
\socratesspeaks
Certainly, then, they have a false opinion about pleasure, if there is an essential difference between feeling pleasure and not feeling pain.
\protarchusspeaks
And we certainly found that difference.
\socratesspeaks
Then shall we adopt the view that there are, \stephpag{b} as we said just now, three states, or that there are only two---pain, which is an evil to mankind, and freedom from pain, which is of itself a good and is called pleasure?
\protarchusspeaks
Why do we ask ourselves that question now, Socrates? I do not understand.
\socratesspeaks
No, Protarchus, for you certainly do not understand about the enemies of our friend Philebus.
\protarchusspeaks
Whom do you mean?
\socratesspeaks
Certain men who are said to be master thinkers about nature, and who deny the existence of pleasures altogether.
\protarchusspeaks
Is it possible? \stephpag{c}
\socratesspeaks
They say that what Philebus and his school call pleasures are all merely refuges from pain.
\protarchusspeaks
Do you recommend that we adopt their view, Socrates?
\socratesspeaks
No, but that we make use of them as seers who divine the truth, not by acquired skill, but by some innate and not ignoble repugnance which makes them hate the power of pleasure and think it so utterly unsound that its very attractiveness is mere trickery, not pleasure. \stephpag{d} You may make use of them in this way, considering also their other expressions of dislike; and after that you shall learn of the pleasures which seem to me to be true, in order that we may consider the power of pleasure from both points of view and form our judgement by comparing them.
\protarchusspeaks
You are right.
\socratesspeaks
Let us, then, consider these men as allies and follow them in the track of their dislike. I fancy their method would be to begin somewhere further back \stephpag{e} and ask whether, if we wished to discover the nature of any class---take the hard, for instance---we should be more likely to learn it by looking at the hardest things or at the least hard. Now you, Protarchus, must reply to them as you have been replying to me.
\protarchusspeaks
By all means, and I say to them that we should look at the greatest things.
\socratesspeaks
Then if we wished to discover what the nature of pleasure is, we should look, not at the smallest pleasures, \stephpag{45 a} but at those which are considered most extreme and intense.
\protarchusspeaks
Every one would agree to that now.
\socratesspeaks
And the commonest and greatest pleasures are, as we have often said, those connected with the body, are they not?
\protarchusspeaks
Certainly.
\socratesspeaks
Are they greater, then, and do they become greater in those who are ill or in those who are in health? Let us take care not to answer hastily and fall into error. Perhaps we might say they are greater \stephpag{b} in those who are in health.
\protarchusspeaks
That is reasonable.
\socratesspeaks
Yes, but are not those pleasures the greatest which gratify the greatest desires?
\protarchusspeaks
That is true.
\socratesspeaks
But do not people who are in a fever, or in similar diseases, feel more intensely thirst and cold and other bodily sufferings which they usually have; and do they not feel greater want, followed by greater pleasure when their want is satisfied? Is this true, or not? \stephpag{c}
\protarchusspeaks
Now that you have said it, it certainly appears to be true.
\socratesspeaks
Then should we appear to be right in saying that if we wished to discover the greatest pleasures we should have to look, not at health, but at disease? Now do not imagine that I mean to ask you whether those who are very ill have more pleasures than those who are well, but assume that I am asking about the greatness of pleasure, and where the greatest intensity of such feeling normally occurs. For we say that it is our task to discover the nature of pleasure and what \stephpag{d} those who deny its existence altogether say that it is.\footnote{This paradox means ``what those say it is who deny that it is really pleasure.''}
\protarchusspeaks
I think I understand you.
\socratesspeaks
Presently, Protarchus, you will show that more clearly, for I want you to answer a question. Do you see greater pleasures---I do not mean greater in number, but greater in intensity and degree---in riotous living or in a life of self-restraint? Be careful about your reply.
\protarchusspeaks
I understand you, and I see that there is a great difference. For the self-restrained are always held in check by the advice of the proverbial expression \stephpag{e} ``nothing too much," which guides their actions; but intense pleasure holds sway over the foolish and dissolute even to the point of madness and makes them notorious.
\socratesspeaks
Good; and if that is true, it is clear that the greatest pleasures and the greatest pains originate in some depravity of soul and body, not in virtue.
\protarchusspeaks
Certainly.
\socratesspeaks
Then we must select some of these pleasures and see what there is about them which made us say that they are the greatest. \stephpag{46 a}
\protarchusspeaks
Yes, we must.
\socratesspeaks
Now see what there is about the pleasures which are related to certain diseases.
\protarchusspeaks
What diseases?
\socratesspeaks
Repulsive diseases which the philosophers of dislike whom we mentioned utterly abominate.
\protarchusspeaks
What are the pleasures?
\socratesspeaks
For instance, the relief of the itch and the like by scratching, no other treatment being required. For in HeavenÕs name what shall we say the feeling is which we have in this case? Is it pleasure or pain?
\protarchusspeaks
I think, Socrates, it is a mixed evil. \stephpag{b}
\socratesspeaks
I did not introduce this question on PhilebusÕ account; but unless we consider these pleasures and those that follow in their train, Protarchus, we can probably never settle the point at issue.
\protarchusspeaks
Then we must attack this family of pleasures.
\socratesspeaks
You mean those which are mixed?
\protarchusspeaks
Certainly.
\socratesspeaks
Some mixtures are concerned with the body and are in the body only, and some belong only to the soul and are in the soul; \stephpag{c} and we shall also find some mingled pains and pleasures belonging both to the soul and to the body, and these are sometimes called pleasures, sometimes pains.
\protarchusspeaks
How so?
\socratesspeaks
Whenever, in the process of restoration or destruction, anyone has two opposite feelings, as we sometimes are cold, but are growing warm, or are hot, but are growing cold, the desire of having the one and being free from the other, the mixture of bitter and sweet, as they say, joined with the difficulty in getting rid of the bitter, \stephpag{d} produces impatience and, later, wild excitement.
\protarchusspeaks
What you say is perfectly true.
\socratesspeaks
And such mixtures sometimes consist of equal pains and pleasures and sometimes contain more of one or the other, do they not?
\protarchusspeaks
Of course.
\socratesspeaks
In the case of the mixtures in which the pains are more than the pleasures---say the itch, which we mentioned just now, or tickling---when the burning inflammation is within and is not reached by the rubbing and scratching, \stephpag{e} which separate only such mixtures as are on the surface, sometimes by bringing the affected parts to the fire or to something cold we change from wretchedness to inexpressible pleasures, and sometimes the opposition between the internal and the external produces a mixture of pains and pleasures, whichever happens to preponderate; this is the result of the forcible separation of combined elements, \stephpag{47 a} or the combination of those that were separate, and the concomitant juxtaposition of pains and pleasures.
\protarchusspeaks
Very true.
\socratesspeaks
And when the pleasure is the predominant element in the mixture, the slight tincture of pain tickles a man and makes him mildly impatient, or again an excessive proportion of pleasure excites him and sometimes even makes him leap for joy; it produces in him all sorts of colors, attitudes, and paintings, and even causes great amazement and foolish shouting, does it not? \stephpag{b}
\protarchusspeaks
Certainly.
\socratesspeaks
And it makes him say of himself, and others say of him, that he is pleased to death with these delights, and the more unrestrained and foolish he is, the more he always gives himself up to the pursuit of these pleasures; he calls them the greatest of all things and counts that man the happiest who lives most entirely in the enjoyment of them.
\protarchusspeaks
Socrates, you have described admirably what happens \stephpag{c} in the case of most people.
\socratesspeaks
That may be, Protarchus, so far as concerns purely bodily pleasures in which internal and external sensations unite; but concerning the pleasures in which the soul and the body contribute opposite elements, each adding pain or pleasure to the otherÕs pleasure or pain, so that both unite in a single mixture---concerning these I said before that when a man is empty he desires to be filled, and rejoices in his expectation, but is pained by his emptiness, and now I add, what I did not say at that time, that in all these cases, which are innumerable, \stephpag{d} of opposition between soul and body, there is one single mixture of pain and pleasure.
\protarchusspeaks
I believe you are quite right.
\socratesspeaks
One further mixture of pain and pleasure is left.
\protarchusspeaks
What is it?
\socratesspeaks
That mixture of its own feelings which we said the soul often experiences.
\protarchusspeaks
And what do we call this \stephpag{e}
\socratesspeaks
Do you not regard anger, fear, yearning, mourning, love, jealousy, envy, and the like as pains of the soul and the soul only?
\protarchusspeaks
I do.
\socratesspeaks
And shall we not find them full of ineffable pleasures? Or must I remind you of the anger?
``Which stirs a man, though very wise, to wrath,
And sweeter is than honey from the comb,
"
unknown
\stephpag{48 a} and of the pleasures mixed with pains, which we find in mournings and longings?
\protarchusspeaks
No, you need not remind me; those things occur just as you suggest.
\socratesspeaks
And you remember, too, how people enjoy weeping at tragedies?
\protarchusspeaks
Yes, certainly.
\socratesspeaks
And are you aware of the condition of the soul at comedies, how there also we have a mixture of pain and pleasure?
\protarchusspeaks
I do not quite understand. \stephpag{b}
\socratesspeaks
Indeed it is by no means easy, Protarchus, to understand such a condition under those circumstances.
\protarchusspeaks
No at least I do not find it so.
\socratesspeaks
Well, then, let us take this under consideration, all the more because of its obscurity; then we can more readily understand the mixture of pain and pleasure in other cases.
\protarchusspeaks
Please go on.
\socratesspeaks
Would you say that envy, which was mentioned just now, was a pain of the soul, or not?
\protarchusspeaks
I say it is.
\socratesspeaks
But certainly we see the envious man rejoicing in the misfortunes of his neighbors. \stephpag{c}
\protarchusspeaks
Yes, very much so.
\socratesspeaks
Surely ignorance is an evil, as is also what we call stupidity.
\protarchusspeaks
Surely.
\socratesspeaks
Next, then, consider the nature of the ridiculous.
\protarchusspeaks
Please proceed.
\socratesspeaks
The ridiculous is in its main aspect a kind of vice which gives its name to a condition; and it is that part of vice in general which involves the opposite of the condition mentioned in the inscription at Delphi.
\protarchusspeaks
You mean ``Know thyself," Socrates? \stephpag{d}
\socratesspeaks
Yes; and the opposite of that, in the language of the inscription, would evidently be not to know oneself at all.
\protarchusspeaks
Of course.
\socratesspeaks
Protarchus, try to divide this into three.
\protarchusspeaks
How do you mean? I am afraid I can never do it.
\socratesspeaks
Then you say that I must now make the division?
\protarchusspeaks
Yes, I say so, and I beg you to do so, besides.
\socratesspeaks
Must not all those who do not know themselves be affected by their condition in one of three ways?
\protarchusspeaks
How is that?
\socratesspeaks
First in regard to wealth; such a man thinks he is \stephpag{e} richer than he is.
\protarchusspeaks
Certainly a good many are affected in that way.
\socratesspeaks
And there are still more who think they are taller and handsomer than they are and that they possess better physical qualities in general than is the case.
\protarchusspeaks
Certainly. \stephpag{49 a}
\socratesspeaks
But by far the greatest number, I fancy, err in the third way, about the qualities of, the soul, thinking that they excel in virtue when they do not.
\protarchusspeaks
Yes, most decidedly.
\socratesspeaks
And of all the virtues, is not wisdom the one to which people in general lay claim, thereby filling themselves with strife and false conceit of wisdom?
\protarchusspeaks
Yes, to be sure.
\socratesspeaks
And we should surely be right in calling all that an evil condition.
\protarchusspeaks
Very much so.
\socratesspeaks
Then this must further be divided into two parts, if we are to gain insight into childish envy with its absurd mixture of pleasure and pain. ``How shall we divide it," do you say? All who have this false and foolish conceit \stephpag{b} of themselves fall, like the rest of mankind, into two classes: some necessarily have strength and power, others, as I believe, the reverse.
\protarchusspeaks
Yes, necessarily.
\socratesspeaks
Make the division, then, on that principle; those of them who have this false conceit and are weak and unable to revenge themselves when they are laughed at you may truly call ridiculous, but those who are strong and able to revenge themselves you will define most correctly to yourself \stephpag{c} by calling them powerful, terrible, and hateful, for ignorance in the powerful is hateful and infamous---since whether real or feigned it injures their neighbors---but ignorance in the weak appears to us as naturally ridiculous.
\protarchusspeaks
Quite right. But the mixture of pleasure and pain in all this is not yet clear to me.
\socratesspeaks
First, then, take up the nature of envy.
\protarchusspeaks
Go on. \stephpag{d}
\socratesspeaks
Is envy a kind of unrighteous pain and also a pleasure?
\protarchusspeaks
Undoubtedly.
\socratesspeaks
But it is neither wrong nor envious to rejoice in the misfortunes of our enemies, is it?
\protarchusspeaks
No, of course not.
\socratesspeaks
But when people sometimes see the misfortunes of their friends and rejoice instead of grieving, is not that wrong?
\protarchusspeaks
Of course it is.
\socratesspeaks
And we said that ignorance was an evil to every one, did we not?
\protarchusspeaks
True.
\socratesspeaks
Then the false conceits of our friends concerning their wisdom, their beauty, \stephpag{e} and their other qualities which we mentioned just now, saying that they belong to three classes, are ridiculous when they are weak, but hateful when they are powerful. Shall we, or shall we not, affirm that, as I said just now, this state of mind when possessed in its harmless form by any of our friends, is ridiculous in the eyes of others?
\protarchusspeaks
Certainly it is ridiculous.
\socratesspeaks
And do we not agree that ignorance is in itself a misfortune?
\protarchusspeaks
Yes, a great one.
\socratesspeaks
And do we feel pleasure or pain when we laugh at it? \stephpag{50 a}
\protarchusspeaks
Pleasure, evidently.
\socratesspeaks
Did we not say that pleasure in the misfortunes of friends was caused by envy?
\protarchusspeaks
There can be no other cause.
\socratesspeaks
Then our argument declares that when we laugh at the ridiculous qualities of our friends, we mix pleasure with pain, since we mix it with envy; for we have agreed all along that envy is a pain of the soul, and that laughter is a pleasure, yet these two are present at the same time on such occasions.
\protarchusspeaks
True. \stephpag{b}
\socratesspeaks
So now our argument shows that in mournings and tragedies and comedies, not merely on the stage, but in all the tragedy and comedy of life, and in countless other ways, pain is mixed with pleasure.
\protarchusspeaks
It is impossible not to agree with that, Socrates, even though one be most eager to maintain the opposite opinion.
\socratesspeaks
Again we mentioned anger, yearning, mourning, love, jealousy, envy, and the like, \stephpag{c} as conditions in which we should find a mixture of the two elements we have now often named, did we not?
\protarchusspeaks
Yes.
\socratesspeaks
And we understand that all the details I have been describing just now are concerned only with sorrow and envy and anger?
\protarchusspeaks
Of course we understand that.
\socratesspeaks
Then there are still many others of those conditions left for us to discuss.
\protarchusspeaks
Yes, very many.
\socratesspeaks
Now why do you particularly suppose I pointed out to you the mixture of pain and pleasure in comedy? Was it not for the sake of convincing you, \stephpag{d} because it is easy to show the mixture in love and fear and the rest, and because I thought that when you had made this example your own, you would relieve me from the necessity of discussing those other conditions in detail, and would simply accept the fact that in the affections of the body apart from the soul, of the soul apart from the body, and of the two in common, there are plentiful mixtures of pain and pleasure? So tell me; will you let me off, or will you keep on till midnight? But I think I need say only a few words to induce you to let me off. I will agree to give you an account of all these matters \stephpag{e} tomorrow, but now I wish to steer my bark towards the remaining points that are needful for the judgement which Philebus demands.
\protarchusspeaks
Good, Socrates; just finish what remains in any way you please.
\socratesspeaks
Then after the mixed pleasures we should naturally and almost of necessity proceed in turn to the unmixed. \stephpag{51 a}
\protarchusspeaks
Very good.
\socratesspeaks
So I will turn to them and try to explain them; for I do not in the least agree with those who say that all pleasures are merely surcease from pain, but, as I said, I use them as witnesses to prove that some pleasures are apparent, but not in any way real, and that there are others which appear to be both great and numerous, but are really mixed up with pains and with cessations of the greatest pains and distresses of body and soul. \stephpag{b}
\protarchusspeaks
But what pleasures, Socrates, may rightly be considered true?
\socratesspeaks
Those arising from what are called beautiful colors, or from forms, most of those that arise from odors and sounds, in short all those the want of which is unfelt and painless, whereas the satisfaction furnished by them is felt by the senses, pleasant, and unmixed with pain.
\protarchusspeaks
Once more, Socrates, what do you mean by this?
\socratesspeaks
My meaning is certainly not clear at the first glance, \stephpag{c} and I must try to make it so. For when I say beauty of form, I am trying to express, not what most people would understand by the words, such as the beauty of animals or of paintings, but I mean, says the argument, the straight line and the circle and the plane and solid figures formed from these by turning-lathes and rulers and patterns of angles; perhaps you understand. For I assert that the beauty of these is not relative, like that of other things, but they are always absolutely beautiful by nature \stephpag{d} and have peculiar pleasures in no way subject to comparison with the pleasures of scratching; and there are colors which possess beauty and pleasures of this character. Do you understand?
\protarchusspeaks
I am trying to do so, Socrates; and I hope you also will try to make your meaning still clearer.
\socratesspeaks
I mean that those sounds which are smooth and clear and send forth a single pure note are beautiful, not relatively, but absolutely, and that there are pleasures which pertain to these by nature and result from them.
\protarchusspeaks
Yes, that also is true. \stephpag{e}
\socratesspeaks
The pleasures of smell are a less divine class; but they have no necessary pains mixed with them, and wherever and in whatever we find this freedom from pain, I regard it always as a mark of similarity to those other pleasures. These, then, are two classes of the pleasures of which I am speaking. Do you understand me?
\protarchusspeaks
I understand. \stephpag{52 a}
\socratesspeaks
And further let us add to these the pleasures of knowledge, if they appear to us not to have hunger for knowledge or pangs of such hunger as their source.
\protarchusspeaks
I agree to that.
\socratesspeaks
Well, if men are full of knowledge and then lose it through forgetfulness, do you see any pains in the losses?
\protarchusspeaks
Not by their inherent nature, but sometimes there is pain in reflecting on the event, \stephpag{b} when a man who has lost knowledge is pained by the lack of it.
\socratesspeaks
True, my dear fellow, but just at present we are recounting natural feelings only, not reflection.
\protarchusspeaks
Then you are right in saying that we feel no pain in the loss of knowledge.
\socratesspeaks
Then we may say that these pleasures of knowledge are unmixed with pain and are felt not by the many but only by very few.
\protarchusspeaks
Yes, certainly. \stephpag{c}
\socratesspeaks
And now that we have fairly well separated the pure pleasures and those which may be pretty correctly called impure, let us add the further statement that the intense pleasures are without measure and those of the opposite sort have measure; those which admit of greatness and intensity and are often or seldom great or intense we shall assign to the class of the infinite, which circulates more or less freely through the body and soul alike, \stephpag{d} and the others we shall assign to the class of the limited.
\protarchusspeaks
Quite right, Socrates.
\socratesspeaks
There is still another question about them to be considered.
\protarchusspeaks
What is it?
\socratesspeaks
What kind of thing is most closely related to truth? The pure and unadulterated, or the violent, the widespread, the great, and the sufficient?
\protarchusspeaks
What is your object, Socrates, in asking that question?
\socratesspeaks
My object, Protarchus, is to leave no gap in my test of pleasure \stephpag{e} and knowledge, if some part of each of them is pure and some part impure, in order that each of them may offer itself for judgement in a condition of purity, and thus make the judgement easier for you and me and all our audience.
\protarchusspeaks
Quite right.
\socratesspeaks
Very well, let us adopt that point of view towards all the classes which we call pure. First let us select one of them and examine it. \stephpag{53 a}
\protarchusspeaks
Which shall we select?
\socratesspeaks
Let us first, if agreeable to you, consider whiteness.
\protarchusspeaks
By all means.
\socratesspeaks
How can we have purity in whiteness, and what purity? Is it the greatest and most widespread, or the most unmixed, that in which there is no trace of any other color?
\protarchusspeaks
Clearly it is the most unadulterated.
\socratesspeaks
Right. Shall we not, then, Protarchus, declare that this, and not the most numerous or the greatest, \stephpag{b} is both the truest and the most beautiful of all whitenesses?
\protarchusspeaks
Quite right.
\socratesspeaks
Then we shall be perfectly right in saying that a little pure white is whiter and more beautiful and truer than a great deal of mixed white.
\protarchusspeaks
Perfectly right.
\socratesspeaks
Well then, we shall have no need of many such examples in our discussion of pleasure; we see well enough from this one that any pleasure, \stephpag{c} however small or infrequent, if uncontaminated with pain, is pleasanter and more beautiful than a great or often repeated pleasure without purity.
\protarchusspeaks
Most certainly; and the example is sufficient.
\socratesspeaks
Here is another point. Have we not often heard it said of pleasure that it is always a process or generation and that there is no state or existence of pleasure? There are some clever people who try to prove this theory to us, and we ought to be grateful to them.
\protarchusspeaks
Well, what then?
\socratesspeaks
I will explain this whole matter, Protarchus, \stephpag{d} by asking questions.
\protarchusspeaks
Go on; ask your questions.
\socratesspeaks
There are two parts of existence, the one self-existent, the other always desiring something else.
\protarchusspeaks
What do you mean? What are these two?
\socratesspeaks
The one is by nature more imposing, the other inferior.
\protarchusspeaks
Speak still more plainly.
\socratesspeaks
We have seen beloved boys who are fair and good, and brave lovers of them.
\protarchusspeaks
Yes, no doubt of it.
\socratesspeaks
Try to find another pair like these \stephpag{e} in all the relations we are speaking of.
\protarchusspeaks
Must I say it a third time? Please tell your meaning more plainly, Socrates.
\socratesspeaks
It is no riddle, Protarchus; the talk is merely jesting with us and means that one part of existences always exists for the sake of something, and the other part is that for the sake of which the former is always coming into being.
\protarchusspeaks
I can hardly understand after all your repetition.
\socratesspeaks
Perhaps, my boy, you will understand better \stephpag{54 a} as the discussion proceeds.
\protarchusspeaks
I hope so.
\socratesspeaks
Let us take another pair.
\protarchusspeaks
What are they?
\socratesspeaks
One is the generation of all things (the process of coming into being), the other is existence or being.
\protarchusspeaks
I accept your two, generation and being.
\socratesspeaks
Quite right. Now which of these shall we say is for the sake of the other, generation for the sake of being, or being for the sake of generation?
\protarchusspeaks
You are now asking whether that which is called being is what it is for the sake of generation?
\socratesspeaks
Yes, plainly. \stephpag{b}
\protarchusspeaks
For HeavenÕs sake, is this the kind of question you keep asking me, ``Tell me, Protarchus, whether you think shipbuilding is for the sake of ships, or ships for the sake of shipbuilding," and all that sort of thing?
\socratesspeaks
Yes; that is just what I mean, Protarchus.
\protarchusspeaks
Then why did you not answer it yourself, Socrates?
\socratesspeaks
There is no reason why I should not; but I want you to take part in the discussion.
\protarchusspeaks
Certainly.
\socratesspeaks
I say that drugs and all sorts of instruments \stephpag{c} and materials are always employed for the sake of production or generation, but that every instance of generation is for the sake of some being or other, and generation in general is for the sake of being in general.
\protarchusspeaks
That is very clear.
\socratesspeaks
Then pleasure, if it is a form of generation, would be generated for the sake of some form of being.
\protarchusspeaks
Of course.
\socratesspeaks
Now surely that for the sake of which anything is generated is in the class of the good, and that which is generated for the sake of something else, my friend, must be placed in another class. \stephpag{d}
\protarchusspeaks
Most undeniably.
\socratesspeaks
Then if pleasure is a form of generation, we shall be right in placing it in a class other than that of the good, shall we not?
\protarchusspeaks
Quite right.
\socratesspeaks
Then, as I said when we began to discuss this point, we ought to be grateful to him who pointed out that there is only a generation, but no existence, of pleasure; for he is clearly making a laughing-stock of those who assert that pleasure is a good.
\protarchusspeaks
Yes, most emphatically.
\socratesspeaks
And he will also surely make a laughing-stock of all those \stephpag{e} who find their highest end in forms of generation.
\protarchusspeaks
How is that, and to whom do you refer?
\socratesspeaks
To those who, when cured of hunger or thirst or any of the troubles which are cured by generation are pleased because of the generation, as if it were pleasure, and say that they would not wish to live without thirst and hunger and the like, if they could not experience the feelings which follow after them. \stephpag{55 a}
\protarchusspeaks
That seems to be their view.
\socratesspeaks
We should all agree that the opposite of generation is destruction, should we not?
\protarchusspeaks
Inevitably.
\socratesspeaks
And he who chooses as they do would be choosing destruction and generation, not that third life in which there was neither pleasure nor pain, but only the purest possible thought.
\protarchusspeaks
It is a great absurdity, as it appears, Socrates, to tell us that pleasure is a good.
\socratesspeaks
Yes, a great absurdity, and let us go still further.
\protarchusspeaks
How? \stephpag{b}
\socratesspeaks
Is it not absurd to say that there is nothing good in the body or many other things, but only in the soul, and that in the soul the only good is pleasure, and that courage and self-restraint and understanding and all the other good things of the soul are nothing of the sort; and beyond all this to be obliged to say that he who is not feeling pleasure, and is feeling pain, is bad when he feels pain, though he be the best of men, and that he who feels pleasure is, \stephpag{c} when he feels pleasure, the more excellent in virtue the greater the pleasure he feels?
\protarchusspeaks
All that, Socrates, is the height of absurdity.
\socratesspeaks
Now let us not undertake to subject pleasure to every possible test and then be found to give mind and knowledge very gentle treatment. Let us rather strike them boldly everywhere to see if their metal rings unsound at any point; so we shall find out what is by nature purest in them, and then we can make use of the truest elements of these and of pleasure to form our judgement of both.
\protarchusspeaks
Right. \stephpag{d}
\socratesspeaks
Well, then, one part of knowledge is productive, the other has to do with education and support. Is that true?
\protarchusspeaks
It is.
\socratesspeaks
Let us first consider whether in the manual arts one part is more allied to knowledge, and the other less, and the one should be regarded as purest, the other as less pure.
\protarchusspeaks
Yes, we ought to consider that.
\socratesspeaks
And should the ruling elements of each of them be separated and distinguished from the rest?
\protarchusspeaks
What are they, and how can they be separated? \stephpag{e}
\socratesspeaks
For example, if arithmetic and the sciences of measurement and weighing were taken away from all arts, what was left of any of them would be, so to speak, pretty worthless.
\protarchusspeaks
Yes, pretty worthless.
\socratesspeaks
All that would be left for us would be to conjecture and to drill the perceptions by practice and experience, with the additional use of the powers of guessing, \stephpag{56 a} which are commonly called arts and acquire their efficacy by practice and toil.
\protarchusspeaks
That is undeniable.
\socratesspeaks
Take music first; it is full of this; it attains harmony by guesswork based on practice, not by measurement; and flute music throughout tries to find the pitch of each note as it is produced by guess, so that the amount of uncertainty mixed up in it is great, and the amount of certainty small.
\protarchusspeaks
Very true. \stephpag{b}
\socratesspeaks
And we shall find that medicine and agriculture and piloting and generalship are all in the same case.
\protarchusspeaks
Certainly.
\socratesspeaks
But the art of building, I believe, employs the greatest number of measures and instruments which give it great accuracy and make it more scientific than most arts.
\protarchusspeaks
In what way?
\socratesspeaks
In shipbuilding and house-building, and many other branches of wood-working. For the artisan uses a rule, I imagine, a lathe, compasses, a chalk-line, \stephpag{c} and an ingenious instrument called a vice.
\protarchusspeaks
Certainly, Socrates; you are right.
\socratesspeaks
Let us, then, divide the arts, as they are called, into two kinds, those which resemble music, and have less accuracy in their works, and those which, like building, are more exact.
\protarchusspeaks
Agreed.
\socratesspeaks
And of these the most exact are the arts which I just now mentioned first.
\protarchusspeaks
I think you mean arithmetic and the other arts you mentioned with it just now. \stephpag{d}
\socratesspeaks
Certainly. But, Protarchus, ought not these to be divided into two kinds? What do you say?
\protarchusspeaks
What kinds?
\socratesspeaks
Are there not two kinds of arithmetic, that of the people and that of philosophers?
\protarchusspeaks
How can one kind of arithmetic be distinguished from the other?
\socratesspeaks
The distinction is no small one, Protarchus. For some arithmeticians reckon unequal units, \stephpag{e} for instance, two armies and two oxen and two very small or incomparably large units; whereas others refuse to agree with them unless each of countless units is declared to differ not at all from each and every other unit.
\protarchusspeaks
You are certainly quite right in saying that there is a great difference between the devotees of arithmetic, so it is reasonable to assume that it is of two kinds.
\socratesspeaks
And how about the arts of reckoning and measuring as they are used in building and in trade when compared with philosophical geometry \stephpag{57 a} and elaborate computations---shall we speak of each of these as one or as two?
\protarchusspeaks
On the analogy of the previous example, I should say that each of them was two.
\socratesspeaks
Right. But do you understand why I introduced this subject?
\protarchusspeaks
Perhaps; but I wish you would give the answer to your question.
\socratesspeaks
This discussion of ours is now, I think, no less than when we began it, seeking a counterpart of pleasure, \stephpag{b} and therefore it has introduced the present subject and is considering whether there is one kind of knowledge purer than another, as one pleasure is purer than another.
\protarchusspeaks
That is very clear; it was evidently introduced with that object.
\socratesspeaks
Well, had not the discussion already found in what preceded that the various arts had various purposes and various degrees of exactness?
\protarchusspeaks
Certainly.
\socratesspeaks
And after having given an art a single name in what has preceded, thereby making us think that it was a single art, \stephpag{c} does not the discussion now assume that the same art is two and ask whether the art of the philosophers or that of the non-philosophers possesses the higher degree of clearness and purity?
\protarchusspeaks
Yes, I think that is just the question it asks.
\socratesspeaks
Then what reply shall we make, Protarchus?
\protarchusspeaks
Socrates, we have found a marvelously great difference in the clearness of different kinds of knowledge.
\socratesspeaks
That will make the reply easier, will it not?
\protarchusspeaks
Yes, to be sure; and let our reply be this, that the arithmetical and metrical arts far surpass the others and that of these \stephpag{d} the arts which are stirred by the impulse of the true philosophers are immeasurably superior in accuracy and truth about measures and numbers.
\socratesspeaks
We accept that as our judgement, and relying upon you we make this confident reply to those who are clever in straining arguments---
\protarchusspeaks
What reply?
\socratesspeaks
That there are two arts of arithmetic and two of measuring, and many other arts which, like these, are twofold in this way, but possess a single name in common. \stephpag{e}
\protarchusspeaks
Let us give this answer, Socrates, to those who you say are clever; I hope we shall have luck with it.
\socratesspeaks
These, then, we say, are the most exact arts or sciences?
\protarchusspeaks
Certainly.
\socratesspeaks
But the art of dialectic would spurn us, Protarchus, if we should judge that any other art is preferable to her. \stephpag{58 a}
\protarchusspeaks
But what is the art to which this name belongs?
\socratesspeaks
Clearly anybody can recognize the art I mean; for I am confident that all men who have any intellect whatsoever believe that the knowledge which has to do with being, reality, and eternal immutability is the truest kind of knowledge. What do you think, Protarchus?
\protarchusspeaks
I have often heard Gorgias constantly maintain that the art of persuasion surpasses all others for this, he said, makes all things subject to itself, \stephpag{b} not by force, but by their free will, and is by far the best of all arts; so now I hardly like to oppose either him or you.
\socratesspeaks
It seems to me that you wanted to speak and threw down your arms out of modesty.
\protarchusspeaks
Very well; have it as you like.
\socratesspeaks
Is it my fault that you have misunderstood?
\protarchusspeaks
Misunderstood what?
\socratesspeaks
My question, dear Protarchus, was not as yet what art or science surpasses all others \stephpag{c} by being the greatest and best and most useful to us: what I am trying to find out at present is which art, however little and of little use, has the greatest regard for clearness, exactness, and truth. See; you will not make Gorgias angry if you grant that his art is superior for the practical needs of men, but say that the study of which I spoke is superior in the matter of the most perfect truth, just as I said in speaking about the white that if it was small and pure it was superior to that which was great \stephpag{d} but impure. Now, therefore, with careful thought and due consideration, paying attention neither to the usefulness nor to the reputation of any arts or sciences, but to that faculty of our souls, if such there be, which by its nature loves the truth and does all things for the sake of the truth, let us examine this faculty and say whether it is most likely to possess mind and intelligence in the greatest purity, or we must look for some other faculty \stephpag{e} which has more valid claims.
\protarchusspeaks
I am considering, and I think it is difficult to concede that any other science or art cleaves more closely to truth than this.
\socratesspeaks
In saying that, did you bear in mind that the arts in general, and the men who devote themselves to them, \stephpag{59 a} make use of opinion and persistently investigate things which have to do with opinion? And even if they think they are studying nature, they are spending their lives in the study of the things of this world, the manner of their production, their action, and the forces to which they are subjected. Is not that true?
\protarchusspeaks
Yes, it is.
\socratesspeaks
Such thinkers, then, toil to discover, not eternal verities, but transient productions of the present, the future, or the past?
\protarchusspeaks
Perfectly true.
\socratesspeaks
And can we say that any of these things becomes certain, if tested by the touchstone of strictest truth, \stephpag{b} since none of them ever was, will be, or is in the same state?
\protarchusspeaks
Of course not.
\socratesspeaks
How can we gain anything fixed whatsoever about things which have no fixedness whatsoever?
\protarchusspeaks
In no way, as it seems to me.
\socratesspeaks
Then no mind or science which is occupied with them possesses the most perfect truth.
\protarchusspeaks
No, it naturally does not.
\socratesspeaks
Then we must dismiss the thought of you and me and Gorgias and Philebus, and make this solemn declaration \stephpag{c} on the part of our argument.
\protarchusspeaks
What is the solemn declaration?
\socratesspeaks
That fixed and pure and true and what we call unalloyed knowledge has to do with the things which are eternally the same without change or mixture, or with that which is most akin to them; and all other things are to be regarded as secondary and inferior.
\protarchusspeaks
Very true.
\socratesspeaks
And of the names applied to such matters, it would be fairest to give the finest names to the finest things, would it not?
\protarchusspeaks
That is reasonable. \stephpag{d}
\socratesspeaks
Are not mind, then, and wisdom the names which we should honor most?
\protarchusspeaks
Yes.
\socratesspeaks
Then these names are applied most accurately and correctly to cases of contemplation of true being.
\protarchusspeaks
Certainly.
\socratesspeaks
And these are precisely the names which I brought forward in the first place as parties to our suit.
\protarchusspeaks
Yes, of course they are, Socrates.
\socratesspeaks
Very well. As to the mixture of wisdom and pleasure, \stephpag{e} if anyone were to say that we are like artisans, with the materials before us from which to create our work, the simile would be a good one.
\protarchusspeaks
Certainly.
\socratesspeaks
And is it, then, our next task to try to make the mixture?
\protarchusspeaks
Surely.
\socratesspeaks
Would it not be better first to repeat certain things and recall them to our minds?
\protarchusspeaks
What things?
\socratesspeaks
Those which we mentioned before. I think the proverb ``we ought to repeat twice and even three times that which is good" \stephpag{60 a} is an excellent one.
\protarchusspeaks
Surely.
\socratesspeaks
Well then, in GodÕs name; I think this is the gist of our discussion.
\protarchusspeaks
What is it?
\socratesspeaks
Philebus says that pleasure is the true goal of every living being and that all ought to aim at it, and that therefore this is also the good for all, and the two designations ``good" and ``pleasant" are properly and essentially one; Socrates, however, says that they are not one, \stephpag{b} but two in fact as in name, that the good and the pleasant differ from one another in nature, and that wisdomÕs share in the good is greater than pleasureÕs. Is not and was not that what was said, Protarchus?
\protarchusspeaks
Yes, certainly.
\socratesspeaks
And furthermore, is not and was not this a point of agreement among us?
\protarchusspeaks
What?
\socratesspeaks
That the nature of the good differs from all else in this respect. \stephpag{c}
\protarchusspeaks
In what respect?
\socratesspeaks
That whatever living being possesses the good always, altogether, and in all ways, has no further need of anything, but is perfectly sufficient. We agreed to that?
\protarchusspeaks
We did.
\socratesspeaks
And then we tried in thought to separate each from the other and apply them to individual lives, pleasure unmixed with wisdom and likewise wisdom which had not the slightest alloy of pleasure?
\protarchusspeaks
Yes. \stephpag{d}
\socratesspeaks
And did we think then that either of them would be sufficient for any one?
\protarchusspeaks
By no means.
\socratesspeaks
And if we made any mistake at that time, let any one now take up the question again. Assuming that memory, wisdom, knowledge, and true opinion belong to the same class, let him ask whether anyone would wish to have or acquire anything whatsoever without these not to speak of pleasure, be it never so abundant or intense, if he could have no true opinion that he is pleased, no knowledge whatsoever \stephpag{e} of what he has felt, and not even the slightest memory of the feeling. And let him ask in the same way about wisdom, whether anyone would wish to have wisdom without any, even the slightest, pleasure rather than with some pleasures, or all pleasures without wisdom rather than with some wisdom.
\protarchusspeaks
That is impossible, Socrates; it is useless to ask the same question over and over again. \stephpag{61 a}
\socratesspeaks
Then the perfect, that which is to be desired by all and is altogether good, is neither of these?
\protarchusspeaks
Certainly not.
\socratesspeaks
We must, then, gain a clear conception of the good, or at least an outline of it, that we may, as we said, know to what the second place is to be assigned.
\protarchusspeaks
Quite right.
\socratesspeaks
And have we not found a road which leads to the good?
\protarchusspeaks
What road?
\socratesspeaks
If you were looking for a particular man and \stephpag{b} first found out correctly where he lived, you would have made great progress towards finding him whom you sought.
\protarchusspeaks
Yes, certainly.
\socratesspeaks
And just now we received an indication, as we did in the beginning, that we must seek the good, not in the unmixed, but in the mixed life.
\protarchusspeaks
Certainly.
\socratesspeaks
Surely there is greater hope that the object of our search will be clearly present in the well mixed life than in the life which is not well mixed?
\protarchusspeaks
Far greater.
\socratesspeaks
Let us make the mixture, Protarchus, with a prayer to the gods, \stephpag{c} to Dionysus or Hephaestus, or whoever he be who presides over the mixing.
\protarchusspeaks
By all means.
\socratesspeaks
We are like wine-pourers, and beside us are fountains---that of pleasure may be likened to a fount of honey, and the sober, wineless fount of wisdom to one of pure, health-giving water---of which we must do our best to mix as well as possible.
\protarchusspeaks
Certainly we must. \stephpag{d}
\socratesspeaks
Before we make the mixture, tell me: should we be most likely to succeed by mixing all pleasure with all wisdom?
\protarchusspeaks
Perhaps.
\socratesspeaks
But that is not safe; and I think I can offer a plan by which we can make our mixture with less risk.
\protarchusspeaks
What is it?
\socratesspeaks
We found, I believe, that one pleasure was greater than another and one art more exact than another?
\protarchusspeaks
Certainly.
\socratesspeaks
And knowledge was of two kinds, one turning its eyes towards transitory things, \stephpag{e} the other towards things which neither come into being nor pass away, but are the same and immutable for ever. Considering them with a view to truth, we judged that the latter was truer than the former.
\protarchusspeaks
That is quite right.
\socratesspeaks
Then what if we first mix the truest sections of each and see whether, when mixed together, they are capable of giving us the most adorable life, or whether we still need something more and different? \stephpag{62 a}
\protarchusspeaks
I think that is what we should do.
\socratesspeaks
Let us assume, then, a man who possesses wisdom about the nature of justice itself, and reason in accordance with his wisdom, and has the same kind of knowledge of all other things.
\protarchusspeaks
Agreed.
\socratesspeaks
Now will this man have sufficient knowledge, if he is master of the theory of the divine circle and sphere, but is ignorant of our human sphere and human circles, even when he uses these \stephpag{b} and other kinds of rules or patterns in building houses?
\protarchusspeaks
We call that a ridiculous state of intellect in a man, Socrates, which is concerned only with divine knowledge.
\socratesspeaks
What? Do you mean to say that the uncertain and impure art of the false rule and circle is to be put into our mixture?
\protarchusspeaks
Yes, that is inevitable, if any man is ever to find his own way home.
\socratesspeaks
And must we add music, which we said a little while ago \stephpag{c} was full of guesswork and imitation and lacked purity?
\protarchusspeaks
Yes, I think we must, if our life is to be life at all.
\socratesspeaks
Shall I, then, like a doorkeeper who is pushed and hustled by a mob, give up, open the door, and let all the kinds of knowledge stream in, the impure mingling with the pure? \stephpag{d}
\protarchusspeaks
I do not know, Socrates, what harm it can do a man to take in all the other kinds of knowledge if he has the first.
\socratesspeaks
Shall I, then, let them all flow into what Homer very poetically calls ``the mingling of the vales?'' \footnote{Homer, \textit{Iliad} 4.453.}
\protarchusspeaks
Certainly.
\socratesspeaks
They are let in; and now we must turn again to the spring of pleasure. For our original plan for making the mixture, by taking first the true parts, did not succeed; because of our love of knowledge, \stephpag{e} we let all kinds of knowledge in together before pleasure.
\protarchusspeaks
Very true.
\socratesspeaks
So now it is time for us to consider about pleasures also, whether these, too, shall be all let loose together, or we shall let only the true ones loose at first.
\protarchusspeaks
It is much safer to let loose the true first.
\socratesspeaks
We will let them loose, then. But what next? If there are any necessary pleasures, as there were kinds of knowledge, must we not mix them with the true?
\protarchusspeaks
Of course; the necessary pleasures must certainly be added. \stephpag{63 a}
\socratesspeaks
And as we said it was harmless and useful to know all the arts throughout our life, if we now say the same of pleasures---that is, if it is advantageous and harmless for us to enjoy all pleasures throughout life---they must all form part of the mixture.
\protarchusspeaks
What shall we say about these pleasures, and what shall we do?
\socratesspeaks
There is no use in asking us, Protarchus; we must ask the pleasures and the arts and sciences themselves \stephpag{b} about one another.
\protarchusspeaks
What shall we ask them?
\socratesspeaks
``Dear ones---whether you should be called pleasures or by any other name---would you choose to dwell with all wisdom, or with none at all?" I think only one reply is possible.
\protarchusspeaks
What is it?
\socratesspeaks
What we said before: ``For any class to be alone, solitary, and unalloyed is neither altogether possible nor is it profitable; but of all classes, \stephpag{c} comparing them one with another, we think the best to live with is the knowledge of all other things and, so far as is possible, the perfect knowledge of our individual selves."
\protarchusspeaks
``Your reply is excellent," we shall tell them.
\socratesspeaks
Right. And next we must turn to wisdom and mind, and question them. We shall ask them, ``Do you want any further pleasures in the mixture?" And they might reply, ``What pleasures?"
\protarchusspeaks
Quite likely. \stephpag{d}
\socratesspeaks
Then we should go on to say: ``In addition to those true pleasures, do you want the greatest and most intense pleasures also to dwell with you?" ``How can we want them, Socrates," they might perhaps say, ``since they contain countless hindrances for us, inasmuch as they disturb with maddening pleasures the souls of men in which we dwell, thereby preventing us from being born at all, and utterly destroying \stephpag{e} for the most part, through the carelessness and forgetfulness which they engender, those of our children which are born? But the true and pure pleasures, of which you spoke, you must consider almost our own by nature, and also those which are united with health and self-restraint, and furthermore all those which are handmaids of virtue in general and follow everywhere in its train as if it were a god,---add these to the mixture; but as for the pleasures which follow after folly and all baseness, it would be very senseless for anyone who desires to discover the most beautiful and most restful mixture or compound, \stephpag{64 a} and to try to learn which of its elements is good in man and the universe, and what we should divine its nature to be, to mix these with mind." Shall we not say that this reply which mind has now made for itself and memory and right opinion is wise and reasonable?
\protarchusspeaks
Certainly.
\socratesspeaks
But another addition is surely necessary, without which nothing whatsoever can ever come into being. \stephpag{b}
\protarchusspeaks
What is it?
\socratesspeaks
That in which there is no admixture of truth can never truly come into being or exist.
\protarchusspeaks
No, of course not.
\socratesspeaks
No. But if anything is still wanting in our mixture, you and Philebus must speak of it. For to me it seems that our argument is now completed, as it were an incorporeal order which shall rule nobly a living body.
\protarchusspeaks
And you may say, Socrates, that I am of the same opinion. \stephpag{c}
\socratesspeaks
And if we were to say that we are now in the vestibule of the good and of the dwelling of the good, should we not be speaking the truth after a fashion?
\protarchusspeaks
I certainly think so.
\socratesspeaks
What element, then, of the mixture would appear to us to be the most precious and also the chief cause why such a state is beloved of all? When we have discovered this, we will then consider whether it is more closely attached and more akin to pleasure or to mind in the universe. \stephpag{d}
\protarchusspeaks
Right; for that is most serviceable to us in forming our judgement.
\socratesspeaks
And it is quite easy to see the cause which makes any mixture whatsoever either of the highest value or of none at all.
\protarchusspeaks
What do you mean?
\socratesspeaks
Why, everybody knows that.
\protarchusspeaks
Knows what?
\socratesspeaks
That any compound, however made, which lacks measure and proportion, must necessarily destroy its components and first of all itself; \stephpag{e} for it is in truth no compound, but an uncompounded jumble, and is always a misfortune to those who possess it.
\protarchusspeaks
Perfectly true.
\socratesspeaks
So now the power of the good has taken refuge in the nature of the beautiful; for measure and proportion are everywhere identified with beauty and virtue.
\protarchusspeaks
Certainly.
\socratesspeaks
We said that truth also was mingled with them in the compound.
\protarchusspeaks
Certainly.
\socratesspeaks
Then if we cannot catch the good with the aid of one idea, \stephpag{65 a} let us run it down with three---beauty, proportion, and truth, and let us say that these, considered as one, may more properly than all other components of the mixture be regarded as the cause, and that through the goodness of these the mixture itself has been made good.
\protarchusspeaks
Quite right.
\socratesspeaks
So now, Protarchus, any one would be able to judge about pleasure and wisdom, \stephpag{b} and to decide which of them is more akin to the highest good and of greater value among men and gods.
\protarchusspeaks
That is clear; but still it is better to carry on the discussion to the end.
\socratesspeaks
Let us, then, judge each of the three separately in its relation to pleasure and mind; for it is our duty to see to which of the two we shall assign each of them as more akin.
\protarchusspeaks
You refer to beauty, truth, and measure?
\socratesspeaks
Yes. Take truth first, Protarchus; take it and look at the three---mind, truth, \stephpag{c} and pleasure; take plenty of time, and answer to yourself whether pleasure or mind is more akin to truth.
\protarchusspeaks
Why take time? For the difference, to my mind, is great. For pleasure is the greatest of impostors, and the story goes that in the pleasures of love, which are said to be the greatest, perjury is even pardoned by the gods, as if the pleasures were like children, utterly devoid of all sense. \stephpag{d} But mind is either identical with truth or of all things most like it and truest.
\socratesspeaks
Next, then, consider measure in the same way, and see whether pleasure possesses more of it than wisdom, or wisdom than pleasure.
\protarchusspeaks
That also is an easy thing to consider. For I think nothing in the world could be found more immoderate than pleasure and its transports, and nothing more in harmony with measure than mind and knowledge. \stephpag{e}
\socratesspeaks
However, go on and tell about the third. Has mind or pleasure the greater share in beauty?
\protarchusspeaks
But Socrates, no one, either asleep or awake, ever saw or knew wisdom or mind to be or become unseemly at any time or in any way whatsoever.
\socratesspeaks
Right.
\protarchusspeaks
But pleasures, and the greatest pleasures at that, when we see any one enjoying them and observe the ridiculous or utterly disgraceful element which accompanies them, \stephpag{66 a} fill us with a sense of shame; we put them out of sight and hide them, so far as possible; we confine everything of that sort to the night time, as unfit for the sight of day.
\socratesspeaks
Then you will proclaim everywhere, Protarchus, by messengers to the absent and by speech to those present, that pleasure is not the first of possessions, nor even the second, but first the eternal nature has chosen measure, moderation, fitness, and all which is to be considered similar to these.
\protarchusspeaks
That appears to result from what has now been said. \stephpag{b}
\socratesspeaks
Second, then, comes proportion, beauty, perfection, sufficiency, and all that belongs to that class.
\protarchusspeaks
Yes, so it appears.
\socratesspeaks
And if you count mind and wisdom as the third, you will, I prophesy, not wander far from the truth.
\protarchusspeaks
That may be.
\socratesspeaks
And will you not put those properties fourth which we said belonged especially to the soul---sciences, arts, and true opinions they are called--- \stephpag{c} and say that these come after the first three, and are fourth, since they are more akin than pleasure to the good?
\protarchusspeaks
Perhaps.
\socratesspeaks
And fifth, those pleasures which we separated and classed as painless, which we called pure pleasures of the soul itself, those which accompany knowledge and, sometimes, perceptions?
\protarchusspeaks
May be.
\socratesspeaks
``But with the sixth generation," says Orpheus, ``cease the rhythmic song." It seems that our discussion, too, is likely to cease with the sixth decision. \stephpag{d} So after this nothing remains for us but to give our discussion a sort of head.
\protarchusspeaks
Yes, that should be done.
\socratesspeaks
Come then, let us for the third time call the same argument to witness before Zeus the saviour, and proceed.
\protarchusspeaks
What argument?
\socratesspeaks
Philebus declared that pleasure was entirely and in all respects the good.
\protarchusspeaks
Apparently, Socrates, when you said ``the third time" just now, you meant that we must take up our argument again from the beginning. \stephpag{e}
\socratesspeaks
Yes; but let us hear what follows. For I, perceiving the truths which I have now been detailing, and annoyed by the theory held not only by Philebus but by many thousands of others, said that mind was a far better and more excellent thing for human life than pleasure.
\protarchusspeaks
True.
\socratesspeaks
But suspecting that there were many other things to be considered, I said that if anything should be found better than these two, I should support mind against pleasure in the struggle for the second place, and even the second place would be lost by pleasure. \stephpag{67 a}
\protarchusspeaks
Yes, that is what you said.
\socratesspeaks
And next it was most sufficiently proved that each of these two was insufficient.
\protarchusspeaks
Very true.
\socratesspeaks
In this argument, then, both mind and pleasure were set aside; neither of them is the absolute good, since they are devoid of self-sufficiency, adequacy, and perfection?
\protarchusspeaks
Quite right.
\socratesspeaks
And on the appearance of a third competitor, better than either of these, mind is now found to be ten thousand times more akin than pleasure to the victor.
\protarchusspeaks
Certainly.
\socratesspeaks
Then, according to the judgement which has now been given by our discussion, the power of pleasure would be fifth.
\protarchusspeaks
So it seems. \stephpag{b}
\socratesspeaks
But not first, even if all the cattle and horses and other beasts in the world, in their pursuit of enjoyment, so assert. Trusting in them, as augurs trust in birds, the many judge that pleasures are the greatest blessings in life, and they imagine that the lusts of beasts are better witnesses than are the aspirations and thoughts inspired by the philosophic muse.
\protarchusspeaks
Socrates, we all now declare that what you have said is perfectly true.
\socratesspeaks
Then you will let me go?
\protarchusspeaks
There is still a little left, Socrates. I am sure you will not give up before we do, and I will remind you of what remains.


\end{drama}

\end{document}