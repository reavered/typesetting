\documentclass[letterpaper,12pt]{article}
\usepackage[12pt]{moresize}
\usepackage[utf8x]{inputenc}
\usepackage[LGR, T1]{fontenc}
\usepackage[english]{babel}
\usepackage[margin=2.5cm, right = 3.5cm]{geometry}
\usepackage{ebgaramond}

\usepackage{dramatist}
\usepackage{csquotes}

\newcommand{\textgreek}[1]{\begingroup\fontencoding{LGR}\selectfont#1\endgroup}

\renewcommand{\thefootnote}{[\fontfamily{ppl}\selectfont\arabic{footnote}]}
\setlength{\skip\footins}{1cm}
\usepackage[]{footmisc}
\renewcommand{\footnotemargin}{3mm} %Setting left margin
\renewcommand{\footnotelayout}{\hspace{2mm}} %spacing between the footnote number and the text of footnote

\usepackage{marginnote}
\renewcommand{\raggedrightmarginnote}{\raggedleft}
\newcommand{\stephpag}[1]{\marginnote{\small\itshape\fontfamily{ppl}\selectfont #1}}

\title{\vspace{-2.5cm} \scshape Theaetetus \vspace{-8mm}}
\author{}
\date{
	\vspace{-1em}
		\small \fontfamily{ppl}\selectfont Written by Plato, translated by Harold North Fowler, 1921
	\begin{center}
		$\mathsection$
	\end{center}
		\vspace{-2em}
	}

\newenvironment{setting}
	{
		\setlength{\tabcolsep}{3em}
		\begin{center}
			\section*{\normalsize \fontfamily{ppl}\selectfont \itshape \bfseries Persons of the Dialogue\vspace{-1mm}}
			\par
			\begin{tabular}{ll}
	}
	{
			\end{tabular}
		\end{center}
		\par \fontfamily{ppl}\selectfont{\small \textbf{Scene:} \textit{Euclid and Terpsion meet in front of Euclid's house in Megara; they enter the house, and the dialogue is read to them by a servant..}}
		
		\hrulefill
	}

\begin{document}

\Character[Eucleides]{Eucleides.}{eucleides} % define characters
\Character[Terpsion]{Terpsion.}{terpsion}
\Character[Socrates]{Socrates.}{socrates}
\Character[Theodorus]{Theodorus.}{theodorus}
\Character[Theaetetus]{Theaetetus.}{theaetetus}

\begin{minipage}{15.45cm}
\maketitle

\begin{setting}
	\textsc{Eucleides.}	&	\textsc{Theodorus.} \\
	\textsc{Terpsion.}	&	\textsc{Theaetetus.}\\
	\textsc{Socrates.}	
\end{setting}
\end{minipage}
\begin{drama}
\setlength{\parindent}{2em}

\eucleidesspeaks
\stephpag{142 a}Just in from the country, Terpsion, or did you come some time ago?
\terpsionspeaks
Quite a while ago; and I was looking for you in the market-place and wondering that I could not find you.
\eucleidesspeaks
Well, you see, I was not in the city.
\terpsionspeaks
Where then?
\eucleidesspeaks
As I was going down to the harbor I met Theaetetus being carried to Athens from the camp at Corinth.
\terpsionspeaks
Alive or dead? \stephpag{b}
\eucleidesspeaks
Just barely alive; for he is suffering severely from wounds, and, worse than that, he has been taken with the sickness that has broken out in the army.
\terpsionspeaks
You mean the dysentery?
\eucleidesspeaks
Yes.
\terpsionspeaks
What a man he is who you say is in danger!
\eucleidesspeaks
A noble man, Terpsion, and indeed just now I heard some people praising him highly for his conduct in the battle.
\terpsionspeaks
That is not at all strange; it would have been much more remarkable if he had not so conducted himself. But why did he not \stephpag{c} stop here in Megara?
\eucleidesspeaks
He was in a hurry to get home; for I begged and advised him to stop, but he would not. So I went along with him, and as I was coming back I thought of Socrates and wondered at his prophetic gift, especially in what he said about him. For I think he met him a little before his own death, when Theaetetus was a mere boy, and as a result of acquaintance and conversation with him, he greatly admired his qualities. When I went to Athens he related to me the conversation \stephpag{d} he had with him, which was well worth hearing, and he said he would surely become a notable man if he lived.
\terpsionspeaks
And he was right, apparently. But what was the talk? Could you relate it?
\eucleidesspeaks
No, by Zeus, at least not offhand. \stephpag{143 a} But I made notes at the time as soon as I reached home, then afterwards at my leisure, as I recalled things, I wrote them down, and whenever I went to Athens I used to ask Socrates about what I could not remember, and then I came here and made corrections; so that I have pretty much the whole talk written down.
\terpsionspeaks
That is true. I heard you say so before; and really I have been waiting about here all along intending to ask you to show it to me. What hinders us from reading it now? Certainly I need to rest, since I have come from the country. \stephpag{b}
\eucleidesspeaks
And I myself went with Theaetetus as far as Erineum,\footnote{Erineum was between Eleusis and Athens, near the Cephissus. Apparently Eucleides had walked some thirty miles.} so I also should not be sorry to take a rest. Come, let us go, and while we are resting, the boy shall read to us.
\terpsionspeaks
Very well.
\eucleidesspeaks
Here is the book, Terpsion. Now this is the way I wrote the conversation: I did not represent Socrates relating it to me, as he did, but conversing with those with whom he told me he conversed. And he told me they were the geometrician Theodorus and Theaetetus. Now in order that \stephpag{c} the explanatory words between the speeches might not be annoying in the written account, such as ``and I said" or ``and I remarked," whenever Socrates spoke, or ``he agreed or he did not agree," in the case of the interlocutor, I omitted all that sort of thing and represented Socrates himself as talking with them.
\terpsionspeaks
That is quite fitting, Eucleides.
\eucleidesspeaks
Come, boy, take the book and read. \stephpag{d}
\socratesspeaks
If I cared more for Cyrene and its affairs, Theodorus, I should ask you about things there and about the people, whether any of the young men there are devoting themselves to geometry or any other form of philosophy; but as it is, since I care less for those people than for the people here, I am more eager to know which of our own young men are likely to gain reputation. These are the things I myself investigate, so far as I can, and about which I question those others with whom I see that the young men like to associate. Now a great many of them come to you, and rightly, \stephpag{e} for you deserve it on account of your geometry, not to speak of other reasons. So if you have met with any young man who is worth mentioning, I should like to hear about him.
\theodorusspeaks
Truly, Socrates, it is well worth while for me to talk and for you to hear about a splendid young fellow, one of your fellow-citizens, whom I have met. Now if he were handsome, I should be very much afraid to speak, lest someone should think I was in love with him. But the fact is—now don't be angry with me—he is not handsome, but is like you in his snub nose and protruding eyes, only those features are less marked in him than in you. \stephpag{144 a} You see I speak fearlessly. But I assure you that among all the young men I have ever met—and I have had to do with a great many—I never yet found one of such marvelously fine qualities. He is quick to learn, beyond almost anyone else, yet exceptionally gentle, and moreover brave beyond any other; I should not have supposed such a combination existed, and I do not see it elsewhere. On the contrary, those who, like him, have quick, sharp minds and good memories, have usually also quick tempers; they dart off and are swept away, \stephpag{b} like ships without ballast; they are excitable rather than courageous; those, on the other hand, who are steadier are somewhat dull when brought face to face with learning, and are very forgetful. But this boy advances toward learning and investigation smoothly and surely and successfully, with perfect gentleness, like a stream of oil that flows without a sound, so that one marvels how he accomplishes all this at his age.
\socratesspeaks
That is good news; but which of our citizens is his father?
\theodorusspeaks
I have heard the name, but do not remember it. \stephpag{c} However, it does not matter, for the youth is the middle one of those who are now coming toward us. He and those friends of his were anointing themselves in the outer course,\footnote{The scene is evidently laid in a gymnasium; the young men have been exercising.} and now they seem to have finished and to be coming here. See if you recognize him.
\socratesspeaks
Yes, I do. He is the son of Euphronius of Sunium, who is a man of just the sort you describe, and of good repute in other respects; moreover he left a very large property. But the youth's name I do not know. \stephpag{d}
\theodorusspeaks
Theaetetus is his name, Socrates; but I believe the property was squandered by trustees. Nevertheless, Socrates, he is remarkably liberal with his money, too.
\socratesspeaks
It is a noble man that you describe. Now please tell him to come here and sit by us.
\theodorusspeaks
I will. Theaetetus, come here to Socrates.
\socratesspeaks
Yes, do so, Theaetetus, that I may look at myself and see what sort of a face I have; \stephpag{e} for Theodorus says it is like yours. Now if we each had a lyre, and he said we had tuned them to the same key, should we take his word for it without more ado, or should we inquire first whether he who said it was a musician?
\theaetetusspeaks
We should inquire.
\socratesspeaks
Then if we found that he was a musician, we should believe him, but if not, we should refuse to take his word?
\theaetetusspeaks
Yes.
\socratesspeaks
But now, if we are concerned about the likeness of our faces, \stephpag{145 a} we must consider whether he who speaks is a painter, or not.
\theaetetusspeaks
I think we must.
\socratesspeaks
Well, is Theodorus a painter?
\theaetetusspeaks
Not so far as I know.
\socratesspeaks
Nor a geometrician, either?
\theaetetusspeaks
Oh yes, decidedly, Socrates.
\socratesspeaks
And an astronomer, and an arithmetician, and a musician, and in general an educated man?
\theaetetusspeaks
I think so.
\socratesspeaks
Well then, if he says, either in praise or blame, that we have some physical resemblance, it is not especially worth while to pay attention to him.
\theaetetusspeaks
Perhaps not. \stephpag{b}
\socratesspeaks
But what if he should praise the soul of one of us for virtue and wisdom? Is it not worth while for the one who hears to examine eagerly the one who is praised, and for that one to exhibit his qualities with eagerness?
\theaetetusspeaks
Certainly, Socrates.
\socratesspeaks
Then, my dear Theaetetus, this is just the time for you to exhibit your qualities and for me to examine them; for I assure you that Theodorus, though he has praised many foreigners and citizens to me, never praised anyone as he praised you just now.
\theaetetusspeaks
A good idea, Socrates; but make sure \stephpag{c} that he was not speaking in jest.
\socratesspeaks
That is not Theodorus's way. But do not seek to draw back from your agreement on the pretext that he is jesting, or he will be forced to testify under oath; for certainly no one will accuse him of perjury. Come, be courageous and hold to the agreement.
\theaetetusspeaks
I suppose I must, if you say so.
\socratesspeaks
Now tell me; I suppose you learn some geometry from Theodorus?
\theaetetusspeaks
Yes. \stephpag{d}
\socratesspeaks
And astronomy and harmony and arithmetic?
\theaetetusspeaks
I try hard to do so.
\socratesspeaks
And so do I, my boy, from him and from any others who I think know anything about these things. But nevertheless, although in other respects I get on fairly well in them, yet I am in doubt about one little matter, which should be investigated with your help and that of these others. Tell me, is not learning growing wiser about that which one learns?
\theaetetusspeaks
Of course.
\socratesspeaks
And the wise, I suppose, are wise by wisdom.
\theaetetusspeaks
Yes. \stephpag{e}
\socratesspeaks
And does this differ at all from knowledge?
\theaetetusspeaks
Does what differ?
\socratesspeaks
Wisdom. Or are not people wise in that of which they have knowledge?
\theaetetusspeaks
Of course.
\socratesspeaks
Then knowledge and wisdom are the same thing?
\theaetetusspeaks
Yes.
\socratesspeaks
Well, it is just this that I am in doubt about and cannot fully grasp by my own efforts—what knowledge really is. \stephpag{146 a} Can we tell that? What do you say? Who of us will speak first? And he who fails, and whoever fails in turn, shall go and sit down and be donkey, as the children say when they play ball; and whoever gets through without failing shall be our king and shall order us to answer any questions he pleases. Why are you silent? I hope, Theodorus, I am not rude, through my love of discussion and my eagerness to make us converse and show ourselves friends and ready to talk to one another. \stephpag{b}
\theodorusspeaks
That sort of thing would not be at all rude, Socrates; but tell one of the youths to answer your questions; for I am unused to such conversation and, moreover, I am not of an age to accustom myself to it. But that would be fitting for these young men, and they would improve much more than I; for the fact is, youth admits of improvement in every way. Come, question Theaetetus as you began to do, and do not let him off.
\socratesspeaks
Well, Theaetetus, you hear what Theodorus says, \stephpag{c} and I think you will not wish to disobey him, nor is it right for a young person to disobey a wise man when he gives instructions about such matters. Come, speak up well and nobly. What do you think knowledge is?
\theaetetusspeaks
Well, Socrates, I must, since you bid me. For, if I make a mistake, you are sure to set me right.
\socratesspeaks
Certainly, if we can.
\theaetetusspeaks
Well then, I think the things one might learn from Theodorus are knowledge—geometry and all the things you spoke of just now—and also cobblery and \stephpag{d} the other craftsmen's arts; each and all of these are nothing else but knowledge.
\socratesspeaks
You are noble and generous, my friend, for when you are asked for one thing you give many, and a variety of things instead of a simple answer.
\theaetetusspeaks
What do you mean by that, Socrates?
\socratesspeaks
Nothing, perhaps; but I will tell you what I think I mean. When you say ``cobblery" you speak of nothing else than the art of making shoes, do you?
\theaetetusspeaks
Nothing else. \stephpag{e}
\socratesspeaks
And when you say ``carpentry"? Do you mean anything else than the art of making wooden furnishings?
\theaetetusspeaks
Nothing else by that, either.
\socratesspeaks
Then in both cases you define that to which each form of knowledge belongs?
\theaetetusspeaks
Yes.
\socratesspeaks
But the question, Theaetetus, was not to what knowledge belongs, nor how many the forms of knowledge are; for we did not wish to number them, but to find out what knowledge itself really is. Or is there nothing in what I say?
\theaetetusspeaks
Nay, you are quite right. \stephpag{147 a}
\socratesspeaks
Take this example. If anyone should ask us about some common everyday thing, for instance, what clay is, and we should reply that it is the potters' clay and the oven makers' clay and the brickmakers' clay, should we not be ridiculous?
\theaetetusspeaks
Perhaps.
\socratesspeaks
Yes in the first place for assuming that the questioner can understand from our answer what clay is, when we say ``clay," no matter whether we add ``the image-makers'" \stephpag{b} or any other craftsmen's. Or does anyone, do you think, understand the name of anything when he does not know what the thing is?
\theaetetusspeaks
By no means.
\socratesspeaks
Then he does not understand knowledge of shoes if he does not know knowledge.
\theaetetusspeaks
No.
\socratesspeaks
Then he who is ignorant of knowledge does not understand cobblery or any other art.
\theaetetusspeaks
That is true.
\socratesspeaks
Then it is a ridiculous answer to the question ``what is knowledge?" when we give the name of some art; \stephpag{c} for we give in our answer something that knowledge belongs to, when that was not what we were asked.
\theaetetusspeaks
So it seems.
\socratesspeaks
Secondly, when we might have given a short, everyday answer, we go an interminable distance round; for instance, in the question about clay, the everyday, simple thing would be to say ``clay is earth mixed with moisture" without regard to whose clay it is.
\theaetetusspeaks
It seems easy just now, Socrates, as you put it; but you are probably asking the kind of thing that came up among us lately when \stephpag{d} your namesake, Socrates here, and I were talking together.
\socratesspeaks
What kind of thing was that, Theaetetus?
\theaetetusspeaks
Theodorus here was drawing some figures for us in illustration of roots, showing that squares containing three square feet and five square feet are not commensurable in length with the unit of the foot, and so, selecting each one in its turn up to the square containing seventeen square feet and at that he stopped. Now it occurred to us, since the number of roots appeared to be infinite, to try to collect them under one name, \stephpag{e} by which we could henceforth call all the roots.\footnote{A simple form of the first statement would be: the square roots of 3, 5, etc., are irrational numbers or surds. The word \textgreek{δύναμις} has not the meaning which we give in English to ``power,'' namely the result of multiplication of a number by itself, but that which we give to ``root,'' i.e. the number which, when multiplied by itself, produces a given result. Here Theaetetus is speaking of square roots only; and when he speaks of numbers and of equal factors he evidently thinks of rational whole numbers only, not of irrational numbers or fractions. He is not giving an exhaustive presentation of his investigation, but merely a brief sketch of it to illustrate his understanding of the purpose of Socrates. Toward the end of this sketch the word \textgreek{δύναμις} is limited to the square roots of ``oblong'' numbers, i.e. to surds. The modern reader may be somewhat confused because Theaetetus seems to speak of arithmetical facts in geometrical terms. (Cf. Gow, Short History of Greek Mathematics, p. 85.)}
\socratesspeaks
And did you find such a name?
\theaetetusspeaks
I think we did. But see if you agree.
\socratesspeaks
Speak on.
\theaetetusspeaks
We divided all number into two classes. The one, the numbers which can be formed by multiplying equal factors, we represented by the shape of the square and called square or equilateral numbers.
\socratesspeaks
Well done!
\theaetetusspeaks
The numbers between these, such as three \stephpag{148 a} and five and all numbers which cannot be formed by multiplying equal factors, but only by multiplying a greater by a less or a less by a greater, and are therefore always contained in unequal sides, we represented by the shape of the oblong rectangle and called oblong numbers.
\socratesspeaks
Very good; and what next?
\theaetetusspeaks
All the lines which form the four sides of the equilateral or square numbers we called lengths, and those which form the oblong numbers we called surds, because they are not commensurable with the others \stephpag{b} in length, but only in the areas of the planes which they have the power to form. And similarly in the case of solids.\footnote{That is, cubes and cube roots.}
\socratesspeaks
Most excellent, my boys! I think Theodorus will not be found liable to an action for false witness.
\theaetetusspeaks
But really, Socrates, I cannot answer that question of yours about knowledge, as we answered the question about length and square roots. And yet you seem to me to want something of that kind. So Theodorus appears to be a false witness after all. \stephpag{c}
\socratesspeaks
Nonsense! If he were praising your running and said he had never met any young man who was so good a runner, and then you were beaten in a race by a full grown man who held the record, do you think his praise would be any less truthful?
\theaetetusspeaks
Why, no.
\socratesspeaks
And do you think that the discovery of knowledge, as I was just now saying, is a small matter and not a task for the very ablest men?
\theaetetusspeaks
By Zeus, I think it is a task for the very ablest.
\socratesspeaks
Then you must have confidence in yourself, and believe that Theodorus is right, \stephpag{d} and try earnestly in every way to gain an understanding of the nature of knowledge as well as of other things.
\theaetetusspeaks
If it is a question of earnestness, Socrates, the truth will come to light.
\socratesspeaks
Well then—for you pointed out the way admirably just now—take your answer about the roots as a model, and just as you embraced them all in one class, though they were many, try to designate the many forms of knowledge by one definition. \stephpag{e}
\theaetetusspeaks
But I assure you, Socrates, I have often tried to work that out, when I heard reports of the questions that you asked, but I can neither persuade myself that I have any satisfactory answer, nor can I find anyone else who gives the kind of answer you insist upon; and yet, on the other hand, I cannot get rid of a feeling of concern about the matter.
\socratesspeaks
Yes, you are suffering the pangs of labor, Theaetetus, because you are not empty, but pregnant.
\theaetetusspeaks
I do not know, Socrates; I merely tell you what I feel. \stephpag{149 a}
\socratesspeaks
Have you then not heard, you absurd boy, that I am the son of a noble and burly midwife, Phaenarete?
\theaetetusspeaks
Yes, I have heard that.
\socratesspeaks
And have you also heard that I practise the same art?
\theaetetusspeaks
No, never.
\socratesspeaks
But I assure you it is true; only do not tell on me to the others; for it is not known that I possess this art. But other people, since they do not know it, do not say this of me, but say that I am a most eccentric person and drive men to distraction. Have you heard that also? \stephpag{b}
\theaetetusspeaks
Yes, I have.
\socratesspeaks
Shall I tell you the reason then?
\theaetetusspeaks
Oh yes, do.
\socratesspeaks
Just take into consideration the whole business of the midwives, and you will understand more easily what I mean. For you know, I suppose, that no one of them attends other women while she is still capable of conceiving and bearing but only those do so who have become too old to bear.
\theaetetusspeaks
Yes, certainly.
\socratesspeaks
They say the cause of this is Artemis, because she, a childless goddess, has had childbirth allotted to her as her special province. Now it would seem she did not allow \stephpag{c} barren women to be midwives, because human nature is too weak to acquire an art which deals with matters of which it has no experience, but she gave the office to those who on account of age were not bearing children, honoring them for their likeness to herself.
\theaetetusspeaks
Very likely.
\socratesspeaks
Is it not, then, also likely and even necessary, that midwives should know better than anyone else who are pregnant and who are not?
\theaetetusspeaks
Certainly.
\socratesspeaks
And furthermore, the midwives, by means of drugs \stephpag{d} and incantations, are able to arouse the pangs of labor and, if they wish, to make them milder, and to cause those to bear who have difficulty in bearing; and they cause miscarriages if they think them desirable.
\theaetetusspeaks
That is true.
\socratesspeaks
Well, have you noticed this also about them, that they are the most skillful of matchmakers, since they are very wise in knowing what union of man and woman will produce the best possible children?
\theaetetusspeaks
I do not know that at all.
\socratesspeaks
But be assured that they are prouder of this \stephpag{e} than of their skill in cutting the umbilical cord. Just consider. Do you think the knowledge of what soil is best for each plant or seed belongs to the same art as the tending and harvesting of the fruits of the earth, or to another?
\theaetetusspeaks
To the same art.
\socratesspeaks
And in the case of a woman, do you think, my friend, that there is one art for the sowing and another for the harvesting?
\theaetetusspeaks
It is not likely. \stephpag{150 a}
\socratesspeaks
No; but because there is a wrongful and unscientific way of bringing men and women together, which is called pandering, the midwives, since they are women of dignity and worth, avoid matchmaking, through fear of falling under the charge of pandering. And yet the true midwife is the only proper match-maker.
\theaetetusspeaks
It seems so.
\socratesspeaks
So great, then, is the importance of midwives; but their function is less important than mine. For women do not, like my patients, bring forth \stephpag{b} at one time real children and at another mere images which it is difficult to distinguish from the real. For if they did, the greatest and noblest part of the work of the midwives would be in distinguishing between the real and the false. Do you not think so?
\theaetetusspeaks
Yes, I do.
\socratesspeaks
All that is true of their art of midwifery is true also of mine, but mine differs from theirs in being practised upon men, not women, and in tending their souls in labor, not their bodies. But the greatest thing about my art is this, \stephpag{c} that it can test in every way whether the mind of the young man is bringing forth a mere image, an imposture, or a real and genuine offspring. For I have this in common with the midwives: I am sterile in point of wisdom, and the reproach which has often been brought against me, that I question others but make no reply myself about anything, because I have no wisdom in me, is a true reproach; and the reason of it is this: the god compels me to act as midwife, but has never allowed me to bring forth. I am, then, not at all a wise person myself, \stephpag{d} nor have I any wise invention, the offspring born of my own soul; but those who associate with me, although at first some of them seem very ignorant, yet, as our acquaintance advances, all of them to whom the god is gracious make wonderful progress, not only in their own opinion, but in that of others as well. And it is clear that they do this, not because they have ever learned anything from me, but because they have found in themselves many fair things and have brought them forth. But the delivery is due to the god and me. And the proof of it is this: many before now, \stephpag{e} being ignorant of this fact and thinking that they were themselves the cause of their success, but despising me, have gone away from me sooner than they ought, whether of their own accord or because others persuaded them to do so. Then, after they have gone away, they have miscarried thenceforth on account of evil companionship, and the offspring which they had brought forth through my assistance they have reared so badly that they have lost it; they have considered impostures and images of more importance than the truth, and at last it was evident to themselves, as well as to others, that they were ignorant. One of these was \stephpag{151 a} Aristeides, the son of Lysimachus, and there are very many more. When such men come back and beg me, as they do, with wonderful eagerness to let them join me again, the spiritual monitor that comes to me forbids me to associate with some of them, but allows me to converse with others, and these again make progress. Now those who associate with me are in this matter also like women in childbirth; they are in pain and are full of trouble night and day, much more than are the women; and my art can arouse this pain and cause it to cease. Well, that is what happens to them. \stephpag{b} But in some cases, Theaetetus, when they do not seem to me to be exactly pregnant, since I see that they have no need of me, I act with perfect goodwill as match-maker and, under God, I guess very successfully with whom they can associate profitably, and I have handed over many of them to Prodicus, and many to other wise and inspired men. Now I have said all this to you at such length, my dear boy, because I suspect that you, as you yourself believe, are in pain because you are pregnant with something within you. Apply, then, to me, remembering that I am the son of a midwife \stephpag{c} and have myself a midwife's gifts, and do your best to answer the questions I ask as I ask them. And if, when I have examined any of the things you say, it should prove that I think it is a mere image and not real, and therefore quietly take it from you and throw it away, do not be angry as women are when they are deprived of their first offspring. For many, my dear friend, before this have got into such a state of mind towards me that they are actually ready to bite me, if I take some foolish notion away from them, and they do not believe that I do this in kindness, \stephpag{d} since they are far from knowing that no god is unkind to mortals, and that I do nothing of this sort from unkindness, either, and that it is quite out of the question for me to allow an imposture or to destroy the true. And so, Theaetetus, begin again and try to tell us what knowledge is. And never say that you are unable to do so; for if God wills it and gives you courage, you will be able.
\theaetetusspeaks
Well then, Socrates, since you are so urgent it would be disgraceful for anyone not to exert himself in every way \stephpag{e} to say what he can. I think, then, that he who knows anything perceives that which he knows, and, as it appears at present, knowledge is nothing else than perception.
\socratesspeaks
Good! Excellent, my boy! That is the way one ought to speak out. But come now, let us examine your utterance together, and see whether it is a real offspring or a mere wind-egg. Perception, you say, is knowledge?
\theaetetusspeaks
Yes.
\socratesspeaks
And, indeed, if I may venture to say so, it is not a bad description of knowledge \stephpag{152 a} that you have given, but one which Protagoras also used to give. Only, he has said the same thing in a different way. For he says somewhere that man is ``the measure of all things, of the existence of the things that are and the non-existence of the things that are not." You have read that, I suppose?
\theaetetusspeaks
Yes, I have read it often.
\socratesspeaks
Well, is not this about what he means, that individual things are for me such as they appear to me, and for you in turn such as they appear to you —you and I being ``man"?
\theaetetusspeaks
Yes, that is what he says. \stephpag{b}
\socratesspeaks
It is likely that a wise man is not talking nonsense; so let us follow after him. Is it not true that sometimes, when the same wind blows, one of us feels cold, and the other does not? or one feels slightly and the other exceedingly cold?
\theaetetusspeaks
Certainly.
\socratesspeaks
Then in that case, shall we say that the wind is in itself cold or not cold or shall we accept Protagoras's saying that it is cold for him who feels cold and not for him who does not?
\theaetetusspeaks
Apparently we shall accept that.
\socratesspeaks
Then it also seems cold, or not, to each of the two?
\theaetetusspeaks
Yes.
\socratesspeaks
But ``seems" denotes perceiving?
\theaetetusspeaks
It does. \stephpag{c}
\socratesspeaks
Then seeming and perception are the same thing in matters of warmth and everything of that sort. For as each person perceives things, such they are to each person.
\theaetetusspeaks
Apparently.
\socratesspeaks
Perception, then, is always of that which exists and, since it is knowledge, cannot be false.
\theaetetusspeaks
So it seems.
\socratesspeaks
By the Graces! I wonder if Protagoras, who was a very wise man, did not utter this dark saying to the common herd like ourselves, and tell the truth\footnote{An allusion to the title of Protagoras's book, \emph{Truth}.} in secret to his pupils. \stephpag{d}
\theaetetusspeaks
Why, Socrates, what do you mean by that?
\socratesspeaks
I will tell you and it is not a bad description, either, that nothing is one and invariable, and you could not rightly ascribe any quality whatsoever to anything, but if you call it large it will also appear to be small, and light if you call it heavy, and everything else in the same way, since nothing whatever is one, either a particular thing or of a particular quality; but it is out of movement and motion and mixture with one another that all those things become which we wrongly say ``are"—wrongly, because nothing ever is, but is always becoming. \stephpag{e} And on this subject all the philosophers, except Parmenides, may be marshalled in one line—Protagoras and Heracleitus and Empedocles—and the chief poets in the two kinds of poetry, Epicharmus, in comedy, and in tragedy, Homer, who, in the line
\begin{displayquote}
	Oceanus the origin of the gods, and Tethys their mother,\\
	--- Hom. Il. 14.201, 302
\end{displayquote}
has said that all things are the offspring of flow and motion; or don't you think he means that?
\theaetetusspeaks
I think he does.
\socratesspeaks
Then who could still contend with such a great host, \stephpag{153 a} led by Homer as general, and not make himself ridiculous?
\theaetetusspeaks
It is not easy, Socrates.
\socratesspeaks
No, Theaetetus, it is not. For the doctrine is amply proved by this, namely, that motion is the cause of that which passes for existence, that is, of becoming, whereas rest is the cause of non-existence and destruction; for warmth or fire, which, you know, is the parent and preserver of all other things, is itself the offspring of movement and friction, and these two are forms of motion. Or are not these the source of fire? \stephpag{b}
\theaetetusspeaks
Yes, they are.
\socratesspeaks
And furthermore, the animal kingdom is sprung from these same sources.
\theaetetusspeaks
Of course.
\socratesspeaks
Well, then, is not the bodily habit destroyed by rest and idleness, and preserved, generally speaking, by gymnastic exercises and motions?
\theaetetusspeaks
Yes.
\socratesspeaks
And what of the habit of the soul? Does not the soul acquire information and is it not preserved and made better through learning and practice, which are motions, whereas through rest, which is want of practice and of study, \stephpag{c} it learns nothing and forgets what it has learned?
\theaetetusspeaks
Certainly.
\socratesspeaks
Then the good, both for the soul and for the body, is motion, and rest is the opposite?
\theaetetusspeaks
Apparently.
\socratesspeaks
Now shall I go on and mention to you also windless air, calm sea, and all that sort of thing, and say that stillness causes decay and destruction and that the opposite brings preservation? And shall I add to this the all-compelling and crowning argument that Homer by ``the golden chain"\footnote{Hom. Il. 8.18 ff. especially 26. In this passage Zeus declares that all the gods and goddesses together could not, with a golden chain, drag him from on high, but that if he pulled, he would drag them, with earth and sea, would then bind the chain round the summit of Olympus, and all the rest would hang aloft. This ``crowning argument'' is a \emph{reductio ad absurdum} of the habit of using texts from Homer in support of all kinds of doctrine.} refers to nothing else than the sun, \stephpag{d} and means that so long as the heavens and the sun go round everything exists and is preserved, among both gods and men, but if the motion should stop, as if bound fast, everything would be destroyed and would, as the saying is, be turned upside down?
\theaetetusspeaks
Yes, Socrates, I think he means what you say he does.
\socratesspeaks
Then, my friend, you must apply the doctrine in this way: first as concerns vision, the color that you call white is not to be taken as something separate outside of your eyes, nor yet as something inside of them; and you must not assign any place to it, \stephpag{e} for then it would at once be in a definite position and stationary and would have no part in the process of becoming.
\theaetetusspeaks
But what do you mean?
\socratesspeaks
Let us stick close to the statement we made a moment ago, and assume that nothing exists by itself as invariably one: then it will be apparent that black or white or any other color whatsoever is the result of the impact of the eye upon the appropriate motion, and therefore that which we call color \stephpag{154 a} will be in each instance neither that which impinges nor that which is impinged upon, but something between, which has occurred, peculiar to each individual. Or would you maintain that each color appears to a dog, or any other animal you please, just as it does to you?
\theaetetusspeaks
No, by Zeus, I wouldn't.
\socratesspeaks
Well, does anything whatsoever appear the same to any other man as to you? Are you sure of this? Or are you not much more convinced that nothing appears the same even to you, because you yourself are never exactly the same?
\theaetetusspeaks
Yes, I am much more convinced of the last.
\socratesspeaks
Then, if that with which I compare myself in size, or which I touch, \stephpag{b} were really large or white or hot, it would never have become different by coming in contact with something different, without itself changing; and if, on the other hand, that which did the comparing or the touching were really large or white or hot, it would not have become different when something different approached it or was affected in some way by it, without being affected in some way itself. For nowadays, my friend, we find ourselves rather easily forced to make extraordinary and absurd statements, as Protagoras and everyone who undertakes to agree with him would say.
\theaetetusspeaks
What do you mean? What statements? \stephpag{c}
\socratesspeaks
Take a little example and you will know all I have in mind. Given six dice, for instance, if you compare four with them, we say that they are more than the four, half as many again, but if you compare twelve with them, we say they are less, half as many; and any other statement would be inadmissible; or would you admit any other?
\theaetetusspeaks
Not I.
\socratesspeaks
Well then, if Protagoras, or anyone else, ask you, ``Theaetetus, can anything become greater or more in any other way than by being increased?" what reply will you make?
\theaetetusspeaks
If I am to say what I think, Socrates, with reference to \stephpag{d} the present question, I should say ``no," but if I consider the earlier question, I should say ``yes," for fear of contradicting myself.
\socratesspeaks
Good, by Hera! Excellent, my friend! But apparently, if you answer ``yes" it will be in the Euripidean spirit; for our tongue will be unconvinced, but not our mind.\footnote{Eur. Hipp. 612 \textgreek{ἡ γλῶσσ᾽ ὀμώμοχ᾽, ἡ δὲ φρὴν ἀνώμοτος}, ``my tongue has sworn, but my mind is unsworn.''}
\theaetetusspeaks
True.
\socratesspeaks
Well, if you and I were clever and wise and had found out everything about the mind, we should henceforth spend the rest of our time testing each other out of the fulness of our wisdom, \stephpag{e} rushing together like sophists in a sophistical combat, battering each other's arguments with counter arguments. But, as it is, since we are ordinary people, we shall wish in the first place to look into the real essence of our thoughts and see whether they harmonize with one another or not at all.
\theaetetusspeaks
Certainly that is what I should like.
\socratesspeaks
And so should I. But since this is the case, and we have plenty of time, shall we not quietly, without any impatience, but truly examining ourselves, \stephpag{155 a} consider again the nature of these appearances within us? And as we consider them, I shall say, I think, first, that nothing can ever become more or less in size or number, so long as it remains equal to itself. Is it not so?
\theaetetusspeaks
Yes.
\socratesspeaks
And secondly, that anything to which nothing is added and from which nothing is subtracted, is neither increased nor diminished, but is always equal.
\theaetetusspeaks
Certainly. \stephpag{b}
\socratesspeaks
And should we not say thirdly, that what was not previously could not afterwards be without becoming and having become?
\theaetetusspeaks
Yes, I agree.
\socratesspeaks
These three assumptions contend with one another in our minds when we talk about the dice, or when we say that I, who do not, at my age, either increase in size or diminish, am in the course of a year first larger than you, who are young, and afterwards smaller, when nothing has been taken from my size, \stephpag{c} but you have grown. For I am, it seems, afterwards what I was not before, and I have not become so; for it is impossible to have become without becoming, and without losing anything of my size I could not become smaller. And there are countless myriads of such contradictions, if we are to accept these that I have mentioned. You follow me, I take it, Theaetetus, for I think you are not new at such things.
\theaetetusspeaks
By the gods, Socrates, I am lost in wonder when I think of all these things, and sometimes when I regard them it really makes my head swim. \stephpag{d}
\socratesspeaks
Theodorus seems to be a pretty good guesser about your nature. For this feeling of wonder shows that you are a philosopher, since wonder is the only beginning of philosophy, and he who said that Iris was the child of Thaumas\footnote{Hes. Theog. 750 Iris is the messenger of heaven, and Plato interprets the name of her father as ``Wonder'' (\textgreek{θαῦμα}).} made a good genealogy. But do you begin to understand why these things are so, according to the doctrine we attribute to Protagoras, or do you not as yet?
\theaetetusspeaks
Not yet, I think.
\socratesspeaks
And will you be grateful to me if I help you \stephpag{e} to search out the hidden truth of the thought of a famous man or, I should say, of famous men?
\theaetetusspeaks
Of course I shall be grateful, very grateful.
\socratesspeaks
Look round and see that none of the uninitiated is listening. The uninitiated are those who think nothing is except what they can grasp firmly with their hands, and who deny the existence of actions and generation and all that is invisible.
\theaetetusspeaks
Truly, Socrates, those you speak of are very stubborn \stephpag{156 a} and perverse mortals.
\socratesspeaks
So they are, my boy, quite without culture. But others are more clever, whose secret doctrines I am going to disclose to you. For them the beginning, upon which all the things we were just now speaking of depend, is the assumption that everything is real motion and that there is nothing besides this, but that there are two kinds of motion, each infinite in the number of its manifestations, and of these kinds one has an active, the other a passive force. From the union and friction of these two are born offspring, infinite in number, but always twins, the object of sense \stephpag{b} and the sense which is always born and brought forth together with the object of sense. Now we give the senses names like these: sight and hearing and smell, and the sense of cold and of heat, and pleasures and pains and desires and fears and so forth. Those that have names are very numerous, and those that are unnamed are innumerable. Now the class of objects of sense is akin to each of these; all sorts of colors are akin to all sorts of acts of vision, and in the same way sounds to acts of hearing, \stephpag{c} and the other objects of sense spring forth akin to the other senses. What does this tale mean for us, Theaetetus, with reference to what was said before? Do you see?
\theaetetusspeaks
Not quite, Socrates.
\socratesspeaks
Just listen; perhaps we can finish the tale. It means, of course, that all these things are, as we were saying, in motion, and their motion has in it either swiftness or slowness. Now the slow element keeps its motion in the same place and directed towards such things as draw near it, and indeed it is in this way that it begets. \stephpag{d} But the things begotten in this way are quicker; for they move from one place to another, and their motion is naturally from one place to another. Now when the eye and some appropriate object which approaches beget whiteness and the corresponding perception—which could never have been produced by either of them going to anything else—then, while sight from the eye and \stephpag{e} whiteness from that which helps to produce the color are moving from one to the other, the eye becomes full of sight and so begins at that moment to see, and becomes, certainly not sight, but a seeing eye, and the object which joined in begetting the color is filled with whiteness and becomes in its turn, not whiteness, but white, whether it be a stick or a stone, or whatever it be the hue of which is so colored. And all the rest—hard and hot and so forth—must be regarded in the same way: we must assume, \stephpag{157 a} we said before, that nothing exists in itself, but all things of all sorts arise out of motion by intercourse with each other; for it is, as they say, impossible to form a firm conception of the active or the passive element as being anything separately; for there is no active element until there is a union with the passive element, nor is there a passive element until there is a union with the active; and that which unites with one thing is active and appears again as passive when it comes in contact with something else. And so it results from all this, as we said in the beginning, that nothing exists as invariably one, itself by itself, but everything is always becoming in relation to something, and ``being" should be altogether abolished, \stephpag{b} though we have often—and even just now—been compelled by custom and ignorance to use the word. But we ought not, the wise men say, to permit the use of ``something" or ``somebody's" or ``mine" or ``this" or ``that" or any other word that implies making things stand still, but in accordance with nature we should speak of things as ``becoming" and ``being made" and ``being destroyed" and ``changing"; for anyone who by his mode of speech makes things stand still is easily refuted. And we must use such expressions in relation both to particular objects and collective designations, among which are \stephpag{c} ``mankind" and ``stone" and the names of every animal and class. Do these doctrines seem pleasant to you, Theaetetus, and do you find their taste agreeable?
\theaetetusspeaks
I don't know, Socrates; besides, I can't tell about you, either, whether you are preaching them because you believe them or to test me.
\socratesspeaks
You forget, my friend, that I myself know nothing about such things, and claim none of them as mine, but am incapable of bearing them and am merely acting as a midwife to you, and for that reason am uttering incantations and giving you a taste of each of the philosophical theories, \stephpag{d} until I may help to bring your own opinion to light. And when it is brought to light, I will examine it and see whether it is a mere wind-egg or a real offspring. So be brave and patient, and in good and manly fashion tell what you think in reply to my questions.
\theaetetusspeaks
Very well; ask them.
\socratesspeaks
Then say once more whether the doctrine pleases you that nothing is, but is always becoming—good or beautiful or any of the other qualities we were just enumerating.
\theaetetusspeaks
Why, when I hear you telling about it as you did, it seems to me that it is wonderfully reasonable and ought to be accepted as you have presented it. \stephpag{e}
\socratesspeaks
Let us, then, not neglect a point in which it is defective. The defect is found in connection with dreams and diseases, including insanity, and everything else that is said to cause illusions of sight and hearing and the other senses. For of course you know that in all these the doctrine we were just presenting seems admittedly to be refuted, because \stephpag{158 a} in them we certainly have false perceptions, and it is by no means true that everything is to each man which appears to him; on the contrary, nothing is which appears.
\theaetetusspeaks
What you say is very true, Socrates.
\socratesspeaks
What argument is left, then, my boy, for the man who says that perception is knowledge and that in each case the things which appear are to the one to whom they appear?
\theaetetusspeaks
I hesitate to say, Socrates, that I have no reply to make, because you scolded me just now when I said that. \stephpag{b} But really I cannot dispute that those who are insane or dreaming have false opinions, when some of them think they are gods and others fancy in their sleep that they have wings and are flying.
\socratesspeaks
Don't you remember, either, the similar dispute about these errors, especially about sleeping and waking?
\theaetetusspeaks
What dispute?
\socratesspeaks
One which I fancy you have often heard. The question is asked, what proof you could give if anyone should ask us now, at the present moment, whether we are asleep and our thoughts are a dream, or whether we are awake \stephpag{c} and talking with each other in a waking condition.
\theaetetusspeaks
Really, Socrates, I don't see what proof can be given; for there is an exact correspondence in all particulars, as between the strophe and antistrophe of a choral song. Take, for instance, the conversation we have just had: there is nothing to prevent us from imagining in our sleep also that we are carrying on this conversation with each other, and when in a dream we imagine that we are relating dreams, the likeness between the one talk and the other is remarkable.
\socratesspeaks
So you see it is not hard to dispute the point, since it is even open to dispute whether we are awake or in a dream. \stephpag{d} Now since the time during which we are asleep is equal to that during which we are awake, in each state our spirit contends that the semblances that appear to it at any time are certainly true, so that for half the time we say that this is true, and for half the time the other, and we maintain each with equal confidence.
\theaetetusspeaks
Certainly.
\socratesspeaks
And may not, then, the same be said about insanity and the other diseases, except that the time is not equal?
\theaetetusspeaks
Yes.
\socratesspeaks
Well, then, shall truth be determined by the length or shortness of time? \stephpag{e}
\theaetetusspeaks
That would be absurd in many ways.
\socratesspeaks
But can you show clearly in any other way which of the two sets of opinions is true?
\theaetetusspeaks
I do not think I can.
\socratesspeaks
Listen, then, while I tell you what would be said about them by those who maintain that what appears at any time is true for him to whom it appears. They begin, I imagine, by asking this question: ``Theaetetus, can that which is wholly other have in any way the same quality as its alternative? And we must not assume that the thing in question is partially the same and partially other, but wholly other."
\theaetetusspeaks
It is impossible for it to be the same in anything, either in quality \stephpag{159 a} or in any other respect whatsoever, when it is wholly other.
\socratesspeaks
Must we not, then, necessarily agree that such a thing is also unlike?
\theaetetusspeaks
It seems so to me.
\socratesspeaks
Then if anything happens to become like or unlike anything—either itself or anything else—we shall say that when it becomes like it becomes the same, and when it becomes unlike it becomes other?
\theaetetusspeaks
We must.
\socratesspeaks
Well, we said before, did we not, that the active elements were many—infinite in fact—and likewise the passive elements?
\theaetetusspeaks
Yes.
\socratesspeaks
And furthermore, that any given element, by uniting at different times with different partners, will beget, not the same, but other results? \stephpag{b}
\theaetetusspeaks
Certainly.
\socratesspeaks
Well, then, let us take me, or you, or anything else at hand, and apply the same principle—say Socrates in health and Socrates in illness. Shall we say the one is like the other, or unlike?
\theaetetusspeaks
When you say ``Socrates in illness" do you mean to compare that Socrates as a whole with Socrates in health as a whole?
\socratesspeaks
You understand perfectly; that is just what I mean.
\theaetetusspeaks
Unlike, I imagine.
\socratesspeaks
And therefore other, inasmuch as unlike?
\theaetetusspeaks
Necessarily.
\socratesspeaks
And you would say the same of Socrates asleep or in any of the other states \stephpag{c} we enumerated just now?
\theaetetusspeaks
Yes.
\socratesspeaks
Then each of those elements which by the law of their nature act upon something else, will, when it gets hold of Socrates in health, find me one object to act upon, and when it gets hold of me in illness, another?
\theaetetusspeaks
How can it help it?
\socratesspeaks
And so, in the two cases, that active element and I, who am the passive element, shall each produce a different object?
\theaetetusspeaks
Of course.
\socratesspeaks
So, then, when I am in health and drink wine, it seems pleasant and sweet to me?
\theaetetusspeaks
Yes.
\socratesspeaks
The reason is, in fact, that according to the principles we accepted a while ago, \stephpag{d} the active and passive elements produce sweetness and perception, both of which are simultaneously moving from one place to another, and the perception, which comes from the passive element, makes the tongue perceptive, and the sweetness, which comes from the wine and pervades it, passes over and makes the wine both to be and to seem sweet to the tongue that is in health.
\theaetetusspeaks
Certainly, such are the principles we accepted a while ago.
\socratesspeaks
But when it gets hold of me in illness, in the first place, it really doesn't get hold of the same man, does it? For he to whom it comes is certainly unlike.
\theaetetusspeaks
True. \stephpag{e}
\socratesspeaks
Therefore the union of the Socrates who is ill and the draught of wine produces other results: in the tongue the sensation or perception of bitterness, and in the wine—a bitterness which is engendered there and passes over into the other; the wine is made, not bitterness, but bitter, and I am made, not perception, but perceptive.
\theaetetusspeaks
Certainly.
\socratesspeaks
Then I shall never have this perception of any other thing; for a perception of another thing is another perception, \stephpag{160 a} and makes the percipient different and other: nor can that which acts on me ever by union with another produce the same result or become the same in kind; for by producing another result from another passive element it will become different in kind.
\theaetetusspeaks
That is true.
\socratesspeaks
And neither shall I, furthermore, ever again become the same as I am, nor will that ever become the same as it is.
\theaetetusspeaks
No.
\socratesspeaks
And yet, when I become percipient, I must necessarily become percipient of something, for it is impossible to become percipient and perceive nothing; and that which is perceived must become so to someone, \stephpag{b} when it becomes sweet or bitter or the like; for to become sweet, but sweet to no one, is impossible.
\theaetetusspeaks
Perfectly true.
\socratesspeaks
The result, then, I think, is that we (the active and the passive elements) are or become, whichever is the case, in relation to one another, since we are bound to one another by the inevitable law of our being, but to nothing else, not even to ourselves. The result, then, is that we are bound to one another; and so if a man says anything ``is," he must say it is to or of or in relation to something, and similarly if he says it ``becomes"; but he must not say \stephpag{c} it is or becomes absolutely, nor can he accept such a statement from anyone else. That is the meaning of the doctrine we have been describing.
\theaetetusspeaks
Yes, quite so, Socrates.
\socratesspeaks
Then, since that which acts on me is to me and to me only, it is also the case that I perceive it, and I only?
\theaetetusspeaks
Of course.
\socratesspeaks
Then to me my perception is true; for in each case it is always part of my being; and I am, as Protagoras says, the judge of the existence of the things that are to me and of the non-existence of those that are not to me.
\theaetetusspeaks
So it seems. \stephpag{d}
\socratesspeaks
How, then, if I am an infallible judge and my mind never stumbles in regard to the things that are or that become, can I fail to know that which I perceive?
\theaetetusspeaks
You cannot possibly fail.
\socratesspeaks
Therefore you were quite right in saying that knowledge is nothing else than perception, and there is complete identity between the doctrine of Homer and Heracleitus and all their followers—that all things are in motion, like streams—the doctrine of the great philosopher Protagoras that man is the measure of all things—and the doctrine of Theaetetus that, \stephpag{e} since these things are true, perception is knowledge. Eh, Theaetetus? Shall we say that this is, so to speak, your new-born child and the result of my midwifery? Or what shall we say?
\theaetetusspeaks
We must say that, Socrates.
\socratesspeaks
Well, we have at last managed to bring this forth, whatever it turns out to be; and now that it is born, we must in very truth perform the rite of running round with it in a circle—\footnote{The rite called \emph{amphidromia} took place a few days after the birth of a child. After some ceremonies of purification the nurse, in the presence of the family, carried the infant rapidly about the family hearth, thereby introducing him, as it were, to the family and the family deities. At this time the father decided whether to bring up the child or to expose it. Sometimes, perhaps, the child was named on this occasion. In the evening relatives assembled for a feast at which shell-fish were eaten.} the circle of our argument—and see whether it may not turn out to be after all not worth rearing, but only a wind-egg, \stephpag{161 a} an imposture. But, perhaps, you think that any offspring of yours ought to be cared for and not put away; or will you bear to see it examined and not get angry if it is taken away from you, though it is your first-born?
\theodorusspeaks
Theaetetus will bear it, Socrates, for he is not at all ill-tempered. But for heaven's sake, Socrates, tell me, is all this wrong after all?
\socratesspeaks
You are truly fond of argument, Theodorus, and a very good fellow to think that I am a sort of bag full of arguments and can easily pull one out and say that after all the other one was wrong; \stephpag{b} but you do not understand what is going on: none of the arguments comes from me, but always from him who is talking with me. I myself know nothing, except just a little, enough to extract an argument from another man who is wise and to receive it fairly. And now I will try to extract this thought from Theaetetus, but not to say anything myself.
\theodorusspeaks
That is the better way, Socrates; do as you say.
\socratesspeaks
Do you know, then, Theodorus, what amazes me in your friend Protagoras? \stephpag{c}
\theodorusspeaks
What is it?
\socratesspeaks
In general I like his doctrine that what appears to each one is to him, but I am amazed by the beginning of his book. I don't see why he does not say in the beginning of his \emph{Truth}\footnote{\emph{Truth} was apparently the title, or part of the title, of Protagoras's book.} that a pig or a dog-faced baboon or some still stranger creature of those that have sensations is the measure of all things. Then he might have begun to speak to us very imposingly and condescendingly, showing that while we were honoring him like a god for his wisdom, he was after all no better in intellect than any other man, \stephpag{d} or, for that matter, than a tadpole. What alternative is there, Theodorus? For if that opinion is true to each person which he acquires through sensation, and no one man can discern another's condition better than he himself, and one man has no better right to investigate whether another's opinion is true or false than he himself, but, as we have said several times, each man is to form his own opinions by himself, and these opinions are always right and true, why in the world, my friend, was Protagoras wise, so that he could rightly be thought worthy \stephpag{e} to be the teacher of other men and to be well paid, and why were we ignorant creatures and obliged to go to school to him, if each person is the measure of his own wisdom? Must we not believe that Protagoras was ``playing to the gallery" in saying this? I say nothing of the ridicule that I and my science of midwifery deserve in that case,—and, I should say, the whole practice of dialectics, too. For would not the investigation of one another's fancies and opinions, and the attempt to refute them, when each man's must be right, be tedious \stephpag{162 a} and blatant folly, if the Truth of Protagoras is true and he was not jesting when he uttered his oracles from the shrine of his book?
\theodorusspeaks
Socrates, the man was my friend, as you just remarked. So I should hate to bring about the refutation of Protagoras by agreeing with you, and I should hate also to oppose you contrary to my real convictions. So take Theaetetus again; especially as he seemed just now to follow your suggestions very carefully.
\socratesspeaks
If you went to Sparta, Theodorus, \stephpag{b} and visited the wrestling-schools, would you think it fair to look on at other people naked, some of whom were of poor physique, without stripping and showing your own form, too?
\theodorusspeaks
Why not, if I could persuade them to allow me to do so? So now I think I shall persuade you to let me be a spectator, and not to drag me into the ring, since I am old and stiff, but to take the younger and nimbler man as your antagonist.
\socratesspeaks
Well, Theodorus, if that pleases you, \stephpag{c} it does not displease me, as the saying is. So I must attack the wise Theaetetus again. Tell me, Theaetetus, referring to the doctrine we have just expounded, do you not share my amazement at being suddenly exalted to an equality with the wisest man, or even god? Or do you think Protagoras's ``measure" applies any less to gods than to men?
\theaetetusspeaks
By no means; and I am amazed that you ask such a question at all; for when we were discussing the meaning of the doctrine \stephpag{d} that whatever appears to each one really is to him, I thought it was good; but now it has suddenly changed to the opposite.
\socratesspeaks
You are young, my dear boy; so you are quickly moved and swayed by popular oratory. For in reply to what I have said, Protagoras, or someone speaking for him, will say, ``Excellent boys and old men, there you sit together declaiming to the people, and you bring in the gods, the question of whose \stephpag{e} existence or non-existence I exclude from oral and written discussion, and you say the sort of thing that the crowd would readily accept—that it is a terrible thing if every man is to be no better than any beast in point of wisdom; but you do not advance any cogent proof whatsoever; you base your statements on probability. If Theodorus, or any other geometrician, should base his geometry on probability, he would be of no account at all. So you and Theodorus had better consider whether you will accept arguments founded on plausibility and probabilities in \stephpag{163 a} such important matters."
\theaetetusspeaks
That would not be right, Socrates; neither you nor we would think so.
\socratesspeaks
Apparently, then, you and Theodorus mean we must look at the matter in a different way.
\theaetetusspeaks
Yes, certainly in a different way.
\socratesspeaks
Well, then, let us look at it in this way, raising the question whether knowledge is after all the same as perception, or different. For that is the object of all our discussion, and it was to answer that question than we stirred up all these strange doctrines, was it not?
\theaetetusspeaks
Most assuredly. \stephpag{b}
\socratesspeaks
Shall we then agree that all that we perceive by sight or hearing we know? For instance, shall we say that before having learned the language of foreigners we do not hear them when they speak, or that we both hear and know what they say? And again, if we do not know the letters, shall we maintain that we do not see them when we look at them or that if we really see them we know them?
\theaetetusspeaks
We shall say, Socrates, that we know just so much of them as we hear or see: in the case of the letters, we both see and know the form and color, \stephpag{c} and in the spoken language we both hear and at the same time know the higher and lower notes of the voice; but we do not perceive through sight or hearing, and we do not know, what the grammarians and interpreters teach about them.
\socratesspeaks
First-rate, Theaetetus! and it is a pity to dispute that, for I want you to grow. But look out for another trouble that is yonder coming towards us, and see how we can repel it.
\theaetetusspeaks
What is it?
\socratesspeaks
It is like this: If anyone should ask, ``Is it possible, if a man has ever known a thing and still has \stephpag{d} and preserves a memory of that thing, that he does not, at the time when he remembers, know that very thing which he remembers?" I seem to be pretty long winded; but I merely want to ask if a man who has learned a thing does not know it when he remembers it.
\theaetetusspeaks
Of course he does, Socrates; for what you suggest would be monstrous.
\socratesspeaks
Am I crazy, then? Look here. Do you not say that seeing is perceiving and that sight is perception?
\theaetetusspeaks
I do.
\socratesspeaks
Then, according to what we have just said, the man who has seen a thing has acquired knowledge \stephpag{e} of that which he has seen?
\theaetetusspeaks
Yes.
\socratesspeaks
Well, then, do you not admit that there is such a thing as memory?
\theaetetusspeaks
Yes.
\socratesspeaks
Memory of nothing or of something?
\theaetetusspeaks
Of something, surely.
\socratesspeaks
Of things he has learned and perceived—that sort of things?
\theaetetusspeaks
Of course.
\socratesspeaks
A man sometimes remembers what he has seen, does he not?
\theaetetusspeaks
He does.
\socratesspeaks
Even when he shuts his eyes, or does he forget if he does that?
\theaetetusspeaks
It would be absurd to say that, Socrates. \stephpag{164 a}
\socratesspeaks
We must, though, if we are to maintain our previous argument; otherwise, it is all up with it.
\theaetetusspeaks
I too, by Zeus, have my suspicions, but I don't fully understand you. Tell me how it is.
\socratesspeaks
This is how it is: he who sees has acquired knowledge, we say, of that which he has seen; for it is agreed that sight and perception and knowledge are all the same.
\theaetetusspeaks
Certainly.
\socratesspeaks
But he who has seen and has acquired knowledge of what he saw, if he shuts his eyes, remembers it, but does not see it. Is that right?
\theaetetusspeaks
Yes. \stephpag{b}
\socratesspeaks
But ``does not see" is the same as ``does not know," if it is true that seeing is knowing.
\theaetetusspeaks
True.
\socratesspeaks
Then this is our result. When a man has acquired knowledge of a thing and still remembers it, he does not know it, since he does not see it; but we said that would be a monstrous conclusion.
\theaetetusspeaks
Very true.
\socratesspeaks
So, evidently, we reach an impossible result if we say that knowledge and perception are the same.
\theaetetusspeaks
So it seems.
\socratesspeaks
Then we must say they are different.
\theaetetusspeaks
I suppose so. \stephpag{c}
\socratesspeaks
Then what can knowledge be? We must, apparently, begin our discussion all over again. And yet, Theaetetus, what are we on the point of doing?
\theaetetusspeaks
About what?
\socratesspeaks
It seems to me that we are behaving like a worthless game-cock; before winning the victory we have leapt away from our argument and begun to crow.
\theaetetusspeaks
How so?
\socratesspeaks
We seem to be acting like professional debaters; we have based our agreements on the mere similarity of words and are satisfied to have got the better of the argument in such a way, and we do not see that we, who claim to be, not contestants for a prize, but lovers of wisdom, \stephpag{d} are doing just what those ingenious persons do.
\theaetetusspeaks
I do not yet understand what you mean.
\socratesspeaks
Well, I will try to make my thought clear. We asked, you recollect, whether a man who has learned something and remembers it does not know it. We showed first that the one who has seen and then shuts his eyes remembers, although he does not see, and then we showed that he does not know, although at the same time he remembers; but this, we said, was impossible. And so the Protagorean tale was brought to naught, and yours also about the identity of knowledge and perception. \stephpag{e}
\theaetetusspeaks
Evidently.
\socratesspeaks
It would not be so, I fancy, my friend, if the father of the first of the two tales were alive; he would have had a good deal to say in its defence. But he is dead, and we are abusing the orphan. Why, even the guardians whom Protagoras left—one of whom is Theodorus here—are unwilling to come to the child's assistance. So it seems that we shall have to do it ourselves, assisting him in the name of justice.
\theodorusspeaks
Do so, for it is not I, Socrates, but rather \stephpag{165 a} Callias the son of Hipponicus, who is the guardian of his children. As for me, I turned rather too soon from abstract speculations to geometry. However, I shall be grateful to you if you come to his assistance.
\socratesspeaks
Good, Theodorus! Now see how I shall help him; for a man might find himself involved in still worse inconsistencies than those in which we found ourselves just now, if he did not pay attention to the terms which we generally use in assent and denial. Shall I explain this to you, or only to Theaetetus?
\theodorusspeaks
To both of us, but let the younger answer; \stephpag{b} for he will be less disgraced if he is discomfited.
\socratesspeaks
Very well; now I am going to ask the most frightfully difficult question of all. It runs, I believe, something like this: Is it possible for a person, if he knows a thing, at the same time not to know that which he knows?
\theodorusspeaks
Now, then, what shall we answer, Theaetetus?
\theaetetusspeaks
It is impossible, I should think.
\socratesspeaks
Not if you make seeing and knowing identical. For what will you do with a question from which there is no escape, by which you are, as the saying is, caught in a pit, when your adversary, unabashed, puts his hand over one of your eyes and asks \stephpag{c} if you see his cloak with the eye that is covered?
\theaetetusspeaks
I shall say, I think, ``Not with that eye, but with the other."
\socratesspeaks
Then you see and do not see the same thing at the same time?
\theaetetusspeaks
After a fashion.
\socratesspeaks
``That," he will reply, ``is not at all what I want, and I did not ask about the fashion, but whether you both know and do not know the same thing. Now manifestly you see that which you do not see. But you have agreed that seeing is knowing and not seeing is not knowing. Very well; from all this, reckon out what the result is." \stephpag{d}
\theaetetusspeaks
Well, I reckon out that the result is the contrary of my hypothesis.
\socratesspeaks
And perhaps, my fine fellow, more troubles of the same sort might have come upon you, if anyone asked you further questions—whether it is possible to know the same thing both sharply and dully, to know close at hand but not at a distance, to know both violently and gently, and countless other questions, such as a nimble fighter, fighting for pay in the war of words, might have lain in wait and asked you, when you said that knowledge and perception were the same thing; he would have charged down upon hearing and smelling and such senses, \stephpag{e} and would have argued persistently and unceasingly until you were filled with admiration of his greatly desired wisdom and were taken in his toils, and then, after subduing and binding you he would at once proceed to bargain with you for such ransom as might be agreed upon between you. What argument, then, you might ask, will Protagoras produce to strengthen his forces? Shall we try to carry on the discussion?
\theaetetusspeaks
By all means.
\socratesspeaks
He will, I fancy, say all that we have said in his defence \stephpag{166 a} and then will close with us, saying contemptuously, ``Our estimable Socrates here frightened a little boy by asking if it was possible for one and the same person to remember and at the same time not to know one and the same thing, and when the child in his fright said 'no,' because he could not foresee what would result, Socrates made poor me a laughing-stock in his talk. But, you slovenly Socrates, the facts stand thus: when you examine any doctrine of mine by the method of questioning, if the person who is questioned makes such replies as I should make and comes to grief, then I am refuted, \stephpag{b} but if his replies are quite different, then the person questioned is refuted, not I. Take this example. Do you suppose you could get anybody to admit that the memory a man has of a past feeling he no longer feels is anything like the feeling at the time when he was feeling it? Far from it. Or that he would refuse to admit that it is possible for one and the same person to know and not to know one and the same thing? Or if he were afraid to admit this, would he ever admit that a person who has become unlike is the same as before he became unlike? In fact, if we are to be on our guard against such verbal entanglements, would he admit that a person is one at all, and not many, who become infinite in number, \stephpag{c} if the process of becoming different continues? But, my dear fellow," he will say, ``attack my real doctrines in a more generous manner, and prove, if you can, that perceptions, when they come, or become, to each of us, are not individual, or that, if they are individual, what appears to each one would not, for all that, become to that one alone—or, if you prefer to say 'be,' would not be—to whom it appears. But when you talk of pigs and dog-faced baboons, you not only act like a pig yourself, but you persuade your hearers to act so toward my writings, \stephpag{d} and that is not right. For I maintain that the truth is as I have written; each one of us is the measure of the things that are and those that are not; but each person differs immeasurably from every other in just this, that to one person some things appear and are, and to another person other things. And I do not by any means say that wisdom and the wise man do not exist; on the contrary, I say that if bad things appear and are to any one of us, precisely that man is wise who causes a change and makes good things appear and be to him. \stephpag{e} And, moreover, do not lay too much stress upon the words of my argument, but get a clearer understanding of my meaning from what I am going to say. Recall to your mind what was said before, that his food appears and is bitter to the sick man, but appears and is the opposite of bitter to the man in health. Now neither of these two is to be made wiser than he is—that is not possible— \stephpag{167 a} nor should the claim be made that the sick man is ignorant because his opinions are ignorant, or the healthy man wise because his are different; but a change must be made from the one condition to the other, for the other is better. So, too, in education a change has to be made from a worse to a better condition; but the physician causes the change by means of drugs, and the teacher of wisdom by means of words. And yet, in fact, no one ever made anyone think truly who previously thought falsely, since it is impossible to think that which is not or to think any other things than those which one feels; and these are always true. But I believe that a man who, on account of a bad condition of soul, \stephpag{b} thinks thoughts akin to that condition, is made by a good condition of soul to think correspondingly good thoughts; and some men, through inexperience, call these appearances true, whereas I call them better than the others, but in no wise truer. And the wise, my dear Socrates, I do not by any means call tadpoles when they have to do with the human body, I call them physicians, and when they have to do with plants, husbandmen; for I assert that these latter, when plants are sickly, instil into them good and healthy sensations, \stephpag{c} and true ones instead of bad sensations, and that the wise and good orators make the good, instead of the evil, seem to be right to their states. For I claim that whatever seems right and honorable to a state is really right and honorable to it, so long as it believes it to be so; but the wise man causes the good, instead of that which is evil to them in each instance, to be and seem right and honorable. And on the same principle the teacher who is able to train his pupils in this manner is not only wise but is also \stephpag{d} entitled to receive high pay from them when their education is finished. And in this sense it is true that some men are wiser than others, and that no one thinks falsely, and that you, whether you will or no, must endure to be a measure. Upon these positions my doctrine stands firm; and if you can dispute it in principle, dispute it by bringing an opposing doctrine against it; or if you prefer the method of questions, ask questions; for an intelligent person ought not to reject this method, on the contrary, he should choose it before all others. However, let me make a suggestion: do not be unfair in your questioning; \stephpag{e} it is very inconsistent for a man who asserts that he cares for virtue to be constantly unfair in discussion; and it is unfair in discussion when a man makes no distinction between merely trying to make points and carrying on a real argument. In the former he may jest and try to trip up his opponent as much as he can, but in real argument he must be in earnest and must set his interlocutor on his feet, pointing out to him those slips only which are due to himself and \stephpag{168 a} his previous associations. For if you act in this way, those who debate with you will cast the blame for their confusion and perplexity upon themselves, not upon you; they will run after you and love you, and they will hate themselves and run away from themselves, taking refuge in philosophy, that they may escape from their former selves by becoming different. But if you act in the opposite way, as most teachers do, you will produce the opposite result, and instead of making your young associates philosophers, you will make them hate philosophy \stephpag{b} when they grow older. If therefore, you will accept the suggestion which I made before, you will avoid a hostile and combative attitude and in a gracious spirit will enter the lists with me and inquire what we really mean when we declare that all things are in motion and that whatever seems is to each individual, whether man or state. And on the basis of that you will consider the question whether knowledge and perception are the same or different, instead of doing as you did a while ago, using as your basis \stephpag{c} the ordinary meaning of names and words, which most people pervert in haphazard ways and thereby cause all sorts of perplexity in one another." Such, Theodorus, is the help I have furnished your friend to the best of my ability—not much, for my resources are small; but if he were living himself he would have helped his offspring in a fashion more magnificent.
\theodorusspeaks
You are joking, Socrates, for you have come to the man's assistance with all the valor of youth.
\socratesspeaks
Thank you, my friend. Tell me, did you observe just now that Protagoras reproached us \stephpag{d} for addressing our words to a boy, and said that we made the boy's timidity aid us in our argument against his doctrine, and that he called our procedure a mere display of wit, solemnly insisting upon the importance of ``the measure of all things," and urging us to treat his doctrine seriously?
\theodorusspeaks
Of course I observed it, Socrates.
\socratesspeaks
Well then, shall we do as he says?
\theodorusspeaks
By all means.
\socratesspeaks
Now you see that all those present, except you and myself are boys. So if we are to do as the man asks, you and I must \stephpag{e} question each other and make reply in order to show our serious attitude towards his doctrine; then he cannot, at any rate, find fault with us on the ground that we examined his doctrine in a spirit of levity with mere boys.
\theodorusspeaks
Why is this? Would not Theaetetus follow an investigation better than many a man with a long beard?
\socratesspeaks
Yes, but not better than you, Theodorus. So you must not imagine that I have to defend your deceased friend \stephpag{169 a} by any and every means, while you do nothing at all; but come, my good man, follow the discussion a little way, just until we can see whether, after all, you must be a measure in respect to diagrams, or whether all men are as sufficient unto themselves as you are in astronomy and the other sciences in which you are alleged to be superior.
\theodorusspeaks
It is not easy, Socrates, for anyone to sit beside you and not be forced to give an account of himself and it was foolish of me just now to say you would excuse me and would not oblige me, as the Lacedaemonians do, to strip; you seem to me to take rather after Sciron.\footnote{Sciron was a mighty man who attacked all who came near him and threw them from a cliff. He was overcome by Theseus. Antaeus, a terrible giant, forced all passersby to wrestle with him. He was invincible until Heracles crushed him in his arms.} For the Lacedaemonians \stephpag{b} tell people to go away or else strip, but you seem to me to play rather the role of Antaeus; for you do not let anyone go who approaches you until you have forced him to strip and wrestle with you in argument.
\socratesspeaks
Your comparison with Sciron and Antaeus pictures my complaint admirably; only I am a more stubborn combatant than they; for many a Heracles and many a Theseus, strong men of words, have fallen in with me and belabored me mightily, but still I do not desist, such a terrible love \stephpag{c} of this kind of exercise has taken hold on me. So, now that it is your turn, do not refuse to try a bout with me; it will be good for both of us.
\theodorusspeaks
I say no more. Lead on as you like. Most assuredly I must endure whatsoever fate you spin for me, and submit to interrogation. However, I shall not be able to leave myself in your hands beyond the point you propose.
\socratesspeaks
Even that is enough. And please be especially careful that we do not inadvertently give a playful turn \stephpag{d} to our argument and somebody reproach us again for it.
\theodorusspeaks
Rest assured that I will try so far as in me lies.
\socratesspeaks
Let us, therefore, first take up the same question as before, and let us see whether we were right or wrong in being displeased and finding fault with the doctrine because it made each individual self-sufficient in wisdom. Protagoras granted that some persons excelled others in respect to the better and the worse, and these he said were wise, did he not?
\theodorusspeaks
Yes.
\socratesspeaks
Now if he himself were present and could agree to this, instead of \stephpag{e} our making the concession for him in our effort to help him, there would be no need of taking up the question again or of reinforcing his argument. But, as it is, perhaps it might be said that we have no authority to make the agreement for him; therefore it is better to make the agreement still clearer on this particular point; for it makes a good deal of difference whether it is so or not.
\theodorusspeaks
That is true.
\socratesspeaks
Let us then get the agreement in as concise a form as possible, not through others, \stephpag{170 a} but from his own statement.
\theodorusspeaks
How?
\socratesspeaks
In this way: He says, does he not? ``that which appears to each person really is to him to whom it appears."
\theodorusspeaks
Yes, that is what he says.
\socratesspeaks
Well then, Protagoras, we also utter the opinions of a man, or rather, of all men, and we say that there is no one who does not think himself wiser than others in some respects and others wiser than himself in other respects; for instance, in times of greatest danger, when people are distressed in war or by diseases or at sea, they regard their commanders as gods and expect them to be their saviors, \stephpag{b} though they excel them in nothing except knowledge. And all the world of men is, I dare say, full of people seeking teachers and rulers for themselves and the animals and for human activities, and, on the other hand, of people who consider themselves qualified to teach and qualified to rule. And in all these instances we must say that men themselves believe that wisdom and ignorance exist in the world of men, must we not?
\theodorusspeaks
Yes, we must.
\socratesspeaks
And therefore they think that wisdom is true thinking and ignorance false opinion, do they not? \stephpag{c}
\theodorusspeaks
Of course.
\socratesspeaks
Well then, Protagoras, what shall we do about the doctrine? Shall we say that the opinions which men have are always true, or sometimes true and sometimes false? For the result of either statement is that their opinions are not always true, but may be either true or false. Just think, Theodorus, would any follower of Protagoras, or you yourself care to contend that no person thinks that another is ignorant and has false opinions?
\theodorusspeaks
No, that is incredible, Socrates. \stephpag{d}
\socratesspeaks
And yet this is the predicament to which the doctrine that man is the measure of all things inevitably leads.
\theodorusspeaks
How so?
\socratesspeaks
When you have come to a decision in your own mind about something, and declare your opinion to me, this opinion is, according to his doctrine, true to you; let us grant that; but may not the rest of us sit in judgement on your decision, or do we always judge that your opinion is true? Do not myriads of men on each occasion oppose their opinions to yours, believing that your judgement and belief are false? \stephpag{e}
\theodorusspeaks
Yes, by Zeus, Socrates, countless myriads in truth, as Homer\footnote{Hom. Od. 16.121 Hom. Od.17.432 Hom. Od. 19.78} says, and they give me all the trouble in the world.
\socratesspeaks
Well then, shall we say that in such a case your opinion is true to you but false to the myriads?
\theodorusspeaks
That seems to be the inevitable deduction.
\socratesspeaks
And what of Protagoras himself? If neither he himself thought, nor people in general think, as indeed they do not, that man is the measure of all things, is it not inevitable that the ``truth" which he wrote is true to no one? But if he himself thought it was true, \stephpag{171 a} and people in general do not agree with him, in the first place you know that it is just so much more false than true as the number of those who do not believe it is greater than the number of those who do.
\theodorusspeaks
Necessarily, if it is to be true or false according to each individual opinion.
\socratesspeaks
Secondly, it involves this, which is a very pretty result; he concedes about his own opinion the truth of the opinion of those who disagree with him and think that his opinion is false, since he grants that the opinions of all men are true.
\theodorusspeaks
Certainly. \stephpag{b}
\socratesspeaks
Then would he not be conceding that his own opinion is false, if he grants that the opinion of those who think he is in error is true?
\theodorusspeaks
Necessarily.
\socratesspeaks
But the others do not concede that they are in error, do they?
\theodorusspeaks
No, they do not.
\socratesspeaks
And he, in turn, according to his writings, grants that this opinion also is true.
\theodorusspeaks
Evidently.
\socratesspeaks
Then all men, beginning with Protagoras, will dispute—or rather, he will grant, after he once concedes that the opinion of the man who holds the opposite view is true—even Protagoras himself, I say, \stephpag{c} will concede that neither a dog nor any casual man is a measure of anything whatsoever that he has not learned. Is not that the case?
\theodorusspeaks
Yes.
\socratesspeaks
Then since the ``truth" of Protagoras is disputed by all, it would be true to nobody, neither to anyone else nor to him.
\theodorusspeaks
I think, Socrates, we are running my friend too hard.
\socratesspeaks
But, my dear man, I do not see that we are running beyond what is right. Most likely, though, he, being older, \stephpag{d} is wiser than we, and if, for example, he should emerge from the ground, here at our feet, if only as far as the neck, he would prove abundantly that I was making a fool of myself by my talk, in all probability, and you by agreeing with me; then he would sink down and be off at a run. But we, I suppose, must depend on ourselves, such as we are, and must say just what we think. And so now must we not say that everybody would agree that some men are wiser and some more ignorant than others?
\theodorusspeaks
Yes, I think at least we must.
\socratesspeaks
And do you think his doctrine might stand most firmly in the form in which we sketched it when defending Protagoras, \stephpag{e} that most things—hot, dry, sweet, and everything of that sort—are to each person as they appear to him, and if Protagoras is to concede that there are cases in which one person excels another, he might be willing to say that in matters of health and disease not every woman or child—or beast, for that matter—knows what is wholesome for it and is able to cure itself, but in this point, if in any, one person excels another?
\theodorusspeaks
Yes, I think that is correct. \stephpag{172 a}
\socratesspeaks
And likewise in affairs of state, the honorable and disgraceful, the just and unjust, the pious and its opposite, are in truth to each state such as it thinks they are and as it enacts into law for itself, and in these matters no citizen and no state is wiser than another; but in making laws that are advantageous to the state, or the reverse, Protagoras again will agree that one counsellor is better than another, and the opinion of one state better than that of another as regards the truth, \stephpag{b} and he would by no means dare to affirm that whatsoever laws a state makes in the belief that they will be advantageous to itself are perfectly sure to prove advantageous. But in the other class of things—I mean just and unjust, pious and impious—they are willing to say with confidence that no one of them possesses by nature an existence of its own; on the contrary, that the common opinion becomes true at the time when it is adopted and remains true as long as it is held; this is substantially the theory of those who do not altogether affirm the doctrine of Protagoras. But, Theodorus, argument after argument, \stephpag{c} a greater one after a lesser, is overtaking us.
\theodorusspeaks
Well, Socrates, we have plenty of leisure, have we not?
\socratesspeaks
Apparently we have. And that makes me think, my friend, as I have often done before, how natural it is that those who have spent a long time in the study of philosophy appear ridiculous when they enter the courts of law as speakers.
\theodorusspeaks
What do you mean?
\socratesspeaks
Those who have knocked about in courts and the like from their youth up seem to me, when compared with those who have been brought up in philosophy \stephpag{d} and similar pursuits, to be as slaves in breeding compared with freemen.
\theodorusspeaks
In what way is this the case?
\socratesspeaks
In this way: the latter always have that which you just spoke of, leisure, and they talk at their leisure in peace; just as we are now taking up argument after argument, already beginning a third, so can they, if as in our case, the new one pleases them better than that in which they are engaged; and they do not care at all whether their talk is long or short, if only they attain the truth. But the men of the other sort are always in a hurry—for the water flowing through the water-clock urges them on— \stephpag{e} and the other party in the suit does not permit them to talk about anything they please, but stands over them exercising the law's compulsion by reading the brief, from which no deviation is allowed (this is called the affidavit);\footnote{In Athenian legal procedure each party to a suit presented a written statement—the charge and the reply—at a preliminary hearing. These statements were subsequently confirmed by oath, and the sworn statement was called \textgreek{διωμοσία} or \textgreek{ἀντωμοσία}, which is rendered above by ``affidavit'' as the nearest English equivalent.} and their discourse is always about a fellow slave and is addressed to a master who sits there holding some case or other in his hands; and the contests never run an indefinite course, but are always directed to the point at issue, and often the race is for the defendant's life. \stephpag{173 a} As a result of all this, the speakers become tense and shrewd; they know how to wheedle their master with words and gain his favor by acts; but in their souls they become small and warped. For they have been deprived of growth and straightforwardness and independence by the slavery they have endured from their youth up, for this forces them to do crooked acts by putting a great burden of fears and dangers upon their souls while these are still tender; and since they cannot bear this burden with uprightness and truth, they turn forthwith to deceit and to requiting wrong with wrong, so that they become greatly bent and stunted. \stephpag{b} Consequently they pass from youth to manhood with no soundness of mind in them, but they think they have become clever and wise. So much for them, Theodorus. Shall we describe those who belong to our band, or shall we let that go and return to the argument, in order to avoid abuse of that freedom and variety of discourse, of which we were speaking just now?
\theodorusspeaks
By all means, Socrates, describe them; \stephpag{c} for I like your saying that we who belong to this band are not the servants of our arguments, but the arguments are, as it were, our servants, and each of them must await our pleasure to be finished; for we have neither judge, nor, as the poets have, any spectator set over us to censure and rule us.
\socratesspeaks
Very well, that is quite appropriate, since it is your wish; and let us speak of the leaders; for why should anyone talk about the inferior philosophers? The leaders, in the first place, from their youth up, remain ignorant of the way to the agora, \stephpag{d} do not even know where the court-room is, or the senate-house, or any other public place of assembly; as for laws and decrees, they neither hear the debates upon them nor see them when they are published; and the strivings of political clubs after public offices, and meetings, and banquets, and revellings with chorus girls—it never occurs to them even in their dreams to indulge in such things. And whether anyone in the city is of high or low birth, or what evil has been inherited by anyone from his ancestors, male or female, are matters to which they pay no more attention than to the number of pints in the sea, as the saying is. \stephpag{e} And all these things the philosopher does not even know that he does not know; for he does not keep aloof from them for the sake of gaining reputation, but really it is only his body that has its place and home in the city; his mind, considering all these things petty and of no account, disdains them and is borne in all directions, as Pindar\footnote{This may refer to Pind. Nem. 10.87 f.:\textgreek{ἥμισυ μέν κε πνέοις γαίας ὑπένερθεν ἐών, ἥμισυ δ᾽ οὐρανοῦ ἐν χρυσέοις δόμοσιν} ``Thou (Polydeuces) shalt live being half the time under the earth and half the time in the golden dwellings of heaven,'' but it may be a quotation from one of the lost poems of Pindar.} says, ``both below the earth," and measuring the surface of the earth, and ``above the sky," studying the stars, and investigating the universal nature \stephpag{174 a} of every thing that is, each in its entirety, never lowering itself to anything close at hand.
\theodorusspeaks
What do you mean by this, Socrates?
\socratesspeaks
Why, take the case of Thales, Theodorus. While he was studying the stars and looking upwards, he fell into a pit, and a neat, witty Thracian servant girl jeered at him, they say, because he was so eager to know the things in the sky that he could not see what was there before him at his very feet. The same jest applies to all who pass their lives in philosophy. \stephpag{b} For really such a man pays no attention to his next door neighbor; he is not only ignorant of what he is doing, but he hardly knows whether he is a human being or some other kind of a creature; but what a human being is and what is proper for such a nature to do or bear different from any other, this he inquires and exerts himself to find out. Do you understand, Theodorus, or not?
\theodorusspeaks
Yes, I do; you are right.
\socratesspeaks
Hence it is, my friend, such a man, both in private, when he meets with individuals, and in public, as I said in the beginning, \stephpag{c} when he is obliged to speak in court or elsewhere about the things at his feet and before his eyes, is a laughing-stock not only to Thracian girls but to the multitude in general, for he falls into pits and all sorts of perplexities through inexperience, and his awkwardness is terrible, making him seem a fool; for when it comes to abusing people he has no personal abuse to offer against anyone, because he knows no evil of any man, never having cared for such things; so his perplexity makes him appear ridiculous; and as to laudatory speeches \stephpag{d} and the boastings of others, it becomes manifest that he is laughing at them—not pretending to laugh, but really laughing—and so he is thought to be a fool. When he hears a panegyric of a despot or a king he fancies he is listening to the praises of some herdsman—a swineherd, a shepherd, or a neatherd, for instance—who gets much milk from his beasts; but he thinks that the ruler tends and milks a more perverse and treacherous creature than the herdsmen, and that he must grow coarse and uncivilized, \stephpag{e} no less than they, for he has no leisure and lives surrounded by a wall, as the herdsmen live in their mountain pens. And when he hears that someone is amazingly rich, because he owns ten thousand acres of land or more, to him, accustomed as he is to think of the whole earth, this seems very little. And when people sing the praises of lineage and say someone is of noble birth, because he can show seven wealthy ancestors, he thinks that such praises betray an altogether dull and narrow vision on the part of those who utter them; \stephpag{175 a} because of lack of education they cannot keep their eyes fixed upon the whole and are unable to calculate that every man has had countless thousands of ancestors and progenitors, among whom have been in any instance rich and poor, kings and slaves, barbarians and Greeks. And when people pride themselves on a list of twenty-five ancestors and trace their pedigree back to Heracles, the son of Amphitryon, the pettiness of their ideas seems absurd to him; he laughs at them because they cannot free their silly minds of vanity by calculating that \stephpag{b} Amphitryon's twenty-fifth ancestor was such as fortune happened to make him, and the fiftieth for that matter. In all these cases the philosopher is derided by the common herd, partly because he seems to be contemptuous, partly because he is ignorant of common things and is always in perplexity.
\theodorusspeaks
That all happens just as you say, Socrates.
\socratesspeaks
But when, my friend, \stephpag{c} he draws a man upwards and the other is willing to rise with him above the level of ``What wrong have I done you or you me?" to the investigation of abstract right and wrong, to inquire what each of them is and wherein they differ from each other and from all other things, or above the level of ``Is a king happy?" or, on the other hand, ``Has he great wealth?" to the investigation of royalty and of human happiness and wretchedness in general, to see what the nature of each is and in what way man is naturally fitted to gain the one and escape the other— \stephpag{d} when that man of small and sharp and pettifogging mind is compelled in his turn to give an account of all these things, then the tables are turned; dizzied by the new experience of hanging at such a height, he gazes downward from the air in dismay and perplexity; he stammers and becomes ridiculous, not in the eyes of Thracian girls or other uneducated persons, for they have no perception of it, but in those of all men who have been brought up as free men, not as slaves. Such is the character of each of the two classes, Theodorus, of the man who has truly been brought up in freedom \stephpag{e} and leisure, whom you call a philosopher—who may without censure appear foolish and good for nothing when he is involved in menial services, if, for instance, he does not know how to pack up his bedding, much less to put the proper sweetening into a sauce or a fawning speech—and of the other, who can perform all such services smartly and quickly, but does not know how to wear his cloak as a freeman should, properly draped,\footnote{The Athenians regarded the proper draping of the cloak as a sign of good breeding. The well-bred Athenian first threw his cloak over the left shoulder, then passed it round the back to the right side, then either above or below the right arm, and finally over the left arm or shoulder. See Aristophanes, \emph{Birds}, 1567 f., with Blaydes's notes.} still less to acquire the true harmony of speech \stephpag{176 a} and hymn aright the praises of the true life of gods and blessed men.
\theodorusspeaks
If, Socrates, you could persuade all men of the truth of what you say as you do me, there would be more peace and fewer evils among mankind.
\socratesspeaks
But it is impossible that evils should be done away with, Theodorus, for there must always be something opposed to the good; and they cannot have their place among the gods, but must inevitably hover about mortal nature and this earth. Therefore we ought to try to escape from earth to the dwelling of the gods as quickly as we can; \stephpag{b} and to escape is to become like God, so far as this is possible; and to become like God is to become righteous and holy and wise. But, indeed, my good friend, it is not at all easy to persuade people that the reason generally advanced for the pursuit of virtue and the avoidance of vice—namely, in order that a man may not seem bad and may seem good—is not the reason why the one should be practiced and the other not; that, I think, is merely old wives' chatter, as the saying is. \stephpag{c} Let us give the true reason. God is in no wise and in no manner unrighteous, but utterly and perfectly righteous, and there is nothing so like him as that one of us who in turn becomes most nearly perfect in righteousness. It is herein that the true cleverness of a man is found and also his worthlessness and cowardice; for the knowledge of this is wisdom or true virtue, and ignorance of it is folly or manifest wickedness; and all the other kinds of seeming cleverness and wisdom are paltry when they appear in public affairs and vulgar in the arts. Therefore by far the best thing for the unrighteous man \stephpag{d} and the man whose words or deeds are impious is not to grant that he is clever through knavery; for such men glory in that reproach, and think it means that they are not triflers, ``useless burdens upon the earth,"\footnote{Hom. Il. 18.104; Hom. Od. 20. 379} but such as men should be who are to live safely in a state. So we must tell them the truth—that just because they do not think they are such as they are, they are so all the more truly; for they do not know the penalty of unrighteousness, which is the thing they most ought to know. For it is not what they think it is—scourgings and death, which they sometimes escape entirely when they have done wrong—but a penalty which it is impossible \stephpag{e} to escape.
\theodorusspeaks
What penalty do you mean?
\socratesspeaks
Two patterns, my friend, are set up in the world, the divine, which is most blessed, and the godless, which is most wretched. But these men do not see that this is the case, and their silliness and extreme foolishness blind them to the fact that \stephpag{177 a} through their unrighteous acts they are made like the one and unlike the other. They therefore pay the penalty for this by living a life that conforms to the pattern they resemble; and if we tell them that, unless they depart from their ``cleverness," the blessed place that is pure of all things evil will not receive them after death, and here on earth they will always live the life like themselves—evil men associating with evil—when they hear this, they will be so confident in their unscrupulous cleverness that they will think our words the talk of fools.
\theodorusspeaks
Very true, Socrates. \stephpag{b}
\socratesspeaks
Yes, my friend, I know. However, there is one thing that has happened to them: whenever they have to carry on a personal argument about the doctrines to which they object, if they are willing to stand their ground for a while like men and do not run away like cowards, then, my friend, they at last become strangely dissatisfied with themselves and their arguments; their brilliant rhetoric withers away, so that they seem no better than children. But this is a digression. Let us turn away from these matters—if we do not, \stephpag{c} they will come on like an ever-rising flood and bury in silt our original argument—and let us, if you please, proceed.
\theodorusspeaks
To me, Socrates, such digressions are quite as agreeable as the argument; for they are easier for a man of my age to follow. However, if you prefer, let us return to our argument.
\socratesspeaks
Very well. We were at about the point in our argument where we said that those who declare that only motion is reality, and that whatever seems to each man really is to him to whom it seems, are willing to maintain their position in regard to other matters \stephpag{d} and to maintain especially in regard to justice that whatever laws a state makes, because they seem to it just, are just to the state that made them, as long as they remain in force; but as regards the good, that nobody has the courage to go on and contend that whatever laws a state passes thinking them advantageous to it are really advantageous as long as they remain in force, unless what he means is merely the name ``advantageous"\footnote{The legislator may call his laws advantageous, and that name, if it is given them when they are enacted, will belong to them, whatever their character may be.}; and that would be making a joke of our argument. Am I right?
\theodorusspeaks
Certainly. \stephpag{e}
\socratesspeaks
Yes; for he must not mean merely the name, but the thing named must be the object of his attention.
\theodorusspeaks
True.
\socratesspeaks
But the state, in making laws, aims, of course, at advantage, whatever the name it gives it, and makes all its laws as advantageous as possible to itself, to the extent of its belief and ability; or has it in making laws anything else in view? \stephpag{178 a}
\theodorusspeaks
Certainly not.
\socratesspeaks
And does it always hit the mark, or does every state often miss it?
\theodorusspeaks
I should say they do often miss it!
\socratesspeaks
Continuing, then, and proceeding from this point, every one would more readily agree to this assertion, if the question were asked concerning the whole class to which the advantageous belongs; and that whole class, it would seem, pertains to the future. For when we make laws, we make them with the idea that they will be advantageous in after time; and this is rightly called the future. \stephpag{b}
\theodorusspeaks
Certainly.
\socratesspeaks
Come then, on this assumption, let us question Protagoras or someone of those who agree with him. Man is the measure of all things, as your school says, Protagoras, of the white, the heavy, the light, everything of that sort without exception; for he possesses within himself the standard by which to judge them, and when his thoughts about them coincide with his sensations, he thinks what to him is true and really is. Is not that what they say?
\theodorusspeaks
Yes.
\socratesspeaks
Does he, then, also, Protagoras, we shall say, possess within himself the standard by which to judge of the things which are yet to be, and do those things \stephpag{c} which he thinks will be actually come to pass for him who thought them? Take, for instance, heat; if some ordinary man thinks he is going to take a fever, that is to say, that this particular heat will be, and some other man, who is a physician, thinks the contrary, whose opinion shall we expect the future to prove right? Or perhaps the opinion of both, and the man will become, not hot or feverish to the physician, but to himself both?
\theodorusspeaks
No, that would be ridiculous.
\socratesspeaks
But, I imagine, in regard to the sweetness or dryness \stephpag{d} which will be in a wine, the opinion of the husbandman, not that of the lyre-player, will be valid.
\theodorusspeaks
Of course.
\socratesspeaks
And again, in a matter of discord or tunefulness in music that has never been played, a gymnastic teacher could not judge better than a musician what will, when performed, seem tuneful even to a gymnastic teacher himself.
\theodorusspeaks
Certainly not.
\socratesspeaks
Then, too, when a banquet is in preparation the opinion of him who is to be a guest, unless he has training in cookery, is of less value concerning the pleasure that will be derived from the viands than that of the cook. \stephpag{e} For we need not yet argue about that which already is or has been pleasant to each one but concerning that which will in the future seem and be pleasant to each one, is he himself the best judge for himself, or would you, Protagoras—at least as regards the arguments which will be persuasive in court to each of us—be able to give an opinion beforehand better than anyone whatsoever who has no especial training?
\theodorusspeaks
Certainly, Socrates, in this, at any rate, he used to declare emphatically that he himself excelled everyone.
\socratesspeaks
Yes, my friend, he certainly did; otherwise nobody would have paid him a high fee \stephpag{179 a} for his conversations, if he had not made his pupils believe that neither a prophet nor anyone else could judge better than himself what was in the future to be and seem.
\theodorusspeaks
Very true.
\socratesspeaks
Both lawmaking, then, and the advantageous are concerned with the future, and everyone would agree that a state in making laws must often fail to attain the greatest advantage?
\theodorusspeaks
Assuredly.
\socratesspeaks
Then it will be a fair answer if we say to your master \stephpag{b} that he is obliged to agree that one man is wiser than another, and that such a wise man is a measure, but that I, who am without knowledge, am not in the least obliged to become a measure, as the argument in his behalf just now tried to oblige me to be, whether I would or no.
\theodorusspeaks
In that respect, Socrates, I think that the argument is most clearly proved to be wrong, and it is proved wrong in this also, in that it declares the opinions of others to be valid, whereas it was shown that they do not consider his arguments true at all. \stephpag{c}
\socratesspeaks
In many other respects, Theodorus, it could be proved that not every opinion of every person is true, at any rate in matters of that kind; but it is more difficult to prove that opinions are not true in regard to the momentary states of feeling of each person, from which our perceptions and the opinions concerning them arise. But perhaps I am quite wrong; for it may be impossible to prove that they are not true, and those who say that they are manifest and are forms of knowledge may perhaps be right, and Theaetetus here was not far from the mark in saying that perception and knowledge are identical. \stephpag{d} So we must, as the argument in behalf of Protagoras\footnote{See 168 b.} enjoined upon us, come up closer and examine this doctrine of motion as the fundamental essence, rapping on it to see whether it rings sound or unsound. As you know, a strife has arisen about it, no mean one, either, and waged by not a few combatants.
\theodorusspeaks
Yes, far from mean, and it is spreading far and wide all over Ionia; for the disciples of Heracleitus are supporting this doctrine very vigorously.
\socratesspeaks
Therefore, my dear Theodorus, we must all the more examine it \stephpag{e} from the beginning as they themselves present it.
\theodorusspeaks
Certainly we must. For it is no more possible, Socrates, to discuss these doctrines of Heracleitus (or, as you say, of Homer or even earlier sages) with the Ephesians themselves—those, at least, who profess to be familiar with them—than with madmen. For they are, quite in accordance with their text-books, in perpetual motion; but as for keeping to an argument or a question and quietly answering and asking in turn, \stephpag{180 a} their power of doing that is less than nothing; or rather the words ``nothing at all" fail to express the absence from these fellows of even the slightest particle of rest. But if you ask one of them a question, he pulls out puzzling little phrases, like arrows from a quiver, and shoots them off; and if you try to get hold of an explanation of what he has said, you will be struck with another phrase of novel and distorted wording, and you never make any progress whatsoever with any of them, nor do they themselves with one another, for that matter, \stephpag{b} but they take very good care to allow nothing to be settled either in an argument or in their own minds, thinking, I suppose, that this is being stationary; but they wage bitter war against the stationary, and, so far as they can, they banish it altogether.
\socratesspeaks
Perhaps, Theodorus, you have seen the men when they are fighting, but have not been with them when they are at peace; for they are no friends of yours; but I fancy they utter such peaceful doctrines at leisure to those pupils whom they wish to make like themselves.
\theodorusspeaks
What pupils, my good man? Such people do not become \stephpag{c} pupils of one another, but they grow up of themselves, each one getting his inspiration from any chance source, and each thinks the other knows nothing. From these people, then, as I was going to say, you would never get an argument either with their will or against it; but we must ourselves take over the question and investigate it as if it were a problem of mathematics.
\socratesspeaks
Yes, what you say is reasonable. Now as for the problem, have we not heard from the ancients, who concealed their meaning from the multitude \stephpag{d} by their poetry, that the origin of all things is Oceanus and Tethys, flowing streams, and that nothing is at rest and likewise from the modern, who, since they are wiser, declare their meaning openly, in order that even cobblers may hear and know their wisdom and may cease from the silly belief that some things are at rest and others in motion, and, after learning that everything is in motion, may honor their teachers? But, Theodorus, I almost forgot that others teach the opposite of this, \stephpag{e} 
\begin{displayquote}So that it is motionless, the name of which is the All,\\
---\emph{Parmenides}, line 98 (ed. Mullach)\footnote{In its context the infinitive is necessary; but Plato may have quoted carelessly and may have used the indicative.}
\end{displayquote}
 and all the other doctrines maintained by Melissus and Parmenides and the rest, in opposition to all these they maintain that everything is one and is stationary within itself, having no place in which to move. What shall we do with all these people, my friend? For, advancing little by little, we have unwittingly fallen between the two parties, and, \stephpag{181 a} unless we protect ourselves and escape somehow, we shall pay the penalty, like those in the palaestra, who in playing on the line are caught by both sides and dragged in opposite directions.\footnote{In the game referred to (called \textgreek{διελκυστίνδα} by Pollux, ix. 112) the players were divided into two parties, each of which tried to drag its opponents over a line drawn across the palaestra.} I think, then, we had better examine first the one party, those whom we originally set out to join, the flowing ones, and if we find their arguments sound, we will help them to pull us over, trying thus to escape the others; but if we find that the partisans of ``the whole" seem to have truer doctrines, we will take refuge with them from those who would move what is motionless. \stephpag{b} But if we find that neither party has anything reasonable to say, we shall be ridiculous if we think that we, who are of no account, can say anything worth while after having rejected the doctrines of very ancient and very wise men. Therefore, Theodorus, see whether it is desirable to go forward into so great a danger.
\theodorusspeaks
Oh, it would be unendurable, Socrates, not to examine thoroughly the doctrines of both parties.
\socratesspeaks
Then they must be examined, since you are so urgent. Now I think the starting-point of our examination of the doctrine of motion is this: \stephpag{c} Exactly what do they mean, after all, when they say that all things are in motion? What I wish to ask is this: Do they mean to say that there is only one kind of motion or, as I believe, two? But it must not be my belief alone; you must share it also, that if anything happens to us we may suffer it in common. Tell me, do you call it motion when a thing changes its place or turns round in the same place?
\theodorusspeaks
Yes.
\socratesspeaks
Let this, then, be one kind of motion. Now when a thing \stephpag{d} remains in the same place, but grows old, or becomes black instead of white, or hard instead of soft, or undergoes any other kind of alteration, is it not proper to say that this is another kind of motion?
\theodorusspeaks
I think so.
\socratesspeaks
Nay, it must be true. So I say that there are these two kinds of motion: ``alteration," and ``motion in space."
\theodorusspeaks
And you are right.
\socratesspeaks
Now that we have made this distinction, let us at once converse with those who say that all things are in motion, and let us ask them, ``Do you mean that everything moves in both ways, \stephpag{e} moving in space and undergoing alteration, or one thing in both ways and another in one of the two ways only?"
\theodorusspeaks
By Zeus, I cannot tell! But I think they would say that everything moves in both ways.
\socratesspeaks
Yes; otherwise, my friend, they will find that things in motion are also things at rest, and it will be no more correct to say that all things are in motion than that all things are at rest.
\theodorusspeaks
What you say is very true.
\socratesspeaks
Then since they must be in motion, and since absence of motion must be impossible for anything, all things are \stephpag{182 a} always in all kinds of motion.
\theodorusspeaks
Necessarily.
\socratesspeaks
Then just examine this point of their doctrine. Did we not find that they say that heat or whiteness or anything you please arises in some such way as this, namely that each of these moves simultaneously with perception between the active and the passive element, and the passive becomes percipient, but not perception, and the active becomes, not a quality, but endowed with a quality? Now perhaps quality seems an extraordinary word, and you do not understand it when used with general application, so let me give particular examples. \stephpag{b} For the active element becomes neither heat nor whiteness, but hot or white, and other things in the same way; you probably remember that this was what we said earlier in our discourse, that nothing is in itself unvaryingly one, neither the active nor the passive, but from the union of the two with one another the perceptions and the perceived give birth and the latter become things endowed with some quality while the former become percipient.
\theodorusspeaks
I remember, of course.
\socratesspeaks
Let us then pay no attention to other matters, whether they teach \stephpag{c} one thing or another; but let us attend strictly to this only, which is the object of our discussion. Let us ask them, ``Are all things, according to your doctrine, in motion and flux?" Is that so?
\theodorusspeaks
Yes.
\socratesspeaks
Have they then both kinds of motion which we distinguished? Are they moving in space and also undergoing alteration?
\theodorusspeaks
Of course; that is, if they are to be in perfect motion.
\socratesspeaks
Then if they moved only in space, but did not undergo alteration, we could perhaps say what qualities belong to those moving things which are in flux, could we not?
\theodorusspeaks
That is right. \stephpag{d}
\socratesspeaks
But since not even this remains fixed—that the thing in flux flows white, but changes, so that there is a flux of the very whiteness, and a change of color, that it may not in that way be convicted of remaining fixed, is it possible to give any name to a color, and yet to speak accurately?
\theodorusspeaks
How can it be possible, Socrates, or to give a name to anything else of this sort, if while we are speaking it always evades us, being, as it is, in flux?
\socratesspeaks
But what shall we say of any of the perceptions, such as seeing or hearing? Does it perhaps remain fixed in the condition of \stephpag{e} seeing or hearing?
\theodorusspeaks
It must be impossible, if all things are in motion.
\socratesspeaks
Then we must not speak of seeing more than not seeing, or of any other perception more than of non-perception, if all things are in all kinds of motion.
\theodorusspeaks
No, we must not.
\socratesspeaks
And yet perception is knowledge, as Theaetetus and I said.
\theodorusspeaks
Yes, you did say that.
\socratesspeaks
Then when we were asked ``what is knowledge?" we answered no more what knowledge is than what not-knowledge is. \stephpag{183 a}
\theodorusspeaks
So it seems.
\socratesspeaks
This would be a fine result of the correction of our answer, when we were so eager to show that all things are in motion, just for the purpose of making that answer prove to be correct. But this, I think, did prove to be true, that if all things are in motion, every answer to any question whatsoever is equally correct, and we may say it is thus or not thus—or, if you prefer, ``becomes thus," to avoid giving them fixity by using the word ``is."
\theodorusspeaks
You are right.
\socratesspeaks
Except, Theodorus, that I said ``thus," and ``not thus"; but we ought not even to say ``thus"; \stephpag{b} for ``thus" would no longer be in motion; nor, again, ``not thus." For there is no motion in ``this" either; but some other expression must be supplied for those who maintain this doctrine, since now they have, according to their own hypothesis, no words, unless it be perhaps the word ``nohow." That might be most fitting for them, since it is indefinite.
\theodorusspeaks
At any rate that is the most appropriate form of speech for them.
\socratesspeaks
So, Theodorus, we have got rid of your friend, and we do not yet concede to him that every man is a measure of all things, unless he be a sensible man; \stephpag{c} and we are not going to concede that knowledge is perception, at least not by the theory of universal motion, unless Theaetetus here has something different to say.
\theodorusspeaks
An excellent idea, Socrates; for now that this matter is settled, I too should be rid of the duty of answering your questions according to our agreement, since the argument about Protagoras is ended.
\theaetetusspeaks
No, Theodorus, not until you and Socrates \stephpag{d} have discussed those who say all things are at rest, as you proposed just now.
\theodorusspeaks
A young man like you, Theaetetus, teaching your elders to do wrong by breaking their agreements! No; prepare to answer Socrates yourself for the rest of the argument.
\theaetetusspeaks
I will if he wishes it. But I should have liked best to hear about the doctrine I mentioned.
\theodorusspeaks
Calling Socrates to an argument is calling cavalry into an open plain.\footnote{A proverbial expression. An open plain is just what cavalry desires.} Just ask him a question and you shall hear.
\socratesspeaks
Still I think, Theodorus, \stephpag{e} I shall not comply with the request of Theaetetus.
\theodorusspeaks
Why will you not comply with it?
\socratesspeaks
Because I have a reverential fear of examining in a flippant manner Melissus and the others who teach that the universe is one and motionless, and because I reverence still more one man, Parmenides. Parmenides seems to me to be, in Homer's words, ``one to be venerated" and also ``awful."\footnote{Il. 3.172; Od. 8.22; xiv. 234} For I met him when I was very young and he was very old, and he appeared to me to possess an absolutely noble depth of mind. \stephpag{184 a} So I am afraid we may not understand his words and may be still farther from understanding what he meant by them; but my chief fear is that the question with which we started, about the nature of knowledge, may fail to be investigated, because of the disorderly crowd of arguments which will burst in upon us if we let them in; especially as the argument we are now proposing is of vast extent, and would not receive its deserts if we treated it as a side issue, and if we treat it as it deserves, it will take so long as to do away with the discussion about knowledge. Neither of these things ought to happen, but we ought to try by the science of midwifery to deliver Theaetetus of the thoughts \stephpag{b} about knowledge with which he is pregnant.
\theodorusspeaks
Yes, if that is your opinion, we ought to do so.
\socratesspeaks
Consider, then, Theaetetus, this further point about what has been said. Now you answered that perception is knowledge, did you not?
\theaetetusspeaks
Yes.
\socratesspeaks
If, then, anyone should ask you, ``By what does a man see white and black colors and by what does he hear high and low tones?" you would, I fancy, say, ``By his eyes and ears."
\theaetetusspeaks
Yes, I should. \stephpag{c}
\socratesspeaks
The easy use of words and phrases and the avoidance of strict precision is in general a sign of good breeding; indeed, the opposite is hardly worthy of a gentleman, but sometimes it is necessary, as now it is necessary to object to your answer, in so far as it is incorrect. Just consider; which answer is more correct, that our eyes are that by which we see or that through which we see, and our ears that by which or that through which we hear?
\theaetetusspeaks
I think, Socrates, we perceive through, rather than by them, in each case. \stephpag{d}
\socratesspeaks
Yes, for it would be strange indeed, my boy, if there are many senses ensconced within us, as if we were so many wooden horses of Troy, and they do not all unite in one power, whether we should call it soul or something else, by which we perceive through these as instruments the objects of perception.
\theaetetusspeaks
I think what you suggest is more likely than the other way.
\socratesspeaks
Now the reason why I am so precise about the matter is this: I want to know whether there is some one and the same power within ourselves by which we perceive black and white through the eyes, and again other qualities \stephpag{e} through the other organs, and whether you will be able, if asked, to refer all such activities to the body. But perhaps it is better that you make the statement in answer to a question than that I should take all the trouble for you. So tell me: do you not think that all the organs through which you perceive hot and hard and light and sweet are parts of the body? Or are they parts of something else?
\theaetetusspeaks
Of nothing else.
\socratesspeaks
And will you also be ready to agree that it is impossible to perceive through one sense \stephpag{185 a} what you perceive through another; for instance, to perceive through sight what you perceive through hearing, or through hearing what you perceive through sight?
\theaetetusspeaks
Of course I shall.
\socratesspeaks
Then if you have any thought about both of these together, you would not have perception about both together either through one organ or through the other.
\theaetetusspeaks
No.
\socratesspeaks
Now in regard to sound and color, you have, in the first place, this thought about both of them, that they both exist?
\theaetetusspeaks
Certainly.
\socratesspeaks
And that each is different from the other and the same as itself? \stephpag{b}
\theaetetusspeaks
Of course.
\socratesspeaks
And that both together are two and each separately is one?
\theaetetusspeaks
Yes, that also.
\socratesspeaks
And are you able also to observe whether they are like or unlike each other?
\theaetetusspeaks
May be.
\socratesspeaks
Now through what organ do you think all this about them? For it is impossible to grasp that which is common to them both either through hearing or through sight. Here is further evidence for the point I am trying to make: if it were possible to investigate the question whether the two, sound and color, are bitter or not, you know that you will be able to tell by what faculty you will investigate it, and that is clearly \stephpag{c} neither hearing nor sight, but something else.
\theaetetusspeaks
Of course it is,—the faculty exerted through the tongue.
\socratesspeaks
Very good. But through what organ is the faculty exerted which makes known to you that which is common to all things, as well as to these of which we are speaking—that which you call being and not-being, and the other attributes of things, about which we were asking just now? What organs will you assign for all these, through which that part of us which perceives gains perception of each and all of them?
\theaetetusspeaks
You mean being and not-being, and likeness and unlikeness, and identity and difference, \stephpag{d} and also unity and plurality as applied to them. And you are evidently asking also through what bodily organs we perceive by our soul the odd and the even and everything else that is in the same category.
\socratesspeaks
Bravo, Theaetetus! you follow me exactly; that is just what I mean by my question.
\theaetetusspeaks
By Zeus, Socrates, I cannot answer, except that I think there is no special organ at all for these notions, as there are for those others; but it appears to me that the soul views by itself directly \stephpag{e} what all things have in common.
\socratesspeaks
Why, you are beautiful, Theaetetus, and not, as Theodorus said, ugly; for he who speaks beautifully is beautiful and good. But besides being beautiful, you have done me a favor by relieving me from a long discussion, if you think that the soul views some things by itself directly and others through the bodily faculties; for that was my own opinion, and I wanted you to agree. \stephpag{186 a}
\theaetetusspeaks
Well, I do think so.
\socratesspeaks
To which class, then, do you assign being; for this, more than anything else, belongs to all things?
\theaetetusspeaks
I assign them to the class of notions which the soul grasps by itself directly.
\socratesspeaks
And also likeness and unlikeness and identity and difference?
\theaetetusspeaks
Yes.
\socratesspeaks
And how about beautiful and ugly, and good and bad?
\theaetetusspeaks
I think that these also are among the things the essence of which the soul most certainly views in their relations to one another, reflecting within itself upon the past and present \stephpag{b} in relation to the future.
\socratesspeaks
Stop there. Does it not perceive the hardness of the hard through touch, and likewise the softness of the soft?
\theaetetusspeaks
Yes.
\socratesspeaks
But their essential nature and the fact that they exist, and their opposition to one another, and, in turn, the essential nature of this opposition, the soul itself tries to determine for us by reverting to them and comparing them with one another.
\theaetetusspeaks
Certainly.
\socratesspeaks
Is it not true, then, that all sensations which reach the soul through the body, \stephpag{c} can be perceived by human beings, and also by animals, from the moment of birth; whereas reflections about these, with reference to their being and usefulness, are acquired, if at all, with difficulty and slowly, through many troubles, in other words, through education?
\theaetetusspeaks
Assuredly.
\socratesspeaks
Is it, then, possible for one to attain ``truth" who cannot even get as far as ``being"?
\theaetetusspeaks
No.
\socratesspeaks
And will a man ever have knowledge of anything the truth of which he fails to attain? \stephpag{d}
\theaetetusspeaks
How can he, Socrates?
\socratesspeaks
Then knowledge is not in the sensations, but in the process of reasoning about them; for it is possible, apparently, to apprehend being and truth by reasoning, but not by sensation.
\theaetetusspeaks
So it seems.
\socratesspeaks
Then will you call the two by the same name, when there are so great differences between them?
\theaetetusspeaks
No, that would certainly not be right.
\socratesspeaks
What name will you give, then, to the one which includes seeing, hearing, smelling, being cold, and being hot? \stephpag{e}
\theaetetusspeaks
Perceiving. What other name can I give it?
\socratesspeaks
Collectively you call it, then, perception?
\theaetetusspeaks
Of course.
\socratesspeaks
By which, we say, we are quite unable to apprehend truth, since we cannot apprehend being, either.
\theaetetusspeaks
No; certainly not.
\socratesspeaks
Nor knowledge either, then.
\theaetetusspeaks
No.
\socratesspeaks
Then, Theaetetus, perception and knowledge could never be the same.
\theaetetusspeaks
Evidently not, Socrates; and indeed now at last it has been made perfectly clear that knowledge is something different from perception. \stephpag{187 a}
\socratesspeaks
But surely we did not begin our conversation in order to find out what knowledge is not, but what it is. However, we have progressed so far, at least, as not to seek for knowledge in perception at all, but in some function of the soul, whatever name is given to it when it alone and by itself is engaged directly with realities.
\theaetetusspeaks
That, Socrates, is, I suppose, called having opinion.
\socratesspeaks
You suppose rightly, my friend. Now begin again \stephpag{b} at the beginning. Wipe out all we said before, and see if you have any clearer vision, now that you have advanced to this point. Say once more what knowledge is.
\theaetetusspeaks
To say that all opinion is knowledge is impossible, Socrates, for there is also false opinion; but true opinion probably is knowledge. Let that be my answer. For if it is proved to be wrong as we proceed, I will try to give another, just as I have given this.
\socratesspeaks
That is the right way, Theaetetus. It is better to speak up boldly than to hesitate about answering, as you did at first. For if we act in this way, one of two things will happen: either we shall find what we are after, \stephpag{c} or we shall be less inclined to think we know what we do not know at all; and surely even that would be a recompense not to be despised. Well, then, what do you say now? Assuming that there are two kinds of opinion, one true and the other false, do you define knowledge as the true opinion?
\theaetetusspeaks
Yes. That now seems to me to be correct.
\socratesspeaks
Is it, then, still worth while, in regard to opinion, to take up again—?
\theaetetusspeaks
What point do you refer to?
\socratesspeaks
Somehow I am troubled now and have often been troubled before, \stephpag{d} so that I have been much perplexed in my own reflections and in talking with others, because I cannot tell what this experience is which we human beings have, and how it comes about.
\theaetetusspeaks
What experience?
\socratesspeaks
That anyone has false opinions. And so I am considering and am still in doubt whether we had better let it go or examine it by another method than the one we followed a while ago.
\theaetetusspeaks
Why not, Socrates, if there seems to be the least need of it? For just now, in talking about leisure, you and Theodorus said very truly that there is no hurry in discussions of this sort. \stephpag{e}
\socratesspeaks
You are right in reminding me. For perhaps this is a good time to retrace our steps. For it is better to finish a little task well than a great deal imperfectly.
\theaetetusspeaks
Of course.
\socratesspeaks
How, then, shall we set about it? What is it that we do say? Do we say that in every case of opinion there is a false opinion, and one of us has a false, and another a true opinion, because, as we believe, it is in the nature of things that this should be so?
\theaetetusspeaks
Yes, we do. \stephpag{188 a}
\socratesspeaks
Then this, at any rate, is possible for us, is it not, regarding all things collectively and each thing separately, either to know or not to know them? For learning and forgetting, as intermediate stages, I leave out of account for the present, for just now they have no bearing upon our argument.
\theaetetusspeaks
Certainly, Socrates, nothing is left in any particular case except knowing or not knowing it.
\socratesspeaks
Then he who forms opinion must form opinion either about what he knows or about what he does not know?
\theaetetusspeaks
Necessarily.
\socratesspeaks
And it is surely impossible that one who knows a thing does not know it, or that one who does not know it \stephpag{b} knows it.
\theaetetusspeaks
Certainly.
\socratesspeaks
Then does he who forms false opinions think that the things which he knows are not these things, but some others of the things he knows, and so, knowing both, is he ignorant of both?
\theaetetusspeaks
That is impossible, Socrates.
\socratesspeaks
Well then, does he think that the things he does not know are other things which he does not know—which is as if a man who knows neither Theaetetus nor Socrates should conceive the idea that Socrates is Theaetetus or Theaetetus Socrates? \stephpag{c}
\theaetetusspeaks
That is impossible.
\socratesspeaks
But surely a man does not think that the things he knows are the things he does not know, or again that the things he does not know are the things he knows.
\theaetetusspeaks
That would be a monstrous absurdity.
\socratesspeaks
Then how could he still form false opinions? For inasmuch as all things are either known or unknown to us, it is impossible, I imagine, to form opinions outside of these alternatives, and within them it is clear that there is no place for fake opinion.
\theaetetusspeaks
Very true.
\socratesspeaks
Had we, then, better look for what we are seeking, not by this method of knowing and not knowing, but by that of being \stephpag{d} and not being?
\theaetetusspeaks
What do you mean?
\socratesspeaks
We may simply assert that he who on any subject holds opinions which are not, will certainly think falsely, no matter what the condition of his mind may be in other respects.
\theaetetusspeaks
That, again, is likely, Socrates.
\socratesspeaks
Well then, what shall we say, Theaetetus, if anyone asks us, ``Is that which is assumed in common speech possible at all, and can any human being hold an opinion which is not, whether it be concerned with any of the things which are, or be entirely independent of them?" We, I fancy, \stephpag{e} shall reply, ``Yes, when, in thinking, he thinks what is not true," shall we not?
\theaetetusspeaks
Yes.
\socratesspeaks
And is the same sort of thing possible in any other field?
\theaetetusspeaks
What sort of thing?
\socratesspeaks
For instance, that a man sees something, but sees nothing.
\theaetetusspeaks
How can he?
\socratesspeaks
Yet surely if a man sees any one thing, he sees something that is. Or do you, perhaps, think ``one" is among the things that are not?
\theaetetusspeaks
No, I do not.
\socratesspeaks
Then he who sees any one thing, sees something that is.
\theaetetusspeaks
That is clear. \stephpag{189 a}
\socratesspeaks
And therefore he who hears anything, hears some one thing and therefore hears what is.
\theaetetusspeaks
Yes.
\socratesspeaks
And he who touches anything, touches some one thing, which is, since it is one?
\theaetetusspeaks
That also is true.
\socratesspeaks
So, then, does not he who holds an opinion hold an opinion of some one thing?
\theaetetusspeaks
He must do so.
\socratesspeaks
And does not he who holds an opinion of some one thing hold an opinion of something that is?
\theaetetusspeaks
I agree.
\socratesspeaks
Then he who holds an opinion of what is not holds an opinion of nothing.
\theaetetusspeaks
Evidently.
\socratesspeaks
Well then, he who holds an opinion of nothing, holds no opinion at all.
\theaetetusspeaks
That is plain, apparently. \stephpag{b}
\socratesspeaks
Then it is impossible to hold an opinion of that which is not, either in relation to things that are, or independently of them.
\theaetetusspeaks
Evidently.
\socratesspeaks
Then holding false opinion is something different from holding an opinion of that which is not?
\theaetetusspeaks
So it seems.
\socratesspeaks
Then false opinion is not found to exist in us either by this method or by that which we followed a little while ago.
\theaetetusspeaks
No, it certainly is not.
\socratesspeaks
But does not that which we call by that name arise after the following manner?
\theaetetusspeaks
After what manner?
\socratesspeaks
We say that false opinion is a kind of interchanged opinion, \stephpag{c} when a person makes an exchange in his mind and says that one thing which exists is another thing which exists. For in this way he always holds an opinion of what exists, but of one thing instead of another; so he misses the object he was aiming at in his thought and might fairly be said to hold a false opinion.
\theaetetusspeaks
Now you seem to me to have said what is perfectly right. For when a man, in forming an opinion, puts ugly instead of beautiful, or beautiful instead of ugly, he does truly hold a false opinion.
\socratesspeaks
Evidently, Theaetetus, you feel contempt of me, and not fear.
\theaetetusspeaks
Why in the world do you say that?
\socratesspeaks
You think, I fancy, that I would not attack your ``truly false" \stephpag{d} by asking whether it is possible for a thing to become slowly quick or heavily light, or any other opposite, by a process opposite to itself, in accordance, not with its own nature, but with that of its opposite. But I let this pass, that your courage may not fail. You are satisfied, you say, that false opinion is interchanged opinion?
\theaetetusspeaks
I am.
\socratesspeaks
It is, then, in your opinion, possible for the mind to regard one thing as another and not as what it is.
\theaetetusspeaks
Yes, it is.
\socratesspeaks
Now when one's mind does this, does it not necessarily \stephpag{e} have a thought either of both things together or of one or the other of them?
\theaetetusspeaks
Yes, it must; either of both at the same time or in succession.
\socratesspeaks
Excellent. And do you define thought as I do?
\theaetetusspeaks
How do you define it?
\socratesspeaks
As the talk which the soul has with itself about any subjects which it considers. You must not suppose that I know this that I am declaring to you. But the soul, as the image presents itself to me, when it thinks, is merely conversing with itself, asking itself questions and answering, \stephpag{190 a} affirming and denying. When it has arrived at a decision, whether slowly or with a sudden bound, and is at last agreed, and is not in doubt, we call that its opinion; and so I define forming opinion as talking and opinion as talk which has been held, not with someone else, nor yet aloud, but in silence with oneself. How do you define it?
\theaetetusspeaks
In the same way.
\socratesspeaks
Then whenever a man has an opinion that one thing is another, he says to himself, we believe, that the one thing is the other. \stephpag{b}
\theaetetusspeaks
Certainly.
\socratesspeaks
Now call to mind whether you have ever said to yourself that the beautiful is most assuredly ugly, or the wrong right, or—and this is the sum of the whole matter—consider whether you have ever tried to persuade yourself that one thing is most assuredly another, or whether quite the contrary is the case, and you have never ventured, even in sleep, to say to yourself that the odd is, after all, certainly even, or anything of that sort.
\theaetetusspeaks
You are right. \stephpag{c}
\socratesspeaks
Do you imagine that anyone else, sane or insane, ever ventured to say to himself seriously and try to persuade himself that the ox must necessarily be a horse, or two one?
\theaetetusspeaks
No, by Zeus, I do not.
\socratesspeaks
Then if forming opinion is talking to oneself, no one who talks and forms opinion of two objects and apprehends them both with his soul, could say and have the opinion that one is the other. But you will also have to give up the expression ``one and other." This is what I mean, that nobody holds the opinion that the ugly is beautiful, or \stephpag{d} anything of that sort.
\theaetetusspeaks
Well, Socrates, I do give it up; and I agree with you in what you say.
\socratesspeaks
You agree, therefore, that he who holds an opinion of both things cannot hold the opinion that one is the other.
\theaetetusspeaks
So it seems.
\socratesspeaks
But surely he who holds an opinion of one only, and not of the other at all, will never hold the opinion that one is the other.
\theaetetusspeaks
You are right; for he would be forced to apprehend also that of which he holds no opinion.
\socratesspeaks
Then neither he who holds opinion of both nor he who holds it of one can hold the opinion that a thing is something else. \stephpag{e} And so anyone who sets out to define false opinion as interchanged opinion would be talking nonsense. Then neither by this method nor by our previous methods is false opinion found to exist in us.
\theaetetusspeaks
Apparently not.
\socratesspeaks
But yet, Theaetetus, if this is found not to exist, we shall be forced to admit many absurdities.
\theaetetusspeaks
What absurdities?
\socratesspeaks
I will not tell you until I have tried to consider the matter in every way. For I should be ashamed of us, if, in our perplexity, we were forced to make such admissions as those to which I refer. But if we find the object of our quest, \stephpag{191 a} and are set free from perplexity, then, and not before, we will speak of others as involved in those absurdities, and we ourselves shall stand free from ridicule. But if we find no escape from our perplexity, we shall, I fancy, become low-spirited, like seasick people, and shall allow the argument to trample on us and do to us anything it pleases. Hear, then, by what means I still see a prospect of success for our quest.
\theaetetusspeaks
Do speak.
\socratesspeaks
I shall deny that we were right when we agreed that it is impossible for a man to have opinion that the things he does not know are the things which he knows, \stephpag{b} and thus to be deceived. But there is a way in which it is possible.
\theaetetusspeaks
Do you mean what I myself suspected when we made the statement to which you refer, that sometimes I, though I know Socrates, saw at a distance someone whom I did not know, and thought it was Socrates whom I do know? In such a case false opinion does arise.
\socratesspeaks
But did not we reject that, because it resulted in our knowing and not knowing the things which we know?
\theaetetusspeaks
Certainly we did.
\socratesspeaks
Let us, then, not make that assumption, but another; perhaps \stephpag{c} it will turn out well for us, perhaps the opposite. But we are in such straits that we must turn every argument round and test it from all sides. Now see if this is sensible: Can a man who did not know a thing at one time learn it later?
\theaetetusspeaks
To be sure he can.
\socratesspeaks
Again, then, can he learn one thing after another?
\theaetetusspeaks
Why not?
\socratesspeaks
Please assume, then, for the sake of argument, that there is in our souls a block of wax, in one case larger, in another smaller, in one case the wax is purer, in another more impure and harder, in some cases softer, \stephpag{d} and in some of proper quality.
\theaetetusspeaks
I assume all that.
\socratesspeaks
Let us, then, say that this is the gift of Memory, the mother of the Muses, and that whenever we wish to remember anything we see or hear or think of in our own minds, we hold this wax under the perceptions and thoughts and imprint them upon it, just as we make impressions from seal rings; and whatever is imprinted we remember and know as long as its image lasts, but whatever is rubbed out or \stephpag{e} cannot be imprinted we forget and do not know.
\theaetetusspeaks
Let us assume that.
\socratesspeaks
Now take a man who knows the things which he sees and hears, and is considering some one of them; observe whether he may not gain a false opinion in the following manner.
\theaetetusspeaks
In what manner?
\socratesspeaks
By thinking that the things which he knows are sometimes things which he knows and sometimes things which he does not know. For we were wrong before in agreeing that this is impossible.
\theaetetusspeaks
What do you say about it now? \stephpag{192 a}
\socratesspeaks
We must begin our discussion of the matter by making the following distinctions: It is impossible for anyone to think that one thing which he knows and of which he has received a memorial imprint in his soul, but which he does not perceive, is another thing which he knows and of which also he has an imprint, and which he does not perceive. And, again, he cannot think that what he knows is that which he does not know and of which he has no seal; nor that what he does not know is another thing which he does not know; nor that what he does not know is what he knows; nor can he think that what he perceives is something else which he perceives; nor that what he perceives is something which he does not perceive; nor that what he does not perceive is something else which he does not perceive; nor that what he does not perceive \stephpag{b} is something which he perceives. And, again, it is still more impossible, if that can be, to think that a thing which he knows and perceives and of which he has an imprint which accords with the perception is another thing which he knows and perceives and of which he has an imprint which accords with the perception. And he cannot think that what he knows and perceives and of which he has a correct memorial imprint is another thing which he knows; nor that a thing which he knows and perceives and of which he has such an imprint is another thing which he perceives; \stephpag{c} nor again that a thing which he neither knows nor perceives is another thing which he neither knows nor perceives; nor that a thing which he neither knows nor perceives is another thing which he does not know; nor that a thing which he neither knows nor perceives is another thing which he does not perceive. In all these cases it is impossible beyond everything for false opinion to arise in the mind of anyone. The possibility that it may arise remains, if anywhere, in the following cases.
\theaetetusspeaks
What cases are they? I hope they may help me to understand better; for now I cannot follow you.
\socratesspeaks
The cases in which he may think that things which he knows are some other things which he knows and perceives; or which he does not know, but perceives; or that things which he knows and perceives are other things which \stephpag{d} he knows and perceives.
\theaetetusspeaks
Now I am even more out of the running than before.
\socratesspeaks
Then let me repeat it in a different way. I know Theodorus and remember within myself what sort of a person he is, and just so I know Theaetetus, but sometimes I see them, and sometimes I do not, sometimes I touch them, sometimes not, sometimes I hear them or perceive them through some other sense, and sometimes I have no perception of you at all, but I remember you none the less and know you in my own mind. Is it not so? \stephpag{e}
\theaetetusspeaks
Certainly.
\socratesspeaks
This, then, is the first of the points which I wish to make clear. Note that one may perceive or not perceive that which one knows.
\theaetetusspeaks
That is true.
\socratesspeaks
So, too, with that which he does not know—he may often not even perceive it, and often he may merely perceive it?
\theaetetusspeaks
That too is possible.
\socratesspeaks
See if you follow me better now. If Socrates \stephpag{193 a} knows Theodorus and Theaetetus, but sees neither of them and has no other perception of them, he never could have the opinion within himself that Theaetetus is Theodorus. Am I right or wrong?
\theaetetusspeaks
You are right.
\socratesspeaks
Now that was the first of the cases of which I spoke.
\theaetetusspeaks
Yes, it was.
\socratesspeaks
The second is this: knowing one of you and not knowing the other, and not perceiving either of you, I never could think that the one whom I know is the one whom I do not know.
\theaetetusspeaks
Right.
\socratesspeaks
And this is the third case: not knowing and not perceiving either of you, \stephpag{b} I could not think that he whom I do not know is someone else whom I do not know. And imagine that you have heard all the other cases again in succession, in which I could never form false opinions about you and Theodorus, either when I know or do not know both of you, or when I know one and not the other; and the same is true if we say ``perceive" instead of ``know." Do you follow me?
\theaetetusspeaks
I follow you.
\socratesspeaks
Then the possibility of forming false opinion remains in the following case: when, for example, knowing you and Theodorus, and having on that block of wax \stephpag{c} the imprint of both of you, as if you were signet-rings, but seeing you both at a distance and indistinctly, I hasten to assign the proper imprint of each of you to the proper vision, and to make it fit, as it were, its own footprint, with the purpose of causing recognition;\footnote{Aesch. Lib. 197 ff. makes Electra recognize the presence of her brother Orestes by the likeness of his footprints to her own.} but I may fail in this by interchanging them, and put the vision of one upon the imprint of the other, as people put a shoe on the wrong foot; or, again, I may be affected as the sight is affected when we use a mirror and the sight as it flows makes a change from right to left, \stephpag{d} and thus make a mistake; it is in such cases, then, that interchanged opinion occurs and the forming of false opinion arises.
\theaetetusspeaks
I think it does, Socrates. You describe what happens to opinion marvelously well.
\socratesspeaks
There is still the further case, when, knowing both of you, I perceive one in addition to knowing him, but do not perceive the other, and the knowledge which I have of that other is not in accord with my perception. This is the case I described in this way before, and at that time you did not understand me.
\theaetetusspeaks
No, I did not.
\socratesspeaks
This is what I meant, that if anyone knows \stephpag{e} and perceives one of you, and has knowledge of him which accords with the perception, he will never think that he is someone else whom he knows and perceives and his knowledge of whom accords with the perception. That was the case, was it not?
\theaetetusspeaks
Yes.
\socratesspeaks
But we omitted, I believe, the case of which I am speaking now—the case in which we say the false opinion arises: when a man knows both and sees both (or has some other perception of them), \stephpag{194 a} but fails to hold the two imprints each under its proper perception; like a bad archer he shoots beside the mark and misses it; and it is just this which is called error or deception.
\theaetetusspeaks
And properly so.
\socratesspeaks
Now when perception is present to one of the imprints but not to the other, and the mind applies the imprint of the absent perception to the perception which is present, the mind is deceived in every such instance. In a word, if our present view is sound, false opinion or deception seems to be impossible in relation to things which one does not know \stephpag{b} and has never perceived; but it is precisely in relation to things which we know and perceive that opinion turns and twists, becoming false and true—true when it puts the proper imprints and seals fairly and squarely upon one another, and false when it applies them sideways and aslant.
\theaetetusspeaks
Well, then, Socrates, is that view not a good one? \stephpag{c}
\socratesspeaks
After you have heard the rest, you will be still more inclined to say so. For to hold a true opinion is a good thing, but to be deceived is a disgrace.
\theaetetusspeaks
Certainly.
\socratesspeaks
They say the cause of these variations is as follows: When the wax in the soul of a man is deep and abundant and smooth and properly kneaded, the images that come through the perceptions are imprinted upon this heart of the soul—as Homer calls it in allusion to its similarity to wax\footnote{The similarity is in the Greek words \textgreek{κέαρ} or \textgreek{κῆρ}, ``heart'', and \textgreek{κηρός}, wax. The shaggy heart is mentioned in the Hom. Il. 2.851; Hom. Il. 16.554 The citation of Homer, here and below, is probably sarcastic—in reference to the practice of some of the sophists who used and perverted his words in support of their doctrines.}—; when this is the case, and in such men, the imprints, \stephpag{d} being clear and of sufficient depth, are also lasting. And men of this kind are in the first place quick to learn, and secondly they have retentive memories, and moreover they do not interchange the imprints of their perceptions, but they have true opinions. For the imprints are clear and have plenty of room, so that such men quickly assign them to their several moulds, which are called realities; and these men, then, are called wise. Or do you not agree?
\theaetetusspeaks
Most emphatically. \stephpag{e}
\socratesspeaks
Now when the heart of anyone is shaggy (a condition which the all-wise poet commends), or when it is unclean or of impure wax, or very soft or hard, those whose wax is soft are quick to learn, but forgetful, and those in whom it is hard are the reverse. But those in whom it is shaggy and rough and stony, infected with earth or dung which is mixed in it, receive indistinct imprints from the moulds. So also do those whose wax is hard; for the imprints lack depth. And imprints in soft wax are also indistinct, because \stephpag{195 a} they melt together and quickly become blurred; but if besides all this they are crowded upon one another through lack of room, in some mean little soul, they are still more indistinct. So all these men are likely to have false opinions. For when they see or hear or think of anything, they cannot quickly assign things to the right imprints, but are slow about it, and because they assign them wrongly they usually see and hear and think amiss. These men, in turn, are accordingly said to be deceived about realities and ignorant. \stephpag{b}
\theaetetusspeaks
You are right as right could be, Socrates.
\socratesspeaks
Shall we, then, say that false opinions exist in us?
\theaetetusspeaks
Assuredly.
\socratesspeaks
And true opinions, no doubt?
\theaetetusspeaks
And true ones also.
\socratesspeaks
Then now at last we think we have reached a valid agreement, that these two kinds of opinion incontestably exist?
\theaetetusspeaks
Most emphatically.
\socratesspeaks
Truly, Theaetetus, a garrulous man is a strange and unpleasant creature!
\theaetetusspeaks
Eh? What makes you say that? \stephpag{c}
\socratesspeaks
Vexation at my own stupidity and genuine garrulity. For what else could you call it when a man drags his arguments up and down because he is so stupid that he cannot be convinced, and is hardly to be induced to give up any one of them?
\theaetetusspeaks
But you, why are you vexed?
\socratesspeaks
I am not merely vexed, I am actually afraid; for I do not know what answer to make if anyone asks me: ``Socrates, have you found out, I wonder, that false opinion exists neither in the relations of the perceptions to one another nor in the thoughts, \stephpag{d} but in the combination of perception with thought?" I shall say ``yes," I suppose, and put on airs, as if we had made a fine discovery.
\theaetetusspeaks
It seems to me, Socrates, that the result we have now brought out is not half bad.
\socratesspeaks
``Do you go on and assert, then," he will say, ``that we never could imagine that the man whom we merely think of, but do not see, is a horse which also we do not see or touch or perceive by any other sense, but merely think of?" I suppose I shall say that I do make that assertion.
\theaetetusspeaks
Yes, and you will be right. \stephpag{e}
\socratesspeaks
``Then," he will say, ``according to that, could we ever imagine that the number eleven which is merely thought of, is the number twelve which also is merely thought of?" Come now, it is for you to answer.
\theaetetusspeaks
Well, my answer will be that a man might imagine the eleven that he sees or touches to be twelve, but that he could never have that opinion concerning the eleven that he has in his mind.
\socratesspeaks
Well, then, do you think that anyone ever considered in his own mind five and seven,— \stephpag{196 a} I do not mean by setting before his eyes seven men and five men and considering them, or anything of that sort, but seven and five in the abstract, which we say are imprints in the block of wax, and in regard to which we deny the possibility of forming false opinions—taking these by themselves, do you imagine that anybody in the world has ever considered them, talking to himself and asking himself what their sum is, and that one person has said and thought eleven, and another twelve, or do all say and think that it is twelve?
\theaetetusspeaks
No, by Zeus; many say eleven, \stephpag{b} and if you take a larger number for consideration, there is greater likelihood of error. For I suppose you are speaking of any number rather than of these only.
\socratesspeaks
You are right in supposing so; and consider whether in that instance the abstract twelve in the block of wax is not itself imagined to be eleven.
\theaetetusspeaks
It seems so.
\socratesspeaks
Have we not, then, come back again to the beginning of our talk? For the man who is affected in this way imagines that one thing which he knows is another thing which be knows. This we said was impossible, and \stephpag{c} by this very argument we were forcing false opinion out of existence, that the same man might not be forced to know and not know the same things at the same time.
\theaetetusspeaks
Very true.
\socratesspeaks
Then we must show that forming false opinion is something or other different from the interchange of thought and perception. For if it were that, we should never be deceived in abstract thoughts. But as the case now stands, either there is no false opinion or it is possible for a man not to know that which he knows. Which alternative will you choose?
\theaetetusspeaks
There is no possible choice, Socrates. \stephpag{d}
\socratesspeaks
And yet the argument is not likely to admit both. But still, since we must not shrink from any risk, what if we should try to do a shameless deed?
\theaetetusspeaks
What is it?
\socratesspeaks
To undertake to tell what it really is to know.
\theaetetusspeaks
And why is that shameless?
\socratesspeaks
You seem not to remember that our whole talk from the beginning has been a search for knowledge, because we did not know what it is.
\theaetetusspeaks
Oh yes, I remember.
\socratesspeaks
Then is it not shameless to proclaim what it is to know, when we are ignorant of knowledge? \stephpag{e} But really, Theaetetus, our talk has been badly tainted with unclearness all along; for we have said over and over again ``we know" and ``we do not know" and ``we have knowledge" and ``we have no knowledge," as if we could understand each other, while we were still ignorant of knowledge; and at this very moment, if you please, we have again used the terms ``be ignorant" and ``understand," as though we had any right to use them if we are deprived of knowledge.
\theaetetusspeaks
But how will you converse, Socrates, if you refrain from these words? \stephpag{197 a}
\socratesspeaks
Not at all, being the man I am; but I might if I were a real reasoner; if such a man were present at this moment he would tell us to refrain from these terms, and would criticize my talk scathingly. But since we are poor creatures, shall I venture to say what the nature of knowing is? For it seems to me that would be of some advantage.
\theaetetusspeaks
Venture it then, by Zeus. You shall have full pardon for not refraining from those terms.
\socratesspeaks
Have you heard what they say nowadays that knowing is?
\theaetetusspeaks
Perhaps; however, I don't remember just at this moment. \stephpag{b}
\socratesspeaks
They say it is having knowledge.
\theaetetusspeaks
True.
\socratesspeaks
Let us make a slight change and say possessing knowledge.
\theaetetusspeaks
Why, how will you claim that the one differs from the other?
\socratesspeaks
Perhaps it doesn't; but first hear how it seems to me to differ, and then help me to test my view.
\theaetetusspeaks
I will if I can.
\socratesspeaks
Well, then, having does not seem to me the same as possessing. For instance, if a man bought a cloak and had it under his control, but did not wear it, we should certainly say, not that he had it, but that he possessed it.
\theaetetusspeaks
And rightly. \stephpag{c}
\socratesspeaks
Now see whether it is possible in the same way for one who possesses knowledge not to have it, as, for instance, if a man should catch wild birds—pigeons or the like—and should arrange an aviary at home and keep them in it, we might in a way assert that he always has them because he possesses them, might we not?
\theaetetusspeaks
Yes.
\socratesspeaks
And yet in another way that he has none of them, but that he has acquired power over them, since he has brought them under his control in his own enclosure, \stephpag{d} to take them and hold them whenever he likes, by catching whichever bird he pleases, and to let them go again; and he can do this as often as be sees fit.
\theaetetusspeaks
That is true.
\socratesspeaks
Once more, then, just as a while ago we contrived some sort of a waxen figment in the soul, so now let us make in each soul an aviary stocked with all sorts of birds, some in flocks apart from the rest, others in small groups, and some solitary, flying hither and thither among them all. \stephpag{e}
\theaetetusspeaks
Consider it done. What next?
\socratesspeaks
We must assume that while we are children this receptacle is empty, and we must understand that the birds represent the varieties of knowledge. And whatsoever kind of knowledge a person acquires and shuts up in the enclosure, we must say that he has learned or discovered the thing of which this is the knowledge, and that just this is knowing.
\theaetetusspeaks
So be it. \stephpag{198 a}
\socratesspeaks
Consider then what expressions are needed for the process of recapturing and taking and holding and letting go again whichever he please of the kinds of knowledge, whether they are the same expressions as those needed for the original acquisition, or others. But you will understand better by an illustration. You admit that there is an art of arithmetic?
\theaetetusspeaks
Yes.
\socratesspeaks
Now suppose this to be a hunt after the kinds of knowledge, or sciences, of all odd and even numbers.
\theaetetusspeaks
I do so.
\socratesspeaks
Now it is by this art, I imagine, that a man has \stephpag{b} the sciences of numbers under his own control and also that any man who transmits them to another does this.
\theaetetusspeaks
Yes.
\socratesspeaks
And we say that when anyone transmits them he teaches, and when anyone receives them he learns, and when anyone, by having acquired them, has them in that aviary of ours, he knows them.
\theaetetusspeaks
Certainly.
\socratesspeaks
Now pay attention to what follows from this. Does not the perfect arithmetician understand all numbers; for he has the sciences of all numbers in his mind?
\theaetetusspeaks
To be sure. \stephpag{c}
\socratesspeaks
Then would such a man ever count anything—either any abstract numbers in his head, or any such external objects as possess number?
\theaetetusspeaks
Of course,
\socratesspeaks
But we shall affirm that counting is the same thing as considering how great any number in question is.
\theaetetusspeaks
We shall.
\socratesspeaks
Then he who by our previous admission knows all number is found to be considering that which he knows as if he did not know it. You have doubtless heard of such ambiguities.
\theaetetusspeaks
Yes, I have.
\socratesspeaks
Continuing, then, our comparison with the acquisition \stephpag{d} and hunting of the pigeons, we shall say that the hunting is of two kinds, one before the acquisition for the sake of possessing, the other carried on by the possessor for the sake of taking and holding in his hands what he had acquired long before. And just so when a man long since by learning came to possess knowledge of certain things, and knew them, he may have these very things afresh by taking up again the knowledge of each of them separately and holding it—the knowledge which he had acquired long before, but had not at hand in his mind?
\theaetetusspeaks
That is true. \stephpag{e}
\socratesspeaks
This, then, was my question just now: How should we express ourselves in speaking about them when an arithmetician undertakes to count or a man of letters to read something? In such a case shall we say that although he knows he sets himself to learn again from himself that which he knows?
\theaetetusspeaks
But that is extraordinary, Socrates.
\socratesspeaks
But shall we say that he is going to read or count that which he does not know, when we have granted that he knows all letters and all numbers? \stephpag{199 a}
\theaetetusspeaks
But that too is absurd.
\socratesspeaks
Shall we then say that words are nothing to us, if it amuses anyone to drag the expressions ``know" and ``learn" one way and another, but since we set up the distinction that it is one thing to possess knowledge and another thing to have it, we affirm that it is impossible not to possess what one possesses, so that it never happens that a man does not know that which he knows, but that it is possible to conceive a false opinion about it? \stephpag{b} For it is possible to have not the knowledge of this thing, but some other knowledge instead, when in hunting for some one kind of knowledge, as the various kinds fly about, he makes a mistake and catches one instead of another; so in one example he thought eleven was twelve, because he caught the knowledge of twelve, which was within him, instead of that of eleven, caught a ringdove, as it were, instead of a pigeon.
\theaetetusspeaks
Yes, that is reasonable.
\socratesspeaks
But when he catches the knowledge he intends to catch, he is not deceived and has true opinion, and so true and false opinion exist and none of the things \stephpag{c} which formerly annoyed us interferes? Perhaps you will agree to this; or what will you do?
\theaetetusspeaks
I will agree.
\socratesspeaks
Yes, for we have got rid of our difficulty about men not knowing that which they know; for we no longer find ourselves not possessing that which we possess, whether we are deceived about anything or not. However, another more dreadful disaster seems to be coming in sight.
\theaetetusspeaks
What disaster?
\socratesspeaks
If the interchange of kinds of knowledge should ever turn out to be false opinion.
\theaetetusspeaks
How so? \stephpag{d}
\socratesspeaks
Is it not the height of absurdity, in the first place for one who has knowledge of something to be ignorant of this very thing, not through ignorance but through his knowledge; secondly, for him to be of opinion that this thing is something else and something else is this thing—for the soul, when knowledge has come to it, to know nothing and be ignorant of all things? For by this argument there is nothing to prevent ignorance from coming to us and making us know something and blindness from making us see, if knowledge is ever to make us ignorant. \stephpag{e}
\theaetetusspeaks
Perhaps, Socrates, we were not right in making the birds represent kinds of knowledge only, but we ought to have imagined kinds of ignorance also flying about in the soul with the others; then the hunter would catch sometimes knowledge and sometimes ignorance of the same thing, and through the ignorance he would have false, but through the knowledge true opinion.
\socratesspeaks
It is not easy, Theaetetus, to refrain from praising you. However, examine your suggestion once more. Let it be as you say: \stephpag{200 a} the man who catches the ignorance will, you say, have false opinion. Is that it?
\theaetetusspeaks
Yes.
\socratesspeaks
But surely he will not also think that he has false opinion.
\theaetetusspeaks
Certainly not.
\socratesspeaks
No, but true opinion, and will have the attitude of knowing that about which he is deceived.
\theaetetusspeaks
Of course.
\socratesspeaks
Hence he will fancy that he has caught, and has, knowledge, not ignorance.
\theaetetusspeaks
Evidently.
\socratesspeaks
Then, after our long wanderings, we have come round again to our first difficulty. For the real reasoner \stephpag{b} will laugh and say, ``Most excellent Sirs, does a man who knows both knowledge and ignorance think that one of them, which he knows, is another thing which he knows; or, knowing neither of them, is he of opinion that one, which he does not know, is another thing which he does not know; or, knowing one and not the other, does he think that the one he does not know is the one he knows; or that the one he knows is the one he does not know? Or will you go on and tell me that there are kinds of knowledge of the kinds of knowledge and of ignorance, and that he who possesses these kinds of knowledge and has enclosed them in some sort of other ridiculous aviaries \stephpag{c} or waxen figments, knows them, so long as he possesses them, even if he has them not at hand in his soul? And in this fashion are you going to be compelled to trot about endlessly in the same circle without making any progress?" What shall we reply to this, Theaetetus?
\theaetetusspeaks
By Zeus, Socrates, I don't know what to say.
\socratesspeaks
Then, my boy, is the argument right in rebuking us and in pointing out that we were wrong to abandon knowledge and seek first for false opinion? \stephpag{d} It is impossible to know the latter until we have adequately comprehended the nature of knowledge.
\theaetetusspeaks
As the case now stands, Socrates, we cannot help thinking as you say.
\socratesspeaks
To begin, then, at the beginning once more, what shall we say knowledge is? For surely we are not going to give it up yet, are we?
\theaetetusspeaks
Not by any means, unless, that is, you give it up.
\socratesspeaks
Tell us, then, what definition will make us contradict ourselves least. \stephpag{e}
\theaetetusspeaks
The one we tried before, Socrates; at any rate, I have nothing else to offer.
\socratesspeaks
What one?
\theaetetusspeaks
That knowledge is true opinion; for true opinion is surely free from error and all its results are fine and good.
\socratesspeaks
The man who was leading the way through the river,\footnote{A man who was leading the way through a river was asked if the water was deep. He replied \textgreek{αὐτὸ δείξει}, ``the event itself will show'' (i.e. you can find out by trying). The expression became proverbial.} Theaetetus, said: ``The result itself will show;" and so in this matter, if we go on with our search, perhaps the thing will turn up in our path and of itself reveal the object of our search; \stephpag{201 a} but if we stay still, we shall discover nothing.
\theaetetusspeaks
You are right; let us go on with our investigation.
\socratesspeaks
Well, then, this at least calls for slight investigation; for you have a whole profession which declares that true opinion is not knowledge.
\theaetetusspeaks
How so? What profession is it?
\socratesspeaks
The profession of those who are greatest in wisdom, who are called orators and lawyers; for they persuade men by the art which they possess, not teaching them, but making them have whatever opinion they like. Or do you think there are any teachers so clever as to be able, in the short time allowed by the water-clock,\footnote{The length of speeches in the Athenian law courts was limited by a water-clock.} \stephpag{b} satisfactorily to teach the judges the truth about what happened to people who have been robbed of their money or have suffered other acts of violence, when there were no eyewitnesses?
\theaetetusspeaks
I certainly do not think so; but I think they can persuade them.
\socratesspeaks
And persuading them is making them have an opinion, is it not?
\theaetetusspeaks
Of course.
\socratesspeaks
Then when judges are justly persuaded about matters which one can know only by having seen them and in no other way, in such a case, judging of them from hearsay, having acquired a true opinion of them, \stephpag{c} they have judged without knowledge, though they are rightly persuaded, if the judgement they have passed is correct, have they not?
\theaetetusspeaks
Certainly.
\socratesspeaks
But, my friend, if true opinion and knowledge were the same thing in law courts, the best of judges could never have true opinion without knowledge; in fact, however, it appears that the two are different.
\theaetetusspeaks
Oh yes, I remember now, Socrates, having heard someone make the distinction, but I had forgotten it. He said that knowledge was true opinion accompanied by reason, \stephpag{d} but that unreasoning true opinion was outside of the sphere of knowledge; and matters of which there is not a rational explanation are unknowable—yes, that is what he called them—and those of which there is are knowable.
\socratesspeaks
I am glad you mentioned that. But tell us how he distinguished between the knowable and the unknowable, that we may see whether the accounts that you and I have heard agree.
\theaetetusspeaks
But I do not know whether I can think it out; but if someone else were to make the statement of it, I think I could follow.
\socratesspeaks
Listen then, while I relate it to you—``a dream for a dream." I in turn \stephpag{e} used to imagine that I heard certain persons say that the primary elements of which we and all else are composed admit of no rational explanation; for each alone by itself can only be named, and no qualification can be added, neither that it is nor that it is not, \stephpag{202 a} for that would at once be adding to it existence or non-existence, whereas we must add nothing to it, if we are to speak of that itself alone. Indeed, not even ``itself" or ``that" or ``each" or ``alone" or ``this" or anything else of the sort, of which there are many, must be added; for these are prevalent terms which are added to all things indiscriminately and are different from the things to which they are added; but if it were possible to explain an element, and it admitted of a rational explanation of its own, it would have to be explained apart from everything else. But in fact none of the primal elements can be expressed by reason; \stephpag{b} they can only be named, for they have only a name; but the things composed of these are themselves complex, and so their names are complex and form a rational explanation; for the combination of names is the essence of reasoning. Thus the elements are not objects of reason or of knowledge, but only of perception, whereas the combinations of them are objects of knowledge and expression and true opinion. When therefore a man acquires without reasoning the true opinion about anything, \stephpag{c} his mind has the truth about it, but has no knowledge; for he who cannot give and receive a rational explanation of a thing is without knowledge of it; but when he has acquired also a rational explanation he may possibly have become all that I have said and may now be perfect in knowledge. Is that the version of the dream you have heard, or is it different?
\theaetetusspeaks
That was it exactly.
\socratesspeaks
Are you satisfied, then, and do you state it in this way, that true opinion accompanied by reason is knowledge?
\theaetetusspeaks
Precisely. \stephpag{d}
\socratesspeaks
Can it be, Theaetetus, that we now, in this casual manner, have found out on this day what many wise men have long been seeking and have grown grey in the search?
\theaetetusspeaks
I, at any rate, Socrates, think our present statement is good.
\socratesspeaks
Probably this particular statement is so; for what knowledge could there still be apart from reason and right opinion? One point, however, in what has been said is unsatisfactory to me.
\theaetetusspeaks
What point?
\socratesspeaks
Just that which seems to be the cleverest; the assertion that the elements are unknowable and the class of combinations \stephpag{e} is knowable.
\theaetetusspeaks
Is that not right?
\socratesspeaks
We are sure to find out, for we have as hostages the examples which he who said all this used in his argument.
\theaetetusspeaks
What examples?
\socratesspeaks
The elements in writing, the letters of the alphabet, and their combinations, the syllables\footnote{\textgreek{Στοιχεῖον} and \textgreek{συλλαβή} originally general terms for element and combination, became the common words for letter and syllable.}; or do you think the author of the statements we are discussing had something else in view?
\theaetetusspeaks
No; those are what he had in view. \stephpag{203 a}
\socratesspeaks
Let us, then, take them up and examine them, or rather, let us examine ourselves and see whether it was in accordance with this theory, or not, that we learned letters. First then, the syllables have a rational explanation, but the letters have not?
\theaetetusspeaks
I suppose so.
\socratesspeaks
I think so, too, decidedly. Now if anyone should ask about the first syllable of Socrates; ``Theaetetus, tell me, what is SO?" What would you reply?
\theaetetusspeaks
I should say ``S and O."
\socratesspeaks
This, then, is your explanation of the syllable?
\theaetetusspeaks
Yes. \stephpag{b}
\socratesspeaks
Come now, in the same manner give me the explanation of the S.
\theaetetusspeaks
How can one give any elements of an element? For really, Socrates, the S is a voiceless letter,\footnote{The distinction here made is that which we make between vowels and consonants. The seven Greek vowels are \textgreek{α, ε, η, ι, ο, υ, ω} called \textgreek{φωνήεντα}.} a mere noise, as of the tongue hissing; B again has neither voice nor noise, nor have most of the other letters; and so it is quite right to say that they have no explanation, seeing that the most distinct of them, the seven vowels, have only voice, but no explanation whatsoever.
\socratesspeaks
In this point, then, my friend, it would seem that we have reached a right conclusion about knowledge.
\theaetetusspeaks
I think we have. \stephpag{c}
\socratesspeaks
But have we been right in laying down the principle that whereas the letter is unknowable, yet the syllable is knowable?
\theaetetusspeaks
Probably.
\socratesspeaks
Well then, shall we say that the syllable is the two letters, or, if there be more than two, all of them, or is it a single concept that has arisen from their combination?
\theaetetusspeaks
I think we mean all the letters it contains.
\socratesspeaks
Now take the case of two, S and O. The two together are the first syllable of my name. He who knows it knows the two letters, does he not? \stephpag{d}
\theaetetusspeaks
Of course.
\socratesspeaks
He knows, that is, the S and the O.
\theaetetusspeaks
Yes.
\socratesspeaks
How is that? He is ignorant of each, and knowing neither of them he knows them both?
\theaetetusspeaks
That is monstrous and absurd, Socrates.
\socratesspeaks
And yet if a knowledge of each letter is necessary before one can know both, he who is ever to know a syllable must certainly know the letters first, and so our fine theory will have run away and vanished! \stephpag{e}
\theaetetusspeaks
And very suddenly, too.
\socratesspeaks
Yes, for we are not watching it carefully. Perhaps we ought to have said that the syllable is not the letters, but a single concept that has arisen from them, having a single form of its own, different from the letters.
\theaetetusspeaks
Certainly; and perhaps that will be better than the other way.
\socratesspeaks
Let us look into that; we must not give up in such unmanly fashion a great and impressive theory.
\theaetetusspeaks
No, we must not. \stephpag{204 a}
\socratesspeaks
Let it be, then, as we say now, that the syllable or combination is a single form arising out of the several conjoined elements, and that it is the same in words and in all other things.
\theaetetusspeaks
Certainly.
\socratesspeaks
Therefore there must be no parts of it.
\theaetetusspeaks
How so?
\socratesspeaks
Because if there are parts of anything, the whole must inevitably be all the parts; or do you assert also that the whole that has arisen out of the parts is a single concept different from all the parts?
\theaetetusspeaks
Yes, I do.
\socratesspeaks
Do you then say that all and the whole are the same, \stephpag{b} or that each of the two is different from the other?
\theaetetusspeaks
I am not sure; but you tell me to answer boldly, so I take the risk and say that they are different.
\socratesspeaks
Your boldness, Theaetetus, is right; but whether your answer is so remains to be seen.
\theaetetusspeaks
Yes, certainly, we must see about that.
\socratesspeaks
The whole, then, according to our present view, would differ from all?
\theaetetusspeaks
Yes.
\socratesspeaks
How about this? Is there any difference between all in the plural and all in the singular? For instance, if we say one, two, three, \stephpag{c} four, five, six, or twice three, or three times two, or four and two, or three and two and one, are we in all these forms speaking of the same or of different numbers?
\theaetetusspeaks
Of the same.
\socratesspeaks
That is, of six?
\theaetetusspeaks
Yes.
\socratesspeaks
Then in each form of speech we have spoken of all the six?
\theaetetusspeaks
Yes.
\socratesspeaks
And again do we not speak of one thing when we speak of them all?
\theaetetusspeaks
Assuredly.
\socratesspeaks
That is, of six?
\theaetetusspeaks
Yes. \stephpag{d}
\socratesspeaks
Then in all things that are made up of number, we apply the same term to all in the plural and all in the singular?
\theaetetusspeaks
Apparently.
\socratesspeaks
Here is another way of approaching the matter. The number of the fathom and the fathom are the same, are they not?
\theaetetusspeaks
Yes.
\socratesspeaks
And of the furlong likewise.
\theaetetusspeaks
Yes.
\socratesspeaks
And the number of the army is the same as the army, and all such cases are alike? In each of them all the number is all the thing.
\theaetetusspeaks
Yes.
\socratesspeaks
And is the number of each anything but \stephpag{e} the parts of each?
\theaetetusspeaks
No.
\socratesspeaks
Everything that has parts, accordingly, consists of parts, does it not?
\theaetetusspeaks
Evidently.
\socratesspeaks
But we are agreed that the all must be all the parts if all the number is to be the all.\footnote{Cf. Plat. Theaet. 204b}
\theaetetusspeaks
Yes.
\socratesspeaks
Then the whole does not consist of parts, for if it consisted of all the parts it would be the all.
\theaetetusspeaks
That seems to be true.
\socratesspeaks
But is a part a part of anything in the world but the whole?
\theaetetusspeaks
Yes, of the all. \stephpag{205 a}
\socratesspeaks
You are putting up a brave fight, Theaetetus. But is not the all precisely that of which nothing is wanting?
\theaetetusspeaks
Necessarily.
\socratesspeaks
And is not just this same thing, from which nothing whatsoever is lacking, a whole? For that from which anything is lacking is neither a whole nor all, which have become identical simultaneously and for the same reason.
\theaetetusspeaks
I think now that there is no difference between all and whole.
\socratesspeaks
We were saying, were we not, that if there are parts of anything, the whole and all of it will be all the parts?
\theaetetusspeaks
Certainly.
\socratesspeaks
Once more, then, as I was trying to say just now, if the syllable is not the letters, does it not follow necessarily \stephpag{b} that it contains the letters, not as parts of it, or else that being the same as the letters, it is equally knowable with them?
\theaetetusspeaks
It does.
\socratesspeaks
And it was in order to avoid this that we assumed that it was different from them?
\theaetetusspeaks
Yes.
\socratesspeaks
Well then, if the letters are not parts of the syllable, can you mention any other things which are parts of it, but are not the letters\footnote{The reader is reminded that words \textgreek{στοιχεῖον} and \textgreek{συλλαβή} have the meanings ``element'' and ``combination'' as well as ``letter'' and ``syllable.''} of it?
\theaetetusspeaks
Certainly not. For if I grant that there are parts of the syllable, it would be ridiculous to give up the letters and look for other things as parts. \stephpag{c}
\socratesspeaks
Without question, then, Theaetetus, the syllable would be, according to our present view, some indivisible concept.
\theaetetusspeaks
I agree.
\socratesspeaks
Do you remember, then, my friend, that we admitted a little while ago, on what we considered good grounds, that there can be no rational explanation of the primary elements of which other things are composed, because each of them, when taken by itself, is not composite, and we could not properly apply to such an element even the expression ``be" or ``this," because these terms are different and alien, and for this reason it is irrational and unknowable?
\theaetetusspeaks
I remember. \stephpag{d}
\socratesspeaks
And is not this the sole reason why it is single in form and indivisible? I can see no other.
\theaetetusspeaks
There is no other to be seen.
\socratesspeaks
Then the syllable falls into the same class with the letter, if it has no parts and is a single form?
\theaetetusspeaks
Yes, unquestionably.
\socratesspeaks
If, then, the syllable is a plurality of letters and is a whole of which the letters are parts, the syllables and the letters are equally knowable and expressible, if all the parts were found to be the same as the whole. \stephpag{e}
\theaetetusspeaks
Certainly.
\socratesspeaks
But if one and indivisible, then syllable and likewise letter are equally irrational and unknowable; for the same cause will make them so.
\theaetetusspeaks
I cannot dispute it.
\socratesspeaks
Then we must not accept the statement of any one who says that the syllable is knowable and expressible, but the letter is not.
\theaetetusspeaks
No, not if we are convinced by our argument. \stephpag{206 a}
\socratesspeaks
But would you not rather accept the opposite belief, judging by your own experience when you were learning to read?
\theaetetusspeaks
What experience?
\socratesspeaks
In learning, you were merely constantly trying to distinguish between the letters both by sight and by hearing, keeping each of them distinct from the rest, that you might not be disturbed by their sequence when they were spoken or written.
\theaetetusspeaks
That is very true.
\socratesspeaks
And in the music school was not perfect attainment \stephpag{b} the ability to follow each note and tell which string produced it; and everyone would agree that the notes are the elements of music?
\theaetetusspeaks
Yes, that is all true.
\socratesspeaks
Then if we are to argue from the elements and combinations in which we ourselves have experience to other things in general, we shall say that the elements as a class admit of a much clearer knowledge than the compounds and of a knowledge that is much more important for the complete attainment of each branch of learning, and if anyone says that the compound is by its nature knowable and the element unknowable, we shall consider that he is, intentionally or unintentionally, joking.
\theaetetusspeaks
Certainly. \stephpag{c}
\socratesspeaks
Still other proofs of this might be brought out, I think; but let us not on that account lose sight of the question before us, which is: What is meant by the doctrine that the most perfect knowledge arises from the addition of rational explanation to true opinion?
\theaetetusspeaks
No, we must not.
\socratesspeaks
Now what are we intended to understand by ``rational explanation"? I think it means one of three things.
\theaetetusspeaks
What are they? \stephpag{d}
\socratesspeaks
The first would be making one's own thought clear through speech by means of verbs and nouns, imaging the opinion in the stream that flows through the lips, as in a mirror or water. Do you not think the rational explanation is something of that sort?
\theaetetusspeaks
Yes, I do. At any rate, we say that he who does that speaks or explains.
\socratesspeaks
Well, that is a thing that anyone can do sooner or later; he can show what he thinks about anything, unless he is deaf or dumb from the first; and so \stephpag{e} all who have any right opinion will be found to have it with the addition of rational explanation, and there will henceforth be no possibility of right opinion apart from knowledge.
\theaetetusspeaks
True.
\socratesspeaks
Let us not, therefore, carelessly accuse him of talking nonsense who gave the definition of knowledge which we are now considering; for perhaps that is not what he meant. He may have meant that each person if asked about anything must be able in reply \stephpag{207 a} to give his questioner an account of it in terms of its elements.
\theaetetusspeaks
As for example, Socrates?
\socratesspeaks
As, for example, Hesiod, speaking of a wagon, says, ``a hundred pieces of wood in a wagon."\footnote{Hes. WD 456} Now I could not name the pieces, nor, I fancy, could you; but if we were asked what a wagon is, we should be satisfied if we could say ``wheels, axle, body, rims, yoke."
\theaetetusspeaks
Certainly.
\socratesspeaks
But he, perhaps, would think we were ridiculous, just as he would if, on being asked about your name, we should reply by telling the syllables, \stephpag{b} holding a right opinion and expressing correctly what we have to say, but should think we were grammarians and as such both possessed and were expressing as grammarians would the rational explanation of the name Theaetetus. He would say that it is impossible for anyone to give a rational explanation of anything with knowledge, until he gives a complete enumeration of the elements, combined with true opinion. That, I believe, is what was said before.
\theaetetusspeaks
Yes, it was.
\socratesspeaks
So, too, he would say that we have right opinion about a wagon, but that he who can give an account of its essential nature \stephpag{c} in terms of those one hundred parts has by this addition added rational explanation to true opinion and has acquired technical knowledge of the essential nature of a wagon, in place of mere opinion, by describing the whole in terms of its elements.
\theaetetusspeaks
Do you agree to that, Socrates?
\socratesspeaks
If you, my friend, agree to it and accept the view that orderly description in terms of its elements is a rational account of anything, but that description in terms of syllables or still larger units is irrational, \stephpag{d} tell me so, that we may examine the question.
\theaetetusspeaks
Certainly I accept it.
\socratesspeaks
Do you accept it in the belief that anyone has knowledge of anything when he thinks that the same element is a part sometimes of one thing and sometimes of another or when he is of opinion that the same thing has as a part of it sometimes one thing and sometimes another?
\theaetetusspeaks
Not at all, by Zeus.
\socratesspeaks
Then do you forget that when you began to learn to read you and the others did just that?
\theaetetusspeaks
Do you mean when we thought that sometimes one letter \stephpag{e} and sometimes another belonged to the same syllable, and when we put the same letter sometimes into the proper syllable and sometimes into another?
\socratesspeaks
That is what I mean.
\theaetetusspeaks
By Zeus, I do not forget, nor do I think that those have knowledge who are in that condition.
\socratesspeaks
Take an example: When at such a stage in his progress a person in writing ``Theaetetus" thinks he ought to write, \stephpag{208 a} and actually does write, TH and E, and again in trying to write ``Theodorus" thinks he ought to write, and does write, T and E, shall we say that he knows the first syllable of your names?
\theaetetusspeaks
No, we just now agreed that a person in such a condition has not yet gained knowledge.
\socratesspeaks
Then there is nothing to prevent the same person from being in that condition with respect to the second and third and fourth syllables?
\theaetetusspeaks
No, nothing.
\socratesspeaks
Then, in that case, he has in mind the orderly description in terms of letters, and will write ``Theaetetus" with right opinion, when he writes the letters in order?
\theaetetusspeaks
Evidently. \stephpag{b}
\socratesspeaks
But he is still, as we say, without knowledge, though he has right opinion?
\theaetetusspeaks
Yes.
\socratesspeaks
Yes, but with his opinion he has rational explanation; for he wrote with the method in terms of letters in his mind, and we agreed that that was rational explanation.
\theaetetusspeaks
True.
\socratesspeaks
There is, then, my friend, a combination of right opinion with rational explanation, which cannot as yet properly be called knowledge?
\theaetetusspeaks
There is not much doubt about it.
\socratesspeaks
So it seems that the perfectly true definition of knowledge, which we thought we had, was but a golden dream. Or shall we wait a bit before we condemn it? Perhaps the definition to be adopted is not this, \stephpag{c} but the remaining one of the three possibilities one of which we said must be affirmed by anyone who asserts that knowledge is right opinion combined with rational explanation.
\theaetetusspeaks
I am glad you called that to mind. For there is still one left. The first was a kind of vocal image of the thought, the second the orderly approach to the whole through the elements, which we have just been discussing, and what is the third?
\socratesspeaks
It is just the definition which most people would give, that knowledge is the ability to tell some characteristic by which the object in question differs from all others.
\theaetetusspeaks
As an example of the method, what explanation can you give me, and of what thing? \stephpag{d}
\socratesspeaks
As an example, if you like, take the sun: I think it is enough for you to be told that it is the brightest of the heavenly bodies that revolve about the earth.
\theaetetusspeaks
Certainly.
\socratesspeaks
Understand why I say this. It is because, as we were just saying, if you get hold of the distinguishing characteristic by which a given thing differs from the rest, you will, as some say, get hold of the definition or explanation of it; but so long as you cling to some common quality, your explanation will pertain to all those objects to which the common quality belongs. \stephpag{e}
\theaetetusspeaks
I understand; and it seems to me that it is quite right to call that kind a rational explanation or definition.
\socratesspeaks
Then he who possesses right opinion about anything and adds thereto a comprehension of the difference which distinguishes it from other things will have acquired knowledge of that thing of which he previously had only opinion.
\theaetetusspeaks
That is what we affirm.
\socratesspeaks
Theaetetus, now that I have come closer to our statement, I do not understand it at all. It is like coming close to a scene-painting.\footnote{In which perspective is the main thing.} While I stood off at a distance, I thought there was something in it.
\theaetetusspeaks
What do you mean? \stephpag{209 a}
\socratesspeaks
I will tell you if I can. Assume that I have right opinion about you; if I add the explanation or definition of you, then I have knowledge of you, otherwise I have merely opinion.
\theaetetusspeaks
Yes.
\socratesspeaks
But explanation was, we agreed, the interpretation of your difference.
\theaetetusspeaks
It was.
\socratesspeaks
Then so long as I had merely opinion, I did not grasp in my thought any of the points in which you differ from others?
\theaetetusspeaks
Apparently not.
\socratesspeaks
Therefore I was thinking of some one of the common traits which you possess no more than other men. \stephpag{b}
\theaetetusspeaks
You must have been.
\socratesspeaks
For heaven's sake! How in the world could I in that case have any opinion about you more than about anyone else? Suppose that I thought ``That is Theaetetus which is a man and has nose and eyes and mouth" and so forth, mentioning all the parts. Can this thought make me think of Theaetetus any more than of Theodorus or of the meanest of the Mysians,\footnote{The Mysians were despised as especially effeminate and worthless.} as the saying is?
\theaetetusspeaks
Of course not.
\socratesspeaks
But if I think not only of a man with nose and eyes, \stephpag{c} but of one with snub nose and protruding eyes, shall I then have an opinion of you any more than of myself and all others like me?
\theaetetusspeaks
Not at all.
\socratesspeaks
No; I fancy Theaetetus will not be the object of opinion in me until this snubnosedness of yours has stamped and deposited in my mind a memorial different from those of the other examples of snubnosedness that I have seen, and the other traits that make up your personality have done the like. Then that memorial, if I meet you again tomorrow, will awaken my memory and make me have right opinion about you.
\theaetetusspeaks
Very true. \stephpag{d}
\socratesspeaks
Then right opinion also would have to do with differences in any given instance?
\theaetetusspeaks
At any rate, it seems so.
\socratesspeaks
Then what becomes of the addition of reason or explanation to right opinion? For if it is defined as the addition of an opinion of the way in which a given thing differs from the rest, it is an utterly absurd injunction.
\theaetetusspeaks
How so?
\socratesspeaks
When we have a right opinion of the way in which certain things differ from other things, we are told to acquire a right opinion of the way in which those same things differ from other things! On this plan the twirling of a scytale\footnote{\textgreek{Aσκυτάλη} was a staff, especially a staff about which a strip of leather was rolled, on which dispatches were so written that when unrolled they were illegible until rolled again upon another staff of the same size and shape.} or a pestle or anything of the sort would be as nothing \stephpag{e} compared with this injunction. It might more justly be called a blind man's giving directions; for to command us to acquire that which we already have, in order to learn that of which we already have opinion, is very like a man whose sight is mightily darkened.
\theaetetusspeaks
Tell me now, what did you intend to say when you asked the question a while ago?
\socratesspeaks
If, my boy, the command to add reason or explanation means learning to know and not merely getting an opinion about the difference, our splendid definition of knowledge would be a fine affair! For learning to know is acquiring knowledge, \stephpag{210 a} is it not?
\theaetetusspeaks
Yes.
\socratesspeaks
Then, it seems, if asked, ``What is knowledge?" our leader will reply that it is right opinion with the addition of a knowledge of difference; for that would, according to him, be the addition of reason or explanation.
\theaetetusspeaks
So it seems.
\socratesspeaks
And it is utterly silly, when we are looking for a definition of knowledge, to say that it is right opinion with knowledge, whether of difference or of anything else whatsoever. So neither perception, Theaetetus, nor true opinion, nor reason or explanation \stephpag{b} combined with true opinion could be knowledge.
\theaetetusspeaks
Apparently not.
\socratesspeaks
Are we then, my friend, still pregnant and in travail with knowledge, or have we brought forth everything?
\theaetetusspeaks
Yes, we have, and, by Zeus, Socrates, with your help I have already said more than there was in me.
\socratesspeaks
Then does our art of midwifery declare to us that all the offspring that have been born are mere wind-eggs and not worth rearing?
\theaetetusspeaks
It does, decidedly.
\socratesspeaks
If after this you ever undertake to conceive other thoughts, Theaetetus, and do conceive, \stephpag{c} you will be pregnant with better thoughts than these by reason of the present search, and if you remain barren, you will be less harsh and gentler to your associates, for you will have the wisdom not to think you know that which you do not know. So much and no more my art can accomplish; nor do I know aught of the things that are known by others, the great and wonderful men who are today and have been in the past. This art, however, both my mother and I received from God, she for women and I for young and noble men and for all who are fair. \stephpag{d} And now I must go to the Porch of the King, to answer to the suit which Meletus\footnote{Meletus was one of those who brought the suit which led to the condemnation and death of Socrates.} has brought against me. But in the morning, Theodorus, let us meet here again.

\end{drama}
\end{document}