\documentclass[letterpaper,12pt]{article}
\usepackage[12pt]{moresize}
\usepackage[utf8x]{inputenc}
\usepackage[english]{babel}
\usepackage[margin=2.5cm, right = 3.5cm]{geometry}
\usepackage{ebgaramond}

\renewcommand{\thefootnote}{[\fontfamily{ppl}\selectfont\arabic{footnote}]}
\setlength{\skip\footins}{1cm}
\usepackage[]{footmisc}
\renewcommand{\footnotemargin}{3mm} %Setting left margin
\renewcommand{\footnotelayout}{\hspace{2mm}} %spacing between the footnote number and the text of footnote

\usepackage{marginnote}
\renewcommand{\raggedrightmarginnote}{\raggedleft}
\newcommand{\stephpag}[1]{\marginnote{\small\itshape\fontfamily{ppl}\selectfont #1}}

\title{\vspace{-2.5cm} \scshape Parmenides \vspace{-8mm}}
\author{}
\date{
	\vspace{-1em}
		\small \fontfamily{ppl}\selectfont Written by Plato, translated by Harold North Fowler, 1925
	\begin{center}
		$\mathsection$
	\end{center}
		\vspace{-2em}
	}

\newenvironment{setting}
	{
		\setlength{\tabcolsep}{4em}
		\begin{center}
			\section*{\normalsize \fontfamily{ppl}\selectfont \itshape \bfseries Persons of the Dialogue\vspace{-1mm}}
			\par
			\begin{tabular}{ll}
	}
	{
			\end{tabular}
		\end{center}
		\par \fontfamily{ppl}\selectfont{\small \textbf{Scene:} \textit{Cephalus rehearses a dialogue which is supposed to have been narrated in his presence by Antiphon, the half-brother of Adeimantus and Glaucon, to certain Clazomenians.}}
		
		\hrulefill		
	}

\begin{document}

\begin{minipage}{15.45cm}
\maketitle

\begin{setting}
	\textsc{Cephalus.}	& \textsc{Adeimantus.} \\
	\textsc{Antiphon.}	& \textsc{Glaucon.} \\
	\textsc{Pythodorus.}	& \textsc{Socrates.} \\
	\textsc{Zeno.}		& \textsc{Parmenides.} \\
	\textsc{Aristoteles.}
\end{setting}
\end{minipage}

\vspace{1em}
\setlength{\parindent}{2em}

\stephpag{126 a}When we came from our home at Clazomenae to Athens, we met Adeimantus and Glaucon in the market-place. Adeimantus took me by the hand and said, ``Welcome, Cephalus if there is anything we can do for you here, let us know."

``Why," said I, ``that is just why I am here, to ask a favour of you."

``Tell us," said he, ``what it is." \stephpag{b} And I said, ``What was your half-brother's name? I don't remember. He was only a boy when I came here from Clazomenae before and that is now a long time ago. His father's name, I believe, was Pyrilampes."

``Yes," said he.

``And what is his own name?"

``Antiphon. Why do you ask?"

``These gentlemen," I said, ``are fellow-citizens of mine, who are very fond of philosophy. They have heard that this Antiphon had a good deal to do with a friend of Zeno's named Pythodorus, that Pythodorus often repeated to him the conversation \stephpag{c} which Socrates, Zeno, and Parmenides once had together, and that he remembers it."

``That is true," said he.

``Well," I said, ``we should like to hear it."

``There is no difficulty about that," said he ``for when he was a youth he studied it with great care though now he devotes most of his time to horses, like his grandfather Antiphon. If that is what you want, let us go to him. He has just gone home from here, and he lives close by in Melite." \stephpag{127 a} Thereupon we started, and we found Antiphon at home, giving a smith an order to make a bridle. When he had got rid of the smith and his brother told him what we were there for, he remembered me from my former visit and greeted me cordially, and when we asked him to repeat the conversation, he was at first unwilling—for he said it was a good deal of trouble—but afterwards he did so. Antiphon, then, said that Pythodorus told him \stephpag{b} that Zeno and Parmenides once came to the Great Panathenaea; that Parmenides was already quite elderly, about sixty-five years old, very white-haired, and of handsome and noble countenance; Zeno was at that time about forty years of age; he was tall and good-looking, and there was a story that Parmenides had been in love with him. \stephpag{c} He said that they lodged with Pythodorus outside of the wall, in Cerameicus, and that Socrates and many others with him went there because they wanted to hear Zeno's writings, which had been brought to Athens for the first time by them. Socrates was then very young. So Zeno himself read aloud to them, and Parmenides was not in the house. \stephpag{d} Pythodorus said the reading of the treatises was nearly finished when he came in himself with Parmenides and Aristoteles (the one who was afterwards one of the thirty), so they heard only a little that remained of the written works. He himself, however, had heard Zeno read them before.

Socrates listened to the end, and then asked that the first thesis of the first treatise be read again. When this had been done, he said: \stephpag{e} ``Zeno, what do you mean by this? That if existences are many, they must be both like and unlike, which is impossible; for the unlike cannot be like, nor the like unlike? Is not that your meaning?"

``Yes," said Zeno.

``Then if it is impossible for the unlike to be like and the like unlike, it is impossible for existences to be many; for if they were to be many, they would experience the impossible. Is that the purpose of your treatises, to maintain against all arguments that existences are not many? And you think each of your treatises is a proof of this very thing, and therefore you believe that the proofs you offer that existences are not many are as many as the treatises you have written? Is that your meaning, \stephpag{128 a} or have I misunderstood?"

``No," said Zeno, ``you have grasped perfectly the general intent of the work."

``I see, Parmenides," said Socrates, ``that Zeno here wishes to be very close to you not only in his friendship, but also in his writing. For he has written much the same thing as you, but by reversing the process he tries to cheat us into the belief that he is saying something new. For you, in your poems, say that the all is one, \stephpag{b} and you furnish proofs of this in fine and excellent fashion; and he, on the other hand, says it is not many, and he also furnishes very numerous and weighty proofs. That one of you says it is one, and the other that it is not many, and that each of you expresses himself so that although you say much the same you seem not to have said the same things at all, appears to the rest of us a feat of expression quite beyond our power."

``Yes, Socrates," said Zeno, ``but you have not perceived all aspects of the truth about my writings. You follow the arguments with a scent \stephpag{c} as keen as a Laconian hound's, but you do not observe that my treatise is not by any means so pretentious that it could have been written with the intention you ascribe to it, of disguising itself as a great performance in the eyes of men. What you mentioned is a mere accident, but in truth these writings are meant to support the argument of Parmenides against those who attempt to jeer at him and assert that \stephpag{d} if the all is one many absurd results follow which contradict his theory. Now this treatise opposes the advocates of the many and gives them back their ridicule with interest, for its purpose is to show that their hypothesis that existences are many, if properly followed up, leads to still more absurd results than the hypothesis that they are one. It was in such a spirit of controversy that I wrote it when I was young, \stephpag{e} and when it was written some one stole it, so that I could not even consider whether it should be published or not. So, Socrates, you are not aware of this and you think that the cause of its composition was not the controversial spirit of a young man, but the ambition of an old one. In other respects, as I said, you guessed its meaning pretty well."

``I see," said Socrates, ``and I accept your explanation. But tell me, do you not believe there is an idea of likeness in the abstract, \stephpag{129 a} and another idea of unlikeness, the opposite of the first, and that you and I and all things which we call many partake of these two? And that those which partake of likeness become like, and those which partake of unlikeness become unlike, and those which partake of both become both like and unlike, all in the manner and degree of their participation? And even if all things partake of both opposites, and are enabled by their participation to be both like and unlike themselves, \stephpag{b} what is there wonderful about that? For if anyone showed that the absolute like becomes unlike, or the unlike like, that would, in my opinion, be a wonder; but if he shows that things which partake of both become both like and unlike, that seems to me, Zeno, not at all strange, not even if he shows that all things are one by participation in unity and that the same are also many by participation in multitude; but if he shows that absolute unity is also many and the absolute many again are one, then I shall be amazed. \stephpag{c} The same applies to all other things. If he shows that the kinds and ideas in and by themselves possess these opposite qualities, it is marvellous but if he shows that I am both one and many, what marvel is there in that? He will say, when he wishes to show that I am many, that there are my right parts and my left parts, my front parts and my back parts, likewise upper and lower, all different; for I do, I suppose, partake of multitude; \stephpag{d} and when he wishes to show that I am one, he will say that we here are seven persons, of whom I am one, a man, partaking also of unity and so he shows that both assertions are true. If anyone then undertakes to show that the same things are both many and one—I mean such things as stones, sticks, and the like—we shall say that he shows that they are many and one, but not that the one is many or the many one; he says nothing wonderful, but only what we should all accept. If, however, as I was saying just now, he first distinguishes the abstract ideas, such as likeness and unlikeness, \stephpag{e} multitude and unity, rest and motion, and the like, and then shows that they can be mingled and separated, I should," said he, ``be filled with amazement, Zeno. Now I think this has been very manfully discussed by you; but I should, as I say, be more amazed if anyone could show in the abstract ideas, which are intellectual conceptions, \stephpag{130 a} this same multifarious and perplexing entanglement which you described in visible objects."

Pythodorus said that he thought at every word, while Socrates was saying this, Parmenides and Zeno would be angry, but they paid close attention to him and frequently looked at each other and smiled, as if in admiration of Socrates, and when he stopped speaking Parmenides expressed their approval. ``Socrates," \stephpag{b} he said, ``what an admirable talent for argument you have! Tell me, did you invent this distinction yourself, which separates abstract ideas from the things which partake of them? And do you think there is such a thing as abstract likeness apart from the likeness which we possess, and abstract one and many, and the other abstractions of which you heard Zeno speaking just now?"

``Yes, I do," said Socrates.

``And also," said Parmenides, ``abstract ideas of the just, the beautiful, the good, and all such conceptions?"

``Yes," he replied. \stephpag{c} ``And is there an abstract idea of man, apart from us and all others such as we are, or of fire or water?"

``I have often," he replied, ``been very much troubled, Parmenides, to decide whether there are ideas of such things, or not."

``And are you undecided about certain other things, which you might think rather ridiculous, such as hair, mud, dirt, or anything else particularly vile and worthless? Would you say that there is an idea of each of these distinct and different from the things \stephpag{d} with which we have to do, or not?"

``By no means," said Socrates. ``No, I think these things are such as they appear to us, and it would be quite absurd to believe that there is an idea of them; and yet I am sometimes disturbed by the thought that perhaps what is true of one thing is true of all. Then when I have taken up this position, I run away for fear of falling into some abyss of nonsense and perishing; so when I come to those things which we were just saying do have ideas, I stay and busy myself with them." \stephpag{e} ``Yes, for you are still young," said Parmenides, ``and philosophy has not yet taken hold upon you, Socrates, as I think it will later. Then you will not despise them; but now you still consider people's opinions, on account of your youth. Well, tell me do you think that, as you say, there are ideas, and that these other things which partake of them are named from them, \stephpag{131 a} as, for instance, those that partake of likeness become like, those that partake of greatness great, those that partake of beauty and justice just and beautiful?"

``Certainly," said Socrates.

``Well then, does each participant object partake of the whole idea, or of a part of it? Or could there be some other third kind of participation?"

``How could there be?" said he.

``Do you think the whole idea, being one, is in each of the many participants, or what?"

``Yes, for what prevents it from being in them, Parmenides?" said Socrates. \stephpag{b} ``Then while it is one and the same, the whole of it would be in many separate individuals at once, and thus it would itself be separate from itself."

``No," he replied, ``for it might be like day, which is one and the same, is in many places at once, and yet is not separated from itself; so each idea, though one and the same, might be in all its participants at once."

``That," said he, ``is very neat, Socrates you make one to be in many places at once, just as if you should spread a sail over many persons and then should say it was one and all of it was over many. \stephpag{c} Is not that about what you mean?"

``Perhaps it is," said Socrates.

``Would the whole sail be over each person, or a particular part over each?"

``A part over each."

``Then," said he, ``the ideas themselves, Socrates, are divisible into parts, and the objects which partake of them would partake of a part, and in each of them there would be not the whole, but only a part of each idea."

``So it appears."

``Are you, then, Socrates, willing to assert that the one idea is really divided and will still be one?"

``By no means," he replied.

``No," said Parmenides, ``for if you divide absolute greatness, \stephpag{d} and each of the many great things is great by a part of greatness smaller than absolute greatness, is not that unreasonable?"

``Certainly," he said.

``Or again, will anything by taking away a particular small part of equality possess something by means of which, when it is less than absolute equality, its possessor will be equal to anything else?"

``That is impossible."

``Or let one of us have a part of the small; the small will be greater than this, since this is a part of it, and therefore the absolute small will be greater; but that to which the part of the small is added will be smaller, \stephpag{e} not greater, than before."

``That," said he, ``is impossible."

``How, then, Socrates, will other things partake of those ideas of yours, if they cannot partake of them either as parts or as wholes?"

``By Zeus," he replied, ``I think that is a very hard question to determine."

``Well, what do you think of this?"

``Of what?" \stephpag{132 a} ``I fancy your reason for believing that each idea is one is something like this; when there is a number of things which seem to you to be great, you may think, as you look at them all, that there is one and the same idea in them, and hence you think the great is one."

``That is true," he said.

``But if with your mind's eye you regard the absolute great and these many great things in the same way, will not another great appear beyond, by which all these must appear to be great?"

``So it seems."

``That is, another idea of greatness will appear, in addition to absolute greatness and the objects which partake of it; \stephpag{b} and another again in addition to these, by reason of which they are all great; and each of your ideas will no longer be one, but their number will be infinite."

``But, Parmenides," said Socrates, ``each of these ideas may be only a thought, which can exist only in our minds then each might be one, without being exposed to the consequences you have just mentioned."

``But," he said, ``is each thought one, but a thought of nothing?"

``That is impossible," he replied.

``But of something?"

``Yes." \stephpag{c} ``Of something that is, or that is not?"

``Of something that is."

``A thought of some single element which that thought thinks of as appertaining to all and as being one idea?"

``Yes."

``Then will not this single element, which is thought of as one and as always the same in all, be an idea?"

``That, again, seems inevitable."

``Well then," said Parmenides, ``does not the necessity which compels you to say that all other things partake of ideas, oblige you also to believe either that everything is made of thoughts, and all things think, or that, being thoughts, they are without thought?"

``That is quite unreasonable, too," he said, \stephpag{d} ``but Parmenides, I think the most likely view is, that these ideas exist in nature as patterns, and the other things resemble them and are imitations of them; their participation in ideas is assimilation to them, that and nothing else."

``Then if anything," he said, ``resembles the idea, can that idea avoid being like the thing which resembles it, in so far as the thing has been made to resemble it; or is there any possibility that the like be unlike its like?"

``No, there is none."

``And must not necessarily the like partake of \stephpag{e} the same idea as its like?"

``It must."

``That by participation in which like things are made like, will be the absolute idea, will it not?"

``Certainly."

``Then it is impossible that anything be like the idea, or the idea like anything; for if they are alike, some further idea, in addition to the first, will always appear, and if that is like anything, still another, \stephpag{133 a} and a new idea will always be arising, if the idea is like that which partakes of it."

``Very true."

``Then it is not by likeness that other things partake of ideas we must seek some other method of participation."

``So it seems."

``Do you see, then, Socrates, how great the difficulty is, if we maintain that ideas are separate, independent entities?"

``Yes, certainly."

``You may be sure," he said, ``that you do not yet, if I may say so, \stephpag{b} grasp the greatness of the difficulty involved in your assumption that each idea is one and is something distinct from concrete things."

``How is that?" said he.

``There are many reasons," he said, ``but the greatest is this; if anyone should say that the ideas cannot even be known if they are such as we say they must be, no one could prove to him that he was wrong, unless he who argued that they could be known were a man of wide education and ability and were willing to follow the proof through many long and elaborate details; \stephpag{c} he who maintains that they cannot be known would be unconvinced."

``Why is that, Parmenides?" said Socrates.

``Because, Socrates, I think that you or anyone else who claims that there is an absolute idea of each thing would agree in the first place that none of them exists in us."

``No, for if it did, it would no longer be absolute," said Socrates.

``You are right," he said. ``Then those absolute ideas which are relative to one another have their own nature in relation to themselves, and not in relation to the likenesses, \stephpag{d} or whatever we choose to call them, which are amongst us, and from which we receive certain names as we participate in them. And these concrete things, which have the same names with the ideas, are likewise relative only to themselves, not to the ideas, and, belong to themselves, not to the like-named ideas."

``What do you mean?" said Socrates.

``For instance," said Parmenides, ``if one of us is master or slave of anyone, he is not the slave of master in the abstract, \stephpag{e} nor is the master the master of slave in the abstract; each is a man and is master or slave of a man but mastership in the abstract is mastership of slavery in the abstract, and likewise slavery in the abstract is slavery to mastership in the abstract, but our slaves and masters are not relative to them, nor they to us; \stephpag{134 a} they, as I say, belong to themselves and are relative to themselves and likewise our slaves and masters are relative to themselves. You understand what I mean, do you not?"

``Certainly," said Socrates, ``I understand."

``Then knowledge also, if abstract or absolute, would be knowledge of abstract or absolute truth?"

``Certainly."

``And likewise each kind of absolute knowledge would be knowledge of each kind of absolute being, would it not?"

``Yes."

``And would not the knowledge that exists among us be the knowledge of the truth that exists among us, and each kind of our knowledge \stephpag{b} be the knowledge of each kind of truth that exists among us?"

``Yes, that is inevitable."

``But the ideas themselves, as you, agree, we have not, neither can they be among us."

``No, they cannot."

``And the various classes of ideas are known by the absolute idea of knowledge?"

``Yes."

``Which we do not possess."

``No, we do not."

``Then none of the ideas is known by us, since we do not partake of absolute knowledge."

``Apparently not."

``Then the absolute good and the beautiful and all \stephpag{c} which we conceive to be absolute ideas are unknown to us."

``I am afraid they are."

``Now we come to a still more fearful consequence."

``What is it?"

``You would say, no doubt, that if there is an absolute kind of knowledge, it is far more accurate than our knowledge, and the same of beauty and all the rest?"

``Yes."

``And if anything partakes of absolute knowledge, you would say that there is no one more likely than God to possess this most accurate knowledge?"

``Of course." \stephpag{d} ``Then will it be possible for God to know human things, if he has absolute knowledge?"

``Why not?"

``Because," said Parmenides, ``we have agreed that those ideas are not relative to our world, nor our world to them, but each only to themselves."

``Yes, we have agreed to that."

``Then if this most perfect mastership and this most accurate knowledge are with God, his mastership can never rule us, \stephpag{e} nor his knowledge know us or anything of our world; we do not rule the gods with our authority, nor do we know anything of the divine with our knowledge, and by the same reasoning, they likewise, being gods, are not our masters and have no knowledge of human affairs."

``But surely this," said he, ``is a most amazing argument, if it makes us deprive God of knowledge."

``And yet, Socrates," said Parmenides, \stephpag{135 a} ``these difficulties and many more besides are inseparable from the ideas, if these ideas of things exist and we declare that each of them is an absolute idea. Therefore he who hears such assertions is confused in his mind and argues that the ideas do not exist, and even if they do exist cannot by any possibility be known by man; and he thinks that what he says is reasonable, and, as I was saying just now, he is amazingly hard to convince. Only a man of very great natural gifts will be able to understand that everything has a class and absolute essence, \stephpag{b} and only a still more wonderful man can find out all these facts and teach anyone else to analyze them properly and understand them."

``I agree with you, Parmenides," said Socrates, ``for what you say is very much to my mind."

``But on the other hand," said Parmenides, ``if anyone, with his mind fixed on all these objections and others like them, denies the existence of ideas of things, and does not assume an idea under which each individual thing is classed, he will be quite at a loss, \stephpag{c} since he denies that the idea of each thing is always the same, and in this way he will utterly destroy the power of carrying on discussion. You seem to have been well aware of this."

``Quite true," he said.

``Then what will become of philosophy? To what can you turn, if these things are unknown?"

``I do not see at all, at least not at present."

``No, Socrates," he said, ``for you try too soon, before you are properly trained, to define the beautiful, the just, the good, and all the other ideas. \stephpag{d} You see I noticed it when I heard you talking yesterday with Aristoteles here. Your impulse towards dialectic is noble and divine, you may be assured of that; but exercise and train yourself while you are still young in an art which seems to be useless and is called by most people mere loquacity; otherwise the truth will escape you."

``What, then, Parmenides," he said, ``is the method of training?"

``That which you heard Zeno practising," said he. \stephpag{e} ``However, even when you were speaking to him I was pleased with you, because you would not discuss the doubtful question in terms of visible objects or in relation to them, but only with reference to what we conceive most entirely by the intellect and may call ideas."

``Yes," he said, ``that is because I think that in that way it is quite easy to show that things experience likeness or unlikeness or anything else."

``Quite right," said he, ``but if you wish to get better training, you must do something more than that; \stephpag{136 a} you must consider not only what happens if a particular hypothesis is true, but also what happens if it is not true."

``What do you mean?" he said.

``Take, for instance," he replied, ``that hypothesis of Zeno's if the many exist, you should inquire what will happen to the many themselves in relation to themselves and to the one, and to the one in relation to itself and to the many, and also what will happen to the one and the many in relation to themselves and to each other, if the many do not exist. \stephpag{b} And likewise if you suppose the existence or non-existence of likeness, what will happen to the things supposed and to other things in relation to themselves and to each other under each of the two hypotheses. The same applies to unlikeness and to motion and rest, creation and destruction, and even to being and not being. In brief, whatever the subject of your hypothesis, if you suppose that it is or is not, or that it experiences any other affection, you must consider what happens to it and to any other particular things you may choose, and to a greater number and to all in the same way; \stephpag{c} and you must consider other things in relation to themselves and to anything else you may choose in any instance, whether you suppose that the subject of your hypothesis exists or does not exist, if you are to train yourself completely to see the truth perfectly."

``Parmenides," he said, ``it is a stupendous amount of study which you propose, and I do not understand very well. Why do you not yourself frame an hypothesis and discuss it, to make me understand better?" \stephpag{d} ``That is a great task, Socrates," he said, ``to impose upon a man of my age."

``But you, Zeno," said Socrates, ``why do not you do it for us?"

Pythodorus said that Zeno answered with a smile: ``Let us ask it of Parmenides himself, Socrates; for there is a great deal in what he says, and perhaps you do not see how heavy a task you are imposing upon him. If there were more of us, it would not be fair to ask it of him; for it is not suitable for him to speak on such subjects before many, especially at his age; \stephpag{e} for the many do not know that except by this devious passage through all things the mind cannot attain to the truth. So I, Parmenides, join Socrates in his request, that I myself may hear the method, which I have not heard for a long time."

Antiphon said that Pythodorus told him that when Zeno said this he himself and Antisthenes and the rest begged Parmenides to show his meaning by an example and not to refuse. And Parmenides said: ``I must perforce do as you ask. \stephpag{137 a} And yet I feel very much like the horse in the poem of Ibycus\footnote{Ibycus fragm. Bergk.}—an old race-horse who was entered for a chariot race and was trembling with fear of what was before him, because he knew it by experience. Ibycus says he is compelled to fall in love against his will in his old age, and compares himself to the horse. So I am filled with terror when I remember through what a fearful ocean of words I must swim, old man that I am. However, I will do it, for I must be obliging, especially since we are, as Zeno says, alone. \stephpag{b} Well, how shall we begin? What shall be our first hypothesis? Or, since you are determined that I must engage in a laborious pastime, shall I begin with myself, taking my own hypothesis and discussing the consequences of the supposition that the one exists or that it does not exist?"

``By all means," said Zeno.

``Who then," said he, ``to answer my questions? Shall we say the youngest? He would be least likely to be over-curious and most likely to say what he thinks and moreover his replies would give me a chance to rest." \stephpag{c} ``I am ready, Parmenides, to do that," said Aristoteles, ``for I am the youngest, so you mean me. Ask your questions and I will answer."

``Well then," said he, ``if the one exists, the one cannot be many, can it?" ``No, of course not." ``Then there can be no parts of it, nor can it be a whole." ``How is that?" ``The part surely is part of a whole." ``Yes." ``And what is the whole? Is not a whole that of which no part is wanting?" \stephpag{d} ``Certainly." ``Then in both cases the one would consist of parts, being a whole and having parts." ``Inevitably." ``Then in both cases the one would be many, not one." ``True." ``Yet it must be not many, but one." ``Yes." ``Then the one, if it is to be one, will not be a whole and will not have parts." ``No."

``And if it has no parts, it can have no beginning, or middle, or end, for those would be parts of it?" ``Quite right." ``Beginning and end are, however, the limits of everything." ``Of course." ``Then the one, if it has neither beginning nor end, is unlimited." ``Yes, it is unlimited." ``And it is without form, \stephpag{e} for it partakes neither of the round nor of the straight." ``How so?" ``The round, of course, is that of which the extremes are everywhere equally distant from the center." ``Yes." ``And the straight, again, is that of which the middle is in the nearest line between the two extremes." ``It is." ``Then the one would have parts and would be many, whether it partook of straight or of round form." ``Certainly." ``Then it is neither straight nor round, since it has no parts." \stephpag{138 a} ``Right."

``Moreover, being of such a nature, it cannot be anywhere, for it could not be either in anything else or in itself." ``How is that?" ``If it were in something else, it would be encircled by that in which it would be and would be touched in many places by many parts of it; but that which is one and without parts and does not partake of the circular nature cannot be touched by a circle in many places." ``No, it cannot." ``But, furthermore, being in itself it would also be surrounding with itself naught other than itself, \stephpag{b} if it were in itself; for nothing can be in anything which does not surround it." ``No, it cannot." ``Then that which surrounds would be other than that which is surrounded; for a whole cannot be both active and passive in the same action; and thus one would be no longer one, but two." ``True." ``Then the one is not anywhere, neither in itself nor in something else." ``No, it is not."

``This being the case, see whether it can be either at rest or in motion." ``Why not?" \stephpag{c} ``Because if in motion it would be either moving in place or changing; for those are the only kinds of motion." ``Yes." ``But the one, if changing to something other than itself, cannot any longer be one." ``It cannot." ``Then it is not in motion by the method of change." ``Apparently not." ``But by moving in place?" ``Perhaps." ``But if the one moved in place, it would either revolve in the same spot or pass from one place to another." ``Yes, it must do so." ``And that which revolves must rest upon a center and have other parts which turn about the center; \stephpag{d} but what possible way is there for that which has no center and no parts to revolve upon a center?" ``There is none." ``But does it change its place by coming into one place at one time and another at another, and move in that way?" ``Yes, if it moves at all." ``Did we not find that it could not be in anything?" ``Yes." ``And is it not still more impossible for it to come into anything?" ``I do not understand why." ``If anything comes into anything, it must be not yet in it, while it is still coming in, nor still entirely outside of it, if it is already coming in, must it not?" ``It must." \stephpag{e} ``Now if anything goes through this process, it can be only that which has parts; for a part of it could be already in the other, and the rest outside; but that which has no parts cannot by any possibility be entirely neither inside nor outside of anything at the same time." ``True." ``But is it not still more impossible for that which has no parts and is not a whole to come into anything, since it comes in neither in parts nor as a whole?" ``Clearly." \stephpag{139 a} ``Then it does not change its place by going anywhere or into anything, nor does it revolve in a circle, nor change." ``Apparently not." ``Then the one is without any kind of motion." ``It is motionless." ``Furthermore, we say that it cannot be in anything." ``We do." ``Then it is never in the same." ``Why is that?" ``Because it would then be in that with which the same is identical." ``Certainly." ``But we saw that it cannot be either in itself or in anything else." ``No, it cannot." ``Then the one is never in the same." \stephpag{b} ``Apparently not." ``But that which is never in the same is neither motionless nor at rest." ``No, it cannot be so." ``The one, then, it appears, is neither in motion nor at rest." ``No, apparently not."

``Neither, surely, can it be the same with another or with itself; nor again other than itself or another." ``Why not?" ``If it were other than itself, it would be other than one and would not be one." ``True." ``And, surely, if it were the same with another, it would be that other, and would not be itself; \stephpag{c} therefore in this case also it would not be that which it is, namely one, but other than one." ``Quite so." ``Then it will not be the same as another, nor other than itself." ``No." ``But it will not be other than another, so long as it is one. For one cannot be other than anything; only other, and nothing else, can be other than another." ``Right." ``Then it will not be other by reason of being one, will it?" ``Certainly not." ``And if not for this reason, not by reason of itself; and if not by reason of itself, not itself; but since itself is not other at all, \stephpag{d} it will not be other than anything." ``Right." ``And yet one will not be the same with itself." ``Why not?" ``The nature of one is surely not the same as that of the same." ``Why?" ``Because when a thing becomes the same as anything, it does not thereby become one." ``But why not?" ``That which becomes the same as many, becomes necessarily many, not one." ``True." ``But if the one and the same were identical, whenever anything became the same it would always become one, and when it became one, the same." ``Certainly." ``Then if the one is the same with itself, \stephpag{e} it will not be one with itself; and thus, being one, it will not be one; this, however, is impossible; it is therefore impossible for one to be either the other of other or the same with itself." ``Impossible." ``Thus the one cannot be either other or the same to itself or another." ``No, it cannot." ``And again it will not be like or unlike anything, either itself or another." ``Why not?" ``Because the like is that which is affected in the same way." ``Yes." ``But we saw that the same was of a nature distinct from that of the one." ``Yes, so we did." \stephpag{140 a} ``But if the one were affected in any way apart from being one, it would be so affected as to be more than one, and that is impossible." ``Yes." ``Then the one cannot possibly be affected in the same way as another or as itself." ``Evidently not." ``Then it cannot be like another or itself." ``No, so it appears." ``Nor can the one be so affected as to be other; for in that case it would be so affected as to be more than one." ``Yes, it would be more." ``But that which is affected in a way other than itself or other, \stephpag{b} would be unlike itself or other, if that which is affected in the same way is like." ``Right." ``But the one, as it appears, being never affected in a way other than itself or other, is never unlike either itself or other." ``Evidently not." ``Then the one will be neither like nor unlike either other or itself." ``So it seems."

``Since, then, it is of such a nature, it can be neither equal nor unequal to itself or other." ``Why not?" ``If it is equal, it is of the same measures as that to which it is equal." ``Yes." ``And if it is greater or less than things \stephpag{c} with which it is commensurate, it will have more measures than the things which are less and less measures than the things which are greater." ``Yes." ``And in the case of things with which it is not commensurate, it will have smaller measures than some and greater measures than others." ``Of course." ``Is it not impossible for that which does not participate in sameness to have either the same measures or anything else the same?" ``Impossible." ``Then not having the same measures, it cannot be equal either to itself or to anything else." ``No, apparently not." ``But whether it have more measures or less, \stephpag{d} it will have as many parts as measures and thus one will be no longer one, but will be as many as are its measures." ``Right." ``But if it were of one measure, it would be equal to the measure; but we have seen that it cannot be equal to anything." ``Yes, so we have." ``Then it will partake neither of one measure, nor of many, nor of few; nor will it partake at all of the same, nor will it ever, apparently, be equal to itself or to anything else; nor will it be greater or less than itself or another." ``Perfectly true." \stephpag{e} ``Well, does anyone believe that the one can be older or younger or of the same age?" ``Why not?" ``Because if it has the same age as itself or as anything else, it will partake of equality and likeness of time, and we said the one had no part in likeness or equality." ``Yes, we said that." ``And we said also that it does not partake of unlikeness or inequality." ``Certainly." ``How, then, being of such a nature, \stephpag{141 a} can it be either younger or older or of the same age as anything?" ``In no way." ``Then the one cannot be younger or older or of the same age as anything." ``No, evidently not." ``And can the one exist in time at all, if it is of such a nature? Must it not, if it exists in time, always be growing older than itself?" ``It must." ``And the older is always older than something younger?" ``Certainly." ``Then that which grows older than itself grows at the same time younger than itself, if it is to have something than which it grows older." ``What do you mean?" \stephpag{b} ``This is what I mean: A thing which is different from another does not have to become different from that which is already different, but it must be different from that which is already different, it must have become different from that which has become so, it will have to be different from that which will be so, but from that which is becoming different it cannot have become, nor can it be going to be, nor can it already be different: it must become different, and that is all." ``There is no denying that." \stephpag{c} ``But surely the notion 'older' is a difference with respect to the younger and to nothing else." ``Yes, so it is." ``But that which is becoming older than itself must at the same time be becoming younger than itself." ``So it appears." ``But surely it cannot become either for a longer or for a shorter time than itself; it must become and be and be about to be for an equal time with itself." ``That also is inevitable." ``Apparently, then, it is inevitable that everything which exists in time and partakes of time \stephpag{d} is of the same age as itself and is also at the same time becoming older and younger than itself." ``I see no escape from that." ``But the one had nothing to do with such affections." ``No, it had not." ``It has nothing to do with time, and does not exist in time." ``No, that is the result of the argument."

``Well, and do not the words 'was,' 'has become,' and 'was becoming' appear to denote participation in past time?" ``Certainly." \stephpag{e} ``And 'will be,' 'will become,' and 'will be made to become,' in future time?" ``Yes." ``And 'is' and 'is becoming' in the present?" ``Certainly." ``Then if the one has no participation in time whatsoever, it neither has become nor became nor was in the past, it has neither become nor is it becoming nor is it in the present, and it will neither become nor be made to become nor will it be in the future." ``Very true." ``Can it then partake of being in any other way than in the past, present, or future?" ``It cannot." ``Then the one has no share in being at all." ``Apparently not." ``Then the one is not at all." " Evidently not." " Then it has no being even so as to be one, for if it were one, it would be and would partake of being; but apparently one neither is nor is one, if this argument is to be trusted." \stephpag{142 a} ``That seems to be true." ``But can that which does not exist have anything pertaining or belonging to it?" ``Of course not." ``Then the one has no name, nor is there any description or knowledge or perception or opinion of it." ``Evidently not." ``And it is neither named nor described nor thought of nor known, nor does any existing thing perceive it." ``Apparently not." ``Is it possible that all this is true about the one ?" ``I do not think so."

``Shall we then return to our hypothesis and see \stephpag{b} if a review of our argument discloses any new point of view?" ``By all means." ``We say, then, that if the one exists, we must come to an agreement about the consequences, whatever they may be, do we not?" ``Yes." ``Now consider the first point. If one is, can it be and not partake of being?" ``No, it cannot." ``Then the being of one will exist, but will not be identical with one; for if it were identical with one, it would not be the being of one, nor would one partake of it, \stephpag{c} but the statement that one is would be equivalent to the statement that one is one but our hypothesis is not if one is one, what will follow, but if one is. Do you agree?" ``Certainly." ``In the belief that one and being differ in meaning?" ``Most assuredly." ``Then if we say concisely 'one is,' it is equivalent to saying that one partakes of being?" ``Certainly." ``Let us again say what will follow if one is and consider whether this hypothesis must not necessarily show that one is of such a nature as to have parts." ``How does that come about ?" ``In this way: \stephpag{d} If being is predicated of the one which exists and unity is predicated of being which is one, and being and the one are not the same, but belong to the existent one of our hypothesis, must not the existent one be a whole of which the one and being are parts?" ``Inevitably." ``And shall we call each of these parts merely a part, or must it, in so far as it is a part, be called a part of the whole?" ``A part of the whole." ``Whatever one, then, exists is a whole and has a part." ``Certainly." ``Well then, can either of these two parts of existent one—unity and being—abandon the other? \stephpag{e} Can unity cease to be a part of being or being to be a part of unity?" ``No." ``And again each of the parts possesses unity and being, and the smallest of parts is composed of these two parts, and thus by the same argument any part whatsoever has always these two parts; for always unity has being and being has unity; \stephpag{143 a} and, therefore, since it is always becoming two, it can never be one." ``Certainly." ``Then it results that the existent one would be infinite in number?" ``Apparently."

``Let us make another fresh start." ``In what direction?" ``We say that the one partakes of being, because it is?" ``Yes." ``And for that reason the one, because it is, was found to be many." ``Yes." ``Well then, will the one, which we say partakes of being, if we form a mental conception of it alone by itself, without that of which we say it partakes, be found to be only one, or many?" ``One, I should say." \stephpag{b} ``Just let us see; must not the being of one be one thing and one itself another, if the one is not being, but, considered as one, partakes of being?" ``Yes, that must be so." ``Then if being is one thing and one is another, one is not other than being because it is one, nor is being other than one because it is being, but they differ from each other by virtue of being other and different." ``Certainly." ``Therefore the other is neither the same as one nor as being." ``Certainly not." ``Well, then, if we make a selection among them, \stephpag{c} whether we select being and the other, or being and one, or one and the other, in each instance we select two things which may properly be called both?" ``What do you mean?" ``I will explain. We can speak of being?" ``Yes." ``And we can also speak of one?" ``Yes, that too." ``Then have we not spoken of each of them?" ``Yes." ``And when I speak of being and one, do I not speak of both?" ``Certainly." ``And also when I speak of being and other, or other and one, in every case I speak of each pair as both?" \stephpag{d} ``Yes." ``If things are correctly called both, can they be both without being two?" ``They cannot." ``And if things are two, must not each of them be one?" ``Certainly." ``Then since the units of these pairs are together two, each must be individually one." ``That is clear." ``But if each of them is one, by the addition of any sort of one to any pair whatsoever the total becomes three?" ``Yes." ``And three is an odd number, and two is even?" ``Of course." \stephpag{e} ``Well, when there are two units, must there not also be twice, and when there are three, thrice, that is, if two is twice one and three is thrice one?" ``There must." ``But if there are two and twice, must there not also be twice two? And again, if there are three and thrice, must there not be thrice three?" ``Of course." ``Well then, if there are three and twice and two and thrice, must there not also be twice three and thrice two?" ``Inevitably." ``Then there would be even times even, \stephpag{144 a} odd times odd, odd times even, and even times odd." ``Yes." ``Then if that is true, do you believe any number is left out, which does not necessarily exist?" ``By no means." ``Then if one exists, number must also exist." ``It must." ``But if number exists, there must be many, indeed an infinite multitude, of existences; or is not number infinite in multitude and participant of existence?" ``Certainly it is." ``Then if all number partakes of existence, every part of number will partake of it?" ``Yes." \stephpag{b} ``Existence, then, is distributed over all things, which are many, and is not wanting in any existing thing from the greatest to the smallest? Indeed, is it not absurd even to ask that question? For how can existence be wanting in any existing thing?" ``It cannot by any means." ``Then it is split up into the smallest and greatest and all kinds of existences nothing else is so much divided, \stephpag{c} and in short the parts of existence are infinite." ``That is true." ``Its parts are the most numerous of all." ``Yes, they are the most numerous." ``Well, is there any one of them which is a part of existence, but is no part?" ``How could that be?" ``But if there is, it must, I imagine, so long as it is, be some one thing; it cannot be nothing." ``That is inevitable." ``Then unity is an attribute of every part of existence and is not wanting to a smaller or larger or any other part." ``True." \stephpag{d} ``Can the one be in many places at once and still be a whole? Consider that question." ``I am considering and I see that it is impossible." ``Then it is divided into parts, if it is not a whole; for it cannot be attached to all the parts of existence at once unless it is divided." ``I agree." ``And that which is divided into parts must certainly be as numerous as its parts." ``It must." ``Then what we said just now—that existence was divided into the greatest number of parts—was not true for it is not divided, you see, into any more parts than one, \stephpag{e} but, as it seems, into the same number as one for existence is not wanting to the one, nor the one to existence, but being two they are equal throughout." ``That is perfectly clear." ``The one, then, split up by existence, is many and infinite in number." ``Clearly." ``Then not only the existent one is many, but the absolute one divided by existence, must be many." ``Certainly."

``And because the parts are parts of a whole, the one would be limited by the whole; \stephpag{145 a} or are not the parts included by the whole?" ``They must be so." ``But surely that which includes is a limit." ``Of course." ``Then the existent one is, apparently, both one and many, a whole and parts, limited and of infinite number." ``So it appears." ``Then if limited it has also extremes ?" ``Certainly." ``Yes, and if it is a whole, will it not have a beginning, a middle, and an end? Or can there be any whole without these three? And if any one of these is wanting, will it still be a whole?" ``It will not." \stephpag{b} ``Then the one, it appears, will have a beginning, a middle, and an end." ``It will." ``But surely the middle is equally distant from the extremes for otherwise it would not be a middle." ``No." ``And the one, apparently, being of such a nature, will partake of some shape, whether straight or round or a mixture of the two." ``Yes, it will."

``This being the case, will not the one be in itself and in other?" ``How is that?" ``Each of the parts doubtless is in the whole and none is outside of the whole." ``True." ``And all the parts are included in the whole ?" \stephpag{c} ``Yes." ``And surely the one is all its parts, neither more nor less than all." ``Certainly." ``But the whole is the one, is it not?" ``Of course." ``Then if all the parts are in the whole and all the parts are the one and the one is also the whole, and all the parts are included in the whole, the one will be included in the one, and thus the one will be in itself." ``Evidently." ``But the whole is not in the parts, neither in all of them nor in any. \stephpag{d} For if it is in all, it must be in one, for if it were wanting in any one it could no longer be in all; for if this one is one of all, and the whole is not in this one, how can it still be in all?" ``It cannot in any way." ``Nor can it be in some of the parts; for if the whole were in some parts, the greater would be in the less, which is impossible." ``Yes, it is impossible." ``But not being in one or several or all of the parts, it must be in something else or cease to be anywhere at all?" ``It must." ``And if it were nowhere, it would be nothing, but being a whole, since it is not in itself, it must be in something else, must it not?" \stephpag{e} ``Certainly." ``Then the one, inasmuch as it is a whole, is in other and inasmuch as it is all its parts, it is in itself; and thus one must be both in itself and in other." ``It must." ``This being its nature, must not the one be both in motion and at rest?" ``How is that?" ``It is at rest, no doubt, if it is in itself; for being in one, \stephpag{146 a} and not passing out from this, it is in the same, namely in itself." ``It is." ``But that which is always in the same, must always be at rest." ``Certainly." ``Well, then, must not, on the contrary, that which is always in other be never in the same, and being never in the same be not at rest, and being not at rest be in motion?" ``True." ``Then the one, being always in itself and in other, must always be in motion and at rest." ``That is clear."

``And again, it must be the same with itself and other than itself, \stephpag{b} and likewise the same with all other things and other than they, if what we have said is true." ``How is that?" ``Everything stands to everything in one of the following relations: it is either the same or other; or if neither the same or other, its relation is that of a part to a whole or of a whole to a part." ``Obviously." ``Now is the one a part of itself?" ``By no means." ``Then it cannot, by being a part in relation to itself, be a whole in relation to itself, as a part of itself." ``No, that is impossible." ``Nor can it be other than itself." \stephpag{c} ``Certainly not." ``Then if it is neither other nor a part nor a whole in relation to itself, must it not therefore be the same with itself?" ``It must." ``Well, must not that which is in another place than itself—the self being in the same place with itself—be other than itself, if it is to be in another place?" ``I think so." ``Now we saw that this was the case with one, for it was in itself and in other at the same time." ``Yes, we saw that it was so." ``Then by this reasoning the one appears to be other than itself." \stephpag{d} ``So it appears." ``Well then, if a thing is other than something, will it not be other than that which is other than it?" ``Certainly." ``Are not all things which are not one, other than one, and the one other than the not one?" ``Of course." ``Then the one would be other than the others." ``Yes, it is other." ``Consider; are not the absolute same and the absolute other opposites of one another?" ``Of course." ``Then will the same ever be in the other, or the other in the same?" ``No." ``Then if the other can never be in the same, there is no existing thing \stephpag{e} in which the other is during any time; for if it were in anything during any time whatsoever, the other would be in the same, would it not?" ``Yes, it would." ``But since the other is never in the same, it can never be in any existing thing." ``True." ``Then the other cannot be either in the not one or in the one." ``No, it cannot." ``Then not by reason of the other will the one be other than the not one or the not one other than the one." ``No." ``And surely they cannot by reason of themselves be other than one another, if they do not partake of the other." \stephpag{147 a} ``Of course not." ``But if they are not other than one another either by reason of themselves or by reason of the other, will it not be quite impossible for them to be other than one another at all?" ``Quite impossible." ``But neither can the not one partake of the one; for in that case they would not be not one, but would be one." ``True." ``Nor can the not one be a number; for in that case, too, since they would possess number, they would not be not one at all." ``No, they would not." ``Well, then, are the not one parts of the one?" ``Or would the not one in that case also partake of the one?" ``Yes, they would partake of it." \stephpag{b} ``If, then, in every way the one is one and the not one are not one, the one cannot be a part of the not one, nor a whole of which the not one are parts, nor are the not one parts of the one, nor a whole of which the one is a part." ``No." ``But we said that things which are neither parts nor wholes of one another, nor other than one another, are the same as one another." ``Yes, we did." ``Shall we say, then, that since the relations of the one and the not one are such as we have described, the two are the same as one another?" ``Yes, let us say that." ``The one, then, is, it appears, other than all other things and than itself, and is also the same as other things and as itself." \stephpag{c} ``That appears to be the result of our argument."

``Is it, then, also like and unlike itself and others?" ``Perhaps." ``At any rate, since it was found to be other than others, the others must also be other than it." ``Of course." ``Then it is other than the others just as the others are other than it, neither more nor less?" ``Certainly." ``And if neither more nor less, then in like degree?" ``Yes." ``In so far as it is so affected as to be other than the others and the others are affected in the same way in relation to the one, to that degree the one will be affected \stephpag{d} in the same way as the others and the others in the same way as the one." ``What do you mean?" ``I will explain. You give a particular name to a thing?" ``Yes." ``Well, you can utter the same name once or more than once?" ``Yes." ``And do you name that to which the name belongs when you utter it once, but not when you utter it many times? Or must you always mean the same thing when you utter the same name, whether once or repeatedly?" ``The same thing, of course." ``The word other is the name of something, is it not?" ``Certainly." \stephpag{e} ``Then when you utter it, whether once or many times, you apply it to nothing else, and you name nothing else, than that of which it is the name." ``Assuredly." ``Now when we say that the others are other than the one, and the one is other than the others, though we use the word other twice, we do not for all that apply it to anything else, but we always apply it to that nature of which it is the name." ``Certainly." \stephpag{148 a} ``In so far as the one is other than the others and the others are other than the one, the one and the others are not in different states, but in the same state; but whatever is in the same state is like, is it not?" ``Yes." ``Then in so far as the one is in the state of being other than the others, just so far everything is like all other things; for everything is other than all other things." ``So it appears." ``But the like is opposed to the unlike." ``Yes." ``And the other to the same." ``That is also true." ``But this, too, was shown, that the one is the same as the others." \stephpag{b} ``Yes, it was." ``And being the same as the others is the opposite of being other than the others." ``Certainly." ``In so far as it was other it was shown to be like." ``Yes." ``Then in so far as it is the same it will be unlike, since it has a quality which is the opposite of the quality which makes it like, for the other made it like." ``Yes." ``Then the same will make it unlike; otherwise the same will not be the opposite of the other." \stephpag{c} ``So it appears." ``Then the one will be both like and unlike the others, like in so far as it is other, unlike in so far as it is the same." ``Yes, that sort of conclusion seems to be tenable." ``But there is another besides." ``What is it?" ``In so far as it is in the same state, the one is not in another state, and not being in another state it is not unlike, and not being unlike it is like but in so far as it is in another state, it is of another sort, and being of another sort it is unlike." ``True." ``Then the one, because it is the same as the others and because it is other than the others, for both these reasons or for either of them would be both like and unlike the others." \stephpag{d} ``Certainly." ``And likewise, since it has been shown to be other than itself and the same as itself, the one will for both these reasons or for either of them be both like and unlike itself." ``That is inevitable." ``Now, then, consider the question whether the one touches or does not touch itself and other things." ``I am considering." ``The one was shown, I think, to be in the whole of itself." ``Right." ``And the one is also in other things?" ``Yes." ``Then by reason of being in the others \stephpag{e} it would touch them, and by reason of being in itself it would be prevented from touching the others, but would touch itself, since it is in itself." ``That is clear." ``Thus the one would touch itself and the other things." ``It would." ``But how about this? Must not everything which is to touch anything be next to that which it is to touch, and occupy that position which, being next to that of the other, touches it?" ``It must." ``Then the one, if it is to touch itself, must lie next to itself and occupy the position next to that in which it is." ``Yes, it must." \stephpag{149 a} ``The one, then, might do this if it were two, and might be in two places at once; but so long as it is one, it will not?" ``No, it will not." ``The one can no more touch itself than it can be two." ``No." ``Nor, again, will it touch the others." ``Why not?" ``Because, as we agreed, that which is to touch anything must be outside of that which it is to touch, and next it, and there must be no third between them." ``True." ``Then there must be two, at least, if there is to be contact." ``There must." ``And if \stephpag{b} to the two a third be added in immediate succession, there will be three terms and two contacts." ``Yes." ``And thus whenever one is added, one contact also is added, and the number of contacts is always one less than the number of terms; for every succeeding number of terms exceeds the number of all the contacts just as much as the first two terms exceeded the number of their contacts. \stephpag{c} For after the first every additional term adds one to the number of contacts." ``Right." ``Then whatever the number of terms, the contacts are always one less." ``True." ``But if only one exists, and not two, there can be no contact." ``Of course not." ``We affirm that those things which are other than one are not one and do not partake of oneness, since they are other." ``They do not." ``Then there is no number in others, if one is not in them." ``Of course not." ``Then the others are neither one nor two, \stephpag{d} nor have they the name of any other number." ``No." ``The one is, then, only one, and there can be no two." ``That is clear." ``There is no contact if there are no two terms." ``No, there is none." ``Then the one does not touch the others, nor the others the one, since there is no contact." ``No, certainly not." ``Thus on all these grounds the one touches and does not touch itself and the others." ``So it appears."

``And is the one both equal and unequal to itself and the others?" ``How is that?" ``If the one were greater or less than the others, \stephpag{e} or, again, the others greater or less than the one, is it not true that the one, considered merely as one, and the others, considered merely as others, would be neither greater nor less than one another, so far as their own natures are concerned; but if in addition to their own natures, they both possessed equality, they would be equal to one another or if the others possessed greatness and the one smallness, or vice versa, that class to which greatness was added would be greater, and that to which smallness was added would be smaller?" ``Certainly." ``These two ideas, greatness and smallness, exist, do they not?" ``For if they did not exist, they could not be opposites of one another and could not come into being in things." ``That is obvious." \stephpag{150 a} ``Then if smallness comes into being in the one, it would be either in a part or in the whole of it." ``Necessarily." ``What if it be in the whole of one?" ``Will it not either be on an equality with the one, extending throughout the whole of it, or else contain it?" ``Clearly." ``And if smallness be on an equality with the one, will it not be equal to the one, and if it contain the one, greater than the one?" ``Of course." ``But can smallness be equal to anything or greater than anything, performing the functions of greatness or equality and not its own functions?" \stephpag{b} ``No, it cannot." ``Then smallness cannot exist in the whole of the one, but, if at all, only in a part of it." ``Yes." ``And neither can it exist in a whole part, for then it will behave just as it did in relation to the whole; it will be equal to or greater than the part in which it happens to exist." ``Inevitably." ``Then smallness will never exist in anything, either in a part or in a whole, nor will anything be small except absolute smallness." ``So it appears." ``Nor will greatness exist in the one. \stephpag{c} For in that case, something other than absolute greatness and differing from it, namely that in which greatness exists, would be greater, and that although there is no smallness in it, which greatness must exceed, if it be great. But this is impossible, since smallness exists nowhere." ``True." ``But absolute greatness is not greater than anything but absolute smallness, and absolute smallness is not smaller than anything but absolute greatness." ``No." ``Then other things are neither greater nor smaller than the one, if they have neither greatness nor smallness, \stephpag{d} nor have even these two the power of exceeding or being exceeded in relation to the one, but only in relation to each other, nor can the one be greater or less than these two or than other things, since it has neither greatness nor smallness." ``Evidently not." ``Then if the one is neither greater nor smaller than the others, it can neither exceed them nor be exceeded by them?" ``Certainly not." ``Then that which neither exceeds nor is exceeded must be on an equality, and being on an equality, must be equal." ``Of course." \stephpag{e} ``And the one will be in the same relation to itself also; if it have in itself neither greatness nor smallness, it cannot be exceeded by itself or exceed itself; it would be on an equality with and equal to itself." ``Certainly." ``The one is, then, equal to itself and to the others." ``Evidently." ``But the one, being within itself, would also be contained by itself, and since it contains itself it would be greater than itself, and since it is contained by itself it would be less than itself; \stephpag{151 a} thus the one would be both greater and less than itself." ``Yes, it would." ``And is it true, moreover, that nothing can exist outside of the one and the others?" ``Of course." ``But that which exists must always exist somewhere." ``Yes." ``And that which exists in anything will be smaller and will exist in the greater? One thing cannot exist in another in any other way, can it?" ``No, it cannot." ``But since there is nothing else apart from the one and the others, and they must be in something, must they not be in one another, the others in the one and the one in the others, \stephpag{b} or else be nowhere at all?" ``Clearly." ``And because the one is in the others, the others will be greater than the one, since they contain it, and the one less than the others, since it is contained; but because the others are in the one, the one will by the same reasoning be greater than the others, and the others less than the one." ``So it appears." ``Then the one is equal to and greater and less than itself and the others." ``Evidently." ``And if equal and greater and less, it will be of equal and more and \stephpag{c} less measures with itself and the others, and since of equal, more, and less measures, of equal, more, and less parts." ``Of course." ``And being of equal and more and less measures, it will be less and more in number than itself and the others and likewise equal in number to itself and the others." ``How is that?" ``If it is greater than any things, it will be of more measures than they; and of as many parts as measures. Similarly if it is less or equal, the number of parts will be less or equal." ``True." ``Then one, being greater and less than itself \stephpag{d} and equal to itself, will be of more and less measures than itself and of equal measures with itself, and if of measures, of parts also?" ``Of course." ``And being of equal parts with itself, it will also be equal in number to itself, and if of more parts, more in number, and if of less parts, less in number than itself." ``Clearly." ``And will not the one possess the same relation towards other things?" ``Because it is shown to be greater than they, must it not also be more in number than they and because it is smaller, less in number? And because it is equal in size, must it not be also, equal in number to the others?" ``Yes, it must." \stephpag{e} ``And so once more, as it appears, the one will be equal to, greater than, and less than itself and other things in number." ``Yes, it will."

``And does the one partake of time and if it partakes of time, is it and does it become younger and older than itself and other things, and neither younger nor older than itself and the others?" ``What do you mean?" ``If one is, it is thereby shown to be." ``Yes." ``But is 'to be' anything else than participation in existence together with present time, \stephpag{152 a} just as 'was' denotes participation in existence together with past time, and 'will be' similar participation together with future time?" ``True." ``Then the one partakes of time if it partakes of being." ``Certainly." ``And the time in which it partakes is always moving forward?" ``Yes." ``Then it is always growing older than itself, if it moves forward with the time." ``Certainly." ``Now, do we not remember that there is something becoming younger when the older becomes older than it?" ``Yes, we do." ``Then the one, since it becomes older than itself, \stephpag{b} would become older than a self which becomes younger?" ``There is no doubt of it." ``Thus the one becomes older and younger than itself." ``Yes." ``And it is older (is it not) when in becoming older it is in the present time, between the past and the future; for in going from the past to the future it cannot avoid the present." ``No, it cannot." ``Then is it not the case that it ceases to become older \stephpag{c} when it arrives at the present, and no longer becomes, but actually is older? For while it moves forward it can never be arrested by the present, since that which moves forward touches both the present and the future, letting the present go and seizing upon the future, proceeding or becoming between the two, the present and the future." ``True." ``But if everything that is becoming is unable to avoid and pass by the present, then when it reaches the present it always ceases to become \stephpag{d} and straightway is that which it happens to be becoming." ``Clearly." ``The one, then, when in becoming older it reaches the present, ceases to become and straightway is older." ``Certainly." ``It therefore is older than that than which it was becoming older; and it was becoming older than itself." ``Yes." ``And that which is older is older than that which is younger, is it not?" ``It is." ``Then the one is younger than itself, when in becoming older it reaches the present." \stephpag{e} ``Undoubtedly." ``But the present is inseparable from the one throughout its whole existence; for it always is now whenever it is." ``Of course." ``Always, then, the one is and is becoming younger than itself." ``So it appears." ``And is it or does it become for a longer time than itself, or for an equal time?" ``For an equal time." ``But that which is or becomes for an equal time is of the same age." ``Of course." ``But that which is of the same age is neither older nor younger." ``No." ``Then the one, since it is and becomes for an equal time with itself, neither is nor becomes older or younger than itself." ``I agree." ``Well, then, is it or does it become older or younger than other things?" \stephpag{153 a} ``I cannot tell." ``But you can at any rate tell that the others, if they are others, not an other—plural, not singular—are more than one; for if they were an other, they would be one; but since they are others, they are more than one and have multitude." ``Yes, they have." ``And being a multitude, they would partake of a number greater than one." ``Of course." ``Well, which shall we say come and have come into being first, the greater or the smaller numbers?" ``The smaller." ``Then the smallest comes into being first and that is the one, is it not?" \stephpag{b} ``Yes." ``The one, therefore, has come into being first of all things that have number; but all others also have number, if they are others and not an other." ``They have." ``And since it came into being first, it came into being, I suppose, before the others, and the others later; but things which have come into being later are younger than that which came into being before them and thus the other things would be younger than the one, and the one older than the other things." ``Yes, they would."

``Here is another question: Can the one have come into being contrary to its own nature, or is that impossible?" ``It is impossible." \stephpag{c} ``But surely the one was shown to have parts, a beginning, a middle, and an end." ``Yes." ``And the beginning of everything—of one and everything else alike—comes into being first, and after the beginning come all the other parts until the end arrives, do they not?" ``Certainly." ``And we shall say also that all these others are parts of the whole and the one, and that it has become one and whole at the moment when the end arrives." ``Yes, we shall say that." ``The end, I imagine, comes into being last; and at that moment the one naturally comes into being; \stephpag{d} so that if the absolute one cannot come into being contrary to its own nature, since it has come into being simultaneously with the end, its nature must be such that it comes into being after all the others." ``That is clear." ``Then the one is younger than the others and the others are older than the one." ``I think that is clear, too." ``Well, must not a beginning or any other part whatsoever of one or of anything else whatsoever, if it be a part, not parts, be one, since it is a part?" ``It must." \stephpag{e} ``Then the one would come into being simultaneously with the first part and with the second, and it is not wanting in any part which comes into being in addition to any part whatsoever which may precede it, until it reaches the end and becomes complete one; it will not be wanting in the middle, nor in the first, nor in the last, nor in any other part in the process of coming into being." ``True." ``Then one has the same age as all the others so that the absolute one, unless it is naturally contrary to nature, could not have come into being either before or after the others, but only simultaneously with them. \stephpag{154 a} And by this reasoning the one would be neither older nor younger than the others nor the others than the one, but of the same age; but by the previous reasoning the one would be both older and younger than the others, and likewise the others than the one." ``Certainly." ``In this state, then, it is and in this way it has come into being. But what about the one becoming older and younger than the others, and the others than the one, and becoming neither older nor younger? Is it the same with becoming as with being, or otherwise?" \stephpag{b} ``I cannot say." ``But I can say as much as this, that even if one thing be older than another, it cannot become older by any greater difference in age than that which existed at first, nor if younger can it become younger by any greater difference; for the addition of equals to unequals, whether in time or anything else whatsoever, makes the difference always equal to that which existed at first." ``Yes, of course." ``Then that which exists \stephpag{c} can never become older or younger than that which exists, if the difference in age is always the same; but it is and has become older, and the other is and has become younger, but it does not become so." ``True." ``And the one, since it exists, never becomes either older or younger than the other things." ``No, it does not." ``But see whether they become older and younger in this way." ``In what way?" ``Because the one was found to be older than the others, and the others than the one." ``What then?" ``When the one is older than the others, \stephpag{d} it has come into being a longer time than the others." ``Yes." ``Then consider again. If we add an equal to a greater and to a less time, will the greater differ from the less by the same or by a smaller fraction?" ``By a smaller fraction." ``Then the proportional difference in age which existed originally between the one and the others will not continue afterwards, but if an equal time be added to the one and the others, the difference in their ages will constantly diminish, will it not?" \stephpag{e} ``Yes." ``And that which differs less in age from something than before becomes younger than before in relation to those things than which it formerly was older?" ``Yes, it becomes younger." ``But if the one becomes younger, must not those other things in turn become older than formerly in relation to the one?" ``Certainly." ``Then that which came into being later, becomes older in relation to the older, which came into being earlier; yet it never is older, but is always becoming older; for the latter always tends towards being younger, \stephpag{155 a} and the former towards being older. And conversely the older becomes in the same way younger than the younger. For as they are moving in opposite directions, they are becoming the opposites of one another, the younger older than the older, and the older younger than the younger; but they cannot finish the process of becoming; for if they finished the process of becoming, they would no longer be becoming, they would be. But as the case is, they become older and younger than one another—the one becomes younger than the others, because, as we saw, it is older and came into being earlier, \stephpag{b} and the others are becoming older than the one, because they came into being later. By the same reasoning the others stand in the same relation to the one, since they were seen to be older than the one and to have come into being earlier." ``Yes, that is clear." ``Then from the point of view that no one thing becomes older or younger than another, inasmuch as they always differ by an equal number, the one cannot become older or younger than the others, nor the others than the one; but in so far as that which comes into being earlier must always differ by a different proportional part from that which comes into being later, \stephpag{c} and vice versa—from this point of view the one and the others must necessarily become both older and younger than one another, must they not?" ``Certainly." ``For all these reasons, then, the one both is and becomes both older and younger than both itself and the others, and neither is nor becomes either older or younger than either itself or the others." ``Perfectly true." ``But since the one partakes of time and can become older and younger, \stephpag{d} must it not also partake of the past, the future, and the present?" ``It must." ``Then the one was and is and will be and was becoming and is becoming and will become." ``Certainly." ``And there would be and was and is and will be something which is in relation to it and belongs to it?" ``Certainly." ``And there would be knowledge and opinion and perception of it; there must be, if we are now carrying on all this discussion about it." ``You are right." ``And it has a name and definition, is named and defined, \stephpag{e} and all the similar attributes which pertain to other things pertain also to the one." ``That is perfectly true."

``Let us discuss the matter once more and for the third time. If the one is such as we have described it, being both one and many and neither one nor many, and partakes of time, must it not, because one is, sometimes partake of being, and again because one is not, sometimes not partake of being?" ``Yes, it must." ``And can one, when it partakes of being, not partake of it, or partake of it when it does not partake of it?" ``No, it cannot." ``Then it partakes at one time and does not partake at another; for that is the only way in which it can partake and not partake of the same thing." \stephpag{156 a} ``True." ``And is there not also a time when it assumes being and when it gives it up? How can it sometimes have and sometimes not have the same thing, unless it receives it at some time and again loses it?" ``There is no other way at all." ``But would you not say that receiving existence is generation or becoming?" ``Yes." ``And losing existence is destruction?" ``Certainly." ``The one, then, as it appears, since it receives and loses existence, is generated and destroyed." \stephpag{b} ``Inevitably." ``And being one and many and being generated and destroyed, when it becomes one its existence as many is destroyed, and when it becomes many its existence as one is destroyed, is it not?" ``Certainly." ``And in becoming one and many, must it not be separated and combined?" ``Inevitably." ``And when it becomes like and unlike, it must be assimilated and dissimilated?" ``Yes." ``And when it becomes greater and smaller and equal, it must be increased and diminished and equalized?" \stephpag{c} ``Yes." ``And when being in motion it comes to rest, and when being at rest it changes to motion, it must itself be in no time at all." ``How is that?" ``It is impossible for it to be previously at rest and afterwards in motion, or previously in motion and afterwards at rest, without changing." ``Of course." ``And there is no time in which anything can be at once neither in motion nor at rest." ``No, there is none." ``And certainly it cannot change without changing." ``I should say not." ``Then when does it change? For it does not change when it is at rest or when it is in motion or when it is in time." \stephpag{d} ``No, it does not." ``Does this strange thing, then, exist, in which it would be at the moment when it changes?" ``What sort of thing is that?" ``The instant. For the instant seems to indicate a something from which there is a change in one direction or the other. For it does not change from rest while it is still at rest, nor from motion while it is still moving; but there is this strange instantaneous nature, something interposed between \stephpag{e} motion and rest, not existing in any time, and into this and out from this that which is in motion changes into rest and that which is at rest changes into motion." ``Yes, that must be so." ``Then the one, if it is at rest and in motion, must change in each direction; for that is the only way in which it can do both. But in changing, it changes instantaneously, and when it changes it can be in no time, and at that instant it will be neither in motion nor at rest." ``No." ``And will the case not be the same in relation to other changes?" ``When it changes from being to destruction \stephpag{157 a} or from not being to becoming, does it not pass into an intermediate stage between certain forms of motion and rest, so that it neither is nor is not, neither comes into being nor is destroyed?" ``Yes, so it appears." ``And on the same principle, when it passes from one to many or from many to one, it is neither one nor many, is neither in a process of separation nor in one of combination. And in passing from like to unlike or from unlike to like, it is neither like nor unlike, neither in a process of assimilation nor in one of dissimilation; \stephpag{b} and in passing from small to great and to equal and vice versa, it is neither small nor great nor equal, neither in a process of increase, nor of diminution, nor of equality." ``Apparently not." ``All this, then, would happen to the one, if the one exists." ``Yes, certainly."

``Must we not consider what is likely to happen to the other things, if the one exists?" ``We must." ``Shall we tell, then, what must happen to the things other than one, if one exists?" ``Let us do so." ``Well, since they are other than the one, the other things are not the one for if they were, they would not be other than the one." ``True." \stephpag{c} ``And yet surely the others are not altogether deprived of the one, but they partake of it in a certain way." ``In what way?" ``Because the others are other than the one by reason of having parts; for if they had no parts, they would be altogether one." ``True." ``But parts, we affirm, belong to that which is a whole." ``Yes, we affirm that they do." ``But the whole must be one composed of many and of this the parts are parts. For each of the parts must be a part, not of many, but of a whole." ``How is that?" ``If anything is a part of many, and is itself one of the many, it will be a part of itself, \stephpag{d} which is impossible, and of each one of the others, if it is a part of all. For if it is not a part of some particular one, it will be a part of the rest, with the exception of that one, and thus it will not be a part of each one, and not being a part of each one, it will not be a part of any one of the many. But that which belongs to none cannot belong, whether as a part or as anything else, to all those things to none of which it belongs." ``That is clear." ``Then the part is a part, not of the many nor of all, but of a single form and a single concept \stephpag{e} which we call a whole, a perfect unity created out of all this it is of which the part is a part." ``Certainly." ``If, then, the others have parts, they will partake of the whole and of the one." ``True." ``Then the things which are other than one must be a perfect whole which has parts." ``Yes, they must." ``And the same reasoning applies to each part for the part must partake of the one. For if each of the parts is a part, \stephpag{158 a} the word 'each' implies that it is one, separated from the rest, and existing by itself; otherwise it will not be 'each.'" ``True." ``But its participation in the one clearly implies that it is other than the one, for if not, it would not partake of the one, but would actually be one; but really it is impossible for anything except one itself to be one." ``Yes, it is impossible." ``And both the whole and the part must necessarily participate in the one; for the one will be a whole of which the parts are parts, and again each individual one which is a part of a whole will be a part of the whole." ``Yes." \stephpag{b} ``And will not the things which participate in the one be other than the one while participating in it?" ``Of course." ``But the things which are other than the one will be many; for if they were neither one nor more than one, they would not be anything." ``No."

``But since the things which participate in the one as a part and the one as a whole are more than one, must not those participants in the one be infinite in number?" ``How so?" ``Let us look at the question in this way. Is it not true that at the moment when they begin to participate in the one they are not one and do not participate in one?" \stephpag{c} ``Clearly." ``Then they are multitudes, in which the one is not, are they not?" ``Yes, they are multitudes." ``Well, then, if we should subtract from them in thought the smallest possible quantity, must not that which is subtracted, if it has no participation in one, be also a multitude, and not one?" ``It must." ``And always when we consider the nature of the class, which makes it other than one, whatever we see of it at any time will be unlimited in number, will it not?" ``Certainly." ``And, further, when each part becomes a part, \stephpag{d} straightway the parts are limited in relation to each other and to the whole, and the whole in relation to the parts." ``Undoubtedly." ``The result, then, to the things which are other than one, that from the one and the union of themselves with it there arises, as it appears, something different within themselves which gives them a limitation in relation to one another; but their own nature, when they are left to themselves, gives them no limits." ``So it appears." ``Then the things which are other than one, both as wholes and as parts, are both unlimited and partake of limitation." ``Certainly." \stephpag{e} ``And are they also both like and unlike one another and themselves?" ``How is that?" ``Inasmuch as they are all by their own nature unlimited, they are all in that respect affected in the same way." ``Certainly." ``And surely inasmuch as they all partake of limitation, they are all affected in the same way in that respect also." ``Obviously." ``And inasmuch as they are so affected as to be both limited and limitless, they are affected by affections which are the opposites of one another." \stephpag{159 a} ``Yes." ``But opposites are as unlike as possible." ``To be sure." ``Then with regard to either one of their two affections they are like themselves and each other, but with regard to both of them together they are utterly opposed and unlike." ``Yes, that must be true." ``Therefore the others are both like and unlike themselves and one another." ``So they are." ``And they are the same as one another and also other than one another, they are both in motion and at rest, and since we have proved these cases, we can easily show that the things \stephpag{b} which are other than one experience all the opposite affections." ``You are right."

``Then what if we now drop these matters as evident and again consider whether, if one is, the things other than one are as we have said, and there is no alternative." ``Certainly." ``Let us then begin at the beginning and ask, if one is, what must happen to the things which are other than one." ``By all means." ``Must not the one be separate from the others, and the others from the one?" ``Why is that?" ``Because there is nothing else besides these, \stephpag{c} which is other than one and other than the others. For when we have said 'one and the others' we have included all things." ``Yes, all things." ``Then there is nothing other than these, in which both the one and the others may be." ``No." ``Then the one and the others can never be in the same." ``Apparently not." ``Then they are separate?" ``Yes." ``And surely we say that what is truly one has no parts." ``How can it have parts?" ``Then the one cannot be in the others as a whole, nor can parts of it, if it is separate from the others and has no parts." ``Of course not." \stephpag{d} ``Then the others cannot partake of the one in any way; they can neither partake of any part of it nor of the whole." ``No, apparently not." ``The others are, then, not one in any sense, nor have they in themselves any unity." ``No." ``But neither are the others many; for if they were many, each of them would be one part of the whole; but actually the things that are other than one are not many nor a whole nor parts, since they do not participate in the one in any way." ``Right." ``Neither are the others two or three, nor are two or three in them, if they are entirely deprived of unity." \stephpag{e} ``True."

``Nor are the others either themselves like and unlike the one, nor are likeness and unlikeness in them; for if they were like and unlike or had likeness and unlikeness in them, the things which are other than the one would have in them two elements opposite to one another." ``That is clear." ``But it is impossible for that to partake of two things which does not even partake of one." ``Impossible." ``The others are, then, not like nor unlike nor both. \stephpag{160 a} For if they were like or unlike, they would partake of one of the two elements, and if they were both, of the two opposites and that was shown to be impossible." ``True."

``They are, then, neither the same nor other, nor in motion nor at rest, nor becoming nor being destroyed, nor greater nor less nor equal, and they experience no similar affections; for if the others are subject to such affections, they will participate in one and two and three and odd and even, \stephpag{b} in which we saw that they cannot participate, if they are in every way utterly deprived of unity." ``Very true." ``Therefore if one exists, the one is all things and nothing at all in relation both to itself and to all others." ``Perfectly true."

``Well, and ought we not next to consider what must happen if one does not exist?" ``Yes, we ought." ``What, then, is the sense of this hypothesis—if one does not exist?" ``Is it different in any way from this—if not one does not exist?" ``Certainly it is different." ``Is it merely different, \stephpag{c} or are the two expressions—if not one does not exist and if one does not exist—complete opposites?" ``They are complete opposites." ``Now if a person should say 'if greatness does not exist', 'if smallness does not exist,' or anything of that sort, would he not make it clear that in each case the thing he speaks of as not existing is different?" ``Certainly." ``And in our case does he not make it clear that he means, when he says 'if one is not,' that the thing which is not is different from other things, and do we not know what he means?" ``Yes, we do know." ``In the first place, then, he speaks of something which is known, and secondly of something different from other things, when he says 'one,' whether he adds to it that it is or that it is not; \stephpag{d} for that which is said to be non-existent is known none the less, and is known to be different from other things, is it not?" ``Certainly." ``Then we should begin at the beginning by asking: if one is not, what must follow? In the first place this must be true of the one, that there is knowledge of it, or else not even the meaning of the words 'if the one does not exist' would be known." ``True." ``And is it not also true that the others differ from the one, or it cannot be said to differ from the others?" ``Certainly." ``Then a difference belongs to the one in addition to knowledge; for when we say that the one differs from the others, \stephpag{e} we speak of a difference in the one, not in the others." ``That is clear." ``And the non-existent one partakes of 'that' and 'some' and 'this' and 'relation to this' and 'these' and all notions of that sort; for the one could not be spoken of, nor could the things which are other than one, nor could anything in relation to the one or belonging to it be or be spoken of, if the one did not partake of the notion some or of those other notions." ``True." ``It is impossible for the one to be, if it does not exist, \stephpag{161 a} but nothing prevents its partaking of many things; indeed it must do so, if that one of which we are speaking, and not something else, is not. But if neither the one, nor 'that,' is not, but we are speaking of something else, there is no use in saying anything at all;\footnote{i.e. if non-existence cannot be predicated either of the one (unitas) or of that (illuditas), but that of which we predicate non-existence is something else, then we may as well stop talking. It has just been affirmed that if that one of which we are speaking, and not something else, is not, then the one must partake of numerous attributes. Now it is affirmed that if the converse is true, further discussion is futile.} but if non-existence is the property of that one, and not of something else, then the one must partake of 'that' and of many other attributes." ``Yes, certainly."

``And it will possess unlikeness in relation to other things for the things which are other than one, being different, will be of a different kind." ``Yes." ``And are not things which are of a different kind also of another kind?" ``Of course." ``And things which are of another kind are unlike, are they not?" \stephpag{b} ``Yes, they are unlike." ``Then if they are unlike the one, the one is evidently unlike the things which are unlike it." ``Evidently." ``Then the one possesses unlikeness in relation to which the others are unlike." ``So it appears." ``But if it possesses unlikeness to the others, must it not possess likeness to itself?" ``How is that?" ``If the one possesses unlikeness to the one, our argument will not be concerned with that which is of the nature of the one, and our hypothesis will not relate to the one, but to something other than one." \stephpag{c} ``Certainly." ``But that is inadmissible." ``It certainly is." ``Then the one must possess likeness to itself." ``It must."

``And neither is the one equal to the others; for if it were equal, then it would both be and be like them in respect to equality, both of which are impossible, if one does not exist." ``Yes, they are impossible." ``And since it is not equal to the others, they cannot be equal to it, can they?" ``Certainly not." ``And things which are not equal are unequal, are they not?" ``Yes." ``And things which are unequal are unequal to something which is unequal to them?" ``Of course." ``Then the one partakes of inequality, in respect to which the others are unequal to it?" \stephpag{d} ``Yes, it does." ``But greatness and smallness are constituents of inequality." ``Yes." ``Then the one, such as we are discussing, possesses greatness and smallness?" ``So it appears." ``Now surely greatness and smallness always keep apart from one another." ``Certainly." ``Then there is always something between them." ``There is." ``Can you think of anything between them except equality?" ``No, only equality." ``Then anything which has greatness and smallness has also equality, which is between the two." ``That is clear." \stephpag{e} ``Then the non-existent one, it appears, partakes of equality and greatness and smallness." ``So it appears."

``And it must also, in a way, partake of existence." ``How is that?" ``It must be in such conditions as we have been saying; for if it were not, we should not be speaking the truth in saying that the one is not. And if we speak the truth, it is clear that we say that which is. Am I not right?" ``You are." ``Then inasmuch as we assert that we are speaking the truth, \stephpag{162 a} we necessarily assert that we say that which is." ``Necessarily." ``Then, as it appears, the non-existent one exists. For if it is not non-existent, but gives up something of being to not-being,\footnote{i.e. if it ceases to be non-existent, gives up something of being (as applied to non-existence) to not-being, so that it no longer is non-existent, but is not non-existent.} then it will be existent." ``Certainly." ``Then if it does not exist and is to continue to be non-existent, it must have the existence of not-being as a bond, just as being has the non-existence of not-being, in order to attain its perfect existence. For in this way the existence of the existent and the non-existence of the non-existent would be best assured, when the existent partakes of the existence of being existent and of the non-existence of not being non-existent, \stephpag{b} thus assuring its own perfect existence, and the non-existent partakes of the non-existence of not being existent and the existence of being non-existent, and thus the non-existent also secures its perfect non-existence." ``Very true." ``Then since the existent partakes of non-existence and the non-existent of existence, the one, since it does not exist, necessarily partakes of existence to attain non-existence." ``Yes, necessarily." ``Clearly, then, the one, if it does not exist, has existence." ``Clearly." ``And non-existence also, if it does not exist." ``Of course."

``Well, can anything which is in a certain condition be not in that condition without changing from it?" ``No, it cannot." ``Then everything of that sort—if a thing is and is not in a given condition—signifies a change." \stephpag{c} ``Of course." ``But change is motion; we agree to that?" ``It is motion." ``And did we not see that the one is and is not?" ``Yes." ``Then we see that it both is and is not in a given condition." ``So it appears." ``And we have seen that the non-existent one has motion, since it changes from being to not-being." ``There is not much doubt of that." ``But if it is nowhere among existing things—and it is nowhere, if it does not exist—it cannot move from any place to another." ``Of course not." ``Then its motion cannot be change of place." ``No, it cannot." \stephpag{d} ``Nor surely can it turn in the same spot, for it nowhere touches the same for the same is existent, and the non-existent cannot be in any existent thing." ``No, it is impossible." ``Then the one, being non-existent, cannot turn in that in which it is not." ``No." ``And the one, whether existent or non-existent, cannot change into something other than itself; for if it changed into something other than itself, our talk would no longer be about the one, but about something else." ``Quite right." ``But if it neither changes into something else, \stephpag{e} nor turns in the same spot, nor changes its place, can it still move in any way?" ``No how can it?" ``But surely that which is without motion must keep still, and that which keeps still must be at rest." ``Yes, it must." ``Then the non-existent one is both at rest and in motion." ``So it appears." ``And if it is in motion, it certainly must change in its nature; \stephpag{163 a} for if anything is moved in any way, in so far as it is moved it is no longer in its former condition, but in a different one." ``True." ``Then in moving, the one changes in nature." ``Yes." ``And yet when it does not move in any way, it will not change its nature in any way." ``No." ``Then in so far as the non-existent one moves, it changes, and in so far as it does not move, it does not change." ``True." ``Then the non-existent one both changes and does not change." ``So it appears." ``And must not that which changes come into a state of being other than its previous one, and perish, so far as its previous state is concerned; \stephpag{b} whereas that which does not change neither comes into being nor perishes?" ``That is inevitable." ``Then the non-existent one, when it is changed, comes into being and perishes, and when it is not changed, neither comes into being nor perishes and thus the non-existent one both comes into being and perishes and neither comes into being nor perishes." ``Quite true."

``Let us now go back again to the beginning and see whether the conclusions we reach will be the same as at present, or different." ``Yes, we should do that." ``We ask, then, if the one is not, \stephpag{c} what will be the consequences in regard to it?" ``Yes." ``Does the expression 'is not' denote anything else than the absence of existence in that of which we say that it is not?" ``No, nothing else." ``And when we say that a thing is not, do we mean that it is in a way and is not in a way? Or does the expression 'is not' mean without any qualifications that the non-existent is not in any way, shape, or manner, and does not participate in being in any way?" ``Without any qualifications whatsoever." ``Then the non-existent cannot be and cannot in any other way partake of existence." \stephpag{d} ``No." ``But were coming into being and perishing anything else than receiving and losing existence." ``No, nothing else." ``But that which has no participation in it can neither receive it nor lose it." ``Of course not." ``Then the one, since it does not exist in any way, cannot possess or lose or share in existence at all." ``That is reasonable." ``Then the non-existent one neither perishes nor comes into being, since it participates in no way in existence." ``No; that is clear." ``Then it is not changed in nature at all; \stephpag{e} for such change involves coming into being and perishing." ``True." ``And if it is not changed, it cannot move, either, can it?" ``Certainly not." ``And we cannot say that that which is nowhere is at rest; for that which is at rest must always be in some place which is the same." ``Yes, of course, the same place." ``Thus we shall say again that the non-existent one is neither at rest nor in motion." ``No, neither." ``Nor can anything which exists pertain to it for the moment it partook of anything which exists it would partake of existence." \stephpag{164 a} ``That is plain." ``Then neither greatness nor smallness nor equality pertains to it." ``No." ``Nor likeness nor difference, either in relation to itself or to other things." ``Clearly not." ``And can other things pertain to it, if nothing pertains to it?" ``Impossible." ``Then the other things are neither like it nor unlike it, nor the same nor different." ``No." ``Well, then, will the notions 'of that' or 'to that' or 'some,' or 'this' or 'of this' \stephpag{b} or 'of another' or 'to another' or past or future or present or knowledge or opinion or perception or definition or name or anything else which exists pertain to the non-existent?" ``No." ``Then the non-existent one has no state or condition whatsoever." ``It appears to have none whatsoever." ``Let us then discuss further what happens to the other things, if the one does not exist." ``Let us do so." ``Well, they must exist; for if others do not even exist, there could be no talking about the others." ``True." ``But if we talk about the others, the others are different. Or do you not regard the words other and different as synonymous?" \stephpag{c} ``Yes, I do." ``And we say that the different is different from the different, and the other is other than the other?" ``Yes." ``Then if the others are to be others, there must be something of which they will be others." ``Yes, there must be." ``Now what can that be? For they cannot be others of the one, if it does not exist." ``No." ``Then they are others of each other; for they have no alternative, except to be others of nothing." ``True." ``They are each, then, others of each other, in groups; for they cannot be so one at a time, if one does not exist. \stephpag{d} But each mass of them is unlimited in number, and even if you take what seems to be the smallest bit, it suddenly changes, like something in a dream that which seemed to be one is seen to be many, and instead of very small it is seen to be very great in comparison with the minute fractions of it." ``Very true." ``Such masses of others would be others of each other, if others exist and one does not exist." ``Certainly." ``There will, then, be many masses, each of which appears to be one, but is not one, if one does not exist?" ``Yes." \stephpag{e} ``And they will seem to possess, number, if each seems to be one and they are many." ``Certainly." ``And some will seem to be even and others odd, but all that will be unreal, if the one does not exist." ``True." ``And there will, we assert, seem to be a smallest among them but this proves to be many and great in comparison with each of the many minute fractions." \stephpag{165 a} ``Of course." ``And each mass will be considered equal to the many minute fractions for it could not appear to pass from greater to smaller, without seeming to enter that which is between them; hence the appearance of equality." ``That is reasonable." ``And although it has a limit in relation to another mass, it has neither beginning nor limit nor middle in relation to itself?" ``Why is that?" ``Because whenever the mind conceives of any of these as belonging to the masses, another beginning appears before the beginning, \stephpag{b} another end remains after the end, and in the middle are other more central middles than the middle, but smaller, because it is impossible to conceive of each one of them, since the one does not exist." ``Very true." ``So all being which is conceived by any mind must, it seems to me, be broken up into minute fractions; for it would always be conceived as a mass devoid of one." ``Certainly." ``Now anything of that sort, if seen from a distance and dimly, must appear to be one, \stephpag{c} but if seen from close at hand and with keen vision, each apparent one must prove to be unlimited in number, if it is really devoid of one, and one does not exist. Am I right?" ``That is perfectly conclusive." ``Therefore the other things must each and all appear to be unlimited and limited and one and many, if the things other than one exist and one does not." ``Yes, they must." ``And will they not also appear to be like and unlike?" ``Why?" ``Just as things in a picture, when viewed from a distance, appear to be all in one and the same condition and alike." \stephpag{d} ``Certainly." ``But when you come close to them they appear to be many and different, and, because of their difference in appearance, different in kind and unlike each other." ``Yes." ``And so the groups of the other things must appear to be like and unlike themselves and each other." ``Certainly." ``And also the same and different, and in contact with one another and separated, and in all kinds of motion and in every sort of rest, and coming into being and perishing, and neither of the two, and all that sort of thing, which we can easily mention in detail, \stephpag{e} if the many exist and the one does not." ``Very true."

``Let us, then, go back once more to the beginning and tell the consequences, if the others exist and the one does not." ``Let us do so." ``Well, the others will not be one?" ``Of course not." ``Nor will they be many for if they were many, one would be contained in them. And if none of them is one, they are all nothing, so that they cannot be many." ``True." ``If one is not contained in the others, the others are neither many nor one." \stephpag{166 a} ``No." ``And they do not even appear to be one or many." ``Why is that?" ``Because the others have no communion in any way whatsoever with anything which is non-existent, and nothing that is non-existent pertains to any of the others, for things that are non-existent have no parts." ``True." ``Nor is there any opinion or appearance of the non-existent in connection with the others, nor is the non-existent conceived of in any way whatsoever as related to the others." ``No." ``Then if one does not exist, \stephpag{b} none of the others will be conceived of as being one or as being many, either; for it is impossible to conceive of many without one." ``True, it is impossible." ``Then if one does not exist, the others neither are nor are conceived to be either one or many." ``No so it seems." ``Nor like nor unlike." ``No." ``Nor the same nor different, nor in contact nor separate, nor any of the other things which we were saying they appeared to be. The others neither are nor appear to be any of these, if the one does not exist." \stephpag{c} ``True." ``Then if we were to say in a word, 'if the one is not, nothing is,' should we be right?" ``Most assuredly." ``Then let us say that, and we may add, as it appears, that whether the one is or is not, the one and the others in relation to themselves and to each other all in every way are and are not and appear and do not appear." ``Very true."
\end{document}
