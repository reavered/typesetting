\begin{enumerate}
\item What is badness? It is that which thou hast often seen. And on the occasion of everything which happens keep this in mind, that it is that which thou hast often seen. Everywhere up and down thou wilt find the same things, with which the old histories are filled, those of the middle ages and those of our own day; with which cities and houses are filled now. There is nothing new: all things are both familiar and short-lived.

\item How can our principles become dead, unless the impressions [{\clarify thoughts}] which correspond to them are extinguished? But it is in thy power continuously to fan these thoughts into a flame. I can have that opinion about anything which I ought to have. If I can, why am I disturbed? The things which are external to my mind have no relation at all to my mind.—Let this be the state of thy affects, and thou standest erect. To recover thy life is in thy power. Look at things again as thou didst use to look at them; for in this consists the recovery of thy life.

\item The idle business of show, plays on the stage, flocks of sheep, herds, exercises with spears, a bone cast to little dogs, a bit of bread into fishponds, laborings of ants and burden-carrying, runnings about of frightened little mice, puppets pulled by strings—[{\clarify all alike}]. It is thy duty then in the midst of such things to show good humor and not a proud air; to understand however that every man is worth just so much as the things are worth about which he busies himself.

\item In discourse thou must attend to what is said, and in every movement thou must observe what is doing. And in the one thou shouldst see immediately to what end it refers, but in the other watch carefully what is the thing signified.

\item Is my understanding sufficient for this or not? If it is sufficient, I use it for the work as an instrument given by the universal nature. But if it is not sufficient, then either I retire from the work and give way to him who is able to do it better, unless there be some reason why I ought not to do so; or I do it as well as I can, taking to help me the man who with the aid of my ruling principle can do what is now fit and useful for the general good. For what-soever either by myself or with another I can do, ought to be directed to this only, to that which is useful and well suited to society.

\item How many after being celebrated by fame have been given up to oblivion; and how many who have celebrated the fame of others have long been dead.

\item Be not ashamed to be helped; for it is thy business to do thy duty like a soldier in the assault on a town. How then, if being lame thou canst not mount up on the battlements alone, but with the help of another it is possible?

\item Let not future things disturb thee, for thou wilt come to them, if it shall be necessary, having with thee the same reason which now thou usest for present things.

\item All things are implicated with one another, and the bond is holy; and there is hardly anything unconnected with any other thing. For things have been co-ordinated, and they combine to form the same universe [{\clarify order}]. For there is one universe made up of all things, and one god who pervades all things, and one substance,\footnote{ ``One substance," p. 42, note 1.} and one law, [{\clarify one}] common reason in all intelligent animals, and one truth; if indeed there is also one perfection for all animals which are of the same stock and participate in the reason.

\item Everything material soon disappears in the substance of the whole; and everything formal [{\clarify causal}] is very soon taken back into the universal reason; and the memory of everything is very soon overwhelmed in time.

\item To the rational animal the same act is according to nature and according to reason.

\item Be thou erect, or be made erect (iii. 5).

\item Just as it is with the members in those bodies which are united in one, so it is with rational beings which exist separate, for they have been constituted for one co-operation. And the perception of this will be more apparent to thee if thou often sayest to thyself that I am a member \textgreek{μέλος} of the system of rational beings. But if [{\clarify using the letter r}] thou sayest that thou art a part \textgreek{μέρος}, thou dost not yet love men from thy heart; beneficence does not yet delight thee for its own sake;\footnote{I have used Gataker's conjecture \textgreek{καταληκτικῶς} instead of the common reading \textgreek{καταληπτικῶς}: compare iv. 20; ix. 42.} thou still doest it barely as a thing of propriety, and not yet as doing good to thyself.

\item Let there fall externally what will on the parts which can feel the effects of this fall. For those parts which have felt will complain, if they choose. But I, unless I think that what has happened is an evil, am not injured. And it is in my power not to think so.

\item Whatever any one does or says, I must be good; just as if the gold, or the emerald, or the purple, were always saying this. Whatever any one does or says, I must be emerald and keep my color.

\item The ruling faculty does not disturb itself; I mean, does not frighten itself or cause itself pain. But if any one else can frighten or pain it, let him do so. For the faculty itself will not by its own opinion turn itself into such ways. Let the body itself take care, if it can, that it suffer nothing, and let it speak, if it suffers. But the soul itself, that which is subject to fear, to pain, which has completely the power of forming an opinion about these things, will suffer nothing, for it will never deviate into such a judgment. The leading principle in itself wants nothing, unless it makes a want for itself; and therefore it is both free from perturbation and unimpeded, if it does not disturb and impede itself.

\item Eudaemonia [{\clarify happiness}] is a good daemon, or a good thing. What then art thou doing here, O imagination? Go away, I entreat thee by the gods, as thou didst come, for I want thee not. But thou art come according to thy old fashion. I am not angry with thee: only go away.

\item Is any man afraid of change? Why, what can take place without change? What then is more pleasing or more suitable to the universal nature? And canst thou take a bath unless the wood undergoes a change? and canst thou be nourished, unless the food undergoes a change? And can anything else that is useful be accomplished without change? Dost thou not see then that for thyself also to change is just the same, and equally necessary for the universal nature?

\item Through the universal substance as through a furious torrent all bodies are carried, being by their nature united with and co-operating with the whole, as the parts of our body with one another. How many a Chrysippus, how many a Socrates, how many an Epictetus has time already swallowed up! And let the same thought occur to thee with reference to every man and thing (v. 23; vi. 15).

\item One thing only troubles me, lest I should do something which the constitution of man does not allow, or in the way which it does not allow, or what it does not allow now.

\item Near is thy forgetfulness of all things; and near the forgetfulness of thee by all.

\item It is peculiar to man to love even those who do wrong. And this happens, if when they do wrong it occurs to thee that they are kinsmen, and that they do wrong through ignorance and unintentionally, and that soon both of you will die; and above all, that the wrong-doer has done thee no harm, for he has not made thy ruling faculty worse than it was before.

\item The universal nature out of the universal substance, as if it were wax, now moulds a horse, and when it has broken this up, it uses the material for a tree, then for a man, then for something else; and each of these things subsists for a very short time. But it is no hardship for the vessel to be broken up, just as there was none in its being fastened together (viii. 50).

\item A scowling look is altogether unnatural; when it is often assumed,\footnote{This is corrupt.} the result is that all comeliness dies away, and at last is so completely extinguished that it cannot be again lighted up at all. Try to conclude from this very fact that it is contrary to reason. For if even the perception of doing wrong shall depart, what reason is there for living any longer?

\item Nature which governs the whole will soon change all things thou seest, and out of their substance will make other things, and again other things from the substance of them, in order that the world may be ever new (xii. 23).

\item When a man has done thee any wrong, immediately consider with what opinion about good or evil he has done wrong. For when thou hast seen this, thou wilt pity him, and wilt neither wonder nor be angry. For either thou thyself thinkest the same thing to be good that he does, or another thing of the same kind. It is thy duty then to pardon him. But if thou dost not think such things to be good or evil, thou wilt more readily be well disposed to him who is in error.

\item Think not so much of what thou hast not as of what thou hast: but of the things which thou hast select the best, and then reflect how eagerly they would have been sought, if thou hadst them not. At the same time, however, take care that thou dost not through being so pleased with them accustom thyself to overvalue them, so as to be disturbed if ever thou shouldst not have them.

\item Retire into thyself. The rational principle which rules has this nature, that it is content with itself when it does what is just, and so secures tranquility.

\item Wipe out the imagination. Stop the pulling of the strings. Confine thyself to the present. Understand well what happens either to thee or to another. Divide and distribute every object into the causal [{\clarify formal}] and the material. Think of thy last hour. Let the wrong which is done by a man stay there where the wrong was done (viii. 29).

\item Direct thy attention to what is said. Let thy understanding enter into the things that are doing and the things which do them (vii. 4).

\item Adorn thyself with simplicity and modesty, and with indifference towards the things which lie between virtue and vice. Love mankind. Follow God. The poet says that law rules all— And it is enough to remember that law rules all.\footnote{The end of this section is unintelligible.}

\item About death: whether it is a dispersion, or a resolution into atoms, or annihilation, it is either extinction or change.

\item About pain: the pain which is intolerable carries us off; but that which lasts a long time is tolerable; and the mind maintains its own tranquility by retiring into itself, and the ruling faculty is not made worse. But the parts which are harmed by pain, let them, if they can, give their opinion about it.

\item About fame: look at the minds [{\clarify of those who seek fame}], observe what they are, and what kind of things they avoid, and what kind of things they pursue. And consider that as the heaps of sand piled on one another hide the former sands; so in life the events which go before are soon covered by those which come after.

\item From Plato:\footnote{Plato, \textit{Politics}, vi. 486.} The man who has an elevated mind and takes a view of all time and of all substance, dost thou suppose it possible for him to think that human life is anything great? It is not possible, he said.—Such a man then will think that death also is no evil.—Certainly not.

\item From Antisthenes: It is royal to do good and to be abused.

\item It is a base thing for the countenance to be obedient and to regulate and compose itself as the mind commands, and for the mind not to be regulated and composed by itself.

\item It is not right to vex ourselves at things, For they care nought about it.\footnote{From the \textit{Bellerophon} of Euripides.}

\item To the immortal gods and us give joy.

\item Life must be reaped like the ripe ears of corn. One man is born; another dies.\footnote{From the \textit{Hypsipyle} of Euripides. Cicero (\textit{Tusculanae Disputationes}, iii. 25) has translated six lines from Euripides, and among them are these two lines—
	\begin{displayquote}
		``Reddenda terrae est terra: tum vita omnibus	\\
		Metenda ut fruges: Sic jubet necessitas."
	\end{displayquote}
}

\item If gods care not for me and my children, There is a reason for it.

\item For the good is with me, and the just.\footnote{See Aristophanes, \textit{Acharnenses}, v. 661.}

\item No joining others in their wailing, no violent emotion.

\item From Plato:\footnote{From the \textit{Apologia}, c. 16.} But I would make this man a sufficient answer, which is this: Thou sayest not well, if thou thinkest that a man who is good for anything at all ought to compute the hazard of life or death, and should not rather look to this only in all that he does, whether he is doing what is just or unjust, and the works of a good or bad man.

\item For\footnote{From the \textit{Apologia}, c. 16.} thus it is, men of Athens, in truth: wherever a man has placed himself thinking it the best place for him, or has been placed by a commander, there in my opinion he ought to stay and to abide the hazard, taking nothing into the reckoning, either death or anything else, before the baseness [{\clarify of deserting his post}].

\item But, my good friend, reflect whether that which is noble and good is not something different from saving and being saved; for as to a man living such or such a time, at least one who is really a man, consider if this is not a thing to be dismissed from the thoughts: and there must be no love of life: but as to these matters a man must intrust them to the Deity and believe what the women say, that no man can escape his destiny, the next inquiry being how he may best live the time that he has to live.\footnote{Plato, \textit{Gorgias}, c. 68 (512). In this passage the text of Antoninus has \textgreek{ἐατέον}, which is perhaps right; but there is a difficulty in the words \textgreek{μὴ γὰρ τοῠτο μέν, τὸ ζῆν ὁποσονδὴ χρόνον τόνγε ὡς ἀληθῶς ἄνδρα ἐατέον ἐστί καὶ οὐ} , etc. The conjecture \textgreek{εὐκτέον} for \textgreek{ἐατέον} does not mend the matter.}

\item Look round at the courses of the stars, as if thou wert going along with them; and constantly consider the changes of the elements into one another, for such thoughts purge away the filth of the terrene life.

\item This is a fine saying of Plato:\footnote{It is said that this is not in the extant writings of Plato.} That he who is discoursing about men should look also at earthly things as if he viewed them from some higher place; should look at them in their assemblies, armies, agricultural labors, marriages, treaties, births, deaths, noise of the courts of justice, desert places, various nations of barbarians, feasts, lamentations, markets, a mixture of all things and an orderly combination of contraries.

\item Consider the past—such great changes of political supremacies; thou mayest foresee also the things which will be. For they will certainly be of like form, and it is not possible that they should deviate from the order of the things which take place now; accordingly to have contemplated human life for forty years is the same as to have contemplated it for ten thousand years. For what more wilt thou see?

\item

\par

\begin{displayquote}
 \hspace{-1.9em}
That which has grown from the earth to the earth, \\
But that which has sprung from heavenly seed, \\
Back to the heavenly realms returns.\footnote{From the \textit{Chrysippus} of Euripides.}
\end{displayquote}

This is either a dissolution of the mutual involution of the atoms, or a similar dispersion of the unsentient elements.

\item

\par

\begin{displayquote}
\hspace{-1.9em}
With food and drinks and cunning magic arts \\
Turning the channel's course to `scape from death. \\
The breeze which heaven has sent \\
We must endure, and toil without complaining.\footnote{The first two lines are from the \textit{Supplices} of Euripides, v. 1110}
\end{displayquote}

\item Another may be more expert in casting his opponent; but he is not more social, nor more modest, nor better disciplined to meet all that happens, nor more considerate with respect to the faults of his neighbors.

\item Where any work can be done conformably to the reason which is common to gods and men, there we have nothing to fear; for where we are able to get profit by means of the activity which is successful and proceeds according to our constitution, there no harm is to be suspected.

\item Everywhere and at all times it is in thy power piously to acquiesce in thy present condition, and to behave, justly to those who are about thee, and to exert thy skill upon thy present thoughts, that nothing shall steal into them without being well examined.

\item Do not look around thee to discover other men's ruling principles, but look straight to this, to what nature leads thee, both the universal nature through the things which happen to thee, and thy own nature through the acts which must be done by thee. But every being ought to do that which is according to its constitution; and all other things have been constituted for the sake of rational beings, just as among irrational things the inferior for the sake of the superior, but the rational for the sake of one another.

The prime principle then in man's constitution is the social. And the second is not to yield to the persuasions of the body—for it is the peculiar office of the rational and intelligent motion to circumscribe itself, and never to be overpowered either by the motion of the senses or of the appetites, for both are animal: but the intelligent motion claims superiority, and does not permit itself to be overpowered by the others. And with good reason, for it is formed by nature to use all of them. The third thing in the rational constitution is freedom from error and from deception. Let then the ruling principle holding fast to these things go straight on, and it has what is its own.

\item Consider thyself to be dead, and to have completed thy life up to the present time; and live according to nature the remainder which is allowed thee.

\item Love that only which happens to thee and is spun with the thread of thy destiny. For what is more suitable?

\item In everything which happens keep before thy eyes those to whom the same things happened, and how they were vexed, and treated them as strange things, and found fault with them: and now where are they? Nowhere. Why then dost thou too choose to act in the same way? and why dost thou not leave these agitations which are foreign to nature to those who cause them and those who are moved by them; and why art thou not altogether intent upon the right way of making use of the things which happen to thee? For then thou wilt use them well, and they will be a material for thee [{\clarify to work on}]. Only attend to thyself, and resolve to be a good man in every act which thou doest: and remember\ldots\footnote{This section is obscure, and the conclusion is so corrupt that it is impossible to give any probable meaning to it. It is better to leave it as it is than to patch it up, as some critics and translators have done.}

\item Look within. Within is the fountain of good, and it will ever bubble up, if thou wilt ever dig.

\item The body ought to be compact, and to show no irregularity either in motion or attitude. For what the mind shows in the face by maintaining in it the expression of intelligence and propriety, that ought to be required also in the whole body. But all these things should be observed without affectation.

\item The art of life is more like the wrestler's art than the dancer's, in respect of this, that it should stand ready and firm to meet onsets which are sudden and unexpected.

\item Constantly observe who those are whose approbation thou wishest to have, and what ruling principles they possess. For then thou wilt neither blame those who offend involuntarily, nor wilt thou want their approbation, if thou lookest to the sources of their opinions and appetites.

\item Every soul, the philosopher says, is involuntarily deprived of truth; consequently in the same way it is deprived of justice and temperance and benevolence and everything of the kind. It is most necessary to bear this constantly in mind, for thus thou wilt be more gentle towards all.

\item In every pain let this thought be present, that there is no dishonor in it, nor does it make the governing intelligence worse, for it does not damage the intelligence either so far as the intelligence is rational\footnote{The text has \textgreek{ὑλίκῄ}, which it has been proposed to alter to \textgreek{λογίκῄ}, and this change is necessary. We shall then have in this section \textgreek{λογίκῄ} and \textgreek{κοίνωνίκῄ} associated, as we have in s. 68 \textgreek{λογίκῄ}; and \textgreek{πολίτίκῄ}, and in s. 72.} or so far as it is social. Indeed in the case of most pains let this remark of Epicurus aid thee, that pain is neither intolerable nor everlasting, if thou bearest in mind that it has its limits, and if thou addest nothing to it in imagination: and remember this too, that we do not perceive that many things which are disagreeable to us are the same as pain, such as excessive drowsiness, and the being scorched by heat, and the having no appetite. When then thou art discontented about any of these things, say to thyself that thou art yielding to pain.

\item Take care not to feel towards the inhuman as they feel towards men.\footnote{I have followed Gataker's conjecture \textgreek{οἱ ἀπάνθρωποι} instead of the MSS. reading \textgreek{οἱ ἄνθρωποι}.}

\item How do we know if Telauges was not superior in character to Socrates? For it is not enough that Socrates died a more noble death, and disputed more skilfully with the sophists, and passed the night in the cold with more endurance, and that when he was bid to arrest Leon\footnote{Leon of Salamis. See Plato, \textit{Epistles} 7; \textit{Apologia} c, 20; Epictetus, iv. I, 160; iv. 7, 30.} of Salamis, he considered it more noble to refuse, and that he walked in a swaggering way in the streets\footnote{Aristophanes \textit{The Clouds}, 362. \textgreek{ὅτι βρενθύεί τ᾽ ἐν ταὶσίν ὁδοῑς καὶ τὼ ὀφθαλμὼ παραβάλλεί}.}—though as to this fact one may have great doubts if it was true. But we ought to inquire what kind of a soul it was that Socrates possessed, and if he was able to be content with being just towards men and pious towards the gods, neither idly vexed on account of men's villainy, nor yet making himself a slave to any man's ignorance, nor receiving as strange anything that fell to his share out of the universal, nor enduring it as intolerable, nor allowing his understanding to sympathize with the affects of the miserable flesh.

\item Nature has not so mingled [{\clarify the intelligence}] with the composition of the body, as not to have allowed thee the power of circumscribing thyself and of bringing under subjection to thyself all that is thy own; for it is very possible to be a divine man and to be recognized as such by no one. Always bear this in mind; and another thing too, that very little indeed is necessary for living a happy life. And because thou hast despaired of becoming a dialectician and skilled in the knowledge of nature, do not for this reason renounce the hope of being both free and modest, and social and obedient to God.

\item It is in thy power to live free from all compulsion in the greatest tranquility of mind, even if all the world cry out against thee as much as they choose, and even if wild beasts tear in pieces the members of this kneaded matter which has grown around thee. For what hinders the mind in the midst of all this from maintaining itself in tranquility and in a just judgment of all surrounding things and in a ready use of the objects which are presented to it, so that the judgment may say to the thing which falls under its observation: This thou art in substance [{\clarify reality}], though in men's opinion thou mayest appear to be of a different kind; and the use shall say to that which falls under the hand: Thou art the thing that I was seeking; for to me that which presents itself is always a material for virtue both rational and political, and in a word, for the exercise of art, which belongs to man or God. For everything which happens has a relationship either to God or man, and is neither new nor difficult to handle, but usual and apt matter to work on.

\item The perfection of moral character consists in this, in passing every day as the last, and in being neither violently excited nor torpid nor playing the hypocrite.

\item The gods who are immortal are not vexed because during so long a time they must tolerate continually men such as they are and so many of them bad; and besides this, they also take care of them in all ways. But thou, who art destined to end so soon, art thou wearied of enduring the bad, and this too when thou art one of them?

\item It is a ridiculous thing for a man not to fly from his own badness, which is indeed possible, but to fly from other men's badness, which is impossible.

\item Whatever the rational and political [{\clarify social}] faculty finds to be neither intelligent nor social, it properly judges to be inferior to itself.

\item When thou hast done a good act and another has received it, why dost thou still look for a third thing besides these, as fools do, either to have the reputation of having done a good act or to obtain a return?

\item No man is tired of receiving what is useful. But it is useful to act according to nature. Do not then be tired of receiving what is useful by doing it to others.

\item The nature of the All moved to make the universe. But now either everything that takes place comes by way of consequence or [{\clarify continuity}]; or even the chief things towards which the ruling power of the universe directs its own movement are governed by no rational principle. If this is remembered, it will make thee more tranquil in many things (vi. 44; ix. 28).\footnote{It is not easy to understand this section. It has been suggested that there is some error in \textgreek{ἢ ἀλόγιστα}, etc. Some of the translators have made nothing of the passage, and they have somewhat perverted the words. The first proposition is, that the universe was made by some sufficient power. A beginning of the universe is assumed, and a power which framed an order. The next question is, How are things produced now? Or, in other words, by what power do forms appear in continuous succession? The answer, according to Antoninus, may be this: It is by virtue of the original constitution of things that all change and succession have been effected and are effected. And this is intelligible in a sense, if we admit that the universe is always one and the same, a continuity of identity; as much one and the same as man is one and the same—which he believes himself to be, though he also believes, and cannot help believing, that both in his body and in his thoughts there is change and succession. There is no real discontinuity then in the universe; and if we say that there was an order framed in the beginning, and that the things which are now produced are a consequence of a previous arrangement, we speak of things as we are compelled to view them, as forming a series of succession, just as we speak of the changes in our own bodies and the sequence of our own thoughts. But as there are no intervals, not even intervals infinitely small, between any two supposed states of any one thing, so there are no intervals, not even infinitely small, between what we call one thing and any other thing which we speak of as immediately preceding or following it. What we call time is an idea derived from our notion of a succession of things or events, an idea which is a part of our constitution, but not an idea which we can suppose to belong to an infinite intelligence and power. The conclusion then is certain that the present and the past, the production of present things and the supposed original order, out of which we say that present things now come, are one, and the present productive power and the so-called past arrangement are only different names for one thing. I suppose then that Antoninus wrote here as people sometimes talk now, and that his real meaning is not exactly expressed by his words. There are certainly other passages from which I think that we may collect that he had notions of production something like what I have expressed.

We now come to the alternate: ``or even the chief things\ldots principle." I do not exactly know what he means by \textgreek{τὰ κυριώτατα} ``the chief," or ``the most excellent," or whatever it is. But as he speaks elsewhere of inferior and superior things, and of the inferior being for the use of the superior, and of rational beings being the highest, he may here mean rational beings. He also in this alternative assumes a governing power of the universe, and that it acts by directing its power towards these chief objects, or making its special, proper motion towards them. And here he uses the noun (\textgreek{ὁρμῄ}) ``movement," which contains the same notion as the verb (\textgreek{ὡρμησε}) ``moved," which he used at the beginning of the paragraph, when he was speaking of the making of the universe. If we do not accept the first hypothesis, he says, we must take the conclusion of the second, that the ``chief things towards which the ruling power of the universe directs its own movement are governed by no rational principle." The meaning then is, if there is a meaning in it, that though there is a governing power which strives to give effect to its efforts, we must conclude that there is no rational direction of anything, if the power which first made the universe does not in some way govern it still. Besides, if we assume that anything is now produced or now exists without the action of the supreme intelligence, and yet that this intelligence makes an effort to act, we obtain a conclusion which cannot be reconciled with the nature of a supreme power, whose existence Antoninus always assumes. The tranquility that a man may gain from these reflections must result from his rejecting the second hypothesis and accepting the first—whatever may be the exact sense in which the emperor understood the first. Or, as he says elsewhere, if there is no Providence which governs the world, man has at least the power of governing himself according to the constitution of his nature; and so he may be tranquil if he does the best that he can.
If there is no error in the passage, it is worth the labor to discover the writer's exact meaning—for I think that he had a meaning, though people may not agree what it was. (Compare ix. 28.) If I have rightly explained the emperor's meaning in this and other passages, he has touched the solution of a great question.}
\end{enumerate}