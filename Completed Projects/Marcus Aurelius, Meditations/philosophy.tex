It has been said that the Stoic philosophy first showed its real value when it passed from Greece to Rome. The doctrines of Zeno and his successors were well suited to the gravity and practical good sense of the Romans; and even in the Republican period we have an example of a man, M. Cato Uticensis, who lived the life of a Stoic and died consistently with the opinions which he professed. He was a man, says Cicero, who embraced the Stoic philosophy from conviction; not for the purpose of vain discussion, as most did, but in order to make his life conformable to the Stoic precepts. In the wretched times from the death of Augustus to the murder of Domitian, there was nothing but the Stoic philosophy which could console and support the followers of the old religion under imperial tyranny and amidst universal corruption. There were even then noble minds that could dare and endure, sustained by a good conscience and an elevated idea of the purposes of man's existence. Such were Paetus Thrasae, Helvidius Priscus, Cornutus, C. Musonius Rufus,[1] and the poets Persius and Juvenal, whose energetic language and manly thoughts may be as instructive to us now as they might have been to their contemporaries. Persius died under Nero's bloody reign; but Juvenal had the good fortune to survive the tyrant Domitian and to see the better times of Nerva, Trajan, and Hadrian.[2] His best precepts are derived from the Stoic school, and they are enforced in his finest verses by the unrivalled vigor of the Latin language.

The best two expounders of the later Stoical philosophy were a Greek slave and a Roman emperor. Epictetus, a Phrygian Greek, was brought to Rome, we know not how, but he was there the slave and afterwards the freedman of an unworthy master, Epaphroditus by name, himself a freedman and a favorite of Nero. Epictetus may have been a hearer of C. Musonius Rufus, while he was still a slave, but he could hardly have been a teacher before he was made free. He was one of the philosophers whom Domitian's order banished from Rome. He retired to Nicopolis in Epirus, and he may have died there. Like other great teachers he wrote nothing, and we are indebted to his grateful pupil Arrian for what we have of Epictetus' discourses. Arrian wrote eight books of the discourses of Epictetus, of which only four remain and some fragments. We have also from Arrian's hand the small Enchiridion or Manual of the chief precepts of Epictetus. This is a valuable commentary on the Enchiridion by Simplicius, who lived in the time of the emperor Justinian.[3]

Antoninus in his first book (i. 7), in which he gratefully commemorates his obligations to his teachers, says that he was made acquainted by Junius Rusticus with the discourses of Epictetus, whom he mentions also in other passages (iv. 41; xi. 34, 36). Indeed, the doctrines of Epictetus and Antoninus are the same, and Epictetus is the best authority for the explanation of the philosophical language of Antoninus and the exposition of his opinions. But the method of the two philosophers is entirely different. Epictetus addressed himself to his hearers in a continuous discourse and in a familiar and simple manner. Antoninus wrote down his reflections for his own use only, in short, unconnected paragraphs, which are often obscure.

The Stoics made three divisions of philosophy,—Physic (\textgreek{φυσικόν}), Ethic (\textgreek{ἠθικόν}), and Logic (\textgreek{λογικόν}) (viii. 13). This division, we are told by Diogenes, was made by Zeno of Citium, the founder of the Stoic sect, and by Chrysippus; but these philosophers placed the three divisions in the following order,—Logic, Physic, Ethic. It appears, however, that this division was made before Zeno's time, and acknowledged by Plato, as Cicero remarks (Acad. Post. i. 5). Logic is not synonymous with our term Logic in the narrower sense of that word.

Cleanthes, a Stoic, subdivided the three divisions and made six,—Dialectic and Rhetoric, comprised in Logic; Ethic and Politic; Physic and Theology. This division was merely for practical use, for all Philosophy is one. Even among the earliest Stoics Logic, or Dialectic, does not occupy the same place as in Plato: it is considered only as an instrument which is to be used for the other divisions of Philosophy. An exposition of the earlier Stoic doctrines and of their modifications would require a volume. My object is to explain only the opinions of Antoninus, so far as they can be collected from his book.

According to the subdivision of Cleanthes, Physic and Theology go together, or the study of the nature of Things, and the study of the nature of the Deity, so far as man can understand the Deity, and of his government of the universe. This division or subdivision is not formally adopted by Antoninus, for, as already observed, there is no method in his book; but it is virtually contained in it.

Cleanthes also connects Ethic and Politic, or the study of the principles of morals and the study of the constitution of civil society; and undoubtedly he did well in subdividing Ethic into two parts. Ethic in the narrower sense and Politic; for though the two are intimately connected, they are also very distinct, and many questions can only be properly discussed by carefully observing the distinction. Antoninus does not treat of Politic. His subject is Ethic, and Ethic in its practical application to his own conduct in life as a man and as a governor. His Ethic is founded on his doctrines about man's nature, the Universal Nature, and the relation of every man to everything else. It is therefore intimately and inseparably connected with Physic, or the Nature of Things, and with Theology, or the Nature of the Deity. He advises us to examine well all the impressions on our minds (\textgreek{φαντασίαι}) and to form a right judgment of them, to make just conclusions, and to inquire into the meanings of words, and so far to apply Dialectic; but he has no attempt at any exposition of Dialectic, and his philosophy is in substance purely moral and practical. He says (viii. 13), "Constantly and, if it be possible, on the occasion of every impression on the soul,[4] apply to it the principles of Physic, of Ethic, and of Dialectic:"[5] which is only another way of telling us to examine the impression in every possible way. In another passage (iii. 11) he says, "To the aids which have been mentioned, let this one still be added: make for thyself a definition or description of the object (\textgreek{τὸ φανταστόν}) which is presented to thee, so as to see distinctly what kind of a thing it is in its substance, in its nudity, in its complete entirety, and tell thyself its proper name, and the names of the things of which it has been compounded, and into which it will be resolved." Such an examination implies a use of Dialectic, which Antoninus accordingly employed as a means toward establishing his Physical, Theological, and Ethical principles.

There are several expositions of the Physical, Theological, and Ethical principles, which are contained in the work of Antoninus; and more expositions than I have read. Ritter (Geschichte der Philosophie, iv. 241), after explaining the doctrines of Epictetus, treats very briefly and insufficiently those of Antoninus. But he refers to a short essay, in which the work is done better[6]. There is also an essay on the Philosophical Principles of M. Aurelius Antoninus by J.M. Schultz, placed at the end of his German translation of Antoninus (Schleswig, 1799). With the assistance of these two useful essays and his own diligent study, a man may form a sufficient notion of the principles of Antoninus; but he will find it more difficult to expound them to others. Besides the want of arrangement in the original and of connection among the numerous paragraphs, the corruption of the text, the obscurity of the language and the style, and sometimes perhaps the confusion in the writer's own ideas—besides all this, there is occasionally an apparent contradiction in the emperor's thoughts, as if his principles were sometimes unsettled, as if doubt sometimes clouded his mind. A man who leads a life of tranquility and reflection, who is not disturbed at home and meddles not with the affairs of the world, may keep his mind at ease and his thoughts in one even course. But such a man has not been tried. All his Ethical philosophy and his passive virtue might turn out to be idle words, if he were once exposed to the rude realities of human existence. Fine thoughts and moral dissertations from men who have not worked and suffered may be read, but they will be forgotten. No religion, no Ethical philosophy is worth anything, if the teacher has not lived the "life of an apostle," and been ready to die "the death of a martyr." "Not in passivity (the passive effects) but in activity lie the evil and the good of the rational social animal, just as his virtue and his vice lie not in passivity, but in activity" (ix. 16). The emperor Antoninus was a practical moralist. From his youth he followed a laborious discipline, and though his high station placed him above all want or the fear of it, he lived as frugally and temperately as the poorest philospher. Epictetus wanted little, and it seems that he always had the little that he wanted and he was content with it, as he had been with his servile station! But Antoninus after his accession to the empire sat on an uneasy seat. He had the administration of an empire which extended from the Euphrates to the Atlantic, from the cold mountains of Scotland to the hot sands of Africa; and we may imagine, though we cannot know it by experience, what must be the trials, the troubles, the anxiety, and the sorrows of him who has the world's business on his hands, with the wish to do the best that he can, and the certain knowledge that he can do very little of the good which he wishes.

In the midst of war, pestilence, conspiracy, general corruption, and with the weight of so unwieldy an empire upon him, we may easily comprehend that Antoninus often had need of all his fortitude to support him. The best and the bravest men have moments of doubt and of weakness; but if they are the best and the bravest, they rise again from their depression by recurring to first principles, as Antoninus does. The emperor says that life is smoke, a vapor, and St. James in his Epistle is of the same mind; that the world is full of envious, jealous, malignant people, and a man might be well content to get out of it. He has doubts perhaps sometimes even about that to which he holds most firmly. There are only a few passages of this kind, but they are evidence of the struggles which even the noblest of the sons of men had to maintain against the hard realities of his daily life. A poor remark it is which I have seen somewhere, and made in a disparaging way, that the emperor's reflections show that he had need of consolation and comfort in life, and even to prepare him to meet his death. True that he did need comfort and support, and we see how he found it. He constantly recurs to his fundamental principle that the universe is wisely ordered, that every man is a part of it and must conform to that order which he cannot change, that whatever the Deity has done is good, that all mankind are a man's brethren, that he must love and cherish them and try to make them better, even those who would do him harm. This is his conclusion (ii. 17): "What then is that which is able to conduct a man? One thing and only one, Philosophy. But this consists in keeping the divinity within a man free from violence and unharmed, superior to pains and pleasures, doing nothing without a purpose nor yet falsely and with hypocrisy, not feeling the need of another man's doing or not doing anything; and besides, accepting all that happens and all that is allotted, as coming from thence, wherever it is, from whence he himself came; and finally waiting for death with a cheerful mind as being nothing else than a dissolution of the elements of which every living being is compounded. But if there is no harm, to the elements themselves in each continually changing into another, why should a man have any apprehension about the change and dissolution of all the elements [himself]? for it is according to nature; and nothing is evil that is according to nature."

The Physic of Antoninus is the knowledge of the Nature of the Universe, of its government, and of the relation of man's nature to both. He names the universe (\textgreek{ἡ τῶν ὑλων οίσία} vi. 1),[7] "the universal substance," and he adds that "reason " (\textgreek{λόγος}) governs the universe. He also (vi. 9) uses the terms "universal nature" or "nature of the universe." He (vi. 25) calls the universe "the one and all, which we name Cosmos or Order" \textgreek{κόσμος}). If he ever seems to use these general terms as significant of the All, of all that man can in any way conceive to exist, he still on other occasions plainly distinguishes between Matter, Material things (\textgreek{ὕλη, ὑλίκόν}), and Cause, Origin, Reason (\textgreek{αἰτία, αἰτίῶδες, λόγος}).[8] This is conformable to Zeno's doctrine that there are two original principles (\textgreek{ἀρχαί}) of all things, that which acts (\textgreek{τὸ ποίοῦν}) and that which is acted upon (\textgreek{τὸ πάσχον}). That which is acted on is the formless matter (\textgreek{ὕλη}): that which acts is the reason (\textgreek{λόγος}), God, who is eternal and operates through all matter, and produces all things. So Antoninus (v. 32) speaks of the reason (\textgreek{λόγος}) which pervades all substance (\textgreek{οὐσία}), and through all time by fixed periods (revolutions) administers the universe (\textgreek{τὸ πᾶν}). God is eternal, and Matter is eternal. It is God who gives form to matter, but he is not said to have created matter. According to this view, which is as old as Anaxagoras, God and matter exist independently, but God governs matter. This doctrine is simply the expression of the fact of the existence both of matter and of God. The Stoics did not perplex themselves with the in-soluble question of the origin and nature of matter.[9] Antoninus also assumes a beginning of things, as we now know them; but his language is sometimes very obscure. I have endeavored to explain the meaning of one difficult passage (vii. 75, and the note).

Matter consists of elemental parts (\textgreek{στοίχεῖα}) of which all material objects are made. But nothing is permanent in form. The nature of the universe, according to Antoninus' expression (iv. 36), "loves nothing so much as to change the things which are, and to make new things like them. For everything that exists is in a manner the seed of that which will be. But thou art thinking only of seeds which are cast into the earth or into a womb: but this is a very vulgar notion." All things then are in a constant flux and change; some things are dissolved into the elements, others come in their places; and so the "whole universe continues ever young and perfect" (xii. 23).

Antoninus has some obscure expressions about what he calls "seminal principles" (\textgreek{σπερματίκοὶ λόγοί}). He opposes them to the Epicurean atoms (vi. 24), and consequently his "seminal principles" are not material atoms which wander about at hazard, and combine nobody knows how. In one passage (iv. 21) he speaks of living principles, souls (\textgreek{ψυχαὶ}) after the dissolution of their bodies being received into the "seminal principle of the universe." Schultz thinks that by "seminal principles Antoninus means the relations of the various elemental principles, which relations are determined by the Deity and by which alone the production of organized beings is possible." This may be the meaning; but if it is, nothing of any value can be derived from it.[10] Antoninus often uses the word "Nature" (\textgreek{φύσις}), and we must attempt to fix its meaning, The simple etymological sense of \textgreek{φύσις} is "production," the birth of what we call Things. The Romans used Natura, which also means "birth" originally. But neither the Greeks nor the Romans stuck to this simple meaning, nor do we. Antoninus says (x. 6): "Whether the universe is [a concourse of] atoms or Nature [is a system], let this first be established, that I am a part of the whole which is governed by nature." Here it might seem as if nature were personified and viewed as an active, efficient power; as something which, it not independent of the Deity, acts by a power which is given to it by the Deity. Such, if I understand the expression right, is the way in which the word Nature is often used now, though it is plain that many writers use the word without fixing any exact meaning to it. It is the same with the expression Laws of Nature, which some writers may use in an intelligible sense, but others as clearly use in no definite sense at all. There is no meaning in this word Nature, except that which Bishop Butler assigns to it, when he says, "The only distinct meaning of that word Natural is Stated, Fixed, or Settled; since what is natural as much requires and presupposes an intelligent agent to render it so, i.e., to effect it continually or at stated times, as what is supernatural or miraculous does to effect it at once." This is Plato's meaning (De Leg., iv. 715) when he says that God holds the beginning and end and middle of all that exists, and proceeds straight on his course, making his circuit according to nature (that is by a fixed order); and he is continually accompanied by justice, who punishes those who deviate from the divine law, that is, from the order or course which God observes.

When we look at the motions of the planets, the action of what we call gravitation, the elemental combination of unorganized bodies and their resolution, the production of plants and of living bodies, their generation, growth, and their dissolution, which we call their death, we observe a regular sequence of phenomena, which within the limits of experience present and past, so far as we know the past, is fixed and invariable. But if this is not so, if the order and sequence of phenomena, as known to us, are subject to change in the course of an infinite progression,—and such change is conceivable,—we have not discovered, nor shall we ever discover, the whole of the order and sequence of phenomena, in which sequence there may be involved according to its very nature, that is, according to its fixed order, some variation of what we now call the Order or Nature of Things. It is also conceivable that such changes have taken place,—changes in the order of things, as we are compelled by the imperfection of language to call them, but which are no changes; and further it is certain that our knowledge of the true sequence of all actual phenomena, as for instance the phenomena of generation, growth, and dissolution, is and ever must be imperfect.

We do not fare much better when we speak of Causes and Effects than when we speak of Nature. For the practical purposes of life we may use the terms cause and effect conveniently, and we may fix a distinct meaning to them, distinct enough at least to prevent all misunderstanding. But the case is different when we speak of causes and effects as of Things. All that we know is phenomena, as the Greeks called them, or appearances which follow one another in a regular order, as we conceive it, so that if some one phenomenon should fail in the series, we conceive that there must either be an interruption of the series, or that something else will appear after the phenomenon which has failed to appear, and will occupy the vacant place; and so the series in its progression may be modified or totally changed. Cause and effect then mean nothing in the sequence of natural phenomena beyond what I have said; and the real cause, or the transcendent cause, as some would call it, of each successive phenomenon is in that which is the cause of all things which are, which have been, and which will be forever. Thus the word Creation may have a real sense if we consider it as the first, if we can conceive a first, in the present order of natural phenomena; but in the vulgar sense a creation of all things at a certain time, followed by a quiescence of the first cause and an abandonment of all sequences of Phenomena to the laws of Nature, or to the other words that people may Use, is absolutely absurd.[11]

Now, though there is great difficulty in understanding all the passages of Antoninus, in which he speaks of Nature, of the changes of things and of the economy of the universe, I am convinced that his sense of Nature and Natural is the same as that which I have stated; and as he was a man who knew how to use words in a clear way and with strict consistency, we ought to assume, even if his meaning in some passages is doubtful, that his view of Nature was in harmony with his fixed belief in the all-pervading, ever present, and ever active energy of God. (ii. 4; iv. 40; x. 1; vi. 40; and other passages. Compare Seneca, De Benef., iv. 7. Swedenborg, Angelic Wisdom, 349-357.)

There is much in Antoninus that is hard to understand, and it might be said that he did not fully comprehend all that he wrote; which would however be in no way remarkable, for it happens now that a man may write what neither he nor anybody can understand. Antoninus tells us (xii. 10) to look at things and see what they are, resolving them into the material (\textgreek{ὕλη}) , the casual (\textgreek{αἲτιον}), and the relation (\textgreek{ἀναφορά}), or the purpose, by which he seems to mean something in the nature of what we call effect, or end. The word Cause (\textgreek{αἰτία}) is the difficulty. There is the same word in the Sanscrit (hétu); and the subtle philosophers of India and of Greece, and the less subtle philosophers of modern times, have all used this word, or an equivalent word, in a vague way. Yet the confusion sometimes may be in the inevitable ambiguity of language rather than in the mind of the writer, for I cannot think that some of the wisest of men did not know what they intended to say. When Antoninus says (iv. 36), "that everything that exists is in a manner the seed of that which will be," he might be supposed to say what some of the Indian philosophers have said, and thus a profound truth might be converted into a gross absurdity. But he says, "in a manner," and in a manner he said true; and in another manner, if you mistake his meaning, he said false. When Plato said, "Nothing ever is, but is always becoming" (\textgreek{ἀεὶ γίγνεται}), he delivered a text, out of which we may derive something; for he destroys by it not all practical, but all speculative notions of cause and effect. The whole series of things, as they appear to us, must be contemplated in time, that is in succession, and we conceive or suppose intervals between one state of things and another state of things, so that there is priority and sequence, and interval, and Being, and a ceasing to Be, and beginning and ending. But there is nothing of the kind in the Nature of Things. It is an everlasting continuity (iv. 45; vii. 75). When Antoninus speaks of generation (x. 26), he speaks of one cause (\textgreek{αἰτία}) acting, and then another cause taking up the work, which the former left in a certain state, and so on; and we might perhaps conceive that he had some notion like what has been called "the self-evolving power of nature;" a fine phrase indeed, the full import of which I believe that the writer of it did not see, and thus he laid himself open to the imputation of being a follower of one of the Hindu sects, which makes all things come by evolution out of nature or matter, or out of something which takes the place of Deity, but is not Deity. I would have all men think as they please, or as they can, and I only claim the same freedom which I give. When a man writes anything, we may fairly try to find out all that his words must mean, even if the result is that they mean what he did not mean; and if we find this contradiction, it is not our fault, but his misfortune. Now Antoninus is perhaps somewhat in this condition in what he says (x. 26), though he speaks at the end of the paragraph of the power which acts, unseen by the eyes, but still no less clearly. But whether in this passage (x. 26) lie means that the power is conceived to be in the different successive causes (\textgreek{αἰτίαι}), or in something else, nobody can tell. From other passages, however, I do collect that his notion of the phenomena of the universe is what I have stated. The Deity works unseen, if we may use such language, and perhaps I may, as Job did, or he who wrote the book of Job. "In him we live and move and are," said St. Paul to the Athenians; and to show his hearers that this was no new doctrine, he quoted the Greek poets. One of these poets was the Stoic Cleauthes, whose noble hymn to Zeus, or God, is an elevated expression of devotion and philosophy. It deprives Nature of her power, and puts her under the immediate government of the Deity.

"Thee all this heaven, which whirls around the earth,
Obeys, and willing follows where thou leadest.
Without thee, God, nothing is done on earth,
Nor in the ethereal realms, nor in the sea,
Save what the wicked through their folly do."

Antoninus' conviction of the existence of a divine power and government was founded on his perception of the order of the universe. Like Socrates (Xen. Mem., iv. 3, 13, etc.) he says that though we cannot see the forms of divine powers, we know that they exist because we see their works.

"To those who ask, Where hast thou seen the gods, or how dost thou comprehend that they exist and so worshipest them? I answer, in the first place, that they may be seen even with the eyes; in the second place, neither have I seen my own soul, and yet I honor it. Thus then with respect to the gods, from what I constantly experience of their power, from this I comprehend that they exist, and I venerate them." (xii. 28, and the note. Comp. Aristotle de Mundo, c. 6; Xen. Mem. i. 4, 9; Cicero, Tuscul. i. 28, 29; St. Paul's Epistle to the Romans, i. 19, 20; and Montaigne's Apology for Raimond de Sebonde, ii. c. 12.) This is a very old argument, which has always had great weight with most people, and has appeared sufficient. It does not acquire the least additional strength by being developed in a learned treatise. It is as intelligible in its simple enunciation as it can be made. If it is rejected, there is no arguing with him who rejects it: and if it is worked out into innumerable particulars, the value of the evidence runs the risk of being buried under a mass of words.

Man being conscious that he is a spiritual power, or that he has such a power, in whatever way he conceives that he has it—for I wish simply to state a fact—from this power which he has in himself, he is led, as Antoninus says, to believe that there is a greater power, which, as the old Stoics tell us, pervades the whole universe as the intellect[12] (\textgreek{νοῦς}) pervades man. (Compare Epictetus' Discourses, i. 14; and Voltaire à Mad\^e. Necker, vol. lxvii., p. 278, ed. Lequien.)

God exists then, but what do we know of his nature? Antoninus says that the soul of man is an efflux from the divinity. We have bodies like animals, but we have reason, intelligence, as the gods. Animals have life (\textgreek{ψυχή}) and what we call instincts or natural principles of action: but the rational animal man alone has a rational, intelligent soul (\textgreek{ψυχὴ λοική, υοερά}). Antoninus insists on this continually: God is in man,[13] and so we must constantly attend to the divinity within us, for it is only in this way that we can have any knowledge of the nature of God. The human soul is in a sense a portion of the divinity, and the soul alone has any communication with the Deity; for as he says (xii. 2): "With his intellectual part alone God touches the intelligence only which has flowed and been derived from himself into these bodies." In fact he says that which is hidden within a man is life, that is, the man himself. All the rest is vesture, covering, organs, instrument, which the living man, the real[14] man, uses for the purpose of his present existence. The air is universally diffused for him who is able to respire; and so for him who is willing to partake of it the intelligent power, which holds within it all things, is diffused as wide and free as the air (viii. 54). It is by living a divine life that man approaches to a knowledge of the divinity.[15] It is by following the divinity within \textgreek{δαίμων} or \textgreek{θεός}, as Antonius calls it, that man comes nearest to the Deity, the supreme good; for man can never attain to perfect agreement with his internal guide (\textgreek{το ήγεμονικόν}). "Live with the gods. And he does live with the gods who constantly shows to them that his own soul is satisfied with that which is assigned to him, and that it does all the daemon (\textgreek{δαίμων}) wishes, which Zeus hath given to every man for his guardian and guide, a portion of himself. And this daemon is every man's understanding and reason" (v. 27).

There is in man, that is in the reason, the intelligence, a superior faculty which if it is exercised rules all the rest. This is the ruling faculty (\textgreek{τὸ ἡγεμονικόν}), which Cicero (De Natura Deorum, ii. 11) renders by the Latin word Principatus, "to which nothing can or ought to be superior." Antoninus often uses this term and others which are equivalent. He names it (vii. 64) "the governing intelligence." The governing faculty is the master of the soul (v. 26). A man must reverence only his ruling faculty and the divinity within him. As we must reverence that which is supreme in the universe, so we must reverence that which is supreme in ourselves; and this is that which is of like kind with that which is supreme in the universe (v. 21). So, as Plotinus says, the soul of man can only know the divine so far as it knows itself. In one passage (xi. 19) Antoninus speaks of a man's condemnation of himself when the diviner part within him has been overpowered and yields to the less honorable and to the perishable part, the body, and its gross pleasures. In a word, the views of Antoninus on this matter, however his expressions may vary, are exactly what Bishop Butler expresses when he speaks of "the natural supremacy of reflection or conscience," of the faculty "which surveys, approves, or disapproves the several affections of our mind and actions of our lives."

Much matter might be collected from Antoninus on the notion of the Universe being one animated Being. But all that he says amounts to no more, as Schultz remarks, than this: the soul of man is most intimately united to his body, and together they make one animal, which we call man; so the Deity is most intimately united to the world, or the material universe, and together they form one whole. But Antoninus did not view God and the material universe as the same, any more than he viewed the body and soul of man as one. Antoninus has 110 speculations on the absolute nature of the Deity. It was not his fashion to waste his time on what man cannot understand.[16] He was satisfied that God exists, that he governs all things, that man can only have an imperfect knowledge of his nature, and he must attain this imperfect knowledge by reverencing the divinity which is within him, and keeping it pure.

From all that has been said, it follows that the universe is administered by the Providence of God (\textgreek{πρόνοια}), and that all things are wisely ordered. There are passages in which Antoninus expresses doubts, or states different possible theories of the constitution and government of the universe; but he always recurs to his fundamental principle, that if we admit the existence of a deity, we must also admit that he orders all things wisely and well (iv. 27; vi. 1; ix. 28; xii. 5; and many other passages). Epictetus says (i. 6) that we can discern the providence which rules the world, if we possess two things,—the power of seeing all that happens with respect to each thing, and a grateful disposition.

But if all things are wisely ordered, how is the world so full of what we call evil, physical and moral? If instead of saying that there is evil in the world, we use the expression which I have used, "what we call evil," we have partly anticipated the emperor's answer. We see and feel and know imperfectly very few things in the few years that we live, and all the knowledge and all the experience of all the human race is positive ignorance of the whole, which is infinite. Now, as our reason teaches us that everything is in some way related to and connected with every other thing, all notion of evil as being in the universe of things is a contradiction; for if the whole comes from and is governed by an intelligent being, it is impossible to conceive anything in it which tends to the evil or destruction of the whole (viii. 55; x. 6). Everything is in constant mutation, and yet the whole subsists; we might imagine the solar system resolved into its elemental parts, and yet the whole would still subsist "ever young and perfect."

All things, all forms, are dissolved, and new forms appear. All living things undergo the change which we call death. If we call death an evil, then all change is an evil. Living beings also suffer pain, and man suffers most of all, for he suffers both in and by his body and by his intelligent part. Men suffer also from one another, and perhaps the largest part of human suffering comes to man from those whom he calls his brothers. Antoninus says (viii. 55), "Generally, wickedness does no harm at all to the universe; and particularly, the wickedness [of one man] does no harm to another. It is only harmful to him who has it in his power to be released from it as soon as he shall choose." The first part of this is perfectly consistent with the doctrine that the whole can sustain no evil or harm. The second part must be explained by the Stoic principle that there is no evil in anything which is not in our power. What wrong we suffer from another is his evil, not ours. But this is an admission that there is evil in a sort, for he who does wrong does evil, and if others can endure the wrong, still there is evil in the wrong-doer. Antoninus (xi. 18) gives many excellent precepts with respect to wrongs and injuries, and his precepts are practical. He teaches us to bear what we cannot avoid, and his lessons may be just as useful to him who denies the being and the government of God as to him who believes in both. There is no direct answer in Antoninus to the objections which may be made to the existence and providence of God because of the moral disorder and suffering which are in the world, except this answer which he makes in reply to the supposition that even the best men may be extinguished by death. He says if it is so, we may be sure that if it ought to have been otherwise, the gods would have ordered it otherwise (xii. 5). His conviction of the wisdom which we may observe in the government of the world is too strong to be disturbed by any apparent irregularities in the order of things. That these disorders exist is a fact, and those who would conclude from them against the being and government of God conclude too hastily. We all admit that there is an order in the material world, a Nature, in the sense in which that word has been explained, a constitution (\textgreek{κατασκευή}), what we call a system, a relation of parts to one another and a fitness of the whole for something. So in the constitution of plants and of animals there is an order, a fitness for some end. Sometimes the order, as we conceive it, is interrupted, and the end, as we conceive it, is not attained. The seed, the plant, or the animal sometimes perishes before it has passed through all its changes and done all its uses. It is according to Nature, that is a fixed order, for some to perish early and for others to do all their uses and leave successors to take their place. So man has a corporeal and intellectual and moral constitution fit for certain uses, and on the whole man performs these uses, dies, and leaves other men in his place. So society exists, and a social state is manifestly the natural state of man—the state for which his nature fits him, and society amidst innumerable irregularities and disorders still subsists; and perhaps we may say that the history of the past and our present knowledge give us a reasonable hope that its disorders will diminish, and that order, its governing principle, may be more firmly established. As order then, a fixed order, we may say, subject to deviations real or apparent, must be admitted to exist in the whole nature of things, that which we call disorder or evil, as it seems to us, does not in any way alter the fact of the general constitution of things having a nature or fixed order. Nobody will conclude from the existence of disorder that order is not the rule, for the existence of order both physical and moral is proved by daily experience and all past experience. We cannot conceive how the order of the universe is maintained: we cannot even conceive how our own life from day to day is continued, nor how we perform the simplest movements of the body, nor how we grow and think and act, though we know many of the conditions which are necessary for all these functions. Knowing nothing then of the unseen power which acts in ourselves except by what is done, we know nothing of the power which acts through what we call all time and all space; but seeing that there is a nature or fixed order in all things known to us, it is conformable to the nature of our minds to believe that this universal Nature has a cause which operates continually, and that we are totally unable to speculate on the reason of any of those disorders or evils which we perceive. This I believe is the answer which may be collected from all that Antoninus has said.[17]

The origin of evil is an old question. Achilles tells Priam (Iliad, 24, 527) that Zeus has two casks, one filled with good things, and the other with bad, and that he gives to men out of each according to his pleasure; and so we must be content, for we cannot alter the will of Zeus. One of the Greek commentators asks how must we reconcile this doctrine with what we find in the first book of the Odyssey, where the king of the gods says, Men say that evil comes to them from us, but they bring it on themselves through their own folly. The answer is plain enough even to the Greek commentator. The poets make both Achilles and Zeus speak appropriately to their several characters. Indeed, Zeus says plainly that men do attribute their sufferings to their gods, but they do it falsely, for they are the cause of their own sorrows.

Epictetus in his Enchiridion (c. 27) makes short work of the question of evil. He says, ``As a mark is not set up for the purpose of missing it, so neither does the nature of evil exist in the universe." This will appear obscure enough to those who are not acquainted with Epictetus, but he always knows what he is talking about. We do not set up a mark in order to miss it, though we may miss it. God, whose existence Epictetus assumes, has not ordered all things so that his purpose shall fail. Whatever there may be of what we call evil, the nature of evil, as he expresses it, does not exist; that is, evil is not a part of the constitution or nature of things. If there were a principle of evil (\textgreek{ἀρχή}) in the constitution of things, evil would no longer be evil, as Simplicius argues, but evil would be good. Simplicius (c. 34, [27]) has a long and curious discourse on this text of Epictetus, and it is amusing and instructive.

One passage more will conclude this matter. It contains all that the emperor could say (ii. 11): ``To go from among men, if there are gods, is not a thing to be afraid of, for the gods will not involve thee in evil; but if indeed they do not exist, or if they have no concern about human affairs, what is it to me to live in a universe devoid of gods or devoid of providence? But in truth they do exist, and they do care for human things, and they have put all the means in man's power to enable him not to fall into real evils. And as to the rest, if there was anything evil, they would have provided for this also, that it should be altogether in a man's power not to fall into it. But that which does not make a man worse, how can it make a man's life worse? But neither through ignorance, nor having the knowledge but not the power to guard against or correct these things, is it possible that the nature of the universe has overlooked them; nor is it possible that it has made so great a mistake, either through want of power or want of skill, that good and evil should happen indiscriminately to the good and the bad. But death certainly and life, honor and dishonor, pain and pleasure, all these things equally happen to good and bad men, being things which make us neither better nor worse. Therefore they are neither good nor evil."

The Ethical part of Antoninus' Philosophy follows from his general principles. The end of all his philosophy is to live conformably to Nature, both a man's own nature and the nature of the universe. Bishop Butler has explained what the Greek philosophers meant when they spoke of living according to Nature, and he says that when it is explained, as he has explained it and as they understood it, it is "a manner of speaking not loose and undeterminate, but clear and distinct, strictly just and true." To live according to Nature is to live according to a man's whole nature, not according to a part of it, and to reverence the divinity within him as the governor of all his actions. "To the rational animal the same act is according to nature and according to reason"[18] (vii. 11). That which is done contrary to reason is also an act contrary to nature, to the whole nature, though it is certainly conformable to some part of man's nature, or it could not be done. Man is made for action, not for idleness or pleasure. As plants and animals do the uses of their nature, so man must do his (v. 1).

Man must also live conformably to the universal nature, conformably to the nature of all things of which he is one; and as a citizen of a political community he must direct his life and actions with reference to those among whom, among other purposes, he lives.[19] A man must not retire into solitude and cut himself off from his fellow-men. He must be ever active to do his part in the great whole. All men are his kin, not only in blood, but still more by participating in the same intelligence and by being a portion of the same divinity. A man cannot really be injured by his brethren, for no act of theirs can make him bad, and he must not be angry with them nor hate them: "For we are made for co-operation, like feet, like hands, like eyelids, like the rows of the upper and lower teeth. To act against one another then is contrary to nature; and it is acting against one another to be vexed and to turn away" (ii. 1).

Further he says: "Take pleasure in one thing and rest in it in passing from one social act to another social act, thinking of God" (vi. 7). Again: "Love mankind. Follow God" (vii. 31). It is the characteristic of the rational soul for a man to love his neighbor (xi. 1). Antoninus teaches in various passages the forgiveness of injuries, and we know that he also practised what he taught. Bishop Butler remarks that "this divine precept to forgive injuries and to love our enemies, though to be met with in Gentile moralists, yet is in a peculiar sense a precept of Christianity, as our Saviour has insisted more upon it than on any other single virtue." The practice of this precept is the most difficult of all virtues. Antoninus often enforces it and gives us aid towards following it. When we are injured, we feel anger and resentment, and the feeling is natural, just, and useful for the conservation of society. It is useful that wrong-doers should feel the natural consequences of their actions, among which is the disapprobation of society and the resentment of him who is wronged. But revenge, in the proper sense of that word, must not be practised. "The best way of avenging thyself," says the emperor, "is not to become like the wrong-doer." It is plain by this that he does not mean that we should in any case practise revenge; but he says to those who talk of revenging wrongs, Be not like him who has done the wrong. Socrates in the Crito (c. 10) says the same in other words, and St. Paul (Ep. to the Romans, xii. 17). "When a man has done thee any wrong, immediately consider with what opinion about good or evil he has done wrong. For when thou hast seen this, thou wilt pity him and wilt neither wonder nor be angry" (vii. 26). Antoninus would not deny that wrong naturally produces the feeling of anger and resentment, for this is implied in the recommendation to reflect on the nature of the man's mind who has done the wrong, and then you will have pity instead of resentment; and so it comes to the same as St. Paul's advice to be angry and sin not; which, as Butler well explains it, is not a recommendation to be angry, which nobody needs, for anger is a natural passion, but it is a warning against allowing anger to lead us into sin. In short the emperor's doctrine about wrongful acts is this: wrong-doers do not know what good and bad are: they offend out of ignorance, and in the sense of the Stoics this is true. Though this kind of ignorance will never be admitted as a legal excuse, and ought not to be admitted as a full excuse in any way by society, there may be grievous injuries, such as it is in a man's power to forgive without harm to society; and if he forgives because he sees that his enemies know not what they do, he is acting in the spirit of the sublime prayer, "Father, forgive them, for they know not what they do."

The emperor's moral philosophy was not a feeble, narrow system, which teaches a man to look directly to his own happiness, though a man's happiness or tranquility is indirectly promoted by living as he ought to do. A man must live conformably to the universal nature, which means, as the emperor explains it in many passages, that a man's actions must be conformable to his true relations to all other human beings, both as a citizen of a political community and as a member of the whole human family. This implies, and he often expresses it in the most forcible language, that a man's words and actions, so far as they affect others, must be measured by a fixed rule, which is their consistency with the conservation and the interests of the particular society of which he is a member, and of the whole human race. To live conformably to such a rule, a man must use his rational faculties in order to discern clearly the consequences and full effect of all his actions and of the actions of others: he must not live a life of contemplation and reflection only, though he must often retire within himself to calm and purify his soul by thought,[20] but he must mingle in the work of man and be a fellow laborer for the general good.

A man should have an object or purpose in life, that he may direct all his energies to it; of course a good object (ii. 7). He who has not one object or purpose of life, cannot be one and the same all through his life (xi. 21). Bacon has a remark to the same effect, on the best means of "reducing of the mind unto virtue and good estate; which is, the electing and propounding unto a man's self good and virtuous ends of his life, such as may be in a reasonable sort within his compass to attain." He is a happy man who has been wise enough to do this when he was young and has had the opportunities; but the emperor seeing well that a man cannot always be so wise in his youth, encourages himself to do it when he can, and not to let life slip away before he has begun. He who can propose to himself good and virtuous ends of life, and be true to them, cannot fail to live conformably to his own interest and the universal interest, for in the nature of things they are one. If a thing is not good for the hive, it is not good for the bee (vi. 54).

One passage may end this matter. "If the gods have determined about me and about the things which must happen to me, they have determined well, for it is not easy even to imagine a deity without forethought; and as to doing me harm, why should they have any desire towards that? For what advantage would result to them from this or to the whole, which is the special object of their providence? But if they have not determined about me individually, they have certainly determined about the whole at least; and the things which happen by way of sequence in this general arrangement I ought to accept with pleasure and to be content with them. But if they determine about nothing—which it is wicked to believe, or if we do believe it, let us neither sacrifice nor pray nor swear by them, nor do anything else which we do as if the gods were present and lived with us; but if however the gods determine about none of the things which concern us, I am able to determine about myself, and I can inquire about that which is useful: and that is useful to every man which is conformable to his own constitution (\textgreek{κατασκευῄ}) and nature. But my nature is rational and social; and my city and country, so far as I am Antoninus, is Rome; but so far as I am a man, it is the world. The things then which are useful to these cities are alone useful to me" (vi. 44).

It would be tedious, and it is not necessary, to state the emperor's opinions on all the ways in which a man may profitably use his understanding towards perfecting himself in practical virtue. The passages to this purpose are in all parts of his book, but as they are in no order or connection, a man must use the book a long time before he will find out all that is in it. A few words may be added here. If we analyze all other things, we find how insufficient they are for human life, and how truly worthless many of them are. Virtue alone is indivisible, one, and perfectly satisfying. The notion of Virtue cannot be considered vague or unsettled, because a man may find it difficult to explain the notion fully to himself, or to expound it to others in such a way as to prevent cavilling. Virtue is a whole, and no more consists of parts than man's intelligence does; and yet we speak of various intellectual faculties as a convenient way of expressing the various powers which man's intellect shows by his works. In the same way we may speak of various virtues or parts of virtue, in a practical sense, for the purpose of showing what particular virtues we ought to practice in order to the exercise of the whole of virtue, that is, as man's nature is capable of.

The prime principle in man's constitution is social. The next in order is not to yield to the persuasions of the body, when they are not conformable to the rational principle, which must govern. The third is freedom from error and from deception. "Let then the ruling principle holding fast to these things go straight on, and it has what is its own" (vii. 55). The emperor selects justice as the virtue which is the basis of all the rest (x. 11), and this had been said long before his time.

It is true that all people have some notion of what is meant by justice as a disposition of the mind, and some notion about acting in conformity to this disposition; but experience shows that men's notions about justice are as confused as their actions are inconsistent with the true notion of justice. The emperor's notion of justice is clear enough, but not practical enough for all mankind. "Let there be freedom from perturbations with respect to the things which come from the external cause; and let there be justice in the things done by virtue of the internal cause, that is, let there be movement and action terminating in this, in social acts, for this is according to thy nature" (ix. 31). In another place (ix. 1) he says that "he who acts unjustly acts impiously," which follows of course from all that he says in various places. He insists on the practice of truth as a virtue and as a means to virtue, which no doubt it is: for lying even in indifferent things weakens the understanding; and lying maliciously is as great a moral offense as a man can be guilty of, viewed both as showing an habitual disposition, and viewed with respect to consequences. He couples the notion of justice with action. A man must not pride himself on having some fine notion of justice in his head, but he must exhibit his justice in act, like St. James' notion of faith. But this is enough.

The Stoics, and Antoninus among them, call some things beautiful \textgreek{(καλά}) and some ugly (\textgreek{αἰσχρά}), and as they are beautiful so they are good, and as they are ugly so they are evil, or bad (ii. 1). All these things, good and evil, are in our power, absolutely, some of the stricter Stoics would say; in a manner only, as those who would not depart altogether from common sense would say; practically they are to a great degree in the power of some persons and in some circumstances, but in a small degree only in other persons and in other circumstances. The Stoics maintain man's free will as to the things which are in his power; for as to the things which are out of his power, free will terminating in action is of course excluded by the very terms of the expression. I hardly know if we can discover exactly Antoninus' notion of the free will of man, nor is the question worth the inquiry. What he does mean and does say is intelligible. All the things which are not in our power (\textgreek{ἀπροαίρετα}) are indifferent: they are neither good nor bad, morally. Such are life, health, wealth, power, disease, poverty, and death. Life and death are all men's portion. Health, wealth, power, disease, and poverty happen to men, indifferently to the good and to the bad; to those who live according to nature and to those who do not.[21] "Life," says the emperor, "is a warfare and a stranger's sojourn, and after fame is oblivion" (ii. 17). After speaking of those men who have disturbed the world and then died, and of the death of philosophers such as Heraclitus and Democritus, who was destroyed by lice, and of Socrates whom other lice (his enemies) destroyed, he says: "What means all this? Thou hast embarked, thou hast made the voyage, thou art come to shore; get out. If indeed to another life, there is no want of gods, not even there. But if to a state without sensation, thou wilt cease to be held by pains and pleasures, and to be a slave to the vessel which is as much inferior as that which serves it is superior: for the one is intelligence and Deity; the other is earth and corruption" (iii. 3). It is not death that a man should fear, but he should fear never beginning to live according to nature (xii. 1). Every man should live in such a way as to discharge his duty, and to trouble himself about nothing else. He should live such a life that he shall always be ready for death, and shall depart content when the summons comes. For what is death? "A cessation of the impressions through the senses, and of the pulling of the strings which move the appetites, and of the discursive movements of the thoughts, and of the service to the flesh" (vi. 28). Death is such as generation is, a mystery of nature (iv. 5). In another passage, the exact meaning of which is perhaps doubtful (ix. 3), he speaks of the child which leaves the womb, and so he says the soul at death leaves its envelope. As the child is born or comes into life by leaving the womb, so the soul may on leaving the body pass into another existence which is perfect. I am not sure if this is the emperor's meaning. Butler compares it with a passage in Strabo (p. 713) about the Brachmans' notion of death being the birth into real life and a happy life, to those who have philosophized; and he thinks Antoninus may allude to this opinion.[22]

Antoninus' opinion of a future life is nowhere clearly expressed. His doctrine of the nature of the soul of necessity implies that it does not perish absolutely, for a portion of the divinity cannot perish. The opinion is at least as old as the time of Epicharmus and Euripides; what comes from earth goes back to earth, and what comes from heaven, the divinity, returns to him who gave it. But I find nothing clear in Antoninus as to the notion of the man existing after death so as to be conscious of his sameness with that soul which occupied his vessel of clay. He seems to be perplexed on this matter, and finally to have rested in this, that God or the gods will do whatever is best, and consistent with the university of things.

Nor, I think, does he speak conclusively on another Stoic doctrine, which some Stoics practised,—the anticipating the regular course of nature by a man's own act. The reader will find some passages in which this is touched on, and he may make of them what he can. But there are passages in which the emperor encourages himself to wait for the end patiently and with tranquility; and certainly it is consistent with all his best teaching that a man should bear all that falls to his lot and do useful acts as he lives. He should not therefore abridge the time of his usefulness by his own act. Whether he contemplates any possible cases in which a man should die by his own hand, I cannot tell; and the matter is not worth a curious inquiry, for I believe it would not lead to any certain result as to his opinion on this point. I do not think that Antoninus, who never mentions Seneca, though he must have known all about him, would have agreed with Seneca when he gives as a reason for suicide, that the eternal law, whatever he means, has made nothing better for us than this, that it has given us only one way of entering into life and many ways of going out of it. The ways of going out indeed are many, and that is a good reason for a man taking care of himself.[23]

Happiness was not the direct object of a Stoic's life. There is no rule of life contained in the precept that a man should pursue his own happiness. Many men think that they are seeking happiness when they are only seeking the gratification of some particular passion, the strongest that they have. The end of a man is, as already explained, to live conformably to nature, and he will thus obtain happiness, tranquility of mind, and contentment (iii. 12; viii. 1, and other places). As a means of living conformably to nature he must study the four chief virtues, each of which has its proper sphere: wisdom, or the knowledge of good and evil; justice, or the giving to every man his due; fortitude, or the enduring of labor and pain; and temperance, which is moderation in all things. By thus living conformably to nature the Stoic obtained all that he wished or expected. His reward was in his virtuous life, and he was satisfied with that. Some Greek poet long ago wrote:—

"For virtue only of all human things
Takes her reward not from the hands of others.
Virtue herself rewards the toils of virtue."

Some of the Stoics indeed expressed themselves in very arrogant, absurd terms, about the wise man's self-sufficiency; they elevated him to the rank of a deity.[24] But these were only talkers and lecturers, such as those in all ages who utter fine words, know little of human affairs, and care only for notoriety. Epictetus and Antoninus both by precept and example labored to improve themselves and others; and if we discover imperfections in their teaching, we must still honor these great men who attempted to show that there is in man's nature and in the constitution of things sufficient reason for living a virtuous life. It is difficult enough to live as we ought to live, difficult even for any man to live in such a way as to satisfy himself, if he exercises only in a moderate degree the power of reflecting upon and reviewing his own conduct; and if all men cannot be brought to the same opinions in morals and religion, it is at least worth while to give them good reasons for as much as they can be persuaded to accept.