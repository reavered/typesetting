\begin{enumerate}
\item He who acts unjustly acts impiously. For since the universal nature has made rational animals for the sake of one another, to help one another according to their deserts, but in no way to injure one another, he who transgresses her will is clearly guilty of impiety towards the highest divinity. And he too who lies is guilty of impiety to the same divinity; for the universal nature is the nature of things that are; and things that are have a relation to all things that come into existence.\footnote{``As there is not any action or natural event, which we are acquainted with, so single and unconnected as not to have a respect to some other actions and events, so possibly each of them, when it has not an immediate, may yet have a remote, natural relation to other actions and events, much beyond the compass of this present world." Again: ``Things seemingly the most insignificant imaginable are perpetually observed to be necessary conditions to other things of the greatest importance, so that any one thing whatever may, for aught we know to the contrary, be a necessary condition to any other."—Butler's \textit{Analogy}, Chap. 7. See all the chapter. Some critics take \textgreek{τὰ ὑπάρχοντα} in this passage of Antoninus to be the same as \textgreek{τὰ ὄντα}: but if that were so he might have said \textgreek{πρὸς ἄλληλα} instead of \textgreek{πρὸς τὰ ὑπάρχοντα}. Perhaps the meaning of \textgreek{πρὸς τὰ ὑπάρχοντα} may be ``to all prior things." If so, the translation is still correct. See vi. 38.} And further, this universal nature is named truth, and is the prime cause of all things that are true. He then who lies intentionally is guilty of impiety, inasmuch as he acts unjustly by deceiving; and he also who lies unintentionally, inasmuch as he is at variance with the universal nature, and inasmuch as he disturbs the order by fighting against the nature of the world; for he fights against it, who is moved of himself to that which is contrary to truth, for he had received powers from nature through the neglect of which he is not able now to distinguish falsehood from truth. And indeed he who pursues pleasure as good, and avoids pain as evil, is guilty of impiety. For of necessity such a man must often find fault with the universal nature, alleging that it assigns things to the bad and the good contrary to their deserts, because frequently the bad are in the enjoyment of pleasure and possess the things which procure pleasure, but the good have pain for their share and the things which cause pain. And further, he who is afraid of pain will sometimes also be afraid of some of the things which will happen in the world, and even this is impiety. And he who pursues pleasure will not abstain from injustice, and this is plainly impiety. Now with respect to the things towards which the universal nature is equally affected—for it would not have made both, unless it was equally affected towards both—towards these they who wish to follow nature should be of the same mind with it, and equally affected. With respect to pain, then, and pleasure, or death and life, or honor and dishonor, which the universal nature employs equally, whoever is not equally affected is manifestly acting impiously. And I say that the universal nature employs them equally, instead of saying that they happen alike to those who are produced in continuous series and to those who come after them by virtue of a certain original movement of Providence, according to which it moved from a certain beginning to this ordering of things, having conceived certain principles of the things which were to be, and having determined powers productive of beings and of changes and of such like successions (vii. 75).

\item It would be a man's happiest lot to depart from mankind without having had any taste of lying and hypocrisy and luxury and pride. However, to breathe out one's life when a man has had enough of these things is the next best voyage, as the saying is. Hast thou determined to abide with vice, and hast not experience yet induced thee to fly from this pestilence? For the destruction of the understanding is a pestilence, much more, indeed, than any such corruption and change of this atmosphere which surrounds us. For this corruption is a pestilence of animals so far as they are animals; but the other is a pestilence of men so far as they are men.

\item Do not despise death, but be well content with it, since this too is one of those things which nature wills. For such as it is to be young and to grow old, and to increase and to reach maturity, and to have teeth and beard and gray hairs, and to beget and to be pregnant and to bring forth, and all the other natural operations which the seasons of thy life bring, such also is dissolution. This, then, is consistent with the character of a reflecting man—to be neither careless nor impatient nor contemptuous with respect to death, but to wait for it as one of the operations of nature. As thou now waitest for the time when the child shall come out of thy wife's womb, so be ready for the time when thy soul shall fall out of this envelope. But if thou requirest also a vulgar kind of comfort which shall reach thy heart, thou wilt be made best reconciled to death by observing the objects from which thou art going to be removed, and the morals of those with whom thy soul will no longer be mingled. For it is no way right to be offended with men, but it is thy duty to care for them and to bear with them gently; and yet to remember that thy departure will not be from men who have the same principles as thyself. For this is the only thing, if there be any, which could draw us the contrary way and attach us to life—to be permitted to live with those who have the same principles as ourselves. But now thou seest how great is the trouble arising from the discordance of those who live together, so that thou mayst say, Come quick, O death, lest perchance I, too, should forget myself.

\item He who does wrong does wrong against himself. He who acts unjustly acts unjustly to himself, because he makes himself bad.

\item He often acts unjustly who does not do a certain thing; not only he who does a certain thing.

\item Thy present opinion founded on understanding, and thy present conduct directed to social good, and thy present disposition of contentment with everything which happens—that is enough.

\item Wipe out imagination; check desire: extinguish appetite: keep the ruling faculty in its own power.

\item Among the animals which have not reason one life is distributed; but among reasonable animals one intelligent soul is distributed: just as there is one earth of all things which are of an earthly nature, and we see by one light, and breathe one air, all of us that have the faculty of vision and all that have life.

\item All things which participate in anything which is common to them all, move towards that which is of the same kind with themselves. Everything which is earthy turns towards the earth, everything which is liquid flows together, and everything which is of an aerial kind does the same, so that they require something to keep them asunder, and the application of force. Fire indeed moves upwards on account of the elemental fire, but it is so ready to be kindled together with all the fire which is here, that even every substance which is somewhat dry is easily ignited, because there is less mingled with it of that which is a hindrance to ignition. Accordingly, then, everything also which participates in the common intelligent nature moves in like manner towards that which is of the same kind with itself, or moves even more. For so much as it is superior in comparison with all other things, in the same degree also is it more ready to mingle with and to be fused with that which is akin to it. Accordingly among animals devoid of reason we find swarms of bees, and herds of cattle, and the nurture of young birds, and in a manner, loves; for even in animals there are souls, and that power which brings them together is seen to exert itself in a superior degree, and in such a way as never has been observed in plants nor in stones nor in trees. But in rational animals there are political communities and friendships, and families and meetings of people; and in wars, treaties, and armistices. But in the things which are still superior, even though they are separated from one another, unity in a manner exists, as in the stars. Thus the ascent to the higher degree is able to produce a sympathy even in things which are separated. See, then, what now takes place; for only intelligent animals have now forgotten this mutual desire and inclination, and in them alone the property of flowing together is not seen. But still, though men strive to avoid [{\clarify this union}], they are caught and held by it, for their nature is too strong for them; and thou wilt see what I say, if thou only observest. Sooner, then, will one find anything earthy which comes in contact with no earthy thing, than a man altogether separated from other men.

\item Both man and God and the universe produce fruit; at the proper seasons each produces it. But and if usage has especially fixed these terms to the vine and like things, this is nothing. Reason produces fruit both for all and for itself, and there are produced from it other things of the same kind as reason itself.

\item If thou art able, correct by teaching those who do wrong; but if thou canst not, remember that indulgence is given to thee for this purpose. And the gods, too, are indulgent to such persons; and for some purposes they even help them to get health, wealth, reputation; so kind they are. And it is in thy power also; or say, who hinders thee?

\item Labor not as one who is wretched, nor yet as one who would be pitied or admired; but direct thy will to one thing only—to put thyself in motion and to check thyself, as the social reason requires.

\item To-day I have got out of all trouble, or rather I have cast out all trouble, for it was not outside, but within and in my opinions.

\item All things are the same, familiar in experience, and ephemeral in time, and worthless in the matter. Everything now is just as it was in the time of those whom we have buried.

\item Things stand outside of us, themselves by themselves, neither knowing aught of themselves, nor expressing any judgment. What is it, then, which does judge about them? The ruling faculty.

\item Not in passivity but in activity lie the evil and the good of the rational social animal, just as his virtue and his vice lie not in passivity but in activity.\footnote{Virtutis omnis laus in actione consistit.—Cicero, \textit{De Officiis}, 1. 6.}

\item For the stone which has been thrown up it is no evil to come down, nor indeed any good to have been carried up (viii. 20).

\item Penetrate inwards into men's leading principles, and thou wilt see what judges thou art afraid of, and what kind of judges they are of themselves.

\item All things are changing: and thou thyself art in continuous mutation and in a manner in continuous destruction, and the whole universe too.

\item It is thy duty to leave another man's wrongful act there where it is (vii. 29; ix. 38).

\item Termination of activity, cessation from movement and opinion, and in a sense their death, is no evil. Turn thy thoughts now to the consideration of thy life, thy life as a child, as a youth, thy manhood, thy old age, for in these also every change was a death. Is this anything to fear? Turn thy thoughts now to thy life under thy grandfather, then to thy life under thy mother, then to thy life under thy father; and as thou findest many other differences and changes and terminations, ask thyself, Is this anything to fear? In like manner, then, neither are the termination and cessation and change of thy whole life a thing to be afraid of.

\item Hasten [{\clarify to examine}] thy own ruling faculty and that of the universe and that of thy neighbor: thy own, that thou mayst make it just: and that of the universe, that thou mayst remember of what thou art a part; and that of thy neighbor, that thou mayst know whether he has acted ignorantly or with knowledge, and thou mayst also consider that his ruling faculty is akin to thine.

\item As thou thyself art a component part of a social system, so let every act of thine be a component part of social life. Whatever act of thine then has no reference either immediately or remotely to a social end, this tears asunder thy life, and does not allow it to be one, and it is of the nature of a mutiny, just as when in a popular assembly a man acting by himself stands apart from the general agreement.

\item Quarrels of little children and their sports, and poor spirits carrying about dead bodies [{\clarify such is everything}]; and so what is exhibited in the representation of the mansions of the dead\footnote{\textgreek{τὸ τῆς Νεκυίας} may be, as Gataker conjectures, a dramatic representation of the state of the dead. Schultz supposes that it may be also a reference to the \textgreek{Νεκυία} of the \textit{Odyssey} (lib. xi.).} strikes our eyes more clearly.

\item Examine into the quality of the form of an object, and detach it altogether from its material part, and then contemplate it; then determine the time, the longest which a thing of this peculiar form is naturally made to endure.

\item Thou hast endured infinite troubles through not being contented with thy ruling faculty when it does the things which it is constituted by nature to do. But enough [{\clarify of this}].

\item When another blames thee or hates thee, or when men say about thee anything injurious, approach their poor souls, penetrate within, and see what kind of men they are. Thou wilt discover that there is no reason to take any trouble that these men may have this or that opinion about thee. However, thou must be well disposed towards them, for by nature they are friends. And the gods too aid them in all ways, by dreams, by signs, towards the attainment of those things on which they set a value.

\item The periodic movements of the universe are the same, up and down from age to age. And either the universal intelligence puts itself in motion for every separate effect, and if this is so, be thou content with that which is the result of its activity; or it puts itself in motion once, and everything else comes by way of sequence\footnote{The words which immediately follow \textgreek{κατ᾽ ἐπακολούθησιν} are corrupt. But the meaning is hardly doubtful. (Compare vii. 75.)} in a manner; or indivisible elements are the origin of all things.—In a word, if there is a god, all is well; and if chance rules, do not thou also be governed by it (vi. 44; vii. 75).

Soon will the earth cover us all: then the earth, too, will change, and the things also which result from change will continue to change forever, and these again forever. For if a man reflects on the changes and transformations which follow one another like wave after wave and their rapidity, he will despise everything which is perishable (xii. 21).

\item The universal cause is like a winter torrent: it carries everything along with it. But how worthless are all these poor people who are engaged in matters political, and, as they suppose, are playing the philosopher! All drivellers. Well then, man: do what nature now requires. Set thyself in motion, if it is in thy power, and do not look about thee to see if any one will observe it; nor yet expect Plato's Republic:\footnote{Those who wish to know what Plato's \textit{Republic} is may now study it in the accurate translation of Davies and Vaughan.} but be content if the smallest thing goes on well, and consider such an event to be no small matter. For who can change men's opinions? and without a change of opinions what else is there than the slavery of men who groan while they pretend to obey? Come now and tell me of Alexander and Philippus and Demetrius of Phalerum. They themselves shall judge whether they discovered what the common nature required, and trained themselves accordingly. But if they acted like tragedy heroes, no one has condemned me to imitate them. Simple and modest is the work of philosophy. Draw me not aside to insolence and pride.

\item Look down from above on the countless herds of men and their countless solemnities, and the infinitely varied voyagings in storms and calms, and the differences among those who are born, who live together, and die. And consider, too, the life lived by others in olden time, and the life of those who will live after thee, and the life now lived among barbarous nations, and how many know not even thy name, and how many will soon forget it, and how they who perhaps now are praising thee will very soon blame thee, and that neither a posthumous name is of any value, nor reputation, nor anything else.

\item Let there be freedom from perturbations with respect to the things which come from the external cause; and let there be justice in the things done by virtue of the internal cause, that is, let there be movement and action terminating in this, in social acts, for this is according to thy nature.

\item Thou canst remove out of the way many useless things among those which disturb thee, for they lie entirely in thy opinion; and thou wilt then gain for thyself ample space by comprehending the whole universe in thy mind, and by contemplating the eternity of time, and observing the rapid change of every several thing, how short is the time from birth to dissolution, and the illimitable time before birth as well as the equally boundless time after dissolution!

\item All that thou seest will quickly perish, and those who have been spectators of its dissolution will very soon perish too. And he who dies at the extremest old age will be brought into the same condition with him who died prematurely.

\item What are these men's leading principles, and about what kind of things are they busy, and for what kind of reasons do they love and honor? Imagine that thou seest their pool souls laid bare. When they think that they do harm by their blame or good by their praise, what an idea!

\item Loss is nothing else than change. But the universal nature delights in change, and in obedience to her all things are now done well, and from eternity have been in like form, and will be such to time without end. What, then, dost thou say—that all things have been and all things always will be bad, and that no power has ever been found in so many gods to rectify these things, but the world has been condemned to be bound in never ceasing evil (iv. 45, vii. 18)?

\item The rottenness of the matter which is the foundation of everything! water, dust, bones, filth: or again, marble rocks, the callosities of the earth; and gold and silver, the sediments; and garments, only bits of hair; and purple dye, blood; and everything else is of the same kind. And that which is of the nature of breath is also another thing of the same kind, changing from this to that.

\item Enough of this wretched life and murmuring and apish tricks. Why art thou disturbed? What is there new in this? What unsettles thee? Is it the form of the thing? Look at it. Or is it the matter? Look at it. But besides these there is nothing. Towards the gods then, now become at last more simple and better. It is the same whether we examine these things for a hundred years or three.

\item If a man has done wrong the harm is his own. But perhaps he has not done wrong.

\item Either all things proceed from one intelligent source and come together as in one body, and the part ought not to find fault with what is done for the benefit of the whole; or there are only atoms, and nothing else than mixture and dispersion. Why, then, art thou disturbed? Say to the ruling faculty, Art thou dead, art thou corrupted, art thou playing the hypocrite, art thou become a beast, dost thou herd and feed with the rest?\footnote{There is some corruption at the end of this section, but I think that the translation expresses the emperor's meaning. Whether intelligence rules all things or chance rules, a man must not be disturbed. He must use the power that he has and be tranquil.}

\item Either the gods have no power or they have power. If, then, they have no power, why dost thou pray to them? But if they have power, why dost thou not pray for them to give thee the faculty of not fearing any of the things which thou fearest, or of not desiring any of the things which thou desirest, or not being pained at anything, rather than pray that any of these things should not happen or happen? for certainly if they can co-operate with men, they can co-operate for these purposes. But perhaps thou wilt say the gods have placed them in thy power. Well, then, is it not better to use what is in thy power like a free man than to desire in a slavish and abject way what is not in thy power? And who has told thee that the gods do not aid us, even in the things which are in our power? Begin, then, to pray for such things, and thou wilt see. One man prays thus: How shall I be able to lie with that woman? Do thou pray thus: How shall I not desire to lie with her? Another prays thus: How shall I be released from this? Pray thou: How shall I not desire to be released? Another thus: How shall I not lose my little son? Thou thus: How shall I not be afraid to lose him? In fine, turn thy prayers this way, and see what comes.

\item Epicurus says, In my sickness my conversation was not about my bodily sufferings, nor, says he, did I talk on such subjects to those who visited me; but I continued to discourse on the nature of things as before, keeping to this main point, how the mind, while participating in such movements as go on in the poor flesh, shall be free from perturbations and maintain its proper good. Nor did I, he says, give the physicians an opportunity of putting on solemn looks, as if they were doing something great, but my life went on well and happily. Do, then, the same that he did both in sickness, if thou art sick, and in any other circumstances; for never to desert philosophy in any events that may befall us, nor to hold trifling talks either with an ignorant man or with one unacquainted with nature, is a principle of all schools of philosophy; but to be intent only on that which thou art now doing and on the instrument by which thou doest it.

\item When thou art offended with any man's shameless conduct, immediately ask thyself, Is it possible, then, that shameless men should not be in the world? It is not possible. Do not, then, require what is impossible. For this man also is one of those shameless men who must of necessity be in the world. Let the same considerations be present to thy mind in the case of the knave, and the faithless man, and of every man who does wrong in any way. For at the same time that thou dost remind thyself that it is impossible that such kind of men should not exist, thou wilt become more kindly disposed towards every one individually. It is useful to perceive this, too, immediately when the occasion arises, what virtue nature has given to man to oppose to every wrongful act. For she has given to man, as an antidote against the stupid man, mildness, and against another kind of man some other power. And in all cases it is possible for thee to correct by teaching the man who is gone astray; for every man who errs misses his object and is gone astray. Besides, wherein hast thou been injured? For thou wilt find that no one among those against whom thou art irritated has done anything by which thy mind could be made worse; but that which is evil to thee and harmful has its foundation only in the mind. And what harm is done or what is there strange, if the man who has not been instructed does the acts of an uninstructed man? Consider whether thou shouldst not rather blame thyself, because thou didst not expect such a man to err in such a way. For thou hadst means given thee by thy reason to suppose that it was likely that he would commit this error, and yet thou hast forgotten and art amazed that he has erred. But most of all when thou blamest a man as faithless or ungrateful, turn to thyself. For the fault is manifestly thy own, whether thou didst trust that a man who had such a disposition would keep his promise, or when conferring thy kindness thou didst not confer it absolutely, nor yet in such way as to have received from thy very act all the profit. For what more dost thou want when thou hast done a man a service? art thou not content that thou hast done something conformable to thy nature, and dost thou seek to be paid for it? just as if the eye demanded a recompense for seeing, or the feet for walking. For as these members are formed for a particular purpose, and by working according to their several constitutions obtain what is their own;\footnote{\textgreek{Ἀπέχει τὸ ἴδιον}. This sense of \textgreek{ὰπέχειν} occurs in xi. 1, and iv. 49; also in St. Matthew, vi. 2, \textgreek{ἀπέχουσίτὸν μισθον}, and in Epictetus.} so also as man is formed by nature to acts of benevolence, when he has done anything benevolent or in any other way conducive to the common interest, he has acted conformably to his constitution, and he gets what is his own.
\end{enumerate}