\begin{enumerate}
\item In the morning when thou risest unwillingly, let this thought be present—I am rising to the work of a human being. Why then am I dissatisfied if I am going to do the things for which I exist and for which I was brought into the world? Or have I been made for this, to lie in the bed-clothes and keep myself warm?—But this is more pleasant.—Dost thou exist then to take thy pleasure, and not at all for action or exertion? Dost thou not see the little plants, the little birds, the ants, the spiders, the bees working together to put in order their several parts of the universe? And art thou unwilling to do the work of a human being, and dost thou not make haste to do that which, is according to thy nature? But it is necessary to take rest also.—It is necessary. However, Nature has fixed bounds to this too: she has fixed bounds to eating and drinking, and yet thou goest beyond these bounds, beyond what is sufficient; yet in thy acts it is not so, but thou stoppest short of what thou canst do. So thou lovest not thyself, for if thou didst, thou wouldst love thy nature and her will. But those who love their several arts exhaust themselves in working at them unwashed and without food; but thou valuest thy own nature less than the turner values the turning art, or the dancer the dancing art, or the lover of money values his money, or the vain-glorious man his little glory. And such men, when they have a violent affection to a thing, choose neither to eat nor to sleep rather than to perfect the things which they care for. But are the acts which concern society more vile in thy eyes and less worthy of thy labor?

\item How easy it is to repel and to wipe away every impression which is troublesome or unsuitable, and immediately to be in all tranquility.

\item Judge every word and deed which are according to nature to be fit for thee; and be not diverted by the blame which follows from any people, nor by their words, but if a thing is good to be done or said, do not consider it unworthy of thee. For those persons have their peculiar leading principle and follow their peculiar movement; which things do not thou regard, but go straight on, following thy own nature and the common nature; and the way of both is one.

\item I go through the things which happen according to nature until I shall fall and rest, breathing out my breath into that element out of which I daily draw it in, and falling upon that earth out of which my father collected the seed, and my mother the blood, and my nurse the milk; out of which during so many years I have been supplied with food and drink; which bears me when I tread on it and abuse it for so many purposes.

\item Thou sayest, Men cannot admire the sharpness of thy wits.—Be it so: but there are many other things of which thou canst not say, I am not formed from them by nature. Show those qualities then which are altogether in thy power, sincerity, gravity, endurance of labor, aversion to pleasure, contentment with thy portion and with few things, benevolence, frankness, no love of superfluity, freedom from trifling, magnanimity. Dost thou not see how many qualities thou art immediately able to exhibit, in which there is no excuse of natural incapacity and unfitness, and yet thou still remainest voluntarily below the mark? or art thou compelled through being defectively furnished by nature to murmur, and to be stingy, and to flatter, and to find fault with thy poor body, and to try to please men, and to make great display, and to be so restless in thy mind? No, by the gods; but thou mightest have been delivered from these things long ago. Only if in truth thou canst be charged with being rather slow and dull of comprehension, thou must exert thyself about this also, not neglecting it nor yet taking pleasure in thy dullness.

\item One man, when he has done a service to another, is ready to set it down to his account as a favor conferred. Another is not ready to do this, but still in his own mind he thinks of the man as his debtor, and he knows what he has done. A third in a manner does not even know what he has done, but he is like a vine which has produced grapes, and seeks for nothing more after it has once produced its proper fruit. As a horse when he has run, a dog when he has tackled the game, a bee when it has made the honey, so a man when he has done a good act does not call out for others to come and see, but he goes on to another act, as a vine goes on to produce again the grapes in season.—Must a man then be one of these, who in a manner act thus without observing it?—Yes.—But this very thing is necessary, the observation of what a man is doing: for, it may be said, it is characteristic of the social animal to perceive that he is working in a social manner, and indeed to wish that his social partner also should perceive it.—It is true that thou sayest, but thou dost not rightly understand what is now said: and for this reason thou wilt become one of those of whom I spoke before, for even they are misled by a certain show of reason. But if thou wilt choose to understand the meaning of what is said, do not fear that for this reason thou wilt omit any social act.

\item A prayer of the Athenians: Rain, rain, O dear Zeus, down on the ploughed fields of the Athenians and on the plains.—In truth we ought not to pray at all, or we ought to pray in this simple and noble fashion.

\item Just as we must understand when it is said, That Aesculapius prescribed to this man horse-exercise, or bathing in cold water, or going without shoes, so we must understand it when it is said, That the nature of the universe prescribed to this man disease, or mutilation, or loss, or anything else of the kind. For in the first case Prescribed means something like this: he prescribed this for this man as a thing adapted to procure health; and in the second case it means, That which happens\footnote{In this section there is a play on the meaning of \textgreek{συμβα῟νείν}.} to [{\clarify or suits}] every man is fixed in a manner for him suitably to his destiny. For this is what we mean when we say that things are suitable to us, as the workmen say of squared stones in walls or the pyramids, that they are suitable, when they fit them to one another in some kind of connection. For there is altogether one fitness [{\clarify harmony}]. And as the universe is made up out of all bodies to be such a body as it is, so out of all existing causes necessity [{\clarify destiny}] is made up to be such a cause as it is. And even those who are completely ignorant understand what I mean; for they say, It [{\clarify necessity, destiny}] brought this to such a person.—This then was brought and this was prescribed to him. Let us then receive these things, as well as those which Aesculapius prescribes. Many as a matter of course even among his prescriptions are disagreeable, but we accept them in the hope of health. Let the perfecting and accomplishment of the things which the common nature judges to be good, be judged by thee to be of the same kind as thy health. And so accept everything which happens, even if it seem disagreeable, because it leads to this, to the health of the universe and to the prosperity and felicity of Zeus [{\clarify the universe}]. For he would not have brought on any man what he has brought, if it were not useful for the whole. Neither does the nature of anything, whatever it may be, cause anything which is not suitable to that which is directed by it. For two reasons then it is right to be content with that which happens to thee; the one, because it was done for thee and prescribed for thee, and in a manner had reference to thee, originally from the most ancient causes spun with thy destiny; and the other, because even that which comes severally to every man is to the power which administers the universe a cause of felicity and perfection, nay even of its very continuance. For the integrity of the whole is mutilated, if thou cuttest off anything whatever from the conjunction and the continuity either of the parts or of the causes. And thou dost cut off, as far as it is in thy power, when thou art dissatisfied, and in a manner triest to put anything out of the way.

\item Be not disgusted, nor discouraged, nor dissatisfied, if thou dost not succeed in doing everything according to right principles, but when thou hast failed, return back again, and be content if the greater part of what thou doest is consistent with man's nature, and love this to which thou returnest; and do not return to philosophy as if she were a master, but act like those who have sore eyes and apply a bit of sponge and egg, or as another applies a plaster, or drenching with water. For thus thou wilt not fail to obey reason, and thou wilt repose in it. And remember that philosophy requires only things which thy nature requires; but thou wouldst have something else which is not according to nature.—It may be objected, Why, what is more agreeable than this [{\clarify which I am doing}]? But is not this the very reason why pleasure deceives us? And consider if magnanimity, freedom, simplicity, equanimity, piety, are not more agreeable. For what is more agreeable than wisdom itself, when thou thinkest of the security and the happy course of all things which depend on the faculty of understanding and knowledge?

\item Things are in such a kind of envelopment that they have seemed to philosophers, not a few nor those common philosophers, altogether unintelligible; nay even to the Stoics themselves they seem difficult to understand. And all our assent is changeable; for where is the man who never changes? Carry thy thoughts then to the objects themselves, and consider how short-lived they are and worthless, and that they may be in the possession of a filthy wretch or a whore or a robber. Then turn to the morals of those who live with thee, and it is hardly possible to endure even the most agreeable of them, to say nothing of a man being hardly able to endure himself. In such darkness then and dirt, and in so constant a flux both of substance and of time, and of motion and of things moved, what there is worth being highly prized, or even an object of serious pursuit, I cannot imagine. But on the contrary it is a man's duty to comfort himself, and to wait for the natural dissolution, and not to be vexed at the delay, but to rest in these principles only: the one, that nothing will happen to me which is not conformable to the nature of the universe; and the other, that it is in my power never to act contrary to my god and daemon: for there is no man who will compel me to this.

\item About what am I now employing my own soul? On every occasion I must ask myself this question, and inquire, What have I now in this part of me which they call the ruling principle? and whose soul have I now—that of a child, or of a young man, or of a feeble woman, or of a tyrant, or of a domestic animal, or of a wild beast?

\item What kind of things those are which appear good to the many, we may learn even from this. For if any man should conceive certain things as being really good, such as prudence, temperance, justice, fortitude, he would not after having first conceived these endure to listen to anything which should not be in harmony with what is really good. But if a man has first conceived as good the things which appear to the many to be good, he will listen and readily receive as very applicable that which was said by the comic writer. Thus even the many perceive the difference. For were it not so, this saying would not offend and would not be rejected [{\clarify in the first case}], while we receive it when it is said of wealth, and of the means which further luxury and fame, as said fitly and wittily. Go on then and ask if we should value and think those things to be good, to which after their first conception in the mind the words of the comic writer might be aptly applied—that he who has them, through pure abundance has not a place to ease himself in.

\item I am composed of the formal and the material; and neither of them will perish into non-existence, as neither of them came into existence out of non-existence. Every part of me then will be reduced by change into some part of the universe, and that again will change into another part of the universe, and so on forever. And by consequence of such a change I too exist, and those who begot me, and so on forever in the other direction. For nothing hinders us from saying so, even if the universe is administered according to definite periods [{\clarify of revolution}].

\item Reason and the reasoning art [{\clarify philosophy}] are powers which are sufficient for themselves and for their own works. They move then from a first principle which is their own, and they make their way to the end which is proposed to them; and this is the reason why such acts are named Catorthoseis or right acts, which word signifies that they proceed by the right road.

\item None of these things ought to be called a man's, which do not belong to a man, as man. They are not required of a man, nor does man's nature promise them, nor are they the means of man's nature attaining its end. Neither then does the end of man lie in these things, nor yet that which aids to the accomplishment of this end, and that which aids toward this end is that which is good. Besides, if any of these things did belong to man, it would not be right for a man to despise them and to set himself against them; nor would a man be worthy of praise who snowed that he did not want these things, nor would he who stinted himself in any of them be good, if indeed these things were good. But now the more of these things a man deprives himself of, or of other things like them, or even when he is deprived of any of them, the more patiently he endures the loss, just in the same degree he is a better man.

\item Such as are thy habitual thoughts, such also will be the character of thy mind; for the soul is dyed by the thoughts. Dye it then with a continuous series of such thoughts as these: for instance, that where a man can live, there he can also live well. But he must live in a palace; well then, he can also live well in a palace. And again, consider that for whatever purpose each thing has been constituted, for this it has been constituted, and towards this it is carried; and its end is in that towards which it is carried; and where the end is, there also is the advantage and the good of each thing. Now the good for the reasonable animal is society; for that we are made for society has been shown above.\footnote{ii. 1.} Is it not plain that the inferior exists for the sake of the superior? But the things which have life are superior to those which have not life, and of those which have life the superior are those which have reason.

\item To seek what is impossible is madness: and it is impossible that the bad should not do something of this kind.

\item Nothing happens to any man which he is not formed by nature to bear. The same things happen to another, and either because he does not see that they have happened, or because he would show a great spirit, he is firm and remains unharmed. It is a shame then that ignorance and conceit should be stronger than wisdom.

\item Things themselves touch not the soul, not in the least degree; nor have they admission to the soul, nor can they turn or move the soul: but the soul turns and moves itself alone, and whatever judgments it may think proper to make, such it makes for itself the things which present themselves to it.

\item In one respect man is the nearest thing to me, so far as I must do good to men and endure them. But so far as some men make themselves obstacles to my proper acts, man becomes to me one of the things which are indifferent, no less than the sun or wind or a wild beast. Now it is true that these may impede my action, but they are no impediments to my affects and disposition, which have the power of acting conditionally and changing: for the mind converts and changes every hindrance to its activity into an aid; and so that which is a hindrance is made a furtherance to an act; and that which is an obstacle on the road helps us on this road.

\item Reverence that which is best in the universe; and this is that which makes use of all things and directs all things. And in like manner also reverence that which is best in thyself; and this is of the same kind as that. For in thyself also, that which makes use of everything else is this, and thy life is directed by this.

\item That which does no harm to the state, does no harm to the citizen. In the case of every appearance of harm apply this rule: if the state is not harmed by this, neither am I harmed. But if the state is harmed, thou must not be angry with him who does harm to the state. Show him where his error is.

\item Often think of the rapidity with which things pass by and disappear, both the things which are and the things which are produced. For substance is like a river in a continual flow, and the activities of things are in constant change, and the causes work in infinite varieties; and there is hardly anything which stands still. And consider this which is near to thee, this boundless abyss of the past and of the future in which all things disappear. How then is he not a fool who is puffed up with such things or plagued about them and makes himself miserable? for they vex him only for a time, and a short time.

\item Think of the universal substance, of which thou hast a very small portion; and of universal time, of which a short and indivisible interval has been assigned to thee; and of that which is fixed by destiny, and how small a part of it thou art.

\item Does another do me wrong? Let him look to it. He has his own disposition, his own activity. I now have what the universal nature now wills me to have; and I do what my nature now wills me to do.

\item Let the part of thy soul which leads and governs be undisturbed by the movements in the flesh, whether of pleasure or of pain; and let it not unite with them, but let it circumscribe itself and limit those affects to their parts. But when these affects rise up to the mind by virtue of that other sympathy that naturally exists in a body which is all one, then thou must not strive to resist the sensation, for it is natural: but let not the ruling part of itself add to the sensation the opinion that it is either good or bad.

\item Live with the gods. And he does live with the gods who constantly shows to them that his own soul is satisfied with that which is assigned to him, and that it does all that the daemon wishes, which Zeus hath given to every man for his guardian and guide, a portion of himself. And this is every man's understanding and reason.

\item Art thou angry with him whose armpits stink? art thou angry with him whose mouth smells foul? What good will this anger do thee? He has such a mouth, he has such armpits: it is necessary that such an emanation must come from such things: but the man has reason, it will be said, and he is able, if he takes pains, to discover wherein he offends; I wish thee well of thy discovery. Well then, and thou hast reason: by thy rational faculty stir up his rational faculty; show him his error, admonish him. For if he listens, thou wilt cure him, and there is no need of anger. [{\clarify Neither tragic actor nor whore.}]\footnote{This is imperfect or corrupt, or both. There is also something wrong or incomplete in the beginning of S. 29, where he says \textgreek{ὠς ἐξελθὼν ζῇν διανοῇ}, which Gataker translates ``as if thou wast about to quit life;" but we cannot translate \textgreek{ἐξελθών} in that way. Other translations are not much more satisfactory. I have translated it literally and left it imperfect.}

\item As thou intendest to live when them art gone out\ldots so it is in thy power to live here. But if men do not permit thee, then get away out of life, yet so as if thou wert suffering no harm. The house is smoky, and I quit it.\footnote{Epictetus, i. 25, 18.} Why dost thou think that this is any trouble? But so long as nothing of the kind drives me out, I remain, am free, and no man shall hinder me from doing what I choose; and I choose to do what is according to the nature of the rational and social animal.

\item The intelligence of the universe is social. Accordingly it has made the inferior things for the sake of the superior, and it has fitted the superior to one another. Thou seest how it has subordinated, co-ordinated, and assigned to everything its proper portion, and has brought together into concord with one another the things which are the best.

\item How hast thou behaved hitherto to the gods, thy parents, brethren, children, teachers, to those who looked after thy infancy, to thy friends, kinsfolk, to thy slaves? Consider if thou hast hitherto behaved to all in such a way that this may be said of thee—
\begin{displayquote}
	``Never has wronged a man in deed or word."
\end{displayquote}
And call to recollection both how many things thou hast passed through, and how many things thou hast been able to endure, and that the history of thy life is now complete and thy service is ended; and how many beautiful things thou hast seen; and how many pleasures and pains thou hast despised; and how many things called honorable thou hast spurned; and to how many ill-minded folks thou hast shown a kind disposition.

\item Why do unskilled and ignorant souls disturb him who has skill and knowledge? What soul then has skill and knowledge? That which knows beginning and end, and knows the reason which pervades all substance, and though all time by fixed periods [{\clarify revolutions}] administers the universe.

\item Soon, very soon, thou wilt be ashes, or a skeleton, and either a name or not even a name; but name is sound and echo. And the things which are much valued in life are empty and rotten and trifling, and [{\clarify like}] little dogs biting one another, and little children quarreling, laughing, and then straightway weeping. But fidelity and modesty and justice and truth are fled
\begin{displayquote}
	Up to Olympus from the wide-spread earth.\\
	—Hesiod, \textit{Works, etc.} V. 197.
\end{displayquote}
What then is there which still detains thee here, if the objects of sense are easily changed and never stand still, and the organs of perception are dull and easily receive false impressions, and the poor soul itself is an exhalation from blood? But to have good repute amid such a world as this is an empty thing. Why then dost thou not wait in tranquility for thy end, whether it is extinction or removal to another state? And until that time comes, what is sufficient? Why, what else than to venerate the gods and bless them, and to do good to men, and to practise tolerance and self-restraint;\footnote{This is the Stoic precept \textgreek{άνέχον καί άπέχον}. The first part teaches us to be content with men and things as they are. The second part teaches us the virtue of self-restraint, or the government of our passions.} but as to everything which is beyond the limits of the poor flesh and breath, to remember that this is neither thine nor in thy power.

\item Thou canst pass thy life in an equable flow of happiness, if thou canst go by the right way, and think and act in the right way. These two things are common both to the soul of God and to the soul of man, and to the soul of every rational being: not to be hindered by another; and to hold good to consist in the disposition to justice and the practice of it, and in this to let thy desire find its termination.

\item If this is neither my own badness, nor an effect of my own badness, and the common weal is not injured, why am I troubled about it, and what is the harm to the common weal?

\item Do not be carried along inconsiderately by the appearance of things, but give help [{\clarify to all}] according to thy ability and their fitness; and if they should have sustained loss in matters which are indifferent, do not imagine this to be a damage; for it is a bad habit. But as the old man, when he went away, asked back his foster-child's top, remembering that it was a top, so do thou in this case also.

When thou art calling out on the Rostra, hast thou forgotten, man, what these things are?—Yes; but they are objects of great concern to these people—wilt thou too then be made a fool for these things? I was once a fortunate man, but I lost it, I know not how.—But fortunate means that a man has assigned to himself a good fortune: and a good fortune is good disposition of the soul, good emotions, good actions.\footnote{This section is unintelligible. Many of the words may be corrupt, and the general purport of the section cannot be discovered. Perhaps several things have been improperly joined in one section. I have translated it nearly literally. Different translators give the section a different turn, and the critics have tried to mend what they cannot understand.}
\end{enumerate}