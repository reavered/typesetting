\renewcommand\labelitemi{}
\begin{itemize}[leftmargin=*]
\setlength\itemsep{0em}
\item
  \textgreek{ἀδιάφορα} (indifferentia, Cicero, Seneca, Epp. 82); things indifferent,
  neither good nor bad; the same as \textgreek{μέσα}.
\item
  \textgreek{αἰσχρός} (turpis, Cic.), ugly; morally ugly.
\item
  \textgreek{αἰτία}, cause.
\item
  \textgreek{αἰτιῶδες, αἴτιον, τό}, the formal or formative principle, the cause.
\item
  \textgreek{ἀκοινώνητος}, unsocial.
\item
  \textgreek{ἀναφορά}, reference, relation to a purpose.
\item
  \textgreek{ἀνυπεξαιρέτως}, unconditionally.
\item
  \textgreek{ἀπόῤῥοια}, efflux.
\item
  \textgreek{ἀπροαίρετα, τά}, the things which are not in our will or power.
\item
  \textgreek{ἀρχέ}, a first principle.
\item
  \textgreek{ἄτομοι} (corpora individua, Cic.), atoms.
\item
  \textgreek{αὐτάρκεια} est quae parvo contenta omne id respuit quod abundat (Cic.);
  contentment.
\item
  \textgreek{αὐτάρκης}, sufficient in itself; contented.
\item
  \textgreek{ἀφορμαί}, means, principles. The word has also other significations in
  Epictetus. Index ed. Schweig.
\end{itemize}

\begin{itemize}[leftmargin=*]
\setlength\itemsep{0em}
\item
  \textgreek{γιγνόμενα, τά}, things which are produced, come into existence.
\end{itemize}

\begin{itemize}[leftmargin=*]
\setlength\itemsep{0em}
\item
  \textgreek{δαίμων}, god, god in man, man's intelligent principle.
\item
  \textgreek{διάθεσις}, disposition, affection of the mind.
\item
  \textgreek{διαίρεσις}, division of things into their parts, dissection,
  resolution, analysis.
\item
  \textgreek{διαλεκτική}, ars bene disserendi et vera ac falsa dijudicandi (Cic.).
\item
  \textgreek{διάλυσις}, dissolution, the opposite of \textgreek{σύγκρισις}.
\item
  \textgreek{διάνοια}, understanding; sometimes, the mind generally, the whole
  intellectual power.
\item
  \textgreek{δόγματα} (decreta, Cic.), principles.
\item
  \textgreek{δύναμις νοερά}, intellectual faculty.
\end{itemize}

\begin{itemize}[leftmargin=*]
\setlength\itemsep{0em}
\item
  \textgreek{ἐγκράτεια}, temperance, self-restraint.
\item
  \textgreek{εἶδος} in divisione formae sunt, quas Graeci \textgreek{εἶδη} vocant; nostri, si
  qui haec forte tractant, species appellant (Cic.). But \textgreek{εἶδος} is used
  by Epictetus and Antoninus less exactly and as a general term, like
  genus. Index Epict. ed. Schweig.---\textgreek{Ὡς δέ γε αἱ πρῶται οὐσίαι πρὸς τὰ
  ἄλλα ἔχουσιν, οὕτω καὶ τὸ εἶδος πρὸς τὸ γένος ἔχει ὑποκεῖται γὰρ τὸ
  εἶδος τῷ γένει} [\clarify{Transliteration text}] (Aristot. Cat. c. 5.)
\item
  \textgreek{εἰμαρμένη} (fatalis necessitas, fatum, Cic.), destiny, necessity.
\item
  \textgreek{ἐκκλίσεις}, aversions, avoidance, the turning away from things; the
  opposite of \textgreek{ὀρέξεις}.
\item
  \textgreek{ἔμψυχα, τά}, things which have life.
\item
  \textgreek{ἐνέργεια}, action, activity.
\item
  \textgreek{ἕννοια, ἕννοιαι}, notio, notiones (Cic.), or ``notitiae rerum;''
  notions of things. (Notionem appello quam Graeci tum \textgreek{έννοιαν}, tum
  \textgreek{πρόληψιν}, Cic.).
\item
  \textgreek{ἕνωσις, ἡ}, the unity.
\item
  \textgreek{ἐπιστροφή}, attention to an object.
\item
  \textgreek{εὐθυμία}, animi tranquillitas (Cic.).
\item
  \textgreek{εὐμενές, τό, εὐμενεία}, benevolence; \textgreek{εὐμενής} sometimes means
  well-contented.
\item
  \textgreek{εὔνοια}, benevolence.
\item
  \textgreek{ἐξουσία}, power, faculty.
\item
  \textgreek{ἐπακολούθησιν, κατὰ}, by way of sequence.
\end{itemize}

\begin{itemize}[leftmargin=*]
\setlength\itemsep{0em}
\item
  \textgreek{ἡγεμονικόν, τό}, the ruling faculty or part; principatus (Cic.).
\end{itemize}

\begin{itemize}[leftmargin=*]
\setlength\itemsep{0em}
\item
  \textgreek{θεωρήματα}, percepta (Cic.), things perceived, general principles.
\end{itemize}

\begin{itemize}[leftmargin=*]
\setlength\itemsep{0em}
\item
  \textgreek{καθήκειν, τό}, duty, ``officium.''
\item
  \textgreek{καλός}, beautiful.
\item
  \textgreek{κατάληψις}, comprehension; cognitio, perceptio, comprehensio (Cic.).
\item
  \textgreek{κατασκευή}, constitution.
\item
  \textgreek{κατορθώσεις, καταρθώματα} recta, recte facta (Cic.); right acts, those
  acts to which we proceed by the right or straight road.
\item
  \textgreek{κόσμος}, order, world, universe.
\item
  \textgreek{κόσμος, ὁ ὃλος}, the universe, that which is the One and the all (vi.
  25).
\item
  \textgreek{κρίμα}, a judgment.
\item
  \textgreek{κυριεῦον τὸ ἔνδον}, that which rules within (iv. 1), the same as \textgreek{τὸ
  ἡγεμονικόν}. Diogenes Laertius vii., Zeno. \textgreek{ἡγεμονικόν δε εἔναι το
  κυριώτατον τῆς ψυχῆς}.
\end{itemize}

\begin{itemize}[leftmargin=*]
\setlength\itemsep{0em}
\item
  \textgreek{λογικά, τά}, the things which have reason.
\item
  \textgreek{λογικός}, rational.
\item
  \textgreek{λόγος}, reason.
\item
  \textgreek{λόγος σπερματικός}, seminal principle.
\end{itemize}

\begin{itemize}[leftmargin=*]
\setlength\itemsep{0em}
\item
  \textgreek{μέσα, τά}, things indifferent, viewed with respect to virtue.
\end{itemize}

\begin{itemize}[leftmargin=*]
\setlength\itemsep{0em}
\item
  \textgreek{νοερός}, intellectual.
\item
  \textgreek{νόμος}, law.
\item
  \textgreek{νοῦς}, intelligence, understanding.
\end{itemize}

\begin{itemize}[leftmargin=*]
\setlength\itemsep{0em}
\item
  \textgreek{οἴησις}, arrogance, pride. It sometimes means in Antoninus the same as
  \textgreek{τῦφος} but it also means ``opinion.''
\item
  \textgreek{οἰκονομία} (dispositio, ordo, Cic.) has sometimes the peculiar sense of
  artifice, or doing something with an apparent purpose different from
  the real purpose.
\item
  \textgreek{ὅλον, τό}, the universe, the whole: \textgreek{ἡ τῶν ὄλων φύσις}.
\item
  \textgreek{ὄντα, τά}, things which exist; existence, being.
\item
  \textgreek{ὄρεξις}, desire of a thing, which is opposed to \textgreek{ἔκκλισις}, aversion.
\item
  \textgreek{ὁρμή}, movement towards an object, appetite; appetitio, naturalis
  appetitus, appetitus animi (Cic.).
\item
  \textgreek{οὐσία}, substance (vi. 49). Modern writers sometimes incorrectly
  translate it ``essentia.'' It is often used by Epictetus in the same
  sense as \textgreek{ὕλη}. Aristotle (Cat. c. 5) defines \textgreek{οὐσία}, and it is properly
  translated ``substantia'' (ed. Jul. Pacius). Porphyrius (Isag. c. 2):
  \textgreek{ἡ οὐσία ἀνωτάτω οὐσα τῷ μηδὲν πρὸ αὐτῆς γένος ἠν τὸ γενικώτατον}.
\end{itemize}

\begin{itemize}[leftmargin=*]
\setlength\itemsep{0em}
\item
  \textgreek{παρακολουθητικὴ δύναμις, ἡ}, the power which enables us to observe and
  understand.
\item
  \textgreek{πεῑσις}, passivity, opposed to \textgreek{ἐνέργεια}: also, affect.
\item
  \textgreek{περιστάσεις}, circumstances, the things which surround us; troubles,
  difficulties.
\item
  \textgreek{πεπρωμένη, ἡ}, destiny.
\item
  \textgreek{προαίρεσις}, purpose, free will (Aristot. Rhet. i. 13).
\item
  \textgreek{προαίρετά, τά}, things which are within our will or power.
\item
  \textgreek{προαιρετικόν, τό}, free will.
\item
  \textgreek{πρόθεσις}, a purpose, proposition.
\item
  \textgreek{πρόνοια} (providentia, Cic.), providence.
\end{itemize}

\begin{itemize}[leftmargin=*]
\setlength\itemsep{0em}
\item
  \textgreek{σκοπός}, object, purpose.
\item
  \textgreek{στοιχεῖον}, element.
\item
  \textgreek{συγκατάθεσις} (assensio, approbatio, Cic.), assent; \textgreek{συγκαταθέσεις}
  (probationes, Gellius, xix. 1).
\item
  \textgreek{συγκρίματα}, things compounded (ii. 3).
\item
  \textgreek{σύγκρισις}, the act of combining elements out of which a body is
  produced, combination.
\item
  \textgreek{σύνθεσις}, ordering, arrangement (compositio).
\item
  \textgreek{σύστημα}, system, a thing compounded of parts which have a certain
  relation to one another.
\end{itemize}

\begin{itemize}[leftmargin=*]
\setlength\itemsep{0em}
\item
  \textgreek{ὕλη}, matter, material.
\item
  \textgreek{ὑλικόν, τό}, the material principle.
\item
  \textgreek{ὑπεξαίρεσις}, exception, reservation; \textgreek{μεθ᾽ ὑπεξαιρέσεως}, conditionally.
\item
  \textgreek{ὑπόθεσις}, material to work on; thing to employ the reason on;
  proposition, thing assumed as matter for argument and to lead to
  conclusions. (Quaestionum duo sunt genera; alterum infinitum,
  definitum alterum. Definitum est, quod \textgreek{ὑπόθεσιν} Graeci, nos causam:
  infinitum, quod \textgreek{θέσιν} illi appellant, nos propositum possumus
  nominare. Cic. See Aristot. Anal. Post. i. c. 2).
\item
  \textgreek{ὑποκείμενα, τά}, things present or existing, vi. 4; or things which are
  a basis or foundation.
\item
  \textgreek{ὑπόληψις}, opinion.
\item
  \textgreek{ὑπόστασις}, basis, substance, being, foundation (x. 5). Epictetus has
  \textgreek{τὸ ὑποστατικὸν καὶ οὐσιῶδες}. (Justinus ad Diogn. c. 2.)
\item
  \textgreek{ὑφίστασθαι}, to subsist, to be.
\end{itemize}

\begin{itemize}[leftmargin=*]
\setlength\itemsep{0em}
\item
  \textgreek{φαντασίαι} (visus, Cic.); appearances, thoughts, impressions (visa
  animi, Gellius, xix. 1): \textgreek{φαντασία ἐστὶ τύπωσις ἐν ψυχῄ}.
\item
  \textgreek{φάντασμα}, seems to be used by Antoninus in the same sense as \textgreek{φαντασία}.
  Epictetus uses only \textgreek{φαντασία}.
\item
  \textgreek{φανταστόν}, that which produces a \textgreek{φαντασία: φανταστὸν τὸ τεπσιηκὸς τὴν
  φαντασίαν αίσθητόν}.
\item
  \textgreek{φύσις}, nature.
\item
  \textgreek{φύσις ἡ τῶν ὄλων}, the nature of the universe.
\end{itemize}

\begin{itemize}[leftmargin=*]
\setlength\itemsep{0em}
\item
  \textgreek{ψυχή}, soul, life, living principle.
\item
  \textgreek{ψυχὴ λογική, νοερά}, a rational soul, an intelligent soul
\end{itemize}

