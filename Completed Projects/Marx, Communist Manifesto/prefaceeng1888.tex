The Manifesto was published as the platform of the Communist League, a working men's association, first exclusively German, later on international, and under the political conditions of the Continent before 1848, unavoidably a secret society. At a Congress of the League, held in November 1847, Marx and Engels were commissioned to prepare a complete theoretical and practical party programme. Drawn up in German, in January 1848, the manuscript was sent to the printer in London a few weeks before the French Revolution of February 24. A French translation was brought out in Paris shortly before the insurrection of June 1848. The first English translation, by Miss Helen Macfarlane, appeared in George Julian Harney's \emph{Red Republican}, London, 1850. A Danish and a Polish edition had also been published. 

The defeat of the Parisian insurrection of June 1848---the first great battle between proletariat and bourgeoisie---drove again into the background, for a time, the social and political aspirations of the European working class. Thenceforth, the struggle for supremacy was, again, as it had been before the Revolution of February, solely between different sections of the propertied class; the working class was reduced to a fight for political elbow-room, and to the position of extreme wing of the middle-class Radicals. Wherever independent proletarian movements continued to show signs of life, they were ruthlessly hunted down. Thus the Prussian police hunted out the Central Board of the Communist League, then located in Cologne. The members were arrested and, after eighteen months'imprisonment, they were tried in October 1852. This celebrated ``Cologne Communist Trial'' lasted from October 4 till November 12; seven of the prisoners were sentenced to terms of imprisonment in a fortress, varying from three to six years. Immediately after the sentence, the League was formally dissolved by the remaining members. As to the Manifesto, it seemed henceforth doomed to oblivion. 

When the European workers had recovered sufficient strength for another attack on the ruling classes, the International Working Men's Association sprang up. But this association, formed with the express aim of welding into one body the whole militant proletariat of Europe and America, could not at once proclaim the principles laid down in the Manifesto. The International was bound to have a programme broad enough to be acceptable to the English trade unions, to the followers of Proudhon in France, Belgium, Italy, and Spain, and to the Lassalleans in Germany.\footnote{Lassalle personally, to us, always acknowledged himself to be a disciple of Marx, and, as such, stood on the ground of the Manifesto. But in his first public agitation, 1862-1864, he did not go beyond demanding co-operative workshops supported by state credit. }

Marx, who drew up this programme to the satisfaction of all parties, entirely trusted to the intellectual development of the working class, which was sure to result from combined action and mutual discussion. The very events and vicissitudes in the struggle against capital, the defeats even more than the victories, could not help bringing home to men's minds the insufficiency of their various favorite nostrums, and preparing the way for a more complete insight into the true conditions for working-class emancipation. And Marx was right. The International, on its breaking in 1874, left the workers quite different men from what it found them in 1864. Proudhonism in France, Lassalleanism in Germany, were dying out, and even the conservative English trade unions, though most of them had long since severed their connection with the International, were gradually advancing towards that point at which, last year at Swansea, their president could say in their name: ``Continental socialism has lost its terror for us.'' In fact, the principles of the Manifesto had made considerable headway among the working men of all countries. 

The Manifesto itself came thus to the front again. Since 1850, the German text had been reprinted several times in Switzerland, England, and America. In 1872, it was translated into English in New York, where the translation was published in \emph{Woorhull and Claflin's Weekly}. From this English version, a French one was made in \emph{Le Socialiste} of New York. Since then, at least two more English translations, more or less mutilated, have been brought out in America, and one of them has been reprinted in England. The first Russian translation, made by Bakunin, was published at Herzen's Kolokol office in Geneva, about 1863; a second one, by the heroic Vera Zasulich, also in Geneva, in 1882. A new Danish edition is to be found in \emph{Socialdemokratisk Bibliothek}, Copenhagen, 1885; a fresh French translation in \emph{Le Socialiste}, Paris, 1886. From this latter, a Spanish version was prepared and published in Madrid, 1886. The German reprints are not to be counted; there have been twelve altogether at the least. An Armenian translation, which was to be published in Constantinople some months ago, did not see the light, I am told, because the publisher was afraid of bringing out a book with the name of Marx on it, while the translator declined to call it his own production. Of further translations into other languages I have heard but had not seen. Thus the history of the Manifesto reflects the history of the modern working-class movement; at present, it is doubtless the most wide spread, the most international production of all socialist literature, the common platform acknowledged by millions of working men from Siberia to California. 

Yet, when it was written, we could not have called it a \emph{socialist} manifesto. By Socialists, in 1847, were understood, on the one hand the adherents of the various Utopian systems: Owenites in England, Fourierists in France, both of them already reduced to the position of mere sects, and gradually dying out; on the other hand, the most multifarious social quacks who, by all manner of tinkering, professed to redress, without any danger to capital and profit, all sorts of social grievances, in both cases men outside the working-class movement, and looking rather to the ``educated'' classes for support. Whatever portion of the working class had become convinced of the insufficiency of mere political revolutions, and had proclaimed the necessity of total social change, called itself Communist. It was a crude, rough-hewn, purely instinctive sort of communism; still, it touched the cardinal point and was powerful enough amongst the working class to produce the Utopian communism of Cabet in France, and of Weitling in Germany. Thus, in 1847, socialism was a middle-class movement, communism a working-class movement. Socialism was, on the Continent at least, ``respectable''; communism was the very opposite. And as our notion, from the very beginning, was that ``the emancipation of the workers must be the act of the working class itself,'' there could be no doubt as to which of the two names we must take. Moreover, we have, ever since, been far from repudiating it. 

The Manifesto being our joint production, I consider myself bound to state that the fundamental proposition which forms the nucleus belongs to Marx. That proposition is: That in every historical epoch, the prevailing mode of economic production and exchange, and the social organization necessarily following from it, form the basis upon which it is built up, and from that which alone can be explained the political and intellectual history of that epoch; that consequently the whole history of mankind (since the dissolution of primitive tribal society, holding land in common ownership) has been a history of class struggles, contests between exploiting and exploited, ruling and oppressed classes; That the history of these class struggles forms a series of evolutions in which, nowadays, a stage has been reached where the exploited and oppressed class---the proletariat---cannot attain its emancipation from the sway of the exploiting and ruling class---the bourgeoisie---without, at the same time, and once and for all, emancipating society at large from all exploitation, oppression, class distinction, and class struggles. 

This proposition, which, in my opinion, is destined to do for history what Darwin's theory has done for biology, we both of us, had been gradually approaching for some years before 1845. How far I had independently progressed towards it is best shown by my ``Conditions of the Working Class in England.''\footnote{\emph{The Condition of the Working Class in England in 1844}. By Frederick Engels. Translated by Florence K. Wischnewetzky—London, Swan, Sonnenschein \& Co.} But when I again met Marx at Brussels, in spring 1845, he had it already worked out and put it before me in terms almost as clear as those in which I have stated it here. 

From our joint preface to the German edition of 1872, I quote the following: 
\begin{displayquote}
``However much that state of things may have altered during the last twenty-five years, the general principles laid down in the Manifesto are, on the whole, as correct today as ever. Here and there, some detail might be improved. The practical application of the principles will depend, as the Manifesto itself states, everywhere and at all times, on the historical conditions for the time being existing, and, for that reason, no special stress is laid on the revolutionary measures proposed at the end of Section II. That passage would, in many respects, be very differently worded today. In view of the gigantic strides of Modern Industry since 1848, and of the accompanying improved and extended organization of the working class, in view of the practical experience gained, first in the February Revolution, and then, still more, in the Paris Commune, where the proletariat for the first time held political power for two whole months, this programme has in some details been antiquated. One thing especially was proved by the Commune, viz., that ``the working class cannot simply lay hold of ready-made state machinery, and wield it for its own purposes.'' (See The Civil War in France: Address of the General Council of the International Working Men's Association 1871, where this point is further developed.) Further, it is self-evident that the criticism of socialist literature is deficient in relation to the present time, because it comes down only to 1847; also that the remarks on the relation of the Communists to the various opposition parties (Section IV), although, in principle still correct, yet in practice are antiquated, because the political situation has been entirely changed, and the progress of history has swept from off the Earth the greater portion of the political parties there enumerated. 

``But then, the Manifesto has become a historical document which we have no longer any right to alter.'' 
\end{displayquote}
The present translation is by Mr Samuel Moore, the translator of the greater portion of Marx's ``Capital.'' We have revised it in common, and I have added a few notes explanatory of historical allusions. 

\hfill
\begin{tabular}{@{}l@{}}
\scshape Frederick Engels \\
\emph{30 January 1888, London.}
\end{tabular}
