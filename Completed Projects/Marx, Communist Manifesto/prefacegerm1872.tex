The Communist League, an international association of workers, which could of course be only a secret one, under conditions obtaining at the time, commissioned us, the undersigned, at the Congress held in London in November 1847, to write for publication a detailed theoretical and practical programme for the Party. Such was the origin of the following Manifesto, the manuscript of which travelled to London to be printed a few weeks before the February [\clarify{French}] Revolution [\clarify{in 1848}]. First published in German, it has been republished in that language in at least twelve different editions in Germany, England, and America. It was published in English for the first time in 1850 in the Red Republican, London, translated by Miss Helen Macfarlane, and in 1871 in at least three different translations in America. The French version first appeared in Paris shortly before the June insurrection of 1848, and recently in Le Socialiste of New York. A new translation is in the course of preparation. A Polish version appeared in London shortly after it was first published in Germany. A Russian translation was published in Geneva in the sixties. Into Danish, too, it was translated shortly after its appearance. 

However much that state of things may have altered during the last twenty-five years, the general principles laid down in the Manifesto are, on the whole, as correct today as ever. Here and there, some detail might be improved. The practical application of the principles will depend, as the Manifesto itself states, everywhere and at all times, on the historical conditions for the time being existing, and, for that reason, no special stress is laid on the revolutionary measures proposed at the end of Section II. That passage would, in many respects, be very differently worded today. In view of the gigantic strides of Modern Industry since 1848, and of the accompanying improved and extended organization of the working class, in view of the practical experience gained, first in the February Revolution, and then, still more, in the Paris Commune, where the proletariat for the first time held political power for two whole months, this programme has in some details been antiquated. One thing especially was proved by the Commune, viz., that ``the working class cannot simply lay hold of the ready-made state machinery, and wield it for its own purposes.'' (See The Civil War in France: Address of the General Council of the International Working Men’s Association, 1871, where this point is further developed.) Further, it is self-evident that the criticism of socialist literature is deficient in relation to the present time, because it comes down only to 1847; also that the remarks on the relation of the Communists to the various opposition parties (Section IV), although, in principle still correct, yet in practice are antiquated, because the political situation has been entirely changed, and the progress of history has swept from off the earth the greater portion of the political parties there enumerated. 

But then, the Manifesto has become a historical document which we have no longer any right to alter. A subsequent edition may perhaps appear with an introduction bridging the gap from 1847 to the present day; but this reprint was too unexpected to leave us time for that. 

\hfill
\begin{tabular}{@{}l@{}}
\scshape Karl Marx \& Frederick Engels \\
\emph{24 June 1872, London.}
\end{tabular}