\documentclass[12pt,openany]{book}
\usepackage[12pt]{moresize}
\usepackage[utf8x]{inputenc}
\usepackage[english]{babel}
\usepackage[margin=2.5cm]{geometry}
\usepackage{titlesec}
\usepackage{ebgaramond}
\usepackage[]{titletoc}
\usepackage{enumitem}

\usepackage{csquotes}

%=======PARAGRAPH FORMATTING=========%

\setlength{\parskip}{1em}   %single space between paragraphs
\setlength{\parindent}{0em}   %single space between paragraphs
\newcommand{\clarify}[1]{{\fontfamily{cmr}\selectfont #1}}
\newcommand{\germvocab}[1]{{\fontfamily{cmr}\selectfont\emph{#1}}}
%=======CHAPTER FORMATTING=========%
\newcommand{\preface}[1]{\chapter*{#1}\addcontentsline{toc}{chapter}{#1}}

\renewcommand\thechapter{\scshape\Roman{chapter}}
\renewcommand{\thesection}{\scshape\arabic{section}.}
\renewcommand{\thesubsection}{\scshape\Alph{subsection}.}

\titleformat
{\chapter} 
[display]
{\centering\scshape\huge} 
{\thechapter} 
{\leftmargin}{\vspace{-1em}}[]

%=======FOOTNOTE FORMATTING=========%
\renewcommand{\thefootnote}{[\fontfamily{ppl}\selectfont\arabic{footnote}]}
\setlength{\skip\footins}{1cm}
\usepackage[]{footmisc}
\renewcommand{\footnotemargin}{3mm} %Setting left margin
\renewcommand{\footnotelayout}{\hspace{2mm}} %spacing between the footnote number and the text of footnote

\usepackage{fancyhdr}

\fancypagestyle{plain}
{
  \renewcommand{\headrulewidth}{0pt}%
  \fancyhf{}%
}
 
\pagestyle{fancy}
\fancyhf{}
\fancyhead[LE,RO]{\thepage}

\title{\HUGE\scshape Manifesto \\ \vspace{5mm}
\huge of the \\ \vspace{4mm}
\HUGE Communist Party}
\author{\Large Karl Marx and Frederick Engels}
\date{\vspace{-4mm}Translated by Samuel Moore, 1882}

\usepackage{hyperref}
\hypersetup{bookmarksnumbered}

%=======TABLE OF CONTENTS FORMATTING=========%
\titlecontents{part}%formatting-toc-sections
    [-1em] 
    {}
    {}
    {\scshape\large}
    {}

\titlecontents{chapter}% formatting-toc-chapters
    [1.5em]% <left-indent>
    {}% <above-code>
    {\scshape \contentslabel{2.3em} }% <numbered-entry-format>
    {\scshape}% <numberless-entry-format>
    {\titlerule*[1pc]{.}\contentspage}% <filler-page-format>

\titlecontents{section}%formatting-toc-sections
    [4em] 
    {}
    {\contentslabel{2.3em}}
    {}
    {\titlerule*[1pc]{.}\contentspage}
    
\titlecontents{subsection}%formatting-toc-sections
    [7em] 
    {}
    {\contentslabel{2.3em}\itshape}
    {}
    {\titlerule*[1pc]{.}\contentspage}


%=======MAIN DOCUMENT=========%
\begin{document}

\begin{titlepage}
    \maketitle
\end{titlepage}

\titlespacing{\chapter}{0mm}{-0.5em}{1.5em}
\renewcommand*\contentsname{Table of Contents}
\tableofcontents


\fancyhead[RE]{\scshape Manifesto of the Communist Party}

\preface{Preface to the 1872 German Edition}
The Communist League, an international association of workers, which could of course be only a secret one, under conditions obtaining at the time, commissioned us, the undersigned, at the Congress held in London in November 1847, to write for publication a detailed theoretical and practical programme for the Party. Such was the origin of the following Manifesto, the manuscript of which travelled to London to be printed a few weeks before the February [\clarify{French}] Revolution [\clarify{in 1848}]. First published in German, it has been republished in that language in at least twelve different editions in Germany, England, and America. It was published in English for the first time in 1850 in the Red Republican, London, translated by Miss Helen Macfarlane, and in 1871 in at least three different translations in America. The French version first appeared in Paris shortly before the June insurrection of 1848, and recently in Le Socialiste of New York. A new translation is in the course of preparation. A Polish version appeared in London shortly after it was first published in Germany. A Russian translation was published in Geneva in the sixties. Into Danish, too, it was translated shortly after its appearance. 

However much that state of things may have altered during the last twenty-five years, the general principles laid down in the Manifesto are, on the whole, as correct today as ever. Here and there, some detail might be improved. The practical application of the principles will depend, as the Manifesto itself states, everywhere and at all times, on the historical conditions for the time being existing, and, for that reason, no special stress is laid on the revolutionary measures proposed at the end of Section II. That passage would, in many respects, be very differently worded today. In view of the gigantic strides of Modern Industry since 1848, and of the accompanying improved and extended organization of the working class, in view of the practical experience gained, first in the February Revolution, and then, still more, in the Paris Commune, where the proletariat for the first time held political power for two whole months, this programme has in some details been antiquated. One thing especially was proved by the Commune, viz., that ``the working class cannot simply lay hold of the ready-made state machinery, and wield it for its own purposes.'' (See The Civil War in France: Address of the General Council of the International Working Men’s Association, 1871, where this point is further developed.) Further, it is self-evident that the criticism of socialist literature is deficient in relation to the present time, because it comes down only to 1847; also that the remarks on the relation of the Communists to the various opposition parties (Section IV), although, in principle still correct, yet in practice are antiquated, because the political situation has been entirely changed, and the progress of history has swept from off the earth the greater portion of the political parties there enumerated. 

But then, the Manifesto has become a historical document which we have no longer any right to alter. A subsequent edition may perhaps appear with an introduction bridging the gap from 1847 to the present day; but this reprint was too unexpected to leave us time for that. 

\hfill
\begin{tabular}{@{}l@{}}
\scshape Karl Marx \& Frederick Engels \\
\emph{24 June 1872, London.}
\end{tabular}

\preface{Preface to the 1882 Russian Edition}
The first Russian edition of the Manifesto of the Communist Party, translated by Bakunin, was published early in the ‘sixties by the printing office of the Kolokol [\clarify{a reference to the Free Russian Printing House}]. Then the West could see in it (the Russian edition of the Manifesto) only a literary curiosity. Such a view would be impossible today. 

What a limited field the proletarian movement occupied at that time (December 1847) is most clearly shown by the last section: the position of the Communists in relation to the various opposition parties in various countries. Precisely Russia and the United States are missing here. It was the time when Russia constituted the last great reserve of all European reaction, when the United States absorbed the surplus proletarian forces of Europe through immigration. Both countries provided Europe with raw materials and were at the same time markets for the sale of its industrial products. Both were, therefore, in one way of another, pillars of the existing European system. 

How very different today. Precisely European immigration fitted North American for a gigantic agricultural production, whose competition is shaking the very foundations of European landed property --- large and small. At the same time, it enabled the United States to exploit its tremendous industrial resources with an energy and on a scale that must shortly break the industrial monopoly of Western Europe, and especially of England, existing up to now. Both circumstances react in a revolutionary manner upon America itself. Step by step, the small and middle land ownership of the farmers, the basis of the whole political constitution, is succumbing to the competition of giant farms; at the same time, a mass industrial proletariat and a fabulous concentration of capital funds are developing for the first time in the industrial regions. 

And now Russia! During the Revolution of 1848-9, not only the European princes, but the European bourgeois as well, found their only salvation from the proletariat just beginning to awaken in Russian intervention. The Tsar was proclaimed the chief of European reaction. Today, he is a prisoner of war of the revolution in Gatchina, and Russia forms the vanguard of revolutionary action in Europe. 

The Communist Manifesto had, as its object, the proclamation of the inevitable impending dissolution of modern bourgeois property. But in Russia we find, face-to-face with the rapidly flowering capitalist swindle and bourgeois property, just beginning to develop, more than half the land owned in common by the peasants. Now the question is: can the Russian obshchina, though greatly undermined, yet a form of primeval common ownership of land, pass directly to the higher form of Communist common ownership? Or, on the contrary, must it first pass through the same process of dissolution such as constitutes the historical evolution of the West? 

The only answer to that possible today is this: If the Russian Revolution becomes the signal for a proletarian revolution in the West, so that both complement each other, the present Russian common ownership of land may serve as the starting point for a communist development. 

\hfill
\begin{tabular}{@{}l@{}}
\scshape Karl Marx \& Frederick Engels  \\
\emph{21 January 1882, London.}
\end{tabular}


\preface{Preface to the 1883 German Edition}
The preface to the present edition I must, alas, sign alone. Marx, the man to whom the whole working class of Europe and America owes more than to any one else --- rests at Highgate Cemetery and over his grave the first grass is already growing. Since his death [\emph{March 14, 1883}], there can be even less thought of revising or supplementing the Manifesto. But I consider it all the more necessary again to state the following expressly: 

The basic thought running through the Manifesto – that economic production, and the structure of society of every historical epoch necessarily arising therefrom, constitute the foundation for the political and intellectual history of that epoch; that consequently (ever since the dissolution of the primaeval communal ownership of land) all history has been a history of class struggles, of struggles between exploited and exploiting, between dominated and dominating classes at various stages of social evolution; that this struggle, however, has now reached a stage where the exploited and oppressed class (the proletariat) can no longer emancipate itself from the class which exploits and oppresses it (the bourgeoisie), without at the same time forever freeing the whole of society from exploitation, oppression, class struggles –-- this basic thought belongs solely and exclusively to Marx.\footnote{``This proposition,'' I wrote in the preface to the English translation, ``which, in my opinion, is destined to do for history what Darwin’ s theory has done for biology, we both of us, had been gradually approaching for some years before 1845. How far I had independently progressed towards it is best shown by my \emph{Conditions of the Working Class in England}. But when I again met Marx at Brussels, in spring 1845, he had it already worked out and put it before me in terms almost as clear as those in which I have stated it here."}

I have already stated this many times; but precisely now is it necessary that it also stand in front of the Manifesto itself. 


\hfill
\begin{tabular}{@{}l@{}}
\scshape Frederick Engels \\
\emph{28 January 1883, London.}
\end{tabular}


\preface{Preface to the 1888 English Edition}
The Manifesto was published as the platform of the Communist League, a working men's association, first exclusively German, later on international, and under the political conditions of the Continent before 1848, unavoidably a secret society. At a Congress of the League, held in November 1847, Marx and Engels were commissioned to prepare a complete theoretical and practical party programme. Drawn up in German, in January 1848, the manuscript was sent to the printer in London a few weeks before the French Revolution of February 24. A French translation was brought out in Paris shortly before the insurrection of June 1848. The first English translation, by Miss Helen Macfarlane, appeared in George Julian Harney's \emph{Red Republican}, London, 1850. A Danish and a Polish edition had also been published. 

The defeat of the Parisian insurrection of June 1848---the first great battle between proletariat and bourgeoisie---drove again into the background, for a time, the social and political aspirations of the European working class. Thenceforth, the struggle for supremacy was, again, as it had been before the Revolution of February, solely between different sections of the propertied class; the working class was reduced to a fight for political elbow-room, and to the position of extreme wing of the middle-class Radicals. Wherever independent proletarian movements continued to show signs of life, they were ruthlessly hunted down. Thus the Prussian police hunted out the Central Board of the Communist League, then located in Cologne. The members were arrested and, after eighteen months'imprisonment, they were tried in October 1852. This celebrated ``Cologne Communist Trial'' lasted from October 4 till November 12; seven of the prisoners were sentenced to terms of imprisonment in a fortress, varying from three to six years. Immediately after the sentence, the League was formally dissolved by the remaining members. As to the Manifesto, it seemed henceforth doomed to oblivion. 

When the European workers had recovered sufficient strength for another attack on the ruling classes, the International Working Men's Association sprang up. But this association, formed with the express aim of welding into one body the whole militant proletariat of Europe and America, could not at once proclaim the principles laid down in the Manifesto. The International was bound to have a programme broad enough to be acceptable to the English trade unions, to the followers of Proudhon in France, Belgium, Italy, and Spain, and to the Lassalleans in Germany.\footnote{Lassalle personally, to us, always acknowledged himself to be a disciple of Marx, and, as such, stood on the ground of the Manifesto. But in his first public agitation, 1862-1864, he did not go beyond demanding co-operative workshops supported by state credit. }

Marx, who drew up this programme to the satisfaction of all parties, entirely trusted to the intellectual development of the working class, which was sure to result from combined action and mutual discussion. The very events and vicissitudes in the struggle against capital, the defeats even more than the victories, could not help bringing home to men's minds the insufficiency of their various favorite nostrums, and preparing the way for a more complete insight into the true conditions for working-class emancipation. And Marx was right. The International, on its breaking in 1874, left the workers quite different men from what it found them in 1864. Proudhonism in France, Lassalleanism in Germany, were dying out, and even the conservative English trade unions, though most of them had long since severed their connection with the International, were gradually advancing towards that point at which, last year at Swansea, their president could say in their name: ``Continental socialism has lost its terror for us.'' In fact, the principles of the Manifesto had made considerable headway among the working men of all countries. 

The Manifesto itself came thus to the front again. Since 1850, the German text had been reprinted several times in Switzerland, England, and America. In 1872, it was translated into English in New York, where the translation was published in \emph{Woorhull and Claflin's Weekly}. From this English version, a French one was made in \emph{Le Socialiste} of New York. Since then, at least two more English translations, more or less mutilated, have been brought out in America, and one of them has been reprinted in England. The first Russian translation, made by Bakunin, was published at Herzen's Kolokol office in Geneva, about 1863; a second one, by the heroic Vera Zasulich, also in Geneva, in 1882. A new Danish edition is to be found in \emph{Socialdemokratisk Bibliothek}, Copenhagen, 1885; a fresh French translation in \emph{Le Socialiste}, Paris, 1886. From this latter, a Spanish version was prepared and published in Madrid, 1886. The German reprints are not to be counted; there have been twelve altogether at the least. An Armenian translation, which was to be published in Constantinople some months ago, did not see the light, I am told, because the publisher was afraid of bringing out a book with the name of Marx on it, while the translator declined to call it his own production. Of further translations into other languages I have heard but had not seen. Thus the history of the Manifesto reflects the history of the modern working-class movement; at present, it is doubtless the most wide spread, the most international production of all socialist literature, the common platform acknowledged by millions of working men from Siberia to California. 

Yet, when it was written, we could not have called it a \emph{socialist} manifesto. By Socialists, in 1847, were understood, on the one hand the adherents of the various Utopian systems: Owenites in England, Fourierists in France, both of them already reduced to the position of mere sects, and gradually dying out; on the other hand, the most multifarious social quacks who, by all manner of tinkering, professed to redress, without any danger to capital and profit, all sorts of social grievances, in both cases men outside the working-class movement, and looking rather to the ``educated'' classes for support. Whatever portion of the working class had become convinced of the insufficiency of mere political revolutions, and had proclaimed the necessity of total social change, called itself Communist. It was a crude, rough-hewn, purely instinctive sort of communism; still, it touched the cardinal point and was powerful enough amongst the working class to produce the Utopian communism of Cabet in France, and of Weitling in Germany. Thus, in 1847, socialism was a middle-class movement, communism a working-class movement. Socialism was, on the Continent at least, ``respectable''; communism was the very opposite. And as our notion, from the very beginning, was that ``the emancipation of the workers must be the act of the working class itself,'' there could be no doubt as to which of the two names we must take. Moreover, we have, ever since, been far from repudiating it. 

The Manifesto being our joint production, I consider myself bound to state that the fundamental proposition which forms the nucleus belongs to Marx. That proposition is: That in every historical epoch, the prevailing mode of economic production and exchange, and the social organization necessarily following from it, form the basis upon which it is built up, and from that which alone can be explained the political and intellectual history of that epoch; that consequently the whole history of mankind (since the dissolution of primitive tribal society, holding land in common ownership) has been a history of class struggles, contests between exploiting and exploited, ruling and oppressed classes; That the history of these class struggles forms a series of evolutions in which, nowadays, a stage has been reached where the exploited and oppressed class---the proletariat---cannot attain its emancipation from the sway of the exploiting and ruling class---the bourgeoisie---without, at the same time, and once and for all, emancipating society at large from all exploitation, oppression, class distinction, and class struggles. 

This proposition, which, in my opinion, is destined to do for history what Darwin's theory has done for biology, we both of us, had been gradually approaching for some years before 1845. How far I had independently progressed towards it is best shown by my ``Conditions of the Working Class in England.''\footnote{\emph{The Condition of the Working Class in England in 1844}. By Frederick Engels. Translated by Florence K. Wischnewetzky—London, Swan, Sonnenschein \& Co.} But when I again met Marx at Brussels, in spring 1845, he had it already worked out and put it before me in terms almost as clear as those in which I have stated it here. 

From our joint preface to the German edition of 1872, I quote the following: 
\begin{displayquote}
``However much that state of things may have altered during the last twenty-five years, the general principles laid down in the Manifesto are, on the whole, as correct today as ever. Here and there, some detail might be improved. The practical application of the principles will depend, as the Manifesto itself states, everywhere and at all times, on the historical conditions for the time being existing, and, for that reason, no special stress is laid on the revolutionary measures proposed at the end of Section II. That passage would, in many respects, be very differently worded today. In view of the gigantic strides of Modern Industry since 1848, and of the accompanying improved and extended organization of the working class, in view of the practical experience gained, first in the February Revolution, and then, still more, in the Paris Commune, where the proletariat for the first time held political power for two whole months, this programme has in some details been antiquated. One thing especially was proved by the Commune, viz., that ``the working class cannot simply lay hold of ready-made state machinery, and wield it for its own purposes.'' (See The Civil War in France: Address of the General Council of the International Working Men's Association 1871, where this point is further developed.) Further, it is self-evident that the criticism of socialist literature is deficient in relation to the present time, because it comes down only to 1847; also that the remarks on the relation of the Communists to the various opposition parties (Section IV), although, in principle still correct, yet in practice are antiquated, because the political situation has been entirely changed, and the progress of history has swept from off the Earth the greater portion of the political parties there enumerated. 

``But then, the Manifesto has become a historical document which we have no longer any right to alter.'' 
\end{displayquote}
The present translation is by Mr Samuel Moore, the translator of the greater portion of Marx's ``Capital.'' We have revised it in common, and I have added a few notes explanatory of historical allusions. 

\hfill
\begin{tabular}{@{}l@{}}
\scshape Frederick Engels \\
\emph{30 January 1888, London.}
\end{tabular}


\preface{Preface to the 1890 German Edition}
Since [\clarify{the first German preface of 1883}] was written, a new German edition of the Manifesto has again become necessary, and much has also happened to the Manifesto which should be recorded here. 

A second Russian translation --- by Vera Zasulich --- appeared in Geneva in 1882; the preface to that edition was written by Marx and myself. Unfortunately, the original German manuscript has gone astray; I must therefore retranslate from the Russian which will in no way improve the text. It reads: 
\begin{displayquote}
	[\clarify{Reprint of the 1882 Russian Edition}]
\end{displayquote}
At about the same date, a new Polish version appeared in Geneva: \emph{Manifest Kommunistyczny}. 

Furthermore, a new Danish translation has appeared in the \emph{Socialdemokratisk Bibliothek}, Copenhagen, 1885. Unfortunately, it is not quite complete; certain essential passages, which seem to have presented difficulties to the translator, have been omitted, and, in addition, there are signs of carelessness here and there, which are all the more unpleasantly conspicuous since the translation indicates that had the translator taken a little more pains, he would have done an excellent piece of work. 

A new French version appeared in 1886, in \emph{Le Socialiste} of Paris; it is the best published to date. 

From this latter, a Spanish version was published the same year in \emph{El Socialista} of Madrid, and then reissued in pamphlet form: \emph{Manifesto del Partido Communista} por Carlos Marx y F. Engels, Madrid, Administracion de El Socialista, Hernan Cortes 8. 

As a matter of curiosity, I may mention that in 1887 the manuscript of an Armenian translation was offered to a publisher in Constantinople. But the good man did not have the courage to publish something bearing the name of Marx and suggested that the translator set down his own name as author, which the latter however declined. 

After one, and then another, of the more or less inaccurate American translations had been repeatedly reprinted in England, an authentic version at last appeared in 1888. This was my friend Samuel Moore, and we went through it together once more before it went to press. It is entitled: \emph{Manifesto of the Communist Party}, by Karl Marx and Frederick Engels. Authorized English translation, edited and annotated by Frederick Engels, 1888, London, William Reeves, 185 Fleet Street, E.C. I have added some of the notes of that edition to the present one. 

The Manifesto has had a history of its own. Greeted with enthusiasm, at the time of its appearance, by the not at all numerous vanguard of scientific socialism (as is proved by the translations mentioned in the first place), it was soon forced into the background by the reaction that began with the defeat of the Paris workers in June 1848, and was finally excommunicated ``by law'' in the conviction of the Cologne Communists in November 1852. With the disappearance from the public scene of the workers’ movement that had begun with the February Revolution, the Manifesto too passed into the background. 

When the European workers had again gathered sufficient strength for a new onslaught upon the power of the ruling classes, the International Working Men’ s Association came into being. Its aim was to weld together into \emph{one} huge army the whole militant working class of Europe and America. Therefore it could not \emph{set out} from the principles laid down in the Manifesto. It was bound to have a programme which would not shut the door on the English trade unions, the French, Belgian, Italian, and Spanish Proudhonists, and the German Lassalleans. This programme --- the considerations underlying the Statutes of the International --- was drawn up by Marx with a master hand acknowledged even by the Bakunin and the anarchists. For the ultimate final triumph of the ideas set forth in the Manifesto, Marx relied solely upon the intellectual development of the working class, as it necessarily has to ensue from united action and discussion. The events and vicissitudes in the struggle against capital, the defeats even more than the successes, could not but demonstrate to the fighters the inadequacy of their former universal panaceas, and make their minds more receptive to a thorough understanding of the true conditions for working-class emancipation. And Marx was right. The working class of 1874, at the dissolution of the International, was altogether different from that of 1864, at its foundation. Proudhonism in the Latin countries, and the specific Lassalleanism in Germany, were dying out; and even the ten arch-conservative English trade unions were gradually approaching the point where, in 1887, the chairman of their Swansea Congress could say in their name: ``Continental socialism has lost its terror for us.'' Yet by 1887 continental socialism was almost exclusively the theory heralded in the Manifesto. Thus, to a certain extent, the history of the Manifesto reflects the history of the modern working-class movement since 1848. At present, it is doubtless the most widely circulated, the most international product of all socialist literature, the common programme of many millions of workers of all countries from Siberia to California. 

Nevertheless, when it appeared, we could not have called it a \emph{socialist} manifesto. In 1847, two kinds of people were considered socialists. On the one hand were the adherents of the various utopian systems, notably the Owenites in England and the Fourierists in France, both of whom, at that date, had already dwindled to mere sects gradually dying out. On the other, the manifold types of social quacks who wanted to eliminate social abuses through their various universal panaceas and all kinds of patch-work, without hurting capital and profit in the least. In both cases, people who stood outside the labor movement and who looked for support rather to the ``educated'' classes. The section of the working class, however, which demanded a radical reconstruction of society, convinced that mere political revolutions were not enough, then called itself Communist. It was still a rough-hewn, only instinctive and frequently somewhat crude communism. Yet, it was powerful enough to bring into being two systems of utopian communism – in France, the ``Icarian'' communists of Cabet, and in Germany that of Weitling. Socialism in 1847 signified a bourgeois movement, communism a working-class movement. Socialism was, on the Continent at least, quite respectable, whereas communism was the very opposite. And since we were very decidedly of the opinion as early as then that ``the emancipation of the workers must be the task of the working class itself,'' [\clarify{from the General Rules of the International}] we could have no hesitation as to which of the two names we should choose. Nor has it ever occurred to us to repudiate it. 

``Working men of all countries, unite!'' But few voices responded when we proclaimed these words to the world 42 years ago, on the eve of the first Paris Revolution in which the proletariat came out with the demands of its own. On September 28, 1864, however, the proletarians of most of the Western European countries joined hands in the International Working Men’s Association of glorious memory. True, the International itself lived only nine years. But that the eternal union of the proletarians of all countries created by it is still alive and lives stronger than ever, there is no better witness than this day. Because today, as I write these lines, the European and American proletariat is reviewing its fighting forces, mobilized for the first time, mobilized as \emph{one} army, under \emph{one} flag, for \emph{one} immediate aim: the standard eight-hour working day to be established by legal enactment, as proclaimed by the Geneva Congress of the International in 1866, and again by the Paris Workers’ Congress of 1889. And today’s spectacle will open the eyes of the capitalists and landlords of all countries to the fact that today the proletarians of all countries are united indeed. 

If only Marx were still by my side to see this with his own eyes! 


\hfill
\begin{tabular}{@{}l@{}}
\scshape Frederick Engels \\
\emph{1 May 1890, London.}
\end{tabular}


\preface{Preface to the 1892 Polish Edition}
The fact that a new Polish edition of the Communist Manifesto has become necessary gives rise to various thoughts.

First of all, it is noteworthy that of late the Manifesto has become an index, as it were, of the development of large-scale industry on the European continent. In proportion as large-scale industry expands in a given country, the demand grows among the workers of that country for enlightenment regarding their position as the working class in relation to the possessing classes, the socialist movement spreads among them and the demand for the Manifesto increases. Thus, not only the state of the labour movement but also the degree of development of large-scale industry can be measured with fair accuracy in every country by the number of copies of the Manifesto circulated in the language of that country.

Accordingly, the new Polish edition indicates a decided progress of Polish industry. And there can be no doubt whatever that this progress since the previous edition published ten years ago has actually taken place. Russian Poland, Congress Poland, has become the big industrial region of the Russian Empire. Whereas Russian large-scale industry is scattered sporadically --- a part round the Gulf of Finland, another in the centre (Moscow and Vladimir), a third along the coasts of the Black and Azov seas, and still others elsewhere --- Polish industry has been packed into a relatively small area and enjoys both the advantages and disadvantages arising from such concentration. The competing Russian manufacturers acknowledged the advantages when they demanded protective tariffs against Poland, in spit of their ardent desire to transform the Poles into Russians. The disadvantages --- for the Polish manufacturers and the Russian government --- are manifest in the rapid spread of socialist ideas among the Polish workers and in the growing demand for the Manifesto.

But the rapid development of Polish industry, outstripping that of Russia, is in its turn a new proof of the inexhaustible vitality of the Polish people and a new guarantee of its impending national restoration. And the restoration of an independent and strong Poland is a matter which concerns not only the Poles but all of us. A sincere international collaboration of the European nations is possible only if each of these nations is fully autonomous in its own house. The Revolution of 1848, which under the banner of the proletariat, after all, merely let the proletarian fighters do the work of the bourgeoisie, also secured the independence of Italy, Germany and Hungary through its testamentary executors, Louis Bonaparte and Bismarck; but Poland, which since 1792 had done more for the Revolution than all these three together, was left to its own resources when it succumbed in 1863 to a tenfold greater Russian force. The nobility could neither maintain nor regain Polish independence; today, to the bourgeoisie, this independence is, to say the last, immaterial. Nevertheless, it is a necessity for the harmonious collaboration of the European nations. It can be gained only by the young Polish proletariat, and in its hands it is secure. For the workers of all the rest of Europe need the independence of Poland just as much as the Polish workers themselves.

\hfill
\begin{tabular}{@{}l@{}}
\scshape Frederick Engels \\
\emph{10 January 1892, London.}
\end{tabular}


\preface{Preface to the 1893 Italian Edition}
Publication of the Manifesto of the Communist Party coincided, one may say, with March 18, 1848, the day of the revolution in Milan and Berlin, which were armed uprisings of the two nations situated in the centre, the one, of the continent of Europe, the other, of the Mediterranean; two nations until then enfeebled by division and internal strife, and thus fallen under foreign domination. While Italy was subject to the Emperor of Austria, Germany underwent the yoke, not less effective though more indirect, of the Tsar of all the Russias. The consequences of March 18, 1848, freed both Italy and Germany from this disgrace; if from 1848 to 1871 these two great nations were reconstituted and somehow again put on their own, it was as Karl Marx used to say, because the men who suppressed the Revolution of 1848 were, nevertheless, its testamentary executors in spite of themselves.

Everywhere that revolution was the work of the working class; it was the latter that built the barricades and paid with its lifeblood. Only the Paris workers, in overthrowing the government, had the very definite intention of overthrowing the bourgeois regime. But conscious though they were of the fatal antagonism existing between their own class and the bourgeoisie, still, neither the economic progress of the country nor the intellectual development of the mass of French workers had as yet reached the stage which would have made a social reconstruction possible. In the final analysis, therefore, the fruits of the revolution were reaped by the capitalist class. In the other countries, in Italy, in Germany, in Austria, the workers, from the very outset, did nothing but raise the bourgeoisie to power. But in any country the rule of the bourgeoisie is impossible without national independence Therefore, the Revolution of 1848 had to bring in its train the unity and autonomy of the nations that had lacked them up to then: Italy, Germany, Hungary. Poland will follow in turn.

Thus, if the Revolution of 1848 was not a socialist revolution, it paved the way, prepared the ground for the latter. Through the impetus given to large-scaled industry in all countries, the bourgeois regime during the last forty-five years has everywhere created a numerous, concentrated and powerful proletariat. It has thus raised, to use the language of the Manifesto, its own grave-diggers. Without restoring autonomy and unity to each nation, it will be impossible to achieve the international union of the proletariat, or the peaceful and intelligent co-operation of these nations toward common aims. Just imagine joint international action by the Italian, Hungarian, German, Polish and Russian workers under the political conditions preceding 1848!

The battles fought in 1848 were thus not fought in vain. Nor have the forty-five years separating us from that revolutionary epoch passed to no purpose. The fruits are ripening, and all I wish is that the publication of this Italian translation may augur as well for the victory of the Italian proletariat as the publication of the original did for the international revolution.

The Manifesto does full justice to the revolutionary part played by capitalism in the past. The first capitalist nation was Italy. The close of the feudal Middle Ages, and the opening of the modern capitalist era are marked by a colossal figured: an Italian, Dante, both the last poet of the Middle Ages and the first poet \clearpage of modern times. Today, as in 1300, a new historical era is approaching. Will Italy give us the new Dante, who will mark the hour of birth of this new, proletarian era?

\hfill
\begin{tabular}{@{}l@{}}
\scshape Frederick Engels \\
\emph{1 February 1893, London.}
\end{tabular}



\chapter*{Manifesto of the Communist Party}
\addcontentsline{toc}{part}{Manifesto of the Communist Party}
A spectre is haunting Europe—the spectre of Communism. All the powers of old Europe have entered into a holy alliance to exorcise this spectre; Pope and Czar, Metternich and Guizot, French Radicals and German police-spies.

Where is the party in opposition that has not been decried as communistic by its opponents in power? Where the Opposition that has not hurled back the branding reproach of Communism, against the more advanced opposition parties, as well as against its reactionary adversaries?

Two things result from this fact.
\begin{enumerate}[label={\Roman*}.]
	\item Communism is already acknowledged by all European Powers to be itself a Power.
	\item It is high time that Communists should openly, in the face of the whole world, publish their views, their aims, their tendencies, and meet this nursery tale of the Spectre of Communism with a Manifesto of the party itself.
\end{enumerate}
To this end, Communists of various nationalities have assembled in London, and sketched the following manifesto, to be published in the English, French, German, Italian, Flemish and Danish languages.

\titlespacing{\chapter}{0mm}{-4em}{1.5em}
\titlespacing{\section}{0mm}{3mm}{2mm}

\titlespacing{\subsection}{0mm}{3mm}{2mm}
\chapter{Bourgeois and Proletarians}
\fancyhead[LO]{\scshape I. Bourgeois and Proletarians}
\input{manifesto1.tex}
\chapter{Proletarians and Communists}
\fancyhead[LO]{\scshape II. Proletarians and Communists}
\input{manifesto2.tex}
\chapter{Socialist and Communist Literature}
\fancyhead[LO]{\scshape III. Socialist and Communist Literature}
\input{manifesto3.tex}

\titleformat
{\chapter} 
[display]
{\centering\scshape\LARGE} 
{\thechapter} 
{\leftmargin}{\vspace{-1em}}[]

\chapter{Position of the Communists in Relation to the Various Existing Opposition Parties}
\fancyhead[LO]{\scshape IV. Position of the Communists}
\input{manifesto4.tex}

\end{document}