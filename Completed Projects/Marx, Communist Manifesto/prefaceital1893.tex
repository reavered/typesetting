Publication of the Manifesto of the Communist Party coincided, one may say, with March 18, 1848, the day of the revolution in Milan and Berlin, which were armed uprisings of the two nations situated in the centre, the one, of the continent of Europe, the other, of the Mediterranean; two nations until then enfeebled by division and internal strife, and thus fallen under foreign domination. While Italy was subject to the Emperor of Austria, Germany underwent the yoke, not less effective though more indirect, of the Tsar of all the Russias. The consequences of March 18, 1848, freed both Italy and Germany from this disgrace; if from 1848 to 1871 these two great nations were reconstituted and somehow again put on their own, it was as Karl Marx used to say, because the men who suppressed the Revolution of 1848 were, nevertheless, its testamentary executors in spite of themselves.

Everywhere that revolution was the work of the working class; it was the latter that built the barricades and paid with its lifeblood. Only the Paris workers, in overthrowing the government, had the very definite intention of overthrowing the bourgeois regime. But conscious though they were of the fatal antagonism existing between their own class and the bourgeoisie, still, neither the economic progress of the country nor the intellectual development of the mass of French workers had as yet reached the stage which would have made a social reconstruction possible. In the final analysis, therefore, the fruits of the revolution were reaped by the capitalist class. In the other countries, in Italy, in Germany, in Austria, the workers, from the very outset, did nothing but raise the bourgeoisie to power. But in any country the rule of the bourgeoisie is impossible without national independence Therefore, the Revolution of 1848 had to bring in its train the unity and autonomy of the nations that had lacked them up to then: Italy, Germany, Hungary. Poland will follow in turn.

Thus, if the Revolution of 1848 was not a socialist revolution, it paved the way, prepared the ground for the latter. Through the impetus given to large-scaled industry in all countries, the bourgeois regime during the last forty-five years has everywhere created a numerous, concentrated and powerful proletariat. It has thus raised, to use the language of the Manifesto, its own grave-diggers. Without restoring autonomy and unity to each nation, it will be impossible to achieve the international union of the proletariat, or the peaceful and intelligent co-operation of these nations toward common aims. Just imagine joint international action by the Italian, Hungarian, German, Polish and Russian workers under the political conditions preceding 1848!

The battles fought in 1848 were thus not fought in vain. Nor have the forty-five years separating us from that revolutionary epoch passed to no purpose. The fruits are ripening, and all I wish is that the publication of this Italian translation may augur as well for the victory of the Italian proletariat as the publication of the original did for the international revolution.

The Manifesto does full justice to the revolutionary part played by capitalism in the past. The first capitalist nation was Italy. The close of the feudal Middle Ages, and the opening of the modern capitalist era are marked by a colossal figured: an Italian, Dante, both the last poet of the Middle Ages and the first poet \clearpage of modern times. Today, as in 1300, a new historical era is approaching. Will Italy give us the new Dante, who will mark the hour of birth of this new, proletarian era?

\hfill
\begin{tabular}{@{}l@{}}
\scshape Frederick Engels \\
\emph{1 February 1893, London.}
\end{tabular}
