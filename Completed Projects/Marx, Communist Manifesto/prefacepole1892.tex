The fact that a new Polish edition of the Communist Manifesto has become necessary gives rise to various thoughts.

First of all, it is noteworthy that of late the Manifesto has become an index, as it were, of the development of large-scale industry on the European continent. In proportion as large-scale industry expands in a given country, the demand grows among the workers of that country for enlightenment regarding their position as the working class in relation to the possessing classes, the socialist movement spreads among them and the demand for the Manifesto increases. Thus, not only the state of the labour movement but also the degree of development of large-scale industry can be measured with fair accuracy in every country by the number of copies of the Manifesto circulated in the language of that country.

Accordingly, the new Polish edition indicates a decided progress of Polish industry. And there can be no doubt whatever that this progress since the previous edition published ten years ago has actually taken place. Russian Poland, Congress Poland, has become the big industrial region of the Russian Empire. Whereas Russian large-scale industry is scattered sporadically --- a part round the Gulf of Finland, another in the centre (Moscow and Vladimir), a third along the coasts of the Black and Azov seas, and still others elsewhere --- Polish industry has been packed into a relatively small area and enjoys both the advantages and disadvantages arising from such concentration. The competing Russian manufacturers acknowledged the advantages when they demanded protective tariffs against Poland, in spit of their ardent desire to transform the Poles into Russians. The disadvantages --- for the Polish manufacturers and the Russian government --- are manifest in the rapid spread of socialist ideas among the Polish workers and in the growing demand for the Manifesto.

But the rapid development of Polish industry, outstripping that of Russia, is in its turn a new proof of the inexhaustible vitality of the Polish people and a new guarantee of its impending national restoration. And the restoration of an independent and strong Poland is a matter which concerns not only the Poles but all of us. A sincere international collaboration of the European nations is possible only if each of these nations is fully autonomous in its own house. The Revolution of 1848, which under the banner of the proletariat, after all, merely let the proletarian fighters do the work of the bourgeoisie, also secured the independence of Italy, Germany and Hungary through its testamentary executors, Louis Bonaparte and Bismarck; but Poland, which since 1792 had done more for the Revolution than all these three together, was left to its own resources when it succumbed in 1863 to a tenfold greater Russian force. The nobility could neither maintain nor regain Polish independence; today, to the bourgeoisie, this independence is, to say the last, immaterial. Nevertheless, it is a necessity for the harmonious collaboration of the European nations. It can be gained only by the young Polish proletariat, and in its hands it is secure. For the workers of all the rest of Europe need the independence of Poland just as much as the Polish workers themselves.

\hfill
\begin{tabular}{@{}l@{}}
\scshape Frederick Engels \\
\emph{10 January 1892, London.}
\end{tabular}
