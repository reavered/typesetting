Since [\clarify{the first German preface of 1883}] was written, a new German edition of the Manifesto has again become necessary, and much has also happened to the Manifesto which should be recorded here. 

A second Russian translation --- by Vera Zasulich --- appeared in Geneva in 1882; the preface to that edition was written by Marx and myself. Unfortunately, the original German manuscript has gone astray; I must therefore retranslate from the Russian which will in no way improve the text. It reads: 
\begin{displayquote}
	[\clarify{Reprint of the 1882 Russian Edition}]
\end{displayquote}
At about the same date, a new Polish version appeared in Geneva: \emph{Manifest Kommunistyczny}. 

Furthermore, a new Danish translation has appeared in the \emph{Socialdemokratisk Bibliothek}, Copenhagen, 1885. Unfortunately, it is not quite complete; certain essential passages, which seem to have presented difficulties to the translator, have been omitted, and, in addition, there are signs of carelessness here and there, which are all the more unpleasantly conspicuous since the translation indicates that had the translator taken a little more pains, he would have done an excellent piece of work. 

A new French version appeared in 1886, in \emph{Le Socialiste} of Paris; it is the best published to date. 

From this latter, a Spanish version was published the same year in \emph{El Socialista} of Madrid, and then reissued in pamphlet form: \emph{Manifesto del Partido Communista} por Carlos Marx y F. Engels, Madrid, Administracion de El Socialista, Hernan Cortes 8. 

As a matter of curiosity, I may mention that in 1887 the manuscript of an Armenian translation was offered to a publisher in Constantinople. But the good man did not have the courage to publish something bearing the name of Marx and suggested that the translator set down his own name as author, which the latter however declined. 

After one, and then another, of the more or less inaccurate American translations had been repeatedly reprinted in England, an authentic version at last appeared in 1888. This was my friend Samuel Moore, and we went through it together once more before it went to press. It is entitled: \emph{Manifesto of the Communist Party}, by Karl Marx and Frederick Engels. Authorized English translation, edited and annotated by Frederick Engels, 1888, London, William Reeves, 185 Fleet Street, E.C. I have added some of the notes of that edition to the present one. 

The Manifesto has had a history of its own. Greeted with enthusiasm, at the time of its appearance, by the not at all numerous vanguard of scientific socialism (as is proved by the translations mentioned in the first place), it was soon forced into the background by the reaction that began with the defeat of the Paris workers in June 1848, and was finally excommunicated ``by law'' in the conviction of the Cologne Communists in November 1852. With the disappearance from the public scene of the workers’ movement that had begun with the February Revolution, the Manifesto too passed into the background. 

When the European workers had again gathered sufficient strength for a new onslaught upon the power of the ruling classes, the International Working Men’ s Association came into being. Its aim was to weld together into \emph{one} huge army the whole militant working class of Europe and America. Therefore it could not \emph{set out} from the principles laid down in the Manifesto. It was bound to have a programme which would not shut the door on the English trade unions, the French, Belgian, Italian, and Spanish Proudhonists, and the German Lassalleans. This programme --- the considerations underlying the Statutes of the International --- was drawn up by Marx with a master hand acknowledged even by the Bakunin and the anarchists. For the ultimate final triumph of the ideas set forth in the Manifesto, Marx relied solely upon the intellectual development of the working class, as it necessarily has to ensue from united action and discussion. The events and vicissitudes in the struggle against capital, the defeats even more than the successes, could not but demonstrate to the fighters the inadequacy of their former universal panaceas, and make their minds more receptive to a thorough understanding of the true conditions for working-class emancipation. And Marx was right. The working class of 1874, at the dissolution of the International, was altogether different from that of 1864, at its foundation. Proudhonism in the Latin countries, and the specific Lassalleanism in Germany, were dying out; and even the ten arch-conservative English trade unions were gradually approaching the point where, in 1887, the chairman of their Swansea Congress could say in their name: ``Continental socialism has lost its terror for us.'' Yet by 1887 continental socialism was almost exclusively the theory heralded in the Manifesto. Thus, to a certain extent, the history of the Manifesto reflects the history of the modern working-class movement since 1848. At present, it is doubtless the most widely circulated, the most international product of all socialist literature, the common programme of many millions of workers of all countries from Siberia to California. 

Nevertheless, when it appeared, we could not have called it a \emph{socialist} manifesto. In 1847, two kinds of people were considered socialists. On the one hand were the adherents of the various utopian systems, notably the Owenites in England and the Fourierists in France, both of whom, at that date, had already dwindled to mere sects gradually dying out. On the other, the manifold types of social quacks who wanted to eliminate social abuses through their various universal panaceas and all kinds of patch-work, without hurting capital and profit in the least. In both cases, people who stood outside the labor movement and who looked for support rather to the ``educated'' classes. The section of the working class, however, which demanded a radical reconstruction of society, convinced that mere political revolutions were not enough, then called itself Communist. It was still a rough-hewn, only instinctive and frequently somewhat crude communism. Yet, it was powerful enough to bring into being two systems of utopian communism – in France, the ``Icarian'' communists of Cabet, and in Germany that of Weitling. Socialism in 1847 signified a bourgeois movement, communism a working-class movement. Socialism was, on the Continent at least, quite respectable, whereas communism was the very opposite. And since we were very decidedly of the opinion as early as then that ``the emancipation of the workers must be the task of the working class itself,'' [\clarify{from the General Rules of the International}] we could have no hesitation as to which of the two names we should choose. Nor has it ever occurred to us to repudiate it. 

``Working men of all countries, unite!'' But few voices responded when we proclaimed these words to the world 42 years ago, on the eve of the first Paris Revolution in which the proletariat came out with the demands of its own. On September 28, 1864, however, the proletarians of most of the Western European countries joined hands in the International Working Men’s Association of glorious memory. True, the International itself lived only nine years. But that the eternal union of the proletarians of all countries created by it is still alive and lives stronger than ever, there is no better witness than this day. Because today, as I write these lines, the European and American proletariat is reviewing its fighting forces, mobilized for the first time, mobilized as \emph{one} army, under \emph{one} flag, for \emph{one} immediate aim: the standard eight-hour working day to be established by legal enactment, as proclaimed by the Geneva Congress of the International in 1866, and again by the Paris Workers’ Congress of 1889. And today’s spectacle will open the eyes of the capitalists and landlords of all countries to the fact that today the proletarians of all countries are united indeed. 

If only Marx were still by my side to see this with his own eyes! 


\hfill
\begin{tabular}{@{}l@{}}
\scshape Frederick Engels \\
\emph{1 May 1890, London.}
\end{tabular}
