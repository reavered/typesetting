The preface to the present edition I must, alas, sign alone. Marx, the man to whom the whole working class of Europe and America owes more than to any one else --- rests at Highgate Cemetery and over his grave the first grass is already growing. Since his death [\emph{March 14, 1883}], there can be even less thought of revising or supplementing the Manifesto. But I consider it all the more necessary again to state the following expressly: 

The basic thought running through the Manifesto – that economic production, and the structure of society of every historical epoch necessarily arising therefrom, constitute the foundation for the political and intellectual history of that epoch; that consequently (ever since the dissolution of the primaeval communal ownership of land) all history has been a history of class struggles, of struggles between exploited and exploiting, between dominated and dominating classes at various stages of social evolution; that this struggle, however, has now reached a stage where the exploited and oppressed class (the proletariat) can no longer emancipate itself from the class which exploits and oppresses it (the bourgeoisie), without at the same time forever freeing the whole of society from exploitation, oppression, class struggles –-- this basic thought belongs solely and exclusively to Marx.\footnote{``This proposition,'' I wrote in the preface to the English translation, ``which, in my opinion, is destined to do for history what Darwin’ s theory has done for biology, we both of us, had been gradually approaching for some years before 1845. How far I had independently progressed towards it is best shown by my \emph{Conditions of the Working Class in England}. But when I again met Marx at Brussels, in spring 1845, he had it already worked out and put it before me in terms almost as clear as those in which I have stated it here."}

I have already stated this many times; but precisely now is it necessary that it also stand in front of the Manifesto itself. 


\hfill
\begin{tabular}{@{}l@{}}
\scshape Frederick Engels \\
\emph{28 January 1883, London.}
\end{tabular}
