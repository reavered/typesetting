To the average reader the multiplicity of Russian organisations-political groups, Committees and Central Committees, Soviets, Dumas and Unions-will prove extremely confusing. For this reason I am giving here a few brief definitions and explanations.

\section*{Political Parties}

In the elections to the Constituent Assembly, there were seventeen tickets in Petrograd, and in some of the provincial towns as many as forty; but the following summary of the aims and composition of political parties is limited to the groups and factions mentioned in this book. Only the essence of their programmes and the general character of their constituencies can be noticed....
\begin{enumerate}
\item Monarchists, of various shades, Octobrists, etc. These once-powerful factions no longer existed openly; they either worked underground, or their members joined the Cadets, as the Cadets came by degrees to stand for their political programme. Representatives in this book, Rodzianko, Shulgin.

\item Cadets. So-called from the initials of its name, Constitutional Democrats. Its official name is "Party of the People's Freedom." Under the Tsar composed of Liberals from the propertied classes, the Cadets were the great party of political reform, roughly corresponding to the Progressive Party in America. When the Revolution broke out in March, 1917, the Cadets formed the first Provisional Government. The Cadet Ministry was overthrown in April because it declared itself in favour of Allied imperialistic aims, including the imperialistic aims of the Tsar's Government. As the Revolution became more and more a social economic Revolution, the Cadets grew more and more conservative. Its representatives in this book are: Miliukov, Vinaver, Shatsky.

2a. Group of Public Men. After the Cadets had become unpopular through their relations with the Kornilov counter-revolution, the Group of Public Men was formed in Moscow. Delegates from the Group of Public Men were given portfolios in the last Kerensky Cabinet. The Group declared itself non-partisan, although its intellectual leaders were men like Rodzianko and Shulgin. It was composed of the more "modern" bankers, merchants and manufacturers, who were intelligent enough to realise that the Soviets must be fought by their own weapon-economic organisation. Typical of the Group: Lianozov, Konovalov.

\item Populist Socialists, or Trudoviki (Labour Group). Numerically a small party, composed of cautious intellectuals, the leaders of the Cooperative societies, and conservative peasants. Professing to be Socialists, the Populists really supported the interests of the petty bourgeoisie-clerks, shopkeepers, etc. By direct descent, inheritors of the compromising tradition of the Labour Group in the Fourth Imperial Duma, which was composed largely of peasant representatives. Kerensky was the leader of the Trudoviki in the Imperial Duma when the Revolution of March, 1917, broke out. The Populist Socialists are a nationalistic party. Their representatives in this book are: Peshekhanov, Tchaikovsky.

\item Russian Social Democratic Labour Party. Originally Marxian Socialists. At a party congress held in 1903, the party split, on the question of tactics, into two factions-the Majority (Bolshinstvo), and the Minority (Menshinstvo). From this sprang the names "Bolsheviki" and "Mensheviki"-"members of the majority" and "members of the minority." These two wings became two separate parties, both calling themselves "Russian Social Democratic Labour Party," and both professing to be Marxians. Since the Revolution of 1905 the Bolsheviki were really the minority, becoming again the majority in September, 1917.
\begin{enumerate}
\item Mensheviki. This party includes all shades of Socialists who believe that society must progress by natural evolution toward Socialism, and that the working-class must conquer political power first. Also a nationalistic party. This was the party of the Socialist intellectuals, which means: all the means of education having been in the hands of the propertied classes, the intellectuals instinctively reacted to their training, and took the side of the propertied classes. Among their representatives in this book are: Dan, Lieber, Tseretelli.

\item Mensheviki Internationalists. The radical wing of the Mensheviki, internationalists and opposed to all coalition with the propertied classes; yet unwilling to break loose from the conservative Mensheviki, and opposed to the dictatorship of the working-class advocated by the Bolsheviki. Trotzky was long a member of this group. Among their leaders: Martov, Martinov.

\item Bolsheviki. Now call themselves the Communist Party, in order to emphasise their complete separation from the tradition of "moderate" or "parliamentary" Socialism, which dominates the Mensheviki and the so-called Majority Socialists in all countries. The Bolsheviki proposed immediate proletarian insurrection, and seizure of the reins of Government, in order to hasten the coming of Socialism by forcibly taking over industry, land, natural resources and financial institutions. This party expresses the desires chiefly of the factory workers, but also of a large section of the poor peasants. The name "Bolshevik" can not be translated by "Maximalist." The Maximalists are a separate group. (See paragraph 5b). Among the leaders: Lenin, Trotzky, Lunatcharsky.

\item United Social Democrats Internationalists. Also called the Novaya Zhizn (New Life) group, from the name of the very influential newspaper which was its organ. A little group of intellectuals with a very small following among the working-class, except the personal following of Maxim Gorky, its leader. Intellectuals, with almost the same programme as the Mensheviki Internationalists, except that the Novaya Zhizn group refused to be tied to either of the two great factions. Opposed the Bolshevik tactics, but remained in the Soviet Government. Other representatives in this book: Avilov, Kramarov.

\item Yedinstvo. A very small and dwindling group, composed almost entirely of the personal following of Plekhanov, one of the pioneers of the Russian Social Democratic movement in the 80's, and its greatest theoretician. Now an old man, Plekhanov was extremely patriotic, too conservative even for the Mensheviki. After the Bolshevik coup d'etat, Yedinstvo disappeared.

\item Socialist Revolutionary party. Called Essaires from the initials of their name. Originally the revolutionary party of the peasants, the party of the Fighting Organisations-the Terrorists. After the March Revolution, it was joined by many who had never been Socialists. At that time it stood for the abolition of private property in land only, the owners to be compensated in some fashion. Finally the increasing revolutionary feeling of peasants forced the Essaires to abandon the "compensation" clause, and led to the younger and more fiery intellectuals breaking off from the main party in the fall of 1917 and forming a new party, the Left Socialist Revolutionary party. The Essaires, who were afterward always called by the radical groups "Right Socialist Revolutionaries," adopted the political attitude of the Mensheviki, and worked together with them. They finally came to represent the wealthier peasants, the intellectuals, and the politically uneducated populations of remote rural districts. Among them there was, however, a wider difference of shades of political and economic opinion than among the Mensheviki. Among their leaders mentioned in these pages: Avksentiev, Gotz, Kerensky, Tchernov, "Babuschka" Breshkovskaya.

\item Left Socialist Revolutionaries. Although theoretically sharing the Bolshevik programme of dictatorship of the working-class, at first were reluctant to follow the ruthless Bolshevik tactics. However, the Left Socialist Revolutionaries remained in the Soviet Government, sharing the Cabinet portfolios, especially that of Agriculture. They withdrew from the Government several times, but always returned. As the peasants left the ranks of the Essaires in increasing numbers, they joined the Left Socialist Revolutionary party, which became the great peasant party supporting the Soviet Government, standing for confiscation without compensation of the great landed estates, and their disposition by the peasants themselves. Among the leaders: Spiridonova, Karelin, Kamkov, Kalagayev.

\item Maximalists. An off-shoot of the Socialist Revolutionary party in the Revolution of 1905, when it was a powerful peasant movement, demanding the immediate application of the maximum Socialist programme. Now an insignificant group of peasant anarchists.
\end{enumerate}
\section*{Parliamentary Procedure}

Russian meetings and conventions are organised after the continental model rather than our own. The first action is usually the election of officers and the presidium.

The presidium is a presiding committee, composed of representatives of the groups and political factions represented in the assembly, in proportion to their numbers. The presidium arranges the Order of Business, and its members can be called upon by the President to take the chair pro tem.

Each question (vopros) is stated in a general way and then debated, and at the close of the debate resolutions are submitted by the different factions, and each one voted on separately. The Order of Business can be, and usually is, smashed to pieces in the first half hour. On the plea of "emergency," which the crowd almost always grants, anybody from the floor can get up and say anything on any subject. The crowd controls the meeting, practically the only functions of the speaker being to keep order by ringing a little bell, and to recognise speakers. Almost all the real work of the session is done in caucuses of the different groups and political factions, which almost always cast their votes in a body and are represented by floor-leaders. The result is, however, that at every important new point, or vote, the session takes a recess to enable the different groups and political factions to hold a caucus.

The crowd is extremely noisy, cheering or heckling speakers, over-riding the plans of the presidium. Among the customary cries are: "Prosim! Please! Go on!" "Pravilno!" or "Eto vierno! That's true! Right!" "Do volno! Enough!" "Doloi! Down with him!" "Posor! Shame!" and "Teesche! Silence! Not so noisy!"

\section*{Popular Organisations}
\begin{enumerate}
\item Soviet. The word soviet means "council." Under the Tsar the Imperial Council of State was called Gosudarstvennyi Soviet. Since the Revolution, however, the term Soviet has come to be associated with a certain type of parliament elected by members of working-class economic organisations-the Soviet of Workers', of Soldiers', or of Peasants' Deputies. I have therefore limited the word to these bodies, and wherever else it occurs I have translated it "Council."

Besides the local Soviets, elected in every city, town and village of Russia-and in large cities, also Ward (Raionny) Soviets-there are also the oblastne or gubiernsky (district or provincial) Soviets, and the Central Executive Committee of the All-Russian Soviets in the capital, called from its initials Tsay-ee-kah. (See below, "Central Committees").

Almost everywhere the Soviets of Workers' and of Soldiers' Deputies combined very soon after the March Revolution. In special matters concerning their peculiar interests, however, the Workers' and the Soldiers' Sections continued to meet separately. The Soviets of Peasants' Deputies did not join the other two until after the Bolshevik coup d'etat. They, too, were organised like the workers and soldiers, with an Executive Committee of the All-Russian Peasants' Soviets in the capital.

\item Trade Unions. Although mostly industrial in form, the Russian labour unions were still called Trade Unions, and at the time of the Bolshevik Revolution had from three to four million members. These Unions were also organised in an All-Russian body, a sort of Russian Federation of Labour, which had its Central Executive Committee in the capital.

\item Factory-Shop Committees. These were spontaneous organisations created in the factories by the workers in their attempt to control industry, taking advantage of the administrative break-down incident upon the Revolution. Their function was by revolutionary action to take over and run the factories. The Factory-Shop Committees also had their All-Russian organisation, with a Central Committee at Petrograd, which co-operated with the Trade Unions.

\item Dumas. The word duma means roughly "deliberative body." The old Imperial Duma, which persisted six months after the Revolution, in a democratised form, died a natural death in September, 1917. The City Duma referred to in this book was the reorganised Municipal Council, often called "Municipal Self-Government." It was elected by direct and secret ballot, and its only reason for failure to hold the masses during the Bolshevik Revolution was the general decline in influence of all purely political representation in the fact of the growing power of organisations based on economic groups.

\item Zemstvos. May be roughly translated "county councils." Under the Tsar semi-political, semi-social bodies with very little administrative power, developed and controlled largely by intellectual Liberals among the land-owning classes. Their most important function was education and social service among the peasants. During the war the Zemstvos gradually took over the entire feeding and clothing of the Russian Army, as well as the buying from foreign countries, and work among the soldiers generally corresponding to the work of the American Y. M. C. A. at the Front. After the March Revolution the Zemstvos were democratized, with a view to making them the organs of local government in the rural districts. But like the City Dumas, they could not compete with the Soviets.

\item Cooperatives. These were the workers' and peasants' Consumers' Cooperative societies, which had several million members all over Russia before the Revolution. Founded by Liberals and "moderate" Socialists, the Cooperative movement was not supported by the revolutionary Socialist groups, because it was a substitute for the complete transference of means of production and distribution into the hands of the workers. After the March Revolution the Cooperatives spread rapidly, and were dominated by Populist Socialists, Mensheviki and Socialist Revolutionaries, and acted as a conservative political force until the Bolshevik Revolution. However, it was the Cooperatives which fed Russia when the old structure of commerce and transportation collapsed.

\item Army Committees. The Army Committees were formed by the soldiers at the front to combat the reactionary influence of the old regime officers. Every company, regiment, brigade, division and corps had its committee, over all of which was elected the Army Committee. The Central Army Committee cooperated with the General Staff. The administrative break-down in the army incident upon the Revolution threw upon the shoulders of the Army Committees most of the work of the Quartermaster's Department, and in some cases, even the command of troops.

\item Fleet Committees. The corresponding organisations in the Navy.
\end{enumerate}
\section*{Central Committees}

In the spring and summer of 1917, All-Russian conventions of every sort of organisation were held at Petrograd. There were national congresses of Workers', Soldiers' and Peasants' Soviets, Trade Unions, Factory-Shop Committees, Army and Fleet Committees-besides every branch of the military and naval service, Cooperatives, Nationalities, etc. Each of these conventions elected a Central Committee, or a Central Executive Committee, to guard its particular interests at the seat of Government. As the Provisional Government grew weaker, these Central Committees were forced to assume more and more administrative powers.

The most important Central Committees mentioned in this book are:

Union of Unions. During the Revolution of 1905, Professor Miliukov and other Liberals established unions of professional men-doctors, lawyers, physicians, etc. These were united under one central organisation, the Union of Unions. In 1905 the Union of Unions acted with the revolutionary democracy; in 1917, however, the Union of Unions opposed the Bolshevik uprising, and united the Government employees who went on strike against the authority of the Soviets.

Tsay-ee-kah. All-Russian Central Executive Committee of the Soviets of Workers' and Soldiers' Deputies. So called from the initials of its name.

Tsentroflot. "Centre-Fleet"-the Central Fleet Committee.

Vikzhel. All-Russian Central Committee of the Railway Workers' Union. So called from the initials of its name.

\section*{Other Organisations}

Red Guards. The armed factory workers of Russia. The Red Guards were first formed during the Revolution of 1905, and sprang into existence again in the days of March, 1917, when a force was needed to keep order in the city. At that time they were armed, and all efforts of the Provisional Government to disarm them were more or less unsuccessful. At every great crisis in the Revolution the Red Guards appeared on the streets, untrained and undisciplined, but full of Revolutionary zeal.

White Guards. Bourgeois volunteers, who emerged in the last stages of the Revolution, to defend private property from the Bolshevik attempt to abolish it. A great many of them were University students.

Tekhintsi. The so-called "Savage Division" in the army, made up of Mohametan tribesmen from Central Asia, and personally devoted to General Kornilov. The Tekhintsi were noted for their blind obedience and their savage cruelty in warfare.

Death Battalions. Or Shock Battalions. The Women's Battalion is known to the world as the Death Battalion, but there were many Death Battalions composed of men. These were formed in the summer of 1917 by Kerensky, for the purpose of strengthening the discipline and combative fire of the army by heroic example. The Death Battalions were composed mostly of intense young patriots. These came for the most part from among the sons of the propertied classes.

Union of Officers. An organisation formed among the reactionary officers in the army to combat politically the growing power of the Army Committees.

Knights of St. George. The Cross of St. George was awarded for distinguished action in battle. Its holder automatically became a "Knight of St. George." The predominant influence in the organisation was that of the supporters of the military idea.

Peasants' Union. In 1905, the Peasants' Union was a revolutionary peasants' organisation. In 1917, however, it had become the political expression of the more prosperous peasants, to fight the growing power and revolutionary aims of the Soviets of Peasants' Deputies.

\section*{Chronology and Spelling}

I have adopted in this book our Calendar throughout, instead of the former Russian Calendar, which was thirteen days earlier.

In the spelling of Russian names and words, I have made no attempt to follow any scientific rules for transliteration, but have tried to give the spelling which would lead the English-speaking reader to the simplest approximation of their pronunciation.

\section*{Sources}

Much of the material in this book is from my own notes. I have also relied, however, upon a heterogeneous file of several hundred assorted Russian newspapers, covering almost every day of the time described, of files of the English paper, the Russian Daily News, and of the two French papers, Journal de Russie and Entente. But far more valuable than these is the Bulletin de la Presse issued daily by the French Information Bureau in Petrograd, which reports all important happenings, speeches and the comment of the Russian press. Of this I have an almost complete file from the spring of 1917 to the end of January, 1918.

Besides the foregoing, I have in my possession almost every proclamation, decree and announcement posted on the walls of Petrograd from the middle of September, 1917, to the end of January, 1918. Also the official publication of all Government decrees and orders, and the official Government publication of the secret treaties and other documents discovered in the Ministry of Foreign Affairs when the Bolsheviki took it over.