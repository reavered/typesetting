\hermogenesspeaks
Here is Socrates; shall we take him as a partner in our discussion?
\cratylusspeaks
If you like.
\hermogenesspeaks
Cratylus, whom you see here, Socrates, says that everything has a right name of its own, which comes by nature, and that a name is not whatever people call a thing by agreement, just a piece of their own voice applied to the thing, but that there is a kind of inherent correctness in names, which is the same for all men, \stephpag{b} both Greeks and barbarians. So I ask him whether his name is in truth Cratylus, and he agrees that it is. ``And what is Socrates' name?" I said. ``Socrates," said he. ``Then that applies to all men, and the particular name by which we call each person is his name?" And he said, ``Well, your name is not \hermogenesspeaks,1 even if all mankind call you so." Now, though I am asking him \stephpag{384 a} and am exerting myself to find out what in the world he means, he does not explain himself at all; he meets me with dissimulation, claiming to have some special knowledge of his own about it which would, if he chose to speak it out clearly, make me agree entirely with him. Now if you could interpret Cratylus's oracular speech, I should like to hear you; or rather, I should like still better to hear, if you please, what you yourself think about the correctness of names.
\socratesspeaks
\hermogenesspeaks, son of Hipponicus, there is an ancient saying \stephpag{b} that knowledge of high things is hard to gain; and surely knowledge of names is no small matter. Now if I had attended Prodicus's fifty-drachma course of lectures, after which, as he himself says, a man has a complete education on this subject, there would be nothing to hinder your learning the truth about the correctness of names at once; but I have heard only the one-drachma course, \stephpag{c} and so I do not know what the truth is about such matters. However, I am ready to join you and Cratylus in looking for it. But as for his saying that \hermogenesspeaks is not truly your name, I suspect he is making fun of you; for perhaps he thinks that you want to make money and fail every time. But, as I said, it is difficult to know such things. We must join forces and try to find out whether you are right, or Cratylus.
\hermogenesspeaks
For my part, Socrates, I have often talked with Cratylus and many others, \stephpag{d} and cannot come to the conclusion that there is any correctness of names other than convention and agreement. For it seems to me that whatever name you give to a thing is its right name; and if you give up that name and change it for another, the later name is no less correct than the earlier, just as we change the names of our servants; for I think no name belongs to any particular thing by nature, but only by the habit and custom of those who employ it and who established the usage. \stephpag{e} But if this is not the case, I am ready to hear and to learn from Cratylus or anyone else. \stephpag{385 a}
\socratesspeaks
It may be that you are right, \hermogenesspeaks; but let us see. Whatever name we decide to give each particular thing is its name?
\hermogenesspeaks
Yes.
\socratesspeaks
Whether the giver be a private person or a state?
\hermogenesspeaks
Yes.
\socratesspeaks
Well, then, suppose I give a name to some thing or other, designating, for instance, that which we now call ``man" as ``horse" and that which we now call ``horse" as ``man," will the real name of the same thing be ``man" for the public and ``horse" for me individually, and in the other case ``horse" for the public and ``man" for me individually? Is that your meaning? \stephpag{b}
\hermogenesspeaks
Yes, that is my opinion.
\socratesspeaks
Now answer this question. Is there anything which you call speaking the truth and speaking falsehood?
\hermogenesspeaks
Yes.
\socratesspeaks
Then there would be true speech and false speech?
\hermogenesspeaks
Certainly.
\socratesspeaks
Then that speech which says things as they are is true, and that which says them as they are not is false?
\hermogenesspeaks
Yes.
\socratesspeaks
It is possible, then, to say in speech that which is and that which is not?
\hermogenesspeaks
Certainly.
\socratesspeaks
But is true speech true only as a whole, \stephpag{c} and are its parts untrue?
\hermogenesspeaks
No, its parts also are true.
\socratesspeaks
Are the large parts true, but not the small ones, or are all true?
\hermogenesspeaks
All, in my opinion.
\socratesspeaks
Is there, then, anything which you say is a smaller part of speech than a name?
\hermogenesspeaks
No, that is the smallest.
\socratesspeaks
And the name is spoken as a part of the true speech?
\hermogenesspeaks
Yes.
\socratesspeaks
Then it is, according to you, true.
\hermogenesspeaks
Yes.
\socratesspeaks
And a part of false speech is false, is it not?
\hermogenesspeaks
It is.
\socratesspeaks
Then it is possible to utter either a false or a true name, since one may utter speech that is either true or false? \stephpag{d}
\hermogenesspeaks
Yes, of course.
\socratesspeaks
Then whatever each particular person says is the name of anything, that is its name for that person?
\hermogenesspeaks
Yes.
\socratesspeaks
And whatever the number of names anyone says a thing has, it will really have that number at the time when he says it?
\hermogenesspeaks
Yes, Socrates, for I cannot conceive of any other kind of correctness in names than this; I may call a thing by one name, which I gave, and you by another, which you gave. And in the same way, I see that states have their own different names for the same things, \stephpag{e} and Greeks differ from other Greeks and from barbarians in their use of names.
\socratesspeaks
Now, \hermogenesspeaks, let us see. Do you think this is true of the real things, that their reality is a separate one for each person, as Protagoras said with his doctrine \stephpag{386 a} that man is the measure of all things—that things are to me such as they seem to me, and to you such as they seem to you—or do you think things have some fixed reality of their own?
\hermogenesspeaks
It has sometimes happened to me, Socrates, to be so perplexed that I have been carried away even into this doctrine of Protagoras; but I do not at all believe he is right.
\socratesspeaks
Well, have you ever been carried away so far \stephpag{b} as not to believe at all that any man is bad?
\hermogenesspeaks
Lord, no; but I have often been carried away into the belief that certain men, and a good many of them, are very bad.
\socratesspeaks
Well, did you never think any were very good?
\hermogenesspeaks
Very few.
\socratesspeaks
But you did think them so?
\hermogenesspeaks
Yes.
\socratesspeaks
And what is your idea about that? Are the very good very wise and the very bad very foolish? \stephpag{c}
\hermogenesspeaks
Yes, that is my opinion.
\socratesspeaks
Now if Protagoras is right and the truth is as he says, that all things are to each person as they seem to him, is it possible for some of us to be wise and some foolish?
\hermogenesspeaks
No, it is not.
\socratesspeaks
And you are, I imagine, strongly of the opinion that if wisdom and folly exist, it is quite impossible that Protagoras is right, for one man would not in reality be at all wiser than another \stephpag{d} if whatever seems to each person is really true to him.
\hermogenesspeaks
Quite right.
\socratesspeaks
But neither do you believe with Euthydemus that all things belong equally to all men at the same time and perpetually,2 for on this assumption also some could not be good and others bad, if virtue and its opposite were always equally possessed by all.
\hermogenesspeaks
True.
\socratesspeaks
Then if neither all things belong equally to all men at the same time and perpetually nor each thing to each man individually, it is clear that things have some fixed reality of their own, \stephpag{e} not in relation to us nor caused by us; they do not vary, swaying one way and another in accordance with our fancy, but exist of themselves in relation to their own reality imposed by nature.
\hermogenesspeaks
I think, Socrates, that is the case.
\socratesspeaks
Can things themselves, then, possess such a nature as this, and that of their actions be different? Or are not actions also a class of realities?
\hermogenesspeaks
Certainly they are. \stephpag{387 a}
\socratesspeaks
Then actions also are performed according to their own nature, not according to our opinion. For instance, if we undertake to cut anything, ought we to cut it as we wish, and with whatever instrument we wish, or shall we, if we are willing to cut each thing in accordance with the nature of cutting and being cut, and with the natural instrument, succeed in cutting it, and do it rightly, whereas if we try to do it contrary to nature we shall fail and accomplish nothing? \stephpag{b}
\hermogenesspeaks
I think the way is as you suggest.
\socratesspeaks
Then, too, if we undertake to burn anything, we must burn not according to every opinion, but according to the right one? And that is as each thing naturally burns or is burned and with the natural instrument?
\hermogenesspeaks
True.
\socratesspeaks
And all other actions are to be performed In like manner?
\hermogenesspeaks
Certainly.
\socratesspeaks
And speaking is an action, is it not?
\hermogenesspeaks
Yes.
\socratesspeaks
Then if a man speaks as he fancies he ought to speak, \stephpag{c} will he speak rightly, or will he succeed in speaking if he speaks in the way and with the instrument in which and with which it is natural for us to speak and for things to be spoken, whereas otherwise he will fail and accomplish nothing?
\hermogenesspeaks
I think the way you suggest is the right one.
\socratesspeaks
Now naming is a part of speaking, for in naming I suppose people utter speech.
\hermogenesspeaks
Certainly.
\socratesspeaks
Then is not naming also a kind of action, if speaking is a kind of action concerned with things?
\hermogenesspeaks
Yes. \stephpag{d}
\socratesspeaks
But we saw that actions are not merely relative to us, but possess a separate nature of their own?
\hermogenesspeaks
True.
\socratesspeaks
Then in naming also, if we are to be consistent with our previous conclusions, we cannot follow our own will, but the way and the instrument which the nature of things prescribes must be employed, must they not? And if we pursue this course we shall be successful in our naming, but otherwise we shall fail.
\hermogenesspeaks
I think you are right.
\socratesspeaks
And again, what has to be cut, we said, has to be cut with something.
\hermogenesspeaks
Certainly. \stephpag{e}
\socratesspeaks
And what has to be woven, has to be woven with something, and what has to be bored, has to be bored with something?
\hermogenesspeaks
Certainly.
\socratesspeaks
And then what has to be named, has to be named with something? \stephpag{388 a}
\hermogenesspeaks
True.
\socratesspeaks
And what is that with which we have to bore?
\hermogenesspeaks
A borer.
\socratesspeaks
And that with which we weave?
\hermogenesspeaks
A shuttle.
\socratesspeaks
And that with which we must name?
\hermogenesspeaks
A name.
\socratesspeaks
Right. A name also, then, is a kind of instrument.
\hermogenesspeaks
Certainly.
\socratesspeaks
Then if I were to ask ``What instrument is the shuttle?" Is it not that with which we weave?
\hermogenesspeaks
Yes. \stephpag{b}
\socratesspeaks
And what do we do when we weave? Do we not separate the mingled threads of warp and woof?
\hermogenesspeaks
Yes.
\socratesspeaks
And you could give a similar answer about the borer and the rest, could you not?
\hermogenesspeaks
Certainly.
\socratesspeaks
And can you say something of the same kind about a name? The name being an instrument, what do we do with it when we name?
\hermogenesspeaks
I cannot tell.
\socratesspeaks
Do we not teach one another something, and separate things according to their natures?
\hermogenesspeaks
Certainly.
\socratesspeaks
A name is, then, an instrument of teaching and of separating reality, \stephpag{c} as a shuttle is an instrument of separating the web?
\hermogenesspeaks
Yes.
\socratesspeaks
But the shuttle is an instrument of weaving?
\hermogenesspeaks
Of course.
\socratesspeaks
The weaver, then, will use the shuttle well, and well means like a weaver; and a teacher will use a name well, and well means like a teacher.
\hermogenesspeaks
Yes.
\socratesspeaks
Whose work will the weaver use well when he uses the shuttle?
\hermogenesspeaks
The carpenter's.
\socratesspeaks
Is every one a carpenter, or he who has the skill?
\hermogenesspeaks
He who has the skill. \stephpag{d}
\socratesspeaks
And whose work will the hole-maker use when he uses the borer?
\hermogenesspeaks
The smith's.
\socratesspeaks
And is every one a smith, or he who has the skill?
\hermogenesspeaks
He who has the skill.
\socratesspeaks
And whose work will the teacher use when he uses the name?
\hermogenesspeaks
I cannot tell that, either.
\socratesspeaks
And can you not tell this, either, who gives us the names we use?
\hermogenesspeaks
No.
\socratesspeaks
Do you not think it is the law that gives them to us?
\hermogenesspeaks
Very likely. \stephpag{e}
\socratesspeaks
Then the teacher, when he uses a name, will be using the work of a lawgiver?
\hermogenesspeaks
I think so.
\socratesspeaks
Do you think every man is a lawgiver, or only he who has the skill?
\hermogenesspeaks
He who has the skill.
\socratesspeaks
Then it is not for every man, \hermogenesspeaks, \stephpag{389 a} to give names, but for him who may be called the name-maker; and he, it appears, is the lawgiver, who is of all the artisans among men the rarest.
\hermogenesspeaks
So it appears.
\socratesspeaks
See now what the lawgiver has in view in giving names. Look at it in the light of what has gone before. What has the carpenter in view when he makes a shuttle? Is it not something the nature of which is to weave?
\hermogenesspeaks
Certainly.
\socratesspeaks
Well, then, if the shuttle breaks while he making it, \stephpag{b} will he make another with his mind fixed on that which is broken, or on that form with reference to which he was making the one which he broke?
\hermogenesspeaks
On that form, in my opinion.
\socratesspeaks
Then we should very properly call that the absolute or real shuttle?
\hermogenesspeaks
Yes, I think so.
\socratesspeaks
Then whenever he has to make a shuttle for a light or a thick garment, or for one of linen or of wool or of any kind whatsoever, all of them must contain the form or ideal of shuttle, \stephpag{c} and in each of his products he must embody the nature which is naturally best for each?
\hermogenesspeaks
Yes.
\socratesspeaks
And the same applies to all other instruments. The artisan must discover the instrument naturally fitted for each purpose and must embody that in the material of which he makes the instrument, not in accordance with his own will, but in accordance with its nature. He must, it appears, know how to embody in the iron the borer fitted by nature for each special use.
\hermogenesspeaks
Certainly.
\socratesspeaks
And he must embody in the wood the shuttle fitted by nature for each kind of weaving.
\hermogenesspeaks
True. \stephpag{d}
\socratesspeaks
For each kind of shuttle is, it appears, fitted by nature for its particular kind of weaving, and the like is true of other instruments.
\hermogenesspeaks
Yes.
\socratesspeaks
Then, my dear friend, must not the law-giver also know how to embody in the sounds and syllables that name which is fitted by nature for each object? Must he not make and give all his names with his eye fixed upon the absolute or ideal name, if he is to be an authoritative giver of names? And if different lawgivers do not embody it in the same syllables, we must not forget this ideal name on that account; for different smiths do not embody the form in the same iron, \stephpag{e} though making the same instrument for the same purpose, but so long as they reproduce the same ideal, \stephpag{390 a} though it be in different iron, still the instrument is as it should be, whether it be made here or in foreign lands, is it not?
\hermogenesspeaks
Certainly.
\socratesspeaks
On this basis, then, you will judge the law-giver, whether he be here or in a foreign land, so long as he gives to each thing the proper form of the name, in whatsoever syllables, to be no worse lawgiver, whether here or anywhere else, will you not?
\hermogenesspeaks
Certainly. \stephpag{b}
\socratesspeaks
Now who is likely to know whether the proper form of shuttle is embodied in any piece of wood? The carpenter who made it, or the weaver who is to use it ?
\hermogenesspeaks
Probably the one who is to use it, Socrates.
\socratesspeaks
Then who is to use the work of the lyre-maker? Is not he the man who would know best how to superintend the making of the lyre and would also know whether it is well made or not when it is finished?
\hermogenesspeaks
Certainly.
\socratesspeaks
Who is he?
\hermogenesspeaks
The lyre-player.
\socratesspeaks
And who would know best about the work of the ship-builder? \stephpag{c}
\hermogenesspeaks
The navigator.
\socratesspeaks
And who can best superintend the work of the lawgiver and judge of it when it is finished, both here and in foreign countries? The user, is it not?
\hermogenesspeaks
Yes.
\socratesspeaks
And is not this he who knows how to ask questions?
\hermogenesspeaks
Certainly.
\socratesspeaks
And the same one knows also how to make replies?
\hermogenesspeaks
Yes.
\socratesspeaks
And the man who knows how to ask and answer questions you call a dialectician?
\hermogenesspeaks
Yes, that is what I call him. \stephpag{d}
\socratesspeaks
The work of the carpenter, then, is to make a rudder under the supervision of the steersman, if he rudder is to be a good one.
\hermogenesspeaks
Evidently.
\socratesspeaks
And the work of the lawgiver, as it seems, is to make a name, with the dialectician as his supervisor, if names are to be well given.
\hermogenesspeaks
True.
\socratesspeaks
Then, \hermogenesspeaks, the giving of names can hardly be, as you imagine, a trifling matter, or a task for trifling or casual persons: and Cratylus is right in saying that names belong to things by nature \stephpag{e} and that not every one is an artisan of names, but only he who keeps in view the name which belongs by nature to each particular thing and is able to embody its form in the letters and syllables.
\hermogenesspeaks
I do not know how to answer you, Socrates; nevertheless it is not easy to change my conviction suddenly. \stephpag{391 a} I think you would be more likely to convince me, if you were to show me just what it is that you say is the natural correctness of names.
\socratesspeaks
I, my dear \hermogenesspeaks, do not say that there is any. You forget what I said a while ago, that I did not know, but would join you in looking for the truth. And now, as we are looking, you and I, we already see one thing we did not know before, that names do possess a certain natural correctness, and that not every man knows \stephpag{b} how to give a name well to anything whatsoever. Is not that true?
\hermogenesspeaks
Certainly.
\socratesspeaks
Then our next task is to try to find out, if you care to know about it, what kind of correctness that is which belongs to names.
\hermogenesspeaks
To be sure I care to know.
\socratesspeaks
Then investigate.
\hermogenesspeaks
How shall I investigate?
\socratesspeaks
The best way to investigate, my friend, is with the help of those who know; and you make sure of their favour by paying them money. They are the sophists, \stephpag{c} from whom your brother Callias got his reputation for wisdom by paying them a good deal of money. But since you have not the control of your inheritance, you ought to beg and beseech your brother to teach you the correctness which he learned of Protagoras about such matters.
\hermogenesspeaks
It would be an absurd request for me, Socrates, if I, who reject the Truth3 of Protagoras altogether, should desire what is said in such a Truth, as if it were of any value.
\socratesspeaks
Then if you do not like that, \stephpag{d} you ought to learn from Homer and the other poets.
\hermogenesspeaks
Why, Socrates, what does Homer say about names, and where?
\socratesspeaks
In many passages; but chiefly and most admirably in those in which he distinguishes between the names by which gods and men call the same things. Do you not think he gives in those passages great and wonderful information about the correctness of names? For clearly the gods call things \stephpag{e} by the names that are naturally right. Do you not think so?
\hermogenesspeaks
Of course I know that if they call things, they call them rightly. But what are these instances to which you refer?
\socratesspeaks
Do you not know that he says about the river in Troy which had the single combat with Hephaestus,4``whom the gods call Xanthus, but men call Scamander
"Hom. Il. 20.74?
\hermogenesspeaks
Oh yes. \stephpag{392 a}
\socratesspeaks
Well, do you not think this is a grand thing to know, that the name of that river is rightly Xanthus, rather than Scamander? Or, if you like, do you think it is a slight thing to learn about the bird which he says ``gods call chalcis, but men call cymindis,
"Hom. Il. 14.291 that it is much more correct for the same bird to be called chalcis than cymindis? Or to learn that the hill men call Batieia is called by the gods Myrina's tomb,5 and many other such statements by Homer and other poets? \stephpag{b} But perhaps these matters are too high for us to understand; it is, I think, more within human power to investigate the names Scamandrius and Astyanax, and understand what kind of correctness he ascribes to these, which he says are the names of Hector's son. You recall, of course: the lines which contain the words to which I refer.
\hermogenesspeaks
Certainly.
\socratesspeaks
Which of the names of the boy do you imagine Homer thought was more correct, Astyanax or Scamandrius? \stephpag{c}
\hermogenesspeaks
I cannot say.
\socratesspeaks
Look at it in this way: suppose you were asked, ``Do the wise or the unwise give names more correctly?"
\hermogenesspeaks
``The wise, obviously," I should say.
\socratesspeaks
And do you think the women or the men of a city, regarded as a class in general, are the wiser?
\hermogenesspeaks
The men.
\socratesspeaks
And do you not know that Homer says the child of Hector was called Astyanax by the men of Troy;6 \stephpag{d} so he must have been called Scamandrius by the women, since the men called him Astyanax?
\hermogenesspeaks
Yes, probably.
\socratesspeaks
And Homer too thought the Trojan men were wiser than the women?
\hermogenesspeaks
I suppose he did.
\socratesspeaks
Then he thought Astyanax was more rightly the boy's name than Scamandrius?
\hermogenesspeaks
So it appears.
\socratesspeaks
Let us, then, consider the reason for this. Does he not himself indicate the reason most admirably? For he says— \stephpag{e} ``He alone defended their city and long walls.
"Hom. Il. 22.5077 Therefore, as it seems, it is right to call the son of the defender Astyanax \clarify{Lord of the city}, ruler of that which his father, as Homer says, defended.
\hermogenesspeaks
That is clear to me.
\socratesspeaks
Indeed? I do not yet understand about it myself, \hermogenesspeaks. Do you?
\hermogenesspeaks
No, by Zeus, I do not. \stephpag{393 a}
\socratesspeaks
But, my good friend, did not Homer himself also give Hector his name?
\hermogenesspeaks
Why do you ask that?
\socratesspeaks
Because that name seems to me similar to Astyanax, and both names seem to be Greek. For lord \clarify{ἄναξ} and holder \clarify{ἕκτωρ} mean nearly the same thing, indicating that they are names of a king; for surely a man is holder of that of which he is lord; for it is clear that he rules it and possesses it and holds it. \stephpag{b} Or does it seem to you that there is nothing in what I am saying, and am I wrong in imagining that I have found a clue to Homer's opinion about the correctness of names?
\hermogenesspeaks
No, by Zeus, you are not wrong, in my opinion; I think perhaps you have found a clue.
\socratesspeaks
It is right, I think, to call a lion's offspring a lion and a horse's offspring a horse. I am not speaking of prodigies, such as the birth of some other kind of creature from a horse, \stephpag{c} but of the natural offspring of each species after its kind. If a horse, contrary to nature, should bring forth a calf, the natural offspring of a cow, it should be called a calf, not a colt, nor if any offspring that is not human should be born from a human being, should that other offspring be called a human being; and the same applies to trees and all the rest. Do you not agree?
\hermogenesspeaks
Yes.
\socratesspeaks
Good; but keep watch of me, and do not let me trick you; for by the same argument any offspring of a king should be called a king; \stephpag{d} and whether the same meaning is expressed in one set of syllables or another makes no difference; and if a letter is added or subtracted, that does not matter either, so long as the essence of the thing named remains in force and is made plain in the name.
\hermogenesspeaks
What do you mean?
\socratesspeaks
Something quite simple. For instance, when we speak of the letters of the alphabet, you know, we speak their names, not merely the letters themselves, except in the case of four, ε, υ, ο, ω.8 \stephpag{e} We make names for all the other vowels and consonants by adding other letters to them; and so long as we include the letter in question and make its force plain, we may properly call it by that name, and that will designate it for us. Take beta, for instance, The addition of e\clarify{η}, t\clarify{τ}, a\clarify{α} does no harm and does not prevent the whole name from making clear the nature of that letter which the lawgiver wished to designate; he knew so well how to give names to letters.
\hermogenesspeaks
I think you are right.
\socratesspeaks
Does not the same reasoning apply to a king? \stephpag{394 a} A king's son will probably be a king, a good man's good, a handsome man's handsome, and so forth; the offspring of each class will be of the same class, unless some unnatural birth takes place; so they should be called by the same names. But variety in the syllables is admissible, so that names which are the same appear different to the uninitiated, just as the physicians' drugs, when prepared with various colors and perfumes, seem different to us, though they are the same, but to the physician, \stephpag{b} who considers only their medicinal value, they seem the same, and he is not confused by the additions. So perhaps the man who knows about names considers their value and is not confused if some letter is added, transposed, or subtracted, or even if the force of the name is expressed in entirely different letters. So, for instance, in the names we were just discussing, Astyanax and Hector, none of the letters is the same, except T, \stephpag{c} but nevertheless they have the same meaning. And what letters has Archepolis \clarify{ruler of the city} in common with them? Yet it means the same thing; and there are many other names which mean simply ``king." Others again mean ``general," such as Agis \clarify{leader}, Polemarchus \clarify{war-lord}, and Eupolemus \clarify{good warrior}; and others indicate physicians, as Iatrocles \clarify{famous physician} and Acesimbrotus \clarify{healer of mortals}; and we might perhaps find many others which differ in syllables and letters, but express the same meaning. Do you think that is true, or not? \stephpag{d}
\hermogenesspeaks
Certainly.
\socratesspeaks
To those, then, who are born in accordance with nature the same names should be given.
\hermogenesspeaks
Yes.
\socratesspeaks
And how about those who are born contrary to nature as prodigies? For instance, when an impious son is born to a good and pious man, ought he not, as in our former example when a mare brought forth a calf, to have the designation of the class to which he belongs, instead of that of his parent?
\hermogenesspeaks
Certainly. \stephpag{e}
\socratesspeaks
Then the impious son of a pious father ought to receive the name of his class.
\hermogenesspeaks
True.
\socratesspeaks
Not Theophilus \clarify{beloved of God} or Mnesitheus \clarify{mindful of God} or anything of that sort; but something of opposite meaning, if names are correct.
\hermogenesspeaks
Most assuredly, Socrates.
\socratesspeaks
As the name of Orestes \clarify{mountain man} is undoubtedly correct, \hermogenesspeaks, whether it was given him by chance or by some poet who indicated by the name the fierceness, rudeness, and mountain-wildness of his nature. \stephpag{395 a}
\hermogenesspeaks
So it seems, Socrates.
\socratesspeaks
And his father's name also appears to be in accordance with nature.
\hermogenesspeaks
It seems so.
\socratesspeaks
Yes, for Agamemnon \clarify{admirable for remaining} is one who would resolve to toil to the end and to endure, putting the finish upon his resolution by virtue. And a proof of this is his long retention of the host at Troy and his endurance. So the name Agamemnon denotes that this man is admirable for remaining. \stephpag{b} And so, too, the name of Atreus is likely to be correct; for his murder of Chrysippus and the cruelty of his acts to Thyestes are all damaging and ruinous \clarify{ἀτηρά} to his virtue. Now the form of his name is slightly deflected and hidden, so that it does not make the man's nature plain to every one; but to those who understand about names it makes the meaning of Atreus plain enough; for indeed \stephpag{c} in view of his stubbornness \clarify{ἀτειρές} and fearlessness \clarify{ἄτρεστον} and ruinous acts \clarify{ἀτηρά} the name is correctly given to him on every ground. And I think Pelops also has a fitting name; for this name means that he who sees only what is near deserves this designation.
\hermogenesspeaks
How is that?
\socratesspeaks
Why it is said of him that in murdering Myrtilus he was quite unable to forecast or foresee the ultimate effects upon his whole race, and all the misery with which it was overwhelmed, \stephpag{d} because he saw only the near at hand and the immediate— that is to say, πέλας \clarify{near}—in his eagerness to win by all means the hand of Hippodameia. And any one would think the name of Tantalus was given rightly and in accordance with nature, if the stories about him are true.
\hermogenesspeaks
What are the stories?
\socratesspeaks
The many terrible misfortunes that happened to him both in his life, the last of which was the utter overthrow of his country, and in Hades, after his death, \stephpag{e} the balancing \clarify{ταλαντεία} of the stone above his head, in wonderful agreement with his name; and it seems exactly as if someone who wished to call him most wretched \clarify{ταλάντατον} disguised the name and said Tantalus instead; in some such way as that chance seems to have affected his name in the legend. And his father also, who is said to be Zeus, appears to have a very excellent name, but it is not easy to understand; \stephpag{396 a} for the name of Zeus is exactly like a sentence; we divide it into two parts, and some of us use one part, others the other; for some call him Zena \clarify{Ζῆνα}, and others Dia \clarify{Δία}; but the two in combination express the nature of the god, which is just what we said a name should be able to do. For certainly no one is so much the author of life \clarify{ζῆν} for us and all others as the ruler and king of all. \stephpag{b} Thus this god is correctly named, through whom \clarify{δι᾽ ὅν} all living beings have the gift of life \clarify{ζῆν}. But, as I say, the name is divided, though it is one name, into the two parts, Dia and Zena. And it might seem, at first hearing, highly irreverent to call him the son of Cronus and reasonable to say that Zeus is the offspring of some great intellect; and so he is, for κόρος \clarify{for Κρόνος} signifies not child, but the purity \clarify{καθαρόν} and unblemished nature of his mind. And Cronus, according to tradition, is the son of Uranus; but the upward gaze is rightly called by the name urania \clarify{οὐρανία}, \stephpag{c} looking at the things above \clarify{ὁρῶ τὰ ἄνω}, and the astronomers say, \hermogenesspeaks, that from this looking people acquire a pure mind, and Uranus is correctly named. If I remembered the genealogy of Hesiod and the still earlier ancestors of the gods he mentions, I would have gone on examining the correctness of their names until I had made a complete trial whether this wisdom which has suddenly come to me, I know not whence, \stephpag{d} will fail or not.
\hermogenesspeaks
Indeed, Socrates, you do seem to me to be uttering oracles, exactly like an inspired prophet.
\socratesspeaks
Yes, \hermogenesspeaks, and I am convinced that the inspiration came to me from Euthyphro the Prospaltian. For I was with him and listening to him a long time early this morning. So he must have been inspired, and he not only filled my ears but took possession of my soul with his superhuman wisdom. So I think this is our duty: \stephpag{e} we ought today to make use of this wisdom and finish the investigation of names, but tomorrow, if the rest of you agree, we will conjure it away and purify ourselves, when we have found some one, whether priest or sophist, \stephpag{397 a} who is skilled in that kind of purifying.
\hermogenesspeaks
I agree, for I should be very glad to hear the rest of the talk about names.
\socratesspeaks
Very well. Then since we have outlined a general plan of investigation, where shall we begin, that we may discover whether the names themselves will bear witness that they are not at all distributed at haphazard, but have a certain correctness? \stephpag{b} Now the names of heroes and men might perhaps prove deceptive; for they are often given because they were names of ancestors, and in some cases, as we said in the beginning, they are quite inappropriate; many, too, are given as the expression of a prayer, such as Eutychides \clarify{fortunate}, Sosias \clarify{saviour}, Theophilus \clarify{beloved of God}, and many others. I think we had better disregard such as these; but we are most likely to find the correct names in the nature of the eternal and absolute; for there the names ought to have been given with the greatest care, \stephpag{c} and perhaps some of them were given by a power more divine than is that of men.
\hermogenesspeaks
I think you are right, Socrates.
\socratesspeaks
Then is it not proper to begin with the gods and see how the gods are rightly called by that name?
\hermogenesspeaks
That is reasonable.
\socratesspeaks
Something of this sort, then, is what I suspect: I think the earliest men in Greece believed only in those gods in whom many foreigners believe today— \stephpag{d} sun, moon, earth, stars, and sky. They saw that all these were always moving in their courses and running, and so they called them gods \clarify{θεούς} from this running \clarify{θεῖν} nature; then afterwards, when they gained knowledge of the other gods, they called them all by the same name. Is that likely to be true, or not?
\hermogenesspeaks
Yes, very likely.
\socratesspeaks
What shall we consider next? \stephpag{e}
\hermogenesspeaks
Spirits, obviously.
\socratesspeaks
\hermogenesspeaks, what does the name ``spirits" really mean? See if you think there is anything in what I am going to say.
\hermogenesspeaks
Go on and say it.
\socratesspeaks
Do you remember who Hesiod says the spirits are?
\hermogenesspeaks
I do not recall it.
\socratesspeaks
Nor that he says a golden race was the first race of men to be born?
\hermogenesspeaks
Yes, I do know that.
\socratesspeaks
Well, he says of it:``But since Fate has covered up this race,
" \stephpag{398 a}
``They are called holy spirits under the earth,
Noble, averters of evil, guardians of mortal men.
"
Hes. WD 122 ff
\hermogenesspeaks
What of that?
\socratesspeaks
Why, I think he means that the golden race was not made of gold, but was good and beautiful. And I regard it as a proof of this that he further says we are the iron race.
\hermogenesspeaks
True.
\socratesspeaks
Don't you suppose that if anyone of our day is good, \stephpag{b} Hesiod would say he was of that golden race?
\hermogenesspeaks
Quite likely.
\socratesspeaks
But the good are the wise, are they not?
\hermogenesspeaks
Yes, they are the wise.
\socratesspeaks
This, then, I think, is what he certainly means to say of the spirits: because they were wise and knowing \clarify{δαήμονες} he called them spirits \clarify{δαίμονες} and in the old form of our language the two words are the same. Now he and all the other poets are right, who say that when a good man dies \stephpag{c} he has a great portion and honor among the dead, and becomes a spirit, a name which is in accordance with the other name of wisdom. And so I assert that every good man, whether living or dead, is of spiritual nature, and is rightly called a spirit.
\hermogenesspeaks
And I, Socrates, believe I quite agree with you in that. But what is the word ``hero"?
\socratesspeaks
That is easy to understand; for the name has been but slightly changed, and indicates their origin from love \clarify{ἔρως}.
\hermogenesspeaks
What do you mean? \stephpag{d}
\socratesspeaks
Why, they were all born because a god fell in love with a mortal woman, or a mortal man with a goddess. Now if you consider the word ``hero" also in the old Attic pronunciation,9 you will understand better; for that will show you that it has been only slightly altered from the name of love \clarify{Eros}, the source from which the heroes spring, to make a name for them. And either this is the reason why they are called heroes, or it is because they were wise and clever orators and dialecticians, able to ask questions \clarify{ἐρωτᾶν}, for εἴρειν is the same as λέγειν \clarify{speak}. Therefore, when their name is spoken in the Attic dialect, which I was mentioning just now, the heroes turn out to be orators and askers of questions, \stephpag{e} so that the heroic race proves to be a race of orators and sophists. That is easy to understand, but the case of men, and the reason why they are called men \clarify{ἄνθρωποι}, is more difficult. Can you tell me what it is?
\hermogenesspeaks
No, my friend, I cannot; and even if I might perhaps find out, I shall not try, because I think you are more likely to find out than I am. \stephpag{399 a}
\socratesspeaks
You have faith in the inspiration of Euthyphro, it seems.
\hermogenesspeaks
Evidently.
\socratesspeaks
And you are right in having it; for just at this very moment I think I have had a clever thought, and if I am not careful, before the day is over I am likely to be wiser than I ought to be. So pay attention. First we must remember in regard to names that we often put in or take out letters, making the names different from the meaning we intend, and we change the accent. \stephpag{b} Take, for instance, Διὶ φίλος; to change this from a phrase to a name, we took out the second iota and pronounced the middle syllable with the grave instead of the acute accent \clarify{Diphilus}. In other instances, on the contrary, we insert letters and pronounce grave accents as acute.
\hermogenesspeaks
True.
\socratesspeaks
Now it appears to me that the name of men \clarify{ἄνθρωπος} underwent a change of that sort. It was a phrase, but became a noun when one letter, alpha, was removed and the accent of the last syllable was dropped.
\hermogenesspeaks
What do you mean? \stephpag{c}
\socratesspeaks
I will tell you. The name ``man" \clarify{ἄνθρωπος} indicates that the other animals do not examine, or consider, or look up at \clarify{ἀναθρεῖ} any of the things that they see, but man has no sooner seen—that is, ὄπωπε—than he looks up at and considers that which he has seen. Therefore of all the animals man alone is rightly called man \clarify{ἄνθρωπος}, because he looks up at \clarify{ἀναθρεῖ} what he has seen \clarify{ὄπωπε}.
\hermogenesspeaks
Of course. May I ask you about the next word I should like to have explained?
\socratesspeaks
Certainly. \stephpag{d}
\hermogenesspeaks
It seems to me to come naturally next after those you have discussed. We speak of man's soul and body.
\socratesspeaks
Yes, of course.
\hermogenesspeaks
Let us try to analyze these, as we did the previous words.
\socratesspeaks
You mean consider ``soul" \clarify{ψυχή} and see why it is properly called by that name, and likewise ``body" \clarify{σῶμα}?
\hermogenesspeaks
Yes.
\socratesspeaks
To speak on the spur of the moment, I think those who gave the soul its name had something of this sort in mind: they thought when it was present in the body it was the cause of its living, \stephpag{e} giving it the power to breathe and reviving it \clarify{ἀναψῦχον}, and when this revivifying force fails, the body perishes and comes to an end therefore, I think, they called it ψυχή. But—please keep still a moment. I fancy I see something which will carry more conviction \stephpag{400 a} to Euthyphro and his followers; for I think they would despise this attempt and would consider it cheap talk. Now see if you like the new one.
\hermogenesspeaks
I am listening.
\socratesspeaks
Do you think there is anything which holds and carries the whole nature of the body, so that it lives and moves, except the soul?
\hermogenesspeaks
No; nothing.
\socratesspeaks
Well, and do you not believe the doctrine of Anaxagoras, that it is mind or soul which orders and holds the nature of all things?
\hermogenesspeaks
I do. \stephpag{b}
\socratesspeaks
Then there would be an admirable fitness in calling that power which carries and holds \clarify{ἔχει} nature \clarify{φύσιν} φυσέχη and this may be refined and pronounced ψυχή.
\hermogenesspeaks
Certainly; and I think this is a more scientific explanation than the other.
\socratesspeaks
Yes, it is. But it seems actually absurd that the name was given with such truth.
\hermogenesspeaks
Now what shall we say about the next word?
\socratesspeaks
You mean ``body" \clarify{σῶμα}?
\hermogenesspeaks
Yes.
\socratesspeaks
I think this admits of many explanations, if a little, even very little, change is made; for some say it is the tomb \clarify{σῆμα} of the soul, \stephpag{c} their notion being that the soul is buried in the present life; and again, because by its means the soul gives any signs which it gives, it is for this reason also properly called ``sign" \clarify{σῆμα}. But I think it most likely that the Orphic poets gave this name, with the idea that the soul is undergoing punishment for something; they think it has the body as an enclosure to keep it safe, like a prison, and this is, as the name itself denotes, the safe \clarify{σῶμα} for the soul, until the penalty is paid, and not even a letter needs to be changed. \stephpag{d}
\hermogenesspeaks
I think, Socrates, enough has been said about these words; but might we not consider the names of the gods in the same way in which you were speaking about that of Zeus a few minutes ago, and see what kind of correctness there is in them?
\socratesspeaks
By Zeus, \hermogenesspeaks, we, if we are sensible, must recognize that there is one most excellent kind, since of the gods we know nothing, neither of them nor of their names, whatever they may be, by which they call themselves, for it is clear that they use the true names. But there is a second kind of correctness, \stephpag{e} that we call them, as is customary in prayers, by whatever names and patronymics are pleasing to them, since we know no other. \stephpag{401 a} Now I think that is an excellent custom. So, if you like, let us first make a kind of announcement to the gods, saying that we are not going to investigate about them—for we do not claim to be able to do that—but about men, and let us inquire what thought men had in giving them their names; for in that there is no impiety.
\hermogenesspeaks
I think, Socrates, you are right; let us do as you say. \stephpag{b}
\socratesspeaks
Shall we, then, begin with Hestia, according to custom?
\hermogenesspeaks
That is the proper thing.
\socratesspeaks
Then what would you say the man had in mind who gave Hestia her name?
\hermogenesspeaks
By Zeus, I think that is no more easy question than the other.
\socratesspeaks
At any rate, my dear \hermogenesspeaks, the first men who gave names were no ordinary persons, but high thinkers and great talkers.
\hermogenesspeaks
What then?
\socratesspeaks
I am sure the names were given by men of that kind; and if foreign names are examined, \stephpag{c} the meaning of each of them is equally evident. Take, for instance, that which we call οὐσία \clarify{reality, essence}; some people call it ἐσσία, and still others ὠσία. First, then, in connection with the second of these forms, it is reasonable that the essence of things be called Hestia; and moreover, because we ourselves say of that which partakes of reality ``it is" \clarify{ἔστιν}, the name Hestia would be correct in this connection also; for apparently we also called οὐσία \clarify{reality} ἐσσία in ancient times. And besides, if you consider it in connection with sacrifices, \stephpag{d} you would come to the conclusion that those who established them understood the name in that way; for those who called the essence of things ἐσσίαwould naturally sacrifice to Hestia first of all the gods. Those on the other hand, who say ὠσία would agree, well enough with Heracleitus that all things move and nothing remains still. So they would say the cause and ruler of things was the pushing power \clarify{ὠθοῦν}, wherefore it had been rightly named ὠσία. But enough of this, considering that we know nothing. \stephpag{e} After Hestia it is right to consider Rhea and Cronus. The name of Cronus, however, has already been discussed. But perhaps I am talking nonsense.
\hermogenesspeaks
Why, Socrates?
\socratesspeaks
My friend, I have thought of a swarm of wisdom.
\hermogenesspeaks
What is it? \stephpag{402 a}
\socratesspeaks
It sounds absurd, but I think there is some probability in it.
\hermogenesspeaks
What is this probability?
\socratesspeaks
I seem to have a vision of Heracleitus saying some ancient words of wisdom as old as the reign of Cronus and Rhea, which Homer said too.
\hermogenesspeaks
What do you mean by that?
\socratesspeaks
Heracleitus says, you know, that all things move and nothing remains still, and he likens the universe to the current of a river, saying that you cannot step twice into the same stream.
\hermogenesspeaks
True. \stephpag{b}
\socratesspeaks
Well, don't you think he who gave to the ancestors of the other gods the names ``Rhea" and ``Cronus" had the same thought as Heracleitus? Do you think he gave both of them the names of streams merely by chance? Just so Homer, too, says—``Ocean the origin of the gods, and their mother Tethys;
"Hom. Il. 14.201, 302 and I believe Hesiod says that also. Orpheus, too, says—``Fair-flowing Ocean was the first to marry,
" \stephpag{c}
``and he wedded his sister Tethys, daughter of his mother.
"
Orpheus Fr
See how they agree with each other and all tend towards the doctrine of Heracleitus.
\hermogenesspeaks
I think there is something in what you say, Socrates; but I do not know what the name of Tethys means.
\socratesspeaks
Why, the name itself almost tells that it is the name of a spring somewhat disguised; for that which is strained \clarify{διαττώμενον) \stephpag{d} and filtered \clarify{ἠθούμενον} represents a spring, and the name Tethys is compounded of those two words.
\hermogenesspeaks
That is very neat, Socrates.
\socratesspeaks
Of course it is. But what comes next? Zeus we discussed before.
\hermogenesspeaks
Yes.
\socratesspeaks
Let us, then, speak of his brothers, Poseidon and Pluto, including also the other name of the latter.
\hermogenesspeaks
By all means.
\socratesspeaks
I think Poseidon's name was given by him who first applied it, \stephpag{e} because the power the sea restrained him as he was walking and hindered his advance; it acted as a bond \clarify{δεσμός} of his feet \clarify{ποδῶν}. So he called the lord of this power Poseidon, regarding him as a foot-bond \clarify{ποσί-δεσμον}. The e is inserted perhaps for euphony. But possibly that may not be right; possibly two lambdas were originally pronounced instead of the sigma, because the god knew \clarify{εἰδότος} many \clarify{πολλά} things. \stephpag{403 a} Or it may be that from his shaking he was called the Shaker \clarify{ὁ σείων}, and the pi and delta are additions. As for Pluto, he was so named as the giver of wealth \clarify{πλοῦτος}, because wealth comes up from below out of the earth. And Hades—I fancy most people think that this is a name of the Invisible \clarify{ἀειδής}, so they are afraid and call him Pluto. \stephpag{b}
\hermogenesspeaks
And what do you think yourself, Socrates?
\socratesspeaks
I think people have many false notions about the power of this god, and are unduly afraid of him. They are afraid because when we are once dead we remain in his realm for ever, and they are also terrified because the soul goes to him without the covering of the body. But I think all these facts, and the office and the name of the god, point in the same direction.
\hermogenesspeaks
How so? \stephpag{c}
\socratesspeaks
I will tell you my own view. Please answer this question: Which is the stronger bond upon any living being to keep him in any one place, desire, or compulsion?
\hermogenesspeaks
Desire, Socrates, is much stronger.
\socratesspeaks
Then do you not believe there would be many fugitives from Hades, if he did not bind with the strongest bond those who go to him there?
\hermogenesspeaks
Of course there would.
\socratesspeaks
Apparently, then, if he binds them with the strongest bond, he binds them by some kind of desire, not by compulsion.
\hermogenesspeaks
Yes, that is plain.
\socratesspeaks
There are many desires, are there not?
\hermogenesspeaks
Yes. \stephpag{d}
\socratesspeaks
Then he binds with the desire which is the strongest of all, if he is to restrain them with the strongest bond.
\hermogenesspeaks
Yes.
\socratesspeaks
And is there any desire stronger than the thought of being made a better man by association with some one?
\hermogenesspeaks
No, by Zeus, Socrates, there certainly is not.
\socratesspeaks
Then, \hermogenesspeaks, we must believe that this is the reason why no one has been willing to come away from that other world, not even the Sirens, but they and all others have been overcome by his enchantments, \stephpag{e} so beautiful, as it appears, are the words which Hades has the power to speak; and from this point of view this god is a perfect sophist and a great benefactor of those in his realm, he who also bestows such great blessings upon us who are on earth; such abundance surrounds him there below, and for this reason he is called Pluto. Then, too, he refuses to consort with men while they have bodies, but only accepts their society \stephpag{404 a} when the soul is pure of all the evils and desires of the body. Do you not think this shows him to be a philosopher and to understand perfectly that under these conditions he could restrain them by binding them with the desire of virtue, but that so long as they are infected with the unrest and madness of the body, not even his father Cronus could hold them to himself, though he bound them with his famous chains?
\hermogenesspeaks
There seems to be something in that, Socrates. \stephpag{b}
\socratesspeaks
And the name ``Hades" is not in the least derived from the invisible \clarify{ἀειδές}, but far more probably from knowing \clarify{εἰδέναι} all noble things, and for that reason he was called Hades by the lawgiver.
\hermogenesspeaks
Very well; what shall we say of Demeter, Hera, Apollo, Athena, Hephaestus, Ares, and the other gods
\socratesspeaks
Demeter appears to have been called Demeter, because like a mother \clarify{μήτηρ} she gives the gift of food, \stephpag{c} and Hera is a lovely one \clarify{ἐρατή}, as indeed, Zeus is said to have married her for love. But perhaps the lawgiver had natural phenomena in mind, and called her Hera \clarify{Ἥρα} as a disguise for ἀήρ \clarify{air}, putting the beginning at the end. You would understand, if you were to repeat the name Hera over and over. And Pherephatta!—How many people fear this name, and also Apollo! I imagine it is because they do not know about correctness of names. You see they change the name to Phersephone and its aspect frightens them. But really the name indicates that the goddess is wise; \stephpag{d} for since things are in motion \clarify{φερόμενα}, that which grasps \clarify{ἐφαπτόμενον} and touches \clarify{ἐπαφῶν} and is able to follow them is wisdom. Pherepapha, or something of that sort, would therefore be the correct name of the goddess, because she is wise and touches that which is in motion \clarify{ἐπαφὴ τοῦ φερομένου)—and this is the reason why Hades, who is wise, consorts with her, because she is wise—but people have altered her name, attaching more importance to euphony than to truth, and they call her Pherephatta. Likewise in the case of Apollo, \stephpag{e} as I say, many people are afraid because of the name of the god, thinking that it has some terrible meaning. Have you not noticed that?
\hermogenesspeaks
Certainly; what you say is true.
\socratesspeaks
But really the name is admirably appropriate to the power of the god.
\hermogenesspeaks
How is that?
\socratesspeaks
I will try to tell you what I think about it; \stephpag{405 a} for no single name could more aptly indicate the four functions of the god, touching upon them all and in a manner declaring his power in music, prophecy, medicine, and archery.
\hermogenesspeaks
Go on; you seem to imply that it is a remarkable name.
\socratesspeaks
His name and nature are in harmony; you see he is a musical god. For in the first place, purification and purgations used in medicine and in soothsaying, and fumigations with medicinal and magic drugs, \stephpag{b} and the baths and sprinklings connected with that sort of thing all have the single function of making a man pure in body and soul, do they not?
\hermogenesspeaks
Certainly.
\socratesspeaks
But this is the god who purifies and washes away \clarify{ἀπαλοούων} and delivers \clarify{ἀπολύων} from such evils, is he not?
\hermogenesspeaks
Certainly.
\socratesspeaks
With reference, then, to his acts of delivering and his washings, \stephpag{c} as being the physician of such diseases, he might properly be called Apoluon \clarify{ἀπαλούων, the washer}, and with reference to soothsaying and truth and simplicity—for the two are identical—he might most properly be called by the name the Thessalians use; for all Thessalians call the god Aplun. And because he is always by his archery controller of darts \clarify{βολῶν} he is ever darting \clarify{ἀεὶ βάλλων}. And with reference to music we have to understand that alpha often signifies ``together," and here it denotes moving together in the heavens about the poles, as we call them, and harmony in song, \stephpag{d} which is called concord; for, as the ingenious musicians and astronomers tell us, all these things move together by a kind of harmony. And this god directs the harmony, making them all move together, among both gods and men; and so, just as we call the ὁμοκέλευθον \clarify{him who accompanies}, and ὁμόκοιτιν \clarify{bedfellow}, by changing the ὁμο to alpha, ἀκόλουθον and ἄκοιτιν, so also we called him Apollo who was Homopolo, \stephpag{e} and the second lambda was inserted, because without it the name sounded of disaster \clarify{ἀπολῶ, ἀπόλωλα, etc.}. Even as it is, some have a suspicion of this, because they do not properly regard the force of the name, and therefore they fear it, thinking that it denotes some kind of ruin. But in fact, as was said, \stephpag{406 a} the name touches upon all the qualities of the god, as simple, ever-darting, purifying, and accompanying. The Muses and music in general are named, apparently, from μῶσθαι, searching, and philosophy; and Leto from her gentleness, because whatever is asked of her, she is willing \clarify{ἐθελήμων}. But perhaps her name is Letho, as she is called by many foreigners; and those who call her by that name seem to do so \stephpag{b} on account of the mild and gentle \clarify{λεῖον, Ληθώ} kindness of her character. Artemis appears to get her name from her healthy \clarify{ἀρτεμές} and well-ordered nature, and her love of virginity; or perhaps he who named her meant that she is learned in virtue \clarify{ἀρετή}, or possibly, too, that she hates sexual intercourse \clarify{ἄροτον μισεῖ} of man and woman; or he who gave the goddess her name may have given it for any or all of these reasons.
\hermogenesspeaks
What of Dionysus and Aphrodite?
\socratesspeaks
You ask great things of me, son of Hipponicus. You see there is both a serious and a facetious account of the form \stephpag{c} of the name of these deities. You will have to ask others for the serious one; but there is nothing to hinder my giving you the facetious account, for the gods also have a sense of humor. Dionysus, the giver \clarify{διδούς} of wine \clarify{οἶνος}, might be called in jest Didoinysus, and wine, because it makes most drinkers think \clarify{οἴεσθαι} they have wit \clarify{νοῦς} when they have not, might very justly be called Oeonus \clarify{οἰόνους}. As for Aphrodite, we need not oppose Hesiod; we can accept his derivation of the name \stephpag{d} from her birth out of the foam \clarify{ἀφροῦ}.
\hermogenesspeaks
But surely you, as an Athenian, will not forget Athena, nor Hephaestus and Ares.
\socratesspeaks
That is not likely.
\hermogenesspeaks
No.
\socratesspeaks
It is easy to tell the reason of one of her two names.
\hermogenesspeaks
What name?
\socratesspeaks
We call her Pallas, you know.
\hermogenesspeaks
Yes, of course.
\socratesspeaks
Those of us are right, I fancy, \stephpag{e} who think this name is derived from armed dances, for lifting oneself or anything else from the ground or \stephpag{407 a} in the hands is called shaking \clarify{πάλλειν} and being shaken, or dancing and being danced.
\hermogenesspeaks
Yes, certainly.
\socratesspeaks
So that is the reason she is called Pallas.
\hermogenesspeaks
And rightly called so. But what can you say of her other name?
\socratesspeaks
You mean Athena?
\hermogenesspeaks
Yes.
\socratesspeaks
That is a weightier matter, my friend. The ancients seem to have had the same belief about Athena as the interpreters of Homer have now; \stephpag{b} for most of these, in commenting on the poet, say that he represents Athena as mind \clarify{νοῦς} and intellect \clarify{διάνοια}; and the maker of her name seems to have had a similar conception of her, but he gives her the still grander title of ``mind of God" ἡ θεοῦ νόησις, seeming to say that she is a ἁ θεονόα; here he used the alpha in foreign fashion instead of eta, and dropped out the iota and sigma. But perhaps that was not his reason; he may have called her Theonoe because she has unequalled knowledge of divine things \clarify{τὰ θεῖα νοοῦσα}. Perhaps, too, he may have wished to identify the goddess with wisdom of character \clarify{ἐν ἤθει νόησις) \stephpag{c} by calling her Ethonoe; and then he himself or others afterwards improved the name, as they thought, and called her Athenaa.
\hermogenesspeaks
And how do you explain Hephaestus?
\socratesspeaks
You ask about ``the noble master of light"?
\hermogenesspeaks
To be sure.
\socratesspeaks
Hephaestus is Phaestus, with the eta added by attraction; anyone could see that, I should think.
\hermogenesspeaks
Very likely, unless some other explanation occurs to you, as it probably will.
\socratesspeaks
To prevent that, ask about Ares.
\hermogenesspeaks
I do ask. \stephpag{d}
\socratesspeaks
Ares, then, if you like, would be named for his virility and courage, or for his hard and unbending nature, which is called ἄρρατον; so Ares would be in every way a fitting name for the god of war.
\hermogenesspeaks
Certainly.
\socratesspeaks
For God's sake, let us leave the gods, as I am afraid to talk about them; but ask me about any others you please, ``that you may see what" Euthyphro's ``horses are."10 \stephpag{e}
\hermogenesspeaks
I will do so, but first one more god. I want to ask you about Hermes, since Cratylus says I am not \hermogenesspeaks \clarify{son of Hermes}. Let us investigate the name of Hermes, to find out whether there is anything in what he says.
\socratesspeaks
Well then, this name ``Hermes" seems to me to have to do with speech; he is an interpreter \clarify{ἡρμηνεύς} and a messenger, \stephpag{408 a} is wily and deceptive in speech, and is oratorical. All this activity is concerned with the power of speech. Now, as I said before, εἴρειν denotes the use of speech; moreover, Homer often uses the word ἐμήσατο, which means ``contrive." From these two words, then, the lawgiver imposes upon us the name of this god who contrived speech and the use of speech—εἴρεινmeans ``speak"— \stephpag{b} and tells us: ``Ye human beings, he who contrived speech \clarify{εἴρειν ἐμήσατο} ought to be called Eiremes by you." We, however, have beautified the name, as we imagine, and call him Hermes. Iris also seems to have got her name from εἴρειν, because she is a messenger.
\hermogenesspeaks
By Zeus, I believe Cratylus was right in saying I was not \hermogenesspeaks; I certainly am no good contriver of speech.
\socratesspeaks
And it is reasonable, my friend, that Pan is the double-natured son of Hermes. \stephpag{c}
\hermogenesspeaks
How is that?
\socratesspeaks
You know that speech makes all things \clarify{πᾶν} known and always makes them circulate and move about, and is twofold, true and false.
\hermogenesspeaks
Certainly.
\socratesspeaks
Well, the true part is smooth and divine and dwells aloft among the gods, but falsehood dwells below among common men, is rough and like the tragic goat11; for tales and falsehoods are most at home there, in the tragic life.
\hermogenesspeaks
Certainly.
\socratesspeaks
Then Pan, who declares and always moves \clarify{ἀεὶ πολῶν} all, is rightly called goat-herd \clarify{αἰπόλος}, \stephpag{d} being the double-natured son of Hermes, smooth in his upper parts, rough and goat-like in his lower parts. And Pan, if he is the son of Hermes, is either speech or the brother of speech, and that brother resembles brother is not at all surprising. But, as I said, my friend, let us get away from the gods.
\hermogenesspeaks
From such gods as those, if you like, Socrates; but why should you not tell of another kind of gods, such as sun, moon, stars, earth, \stephpag{e} ether, air, fire, water, the seasons, and the year?
\socratesspeaks
You are imposing a good many tasks upon me; however, if it will give you pleasure, I am willing.
\hermogenesspeaks
It will give me pleasure.
\socratesspeaks
What, then, do you wish first? Shall we discuss the sun \clarify{Ἥλιος}, as you mentioned it first?
\hermogenesspeaks
By all means.
\socratesspeaks
I think it would be clearer \stephpag{409 a} if we were to use the Doric form of the name. The Dorians call it Ἅλιος. Now ἅλιοςmight be derived from collecting \clarify{ἁλίζειν} men when he rises, or because he always turns \clarify{ἀεὶ εἱλεῖν} about the earth in his course, or because he variegates the products of the earth, for variegate is identical with αἰολλεῖν.
\hermogenesspeaks
And what of the moon, Selene?
\socratesspeaks
That name appears to put Anaxagoras in an uncomfortable position.
\hermogenesspeaks
How so?
\socratesspeaks
Why, it seems to have anticipated by many years the recent doctrine of Anaxagoras, \stephpag{b} that the moon receives its light from the sun.
\hermogenesspeaks
How is that?
\socratesspeaks
Σέλας \clarify{gleam} and φῶς \clarify{light} are the same thing.
\hermogenesspeaks
Yes.
\socratesspeaks
Now the light is always new and old about the moon, if the Anaxagoreans are right; for they say the sun, in its continuous course about the moon, always sheds new light upon it, and the light of the previous month persists.
\hermogenesspeaks
Certainly.
\socratesspeaks
The moon is often called Σελαναία.
\hermogenesspeaks
Certainly.
\socratesspeaks
Because it has always a new and old gleam \clarify{σέλα νέον τε καὶ ἕνον) \stephpag{c} the very most fitting name for it would be Σελαενονεοάεια, which has been compressed into Σελαναία.
\hermogenesspeaks
That is a regular opera bouffe name, Socrates. But what have you to say of the month \clarify{μήν} and the stars?
\socratesspeaks
The word ``month" \clarify{μείς} would be properly pronounced μείης, from μειοῦσθαι, ``to grow less," and I think the stars \clarify{ἄστερα} get their name from ἀστραπή \clarify{lightning}. But ἀστραπή, because it turns our eyes upwards \clarify{τὰ ὦπαἀναστρέθει}, would be called ἀναστρωπή, which is now pronounced more prettily ἀστραπή.
\hermogenesspeaks
And what of πῦρ \clarify{fire} and ὕδωρ \clarify{water}? \stephpag{d}
\socratesspeaks
Πῦρ is too much for me. It must be that either the muse of Euthyphro has deserted me or this is a very difficult word. Now just note the contrivance I introduce in all cases like this which are too much for me.
\hermogenesspeaks
What contrivance?
\socratesspeaks
I will tell you. Answer me; can you tell the reason of the word πῦρ?
\hermogenesspeaks
Not I, by Zeus.
\socratesspeaks
See what I suspect about it. I know that many Greeks, \stephpag{e} especially those who are subject to the barbarians, have adopted many foreign words.
\hermogenesspeaks
What of that?
\socratesspeaks
If we should try to demonstrate the fitness of those words in accordance with the Greek language, and not in accordance with the language from which they are derived, you know we should get into trouble.
\hermogenesspeaks
Naturally. \stephpag{410 a}
\socratesspeaks
Well, this word πῦρ is probably foreign; for it is difficult to connect it with the Greek language, and besides, the Phrygians have the same word, only slightly altered. The same is the case with ὕδωρ \clarify{water}, κύων \clarify{dog}, and many other words.
\hermogenesspeaks
Yes, that is true.
\socratesspeaks
So we must not propose forced explanations of these words, though something might be said about them. I therefore set aside πῦρ and ὕδωρ in this way. \stephpag{b} But is air called ἀήρ because it raises \clarify{αἴρει} things from the earth, or because it is always flowing \clarify{ἀεὶ ῥεῖ}, or because wind arises from its flow? The poets call the winds ἀήτας, ``blasts." Perhaps the poet means to say ``air-flow" \clarify{ἀητόρρουν}, as he might say ``wind-flow" \clarify{πνευματόρρουν}. The word αἴθηρ \clarify{ether} I understand in this way: because it always runs and flows about the air \clarify{ἀεὶ θεῖ περὶ τὸν ἀέρα ῥέον}, it may properly be called ἀειθεήρα. The word γῆ \clarify{earth} shows the meaning better \stephpag{c} in the form γαῖα; for γαῖαis a correct word for ``mother," as Homer says, for he uses γεγάασιν to mean γεγενῆσθαι \clarify{be born}. Well, now what came next?
\hermogenesspeaks
The seasons, Socrates, and the two words for year.
\socratesspeaks
The word ὧραι \clarify{seasons} should be pronounced in the old Attic fashion, ὅραι, if you wish to know the probable meaning; ΗΟΡΑΙ exist to divide winters and summers and winds and the fruits of the earth; and since they divide \clarify{ὁρίζουσι}, they would rightly be called ὅραι. \stephpag{d} The two words for year, ἐνιαυτός and ἔτος, are really one. For that which brings to light within itself the plants and animals, each in its turn, and examines them, is called by some ἐνιαυτός, because of its activity within itself \clarify{ἐν ἑαυτῷ}, and by others ἔτος, because it examines \clarify{ἐτάζει}, just as we saw before that the name of Zeus was divided and some said Δία and others Ζῆνα. The whole phrase is ``that which examines within itself" \clarify{τὸ ἐν ἁυτῷ ἐτάζον}, and this one phrase is divided in speech so that the two words ἐνιαυτός and ἔτος \stephpag{e} are formed from one phrase.
\hermogenesspeaks
Truly, Socrates, you are going ahead at a great rate.
\socratesspeaks
Yes, I fancy I am already far along on the road of wisdom.
\hermogenesspeaks
I am sure you are.
\socratesspeaks
You will be surer presently. \stephpag{411 a}
\hermogenesspeaks
Now after the class of words you, have explained, I should like to examine the correctness of the noble words that relate to virtue, such as wisdom, intelligence, justice, and all the others of that sort.
\socratesspeaks
You are stirring up a mighty tribe of words, my friend; however, since I have put on the lion helmet, I must not play the coward, but must, it seems, examine wisdom, intelligence, thought, knowledge, \stephpag{b} and all the other noble words of which you speak.
\hermogenesspeaks
Certainly we must not stop until that is done.
\socratesspeaks
By dog, I believe I have a fine intuition which has just come to me, that the very ancient men who invented names were quite like most of the present philosophers who always get dizzy as they turn round and round in their search for the nature of things, and then the things seem to them to turn round and round and be in motion. \stephpag{c} They think the cause of this belief is not an affection within themselves, but that the nature of things really is such that nothing is at rest or stable, but everything is flowing and moving and always full of constant motion and generation. I say this because I thought of it with reference to all these words we are now considering.
\hermogenesspeaks
How is that, Socrates?
\socratesspeaks
Perhaps you did not observe that the names we just mentioned are given under the assumption that the things named are moving and flowing and being generated.
\hermogenesspeaks
No, I did not notice that at all. \stephpag{d}
\socratesspeaks
Surely the first one we mentioned is subject to such assumptions.
\hermogenesspeaks
What is the word ?
\socratesspeaks
Wisdom \clarify{φρόνησις}; for it is perception \clarify{νόησις} of motion \clarify{φορᾶς} and flowing \clarify{ῥοῦ}; or it might be understood as benefit \clarify{ὄνησις} of motion \clarify{φορᾶς}; in either case it has to do with motion. And γνώμη \clarify{thought}, if you please, certainly denotes contemplation and consideration of generation \clarify{γονῆς νώμησις}; for to consider is the same as to contemplate. Or, if you please, νόησις \clarify{intelligence} is merely ἕσις \clarify{desire} τοῦ νεοῦ \clarify{of the new}; but that things are new shows that they are always being generated; \stephpag{e} therefore the soul's desire for generation is declared by the giver of the name νεόεσις; for in antiquity the name was not νόησις, but two epsilons had to be spoken instead of the eta. Σωφροσύνη \clarify{self-restraint} is σωτηρία \clarify{salvation} of φρόνησις \clarify{wisdom}, which we have just been discussing. \stephpag{412 a} And ἐπιστήμη \clarify{knowledge} indicates that the soul which is of any account accompanies \clarify{ἕπεται} things in their motion, neither falling behind them nor running in front of them; therefore we ought to insert an epsilon and call it ἐπεϊστήμη. Σύνεσις \clarify{intelligence} in its turn is a kind of reckoning together; when one says συνιέναι \clarify{understand}, the same thing as ἐπίστασθαι is said; \stephpag{b} for συνιέναι means that the soul goes with things. Certainly σοφία \clarify{wisdom} denotes the touching of motion. This word is very obscure and of foreign origin; but we must remember that the poets often say of something which begins to advance ἐσύθη \clarify{it rushed}. There was a famous Laconian whose name was Σοῦς \clarify{Rush}, for this is the Laconian word for rapid motion. Now σοφίαsignifies the touching \clarify{ἐπαφή} of this rapid motion, the assumption being that things are in motion. And the word ἀγαθόν \clarify{good} \stephpag{c} is intended to denote the admirable \clarify{ἀγαστόν} in all nature. For since all things are in motion, they possess quickness and slowness; now not all that is swift, but only part of it, is admirable; this name ἀγαθόν is therefore given to the admirable \clarify{ἀγαστόν} part of the swift \clarify{θοοῦ}.

It is easy to conjecture that the word δικαιασύνη applies to the understanding \clarify{σύνεσις} of the just \clarify{τοῦ διαίον} but the word δίκαιον \clarify{just} is itself difficult. Up to a certain point, you see, many men seem to agree about it, but beyond that they differ. \stephpag{d} For those who think the universe is in motion believe that the greater part of it is of such a nature as to be a mere receptacle, and that there is some element which passes through all this, by means of which all created things are generated. And this element must be very rapid and very subtle; for it could not pass through all the universe unless it were very subtle, so that nothing could keep it out, and it must be very swift, so that all other things are relatively at rest. Since, then, it superintends and passes through \clarify{διαϊόν} all other things, \stephpag{e} this is rightly called by the name δίκαιον, the sound of the kappa being added merely for the sake of euphony. Up to this point, as I said just now, many men agree about justice \clarify{δίκαιον}; \stephpag{413 a} and I, \hermogenesspeaks, being very much in earnest about it, have persistently asked questions and have been told in secret teachings that this is justice, or the cause—for that through which creation takes place is a cause—and some one told me that it was for this reason rightly called Zeus \clarify{Δία}. But when, after hearing this, I nevertheless ask them quietly, ``What then, my most excellent friend, if this is true, is justice?" they think I am asking too many questions and am leaping over the trenches.12 \stephpag{b} They say I have been told enough; they try to satisfy me by saying all sorts of different things, and they no longer agree. For one says the sun is justice, for the sun alone superintends all things, passing through and burning \clarify{διαϊόντα καὶ καίοντα} them. Then when I am pleased and tell this to some one, thinking it is a fine answer, he laughs at me and asks if I think there is no justice among men when the sun has set. So I beg him to tell me what he thinks it is, \stephpag{c} and he says ``Fire." But this is not easy to understand. He says it is not actual fire, but heat in the abstract that is in the fire. Another man says he laughs at all these notions, and that justice is what Anaxagoras says it is, mind; for mind, he says, is ruled only by itself, is mixed with nothing, orders all things, and passes through them. Then, my friend, I am far more perplexed than before I undertook to learn about the nature of justice. \stephpag{d} But I think the name—and that was the subject of our investigation—was given for the reasons I have mentioned.
\hermogenesspeaks
I think, Socrates, you must have heard this from some one and are not inventing it yourself.
\socratesspeaks
And how about the rest of my talk?
\hermogenesspeaks
I do not at all think you had heard that.
\socratesspeaks
Listen then; perhaps I may deceive you into thinking that all I am going to say is my own. What remains to consider after justice? I think we have not yet discussed courage. \stephpag{e} It is plain enough that injustice \clarify{ἀδικία} is really a mere hindrance of that which passes through \clarify{τοῦ διαϊόντος, but the word ἀδρεία \clarify{courage} implies that courage got its name in battle, and if the universe is flowing, a battle in the universe can be nothing else than an opposite current or flow \clarify{ῥοή}. Now if we remove the delta from the word ἀνδρεία, the word ἀνρεία signifies exactly that activity. Of course it is clear that not the current opposed to every current is courage, but only that opposed to the current which is contrary to justice; \stephpag{414 a} for otherwise courage would not be praised. The words ἄρρεν \clarify{male} and ἀνήρ \clarify{man} refer, like ἀνδρεία, to the upward \clarify{ἄνω} current or flow. The word γυνή \clarify{woman} seems to me to be much the same as γονή \clarify{birth}. I think θῆλυ \clarify{female} is derived from θηλή \clarify{teat}; and is not θηλή, \hermogenesspeaks, so called because it makes things flourish \clarify{τεθηλέναι}, like plants wet with showers?
\hermogenesspeaks
Very likely, Socrates.
\socratesspeaks
And again, the word θάλλειν \clarify{flourish} seems to me to figure the rapid and sudden growth of the young. \stephpag{b} Something of that sort the namegiver has reproduced in the name, which he compounded of θεῖν \clarify{run} and ἅλλεσθαι \clarify{jump}. You do not seem to notice how I rush along outside of the race-course, when I get on smooth ground. But we still have plenty of subjects left which seem to be serious.
\hermogenesspeaks
True.
\socratesspeaks
One of which is to see what the word τέχνη \clarify{art, science} means.
\hermogenesspeaks
Certainly.
\socratesspeaks
Does not this denote possession of mind, if you remove the tau and insert omicron between the chi and the nu \stephpag{c} and the nu and the eta \clarify{making ἐχονόη}?
\hermogenesspeaks
It does it very poorly, Socrates.
\socratesspeaks
My friend, you do not bear in mind that the original words have before now been completely buried by those who wished to dress them up, for they have added and subtracted letters for the sake of euphony and have distorted the words in every way for ornamentation or merely in the lapse of time. Do you not, for instance, think it absurd that the letter rho is inserted in the word κάαπτρον \clarify{mirror}? \stephpag{d} I think that sort of thing is the work of people who care nothing for truth, but only for the shape of their mouths; so they keep adding to the original words until finally no human being can understand what in the world the word means. So the sphinx, for instance, is called sphinx, instead of phix, and there are many other examples.
\hermogenesspeaks
Yes, that is true, Socrates.
\socratesspeaks
And if we are permitted to insert and remove any letters we please in words, it will be perfectly easy to fit any name to anything. \stephpag{e}
\hermogenesspeaks
True.
\socratesspeaks
Yes, quite true. But I think you, as a wise director, must observe the rule of moderation and probability.
\hermogenesspeaks
I should like to do so.
\socratesspeaks
And I, too, \hermogenesspeaks. \stephpag{415 a} But do not, my friend, demand too much precision, lest you ````enfeeble me of my sight.""Hom. Il. 6.265 For now that τέχνη \clarify{art} is disposed of, I am nearing the loftiest height of my subject, when once we have investigated μηχανή\clarify{contrivance}. For I think μηχανή signifies ἄνειν ἐπὶ πολύ \clarify{much accomplishment}; for μῆκος \clarify{length} has about the same meaning as τὸ πολύ \clarify{much}, and the name μηχανή is composed of these two, μῆκος and ἄνειν. But, as I was just saying, we must go on to the loftiest height of our subject; we must search for the meaning of the words ἀρετή \clarify{virtue} and κακία \clarify{wickedness}. Now one of them I cannot yet see; \stephpag{b} but the other seems to be quite clear, since it agrees with everything we have said before. For inasmuch as all things are in motion, everything that moves badly \clarify{κακῶς ἰόν} would be evil \clarify{κακία}; and when this evil motion in relation to its environment exists in the soul, it receives the general name κακία \clarify{evil} in the special sense of wickedness. But the nature of evil motion \clarify{κακῶς ἰέναι} is made clear, I think, also in the word δειλία\clarify{cowardice}, which we have not yet discussed. We passed it by, \stephpag{c} when we ought to have examined it after ἀνδρεία \clarify{courage}; and I fancy we passed over a good many other words. Now the meaning of δειλία is ``a strong bond of the soul"; for λίαν \clarify{excessively} is, in a way, expressive of strength; so δειλία would be the excessive or greatest bond \clarify{δεσμός, δεῖν} of the soul; and so, too, ἀπορία \clarify{perplexity} is an evil, as is everything, apparently, which hinders motion and progress \clarify{πορεύεσθαι}. This, then, seems to be the meaning of evil motion \clarify{κακῶςἰέναι}, that advance is halting and impeded; and the soul that is infected by it becomes filled with wickedness \clarify{κακία}. If these are the reasons for the name of wickedness, virtue \clarify{ἀρετή} would be the opposite of this; it would signify first ease of motion, \stephpag{d} and secondly that the flow of the good soul is always unimpeded, and therefore it has received this name, which designates that which always flows \clarify{ἀεὶ ῥέον} without let or hindrance. It is properly called ἀειρειτή, or perhaps also αἱρετή, indicating that this condition is especially to be chosen; but it has been compressed and is pronounced ἀρετή. Perhaps you will say this is another invention of mine; but I say if what I said just now about κακία is right, \stephpag{e} this about the name of ἀρετή is right too. \stephpag{416 a}
\hermogenesspeaks
But what is the meaning of the word κακόν which you used in many of your derivations?
\socratesspeaks
By Zeus, I think it is a strange word and hard to understand; so I apply to it that contrivance of mine.
\hermogenesspeaks
What contrivance?
\socratesspeaks
The claim of foreign origin, which I advance in this case as in those others.
\hermogenesspeaks
Well, probably you are right. But, if you please, let us drop these words and try to discover the reasons for the words καλόν \clarify{beautiful, noble} and αἰσχρόν \clarify{base}.
\socratesspeaks
I think the meaning of αἰσχρόν is clear, \stephpag{b} and this also agrees with what has been said before. For the giver of names appears to me throughout to denounce that which hinders and restrains things from flowing, and in this instance he gave to that which always restrains the flow \clarify{ἀεὶ ἴσχει τὸν ῥοῦν} this name ἀεισχοροῦν, which is now compressed and pronounced αἰσχρόν.
\hermogenesspeaks
What about καλόν?
\socratesspeaks
That is harder to understand, and yet it expresses its meaning: it has been altered merely in accent and in the length of the O. \stephpag{c}
\hermogenesspeaks
How is that?
\socratesspeaks
I think this word denotes intellect.
\hermogenesspeaks
What do you mean?
\socratesspeaks
Why, what do you think is the cause why anything is called by a name? Is it not the power which gave the name?
\hermogenesspeaks
Why, certainly.
\socratesspeaks
And is not that power the intellect either of gods or of men or both?
\hermogenesspeaks
Yes.
\socratesspeaks
Are not that which called things by name and that which calls them by name \clarify{τὸ καλοῦν} the same thing, namely intellect?
\hermogenesspeaks
Yes, clearly.
\socratesspeaks
And are not all works which are done by mind and intelligence worthy of praise, and those that are not done by them worthy of blame?
\hermogenesspeaks
Certainly. \stephpag{d}
\socratesspeaks
Does not the medical power perform medical works and the power of carpentry works of carpentry? Do you agree to that?
\hermogenesspeaks
I agree.
\socratesspeaks
And the beautiful performs beautiful works?
\hermogenesspeaks
It must do so.
\socratesspeaks
And the beautiful is, we say, intellect?
\hermogenesspeaks
Certainly.
\socratesspeaks
Then this name, the beautiful, is rightly given to mind, since it accomplishes the works which we call beautiful and in which we delight.
\hermogenesspeaks
Evidently. \stephpag{e}
\socratesspeaks
What further words of this sort are left for us?
\hermogenesspeaks
Those that are related to the good and the beautiful, \stephpag{417 a} such as συμφέροντα \clarify{advantageous}, λυσιτελοῦντα\clarify{profitable}, ὠφέλιμα \clarify{useful}, κερδαλέα \clarify{gainful}, and their opposites.
\socratesspeaks
You might by this time be able to find the meaning of συμφέροντα by yourself in the light of the previous explanations, for it appears to be own brother to ἐπιστήμη. It means nothing else but the motion \clarify{φορά} of the soul in company with the world, and naturally things which are done by such a power are called συμφέροντα and σύμφορα because they are carried round with \clarify{συμπεριφέρεσθαι} the world. But κερδαλέον is from κέρδος \clarify{gain}. \stephpag{b} If you restore nu in the word κέρδος in place of the delta, the meaning is plain; it signifies good, but in another way. Because it passes through and is mingled \clarify{κεράννυται} with all things, he who named it gave it this name which indicates that function; but he inserted a delta instead of nu and said κέρδος.
\hermogenesspeaks
And what is λυσιτελοῦν?
\socratesspeaks
I do not think, \hermogenesspeaks, the name-giver gives the meaning to λυσιτελοῦν which it has in the language of tradesfolk, when profit sets free \clarify{ἀπολύει} the sum invested, \stephpag{c} but he means that because it is the swiftest thing in the world it does not allow things to remain at rest and does not allow the motion to come to any end \clarify{τέλος} of movement or to stop or pause, but always, if any end of the motion is attempted, it sets it free, making it unceasing and immortal. It is in this sense, I think, that the good is dubbed λυσιτελοῦν, for it frees \clarify{λύει} the end \clarify{τέλος} of the motion. But the word ὠφέλιμον is a foreign one, which Homer often uses in the verbal form ὀφέλλειν. This is a synonym of ``increase" and ``create." \stephpag{d}
\hermogenesspeaks
What shall be our explanations of the opposites of these?
\socratesspeaks
Those of them that are mere negatives, need, I think, no discussion.
\hermogenesspeaks
Which are those?
\socratesspeaks
Disadvantageous, useless, unprofitable, and ungainful.
\hermogenesspeaks
True.
\socratesspeaks
But βλαβερόν \clarify{harmful} and ζημιῶδες \clarify{hurtful} do need it.
\hermogenesspeaks
Yes.
\socratesspeaks
And βλαβερόν means that which harms \clarify{βλάπτον} the flow \clarify{ῥοῦν}; \stephpag{e} but βλάπτον means ``wishing to fasten" \clarify{ἅπτειν}, and ἅπτειν is the same thing as δεῖν \clarify{bind}, which the name-giver constantly finds fault with. Now τὸβουλόμενον ἅπτειν ῥοῦν \clarify{that which wishes to fasten the flow} would most correctly be called βουλαπτεροῦν, but is called βλαβερόν merely, as I think, to make it prettier.
\hermogenesspeaks
Elaborate names these are, Socrates, that result from your method. Just now, \stephpag{418 a} when you pronounced βουλαπτεροῦν, you looked as if you had made up your mouth to whistle the flute-prelude of the hymn to Athena.
\socratesspeaks
Not I, \hermogenesspeaks, am responsible, but those who gave the name.
\hermogenesspeaks
True. Well, what is the origin of ζημιῶδες?
\socratesspeaks
What can the origin of ζημιῶδες be? See, \hermogenesspeaks, how true my words are when I say that by adding and taking away letters people alter the sense of words so that even by very slight changes they sometimes make them mean the opposite of what they meant before; as, for instance, \stephpag{b} in the case of the word δέον \clarify{obligation, right}, for that just occurred to me and I was reminded of it by what I was going to say to you, that this fine modern language of ours has turned δέον and also ζημιῶδες round, so that each has the opposite of its original meaning, whereas the ancient language shows clearly the real sense of both words.
\hermogenesspeaks
What do you mean?
\socratesspeaks
I will tell you. You know that our ancestors made good use of the sounds of iota and delta, \stephpag{c} and that is especially true of the women, who are most addicted to preserving old forms of speech. But nowadays people change iota to eta or epsilon, and delta to zeta, thinking they have a grander sound.
\hermogenesspeaks
How is that?
\socratesspeaks
For instance, in the earliest times they called day ἱμέρα, others said ἑμέρα, and now they say ἡμέρα.
\hermogenesspeaks
That is true.
\socratesspeaks
Only the ancient word discloses the intention of the name-giver, don't you know? For day comes out of darkness to men; they welcome it and long \clarify{ἱμείρουσι} for it, \stephpag{d} and so they called it ἱμέρα.
\hermogenesspeaks
That is clear.
\socratesspeaks
But now ἡμέρα is masquerading so that you could not guess its meaning. Why, some people think day is called ἡμέρα because it makes things gentle \clarify{ἥμερα}.
\hermogenesspeaks
I believe they do.
\socratesspeaks
And you know the ancients called ζυγόν \clarify{yoke} δυογόν.
\hermogenesspeaks
Certainly.
\socratesspeaks
And ζυγόν conveys no clear meaning, \stephpag{e} but the name δυογόν is quite properly given to that which binds two together for the purpose of draught; now, however, we say ζυγόν. There are a great many other such instances.
\hermogenesspeaks
Yes, that is plain.
\socratesspeaks
Similarly the word δέον \clarify{obligation} at first, when spoken in this way, denotes the opposite of all words connected with the good; for although it is a form of good, δέον seems to be a bond \clarify{δεσμός} and hindrance of motion, own brother, as it were, toβλαβερόν.
\hermogenesspeaks
Yes, Socrates, it certainly does seem so.
\socratesspeaks
But it does not, if you employ the ancient word, \stephpag{419 a} which is more likely to be right than the present one. You will find that it agrees with the previous words for ``good," if instead of the epsilon you restore the iota, as it was in old times for διόν \clarify{going through}, not δέον, signifies good, which the name-giver praises. And so the giver of names does not contradict himself, but δέον \clarify{obligation, right}, ὠφέλιμον \clarify{useful}, λυσιτελοῦν \clarify{profitable}, κερδαλέον\clarify{gainful}, ἀγαθόν \clarify{good}, ξυμφέρον \clarify{advantageous}, and εὔπορον \clarify{prosperous}, are plainly identical, signifying under different names the principle of arrangement and motion which has constantly been praised, \stephpag{b} whereas the principle of constraint and bondage is found fault with. And likewise in the case of ζημιῶδες, if you restore the ancient delta in place of the zeta, you will see that the name, pronounced δημιῶδες, was given to that which binds motion \clarify{δοῦντι τὸ ἰόν}.
\hermogenesspeaks
What of ἡδονή \clarify{pleasure} and λύπη \clarify{pain} and ἐπιθυμία \clarify{desire}, and the like, Socrates?
\socratesspeaks
I do not think they are at all difficult, \hermogenesspeaks, for ἡδονή appears to have this name because it is the action that tends towards advantage \clarify{ἡ πρὸς τὴν ὄνησιν τείνουσα}; the delta is inserted, so that we say ἡδονή instead of ἡονή. \stephpag{c} Λύπη appears to have received its name from the dissolution \clarify{διάλυσις} of the body which takes place through pain. Ἀνία \clarify{sorrow} is that which hinders motion \clarify{ἰέναι}. Ἀλγηδών \clarify{distress} is, I think, a foreign word, derived from ἀλγεινός \clarify{distressing}. Ὀδύνη \clarify{grief} appears to be so called from the putting on of pain \clarify{τῆςἐνδύσεως τῆς λύπης}. Ἀχθηδών \clarify{vexation} has a name, as anyone can see, made in the likeness of the weight \clarify{ἄχθος, burden} which vexation imposes upon motion. Χαρά \clarify{joy} seems to have its name from the plenteous diffusion \clarify{διάχυσις} of the flow of the soul. \stephpag{d} Τέρψις \clarify{delight} is from τερπνόν \clarify{delightful}; and τερπνόν is called from the creeping \clarify{ἕρψις} of the soul, which is likened to a breath \clarify{πνοή}, and would properly be called ἕρπνουν, but the name has been changed in course of time to τερπνόν. Εὐφροσύνη\clarify{mirth} needs no explanation, for it is clear to anyone that from the motion of the soul in harmony \clarify{εὖ} with the universe, it received the name εὐφεροσύνη, as it rightfully is; but we call it ευφροσύνη. \stephpag{e} Nor is there any difficulty about ἐπιθυμία \clarify{desire}, for this name was evidently given to the power that goes \clarify{ἰοῦσα} into the soul \clarify{θυμός}. And θυμός has its name from the raging \clarify{θύσις} and boiling of the soul. The name ἵμερος \clarify{longing} was given to the stream \clarify{ῥοῦς} which most draws the soul; \stephpag{420 a} for because it flows with a rush \clarify{ἱέμενος} and with a desire for things and thus draws the soul on through the impulse of its flowing, all this power gives it the name of ἵμερος. And the word πόθος\clarify{yearning} signifies that it pertains not to that which is present, but to that which is elsewhere \clarify{ἄλλοθί που} or absent, and therefore the same feeling which is called ἵμερος when its object is present, is called πόθος when it is absent. And ἔρως \clarify{love} is so called because it flows in \clarify{ἐσρεῖ} from without, and this flowing is not inherent in him who has it, \stephpag{b} but is introduced through the eyes; for this reason it was in ancient times called ἔσρος, from ἐσρεῖν—for we used to employ omicron instead of omega—but now it is called ἔρως through the change of omicron to omega. Well, what more is there that you want to examine?
\hermogenesspeaks
What is your view about δόξα \clarify{opinion} and the like?
\socratesspeaks
Δόξα is derived either from the pursuit \clarify{δίωξις} which the soul carries on as it pursues the knowledge of the nature of things, or from the shooting of the bow \clarify{τόξον}; the latter is more likely; at any rate οἴησις \clarify{belief} supports this view, \stephpag{c} for it appears to mean the motion \clarify{οἶσις} of the soul towards the essential nature of every individual thing, just as βουλή \clarify{intention} denotes shooting \clarify{βολή} and βούλεσθαι \clarify{wish}, as well as βουλεύεσθαι \clarify{plan}, denotes aiming at something. All these words seem to follow δόξα and to express the idea of shooting, just as ἀβουλία \clarify{ill-advisedness}, on the other hand, appears to be a failure to hit, as if a person did not shoot or hit that which he shot at or wished or planned or desired. \stephpag{d}
\hermogenesspeaks
I think you are hurrying things a bit, Socrates.
\socratesspeaks
Yes, for I am running the last lap now. But I think I must still explain ἀνάγκη \clarify{compulsion} and ἑκούσιον\clarify{voluntary} because they naturally come next. Now by the word ἑκούσιον is expressed the yielding \clarify{εἶκον} and not opposing, but, as I say, yielding to the motion which is in accordance with the will; but the compulsory \clarify{τὸἀναγκαῖον} and resistant, being contrary to the will, is associated with error and ignorance; so it is likened to walking through ravines \clarify{ἄγκη}, \stephpag{e} because they are hard to traverse, rough, and rugged, and retard motion; the word ἀναγκαῖον may, then, originate in a comparison with progress through a ravine. But let us not cease to use my strength, so long as it lasts and do not you cease from asking questions.
\hermogenesspeaks
I ask, then, about the greatest and noblest words, \stephpag{421 a} truth \clarify{ἀλήθεια}, falsehood \clarify{ψεῦδος}, being \clarify{τὸ ὄν}, and why name, the subject of our whole discourse, has the name ὄνομα.
\socratesspeaks
Does the word μαίσθαι \clarify{search} mean anything to you?
\hermogenesspeaks
Yes, it means ``seek."
\socratesspeaks
The word ὄνομα seems to be a word composed from a sentence signifying ``this is a being about which our search is." You can recognize that more readily in the adjective ὀνομαστόν, for that says clearly that this is \stephpag{b} ὄν οὗ μάσμαἐστίν \clarify{being of which the search is}. And ἀλήθεια \clarify{truth} is like the others; for the divine motion of the universe is, I think, called by this name, ἀλήθεια, because it is a divine wandering θεία ἄλη. But ψεῦδος \clarify{falsehood} is the opposite of motion; for once more that which is held back and forced to be quiet is found fault with, and it is compared to slumberers \clarify{εὕουσι}; but the addition of the psi conceals the meaning of the word. The words τὸ ὄν\clarify{being} and οὐσία \clarify{existence} agree with ἀληθής with the loss of iota, for they mean ``going" \clarify{ἰόν}. And οὐκ ὄν \clarify{not being} means οὐκ ἰόν \clarify{not going}, \stephpag{c} and indeed some people pronounce it so.
\hermogenesspeaks
I think you have knocked these words to pieces manfully, Socrates; but if anyone should ask you what propriety or correctness there was in these words that you have employed—ἰόν and ρἕον and δοῦν—
\socratesspeaks
What answer should I make? Is that your meaning?
\hermogenesspeaks
Yes, exactly.
\socratesspeaks
We acquired just now one way of making an answer with a semblance of sense in it.
\hermogenesspeaks
What way was that?
\socratesspeaks
Saying, if there is a word we do not know about, that it is of foreign origin. \stephpag{d} Now this may be true of some of them, and also on account of the lapse of time it may be impossible to find out about the earliest words; for since words get twisted in all sorts of ways, it would not be in the least wonderful if the ancient Greek word should be identical with the modern foreign one.
\hermogenesspeaks
That is not unlikely.
\socratesspeaks
It is indeed quite probable. However, we must play the game13 and investigate these questions vigorously. But let us bear in mind that if a person asks \stephpag{e} about the words by means of which names are formed, and again about those by means of which those words were formed, and keeps on doing this indefinitely, he who answers his questions will at last give up; will he not?
\hermogenesspeaks
Yes, I think so. \stephpag{422 a}
\socratesspeaks
Now at what point will he be right in giving up and stopping? Will it not be when he reaches the names which are the elements of the other names and words? For these, if they are the elements, can no longer rightly appear to be composed of other names. For instance, we said just now that ἀγαθόν was composed of ἀγαστόν and θοόν; and perhaps we might say that θοόν was composed of other words, and those of still others; \stephpag{b} but if we ever get hold of a word which is no longer composed of other words, we should be right ill saying that we had at last reached an element, and that we must no longer refer to other words for its derivation.
\hermogenesspeaks
I think you are right.
\socratesspeaks
Are, then, these words about which you are now asking elements, and must we henceforth investigate their correctness by some other method?
\hermogenesspeaks
Probably.
\socratesspeaks
Yes, probably, \hermogenesspeaks; at any rate, all the previous words were traced back to these. \stephpag{c} But if this be true, as I think it is, come to my aid again and help me in the investigation, that I may not say anything foolish in declaring what principle must underlie the correctness of the earliest names.
\hermogenesspeaks
Go on, and I will help you to the best of my ability.
\socratesspeaks
I think you agree with me that there is but one principle of correctness in all names, the earliest as well as the latest, and that none of them is any more a name than the rest.
\hermogenesspeaks
Certainly. \stephpag{d}
\socratesspeaks
Now the correctness of all the names we have discussed was based upon the intention of showing the nature of the things named.
\hermogenesspeaks
Yes, of course.
\socratesspeaks
And this principle of correctness must be present in all names, the earliest as well as the later ones, if they are really to be names.
\hermogenesspeaks
Certainly.
\socratesspeaks
But the later ones, apparently, were able to accomplish this by means of the earlier ones.
\hermogenesspeaks
Evidently.
\socratesspeaks
Well, then, how can the earliest names, which are not as yet based upon any others, make clear to us the nature of things, so far as that is possible, \stephpag{e} which they must do if they are to be names at all? Answer me this question: If we had no voice or tongue, and wished to make things clear to one another, should we not try, as dumb people actually do, to make signs with our hands and head and person generally?
\hermogenesspeaks
Yes. What other method is there, Socrates? \stephpag{423 a}
\socratesspeaks
If we wished to designate that which is above and is light, we should, I fancy, raise our hand towards heaven in imitation of the nature of the thing in question; but if the things to be designated were below or heavy, we should extend our hands towards the ground; and if we wished to mention a galloping horse or any other animal, we should, of course, make our bodily attitudes as much like theirs as possible.
\hermogenesspeaks
I think you are quite right; there is no other way.
\socratesspeaks
For the expression of anything, I fancy, \stephpag{b} would be accomplished by bodily imitation of that which was to be expressed.
\hermogenesspeaks
Yes.
\socratesspeaks
And when we wish to express anything by voice or tongue or mouth, will not our expression by these means be accomplished in any given instance when an imitation of something is accomplished by them?
\hermogenesspeaks
I think that is inevitable.
\socratesspeaks
A name, then, it appears, is a vocal imitation of that which is imitated, and he who imitates with his voice names that which he imitates.
\hermogenesspeaks
I think that is correct. \stephpag{c}
\socratesspeaks
By Zeus, I do not think it is quite correct, yet, my friend.
\hermogenesspeaks
Why not?
\socratesspeaks
We should be obliged to agree that people who imitate sheep and cocks and other animals were naming those which they imitate.
\hermogenesspeaks
Yes, so we should.
\socratesspeaks
And do you think that is correct?
\hermogenesspeaks
No, I do not; but, Socrates, what sort of an imitation is a name?
\socratesspeaks
In the first place we shall not, in my opinion, be making names, if we imitate things as we do in music, \stephpag{d} although musical imitation also is vocal; and secondly we shall make no names by imitating that which music imitates. What I mean is this: all objects have sound and shape, and many have color, have they not?
\hermogenesspeaks
Certainly.
\socratesspeaks
Well then, the art of naming is not employed in the imitation of those qualities, and has nothing to do with them. The arts which are concerned with them are music and design, are they not?
\hermogenesspeaks
Yes. \stephpag{e}
\socratesspeaks
Here is another point. Has not each thing an essential nature, just as it has a color and the other qualities we just mentioned? Indeed, in the first place, have not color and sound and all other things which may properly be said to exist, each and all an essential nature?
\hermogenesspeaks
I think so.
\socratesspeaks
Well, then, if anyone could imitate this essential nature of each thing by means of letters and syllables, he would show what each thing really is, would he not? \stephpag{424 a}
\hermogenesspeaks
Certainly.
\socratesspeaks
And what will you call him who can do this, as you called the others musician and painter? What will you call this man?
\hermogenesspeaks
I think, Socrates, he is what we have been looking for all along, the name-maker.
\socratesspeaks
If that is the case, is it our next duty to consider whether in these names about which you were asking—flow, motion, and restraint—the namemaker grasps with his letters and syllables the reality \stephpag{b} of the things named and imitates their essential nature, or not?
\hermogenesspeaks
Certainly.
\socratesspeaks
Well now, let us see whether those are the only primary names, or there are others.
\hermogenesspeaks
I think there are others.
\socratesspeaks
Yes, most likely there are. Now what is the method of division with which the imitator begins his imitation? Since the imitation of the essential nature is made with letters and syllables, would not the most correct way be for us to separate the letters first, \stephpag{c} just as those who undertake the practice of rhythms separate first the qualities of the letters, then those of the syllables, and then, but not till then, come to the study of rhythms?
\hermogenesspeaks
Yes.
\socratesspeaks
Must not we, too, separate first the vowels, then in their several classes the consonants or mutes, as they are called by those who specialize in phonetics, and also the letters which are neither vowels nor mutes, as well as the various classes that exist among the vowels themselves? \stephpag{d} And when we have made all these divisions properly, we must in turn give names to the things which ought to have them, if there are any names to which they can all, like the letters, be referred, from which it is possible to see what their nature is and whether there are any classes among them, as there are among letters. When we have properly examined all these points, we must know how to apply each letter with reference to its fitness, whether one letter is to be applied to one thing or many are to be combined; just as painters, when they wish to produce an imitation, sometimes use only red, \stephpag{e} sometimes some other color, and sometimes mix many colors, as when they are making a picture of a man or something of that sort, employing each color, I suppose, as they think the particular picture demands it. In just this way we, too, shall apply letters to things, using one letter for one thing, when that seems to be required, or many letters together, forming syllables, as they are called, and in turn combining syllables, \stephpag{425 a} and by their combination forming nouns and verbs. And from nouns and verbs again we shall finally construct something great and fair and complete. Just as in our comparison we made the picture by the art of painting, so now we shall make language by the art of naming, or of rhetoric, or whatever it be. No, not we; I said that too hastily. For the ancients gave language its existing composite character; and we, if we are to examine all these matters with scientific ability, \stephpag{b} must take it to pieces as they put it together and see whether the words, both the earliest and the later, are given systematically or not; for if they are strung together at haphazard, it is a poor, unmethodical performance, my dear \hermogenesspeaks.
\hermogenesspeaks
By Zeus, Socrates, may be it is.
\socratesspeaks
Well, do you believe you could take them to pieces in that way? I do not believe I could.
\hermogenesspeaks
Then I am sure I could not.
\socratesspeaks
Shall we give up then? Or shall we do the best we can and try to see if we are able to understand even a little about them, \stephpag{c} and, just as we said to the gods a while ago that we knew nothing about the truth but were guessing at human opinion about them, so now, before we proceed, shall we say to ourselves that if anyone, whether we or someone else, is to make any analysis of names, he will have to analyze them in the way we have described, and we shall have to study them, as the saying is, with all our might? Do you agree, or not?
\hermogenesspeaks
Yes, I agree most heartily. \stephpag{d}
\socratesspeaks
It will, I imagine, seem ridiculous that things are made manifest through imitation in letters and syllables; nevertheless it cannot be otherwise. For there is no better theory upon which we can base the truth of the earliest names, unless you think we had better follow the example of the tragic poets, who, when they are in a dilemma, have recourse to the introduction of gods on machines. So we may get out of trouble by saying that the gods gave the earliest names, and therefore they are right. \stephpag{e} Is that the best theory for us? Or perhaps this one, that we got the earliest names from some foreign folk and the foreigners are more ancient than we are? Or that it is impossible to investigate them because of their antiquity, as is also the case with the foreign words? \stephpag{426 a} All these are merely very clever evasions on the part of those who refuse to offer any rational theory of the correctness of the earliest names. And yet if anyone is, no matter why, ignorant of the correctness of the earliest names, he cannot know about that of the later, since they can be explained only by means of the earliest, about which he is ignorant. No, it is clear that anyone who claims to have scientific knowledge of names must be able first of all to explain the earliest names perfectly, \stephpag{b} or he can be sure that what he says about the later will be nonsense. Or do you disagree?
\hermogenesspeaks
No, Socrates, not in the least.
\socratesspeaks
Now I think my notions about the earliest names are quite outrageous and ridiculous. I will impart them to you, if you like; if you can find anything better, please try to impart it to me.
\hermogenesspeaks
I will do so. Go on, and do not be afraid. \stephpag{c}
\socratesspeaks
First, then, the letter rho seems to me to be an instrument expressing all motion. We have not as yet said why motion has the name κίνησις; but it evidently should be ἴεσις, for in old times we did not employ eta, but epsilon. And the beginning of κίνησις is from κίειν, a foreign word equivalent to ἰέναι \clarify{go}. So we should find that the ancient word corresponding to our modern form would be ἴεσις; but now by the employment of the foreign word κίειν, change of epsilon to eta, and the insertion of nu it has become κίνησις, though it ought to be κιείνεσις or εἶσις. \stephpag{d} And στάσις \clarify{rest} signifies the negation of motion, but is called στάσις for euphony. Well, the letter rho, as I was saying, appeared to be a fine instrument expressive of motion to the name-giver who wished to imitate rapidity, and he often applies it to motion. In the first place, in the words ῥεῖν \clarify{flow} and ῥοή \clarify{current} he imitates their rapidity by this letter, \stephpag{e} then in τρόμος \clarify{trembling} and in τρέχειν \clarify{run}, and also in such words as κρούειν \clarify{strike}, θραύειν \clarify{break}, ἐρείκειν \clarify{rend}, θρύπτειν \clarify{crush},κερματίζειν \clarify{crumble}, ῥυμβεῖν \clarify{whirl}, he expresses the action of them all chiefly by means of the letter rho; for he observed, I suppose, that the tongue is least at rest and most agitated in pronouncing this letter, and that is probably the reason why he employed it for these words. Iota again, he employs for everything subtle, which can most readily pass through all things. \stephpag{427 a} Therefore he imitates the nature of ἰέναι \clarify{go} and ἵεσθαι \clarify{hasten} by means of iota, just as he has imitated all such notions as ψυχρόν \clarify{cold, shivering}, ζέον \clarify{seething}, σείεσθαι \clarify{shake}, and σεισμός \clarify{shock} by means of phi, psi, sigma, and zeta, because those letters are pronounced with much breath. Whenever he imitates that which resembles blowing, the giver of names always appears to use for the most part such letters. And again he appears to have thought that the compression and pressure of the tongue in the pronunciation of delta and tau was naturally fitted \stephpag{b} to imitate the notion of binding and rest. And perceiving that the tongue has a gliding movement most in the pronunciation of lambda, he made the words λεῖα \clarify{level}, ὀλισθάναιν \clarify{glide} itself, λιπαρόν \clarify{sleek}, κολλῶδες \clarify{glutinous}, and the like to conform to it. Where the gliding of the tongue is stopped by the sound of gamma he reproduced the nature of γλισχρόν \clarify{glutinous}, γλυκύ \clarify{sweet}, and γλοιῶδες \clarify{gluey}. \stephpag{c} And again, perceiving that nu is an internal sound, he made the words ἔνδον \clarify{inside} and ἐντός \clarify{within}, assimilating the meanings to the letters, and alpha again he assigned to greatness, and eta to length, because the letters are large. He needed the sign Ο for the expression of γόγγυλον \clarify{round}, and made it the chief element of the word. And in this way the lawgiver appears to apply the other letters, making by letters and syllables a name for each and every thing, and from these names he compounds all the rest by imitation. \stephpag{d} This, \hermogenesspeaks, appears to me to be the theory of the correctness of names, unless, indeed, Cratylus has some other view.
\hermogenesspeaks
Truly, Socrates, as I said in the beginning, Cratylus often troubles me a good deal; he declares that there is such a thing as correctness of names, but does not say clearly what it is; and so I cannot tell whether he speaks so obscurely about it on any given occasion intentionally or unintentionally. \stephpag{e} So now, Cratylus, tell me, in the presence of Socrates, do you like what Socrates says about names, or have you a better theory to propose? And if you have, tell us about it; then you will either learn from Socrates or instruct both him and me.
\cratylusspeaks
But, \hermogenesspeaks, do you think it is an easy matter to learn or teach any subject so quickly, especially so important an one as this, which appears to me to be one of the most important? \stephpag{428 a}
\hermogenesspeaks
No, by Zeus, I do not. But I think Hesiod is right in saying:``If you can only add little to little, it is worth while.
"Hes. WD 359 So now if you can make even a little progress, do not shirk the trouble, but oblige Socrates—you owe it to him—and me.
\socratesspeaks
For that matter, Cratylus, I would not positively affirm any of the things I have said. I merely expressed the opinions which I reached with the help of \hermogenesspeaks. So far as I am concerned, you need not hesitate, \stephpag{b} and if your view is better than mine, I will accept it. And I should not be at all surprised if it were better; for I think you have not only investigated such matters yourself but have been taught about them by others. So if you have any better theory to propound, put me down as one of your pupils in the course on the correctness of names.
\cratylusspeaks
Yes, Socrates, I have, as you say, paid attention to these matters, and perhaps I might make you my pupil. However, I am afraid the opposite is the case, \stephpag{c} and I am impelled to say to you what Achilles says in the ``Prayers" to Ajax. He says:``Ajax, descendant of Zeus, son of Telamon, chief of thy people,
All thou hast uttered is good in my sight and pleases my spirit.
"Hom. Il. 9.644 f And so, Socrates, your oracular utterances seem to me to be much to my mind, whether you are inspired by Euthyphro or some other Muse has dwelt within you all along without our knowing it. \stephpag{d}
\socratesspeaks
My excellent Cratylus, I myself have been marvelling at my own wisdom all along, and I cannot believe in it. So I think we ought to reexamine my utterances. For the worst of all deceptions is self-deception. How can it help being terrible, when the deceiver is always present and never stirs from the spot? So I think we must turn back repeatedly to what we have said and must try, as the poet says, to look ``both forwards and backwards.
"Hom. Il. 1.343; 3.109 \stephpag{e} Then let us now see what we have said. Correctness of a name, we say, is the quality of showing the nature of the thing named. Shall we call that a satisfactory statement?
\cratylusspeaks
I am perfectly satisfied with it, Socrates.
\socratesspeaks
Names, then, are given with a view to instruction?
\cratylusspeaks
Certainly.
\socratesspeaks
Shall we, then; say that this instruction is an art and has its artisans?
\cratylusspeaks
Certainly.
\socratesspeaks
Who are they? \stephpag{429 a}
\cratylusspeaks
The lawgivers, as you said in the beginning.
\socratesspeaks
Shall we declare that this art arises in men like the other arts, or not? What I mean is this: Some painters are better, and others worse, are they not?
\cratylusspeaks
Certainly.
\socratesspeaks
And the better produce better works—that is, their paintings—and the others worse works? And likewise some builders build better houses and others worse?
\cratylusspeaks
Yes. \stephpag{b}
\socratesspeaks
Then do some lawgivers produce better, and others worse works?
\cratylusspeaks
No; at that point I cease to agree.
\socratesspeaks
Then you do not think that some laws are better, and some worse?
\cratylusspeaks
No, I do not.
\socratesspeaks
And you do not, it appears, think that one name is better, and another worse?
\cratylusspeaks
No, I do not.
\socratesspeaks
Then all names are correct?
\cratylusspeaks
All that are really names.
\socratesspeaks
How about the name of our friend \hermogenesspeaks, \stephpag{c} which was mentioned a while ago? Shall we say that it is not his name at all, unless he belongs to the race of Hermes, or that it is his name, but is incorrect?
\cratylusspeaks
I think, Socrates, that it is not his name at all; it appears to be his, but is really the name of some one else who possesses the nature that makes the name clear.
\socratesspeaks
And when anyone says that our friend is \hermogenesspeaks, is he not even speaking falsely? For perhaps it is not even possible to say that he is \hermogenesspeaks, if he is not.
\cratylusspeaks
What do you mean?
\socratesspeaks
Do you mean to say that it is impossible to speak falsehood at all? \stephpag{d} For there are, my dear Cratylus, many who do so, and who have done so in the past.
\cratylusspeaks
Why, Socrates, how could anyone who says that which he says, say that which is not? Is not falsehood saying that which is not?
\socratesspeaks
Your reasoning is too clever for me at my age, my friend. However, tell me this: Do you think it is possible to speak falsehood, \stephpag{e} but not to say it?
\cratylusspeaks
Neither to speak nor to say it.
\socratesspeaks
Nor utter it or use it as a form of address? For instance, if some one should meet you in hospitable fashion, should grasp your hand and say, ``Well met, my friend from Athens, son of Smicrion, \hermogenesspeaks," would he be saying or speaking or uttering or addressing these words not to you, but to \hermogenesspeaks—or to nobody?
\cratylusspeaks
I think, Socrates, the man would be producing sounds without sense.
\socratesspeaks
Even that reply is welcome; \stephpag{430 a} for I can ask whether the words he produced would be true, or false, or partly true and partly false. Even that would suffice.
\cratylusspeaks
I should say that the man in such a case was merely making a noise, going through purposeless motions, as if he were beating a bronze pot.
\socratesspeaks
Let us see, Cratylus, if we cannot come to terms somehow. You would agree, would you not, that the name is one thing and the thing of which it is the name is another?
\cratylusspeaks
Yes, I should.
\socratesspeaks
And you agree that the name is an imitation \stephpag{b} of the thing named?
\cratylusspeaks
Most assuredly.
\socratesspeaks
And you agree that paintings also are imitations, though in a different way, of things?
\cratylusspeaks
Yes.
\socratesspeaks
Well then—for perhaps I do not understand, and you may be right—can both of these imitations, the paintings and the names, be assigned and applied to the things which they imitate, or not? \stephpag{c}
\cratylusspeaks
They can.
\socratesspeaks
First, then, consider this question: Can we assign the likeness of the man to the man and that of the woman to the woman, and so forth?
\cratylusspeaks
Certainly.
\socratesspeaks
And can we conversely attribute that of the man to the woman, and the woman's to the man?
\cratylusspeaks
That is also possible.
\socratesspeaks
And are these assignments both correct, or only the former?
\cratylusspeaks
The former.
\socratesspeaks
The assignment, in short, which attributes to each that which belongs to it and is like it.
\cratylusspeaks
That is my view.
\socratesspeaks
To put an end to contentious argument between you and me, \stephpag{d} since we are friends, let me state my position. I call that kind of assignment in the case of both imitations paintings and names—correct, and in the case of names not only correct, but true; and the other kind, which gives and applies the unlike imitation, I call incorrect and, in the case of names, false.
\cratylusspeaks
But it may be, Socrates, that this incorrect assignment is possible in the case of paintings, and not in the case of names, \stephpag{e} which must be always correctly assigned.
\socratesspeaks
What do you mean? What difference is there between the two? Can I not step up to a man and say to him, ``This is your portrait," and show him perhaps his own likeness or, perhaps, that of a woman? And by ``show" I mean bring before the sense of sight.
\cratylusspeaks
Certainly.
\socratesspeaks
Well, then, can I not step up to the same man again and say, ``This is your name"? A name is an imitation, just as a picture is. \stephpag{431 a} Very well; can I not say to him, ``This is your name," and then bring before his sense of hearing perhaps the imitation of himself, saying that it is a man, or perhaps the imitation of the female of the human species, saying that it is a woman? Do you not believe that this is possible and sometimes happens?
\cratylusspeaks
I am willing to concede it, Socrates, and grant that you are right.
\socratesspeaks
That is a good thing for you to do, my friend, if I am right; for now we need no longer argue about the matter. \stephpag{b} If, then, some such assignment of names takes place, we will call one kind speaking truth, and the other speaking falsehood. But if this is accepted, and if it is possible to assign names incorrectly and to give to objects not the names that befit them, but sometimes those that are unfitting, it would be possible to treat verbs in the same way. And if verbs and nouns can be assigned in this way, the same must be true of sentences; for sentences are, I conceive, a combination of verbs and nouns. \stephpag{c} What do you say to that, Cratylus?
\cratylusspeaks
I agree; I think you are right.
\socratesspeaks
If, then, we compare the earliest words to sketches, it is possible in them, as in pictures, to reproduce all the colors and shapes, or not all; some may be wanting, and some may be added, and they may be too many or too large. Is not that true?
\cratylusspeaks
Yes, it is.
\socratesspeaks
Then he who reproduces all, produces good sketches and pictures, and he who adds or takes away produces also sketches and pictures, but bad ones? \stephpag{d}
\cratylusspeaks
Yes.
\socratesspeaks
And how about him who imitates the nature of things by means of letters and syllables? By the same principle, if he gives all that is appropriate, the image—that is to say, the name—will be good, and if he sometimes omits a little, it will be an image, but not a good one; and therefore some names are well and others badly made. Is that not true?
\cratylusspeaks
Perhaps. \stephpag{e}
\socratesspeaks
Perhaps, then, one artisan of names will be good, and another bad?
\cratylusspeaks
Yes.
\socratesspeaks
The name of such an artisan was lawgiver?
\cratylusspeaks
Yes.
\socratesspeaks
Perhaps, then, by Zeus, as is the case in the other arts, one lawgiver may be good and another bad, if we accept our previous conclusions.
\cratylusspeaks
That is true. But you see, Socrates, when by the science of grammar we assign these letters—alpha, beta, and the rest—to names, \stephpag{432 a} if we take away or add or transpose any letter, it is not true that the name is written, but written incorrectly; it is not written at all, but immediately becomes a different word, if any such thing happens to it.
\socratesspeaks
Perhaps we are not considering the matter in the right way.
\cratylusspeaks
Why not?
\socratesspeaks
It may be that what you say would be true of those things which must necessarily consist of a certain number or cease to exist at all, as ten, for instance, or any number you like, \stephpag{b} if you add or subtract anything is immediately another number; but this is not the kind of correctness which applies to quality or to images in general; on the contrary, the image must not by any means reproduce all the qualities of that which it imitates, if it is to be an image. See if I am not right. Would there be two things, Cratylus and the image of Cratylus, if some god should not merely imitate your color and form, as painters do, but should also make all the inner parts like yours, should reproduce \stephpag{c} the same flexibility and warmth, should put into them motion, life, and intellect, such as exist in you, and in short, should place beside you a duplicate of all your qualities? Would there be in such an event Cratylus and an image of Cratylus, or two Cratyluses?
\cratylusspeaks
I should say, Socrates, two Cratyluses.
\socratesspeaks
Then don't you see, my friend, that we must look for some other principle of correctness in images and in names, of which we were speaking, and must not insist that they are no longer images \stephpag{d} if anything be wanting or be added? Do you not perceive how far images are from possessing the same qualities as the originals which they imitate?
\cratylusspeaks
Yes, I do.
\socratesspeaks
Surely, Cratylus, the effect produced by the names upon the things of which they are the names would be ridiculous, if they were to be entirely like them in every respect. For everything would be duplicated, and no one could tell in any case which was the real thing and which the name.
\cratylusspeaks
Quite true.
\socratesspeaks
Then do not be faint-hearted, but have the courage to admit that one name may be correctly and another incorrectly given; \stephpag{e} do not insist that it must have all the letters and be exactly the same as the thing named, but grant that an inappropriate letter may be employed. But if a letter, then grant that also a noun in a clause, and if a noun, then also a clause in a sentence may be employed which is not appropriate to the things in question, and the thing may none the less be named and described, so long as the intrinsic quality of the thing named is retained, \stephpag{433 a} as is the case in the names of the letters of the alphabet, if you remember what \hermogenesspeaks and I were saying a while ago.
\cratylusspeaks
Yes, I remember.
\socratesspeaks
Very well, then. So long as this intrinsic quality is present, even though the name have not all the proper letters, the thing will still be named; well, when it has all the proper letters; badly, when it has only a few of them. Let us, then, grant this, my friend, or we shall get into trouble, like the belated night wanderers in the road at Aegina,14 and in very truth we shall be found to have arrived too late; \stephpag{b} otherwise you must look for some other principle of correctness in names, and must not admit that a name is the representation of a thing in syllables and letters. For if you maintain both positions, you cannot help contradicting yourself.
\cratylusspeaks
Well, Socrates, I think what you say is reasonable, and I accept it.
\socratesspeaks
Then since we are agreed about this, let us consider the next point. If a name, we say, is to be a good one, it must have the proper letters?
\cratylusspeaks
Yes. \stephpag{c}
\socratesspeaks
And the proper letters are those which are like the things named?
\cratylusspeaks
Yes, certainly.
\socratesspeaks
That is, then, the method by which wellgiven names are given. But if any name is not well given, the greater part of it may perhaps, if it is to be an image at all, be made up of proper and like letters, but it may contain some inappropriate element, and is on that account not good or well made. Is that our view?
\cratylusspeaks
I suppose, Socrates, there is no use in keeping up my contention; but I am not satisfied that it can be a name and not be well given.
\socratesspeaks
Are you not satisfied that the name is \stephpag{d} the representation of a thing?
\cratylusspeaks
Yes.
\socratesspeaks
And do you not think it is true that some names are composed of earlier ones and others are primary?
\cratylusspeaks
Yes.
\socratesspeaks
But if the primary names are to be representations of any things, can you suggest any better way of making them representations than by making them as much as possible like the things which they are to represent? \stephpag{e} Or do you prefer the theory advanced by \hermogenesspeaks and many others, who claim that names are conventional and represent things to those who established the convention and knew the things beforehand, and that convention is the sole principle of correctness in names, and it makes no difference whether we accept the existing convention or adopt an opposite one according to which small would be called great and great small? Which of these two theories do you prefer? \stephpag{434 a}
\cratylusspeaks
Representing by likeness the thing represented is absolutely and entirely superior to representation by chance signs.
\socratesspeaks
You are right. Then if the name is like the thing, the letters of which the primary names are to be formed must be by their very nature like the things, must they not? Let me explain. Could a painting, to revert to our previous comparison, ever be made like any real thing, if there were no pigments out of which the painting is composed, \stephpag{b} which were by their nature like the objects which the painter's art imitates? Is not that impossible?
\cratylusspeaks
Yes, it is impossible.
\socratesspeaks
In the same way, names can never be like anything unless those elements of which the names are composed exist in the first place and possess some kind of likeness to the things which the names imitate; and the elements of which they are composed are the letters, are they not?
\cratylusspeaks
Yes.
\socratesspeaks
Then I must now ask you to consider with me the subject which \hermogenesspeaks and I discussed a while ago. \stephpag{c} Do you think I am right in saying that rho is expressive of speed, motion, and hardness, or not?
\cratylusspeaks
You are right.
\socratesspeaks
And lambda is like smoothness, softness, and the other qualities we mentioned?
\cratylusspeaks
Yes.
\socratesspeaks
You know, of course, that we call the same thing σκληρότης \clarify{hardness} which the Eretrians call σκληρότηρ?
\cratylusspeaks
Certainly.
\socratesspeaks
Have rho and sigma both a likeness to the same thing, and does the final rho mean to them just what the sigma means to us, or is there to one of us no meaning? \stephpag{d}
\cratylusspeaks
They mean the same to both.
\socratesspeaks
In so far as rho and sigma are alike, or in so far as they are not?
\cratylusspeaks
In so far as they are alike.
\socratesspeaks
And are they alike in all respects?
\cratylusspeaks
Yes; at least for the purpose of expressing motion equally.
\socratesspeaks
But how about the lambda in σκληρότης? Does it not express the opposite of hardness?
\cratylusspeaks
Well, perhaps it has no right to be there, Socrates; it may be like the cases that came up in your talk with \hermogenesspeaks, when you removed or inserted letters where that was necessary. I think you did right; and in this case perhaps we ought to put a rho in place of the lambda. \stephpag{e}
\socratesspeaks
Excellent. However, do we not understand one another when anyone says σκληρόν, using the present pronunciation, and do you not now know what I mean?
\cratylusspeaks
Yes, but that is by custom, my friend.
\socratesspeaks
In saying ``custom" do you think you are saying anything different from convention? Do you not mean by ``convention" that when I speak I have a definite meaning and you recognize that I have that meaning? Is not that what you mean? \stephpag{435 a}
\cratylusspeaks
Yes.
\socratesspeaks
Then if you recognize my meaning when I speak, that is an indication given to you by me.
\cratylusspeaks
Yes.
\socratesspeaks
The indication comes from something which is unlike my meaning when I speak, if in your example σκληρότης the lambda is unlike hardness; and if this is true, did you not make a convention with yourself, since both like and unlike letters, by the influence of custom and convention, produce indication? And even if custom is entirely distinct from convention, \stephpag{b} we should henceforth be obliged to say that custom, not likeness, is the principle of indication, since custom, it appears, indicates both by the like and by the unlike. And since we grant this, Cratylus—for I take it that your silence gives consent—both convention and custom must contribute something towards the indication of our meaning when we speak. For, my friend, if you will just turn your attention to numbers, where do you think you can possibly get names to apply to each individual number on the principle of likeness, \stephpag{c} unless you allow agreement and convention on your part to control the correctness of names? I myself prefer the theory that names are, so far as is possible, like the things named; but really this attractive force of likeness is, as \hermogenesspeaks says, a poor thing, and we are compelled to employ in addition this commonplace expedient, convention, to establish the correctness of names. Probably language would be, within the bounds of possibility, most excellent when all its terms, or as many as possible, were based on likeness, that is to say, were appropriate, and most deficient under opposite conditions. \stephpag{d} But now answer the next question. What is the function of names, and what good do they accomplish?
\cratylusspeaks
I think, Socrates, their function is to instruct, and this is the simple truth, that he who knows the names knows also the things named.
\socratesspeaks
I suppose, Cratylus, you mean that when anyone knows the nature of the name—and its nature is that of the thing—he will know the thing also, \stephpag{e} since it is like the name, and the science of all things which are like each other is one and the same. It is, I fancy, on this ground that you say whoever knows names will know things also.
\cratylusspeaks
You are perfectly right.
\socratesspeaks
Now let us see what this manner of giving instruction is, to which you refer, and whether there is another method, but inferior to this, or there is no other at all. What do you think? \stephpag{436 a}
\cratylusspeaks
I think there is no other at all; this is both the best and the only method.
\socratesspeaks
Do you think this is also the method of discovering realities, and that he who has discovered the names has discovered also the things named; or do you think inquiry and discovery demand another method, and this belongs to instruction?
\cratylusspeaks
I most certainly think inquiry and discovery follow this same method and in the same way.
\socratesspeaks
Let us consider the matter, Cratylus. Do you not see that he who in his inquiry after things follows names \stephpag{b} and examines into the meaning of each one runs great risks of being deceived?
\cratylusspeaks
How so?
\socratesspeaks
Clearly he who first gave names, gave such names as agreed with his conception of the nature of things. That is our view, is it not?
\cratylusspeaks
Yes.
\socratesspeaks
Then if his conception was incorrect, and he gave the names according to his conception, what do you suppose will happen to us who follow him? Can we help being deceived?
\cratylusspeaks
But, Socrates, surely that is not the case. \stephpag{c} He who gave the names must necessarily have known; otherwise, as I have been saying all along, they would not be names at all. And there is a decisive proof that the name-giver did not miss the truth, one which you must accept; for otherwise his names would not be so universally consistent. Have you not yourself noticed in speaking that all names were formed by the same method and with the same end in view?
\socratesspeaks
But that, Cratylus, is no counter argument. For if the giver of names erred in the beginning \stephpag{d} and thenceforth forced all other names into agreement with his own initial error, there is nothing strange about that. It is just so sometimes in geometrical diagrams; the initial error is small and unnoticed, but all the numerous deductions are wrong, though consistent. Every one must therefore give great care and great attention to the beginning of any undertaking, to see whether his foundation is right or not. If that has been considered with proper care, everything else will follow. \stephpag{e} However, I should be surprised if names are really consistent. Let us review our previous discussion. Names, we said, indicate nature to us, assuming that all things are in motion and flux. Do you not think they do so? \stephpag{437 a}
\cratylusspeaks
Yes, and they indicate it correctly.
\socratesspeaks
Let us first take up again the word ἐπιστήμη \clarify{knowledge} and see how ambiguous it is, seeming to indicate that it makes our soul stand still \clarify{ἵστησιν} at things, rather than that it is carried round with them, so it is better to speak the beginning of it as we now do than to insert the epsilon and say ἐπεϊστήμ; we should insert an iota rather than an epsilon. Then take βέβαιον \clarify{firm}, which expresses position and rest, not motion. \stephpag{b} And ἱστορία \clarify{inquiry} means much the same, that it stops \clarify{ἵστησιν} the flow. And πιστόν \clarify{faithful} most certainly means that which stops \clarify{ἱστόν} motion. Then again, anyone can see that μνήμη \clarify{memory} expresses rest \clarify{μονή} in the soul, not motion. On the other hand, ἁμαρτία \clarify{error} and ξυμφορά \clarify{misfortune}, if you consider merely the form of the names, will appear to be the same as σύνεσις \clarify{intellect} and ἐπιστήμη and all the other names of good significance. Moreover, ἀμαθία \clarify{ignorance} and ἀκολασία \clarify{unrestraint} also appear to be like them; for the former, ἀμαθία, \stephpag{c} seems to be τοῦ ἅμα θεῷ ἰόντος πορεία \clarify{the progress of one who goes with God}, and ἀκολασία seems to be exactly ἀκολουθία τοῖς πράγμασιν \clarify{movement in company with things}. And so names which we believe have the very worst meanings appear to be very like those which have the best. And I think we could, if we took pains, find many other words which would lead us to reverse our judgement and believe that the giver of names meant that things were not in progress or in motion, but were at rest.
\cratylusspeaks
But, Socrates, you see that most of the names \stephpag{d} indicate motion.
\socratesspeaks
What of that, Cratylus? Are we to count names like votes, and shall correctness rest with the majority? Are those to be the true names which are found to have that one of the two meanings which is expressed by the greater number?
\cratylusspeaks
That is not reasonable.
\socratesspeaks
No, not in the least, my friend. \stephpag{438 a} Now let us drop this and return to the point at which we digressed. A little while ago, you may remember, you said he who gave names must have known the things to which he gave them. Do you still hold that opinion, or not?
\cratylusspeaks
I do.
\socratesspeaks
And you say that he who gave the first names also knew the things which he named?
\cratylusspeaks
Yes, he knew them.
\socratesspeaks
But from what names had he learned or discovered the things, \stephpag{b} if the first names had not yet been given, and if we declare that it is impossible to learn or discover things except by learning or ourselves discovering the names?
\cratylusspeaks
I think there is something in what you say, Socrates.
\socratesspeaks
How can we assert that they gave names or were lawgivers with knowledge, before any name whatsoever had been given, and before they knew any names, if things cannot be learned except through their names? \stephpag{c}
\cratylusspeaks
I think the truest theory of the matter, Socrates, is that the power which gave the first names to things is more than human, and therefore the names must necessarily be correct.
\socratesspeaks
Then, in your opinion, he who gave the names, though he was a spirit or a god, would have given names which made him contradict himself? Or do you think there is no sense in what we were saying just now?
\cratylusspeaks
But, Socrates, those that make up one of the two classes are not really names.
\socratesspeaks
Which of the two, my excellent friend; the class of those which point towards rest or of those that point towards motion? We agreed just now that the matter is not to be determined by mere numbers. \stephpag{d}
\cratylusspeaks
No; that would not be right, Socrates.
\socratesspeaks
Then since the names are in conflict, and some of them claim that they are like the truth, and others that they are, how can we decide, and upon what shall we base our decision? Certainly not upon other names differing from these, for there are none. No, it is plain that we must look for something else, not names, which shall show us which of these two kinds are the true names, which of them, that is to say, \stephpag{e} show the truth of things.
\cratylusspeaks
That is my opinion.
\socratesspeaks
Then if that is true, Cratylus, it seems that things may be learned without names.
\cratylusspeaks
So it appears.
\socratesspeaks
What other way is left by which you could expect to know them? What other than the natural and the straightest way, through each other, if they are akin, and through themselves? For that which is other and different from them would signify not them, but something other and different.
\cratylusspeaks
I think that is true. \stephpag{439 a}
\socratesspeaks
Stop for Heaven's sake! Did we not more than once agree that names which are rightly given are like the things named and are images of them?
\cratylusspeaks
Yes.
\socratesspeaks
Then if it be really true that things can be learned either through names or through themselves which would be the better and surer way of learning? To learn from the image whether it is itself a good imitation and also to learn the truth which it imitates, \stephpag{b} or to learn from the truth both the truth itself and whether the image is properly made?
\cratylusspeaks
I think it is certainly better to learn from the truth.
\socratesspeaks
How realities are to be learned or discovered is perhaps too great a question for you or me to determine; but it is worth while to have reached even this conclusion, that they are to be learned and sought for, not from names but much better through themselves than through names.
\cratylusspeaks
That is clear, Socrates.
\socratesspeaks
Then let us examine one further point to avoid being deceived by the fact that most of these names tend in the same direction. \stephpag{c} Suppose it should prove that although those who gave the names gave them in the belief that all things are in motion and flux—I myself think they did have that belief— still in reality that is not the case, and the namegivers themselves, having fallen into a kind of vortex, are whirled about, dragging us along with them. Consider, my worthy Cratylus, a question about which I often dream. Shall we assert that there is any absolute beauty, or good, or any other absolute existence, \stephpag{d} or not?
\cratylusspeaks
I think there is, Socrates.
\socratesspeaks
Then let us consider the absolute, not whether a particular face, or something of that sort, is beautiful, or whether all these things are in flux. Is not, in our opinion, absolute beauty always such as it is?
\cratylusspeaks
That is inevitable.
\socratesspeaks
Can we, then, if it is always passing away, correctly say that it is this, then that it is that, or must it inevitably, in the very instant while we are speaking, become something else and pass away and no longer be what it is?
\cratylusspeaks
That is inevitable. \stephpag{e}
\socratesspeaks
How, then, can that which is never in the same state be anything? For if it is ever in the same state, then obviously at that time it is not changing; and if it is always in the same state and is always the same, how can it ever change or move without relinquishing its own form?
\cratylusspeaks
It cannot do so at all.
\socratesspeaks
No, nor can it be known by anyone. \stephpag{440 a} For at the moment when he who seeks to know it approaches, it becomes something else and different, so that its nature and state can no longer be known; and surely there is no knowledge which knows that which is in no state.
\cratylusspeaks
It is as you say.
\socratesspeaks
But we cannot even say that there is any knowledge, if all things are changing and nothing remains fixed; for if knowledge itself does not change and cease to be knowledge, then knowledge would remain, and there would be knowledge; but if the very essence of knowledge changes, \stephpag{b} at the moment of the change to another essence of knowledge there would be no knowledge, and if it is always changing, there will always be no knowledge, and by this reasoning there will be neither anyone to know nor anything to be known. But if there is always that which knows and that which is known—if the beautiful, the good, and all the other verities exist—I do not see how there is any likeness between these conditions of which I am now speaking and flux or motion. \stephpag{c} Now whether this is the nature of things, or the doctrine of Heracleitus and many others is true, is another question; but surely no man of sense can put himself and his soul under the control of names, and trust in names and their makers to the point of affirming that he knows anything; nor will he condemn himself and all things and say that there is no health in them, but that all things are flowing like leaky pots, \stephpag{d} or believe that all things are just like people afflicted with catarrh, flowing and running all the time. Perhaps, Cratylus, this theory is true, but perhaps it is not. Therefore you must consider courageously and thoroughly and not accept anything carelessly—for you are still young and in your prime; then, if after investigation you find the truth, impart it to me.
\cratylusspeaks
I will do so. However, I assure you, Socrates, that I have already considered the matter, and after toilsome consideration \stephpag{e} I think the doctrine of Heracleitus is much more likely to be true.
\socratesspeaks
Some other time, then, my friend, you will teach me, when you come back; but now go into the country as you have made ready to do; and \hermogenesspeaks here will go with you a bit.
\cratylusspeaks
Very well, Socrates, and I hope you also will continue to think of these matters.
\end{drama}
\end{document}