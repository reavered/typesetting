\documentclass[12pt]{report}
\usepackage[12pt]{moresize}
\usepackage[utf8]{inputenc}
\usepackage[english]{babel}
\usepackage[top=2.5cm, bottom=2.5cm, left=2.5cm, right=2.5cm]{geometry}
\usepackage{ebgaramond}

%=======SECTION HEADERS=========%
\usepackage{titlesec}
\usepackage{titletoc}

%========QUOTES=========%
\usepackage{epigraph}

%=======PARAGRAPH FORMATTING=========%
\setlength{\parindent}{0pt} %no paragraph indents
\setlength{\parskip}{1em}   %single space between paragraphs

\renewcommand{\chaptermark}[1]{\markboth{\MakeUppercase{Book \thechapter}}{}} %Book format- heading

%=======CHAPTER FORMATTING=========%
\renewcommand\thechapter{{\Roman{chapter}}}    %section numbering style

\newcommand{\mychapter}[2]{
\setcounter{chapter}{#1}
    \setcounter{section}{0}
    \chapter*{#2}
    \addcontentsline{toc}{chapter}{#2}
}

\titleformat
{\chapter} 
[display]
{\centering\fontfamily{ppl}\Huge} 
{Chapter \thechapter\vspace{-1em}} 
{\leftmargin}{\normalfont\Large}[]

%=======SECTION HEADER SPACING=========%
\titlespacing{\chapter}{0mm}{-2em}{0em}
\titlespacing{\section}{0mm}{3mm}{2mm}

%=======TITLE PAGE=========%
\title{\HUGE\bfseries{The Principles of Morals and Legislation}}
\author{\Large by Jeremy Bentham}
\date{\vspace{-4mm}1781}

%=======FOOTNOTES=========%
\renewcommand{\thefootnote}{[\arabic{footnote}]}
\setlength{\skip\footins}{1cm}
\usepackage[]{footmisc}
\renewcommand{\footnotemargin}{3mm} %Setting left margin
\renewcommand{\footnotelayout}{\hspace{2mm}} %spacing between the footnote number and the text of footnote

\usepackage{hyperref}
\hypersetup{bookmarksnumbered} %Bookmarks are numbered in the ToC when converted to PDF or EPUB

\titlecontents{chapter}% <section-type>
  [2.5em]% <left>
  {\vspace{.75em}}% <above-code>
  {\contentslabel{2.3em}\quad}% <numbered-entry-format>
  {\contentslabel{2.3em}\quad}% <numberless-entry-format>
  {\titlerule*[1pc]{.}\contentspage}% <filler-page-format>

\begin{document}

\begin{titlepage}
    \maketitle
\end{titlepage}

%=======TABLE OF CONTENTS=========%
\renewcommand*\contentsname{\vspace{-1cm} Table of Contents}
\tableofcontents

%=======MAIN DOCUMENT=========%
\addcontentsline{toc}{chapter}{Preface}
\chapter*{Preface}
The following sheets were, as the note on the opposite page expresses,
printed so long ago as the year 1780. The design, in pursuance of which
they were written, was not so extensive as that announced by the present
title. They had at that time no other destination than that of serving
as an introduction to a plan of a penal code in terminus, designed to
follow them, in the same volume.

The body of the work had received its completion according to the then
present extent of the author's views, when, in the investigation of some
flaws he had discovered, he found himself unexpectedly entangled in an
unsuspected corner of the metaphysical maze. A suspension, at first not
apprehended to be more than a temporary one, necessarily ensued:
suspension brought on coolness, and coolness, aided by other concurrent
causes, ripened into disgust.

Imperfections pervading the whole mass had already been pointed out by
the sincerity of severe and discerning friends; and conscience had
certified the justness of their censure. The inordinate length of some
of the chapters, the apparent inutility of others, and the dry and
metaphysical turn of the whole, suggested an apprehension, that, if
published in its present form, the work would contend under great
disadvantages for any chance, it might on other accounts possess, of
being read, and consequently of being of use.

But, though in this manner the idea of completing the present work slid
insensibly aside, that was not by any means the case with the
considerations which had led him to engage in it. Every opening, which
promised to afford the lights he stood in need of, was still pursued: as
occasion arose the several departments connected with that in which he
had at first engaged, were successively explored; insomuch that, in one
branch or other of the pursuit, his researches have nearly embraced the
whole field of legislation.

Several causes have conspired at present to bring to light, under this
new title, a work which under its original one had been imperceptibly,
but as it had seemed irrevocably, doomed to oblivion. In the course of
eight years, materials for various works, corresponding to the different
branches of the subject of legislation, had been produced, and some
nearly reduced to shape: and, in every one of those works, the
principles exhibited in the present publication had been found so
necessary, that, either to transcribe them piecemeal, or to exhibit them
somewhere where they could be referred to in the lump, was found
unavoidable. The former course would have occasioned repetitions too
bulky to be employed without necessity in the execution of a plan
unavoidably so voluminous: the latter was therefore indisputably the
preferable one.

To publish the materials in the form in which they were already printed,
or to work them up into a new one, was therefore the only alternative:
the latter had all along been his wish, and, had time and the requisite
degree of alacrity been at command, it would as certainly have been
realised. Cogent considerations, however, concur, with the irksomeness
of the task, in placing the accomplishment of it at present at an
unfathomable distance.

Another consideration is, that the suppression of the present work, had
it been ever so decidedly wished, is no longer altogether in his power.
In the course of so long an interval, various incidents have introduced
copies into various hands, from some of which they have been transferred
by deaths and other accidents, into others that are unknown to him.
Detached, but considerable extracts, have even been published, without
any dishonourable views (for the name of the author was very honestly
subjoined to them), but without his privity, and in publications
undertaken without his knowledge.

It may perhaps be necessary to add, to complete his excuse for offering
to the public a work pervaded by blemishes, which have not escaped even
the author's partial eye, that the censure, so justly bestowed upon the
form, did not extend itself to the matter.

In sending it thus abroad into the world with all its imperfections upon
its head, he thinks it may be of assistance to the few readers he can
expect, to receive a short intimation of the chief particulars, in
respect of which it fails of corresponding with his maturer views. It
will thence be observed how in some respects it fails of quadrating with
the design announced by its original title, as in others it does with
that announced by the one it bears at present.

An introduction to a work which takes for its subject the totality of
any science, ought to contain all such matters, and such matters only,
as belong in common to every particular branch of that science, or at
least to more branches of it than one. Compared with its present title,
the present work fails in both ways of being conformable to that rule.
As an introduction to the principles of \emph{morals,} in addition to
the analysis it contains of the extensive ideas signified by the terms
\emph{pleasure, pain, motive,} and \emph{disposition,} it ought to have
given a similar analysis of the not less extensive, though much less
determinate, ideas annexed to the terms \emph{emotion, passion,
appetite, virtue, vice,} and some others, including the names of the
particular \emph{virtues} and \emph{vices.} But as the true, and, if he
conceives right, the only true groundwork for the development of the
latter set of terms, has been laid by the explanation of the former, the
completion of such a dictionary, so to style it, would, in comparison of
the commencement, be little more than a mechanical operation.

Again, as an introduction to the principles of \emph{legislation in
general,} it ought rather to have included matters belonging exclusively
to the \emph{civil} branch, than matters more particularly applicable to
the \emph{penal:} the latter being but a means of compassing the ends
proposed by the former. In preference therefore, or at least in
priority, to the several chapters which will be found relative to
\emph{punishment,} it ought to have exhibited a set of propositions
which have since presented themselves to him as affording a standard for
the operations performed by government, in the creation and distribution
of proprietary and other civil rights. He means certain axioms of what
may be termed \emph{mental pathology,} expressive of the connection
betwixt the feelings of the parties concerned, and the several classes
of incidents, which either call for, or are produced by, operations of
the nature above mentioned. \textsuperscript{1} The consideration of the
division of offences, and every thing else that belongs to offences,
ought, besides, to have preceded the consideration of punishment: for
the idea of \emph{punishment} presupposes the idea of \emph{offence:}
punishment, as such, not being inflicted but in consideration of
offence.

Lastly, the analytical discussions relative to the classification of
offences would, according to his present views, be transferred to a
separate treatise, in which the system of legislation is considered
solely in respect of its form: in other words, in respect of its
\emph{method} and \emph{terminology.}

In these respects the performance fails of coming up to the author's own
ideas of what should have been exhibited in a work, bearing the title he
has now given it. viz., that of an \emph{Introduction to the Principles
of Morals and Legislation.} He knows however of no other that would be
less unsuitable: nor in particular would so adequate an intimation of
its actual contents have been given, by a title corresponding to the
more limited design, with which it was written: viz., that of serving as
an \emph{introduction to a penal code.} \emph{}

Yet more. Dry and tedious as a great part of the discussions it contains
must unavoidably be found by the bulk of readers, he knows not how to
regret the having written them, nor even the having made them public.
Under every head, the practical uses, to which the discussions contained
under that head appeared applicable, are indicated: nor is there, he
believes, a single proposition that he has not found occasion to build
upon in the penning of some article or other of those provisions of
detail, of which a body of law, authoritative or unauthoritative, must
be composed. He will venture to specify particularly, in this view, the
several chapters shortly characterized by the words \emph{Sensibility,
Actions, Intentionality, Consciousness, Motives, Dispositions,
Consequences.} \emph{}

Even in the enormous chapter on the division of offenses, which,
notwithstanding the forced compression the plan has undergone in several
of its parts, in manner there mentioned, occupies no fewer than one
hundred and four closely printed quarto pages, the ten concluding ones
are employed in a statement of the practical advantages that may be
reaped from the plan of classification which it exhibits. Those in whose
sight the \emph{Defence of Usury} has been fortunate enough to find
favour, may reckon as one instance of those advantages the discovery of
the principles developed in that little treatise. In the preface to an
anonymous tract published so long ago as in 1776, \textsuperscript{2} he
had hinted at the utility of a natural classification of offenses, in
the character of a test for distinguishing genuine from spurious ones.
The case of usury is one among a number of instances of the truth of
that observation. A note at the end of Sect. xxxv. chap. xvi. of the
present publication, may serve to show how the opinions, developed in
that tract, owed their origin to the difficulty experienced in the
attempt to find a place in his system for that imaginary offense. To
some readers, as a means of helping them to support the fatigue of
wading through an analysis of such enormous length, he would almost
recommend the beginning with those ten concluding pages.

One good at least may result from the present publication; viz., that
the more he has trespassed on the patience of the reader on this
occasion, the less need he will have so to do on future ones: so that
this may do to those, the office which is done, by books of pure
mathematics, to books of mixed mathematics and natural philosophy. The
narrower the circle of readers is, within which the present work may be
condemned to confine itself, the less limited may be the number of those
to whom the fruits of his succeeding labours may be found accessible. He
may therefore in this respect find himself in the condition of those
philosophers of antiquity, who are represented as having held two bodies
of doctrine, a popular and an occult one: but, with this difference,
that in his instance the occult and the popular will, he hopes, be found
as consistent as in those they were contradictory; and that in his
production whatever there is of occultness has been the pure result of
sad necessity, and in no respect of choice.

Having, in the course of this advertisement, had such frequent occasion
to allude to different arrangements, as having been suggested by more
extensive and maturer views, it may perhaps contribute to the
satisfaction of the reader, to receive a short intimation of their
nature: the rather, as, without such explanation, references, made here
and there to unpublished works, might be productive of perplexity and
mistake. The following then are the titles of the works by the
publication of which his present designs would be completed. They are
exhibited in the order which seemed to him best fitted for apprehension,
and in which they would stand disposed, were the whole assemblage ready
to come out at once: but the order, in which they will eventually
appear, may probably enough be influenced in some degree by collateral
and temporary considerations.

Part the 1st. Principles of legislation in matters of \emph{civil,} more
distinctively termed \emph{private distributive,} or for shortness,
\emph{distributive,} law.

Part the 2nd. Principles of legislation in matters of \emph{penal law.}
\emph{}

Part the 3rd. Principles of legislation in matters of \emph{procedure:}
uniting in one view the \emph{criminal} and \emph{civil} branches,
between which no line can be drawn, but a very indistinct one, and that
continually liable to variation.

Part the 4th. Principles of legislation in matters of \emph{reward.}

Part the 5th. Principles of legislation in matters of \emph{public
distributive,} more concisely as well as familiarly termed
\emph{constitutional,} law.

Part the 6th. Principles of legislation in matters of \emph{political
tactics:} or of the art of maintaining \emph{order} in the proceedings
of political assemblies, so as to direct them to the end of their
institution: viz., by a system of rules, which are to the constitutional
branch, in some respects, what the law of procedure is to the civil and
the penal.

Part the 7th. Principles of legislation in matters betwixt nation and
nation, or, to use a new though not inexpressive appellation, in matters
of \emph{international} law.

Part the 8th. Principles of legislation in matters of \emph{finance.}

Part the 9th. Principles of legislation in matters of \emph{political
economy.}

Part the 10th. Plan of a body of law, complete in all its branches,
considered in respect of its \emph{form;} in other words, in respect of
its method and terminology; including a view of the origination and
connexion of the ideas expressed by the short list of terms, the
exposition of which contains all that can be said with propriety to
belong to the head of \emph{universal jurisprudence.}

The use of the principles laid down under the above several heads is to
prepare the way for the body of law itself exhibited \emph{in terminis;}
and which to be complete, with reference to any political state, must
consequently be calculated for the meridian, and adapted to the
circumstances, of some one such state in particular.

Had he an unlimited power of drawing upon \emph{time,} and every other
condition necessary, it would be his wish to postpone the publication of
each part to the completion of the whole. In particular, the use of the
ten parts, which exhibit what appear to him the dictates of utility in
every line, being no other than to furnish reasons for the several
corresponding provisions contained in the body of law itself, the exact
truth of the former can never be precisely ascertained, till the
provisions, to which they are destined to apply, are themselves
ascertained, and that \emph{in terminis.} But as the infirmity of human
nature renders all plans precarious in the execution, in proportion as
they are extensive in the design, and as he has already made
considerable advances in several branches of the theory, without having
made correspondent advances in the practical applications, he deems it
more than probable, that the eventual order of publication will not
correspond exactly with that which, had it been equally practicable,
would have appeared most eligible. Of this irregularity the unavoidable
result will be, a multitude of imperfections, which, if the execution of
the body of law \emph{in terminis} had kept pace with the development of
the principles, so that each part had been adjusted and corrected by the
other, might have been avoided. His conduct however will be the less
swayed by this inconvenience, from his suspecting it to be of the number
of those in which the personal vanity of the author is much more
concerned, than the instruction of the public: since whatever amendments
may be suggested in the detail of the principles, by the literal
fixation of the provisions to which they are relative, may easily be
made in a corrected edition of the former, succeeding upon the
publication of the latter.

In the course of the ensuing pages, references will be found, as already
intimated, some to the plan of a penal code to which this work was meant
as an introduction, some to other branches of the above-mentioned
general plan, under titles somewhat different from those, by which they
have been mentioned here. The giving this warning is all which it is in
the author's power to do, to save the reader from the perplexity of
looking out for what has not as yet any existence. The recollection of
the change of plan will in like manner account for several similar
incongruities not worth particularizing.

Allusion was made, at the outset of this advertisement, to some
unspecified difficulties, as the causes of the original suspension, and
unfinished complexion, of the present work. Ashamed of his defeat, and
unable to dissemble it, he knows not how to reface himself the benefit
of such an apology as a slight sketch of the nature of those
difficulties may afford.. The discovery of them was produced by the
attempt to solve the questions that will be found at the conclusion of
the volume: \emph{Wherein consisted the identity and} completeness
\emph{of a law? What the distinction, and where the separation, between
a} penal \emph{and a} civil \emph{law? What the distinction, and where
the separation, between the} penal \emph{and} other branches \emph{of}
the law?

To give a complete and correct answer to these questions, it is but too
evident that the relations and dependencies of every part of the
legislative system, with respect to every other, must have been
comprehended and ascertained. But it is only upon a view of these parts
themselves, that such an operation could have been performed. To the
accuracy of such a survey one necessary condition would therefore be,
the complete existence of the fabric to be surveyed. To the performance
of this condition no example is as yet to be met with any where.
\emph{Common} law, as it styles itself in England, \emph{judiciary} law
as it might aptly be styled every where. that fictitious composition
which has no known person for its author, no known assemblage of words
for its substance, forms every where the main body of the legal fabric:
like that fancied ether, which, in default of sensible matter, fills up
the measure of the universe. Shreds and scraps of real law, stuck on
upon that imaginary ground, compose the furniture of every national
code. What follows?'' that he who, for the purpose just
mentioned or for any other, wants an example of a complete body of law
to refer to, must begin with making one.

There is, or rather there ought to be, a \emph{logic} of the
\emph{will.} as well as of the \emph{understanding:} the operations of
the former faculty, are neither less susceptible, nor less worthy, then
those of the latter, of being delineated by rules. Of these two branches
of that recondite art, Aristotle saw only the latter: succeeding
logicians, treading in the steps of their great founder, have concurred
in seeing with no other eyes. Yet so far as a difference can be assigned
between branches so intimately connected, whatever difference there is,
in point of importance, is in favour of the logic of the will. Since it
is only by their capacity of directing the operations of this faculty,
that the operations of the understanding are of any consequence. Of this
logic of the will, the science of \emph{law,} considered in respect of
its \emph{form,} is the most considerable branch,'' the most
important application. It is, to the art of legislation, what the
science of anatomy is to the art of medicine: with this difference, that
the subject of it is what the artist has to work \emph{with,} instead of
being what he has to operate \emph{upon.} Nor is the body politic less
in danger from a want of acquaintance with the one science, than the
body natural from ignorance in the other. One example, amongst a
thousand that might be adduced in proof of this assertion, may be seen
in the note which terminates this volume. Such then were the
difficulties: such the preliminaries:'' an unexampled work to
achieve, and then a new science to create: a new branch to add to one of
the most abstruse of sciences.

Yet more: a body of proposed law, how complete soever, would be
comparatively useless and uninstructive, unless explained and justified,
and that in every tittle, by a continued accompaniment, a perpetual
commentary of reasons: which reasons, that the comparative value of such
as point in opposite directions may be estimated, and the conjunct
force, of such as point in the same direction may be felt. must be
marshalled, and put under subordination to such extensive and leading
ones as are termed \emph{principles.} There must be therefore, not one
system only, but two parallel and connected systems, running on
together. the one of legislative provisions, the other of political
reasons, each affording to the other correction and support.

Are enterprises like these achievable? He knows not. This only he knows,
that they have been undertaken, proceeded in, and that some progress has
been made in all of them. He will venture to add, if at all achievable,
never at least by one, to whom the fatigue of attending to discussions,
as arid as those which occupy the ensuing pages, would either appear
useless, or feel intolerable. He will repeat it boldly (for it has been
said before him), truths that form the basis of political and moral
science are not to be discovered but by investigations as severe as
mathematical ones, and beyond all comparison more intricate and
extensive. The familiarity of the terms is a presumption, but is a most
fallacious one, of the facility of the matter. Truths in general have
been called stubborn things: the truths just mentioned are so in their
own way. They are not to be forced into detached and general
propositions, unincumbered with explanations and exceptions. They will
not compress themselves into epigrams. They recoil from the tongue and
the pen of the declaimer. They flourish not in the same soil with
sentiment. They grow among thorns; and are not to be plucked, like
daisies, by infants as they run. Labour, the inevitable lot of humanity,
is in no track more inevitable than here. In vain would an Alexander
bespeak a peculiar road for royal vanity, or a Ptolemy, a smoother one,
for royal indolence. There is no \emph{King's Road,} no
\emph{Stadtholder's Gate,} to legislative, any more than to mathematic
science.

\titlespacing{\chapter}{0mm}{-2em}{1em}
\chapter{Of The Principle of Utility}

I. Nature has placed mankind under the governance of two sovereign
masters, \emph{pain} and \emph{pleasure.} It is for them alone to point
out what we ought to do, as well as to determine what we shall do. On
the one hand the standard of right and wrong, on the other the chain of
causes and effects, are fastened to their throne. They govern us in all
we do, in all we say, in all we think: every effort we can make to throw
off our subjection, will serve but to demonstrate and confirm it. In
words a man may pretend to abjure their empire: but in reality he will
remain subject to it all the while. The \emph{principle of utility}
recognizes this subjection, and assumes it for the foundation of that
system, the object of which is to rear the fabric of felicity by the
hands of reason and of law. Systems which attempt to question it, deal
in sounds instead of sense, in caprice instead of reason, in darkness
instead of light.

But enough of metaphor and declamation: it is not by such means that
moral science is to be improved.

II. The principle of utility is the foundation of the present work: it
will be proper therefore at the outset to give an explicit and
determinate account of what is meant by it. By the principle of utility
is meant that principle which approves or disapproves of every action
whatsoever. according to the tendency it appears to have to augment or
diminish the happiness of the party whose interest is in question: or,
what is the same thing in other words to promote or to oppose that
happiness. I say of every action whatsoever, and therefore not only of
every action of a private individual, but of every measure of
government.

III. By utility is meant that property in any object, whereby it tends
to produce benefit, advantage, pleasure, good, or happiness, (all this
in the present case comes to the same thing) or (what comes again to the
same thing) to prevent the happening of mischief, pain, evil, or
unhappiness to the party whose interest is considered: if that party be
the community in general, then the happiness of the community: if a
particular individual, then the happiness of that individual.

IV. The interest of the community is one of the most general expressions
that can occur in the phraseology of morals: no wonder that the meaning
of it is often lost. When it has a meaning, it is this. The community is
a fictitious \emph{body,} composed of the individual persons who are
considered as constituting as it were its \emph{members.} The interest
of the community then is, what is it?'' the sum of the interests
of the several members who compose it.

V. It is in vain to talk of the interest of the community, without
understanding what is the interest of the individual. A thing is said to
promote the interest, or to be \emph{for} the interest, of an
individual, when it tends to add to the sum total of his pleasures: or,
what comes to the same thing, to diminish the sum total of his pains.

VI. An action then may be said to be conformable to then principle of
utility, or, for shortness sake, to utility, (meaning with respect to
the community at large) when the tendency it has to augment the
happiness of the community is greater than any it has to diminish it.

VII. A measure of government (which is but a particular kind of action,
performed by a particular person or persons) may be said to be
conformable to or dictated by the principle of utility, when in like
manner the tendency which it has to augment the happiness of the
community is greater than any which it has to diminish it.

VIII. When an action, or in particular a measure of government, is
supposed by a man to be conformable to the principle of utility, it may
be convenient, for the purposes of discourse, to imagine a kind of law
or dictate, called a law or dictate of utility: and to speak of the
action in question, as being conformable to such law or dictate.

IX. A man may be said to be a partizan of the principle of utility, when
the approbation or disapprobation he annexes to any action, or to any
measure, is determined by and proportioned to the tendency which he
conceives it to have to augment or to diminish the happiness of the
community: or in other words, to its conformity or unconformity to the
laws or dictates of utility.

X. Of an action that is conformable to the principle of utility one may
always say either that it is one that ought to be done, or at least that
it is not one that ought not to be done. One may say also, that it is
right it should be done; at least that it is not wrong it should be
done: that it is a right action; at least that it is not a wrong action.
When thus interpreted, the words \emph{ought,} and \emph{right} and
\emph{wrong} and others of that stamp, have a meaning: when otherwise,
they have none.

XI. Has the rectitude of this principle been ever formally contested? It
should seem that it had, by those who have not known what they have been
meaning. Is it susceptible of any direct proof? it should seem not: for
that which is used to prove every thing else, cannot itself be proved: a
chain of proofs must have their commencement somewhere. To give such
proof is as impossible as it is needless.

XII. Not that there is or ever has been that human creature at
breathing, however stupid or perverse, who has not on many, perhaps on
most occasions of his life, deferred to it. By the natural constitution
of the human frame, on most occasions of their lives men in general
embrace this principle, without thinking of it: if not for the ordering
of their own actions, yet for the trying of their own actions, as well
as of those of other men. There have been, at the same time, not many
perhaps, even of the most intelligent, who have been disposed to embrace
it purely and without reserve. There are even few who have not taken
some occasion or other to quarrel with it, either on account of their
not understanding always how to apply it, or on account of some
prejudice or other which they were afraid to examine into, or could not
bear to part with. For such is the stuff that man is made of: in
principle and in practice, in a right track and in a wrong one, the
rarest of all human qualities is consistency.

XIII. When a man attempts to combat the principle of utility, it is with
reasons drawn, without his being aware of it, from that very principle
itself. His arguments, if they prove any thing, prove not that the
principle is \emph{wrong,} but that, according to the applications he
supposes to be made of it, it is \emph{misapplied.} Is it possible for a
man to move the earth? Yes; but he must first find out another earth to
stand upon.

XIV. To disprove the propriety of it by arguments is impossible; but,
from the causes that have been mentioned, or from some confused or
partial view of it, a man may happen to be disposed not to relish it.
Where this is the case, if he thinks the settling of his opinions on
such a subject worth the trouble, let him take the following steps, and
at length, perhaps, he may come to reconcile himself to it.
\begin{enumerate}
	\item Let him settle with himself, whether he would wish to discard this
principle altogether; if so, let him consider what it is that all his
reasonings (in matters of politics especially) can amount to?
	\item If he would, let him settle with himself, whether he would judge and
act without any principle, or whether there is any other he would judge
an act by?
	\item If there be, let him examine and satisfy himself whether the
principle he thinks he has found is really any separate intelligible
principle; or whether it be not a mere principle in words, a kind of
phrase, which at bottom expresses neither more nor less than the mere
averment of his own unfounded sentiments; that is, what in another
person he might be apt to call caprice?
	\item If he is inclined to think that his own approbation or
disapprobation, annexed to the idea of an act, without any regard to its
consequences, is a sufficient foundation for him to judge and act upon,
let him ask himself whether his sentiment is to be a standard of right
and wrong, with respect to every other man, or whether every man's
sentiment has the same privilege of being a standard to itself?
	\item In the first case, let him ask himself whether his principle is not
despotical, and hostile to all the rest of human race?
	\item In the second case, whether it is not anarchial, and whether at this
rate there are not as many different standards of right and wrong as
there are men? and whether even to the same man, the same thing, which
is right today, may not (without the least change in its nature) be
wrong tomorrow? and whether the same thing is not right and wrong in the
same place at the same time? and in either case, whether all argument is
not at an end? and whether, when two men have said, ``I like this,'' and
``I don't like it,'' they can (upon such a principle) have any thing
more to say?
	\item If he should have said to himself, No: for that the sentiment which
he proposes as a standard must be grounded on reflection, let him say on
what particulars the reflection is to turn? if on particulars having
relation to the utility of the act, then let him say whether this is not
deserting his own principle, and borrowing assistance from that very one
in opposition to which he sets it up: or if not on those particulars, on
what other particulars?
	\item If he should be for compounding the matter, and adopting his own
principle in part, and the principle of utility in part, let him say how
far he will adopt it?
	\item When he has settled with himself where he will stop, then let him ask
himself how he justifies to himself the adopting it so far? and why he
will not adopt it any farther?
	\item Admitting any other principle than the principle of utility to be a
right principle, a principle that it is right for a man to pursue;
admitting (what is not true) that the word \emph{right} can have a
meaning without reference to utility, let him say whether there is any
such thing as a \emph{motive} that a man can have to pursue the dictates
of it: if there is, let him say what that motive is, and how it is to be
distinguished from those which enforce the dictates of utility: if not,
then lastly let him say what it is this other principle can be good for?
\end{enumerate}

\chapter{Of Principles Adverse to that of Utility}

I. If the principle of utility be a right principle to be governed by,
and that in all cases, it follows from what has been just observed, that
whatever principle differs from it in any case must necessarily be a
wrong one. To prove any other principle, therefore, to be a wrong one,
there needs no more than just to show it to be what it is, a principle
of which the dictates are in some point or other different from those of
the principle of utility: to state it is to confute it.

II. A principle may be different from that of utility in two ways: 1. By
being constantly opposed to it: this is the case with a principle which
may be termed the principle of \emph{asceticism.} 2. By being sometimes
opposed to it, and sometimes not, as it may happen: this is the case
with another, which may be termed the principle of \emph{sympathy} and
\emph{antipathy.} \emph{}

III. By the principle of asceticism I mean that principle, which, like
the principle of utility, approves or disapproves of any action,
according to the tendency which it appears to have to augment or
diminish the happiness of the party whose interest is in question; but
in an inverse manner: approving of actions in as far as they tend to
diminish his happiness; disapproving of them in as far as they tend to
augment it.

IV. It is evident that any one who reprobates any the least particle of
pleasure, as such, from whatever source derived, is \emph{pro tanto} a
partizan of the principle of asceticism. It is only upon that principles
and not from the principle of utility, that the most abominable pleasure
which the vilest of malefactors ever reaped from his crime would be to
be reprobated, if it stood alone. The case is, that it never does stand
alone; but is necessarily followed by such a quantity of pain (or, what
comes to the same thing, such a chance for a certain quantity of pain)
that, the pleasure in comparison of it, is as nothing: and this is the
true and sole, but perfectly sufficient, reason for making it a ground
for punishment.

V. There are two classes of men of very different complexions, by whom
the principle of asceticism appears to have been embraced; the one a set
of moralists, the other a set of religionists. Different accordingly
have been the motives which appears to have recommended it to the notice
of these different parties. Hope, that is the prospect of pleasure,
seems to have animated the former: hope, the aliment of philosophic
pride: the hope of honour and reputation at the hands of men. Fear, that
is the prospect of pain, the latter: fear, the offspring of
superstitious fancy: the fear of future punishment at the hands of a
splenetic and revengeful Deity. I say in this case fear: for of the
invisible future, fear is more powerful than hope. These circumstances
characterize the two different parties among the partisans of the
principle of asceticism; the parties and their motives different, the
principle the same.

VI. The religious party, however, appear to have carried it farther than
the philosophical: they have acted more consistently and less wisely.
The philosophical party have scarcely gone farther than to reprobate
pleasure: the religious party have frequently gone so far as to make it
a matter of merit and of duty to court pain. The philosophical party
have hardly gone farther than the making pain a matter of indifference.
It is no evil, they have said: they have not said, it is a good. They
have not so much as reprobated all pleasure in the lump. They have
discarded only what they have called the gross; that is, such as are
organical, or of which the origin is easily traced up to such as are
organical: they have even cherished and magnified the refined. Yet this,
however, not under the name of pleasure: to cleanse itself from the
sordes of its impure original, it was necessary it should change its
name: the honourable, the glorious, the reputable, the becoming, the
\emph{honestum,} the \emph{decorum} it was to be called: in short, any
thing but pleasure.

VII. From these two sources have flowed the doctrines from it which the
sentiments of the bulk of mankind have all along received a tincture of
this principle; some from the philosophical, some from the religious,
some from both. Men of education more frequently from the philosophical,
as more suited to the elevation of their sentiments: the vulgar more
frequently from the superstitious, as more suited to the narrowness of
their intellect, undilated by knowledge and to the abjectness of their
condition, continually open to the attacks of fear. The tinctures,
however, derived from the two sources, would naturally intermingle,
insomuch that a man would not always know by which of them he was most
influenced: and they would often serve to corroborate and enliven one
another. It was this conformity that made a kind of alliance between
parties of a complexion otherwise so dissimilar: and disposed them to
unite upon various occasions against the common enemy, the partizan of
the principle of utility, whom they joined in branding with the odious
name of Epicurean.

VIII. The principle of asceticism, however, with whatever warmth it may
have been embraced by its partizans as a rule of Private conduct, seems
not to have been carried to any considerable length, when applied to the
business of government. In a few instances it has been carried a little
way by the philosophical party: witness the Spartan regimen. Though
then, perhaps, it maybe considered as having been a measure of security:
and an application, though a precipitate and perverse application, of
the principle of utility. Scarcely in any instances, to any considerable
length, by the religious: for the various monastic orders, and the
societies of the Quakers, Dumplers, Moravians, and other religionists,
have been free societies, whose regimen no man has been astricted to
without the intervention of his own consent. Whatever merit a man may
have thought there would be in making himself miserable, no such notion
seems ever to have occurred to any of them, that it may be a merit, much
less a duty, to make others miserable: although it should seem, that if
a certain quantity of misery were a thing so desirable, it would not
matter much whether it were brought by each man upon himself, or by one
man upon another. It is true, that from the same source from whence,
among the religionists, the attachment to the principle of asceticism
took its rise, flowed other doctrines and practices, from which misery
in abundance was produced in one man by the instrumentality of another:
witness the holy wars, and the persecutions for religion. But the
passion for producing misery in these cases proceeded upon some special
ground: the exercise of it was confined to persons of particular
descriptions: they were tormented, not as men, but as heretics and
infidels. To have inflicted the same miseries on their fellow believers
and fellow-sectaries, would have been as blameable in the eyes even of
these religionists, as in those of a partizan of the principle of
utility. For a man to give himself a certain number of stripes was
indeed meritorious: but to give the same number of stripes to another
man, not consenting, would have been a sin. We read of saints, who for
the good of their souls, and the mortification of their bodies, have
voluntarily yielded themselves a prey to vermin: but though many persons
of this class have wielded the reins of empire, we read of none who have
set themselves to work, and made laws on purpose, with a view of
stocking the body politic with the breed of highwaymen, housebreakers,
or incendiaries. If at any time they have suffered the nation to be
preyed upon by swarms of idle pensioners, or useless placemen, it has
rather been from negligence and imbecility, than from any settled plan
for oppressing and plundering of the people. If at any time they have
sapped the sources of national wealth, by cramping commerce, and driving
the inhabitants into emigration, it has been with other views, and in
pursuit of other ends. If they have declaimed against the pursuit of
pleasure, and the use of wealth, they have commonly stopped at
declamation: they have not, like Lycurgus, made express ordinances for
the purpose of banishing the precious metals. If they have established
idleness by a law, it has been not because idleness, the mother of vice
and misery, is itself a virtue, but because idleness (say they) is the
road to holiness. If under the notion of fasting, they have joined in
the plan of confining their subjects to a diet, thought by some to be of
the most nourishing and prolific nature, it has been not for the sake of
making them tributaries to the nations by whom that diet was to be
supplied, but for the sake of manifesting their own power, and
exercising the obedience of the people. If they have established, or
suffered to be established, punishments for the breach of celibacy, they
have done no more than comply with the petitions of those deluded
rigorists, who, dupes to the ambitious and deep-laid policy of their
rulers, first laid themselves under that idle obligation by a vow.

IX. The principle of asceticism seems originally to have been the
reverie of certain hasty speculators, who having perceived, or fancied,
that certain pleasures, when reaped in certain circumstances, have, at
the long run, been attended with pains more than equivalent to them,
took occasion to quarrel with every thing that offered itself under the
name of pleasure. Having then got thus far, and having forgot the point
which they set out from, they pushed on, and went so much further as to
think it meritorious to fall in love with pain. Even this, we see, is at
bottom but the principle of utility misapplied.

X. The principle of utility is capable of being consistently pursued;
and it is but tautology to say, that the more consistently it is
pursued, the better it must ever be for humankind. The principle of
asceticism never was, nor ever can be, consistently pursued by any
living creature. Let but one tenth part of the inhabitants of this earth
pursue it consistently, and in a day's time they will have turned it
into a hell.

XI. Among principles adverse to that of utility, that which at this day
seems to have most influence in matters of government, is what may be
called the principle of sympathy and antipathy. By the principle of
sympathy and antipathy, I mean that principle which approves or
disapproves of certain actions, not on account of their tending to
augment the happiness, nor yet on account of their tending to diminish
the happiness of the party whose interest is in question, but merely
because a man finds himself disposed to approve or disapprove of them:
holding up that approbation or disapprobation as a sufficient reason for
itself, and disclaiming the necessity of looking out for any extrinsic
ground. Thus far in the general department of morals: and in the
particular department of politics, measuring out the quantum (as well as
determining the ground) of punishment, by the degree of the
disapprobation.

XII. It is manifest, that this is rather a principle in name than in
reality: it is not a positive principle of itself, so much as a term
employed to signify the negation of all principle. What one expects to
find in a principle is something that points out some external
consideration, as a means of warranting and guiding the internal
sentiments of approbation and disapprobation: this expectation is but
ill fulfilled by a proposition, which does neither more nor less than
hold up each of those sentiments as a ground and standard for itself.

XIII. In looking over the catalogue of human actions (says a partizan of
this principle) in order to determine which of them are to be marked
with the seal of disapprobation, you need but to take counsel of your
own feelings: whatever you find in yourself a propensity to condemn, is
wrong for that very reason. For the same reason it is also meet for
punishment: in what proportion it is adverse to utility, or whether it
be adverse to utility at all, is a matter that makes no difference. In
that same \emph{proportion} also is it meet for punishment: if you hate
much, punish much: if you hate little, punish little: punish as you
hate. If you hate not at all, punish not at all: the fine feelings of
the soul are not to be overborne and tyrannized by the harsh and rugged
dictates of political utility.

XIV. The various systems that have been formed concerning the standard
of right may all be reduced to the principle of sympathy and antipathy.
One account may serve to for all of them. They consist all of them in so
many contrivances for avoiding the obligation of appealing to any
external standard, and for prevailing upon the reader to accept of the
author's sentiment or opinion as a reason for itself. The phrases
different, but the principle the same.

XV. It is manifest, that the dictates of this principle will frequently
coincide with those of utility, though perhaps without intending any
such thing. Probably more frequently than not: and hence it is that the
business of penal justice is carried upon that tolerable sort of footing
upon which we see it carried on in common at this day. For what more
natural or more general ground of hatred to a practice can there be,
than the mischievousness of such practice? What all men are exposed to
suffer by, all men will be disposed to hate. It is far yet, however,
from being a constant ground: for when a man suffers, it is not always
that he knows what it is he suffers by. A man may suffer grievously, for
instance, by a new tax, without being able to trace up the cause of his
sufferings to the injustice of some neighbour, who has eluded the
payment of an old one.

XVI. The principle of sympathy and antipathy is most apt to err on the
side of severity. It is for applying punishment in many cases which
deserve none: in many cases which deserve some, it is for applying more
than they deserve. There is no incident imaginable, be it ever so
trivial, and so remote from mischief, from which this principle may not
extract a ground of punishment. Any difference in taste: any difference
in opinion: upon one subject as well as upon another. No disagreement so
trifling which perseverance and altercation will not render serious.
Each becomes in the other's eyes an enemy, and, if laws permit, a
criminal. This is one of the circumstances by which the human race is
distinguished (not much indeed to its advantage) from the brute
creation.

XVII. It is not, however, by any means unexampled for this principle to
err on the side of lenity. A near and perceptible mischief moves
antipathy. A remote and imperceptible mischief, though not less real,
has no effect. Instances in proof of this will occur in numbers in the
course of the work. 4 It would be breaking in upon the order of it to
give them here.

XVIII. It may be wondered, perhaps, that in all this no mention has been
made of the \emph{theological} principle; meaning that principal which
professes to recur for the standard of right and wrong to the will of
God. But the case is, this is not in fact a distinct principle. It is
never any thing more or less than one or other of the three
before-mentioned principles presenting itself under another shape. The
\emph{will} of God here meant cannot be his revealed will, as contained
in the sacred writings: for that is a system which nobody ever thinks of
recurring to at this time of day, for the details of political
administration: and even before it can be applied to the details of
private conduct, it is universally allowed, by the most eminent divines
of all persuasions, to stand in need of pretty ample interpretations;
else to what use are the works of those divines? And for the guidance of
these interpretations, it is also allowed, that some other standard must
be assumed. The will then which is meant on this occasion, is that which
may be called the \emph{presumptive} will: that is to say, that which is
presumed to be his will by virtue of the conformity of its dictates to
those of some other principle. What then may be this other principle? it
must be one or other of the three mentioned above: for there cannot, as
we have seen, be any more. It is plain, therefore, that, setting
revelation out of the question, no light can ever be thrown upon the
standard of right and wrong, by any thing that can be said upon the
question, what is God's will. We may be perfectly sure, indeed, that
whatever is right is conformable to the will of God: but so far is that
from answering the purpose of showing us what is right, that it is
necessary to know first whether a thing is right, in order to know from
thence whether it be conformable to the will of God.

XIX. There are two things which are very apt to be confounded, but which
it imports us carefully to distinguish:'' the motive or cause,
which, by operating on the mind of an individual, is productive of any
act: and the ground or reason which warrants a legislator, or other
bystander, in regarding that act with an eye of approbation. When the
act happens, in the particular instance in question, to be productive of
effects which we approve of, much more if we happen to observe that the
same motive may frequently be productive, in other instances, of the
like effects, we are apt to transfer our approbation to the motive
itself, and to assume, as the just ground for the approbation we bestow
on the act, the circumstance of its originating from that motive. It is
in this way that the sentiment of antipathy has often been considered as
a just ground of action. Antipathy, for instance, in such or such a
case, is the cause of an action which is attended with good effects: but
this does not make it a right ground of action in that case, any more
than in any other. Still farther. Not only the effects are good, but the
agent sees beforehand that they will be so. This may make the action
indeed a perfectly right action: but it does not make antipathy a right
ground of action. For the same sentiment of antipathy, if implicitly
deferred to, may be, and very frequently is, productive of the very
worst effects. Antipathy, therefore, can never be a right ground of
action. No more, therefore, can resentment, which, as will be seen more
particularly hereafter, is but a modification of antipathy. The only
right ground of action, that can possibly subsist, is, after all, the
consideration of utility which, if it is a right principle of actions
and of approbation any one case, is so in every other. Other principles
in abundance, that is, other motives, may be the reasons why such and
such an act \emph{has} been done: that is, the reasons or causes of its
being done: but it is this alone that can be the reason why it might or
ought to have been done. Antipathy or resentment requires always to be
regulated, to prevent it doing mischief: to be regulated what? always by
the principle of utility. The principle of utility neither requires nor
admits of any another regulator than itself.

\chapter{Of the Four Sanctions or Sources of Pain and
Pleasure}

I. It has been shown that the happiness of the individuals, of whom a
community is composed, that is their pleasures and their security, is
the end and the sole end which the legislator ought to have in view: the
sole standard, in conformity to which each individual ought, as far as
depends upon the legislator, to be \emph{made} to fashion his behaviour.
But whether it be this or any thing else that is to be \emph{done,}
there is nothing by which a man can ultimately be \emph{made} to do it,
but either pain or pleasure. Having taken a general view of these two
grand objects (viz., pleasure, and what comes to the same thing,
immunity from pain) in the character of \emph{final} causes; it will be
necessary to take a view of pleasure and pain itself, in the character
of \emph{efficient} causes or means.

II. There are four distinguishable sources from which pleasure and pain
are in use to flow: considered separately they may be termed the
\emph{physical,} the \emph{political,} the \emph{moral} and the
\emph{religious:} and inasmuch as the pleasures and pains belonging to
each of them are capable of giving a binding force to any law or rule of
conduct, they may all of them termed \emph{sanctions.} \emph{}

III. If it be in the present life, and from the ordinary coursed of
nature, not purposely modified by the interposition of these will of any
human being, nor by any extraordinary interposition of any superior
invisible being, that the pleasure or the pain takes place or is
expected, it may be said to issue from or to belong to the
\emph{physical sanction.} \emph{}

IV. If at the hands of a \emph{particular} person or set of persons in
the community, who under names correspondent to that of \emph{judge,}
are chosen for the particular purpose of dispensing it, according to the
will of the sovereign or supreme ruling power in the state, it may be
said to issue from the \emph{political sanction.} \emph{}

V. If at the hands of such chance persons in the community, as the party
in question may happen in the course of his life to have concerns with,
according to each man's spontaneous disposition, and not according to
any settled or concerted rule, it may be said to issue from the
\emph{moral} or \emph{popular sanction.} \emph{}

VI. If from the immediate hand of a superior invisible being, either in
the present life, or in a future, it may be said to issue from the
\emph{religious sanction.} \emph{}

VII. Pleasures or pains which may be expected to issue from the
\emph{physical, political,} or \emph{moral} sanctions, must all of them
be expected to be experienced, if ever, in the \emph{present} life:
those which may be expected to issue from the \emph{religious} sanction,
may be expected to be experienced either in the \emph{present} life or
in a \emph{future.} \emph{}

VIII. Those which can be experienced in the present life, can of course
be no others than such as human nature in the course of the present life
is susceptible of: and from each of these sources may flow all the
pleasures or pains of which, in the course of the present life, human
nature is susceptible. With regard to these then (with which alone we
have in this place any concern) those of them which belong to any one of
those sanctions, differ not ultimately in kind from those which belong
to any one of the other three: the only difference there is among them
lies in the circumstances that accompany their production. A suffering
which befalls a man in the natural and spontaneous course of things,
shall be styled, for instance, a \emph{calamity;} in which case, if it
be supposed to befall him through any imprudence of his, it may be
styled a punishment issuing from the physical sanction. Now this same
suffering, if inflicted by the law, will be what is commonly called a
\emph{punishment;} if incurred for want of any friendly assistance,
which the misconduct, or supposed misconduct, of the sufferer has
occasioned to be withholden, a punishment issuing from the \emph{moral}
sanction; if through the immediate interposition of a particular
providence, a punishment issuing from the religious sanction.

IX. A man's goods, or his person, are consumed by fire. If this happened
to him by what is called an accident, it was a calamity: if by reason of
his own imprudence (for instance, from his neglecting to put his candle
out) it may be styled a punishment of the physical sanction: if it
happened to him by the sentence of the political magistrate, a
punishment belonging to the political sanction; that is, what is
commonly called a punishment: if for want of any assistance which his
\emph{neighbour} withheld from him out of some dislike to his
\emph{moral} character, a punishment of the \emph{moral} sanction: if by
an immediate act of \emph{God's} displeasure, manifested on account of
some \emph{sin} committed by him, or through any distraction of mind,
occasioned by the dread of such displeasure, a punishment of the
\emph{religious} sanction.

X. As to such of the pleasures and pains belonging to the religious
sanction, as regard a future life, of what kind these may be we cannot
know. These lie not open to our observation. During the present life
they are matter only of expectation: and, whether that expectation be
derived from natural or revealed religion, the particular kind of
pleasure or pain, if it be different from all those which he open to our
observation, is what we can have no idea of. The best ideas we can
obtain of such pains and pleasures are altogether unliquidated in point
of quality. In what other respects our ideas of them \emph{may} be
liquidated will be considered in another place.

XI. Of these four sanctions the physical is altogether, we may observe,
the groundwork of the political and the moral: so is it also of the
religious, in as far as the latter bears relation to the present life.
It is included in each of those other three. This may operate in any
case, (that is, any of the pains or pleasures belonging to it may
operate) independently of \emph{them:} none of \emph{them} can operate
but by means of this. In a word, the powers of nature may operate of
themselves; but neither the magistrate, nor men at large, \emph{can}
operate, nor is God in the case in question \emph{supposed} to operate,
but through the powers of nature.

XII. For these four objects, which in their nature have so much in
common, it seemed of use to find a common name. It seemed of use, in the
first place, for the convenience of giving a name to certain pleasures
and pains, for which a name equally characteristic could hardly
otherwise have been found: in the second place, for the sake of holding
up the efficacy of certain moral forces, the influence of which is apt
not to be sufficiently attended to. Does the political sanction exert an
influence over the conduct of mankind? The moral, the religious
sanctions do so too. In every inch of his career are the operations of
the political magistrate liable to be aided or impeded by these two
foreign powers: who, one or other of them, or both, are sure to be
either his rivals or his allies. Does it happen to him to leave them out
in his calculations? he will be sure almost to find himself mistaken in
the result. Of all this we shall find abundant proofs in the sequel of
this work. It behoves him, therefore, to have them continually before
his eyes; and that under such a name as exhibits the relation they bear
to his own purposes and designs.

\chapter{Value of a Lot of Pleasure or Pain, How to be
Measured}

I. Pleasures then, and the avoidance of pains, are the \emph{ends} that
the legislator has in view; it behoves him therefore to understand their
\emph{value.} Pleasures and pains are the instruments he has to work
with: it behoves him therefore to understand their force, which is
again, in other words, their value.

II. To a person considered by \emph{himself,} the value of a pleasure or
pain considered \emph{by itself,} will be greater or less, according to
the four following \emph{circumstances:}
\begin{enumerate}
	\item Its \emph{intensity.} 
	\item Its \emph{duration.} 
	\item Its \emph{certainty} or \emph{uncertainty.}
	\item Its \emph{propinquity} or \emph{remoteness.} 
\end{enumerate}
III. These are the circumstances which are to be considered in
estimating a pleasure or a pain considered each of them by itself. But
when the value of any pleasure or pain is considered for the purpose of
estimating the tendency of any \emph{act} by which it is produced, there
are two other circumstances to be taken into the account; these are,\\
5. Its \emph{fecundity,} or the chance it has of being followed by
sensations of the \emph{same} kind: that is, pleasures, if it be a
pleasure: pains, if it be a pain.\\
6. Its \emph{purity,} or the chance it has of not being followed by
sensations of the \emph{opposite} kind: that is, pains, if it be a
pleasure: pleasures, if it be a pain.

These two last, however, are in strictness scarcely to be deemed
properties of the pleasure or the pain itself; they are not, therefore,
in strictness to be taken into the account of the value of that pleasure
or that pain. They are in strictness to be deemed properties only of the
act, or other event, by which such pleasure or pain has been produced;
and accordingly are only to be taken into the account of the tendency of
such act or such event.

IV. To a \emph{number} of persons, with reference to each of whom to the
value of a pleasure or a pain is considered, it will be greater or less,
according to seven circumstances: to wit, the six preceding ones;
viz.,\\
1. Its \emph{intensity.} \emph{\\
}2. Its \emph{duration.} \emph{\\
}3. Its \emph{certainty} or \emph{uncertainty.} \emph{\\
}4. Its \emph{propinquity} or \emph{remoteness.} \emph{\\
}5. Its \emph{fecundity.} \emph{\\
}6. Its \emph{purity.} \emph{\\
}And one other; to wit:\\
7. Its \emph{extent;} that is, the number of persons to whom it
\emph{extends;} or (in other words) who are affected by it.

V. To take an exact account then of the general tendency of any act, by
which the interests of a community are affected, proceed as follows.
Begin with any one person of those whose interests seem most immediately
to be affected by it: and take an account,\\
1. Of the value of each distinguishable \emph{pleasure} which appears to
be produced by it in the \emph{first} instance.\\
2. Of the value of each \emph{pain} which appears to be produced by it
in the \emph{first} instance.\\
3. Of the value of each pleasure which appears to be produced by it
\emph{after} the first. This constitutes the \emph{fecundity} of the
first \emph{pleasure} and the \emph{impurity} of the first \emph{pain.}
\emph{\\
}4. Of the value of each \emph{pain} which appears to be produced by it
after the first. This constitutes the \emph{fecundity} of the first
\emph{pain,} and the \emph{impurity} of the first pleasure.\\
5. Sum up all the values of all the \emph{pleasures} on the one side,
and those of all the pains on the other. The balance, if it be on the
side of pleasure, will give the \emph{good} tendency of the act upon the
whole, with respect to the interests of that \emph{individual} person;
if on the side of pain, the \emph{bad} tendency of it upon the whole.\\
6. Take an account of the \emph{number} of persons whose interests
appear to be concerned; and repeat the above process with respect to
each. \emph{Sum up} the numbers expressive of the degrees of \emph{good}
tendency, which the act has, with respect to each individual, in regard
to whom the tendency of it is \emph{good} upon the whole: do this again
with respect to each individual, in regard to whom the tendency of it is
\emph{good} upon the whole: do this again with respect to each
individual, in regard to whom the tendency of it is \emph{bad} upon the
whole. Take the \emph{balance} which if on the side of \emph{pleasure,}
will give the general \emph{good tendency} of the act, with respect to
the total number or community of individuals concerned; if on the side
of pain, the general \emph{evil tendency,} with respect to the same
community.

VI. It is not to be expected that this process should be strictly
pursued previously to every moral judgment, or to every legislative or
judicial operation. It may, however, be always kept in view: and as near
as the process actually pursued on these occasions approaches to it, so
near will such process approach to the character of an exact one.

VII. The same process is alike applicable to pleasure and pain, in
whatever shape they appear: and by whatever denomination they are
distinguished: to pleasure, whether it be called \emph{good} (which is
properly the cause or instrument of pleasure) or \emph{profit} (which is
distant pleasure, or the cause or instrument of, distant pleasure,) or
\emph{convenience,} or \emph{advantage, benefit, emolument, happiness,}
and so forth: to pain, whether it be called \emph{evil,} (which
corresponds to \emph{good)} or \emph{mischief,} or \emph{inconvenience}
or \emph{disadvantage,} or \emph{loss,} or \emph{unhappiness,} and so
forth.

VIII. Nor is this a novel and unwarranted, any more than it is a useless
theory. In all this there is nothing but what the practice of mankind,
wheresoever they have a clear view of their own interest, is perfectly
conformable to. An article of property, an estate in land, for instance,
is valuable, on what account? On account of the pleasures of all kinds
which it enables a man to produce, and what comes to the same thing the
pains of all kinds which it enables him to avert. But the value of such
an article of property is universally understood to rise or fall
according to the length or shortness of the time which a man has in it:
the certainty or uncertainty of its coming into possession: and the
nearness or remoteness of the time at which, if at all, it is to come
into possession. As to the \emph{intensity} of the pleasures which a man
may derive from it, this is never thought of, because it depends upon
the use which each particular person may come to make of it; which
cannot be estimated till the particular pleasures he may come to derive
from it, or the particular pains he may come to exclude by means of it,
are brought to view. For the same reason, neither does he think of the
\emph{fecundity} or \emph{purity} of those pleasures. Thus much for
pleasure and pain, happiness and unhappiness, in \emph{general.} We come
now to consider the several particular kinds of pain and pleasure.

\chapter{Pleasures and Pains, Their Kinds}

I. Having represented what belongs to all sorts of pleasures and pains
alike, we come now to exhibit, each by itself, the several sorts of
pains and pleasures. Pains and pleasures may be called by one general
word, interesting perceptions. Interesting perceptions are either simple
or complex. The simple ones are those which cannot any one of them be
resolved into more: complex are those which are resolvable into divers
simple ones. A complex interesting perception may accordingly be
composed either, 1. Of pleasures alone: 2. Of pains alone: or, 3. Of a
pleasure or pleasures, and a pain or pains together. What determines a
lot of pleasure, for example, to be regarded as one complex pleasure,
rather than as divers simple ones, is the nature of the exciting cause.
Whatever pleasures are excited all at once by the action of the same
cause, are apt to be looked upon as constituting all together but one
pleasure.

II. The several simple pleasures of which human nature is susceptible,
seem to be as follows:\\
1. The pleasures of sense.\\
2. The pleasures of wealth.\\
3. The pleasures of skill.\\
4. The pleasures of amity.\\
5. The pleasures of a good name.\\
6. The pleasures of power.\\
7. The pleasures of piety.\\
8. The pleasures of benevolence.\\
9. The pleasures of malevolence.\\
10. The pleasures of memory.\\
11. The pleasures of imagination.\\
12. The pleasures of expectation.\\
13. The pleasures dependent on association.\\
14. The pleasures of relief.

III. The several simple pains seem to be as follows:\\
1. The pains of privation.\\
2. The pains of the senses.\\
3. The pains of awkwardness.\\
4. The pains of enmity.\\
5. The pains of an ill name.\\
6. The pains of piety.\\
7. The pains of benevolence.\\
8. The pains of malevolence.\\
9. The pains of the memory.\\
10. The pains of the imagination.\\
11. The pains of expectation\\
12. The pains dependent on association.

IV. 1. The pleasures of sense seem to be as follows:\\
1. The pleasures of the taste or palate; including whatever pleasures
are experienced in satisfying the appetites of hunger and thirst.\\
2. The pleasure of intoxication.\\
3. The pleasures of the organ of smelling.\\
4. The pleasures of the touch.\\
5. The simple pleasures of the ear; independent of association. 6. The
simple pleasures of the eye; independent of association.\\
7. The pleasure of the sexual sense.\\
8. The pleasure of health: or, the internal pleasureable feeling or flow
of spirits (as it is called), which accompanies a state of full health
and vigour; especially at times of moderate bodily exertion.\\
9. The pleasures of novelty: or, the pleasures derived from the
gratification of the appetite of curiosity, by the application of new
objects to any of the senses.

V. 2. By the pleasures of wealth may be meant those pleasures which a
man is apt to derive from the consciousness of possessing any article or
articles which stand in the list of instruments of enjoyment or
security, and more particularly at the time of his first acquiring them;
at which time the pleasure may be styled a pleasure of gain or a
pleasure of acquisition: at other times a pleasure of possession.

3. The pleasures of skill, as exercised upon particular objects, are
those which accompany the application of such particular instruments of
enjoyment to their uses, as cannot be so applied without a greater or
less share of difficulty or exertion.

VI. 4. The pleasures of amity, or self-recommendation, are the pleasures
that may accompany the persuasion of a man's being in the acquisition or
the possession of the good-will of such or such assignable person or
persons in particular: or, as the phrase is, of being upon good terms
with him or them: and as a fruit of it, of his being in a way to have
the benefit of their spontaneous and gratuitous services.

VII. 5. The pleasures of a good name are the pleasures that accompany
the persuasion of a man's being in the acquisition or the possession of
the good-will of the world about him; that is, of such members of
society as he is likely to have concerns with; and as a means of it,
either their love or their esteem, or both: and as a fruit of it, of his
being in the way to have the benefit of their spontaneous and gratuitous
services. These may likewise be called the pleasures of good repute, the
pleasures of honour, or the pleasures of the moral sanction.

VIII. 6. The pleasures of power are the pleasures that accompany the
persuasion of a man's being in a condition to dispose people, by means
of their hopes and fears, to give him the benefit of their services:
that is, by the hope of some service, or by the fear of some disservice,
that he may be in the way to render them.

IX. 7. The pleasures of piety are the pleasures that accompany the
belief of a man's being in the acquisition or in possession of the
good-will or favour of the Supreme Being: and as a fruit of it, of his
being in a way of enjoying pleasures to be received by God's special
appointment, either in this life, or in a life to come. These may also
be called the pleasures of religion, the pleasures of a religious
disposition, or the pleasures of the religious sanction.

X. 8. The pleasures of benevolence are the pleasures resulting from the
view of any pleasures supposed to be possessed by the beings who may be
the objects of benevolence; to wit, the sensitive beings we are
acquainted with; under which are commonly included,\\
1. The Supreme Being.\\
2. Human beings.\\
3. Other animals. These may also be called the pleasures of good-will,
the pleasures of sympathy, or the pleasures of the benevolent or social
affections.

XI. 9. The pleasures of malevolence are the pleasures resulting from the
view of any pain supposed to be suffered by the beings who may become
the objects of malevolence: to wit, 2. Other animals. These may also be
styled the pleasures of ill-will, the pleasures of the irascible
appetite, the pleasures of antipathy, or the pleasures of the malevolent
or dissocial affections.

XII. 10. The pleasures of the memory are the pleasures which, after
having enjoyed such and such pleasures, or even in some case after
having suffered such and such pains, a man will now and then experience,
at recollecting them exactly in the order and in the circumstances in
which they were actually enjoyed or suffered. These derivative pleasures
may of course be distinguished into as many species as there are of
original perceptions, from whence they may be copied. They may also be
styled pleasures of simple recollection.

XIII. 11. The pleasures of the imagination are the pleasures which may
be derived from the contemplation of any such pleasures as may happen to
be suggested by the memory, but in a different order, and accompanied by
different groups of circumstances. These may accordingly be referred to
any one of the three cardinal points of time, present, past, or future.
It is evident they may admit of as many distinctions as those of the
former class.

XIV. 12. The pleasures of expectation are the pleasures that result from
the contemplation of any sort of pleasure, referred to time
\emph{future,} and accompanied with the sentiment of \emph{belief.}
These also may admit of the same distinctions.

XV. 13. The pleasures of association are the pleasures which certain
objects or incidents may happen to afford, not of themselves, but merely
in virtue of some association they have contracted in the mind with
certain objects or incidents which are in themselves pleasurable. Such
is the case, for instance, with the pleasure of skill, when afforded by
such a set of incidents as compose a game of chess. This derives its
pleasurable quality from its association partly with the pleasures of
skill, as exercised in the production of incidents pleasurable of
themselves: partly from its association with the pleasures of power.
Such is the case also with the pleasure of good luck, when afforded by
such incidents as compose the game of hazard, or any other game of
chance, when played at for nothing. This derives its pleasurable quality
from its association with one of the pleasures of wealth; to wit, with
the pleasure of acquiring it.

XVI. 14. Farther on we shall see pains grounded upon pleasures; in like
manner may we now see pleasures grounded upon pains. To the catalogue of
pleasures may accordingly be added the pleasures of \emph{relief:} or,
the pleasures which a man experiences when, after he has been enduring a
pain of any kind for a certain time, it comes to cease, or to abate.
These may of course be distinguished into as many species as there are
of pains: and may give rise to so many pleasures of memory, of
imagination, and of expectation.

XVII. 1. Pains of privation are the pains that may results from the
thought of not possessing in the time present any of the several kinds
of pleasures. Pains of privation may accordingly be resolved into as
many kinds as there are of pleasures to which they may correspond, and
from the absence whereof they may be derived.

XVIII. There are three sorts of pains which are only so many
modifications of the several pains of privation. When the enjoyment of
any particular pleasure happens to be particularly desired, but without
any expectation approaching to assurance, the pain of privation which
thereupon results takes a particular name, and is called the pain of
\emph{desire,} or of unsatisfied desire.

XIX. Where the enjoyment happens to have been looked for with a degree
of expectation approaching to assurance, and that expectation is made
suddenly to cease, it is called a pain of disappointment.

XX. A pain of privation takes the name of a pain of regret in two
cases:\\
1. Where it is grounded on the memory of a pleasure, which having been
once enjoyed, appears not likely to be enjoyed again:\\
2. Where it is grounded on the idea of a pleasure, which was never
actually enjoyed, nor perhaps so much as expected, but which might have
been enjoyed (it is supposed,) had such or such a contingency happened,
which, in fact, did not happen.

XXI. 2. The several pains of the senses seem to be as follows:\\
1. The pains of hunger and thirst: or the disagreeable sensations
produced by the want of suitable substances which need at times to be
applied to the alimentary canal.\\
2. The pains of the taste: or the disagreeable sensations produced by
the application of various substances to the palate, and other superior
parts of the same canal.\\
3. The pains of the organ of smell: or the disagreeable sensations
produced by the effluvia of various substances when applied to that
organ.\\
4. The pains of the touch: or the disagreeable sensations produced by
the application of various substances to the skin.\\
5. The simple pains of the hearing: or the disagreeable sensations
excited in the organ of that sense by various kinds of sounds:
independently (as before,) of association.\\
6. The simple pains of the sight: or the disagreeable sensations if any
such there be, that may be excited in the organ of that sense by visible
images, independent of the principle of association.\\
7. The pains resulting from excessive heat or cold, unless these be
referable to the touch.\\
8. The pains of disease: or the acute and uneasy sensations resulting
from the several diseases and indispositions to which human nature is
liable.\\
9. The pain of exertion, whether bodily or mental: or the uneasy
sensation which is apt to accompany any intense effort, whether of mind
or body.

XXII. 3. The pains of awkwardness are the pains which sometimes result
from the unsuccessful endeavour to apply any particular instruments of
enjoyment or security to their uses, or from the difficulty a man
experiences in applying them.

XXIII. 4. The pains of enmity are the pains that may accompany the
persuasion of a man's being obnoxious to the ill-will of such or such an
assignable person or persons in particular: or, as the phrase is, of
being upon ill terms with him or them: and, in consequence, of being
obnoxious to certain pains of some sort or other, of which he may be the
cause.

XXIV. 5. The pains of an ill-name, are the pains that accompany the
persuasion of a man's being obnoxious, or in a way to be obnoxious to
the ill-will of the world about him. These may likewise be called the
pains of ill-repute, the pains of dishonour, or the pains of the moral
sanction.

XXV. 6. The pains of piety are the pains that accompany the belief of a
man's being obnoxious to the displeasure of the Supreme Being: and in
consequence to certain pains to be inflicted by his especial
appointment, either in this life or in a life to come. These may also be
called the pains of religion; the pains of a religious disposition; or
the pains of the religious sanction. When the belief is looked upon as
well-grounded, these pains are commonly called religious terrors; when
looked upon as ill-grounded, superstitious terrors.

XXVI. 7. The pains of benevolence are the pains resulting from the view
of any pains supposed to be endured by other beings. These may also be
called the pains of good-will, of sympathy, or the pains of the
benevolent or social affections.

XXVII. 8. The pains of malevolence are the pains resulting from the view
of any pleasures supposed to be enjoyed by any beings who happen to be
the objects of a man's displeasure. These may also be styled the pains
of ill-will, of antipathy, or the pains of the malevolent or dissocial
affections.

XXVIII. 9. The pains of the memory may be grounded on every one of the
above kinds, as well of pains of privation as of positive pains. These
correspond exactly to the pleasures of the memory.

XXIX. 10. The pains of the imagination may also be grounded on any one
of the above kinds, as well of pains of privation as of positive pains:
in other respects they correspond exactly to the pleasures of the
imagination.

XXX. 11. The pains of expectation may be grounded on each one of the
above kinds, as well of pains of privation as of positive pains. These
may be also termed pains of apprehension.

XXXI. 12. The pains of association correspond exactly to the pleasures
of association.

XXXII. Of the above list there are certain pleasures and pains which
suppose the existence of some pleasure or pain, of some other person, to
which the pleasure or pain of the person in question has regard: such
pleasures and pains may be termed \emph{extra-regarding.} Others do not
suppose any such thing: these may be termed \emph{self-regarding.} The
only pleasures and pains of the extra-regarding class are those of
benevolence and those of malevolence: all the rest are self-regarding.

XXXIII. Of all these several sorts of pleasures and pains, there is
scarce any one which is not liable, on more accounts than one, to come
under the consideration of the law. Is an offense committed? It is the
tendency which it has to destroy, in such or such persons, some of these
pleasures, or to produce some of these pains, that constitutes the
mischief of it, and the ground for punishing it. It is the prospect of
some of these pleasures, or of security from some of these pains, that
constitutes the motive or temptation, it is the attainment of them that
constitutes the profit of the offense. Is the offender to be punished?
It can be only by the production of one or more of these pains, that the
punishment can be inflicted.

\chapter{Chapter VI: Of Circumstances Influencing Sensibility}

I. Pain and pleasure are produced in men's minds by the action of
certain causes. But the quantity of pleasure and pain runs not uniformly
in proportion to the cause; in other words, to the quantity of force
exerted by such cause. The truth of this observation rests not upon any
metaphysical nicety in the import given to the terms \emph{cause,
quantity,} and \emph{force:} it will be equally true in whatsoever
manner such force be measured.

II. The disposition which any one has to feel such or such a quantity of
pleasure or pain, upon the application of a cause of given force, is
what we term the degree or \emph{quantum} of his sensibility. This may
be either \emph{general} referring to the sum of the causes that act
upon him during a given period: or \emph{particular,} referring to the
action of any one particular cause, or sort of cause.

III. But in the same mind such and such causes of pain or pleasure will
produce more pain or pleasure than such or such other causes of pain or
pleasure: and this proportion will in different minds be different. The
disposition which any one has to have the proportion in which he is
affected by two such causes, different from that in which another man is
affected by the same two causes, may be termed the quality or
\emph{bias} of his sensibility. One man, for instance, may be most
affected by the pleasures of the taste; another by those of the ear. So
also, if there be a difference in the nature or proportion of two pains
or pleasures which they respectively experience from the same cause; a
case not so frequent as the former. From the same injury, for instance,
one man may feel the same quantity of grief and resentment together as
another man: but one of them shall feel a greater share of grief than of
resentment: the other, a greater share of resentment than of grief.

IV. Any incident which serves as a cause, either of pleasure or of pain,
may be termed an \emph{exciting} cause: if of pleasure, a pleasurable
cause: if of pain, a painful, afflictive, or dolorific cause.

V. Now the quantity of pleasure, or of pain, which a man is liable to
experience upon the application of an exciting cause, since they will
not depend altogether upon that cause, will depend in some measure upon
some other circumstance or circumstances: these circumstances,
whatsoever they be, maybe termed circumstances influencing sensibility.

VI. These circumstances will apply differently to different exciting
causes; insomuch that to a certain exciting cause, a certain
circumstance shall not apply at all, which shall apply with great force
to another exciting cause. But without entering for the present into
these distinctions, it may be of use to sum up all the circumstances
which can be found to influence the effect of any exciting cause. These,
as on a former occasion, it may be as well first to sum up together in
the concisest manner possible, and afterwards to allot a few words to
the separate explanation of each article. They seem to be as follows:\\
1. Health.\\
2. Strength.\\
3. Hardiness.\\
4. Bodily imperfection.\\
5. Quantity and quality of knowledge.\\
6. Strength of intellectual powers.\\
7. Firmness of mind.\\
8. Steadiness of mind.\\
9. Bent of inclination.\\
10. Moral sensibility.\\
11. Moral biases.\\
12. Religious sensibility.\\
13. Religious biases.\\
14. Sympathetic sensibility.\\
15. Sympathetic biases.\\
16. Antipathetic sensibility.\\
17. Antipathetic biases\\
18. Insanity.\\
19. Habitual occupations.\\
20. Pecuniary circumstances.\\
21. Connexions in the way of sympathy.\\
22. Connexions in the way of antipathy.\\
23. Radical frame of body.\\
24. Radical frame of mind.\\
25. Sex.\\
26. Age.\\
27. Rank.\\
28. Education.\\
29. Climate.\\
30. Lineage.\\
31. Government.\\
32. Religious profession.

VII. 1. Health is the absence of disease, and consequently of all those
kinds of pain which are among the symptoms of disease. A man may be said
to be in a state of health when he is not conscious of any uneasy
sensations, the primary seat of which can be perceived to be anywhere in
his body. In point of general sensibility, a man who is under the
pressure of any bodily indisposition, or, as the phrase is, is in an ill
state of health, is less sensible to the influence of any pleasurable
cause, and more so to that of any afflictive one, than if he were
\emph{well.}

VIII. 2. The circumstance of strength, though in point of causality
closely connected with that of health, is perfectly distinguishable from
it. The same man will indeed generally be stronger in a good state of
health than in a bad one. But one man, even in a bad state of health,
may be stronger than another even in a good one. Weakness is a common
concomitant of disease: but in consequence of his radical frame of body,
a man may be weak all his life long, without experiencing any disease.
Health, as we have observed, is principally a negative circumstance:
strength a positive one. The degree of a man's strength can be measured
with tolerable accuracy.

IX. 3. Hardiness is a circumstance which, though closely connected with
that of strength, is distinguishable from it. Hardiness is the absence
of irritability. Irritability respects either pain, resulting from the
action of mechanical causes; or disease, resulting from the action of
causes purely physiological. Irritability, in the former sense, is the
disposition to undergo a greater or less degree of pain upon the
application of a mechanical cause; such as are most of those
applications by which simple afflictive punishments are inflicted, as
whipping, beating, and the like. In the latter sense, it is the
disposition to contract disease with greater or less facility, upon the
application of any instrument acting on the body by its physiological
properties; as in the case of fevers, or of colds, or other inflammatory
diseases, produced by the application of damp air: or to experience
immediate uneasiness, as in the case of relaxation or chilliness
produced by an over or under proportion of the matter of heat.
Hardiness, even in the sense in which it is opposed to the action of
mechanical causes, is distinguishable from strength. The external
indications of strength are the abundance and firmness of no the
muscular fibres: those of hardiness, in this sense, are the firmness of
the muscular fibres, and the callosity of the skin. Strength is more
peculiarly the gift of nature: hardiness, of education. Of two persons
who have had, the one the education of a gentleman, the other, that of a
common sailor, the first may be the stronger, at the same time that the
other is the hardier.

X. 4. By bodily imperfection may be understood that condition which a
person is in, who either stands distinguished by any remarkable
deformity, or wants any of those parts or faculties, which the ordinary
run of persons of the same sex and age are furnished with: who, for
instance, has a hare-lip, is deaf, or has lost a hand. This
circumstance, like that of ill-health, tends in general to diminish more
or less the effect of any pleasurable circumstance, and to increase that
of any afflictive one. The effect of this circumstance, however, admits
of great variety: inasmuch as there are a great variety of ways in which
a man may suffer in his personal appearance, and in his bodily organs
and faculties: all which differences will be taken notice of in their
proper places.

XI. 5. So much for circumstances belonging to the condition of the body:
we come now to those which concern the condition of the mind: the use of
mentioning these will be seen hereafter. In the first place may be
reckoned the quantity and quality of the knowledge the person in
question happens to possess: that is, of the ideas which he has actually
in stores ready upon occasion to call to mind: meaning such ideas as are
in some way or other of an interesting nature: that is, of a nature in
some way or other to influence his happiness, or that of other men. When
these ideas are many, and of importance, a man is said to be a man of
knowledge; when few, or not of importance, \emph{ignorant.} \emph{}

XII. 6. By strength of intellectual powers may be understood the degree
of facility which a man experiences in his endeavours to call to mind as
well such ideas as have been already aggregated to his stock of
knowledge, as any others, which, upon any occasion that may happen, he
may conceive a desire to place there. It seems to be on some such
occasion as this that the words \emph{parts} and \emph{talents} are
commonly employed. To this head may be referred the several qualities of
readiness of apprehension, accuracy and tenacity of memory, strength of
attention, clearness of discernment, amplitude of comprehension,
vividity and rapidity of imagination. Strength of intellectual powers,
in general, seems to correspond pretty exactly to general strength of
body: as any of these qualities in particular does to particular
strength.

XIII. 7. Firmness of mind on the one hand, and irritability on the
other, regard the proportion between the degrees of efficacy with which
a man is acted upon by an exciting cause, of which the value lies
chiefly in magnitude, and one of which the value lies chiefly in
propinquity. A man may be said to be of a firm mind, when small
pleasures or pains, which are present or near, do not affect him, in a
greater proportion to their value, than greater pleasures or pains,
which are uncertain or remote; Of an irritable mind, when the contrary
is the case.

XIV. 8. Steadiness regards the time during which a given exciting cause
of a given value continues to affect a man in nearly the same manner and
degree as at first, no assignable external event or change of
circumstances intervening to make an alteration in its force.

XV. 9. By the bent of a man's inclinations may be understood the
propensity he has to expect pleasure or pain from certain objects,
rather than from others. A man's inclinations may be said to have such
or such a bent, when, amongst the several sorts of objects which afford
pleasure in some degree to all men, he is apt to expect more pleasure
from one particular sort, than from another particular sort, or more
from any given particular sort, than another man would expect from that
sort; or when, amongst the several sorts of objects, which to one man
afford pleasure, whilst to another they afford none, he is apt to
expect, or not to expect, pleasure from an object of such or such a
sort: so also with regard to pains. This circumstance, though intimately
connected with that of the bias of a man's sensibility, is not
undistinguishable from it. The quantity of pleasure or pain, which on
any given occasion a man may experience from an application of any sort,
may be greatly influenced by the expectations he has been used to
entertain of pleasure or pain from that quarter; but it will not be
absolutely determined by them: for pleasure or pain may come upon him
from a quarter from which he was not accustomed to expect it.

XVI. 10. The circumstances of \emph{moral, religious, sympathetic,} and
\emph{antipathetic} sensibility, when closely considered, will appear to
be included in some sort under that of \emph{bent of inclination.} On
account of their particular importance they may, however, be worth
mentioning apart. A man's moral sensibility may be said to be strong,
when the pains and pleasures of the moral sanction show greater in his
eyes, in comparison with other pleasures and pains (and consequently
exert a stronger influence) than in the eyes of the persons he is
compared with; in other words, when he is acted on with more than
ordinary efficacy by the sense of honour: it may be said to be weak,
when the contrary is the case. \textless{}

XVII. 11. Moral sensibility seems to regard the average effect or
influence of the pains and pleasures of the moral sanction, upon all
sorts of occasions to which it is applicable, or happens to be applied.
It regards the average force or \emph{quantity} of the impulses the mind
receives from that source during a given period. Moral \emph{bias}
regards the particular acts on which, upon so many particular occasions,
the force of that sanction is looked upon as attaching. It regards the
\emph{quality} or direction of those impulses. It admits of as many
varieties, therefore, as there are dictates which the moral sanction may
be conceived to issue forth. A man may be said to have such or such a
\emph{moral bias,} or to have a moral bias in favour of such or such an
action, when he looks upon it as being of the number of those of which
the performance is dictated by the moral sanction.\\
XVIII. 12. What has been said with regard to moral sensibility, may be
applied, \emph{mutatis mutandis,} to religious.

XIX. 13. What has been said with regard to moral biases, may also be
applied, \emph{mutatis mutandis,} to religious biases.

XX. 14. By sympathetic sensibility is to be understood the propensity
that a man has to derive pleasure from the happiness, and pain from the
unhappiness, of other sensitive beings. It is the stronger, the greater
the ratio of the pleasure or pain he feels on their account is to that
of the pleasure or pain which (according to what appears to him) they
feel for themselves.

XXI. 15. Sympathetic bias regards the description of the parties who are
the objects of a man's sympathy: and of the acts or other circumstances
of or belonging to those persons, by which the sympathy is excited.
These parties may be,\\
1. Certain individuals.\\
2. Any subordinate class of individuals.\\
3. The whole nation.\\
4. Human kind in general.\\
5. The whole sensitive creation.\\
According as these objects of sympathy are more numerous, the
\emph{affection,} by which the man is biased, may be said to be the more
\emph{enlarged.} \emph{}

XXII. 16, 17. Antipathetic sensibility and antipathetic biases are just
the reverse of sympathetic sensibility and sympathetic biases. By
antipathetic sensibility is to be understood the propensity that a man
has to derive pain from the happiness, and pleasure from the
unhappiness, of other sensitive beings.

XXIII. 18. The circumstance of insanity of mind corresponds to that of
bodily imperfection. It admits, however, of much less variety, inasmuch
as the soul is (for aught we can perceive) one indivisible thing, not
distinguishable, like the body, into parts. What lesser degrees of
imperfection the mind may be susceptible of, seem to be comprisable
under the already-mentioned heads of ignorance, weakness of mind,
irritability, or unsteadiness; or under such others as are reducible to
them. Those which are here in view are those extraordinary species and
degrees of mental imperfection, which, wherever they take place, are as
conspicuous and as unquestionable as lameness or blindness in the body:
operating partly, it should seem, by inducing an extraordinary degree of
the imperfections above mentioned, partly by giving an extraordinary and
preposterous bent to the inclinations.

XXIV. 19. Under the head of a man's habitual occupations, are to be
understood, on this occasion, as well those which he pursues for the
sake of profit, as those which he pursues for the sake of present
pleasure.

The consideration of the profit itself belongs to the head of a man's
pecuniary circumstances. It is evident, that if by any means a
punishment, or any other exciting cause, has the effect of putting it
out of his power to continue in the pursuit of any such occupation, it
must on that account be much the more distressing. A man's habitual
occupations, though intimately connected in point of causality with the
bent of his inclinations, are not to be looked upon as precisely the
same circumstance. An amusement, or channel of profit, may be the object
of a man's \emph{inclinations,} which has never been the subject of his
\emph{habitual occupations:} for it may be, that though he wished to
betake himself to it, he never did, it not being in his power: a
circumstance which may make a good deal of difference in the effect of
any incident by which he happens to be debarred from it.

XXV. 20. Under the head of pecuniary circumstances, I mean to bring to
view the proportion which a man's means bear to his wants: the sum total
of his means of every kind, to the sum total of his wants of every kind.
A man's means depend upon three circumstances: 1. His property. 2. The
profit of his labour. 3. His connexions in the way of support. His wants
seem to depend upon four circumstances. 1. His habits of expense. 2. His
connexions in the way of burthen. 3. Any present casual demand he may
have. 4. The strength of his expectation. By a man's property is to be
understood, whatever he has in store independent of his labour. By the
profit of his labour is to be understood the growing profit. As to
labour, it may be either of the body principally, or of the mind
principally, or of both indifferently: nor does it matter in what
manner, nor on what subject, it be applied, so it produce a profit. By a
man's connexions in the way of support, are to be understood the
pecuniary assistances, of whatever kind, which he is in a way of
receiving from any persons who, on whatever account, and in whatever
proportion, he has reason to expect should contribute \emph{gratis} to
his maintenance: such as his parents, patrons, and relations. It seems
manifest, that a man can have no other means than these. What he uses,
he must have either of his own, or from other people: if from other
people, either \emph{gratis} or for a price. As to habits of expense, it
is well known, that a man's desires are governed in a great degree by
his habits. Many are the cases in which desire (and consequently the
pain of privation connected with it) would not even subsist at all, but
for previous enjoyment. By a man's connexions in the way of burthen, are
to be understood whatever expense he has reason to look upon himself as
bound to be at in the support of those who by law, or the customs of the
world, are warranted n looking up to him for assistance; such as
children, poor relations, superannuated servants, and any other
dependents whatsoever. As to present casual demand, it is manifest, that
there are occasions on which a given sum will be worth infinitely more
to a man than the same sum would at another time: where, for example, in
a case of extremity, a man stands in need of extraordinary medical
assistance: or wants money to carry on a law-suit, on which his all
depends: or has got a livelihood waiting for him in a distant country,
and wants money for the charges of conveyance. In such cases, any piece
of good or ill fortune, in the pecuniary way, might have a very
different effect from what it would have at any other time. With regard
to strength of expectation; when one man expects to gain or to keep a
thing which another does not, it is plain the circumstance of not having
it will affect the former very differently from the latter; who, indeed,
commonly will not be affected by it at all.

XXVI. 21. Under the head of a man's connexions in the way of sympathy, I
would bring to view the number and description of the persons in whose
welfare he takes such a concern, as that the idea of their happiness
should be productive of pleasure, and that of their unhappiness of pain
to him: for instance, a man's wife, his children, his parents, his near
relations, and intimate friends. This class of persons, it is obvious,
will for the most part include the two classes by which his pecuniary
circumstances are affected: those, to wit, from whose means he may
expect support, and those whose wants operate on him as a burthen. But
it is obvious, that besides these, it may very well include others, with
whom he has no such pecuniary connexion: and even with regard to these,
it is evident that the pecuniary dependence, and the union of
affections, are circumstances perfectly distinguishable. Accordingly,
the connexions here in question, independently of any influence they may
have on a man's pecuniary circumstances, have an influence on the effect
of any exciting causes whatsoever. The tendency of them is to increase a
man's general sensibility; to increase, on the one hand, the pleasure
produced by all pleasurable causes; on the other, the pain produced by
all afflictive ones. When any pleasurable incident happens to a man, he
naturally, in the first moment, thinks of the pleasure it will afford
immediately to himself: presently afterwards, however (except in a few
cases, which is not worth while here to insist on) he begins to think of
the pleasure which his friends will feel upon their coming to know of
it: and this secondary pleasure is commonly no mean addition to the
primary one. First comes the self-regarding pleasure: then comes the
idea of the pleasure of sympathy, which you suppose that pleasure of
yours will give birth to in the bosom of your friend: and this idea
excites again in yours a new pleasure of sympathy, grounded upon his.
The first pleasure issuing from your own bosom, as it were from a
radiant point, illuminates the bosom of your friend: reverberated from
thence, it is reflected with augmented warmth to the point from whence
it first proceeded: and so it is with pains.\\
Nor does this effect depend wholly upon affection. Among near relations,
although there should be no kindness, the pleasures and pains of the
moral sanction are quickly propagated by a peculiar kind of sympathy: no
article, either of honour or disgrace, can well fall upon a man, without
extending to a certain distance within the circle of his family. What
reflects honour upon the father, reflects honour upon the son: what
reflects disgrace, disgrace. The \emph{cause} of this singular and
seemingly unreasonable circumstance (that is, its analogy to the rest of
the phenomena of the human mind,) belongs not to the present purpose. It
is sufficient if the effect be beyond dispute.

XXVII. 22. Of a man's connexions in the way of antipathy, there needs
not any thing very particular to be observed. Happily there is no
primeval and constant source of antipathy in a human nature, as there is
of sympathy. There are no permanent sets of persons who are naturally
and of course the objects of antipathy to a man, as there are who are
the objects of the contrary affection. Sources, however, but too many,
of antipathy, are apt to spring up upon various occasions during the
course of a man's life: and whenever they do, this circumstance may have
a very considerable influence on the effects of various exciting causes.
As on the one hand, a punishment, for instance, which tends to separate
a man from those with whom he is connected in the way of sympathy, so on
the other hand, one which tends to force him into the company of those
with whom he is connected in the way of antipathy, will, on that
account, be so much the more distressing. It is to be observed, that
sympathy itself multiplies the sources of antipathy. Sympathy for your
friend gives birth to antipathy on \emph{your} part against all those
who are objects of antipathy, as well as to sympathy for those who are
objects of sympathy to \emph{him.} In the same manner does antipathy
multiply the sources of sympathy; though commonly perhaps with rather a
less degree of efficacy. Antipathy against your enemy is apt to give
birth to sympathy on \emph{your} part towards those who are objects of
antipathy, as well as to antipathy against those who are objects of
sympathy, to \emph{him.}

XXVIII. 23. Thus much for the circumstances by which the effect of any
exciting cause may be influenced, when applied upon any given occasion,
at any given period. But besides these supervening incidents, there are
other circumstances relative to a man, that may have their influence,
and which are coeval to his birth. In the first place, it seems to be
universally agreed, that in the original frame or texture of every man's
body, there is a something which, independently of all subsequently
intervening circumstances, renders him liable to be affected by causes
producing bodily pleasure or pain, in a manner different from that in
which another man would be affected by the same causes. To the catalogue
of circumstances influencing a man's sensibility, we may therefore add
his original or radical frame, texture, constitution, or temperament of
body.

XXIX. 24. In the next place, it seems to be pretty well agreed, that
there is something also in the original frame or texture of every man's
mind, which, independently of all exterior and subsequently intervening
circumstances, and even of his radical frame of body, makes him liable
to be differently affected by the same exciting causes, from what
another man would be. To the catalogue of circumstances influencing a
man's sensibility, we may therefore further add his original or radical
frame, texture, constitution or temperament of mind.

XXX. It seems pretty certain, all this while, that a man's sensibility
to causes producing pleasure or pain, even of mind, may depend in a
considerable degree upon his original and acquired frame of body. But we
have no reason to think that it can depend altogether upon that frame:
since, on the one hand, we see persons whose frame of body is as much
alike as can be conceived, differing very considerably in respect of
their mental frame: and, on the other hand, persons whose frame of mind
is as much alike as can be conceived, differing very conspicuously in
regard to their bodily frame.

XXXI. It seems indisputable also, that the different sets of a external
occurrences that may befall a man in the course of his life, will make
great differences in the subsequent texture of his mind at any given
period: yet still those differences are not solely to be attributed to
such occurrences. Equally far from the truth seems that opinion to be
(if any such be maintained) which attributes all to nature, and that
which attributes all to education. The two circumstances will therefore
still remain distinct, as well from one another, as from all others.

XXXII. Distinct however as they are, it is manifest, that at no period
in the active part of a man's life can they either of them make their
appearance by themselves. All they do is to constitute the latent
groundwork which the other supervening circumstances have to work upon
and whatever influence those original principles may have, is so changed
and modified, and covered over, as it were, by those other
circumstances, as never to be separately discernible. The effects of the
one influence are indistinguishably blended with those of the other.

XXXIII. The emotions of the body are received, and with reason, as
probable indications of the temperature of the mind. But they are far
enough from conclusive. A man may exhibit, for instance, the exterior
appearances of grief, without really grieving at all, or at least in any
thing near the proportion in which he appears to grieve. Oliver
Cromwell, whose conduct indicated a heart more than ordinarily callous,
was as remarkably profuse in tears. 5 Many men can command the external
appearances of sensibility with very little real feeling. The female sex
commonly with greater facility than the male: hence the proverbial
expression of a woman's tears. To have this kind of command over one's
self, was the characteristic excellence of the orator of ancient times,
and is still that of the player in our own.

XXXIV. The remaining circumstances may, with reference to those already
mentioned, be termed \emph{secondary} influencing circumstances. These
have an influence, it is true, on the quantum or bias of a man's
sensibility, but it is only by means of the other primary ones. The
manner in which these two sets of circumstances are concerned, is such
that the primary ones do the business, while the secondary ones lie most
open to observation. The secondary ones, therefore, are those which are
most heard of; on which account it will be necessary to take notice of
them: at the same time that it is only by means of the primary ones that
their influence can be explained; whereas the influence of the primary
ones will be apparent enough, without any mention of the secondary ones.

XXXV. 25. Among such of the primitive modifications of the corporeal
frame as may appear to influence the quantum and bias of sensibility,
the most obvious and conspicuous are those which constitute the sex. In
point of quantity, the sensibility of the female sex appears in general
to be greater than that of the male. The health of the female is more
delicate than that of the male: in point of strength and hardiness of
body, in point of quantity and quality of knowledge, in point of
strength of intellectual powers, and firmness of mind, she is commonly
inferior: moral, religious, sympathetic, and antipathetic sensibility
are commonly stronger in her than in the male. The quality of her
knowledge, and the bent of her inclinations, are commonly in many
respects different. Her moral biases are also, in certain respects,
remarkably different: chastity, modesty, and delicacy, for instance, are
prized more than courage in a woman: courage, more than any of those
qualities, in a man. The religious biases in the two sexes are not apt
to be remarkably different; except that the female is rather more
inclined than the male to superstition; that is, to observances not
dictated by the principle of utility; a difference that may be pretty
well accounted for by some of the before-mentioned circumstances. Her
sympathetic biases are in many respects different; for her own offspring
all their lives long, and for children in general while young, her
affection is commonly stronger than that of the male. Her affections are
apt to be less enlarged: seldom expanding themselves so much as to take
in the welfare of her country in general, much less that of mankind, or
the whole sensitive creation: seldom embracing any extensive class or
division, even of her own countrymen, unless it be in virtue of her
sympathy for some particular individuals that belong to it. In general,
her antipathetic, as well as sympathetic biases are apt to be less
conformable to the principle of utility than those of the male; owing
chiefly to some deficiency in point of knowledge, discernment, and
comprehension. Her habitual occupations of the amusing kind are apt to
be in many respects different from those of the male. With regard to her
connexions in the way of sympathy, there can be no difference. In point
of pecuniary circumstances, according to the customs of perhaps all
countries, she is in general less independent.

XXXVI. 26. Age is of course divided into divers periods, of which the
number and limits are by no means uniformly ascertained. One might
distinguish it, for the present purpose, into, 1. Infancy. 2.
Adolescence. 3. Youth. 4. Maturity. 5. Decline. 6. Decrepitude. It were
lost time to stop on the present occasion to examine it at each period,
and to observe the indications it gives, with respect to the several
primary circumstances just reviewed. Infancy and decrepitude are
commonly inferior to the other periods, in point of health, strength,
hardiness, and so forth. In infancy, on the part of the female, the
imperfections of that sex are enhanced: on the part of the male,
imperfections take place mostly similar in quality, but greater in
quantity, to those attending the states of adolescence, youth, and
maturity in the female. In the stage of decrepitude both sexes relapse
into many of the imperfections of infancy. The generality of these
observations may easily be corrected upon a particular review.

XXXVII. 27. Station, or rank in life, is a circumstance, that, among a
civilized people, will commonly undergo a multiplicity of variations.
\emph{Cæteris Paribus,} the quantum of sensibility appears to be
greater in the higher ranks of men than in the lower. The primary
circumstances in respect of which this secondary circumstance is apt to
induce or indicate a difference, seem principally to be as follows:\\
1. Quantity and Quality of knowledge.\\
2. Strength of mind.\\
3. Bent of inclination.\\
4. Moral sensibility.\\
5. Moral biases.\\
6. Religious sensibility.\\
7. Religious biases.\\
8. Sympathetic sensibility.\\
9. Sympathetic biases.\\
10. Antipathetic sensibility.\\
11. Antipathetic biases.\\
12. Habitual occupations.\\
13. Nature and productiveness of a man's means of livelihood.\\
14. Connexions importing profit.\\
15. Habit of expense.\\
16. Connexions importing burthen. A man of a certain rank will
frequently have a number of dependents besides those whose dependency is
the result of natural relationship. As to health, strength, and
hardiness, if rank has any influence on these circumstances, it is but
in a remote way chiefly by the influence it may have on its habitual
occupations.

XXXVIII. 28. The influence of education is still more extensive.
Education stands upon a footing somewhat different from that of the
circumstances of age, sex, and rank. These words, though the influence
of the circumstances they respectively denote exerts itself principally,
if not entirely, through the medium of certain of the primary
circumstances before mentioned, present, however, each of them a
circumstance which has a separate existence of itself. This is not the
case with the word education: which means nothing any farther than as it
serves to call up to view some one or more of those primary
circumstances. Education may be distinguished into physical and mental;
the education of the body and that of the mind: mental, again, into
intellectual and moral; the culture of the understanding, and the
culture of the affections. The education a man receives, is given to him
partly by others, partly by himself. By education then nothing more can
be expressed than the condition a man is in in respect of those primary
circumstances, as resulting partly from the management and contrivance
of others, principally of those who in the early periods of his life
have had dominion over him, partly from his own. To the physical part of
his education, belong the circumstances of health, strength, and
hardiness: sometimes, by accident, that of bodily imperfection; as where
by intemperance or negligence an irreparable mischief happens to his
person. To the intellectual part, those of quantity and quality of
knowledge, and in some measure perhaps those of firmness of mind and
steadiness. To the moral part, the bent of his inclinations, the
quantity and quality of his moral, religious, sympathetic, and
antipathetic sensibility: to all three branches indiscriminately, but
under the superior control of external occurrences, his habitual
recreations, his property, his means of livelihood, his connexions in
the way of profit and of burthen, and his habits of expense. With
respect indeed to all these points, the influence of education is
modified, in a manner more or less apparent, by that of exterior
occurrences; and in a manner scarcely at all apparent, and altogether
out of the reach of calculation, by the original texture and
constitution as well of his body as of his mind.

XXXIX. 29. Among the external circumstances by which the influence of
education is modified, the principal are those which come under the head
of \emph{climate.} This circumstance places itself in front, and demands
a separate denomination, not merely on account of the magnitude of its
influence, but also on account of its being conspicuous to every body,
and of its applying indiscriminately to great numbers at a time. This
circumstance depends for its \emph{essence} upon the situation of that
part of the earth which is in question, with respect to the course taken
by the whole planet in its revolution round the sun: but for its
\emph{influence} it depends upon the condition of the bodies which
compose the earth's surface at that part, principally upon the
quantities of sensible heat at different periods, and upon the density,
and purity, and dryness or moisture of the circumambient air. Of the so
often mentioned primary circumstances, there are few of which the
production is not influenced by this secondary one; partly by its
manifest effects upon the body; partly by its less perceptible effects
upon the mind. In hot climates men's health is apt to be more precarious
than in cold: their strength and hardiness less: their vigour, firmness,
and steadiness of mind less: and thence indirectly their quantity of
knowledge: the bent of their inclinations different: most remarkably so
in respect of their superior propensity to sexual enjoyments, and in
respect of the earliness of the period at which that propensity begins
to manifest itself: their sensibilities of all kinds more intense: their
habitual occupations savouring more of sloth than of activity: their
radical frame of body less strong, probably, and less hardy: their
radical frame of mind less vigorous, less firm, less steady.

XL. 30. Another article in the catalogue of secondary circumstances, is
that of \emph{race} or \emph{lineage:} the national race or lineage a
man issues from. This circumstance, independently of that of climate,
will commonly make some difference in point of radical frame of mind and
body. A man of negro race, born in France or England, is a very
different being, in many respects, from a man of French or English race.
A man of Spanish race, born in Mexico or Peru, is at the hour of his
birth a different sort of being, in many respects, from a man of the
original Mexican or Peruvian race. This circumstance, as far as it is
distinct from climate, rank, and education, and from the two just
mentioned, operates chiefly through the medium of moral, religious,
sympathetic, and antipathetic biases.

XLI. 31. The last circumstance but one, is that of government: the
government a man lives under at the time in question; or rather that
under which he has been accustomed most to live. This circumstance
operates principally through the medium of education: the magistrate
operating in the character of a tutor upon all the members of the state,
by the direction he gives to their hopes and to their fears. Indeed
under a solicitous and attentive government, the ordinary preceptor, nay
even the parent himself, is but a deputy, as it were, to the magistrate:
whose controlling influence, different in this respect from that of the
ordinary preceptor, dwells with a man to his life's end. The effects of
the peculiar power of the magistrate are seen more particularly in the
influence it exerts over the quantum and bias of men's moral, religious,
sympathetic, and antipathetic sensibilities. Under a well-constituted,
or even under a well-administered though ill-constituted government,
men's moral sensibility is commonly stronger, and their moral biases
more conformable to the dictates of utility: their religious sensibility
frequently weaker, but their religious biases less unconformable to the
dictates of utility: their sympathetic affections more enlarged,
directed to the magistrate more than to small parties or to individuals,
and more to the whole community than to either: their antipathetic
sensibilities less violent, as being more obsequious to the influence of
well-directed moral biases, and less apt to be excited by that of
ill-directed religious ones: their antipathetic biases more conformable
to well-directed moral ones, more apt (in proportion) to be grounded on
enlarged and sympathetic than on narrow and self-regarding affections,
and accordingly, upon the whole, more conformable to the dictates of
utility.

XLII. 32. The last circumstance is that of religious profession: the
religious profession a man is of: the religious fraternity of which he
is a member. This circumstance operates principally through the medium
of religious sensibility and religious biases. It operates, however, as
an indication more or less conclusive, with respect to several other
circumstances. With respect to some, scarcely but through the medium of
the two just mentioned: this is the case with regard to the quantum and
bias of a man's moral, sympathetic, and antipathetic sensibility:
perhaps in some cases with regard to quantity and quality of knowledge,
strength of intellectual powers, and bent of inclination. With respect
to others, it may operate immediately of itself: this seems to be the
case with regard to a man's habitual occupations, pecuniary
circumstances, and connexions in the way of sympathy and antipathy. A
man who pays very little inward regard to the dictates of the religion
which he finds it necessary to profess, may find it difficult to avoid
joining in the ceremonies of it, and bearing a part in the pecuniary
burthens it imposes. By force of habit and example he may even be led to
entertain a partiality for persons of the same profession, and a
proportionable antipathy against those of a rival one. In particular,
the antipathy against persons of different persuasions is one of the
last points of religion which men part with. Lastly, it is obvious, that
the religious profession a man is of cannot but have a considerable
influence on his education. But, considering the import of the term
education, to say this is perhaps no more than saying in other words
what has been said already.

XLIII. These circumstances, all or many of them, will need to be
attended to as often as upon any occasion any account is taken of any
quantity of pain or pleasure, as resulting from any cause. Has any
person sustained an injury? they will need to be considered in
estimating the mischief of the offense.. Is satisfaction to be made to
him? they will need to be attended to in adjusting the \emph{quantum} of
that satisfaction. Is the injurer to be punished? they will need to be
attended to in estimating the force of the impression that will be made
on him by any given punishment.

XLIV. It is to be observed, that though they seem all of them, on some
account or other, to merit a place in the catalogue, they are not all of
equal use in practice. Different articles among them are applicable to
different exciting causes. Of those that may influence the effect of the
same exciting cause, some apply indiscriminately to whole classes of
persons together; being applicable to all, without any remarkable
difference in degree: these may be directly and pretty fully provided
for by the legislator. This is the case, for instance, with the primary
circumstances of bodily imperfection, and insanity: with the secondary
circumstance of sex: perhaps with that of age: at any rate with those of
rank, of climate, of lineage, and of religious profession. Others,
however they may apply to whole classes of persons, yet in their
application to different individuals are susceptible of perhaps an
indefinite variety of degrees. These cannot be fully provided for by the
legislator; but, as the existence of them, in every sort of case, is
capable of being ascertained, and the degree in which they take place is
capable of being measured, provision may be made for them by the judge,
or other executive magistrate, to whom the several individuals that
happen to be concerned may be made known. This is the case, 1. With the
circumstance of health. 2. In some sort with that of strength. 3.
Scarcely with that of hardiness: still less with those of quantity and
quality of knowledge, strength of intellectual powers, firmness or
steadiness of mind; except in as far as a man's condition, in respect of
those circumstances, maybe indicated by the secondary circumstances of
sex, age, or rank: hardly with that of bent of inclination, except in as
far as that latent circumstance is indicated by the more manifest one of
habitual occupations: hardly with that of a man's moral sensibility or
biases, except in as far as they may be indicated by his sex, age, rank,
and education: not at all with his religious sensibility and religious
biases, except in as far as they may be indicated by the religious
profession he belongs to: not at all with the quantity or quality of his
sympathetic or antipathetic sensibilities, except in as far as they may
be presumed from his sex, age, rank, education, lineage, or religious
profession. It is the case, however, with his habitual occupations, with
his pecuniary circumstances, and with his connexions in the way of
sympathy. Of others, again, either the existence cannot be ascertained,
or the degree cannot be measured. These, therefore, cannot be taken into
account, either by the legislator or the executive magistrate.
Accordingly, they would have no claim to be taken notice of, were it not
for those secondary circumstances by which they are indicated, and whose
influence could not well be understood without them. What these are has
been already mentioned.

XLV. It has already been observed, that different articles in this list
of circumstances apply to different exciting causes: the circumstance of
bodily strength, for instance, has scarcely any influence of itself
(whatever it may have in a roundabout way, and by accident) on the
effect of an incident which should increase or diminish the quantum of a
man's property. It remains to be considered, what the exciting causes
are with which the legislator has to do. These may, by some accident or
other, be any whatsoever: but those which he has principally to do, are
those of the painful or afflictive kind. With pleasurable ones he has
little to do, except now and then by accident: the reasons of which may
be easily enough perceived, at the same time that it would take up too
much room to unfold them here. The exciting causes with which he has
principally to do, are, on the one hand, the mischievous acts, which it
is his business to prevent; on the other hand, the punishments, by the
terror of which it is his endeavour to prevent them. Now of these two
sets of exciting causes, the latter only is of his production: being
produced partly by his own special appointment, partly in conformity to
his general appointment, by the special appointment of the judge. For
the legislator, therefore, as well as for the judge, it is necessary (if
they would know what it is they are doing when they are appointing
punishment) to have an eye to all these circumstances. For the
legislator, lest, meaning to apply a certain quantity of punishment to
all persons who shall put themselves in a given predicament, he should
unawares apply to some of those persons much more or much less than he
himself intended; for the judge, lest, in applying to a particular
person a particular measure of punishment, he should apply much more or
much less than was intended, perhaps by himself, and at any rate by the
legislator. They ought each of them, therefore, to have before him, on
the one hand, a list of the several circumstances by which sensibility
may be influenced; on the other hand, a list of the several species and
degrees of punishment which they purpose to make use of: and then, by
making a comparison between the two, to form a detailed estimate of the
influence of each of the circumstances in question, upon the effect of
each species and degree of punishment.\\
There are two plans or orders of distribution, either of which might be
pursued in the drawing up this estimate. The one is to make the name of
the circumstance take the lead, and under it to represent the different
influences it exerts over the effects of the several modes of
punishment: the other is to make the name of the punishment take the
lead, and under it to represent the different influences which are
exerted over the effects of it by the several circumstances above
mentioned. Now of these two sorts of objects, the punishment is that to
which the intention of the legislator is directed in the first instance.
This is of his own creation, and will be whatsoever he thinks fit to
make it: the influencing circumstance exists independently of him, and
is what it is whether he will or no. What he has occasion to do is to
establish a certain species and degree of punishment: and it is only
with reference to that punishment that he has occasion to make any
inquiry concerning any of the circumstances here in question. The latter
of the two plans therefore is that which appears by far the most useful
and commodious. But neither upon the one nor the other plan can any such
estimate be delivered here.

XLVI. Of the several circumstances contained in this catalogue, it may
be of use to give some sort of analytic view; in order that it may be
the more easily discovered if any which ought to have been inserted are
omitted; and that, with regard to those which are inserted, it may be
seen how they differ and agree.

In the first place, they may be distinguished into \emph{primary} and
\emph{secondary:} those may be termed primary, which operate immediately
of themselves: those secondary, which operate not but by the medium of
the former. To this latter head belong the circumstances of sex, age,
station in life, education, climate, lineage, government, and religious
profession: the rest are primary. These again are either \emph{connate}
or \emph{adventitious:} those which are connate, are radical frame of
body and radical frame of mind. Those which are adventitious, are either
\emph{personal,} or \emph{exterior.} The personal, again, concern either
a man's \emph{dispositions,} or his \emph{actions.} Those which concern
his dispositions, concern either his \emph{body} or his \emph{mind.}
Those which concern his body are health, strength, hardiness, and bodily
imperfection. Those which concern his mind, again, concern either his
\emph{understanding} or his \emph{affections.} To the former head belong
the circumstances of quantity and quality of knowledge, strength of
understanding, and insanity. To the latter belong the circumstances of
firmness of mind, steadiness, bent of inclination, moral sensibility,
moral biases, religious sensibility, religious biases, sympathetic
sensibility, sympathetic biases, antipathetic sensibility, and
antipathetic biases. Those which regard his actions, are his habitual
occupations. Those which are exterior to him, regard either the
\emph{things} or the \emph{persons} which he is concerned with; under
the former head come his pecuniary circumstances; under the latter, his
connexions in the way of sympathy and antipathy.

\chapter{Of Human Actions in General}

I. The business of government is to promote the happiness of the
society, by punishing and rewarding. That part of its business which
consists in punishing, is more particularly the subject of penal law. In
proportion as an act tends to disturb that happiness, in proportion as
the tendency of it is pernicious, will be the demand it creates for
punishment. What happiness consists of we have already seen: enjoyment
of pleasures, security from pains.

II. The general tendency of an act is more or less pernicious, according
to the sum total of its consequences: that is, according to the
difference between the sum of such as are good, and the sum of such as
are evil.

III. It is to be observed, that here, as well as henceforward, wherever
consequences are spoken of, such only are meant as are \emph{material.}
Of the consequences of any act, the multitude and variety must needs be
infinite: but such of them only as are material are worth regarding. Now
among the consequences of an act, be they what they may, such only, by
one who views them in the capacity of a legislator, can be said to be
material (or \emph{of importance)} as either consist of pain or
pleasure, or have an influence in the production of pain or pleasure.

IV. It is also to be observed, that into the account of the consequences
of the act, are to be taken not such only as might have ensued, were
intention out of the question, but such also as depend upon the
connexion there may be between these first-mentioned consequences and
the intention. The connexion there is between the intention and certain
consequences is, as we shall see hereafter, a means of producing other
consequences. In this lies the difference between rational agency and
irrational.

V. Now the intention, with regard to the consequences of an act, will
depend upon two things:\\
1. The state of the will or intention, with respect to the act itself.
And,\\
2. The state of the understanding, or perceptive faculties, with regard
to the circumstances which it is, or may appear to be, accompanied
with.\\
Now with respect to these circumstances, the perceptive faculty is
susceptible of three states: consciousness, unconsciousness, and false
consciousness. Consciousness, when the party believes precisely those
circumstances, and no others, to subsist, which really do subsist:
unconsciousness, when he fails of perceiving certain circumstances to
subsist, which, however, do subsist: false consciousness, when he
believes or imagines certain circumstances to subsist, which in truth do
not subsist.

VI. In every transaction, therefore, which is examined with a view to
punishment, there are four articles to be considered:\\
1. The \emph{act} itself, which is done.\\
2. The \emph{circumstances} in which it is done.\\
3. The \emph{intentionality} that may have accompanied it.\\
4.The \emph{consciousness,} unconsciousness, or false consciousness,
that may have accompanied it. What regards the act and the circumstances
will be the subject of the present chapter: what regards intention and
consciousness, that of the two succeeding.

VII. There are also two other articles on which the general tendency of
an act depends: and on that, as well as on other accounts, the demand
which it creates for punishment. These are,\\
1. The particular \emph{motive} or motives which gave birth to it.\\
2. The general \emph{disposition} which it indicates. These articles
will be the subject of two other chapters.

VIII. Acts may be distinguished in several ways, for several purposes.
They may be distinguished, in the first place, into \emph{positive and
negative.} By positive are meant such as consist in motion or exertion:
by negative, such as consist in keeping at rest; that is, in forbearing
to move or exert one's self in such and such circumstances. thus, to
strike is a positive act: not to strike on a certain occasion, a
negative one. Positive acts are styled also acts of commission;
negative, acts of omission or forbearance.

IX. Such acts, again, as are negative, may either be \emph{absolutely}
so, or \emph{relatively:} absolutely, when they import the negation of
all positive agency whatsoever; for instance, not to strike at all:
relatively, when they import the negation of such or such a particular
mode of agency; for instance, not to strike such a person or such a
thing, or in such a direction.

X. It is to be observed, that the nature of the act, whether positive or
negative, is not to be determined immediately by the form of the
discourse made use of to express it. An act which is positive in its
nature may be characterized by a negative expression: thus, not to be at
rest, is as much as to say to move. So also an act, which is negative in
its nature, may be characterized by a positive expression: thus, to
forbear or omit to bring food to a person in certain circumstances, is
signified by the single and positive term \emph{to starve.}

XI. In the second place, acts may be distinguished into \emph{external}
and \emph{internal.} By external, are meant corporal acts; acts of the
body: by internal, mental acts; acts of the mind. Thus, to strike is an
external or exterior act: to intend to strike, an internal or interior
one.

XII. Acts of \emph{discourse} are a sort of mixture of the two: external
acts, which are no ways material, nor attended with any consequences,
any farther than as they serve to express the existence of internal
ones. To speak to another to strike, to write to him to strike, to make
signs to him to strike, are all so many acts of discourse.

XIII. Third, acts that are external may be distinguished into
\emph{transitive} and \emph{intransitive.} Acts may be called
transitive, when the motion is communicated from the person of the agent
to some foreign body: that is, to such a foreign body on which the
effects of it are considered as being \emph{material;} as where a man
runs against you, or throws water in your face. Acts may be called
intransitive, when the motion is communicated to no other body, on which
the effects of it are regarded as material, than some part of the same
person in whom it originated, as where a man runs, or washes himself.

XIV. An act of the transitive kind may be said to be in its
\emph{commencement,} or in the \emph{first} stage of its progress, while
the motion is confined to the person of the agent, and has not yet been
communicated to any foreign body, on which the effects of it can be
material. It may be said to be in its \emph{termination,} or to be in
the last stage of its progress, as soon as the motion or impulse has
been communicated to some such foreign body. It may be said to be in the
\emph{middle} or intermediate stage or stages of its progress, while the
motion, having passed from the person of the agent, has not yet been
communicated to any such foreign body. Thus, as soon as a man has lifted
up his hand to strike, the act he performs in striking you is in its
commencement: as soon as his hand has reached you, it is in its
termination. If the act be the motion of a body which is separated from
the person of the agent before it reaches the object, it may be said,
during that interval, to be in its intermediate progress, or in
\emph{gradu mediativo:} as in the case where a man throws a stone or
fires a bullet at you.

XV. An act of the intransitive kind may be said to be in its
commencement, when the motion or impulse is as yet confined to the
member or organ in which it originated; and has not yet been
communicated to any member or organ that is distinguishable from the
former. It may be said to be in its termination, as soon as it has been
applied to any other part of the same person. Thus, where a man poisons
himself, while he is lifting up the poison to his mouth, the act is in
its commencement: as soon as it has reached his lips, it is in its
termination.

XVI. In the third place, acts may be distinguished into \emph{transient}
and \emph{continued.} Thus, to strike is a transient act: to lean, a
continued one. To buy, a transient act: to keep in one's possession, a
continued one.

XVII. In strictness of speech there is a difference between a
\emph{continued} act and a \emph{repetition} of acts. It is a repetition
of acts, when there are intervals filled up by acts of different
natures: a continued act, when there are no such intervals. Thus, to
lean, is continued act: to keep striking, a repetition of acts.

XVIII. There is a difference, again, between a \emph{repetition} of
acts, and a \emph{habit} or \emph{practice.} The term repetition of acts
may be employed, let the acts in question be separated by ever such
short intervals, and let the sum total of them occupy ever so short a
space of time. The term habit is not employed but when the acts in
question are supposed to be separated by long-continued intervals, and
the sum total of them to occupy a considerable space of time. It is not
(for instance) the drinking ever so many times, nor ever so much at a
time, in the course of the same sitting, that will constitute a habit of
drunkenness: it is necessary that such sittings themselves be frequently
repeated. Every habit is a repetition of acts; or, to speak more
strictly, when a man has frequently repeated such and such acts after
considerable intervals, he is said to have persevered in or contracted a
habit: but every repetition of acts is not a habit.

XIX. Fourth, acts may be distinguished into \emph{indivisible} and
\emph{divisible.} Indivisible acts are merely imaginary: they may be
easily conceived, but can never be known to be exemplified. Such as are
divisible may be so, with regard either to matter or to to motion. An
act indivisible with regard to matter, is the motion or rest of one
single atom of matter. An act indivisible, with regard to motion, is the
motion of any body, from one single atom of space to the next to it.\\
Fifth, acts may be distinguished into \emph{simple} and \emph{complex:}
simple, such as the act of striking, the act of leaning, or the act of
drinking, above instanced: complex, consisting each of a multitude of
simple acts, which, though numerous and heterogeneous, derive a sort of
unity from the relation they bear to some common design or end; such as
the act of giving a dinner, the act of maintaining a child, the act of
exhibiting a triumph, the act of bearing arms, the act of holding a
court, and so forth.

XX. It has been every now and then made a question, what it is in such a
case that constitutes one act: where one act has ended, and another act
has begun: whether what has happened has been one act or many. These
questions, it is now evident, may frequently be answered, with equal
propriety, in opposite ways: and if there be any occasions on which they
can be answered only in one way, the answer will depend upon the nature
of the occasion, and the purpose for which the question is proposed. A
man is wounded in two fingers at one stroke,'' Is it one wound
or several? A man is beaten at 12 o'clock, and again at 8 minutes after
12.'' Is it one beating or several? You beat one man, and
instantly in the same breath you beat another.'' Is this one
beating or several? In any of these cases it may be one, perhaps, as to
some purposes, and several as to others. These examples are given, that
men may be aware of the ambiguity of language: and neither harass
themselves with unsolvable doubts, nor one another with interminable
disputes.

XXI. So much with regard to acts considered in themselves: we come now
to speak of the \emph{circumstances} with which they may have been
accompanied. These must necessarily be taken into the account before any
thing can be determined relative to the consequences. What the
consequences of an act may be upon the whole can never otherwise be
ascertained: it can never be known whether it is beneficial, or
indifferent, or mischievous. In some circumstances even to kill a man
may be a beneficial act: in others, to set food before him may be a
pernicious one.

XXII. Now the circumstances of an act, are, what? Any objects (or
entities) whatsoever. Take any act whatsoever, there is nothing in the
nature of things that excludes any imaginable object from being a
circumstance to it. Any given object may be a circumstance to any other.

XXIII. We have already had occasion to make mention for a moment of the
\emph{consequences} of an act: these were distinguished into material
and immaterial. In like manner may the circumstances of it be
distinguished. Now \emph{materiality} is a relative term: applied to the
consequences of an act, it bore relation to pain and pleasure: applied
to the circumstances, it bears relation to the consequences. A
circumstance may be said to be material, when it bears a visible
relation in point of causality to the consequences: immaterial, when it
bears no such visible relation. '

XXIV. The consequences of an act are events. A circumstance may be
related to an event in point of causality in any be one of four ways: 1.
In the way of causation or production. 2. In the way of derivation. 3.
In the way of collateral condition. 4. In the way of conjunct influence.
It may be said to be related to the event in the way of causation. when
it is of the number of those that contribute to the production of such
event: in the way of derivation, when it is of the number of the events
to the production of which that in question has been contributory: in
the way of collateral connexion, where the circumstance in question, and
the event in question, without being either of them instrumental in the
production of the other, are related, each of them, to some common
object, which has been concerned in the production of them both: in the
way of conjunct influence, when, whether related in any other way or
not, they have both of them concurred in the production of some common
consequence.

XXV. An example may be of use. In the year 1628, Villiers, Duke of
Buckingham, favourite and minister of Charles I. of England, received a
wound and died. The man who gave it him was one Felton, who, exasperated
at the maladministration of which that minister was accused, went down
from London to Portsmouth, where Buckingham happened then to be, made
his way into his antechamber, and finding him busily engaged in
conversation with a number of people round him, got close to him, drew a
knife and stabbed him. In the effort, the assassin's hat fell off, which
was found soon after, and, upon searching him, the bloody knife. In the
crown of the hat were found scraps of paper, with sentences expressive
of the purpose he was come upon. Here then, suppose the event in
question is the wound received by Buckingham: Felton's drawing out his
knife, his making his way into the chamber, his going down to
Portsmouth, his conceiving an indignation at the idea of Buckingham's
administration, that administration itself, Charles's appointing such a
minister, and so on, higher and higher without end, are so many
circumstances, related to the event of Buckingham's receiving the wound,
in the way of causation or production: the bloodiness of the knife, a
circumstance related to the same event in the way of derivation: the
finding of the hat upon the ground, the finding the sentences in the
hat, and the writing them, so many circumstances related to it in the
way of collateral connexion: and the situation and conversations of the
people about Buckingham, were circumstances related to the circumstances
of Felton's making his way into the room, going down to Portsmouth, and
so forth, in the way of conjunct influence; inasmuch as they contributed
in common to the event of Buckingham's receiving the wound, by
preventing him from putting himself upon his guard upon the first
appearance of the intruder.

XXVI. These several relations do not all of them attach upon an event
with equal certainty. In the first place, it is plain, indeed, that
every event must have some circumstance or other, and in truth, an
indefinite multitude of circumstances, related to it in the way of
production: it must of course have a still greater multitude of
circumstances related to it in the way of collateral connexion. But it
does not appear necessary that every event should have circumstances
related to it in the way of derivation: nor therefore that it should
have any related to it in the way of conjunct influence. But of the
circumstances of all kinds which actually do attach upon an event, it is
only a very small number that can be discovered by the utmost exertion
of the human faculties: it is a still smaller number that ever actually
do attract our notice: when occasion happens, more or fewer of them will
be discovered by a man in proportion to the strength, partly of his
intellectual powers, partly of his inclination. It appears therefore
that the multitude and description of such of the circumstances
belonging to an act, as may appear to be material, will be determined by
two considerations: 1. By the nature of things themselves. 2. By the
strength or weakness of the faculties of those who happen to consider
them.

XXVII. Thus much it seemed necessary to premise in general concerning
acts, and their circumstances, previously to the consideration of the
particular sorts of acts with their particular circumstances, with which
we shall have to do in the body of the work. An act of some sort or
other is necessarily included in the notion of every offense. Together
with this act, under the notion of the same offense, are included
certain circumstances: which circumstances enter into the essence of the
offense, contribute by their conjunct influence to the production of its
consequences, and in conjunction with the act are brought into view by
the name by which it stands distinguished. These we shall have occasion
to distinguish hereafter by the name of \emph{criminative}
circumstances. Other circumstances again entering into combination with
the act and the former set of circumstances, are productive of still
farther consequences. These additional consequences, if they are of the
beneficial kind, bestow, according to the value they bear in that
capacity, upon the circumstances to which they owe their birth the
appellation of \emph{exculpative} or \emph{extenuative} circumstances:
if of the mischievous kind, they bestow on them the appellation of
\emph{aggravative} circumstances. Of all these different sets of
circumstances, the criminative are connected with the consequences of
the original offence, in the way of production; with the act, and with
one another, in the way of conjunct influence: the consequences of the
original offense with them, and with the act respectively, in the way of
derivation: the consequences of the modified offense, with the
criminative, exculpative, and extenuative circumstances respectively, in
the way also of derivation: these different sets of circumstances, with
the consequences of the modified act or offense, in the way of
production: and with one another (in respect of the consequences of the
modified act or offense) in the way of conjunct influence. Lastly,
whatever circumstances can be seen to be connected with the consequences
of the offense, whether directly in the way of derivation, or obliquely
in the way of collateral affinity (to wit, in virtue of its being
connected, in the way of derivation, with some of the circumstances with
which they stand connected in the same manner) bear a \emph{material}
relation to the offense in the way of evidence, they may accordingly be
styled \emph{evidentiary} circumstances, and may become of use, by being
held forth upon occasion as so many proofs, indications, or evidences of
its having been committed.

\chapter{Of Intentionality}

I. So much with regard to the two first of the articles upon which the
evil tendency of an action may depend: viz., the act itself, and the
general assemblage of the circumstances with which it may have been
accompanied. We come now to consider the ways in which the particular
circumstance of intention may be concerned in it.

II. First, then, the intention or will may regard either of two
objects:\\
1. The act itself: or, 2. Its consequences. Of these objects, that which
the intention regards may be styled \emph{intentional.} If it regards
the act, then the act may be said to be intentional: if the
consequences, so also then may the consequences. If it regards both the
act and consequences, the whole \emph{action} may be said to be
intentional. Whichever of those articles is not the object of the
intention, may of course be said to be \emph{unintentional.}

III. The act may very easily be intentional without the consequences;
and often is so. Thus, you may intend to touch a man without intending
to hurt him: and yet, as the consequences turn out, you may chance to
hurt him.

IV. The consequences of an act may also be intentional, without the
act's being intentional throughout; that is, without its being
intentional in every stage of it: but this is not so frequent a case as
the former. You intend to hurt a man, suppose, by running against him,
and pushing him down: and you run towards him accordingly: but a second
man coming in on a sudden between you and the first man, before you can
stop yourself, you run against the second man, and by him push down the
first.

V. But the consequences of an act cannot be intentional, without the
act's being itself intentional in at least the first, stage. If the act
be not intentional in the first stage, it is no act of yours: there is
accordingly no intention on your part to produce the consequences: that
is to say, the individual consequences. All there can have been on your
part is a distant intention to produce other consequences, of the same
nature, by some act of yours, at a future time: or else, without any
intention, a bare \emph{wish} to see such event take place. The second
man, suppose, runs of his own accord against the first, and pushes him
down. You had intentions of doing a thing of the same nature:
\emph{viz.,} To run against him, and push him down yourself; but you had
done nothing in pursuance of those intentions: the individual
consequences therefore of the act, which the second man performed in
pushing down the first, cannot be said to have been on your part
intentional.

VI. Second. A consequence, when it is intentional, may either be
\emph{directly} so, or only \emph{obliquely.} It may be said to be
directly or lineally intentional, when the prospect of producing it
constituted one of the links in the chain of causes by which the person
was determined to do the act. It may be said to be obliquely or
collaterally intentional, when, although the consequence was in
contemplation, and appeared likely to ensue in case of the act's being
performed, yet the prospect of producing such consequence did not
constitute a link in the aforesaid chain.

VII. Third. An incident, which is directly intentional, may or either be
\emph{ultimately} so, or only \emph{mediately.} It may be said to be
ultimately intentional, when it stands last of all exterior events in
the aforesaid chain of motives; insomuch that the prospect of the
production of such incident, could there be a certainty of its taking
place, would be sufficient to determine the will, without the prospect
of its producing any other. It may be said to be mediately intentional,
and no more, when there is some other incident, the prospect of
producing which forms a subsequent link in the same chain: insomuch that
the prospect of producing the former would not have operated as a
motive, but for the tendency which it seemed to have towards the
production of the latter.

VIII. Fourth. When an incident is directly intentional, it may either be
\emph{exclusively} so, or \emph{inexclusively.} It may be said to be
exclusively intentional, when no other but that very individual incident
would have answered the purpose, insomuch that no other incident had any
share in determining the will to the act in question. It may be said to
have been inexclusively (or concurrently) intentional, when there was
some other incident, the prospect of which was acting upon the will at
the same time.

IX. Fifth. When an incident is inexclusively intentional, it may be
either \emph{conjunctively} so, \emph{disjunctively,} or
\emph{indiscriminately.} It may be said to be conjunctively intentional
with regard to such other incident, when the intention is to produce
both: disjunctively, when the intention is to produce either the one or
the other indifferently, but not both: indiscriminately, when the
intention is indifferently to produce either the one or the other, or
both, as it may happen.

X. Sixth. When two incidents are disjunctively intentional, they may be
so with or without \emph{preference.} They may be said to be so with
preference, when the intention is, that one of them in particular should
happen rather than the other: without preference, when the intention is
equally fulfilled, whichever of them happens.

XI. One example will make all this clear. William II. king of England,
being out a stag-hunting, received from Sir Walter Tyrrel a wound, of
which he died . 6 Let us take this case, and diversify it with a variety
of suppositions, correspondent to the distinctions just laid down.\\
I. First then, Tyrrel did not so much as entertain a thought of the
king's death; or, if he did, looked upon it as an event of which there
was no danger. In either of these cases the incident of his killing the
king was altogether unintentional.\\
2. He saw a stag running that way, and he saw the king riding that way
at the same time: what he aimed at was to kill the stag: he did not wish
to kill the king: at the same time he saw, that if he shot, it was as
likely he should kill the king as the stag: yet for all that he shot,
and killed the king accordingly. In this case the incident of his
killing the king was intentional, but obliquely so.\\
3. He killed the king on account of the hatred he bore him, and for no
other reason than the pleasure of destroying him. In this case the
incident of the king's death was not only directly but ultimately
intentional.\\
4. He killed the king, intending fully so to do; not for any hatred he
bore him, but for the sake of plundering him when dead. In this case the
incident of the king's death was directly intentional, but not
ultimately: it was mediately intentional.\\
5. He intended neither more nor less than to kill the king. He had no
other aim nor wish. In this case it was exclusively as well as directly
intentional: exclusively, to wit, with regard to every other material
incident.\\
6. Sir Walter shot the king in the right leg, as he was plucking a thorn
out of it with his left hand. His intention was, by shooting the arrow
into his leg through his hand, to cripple him in both those limbs at the
same time. In this case the incident of the king's being shot in the leg
was intentional: and that conjunctively with another which did not
happen; \emph{viz.,} his being shot in the hand.\\
7. The intention of Tyrrel was to shoot the king either in the hand or
in the leg, but not in both; and rather in the hand than in the leg. In
this case the intention of shooting in the hand was disjunctively
concurrent, with regard to the other incident, and that with
preference.\\
8. His intention was to shoot the king either in the leg or the hand,
whichever might happen: but not in both. In this case the intention was
inexclusive, but disjunctively so: yet that, however, without
preference.\\
9. His intention was to shoot the king either in the leg or the hand, or
in both, as it might happen. In this case the intention was
indiscriminately concurrent, with respect to the two incidents.

XII. It is to be observed, that an act may be unintentional in any stage
or stages of it, though intentional in the preceding: and, on the other
hand, it may be intentional in any stage or stages of it, and yet
unintentional in the succeeding. But whether it be intentional or no in
any preceding stage, is immaterial, with respect to the consequences, so
it be unintentional in the last. The only point, with respect to which
it is material, is the proof. The more stages the act is unintentional
in, the more apparent it will commonly be, that it was unintentional
with respect to the last. If a man, intending to strike you on the
cheek, strikes you in the eye, and puts it out, it will probably be
difficult for him to prove that it was not his intention to strike you
in the eye. It will probably be easier, if his intention was really not
to strike you, or even not to strike at all.

XIII. It is frequent to hear men speak of a good intention, of a bad
intention; of the goodness and badness of a man's intention: a
circumstance on which great stress is generally laid. It is indeed of no
small importance, when properly understood: but the import of it is to
the last degree ambiguous and obscure. Strictly speaking, nothing can be
said to be good or bad, but either in itself; which is the case only
with pain or pleasure: or on account of its effects; which the case only
with things that are the causes or preventives of pain and pleasure. But
in a figurative and less proper way of speech, a thing may also be
styled good or bad, in consideration of its cause. Now the effects of an
intention to do such or such an act, are the same objects which we have
been speaking of under the appellation of its \emph{consequences:} and
the causes of intention are called \emph{motives.} A man's intention
then on any occasion may be styled good or bad, with reference either to
the consequences of the act, or with reference to his motives. If it be
deemed good or bad in any sense, it must be either because it is deemed
to be productive of good or of bad consequences, or because it is deemed
to originate from a good or from a bad motive. But the goodness or
badness of the consequences depend upon the circumstances. Now the
circumstances are no objects of the intention. A man intends the act:
and by his intention produces the act: but as to the circumstances, he
does not intend \emph{them:} he does not, inasmuch as they are
circumstances of it, produce them. If by accident there be a few which
he has been instrumental in producing, it has been by former intentions,
directed to former acts, productive of those circumstances as the
consequences: at the time in question he takes them as he finds them.
Acts, with their consequences, are objects of the will as well as of the
understanding: circumstances, as such, are objects of the understanding
only. All he can do with these, as such, is to know or not to know them:
in other words, to be conscious of them, or not conscious. To the title
of Consciousness belongs what is to be said of the goodness or badness
of a man's intention, as resulting from the consequences of the act: and
to the head of Motives, what is to be said of his intention, as
resulting from the motive.

\chapter{Of Consciousness}

I. So far with regard to the ways in which the will or intention may be
concerned in the production of any incident: we come now to consider the
part which the understanding or perceptive faculty may have borne, with
relation to such incident.

II. A certain act has been done, and that intentionally: that act was
attended with certain circumstances: upon these circumstances depended
certain of its consequences; and amongst the rest, all those which were
of a nature purely physical. Now then, take any one of these
circumstances, it is plain, that a man, at the time of doing the act
from whence such consequences ensued, may have been either conscious,
with respect to this circumstance, or unconscious. In other words, he
may either have been aware of the circumstance, or not aware: it may
either have been present to his mind, or not present. In the first case,
the act may be said to have been an \emph{advised} act, with respect to
that circumstance: in the other case, an \emph{unadvised} one.

III. There are two points, with regard to which an act may have been
advised or unadvised: 1. The \emph{existence} of the circumstance
itself. 2. The \emph{materiality} of it.

IV. It is manifest, that with reference to the time of the act, such
circumstance may have been either \emph{present, past,} or
\emph{future.}

V. An act which is unadvised, is either \emph{heedless,} or not
heedless. It is termed heedless, when the case is thought to be such,
that a person of ordinary prudence, if prompted by an ordinary share of
benevolence, would have been likely to have bestowed such and so much
attention and reflection upon the material circumstances, as would have
effectually disposed him to prevent the mischievous incident from taking
place: not heedless, when the case is not thought to be such as above
mentioned.

VI. Again. Whether a man did or did not suppose the existence or
materiality of a given circumstance, it may be that he \emph{did}
suppose the existence and materiality of some circumstance, which either
did not exist, or which, though existing, was not material. In such case
the act may be said to be \emph{mis-advised,} with respect to such
imagined circumstance: and it maybe said, that there has been an
erroneous supposition, or a \emph{mis-supposal} in the case.

VII. Now a circumstance, the existence of which is thus erroneously
supposed, may be material either, 1. In the way of prevention: or, 2. In
that of compensation. It may be said to be material in the way of
prevention, when its effect or tendency, had it existed, would have been
to prevent the obnoxious consequences: in the way of compensation, when
that effect or tendency would have been to produce other consequences,
the beneficialness of which would have out-weighed the mischievousness
of the others.

VIII. It is manifest that, with reference to the time of the act, such
imaginary circumstance may in either case have been supposed either to
be \emph{present, past,} or \emph{future.} \emph{}

IX. To return to the example exhibited in the preceding chapter.\\
10. Tyrrel intended to shoot in the direction in which he shot; but he
did not know that the king was riding so near that way. In this case the
act he performed in shooting, the act of shooting, was \emph{unadvised,}
with respect to the existence of the circumstance of the king's being so
near riding that way.\\
11. He knew that the king was riding that way: but at the distance at
which the king was, he knew not of the probability there was that the
arrow would reach him. In this case the act was unadvised, with respect
to the \emph{materiality} of the circumstance.\\
12. Somebody had dipped the arrow in poison, without Tyrrel's knowing of
it. In this case the act was unadvised, with respect to the existence of
a \emph{past} circumstance.\\
13. At the very instant that Tyrrel drew the bow, the king being
screened from his view by the foliage of some bushes, was riding
furiously, in such manner as to meet the arrow in a direct line: which
circumstance was also more than Tyrrel knew of. In this case the act was
unadvised, with respect to the existence of a \emph{present}
circumstance.\\
14. The king being at a distance from court, could get nobody to dress
his wound till the next day; of which circumstance Tyrrel was not aware.
In this case the act was unadvised, with respect to what was then
\emph{future} circumstance.\\
15. Tyrrel knew of the king's being riding that way, of his being so
near, and so forth; but being deceived by the foliage of the bushes, he
thought he saw a bank between the spot from which he shot, and that to
which the king was riding. In this case the act was \emph{mis-advised,}
proceeding on the \emph{mis-supposal} of a \emph{preventive}
circumstance.\\
16. Tyrrel knew that every thing was as above, nor was he deceived by
the supposition of any preventive circumstance. But he believed the king
to be an usurper: and supposed he was coming up to attack a person whom
Tyrrel believed to be the rightful king, and who was riding by Tyrrel's
side. In this case the act was also mis-advised, but proceeded on the
mis-supposal of a \emph{compensative} circumstance.

X. Let us observe the connexion there is between intentionality and
consciousness. When the act itself is intentional, and with respect to
the existence of all the circumstances \emph{advised,} as also with
respect to the materiality of those circumstances, in relation to a
given consequence, and there is no mis-supposal with regard to any
preventive circumstance, that consequence must also be intentional: in
other words; advisedness, with respect to the circumstances, if clear
from the mis-supposal of any preventive circumstance, extends the
intentionality from the act to the consequences. Those consequences may
be either directly intentional, or only obliquely so: but at any rate
they cannot be but intentional.

XI. To go on with the example. If Tyrrel intended to shoot in the
direction in which the king was riding up, and knew that the king was
coming to meet the arrow, and knew the probability there was of his
being shot in that same part in which he was shot, or in another as
dangerous, and with that same degree of force, and so forth, and was not
misled by the erroneous supposition of a circumstance by which the shot
would have been prevented from taking place, or any such other
preventive circumstance, it is plain he could not but have intended the
king's death. Perhaps he did not positively wish it; but for all that,
in a certain sense he intended it.

XII. What heedlessness is in the case of an unadvised act, rashness is
in the case of a misadvised one. A misadvised act then may be either
rash or not rash. It may be termed rash, when the case is thought to be
such, that a person of ordinary prudence, if prompted by an ordinary
share of benevolence, would have employed such and so much attention and
reflection to the imagined circumstance, as, by discovering to him the
nonexistence, improbability, or immateriality of it, would have
effectually disposed him to prevent the mischievous incident from taking
place.

XIII. In ordinary discourse, when a man does an act of which the
consequences prove mischievous, it is a common thing to speak of him as
having acted with a good intention or, with a bad intention, of his
intention's being a good one or a bad one. The epithets good and bad are
all this while applied, we see, to the intention: but the application of
them is most commonly governed by a supposition formed with regard to
the nature of the motive. The act, though eventually it prove
mischievous, is said to be done with a good intention, when it is
supposed to issue from a motive which is looked upon as a good motive:
with a bad intention, when it is supposed to be the result of a motive
which is looked upon as a bad motive. But the nature of the consequences
intended, and the nature of the motive which gave birth to the
intention, are objects which, though intimately connected, are perfectly
distinguishable. The intention might therefore with perfect propriety be
styled a good one, whatever were the motive. It might be styled a good
one, when not only the consequences of the act \emph{prove} mischievous,
but the motive which gave birth to it \emph{was} what is called a bad
one. To warrant the speaking of the intention as being a good one, it is
sufficient if the consequences of the act, had they proved what to the
agent they seemed likely to be, \emph{would} have been of a beneficial
nature. And in the same manner the intention may be bad, when not only
the consequences of the act prove beneficial, but the motive which gave
birth to it was a good one.

XIV. Now, when a man has a mind to speak of your intention as being good
or bad, with reference to the consequences, if he speaks of it at all he
must use the word intention, for there is no other. But if a man means
to speak of the \emph{motive} from which your intention originated, as
being a good or a bad one, he is certainly not obliged to use the word
intention: it is at least as well to use the word motive. By the
supposition he means the motive; and very likely he may \emph{not} mean
the intention. For what is true of the one is very often not true of the
other. The motive may be good when the intention is bad: the intention
may be good when the motive is bad: whether they are both good or both
bad, or the one good and the other bad, makes, as we shall see
hereafter, a very essential difference with regard to the consequences.
It is therefore much better, when motive is meant, never to say
intention.

XV. An example will make this clear. Out of malice a man prosecutes you
for a crime of which he believes you to be guilty, but of which in fact
you are not guilty. Here the \emph{consequences} of his conduct are
mischievous: for they are mischievous to you at any rate, in virtue of
the shame and anxiety which you are made to suffer while the prosecution
is depending: to which is to be added, in case of your being convicted,
the evil of the punishment. To you therefore they are mischievous; nor
is there any one to whom they are beneficial. The man's \emph{motive}
was also what is called a bad one: for malice will be allowed by every
body to be a bad motive. However, the \emph{consequences} of his
conduct, had they proved such as he believed them likely to be, would
have been good: for in them would have been included the punishment of a
criminal, which is a benefit to all who are exposed to suffer by a crime
of the like nature. The \emph{Intention} therefore, in this case, though
not in a common way of speaking the motive, might be styled a
\emph{good} one. But of motives more particularly in the next chapter.

XVI. In the same sense the intention, whether it be positively good or
no, so long as it is not bad, may be termed innocent. Accordingly, let
the consequences have proved mischievous, and let the motive have been
what it will, the intention may be termed innocent in either of two
cases:\\
1. In the case of \emph{un-}advisedness with respect to any of the
circumstances on which the mischievousness of the consequences
depended:\\
2. In the case of \emph{mis-}advisedness with respect to any
circumstance, which, had it been what it appeared to be, would have
served either to prevent or to outweigh the mischief.

XVII. A few words for the purpose of applying what has been said to the
Roman law. Unintentionality, and innocence of intention, seem both to be
included in the case of \emph{infortunium,} where there is neither
\emph{dolus} nor \emph{culpa.} Unadvisedness coupled with heedlessness,
and mis-advisedness coupled with rashness, correspond to the \emph{culpa
sine dolo.} Direct intentionality corresponds to \emph{dolus.} Oblique
intentionality seems hardly to have been distinguished from direct; were
it to occur, it would probably be deemed also to correspond to
\emph{dolus.} The division into \emph{culpa, lata, levis,} and
\emph{levissima,} is such as nothing certain can correspond to. What is
it that it expresses? A distinction, not in the case itself, but only in
the sentiments which any person (a judge, for instance) may find himself
disposed to entertain with relation to it: supposing it already
distinguished into three subordinate cases by other means. The word
\emph{dolus} seems ill enough contrived: the word \emph{culpa} as
indifferently. \emph{Dolus,} upon any other occasion, would be
understood to imply deceit, concealments, clandestinity: but here it is
extended to open force. \emph{Culpa,} upon any other occasion, would be
understood to extend to blame of every kind. It would therefore include
dolus.

XVIII. The above-mentioned definitions and distinctions are far from
being mere matters of speculation. They are capable of the most
extensive and constant application, as well to moral discourse as to
legislative practice. Upon the degree and bias of a man's intention,
upon the absence or presence of consciousness or mis-supposal, depend a
great part of the good and bad, more especially of the bad consequences
of an act; and on this, as well as other grounds, a great part of the
demand for punishment. The presence of intention with regard to such or
such a consequence, and of consciousness with regard to such or such a
circumstance, of the act, will form so many eliminative circumstances,
or essential ingredients in the composition of this or that offence:
applied to other circumstances, consciousness will form a ground of
aggravation, annexable to the like offence. In almost all cases, the
absence of intention with regard to certain consequences and the absence
of consciousness, or the presence of mis-supposal, with regard to
certain circumstances, will constitute so many grounds of extenuation.

\chapter{Of Motives}

§1. Different senses of the word motive

I. It is an acknowledged truth, that every kind of act whatever, and
consequently every kind of offense, is apt to assume a different
character, and be attended with different effects, according to the
nature of the \emph{motive} which gives birth to it. This makes it
requisite to take a view of the several motives by which human conduct
is liable to be influenced.

II. By a motive, in the most extensive sense in which the word is ever
used with reference to a thinking being, is meant any thing that can
contribute to give birth to, or even to prevent, any kind of action. Now
the actions of a thinking being is the act either of the body, or only
of the mind: and an act of the mind is an act either of the intellectual
faculty, or of the will. Acts of the intellectual faculty will sometimes
rest in the understanding merely, without exerting any influence in the
production of any acts of the will. Motives, which are not of a nature
to influence any other acts than those, may be styled purely
\emph{speculative} motives, or motives resting in speculation. But as to
these acts, neither do they exercise any influence over external acts,
or over their consequences, nor consequently over any pain or any
pleasure that may be in the number of such consequences. Now it is only
on account of their tendency to produce either pain or pleasure, that
any acts can be material. With acts, therefore, that rest purely in the
understanding, we have not here any concern: nor therefore with any
object, if any such there be, which, in the character of a motive, can
have no influence on any other acts than those.

III. The motives with which alone we have any concern are such as are of
a nature to act upon the will. By a motive then, in this sense of the
word, ls to be understood any thing whatsoever, which, by influencing
the will of a sensitive being, is supposed to serve as a means of
determining him to act, or voluntarily to forbear to act, upon any
occasion. Motives of this sort, in contradistinction to the former, may
be styled \emph{practical} motives, or motives applying to practice.

IV. Owing to the poverty and unsettled state of language, the word
\emph{motive} is employed indiscriminately to denote two kinds of
objects, which, for the better understanding of the subject, it is
necessary should be distinguished. On some occasions it is employed to
denote any of those really existing incidents from whence the act in
question is supposed to take its rise. The sense it bears on these
occasions may be styled its literal or \emph{unfigurative} sense. On
other occasions it is employed to denote a certain fictitious entity, a
passion, an affection of the mind, an ideal being which upon the
happening of any such incident is considered as operating upon the mind,
and prompting it to take that course, towards which it is impelled by
the influence of such incident. Motives of this class are Avarice,
Indolence, Benevolence, and so forth; as we shall see more particularly
farther on. This latter may be styled the \emph{figurative} sense of the
term \emph{motive.}

V. As to the real incidents to which the name of motive is also given,
these too are of two very different kinds. They may be either,\\
1. The \emph{internal} perception of any individual lot of pleasure or
pain, the expectation of which is looked upon as calculated to determine
you to act in such or such a manner; as the pleasure of acquiring such a
sum of money, the pain of exerting yourself on such an occasion, and so
forth: or,\\
2. Any \emph{external} event, the happening whereof is regarded as
having a tendency to bring about the perception of such pleasure or such
pain; for instance, the coming up of a lottery ticket, by which the
possession of the money devolves to you; or the breaking out of a fire
in the house you are in, which makes it necessary for you to quit it.
The former kind of motives may be termed interior, or internal: the
latter exterior, or external.

VI. Two other senses of the term \emph{motive} need also to be
distinguished. Motive refers necessarily to action. It is a pleasure,
pain, or other event, that prompts to action. Motive then, in one sense
of the word, must be previous to such event. But, for a man to be
governed by any motive, he must in every case look beyond that event
which is called his action; he must look to the consequences of it: and
it is only in this way that the idea of pleasure, of pain, or of any
other event, can give birth to it. He must look, therefore, in every
case, to some event posterior to the act in contemplation: an event
which as yet exists not, but stands only in prospect. Now, as it is in
all cases difficult, and in most cases unnecessary, to distinguish
between objects so intimately connected, as the posterior possible
object which is thug looked forward to, and the present existing object
or event which takes place upon a man's looking forward to the other,
they are both of them spoken of under the same appellation,
\emph{motive.} To distinguish them, the one first mentioned may be
termed a motive in \emph{prospect,} the other a motive in \emph{esse:}
and under each of these denominations will come as well exterior as
internal motives. A fire breaks out in your neighbour's house: you are
under apprehension of its extending to your own: you are apprehensive,
that if you stay in it, you will be burnt: you accordingly run out of
it. This then is the act: the others are all motives to it. The event of
the fire's breaking out in your neighbour's house is an external motive,
and that in \emph{esse:} the idea or belief of the probability of the
fire's extending to your own house, that of your being burnt if you
continue, and the pain you feel at the thought of such a catastrophe,
are all so many internal events, but still in \emph{esse:} the event of
the fire's actually extending to your own house, and that of your being
actually burnt by it, external motives in prospect: the pain you would
feel at seeing your house a burning, and the pain you would feel while
you yourself were burning, internal motives in prospect: which events,
according as the matter turns out, may come to be in \emph{esse:} but
then of course they will cease to act as motives.

VII. Of all these motives, which stand nearest to the act, to the
production of which they all contribute, is that internal motive in
\emph{esse} which consists in the expectation of the internal motive in
prospect: the pain or uneasiness you feel at the thoughts of being
burnt. All other motives are more or less remote: the motives in
prospect, in proportion as the period at which they are expected to
happen is more distant from the period at which the act takes place, and
consequently later in point of time: the motives \emph{in esse}, in
proportion as they also are more distant from that period, and
consequently earlier in point of time.

VIII. It has already been observed, that with motives of which the
influence terminates altogether in the understanding, we have nothing
here to do. If then, amongst objects that are spoken of as motives with
reference to the understanding, there be any which concern us here, it
is only in as far as such objects may, through the medium of the
understanding, exercise an influence over the will. It is in this way,
and in this way only, that any objects, in virtue of any tendency they
may have to influence the sentiment of belief, may in a practical sense
act in the character of motives. Any objects, by tending to induce a
belief concerning the existence, actual, or probable, of a practical
motive; that is, concerning the probability of a motive in prospect, or
the existence of a motive \emph{in esse;} may exercise an influence on
the will, and rank with those other motives that have been placed under
the name of practical. The pointing out of motives such as these, is
what we frequently mean when we talk of giving \emph{reasons.} Your
neighbour's house is on fire as before. I observe to you, that at the
lower part of your neighbour's house is some wood-work, which joins on
to yours; that the flames have caught this wood-work, and so forth;
which I do in order to dispose you to believe as I believe, that if you
stay in your house much longer you will be burnt. In doing this, then, I
suggest motives to your understanding; which motives, by the tendency
they have to give birth to or strengthen a pain, which operates upon you
in the character of an internal motive in \emph{esse,} join their force,
and act as motives upon the will. 2. No motives either constantly good
or constantly bad.

2. IX. In all this chain of motives, the principal or original link
seems to be the last internal motive in prospect: it is to this that all
the other motives in prospect owe their materiality: and the immediately
acting motive its existence. This motive in prospect, we see, is always
some pleasure, or some pain; some pleasure, which the act in question is
expected to be a means of continuing or producing: some pain which it is
expected to be a means of discontinuing or preventing. A motive is
substantially nothing more than pleasure or pain, operating in a certain
manner.

X. Now, pleasure is in \emph{itself} a good: nay, even setting aside
immunity from pain, the only good: pain is in itself an evil; and,
indeed, without exception, the only evil; or else the words good and
evil have no meaning. And this is alike true of every sort of pain, and
of every sort of pleasure. It follows, therefore, immediately and
incontestibly, that \emph{there is no such thing as any sort of motive
that is in itself a bad one.}

XI. It is common, however, to speak of actions as proceeding from
\emph{good} or \emph{bad} motives: in which case the motives meant are
such as are internal. The expression is far from being an accurate one;
and as it is apt to occur in the consideration of most every kind of
offence, it will be requisite to settle the precise meaning of it, and
observe how far it quadrates with the truth of things.

XII. With respect to goodness and badness, as it is with very thing else
that is not itself either pain or pleasure, so is it with motives. If
they are good or bad, it is only on account of their effects: good, on
account of their tendency to produce pleasure, or avert pain: bad, on
account of their tendency to produce pain, or avert pleasure. Now the
case is, that from one and the same motive, and from every kind of
motive, may proceed actions that are good, others that are bad, and
others that are indifferent. This we shall proceed to show with respect
to all the different kinds of motives, as determined by the various
kinds of pleasures and pains.

XIII. Such an analysis, useful as it is, will be found to be a matter of
no small difficulty owing, in great measure, to a certain perversity of
structure which prevails more or less throughout all languages. To speak
of motives, as of anything else, one must call them by their names. But
the misfortune is, that it is rare to meet with a motive of which the
name expresses that and nothing more. Commonly along with the very name
of the motive, is tacitly involved a proposition imputing to it a
certain quality; a quality which, in many cases, will appear to include
that very goodness or badness, concerning which we are here inquiring
whether, properly speaking, it be or be not imputable to motives. To use
the common phrase, in most cases, the name of the motive is a word which
is employed either only in a \emph{good sense,} or else only in a
\emph{bad sense.} Now, when a word is spoken of as being used in a good
sense, all that is necessarily meant is this: that in conjunction with
the idea of the object it is put to signify, it conveys an idea of
\emph{approbation:} that is, of a pleasure or satisfaction, entertained
by the person who employs the term at the thoughts of such object. In
like manner, when a word is spoken of as being used in a bad sense, all
that is necessarily meant is this: that, in conjunction with the idea of
the object it is put to signify, it conveys an idea of
\emph{disapprobation:} that is, of a displeasure entertained by the
person who employs the term at the thoughts of such object. Now, the
circumstance on which such approbation is grounded will, as naturally as
any other, be the opinion of the \emph{goodness} of the object in
question, as above explained: such, at least, it must be, upon the
principle of utility: so, on the other hand, the circumstance on which
any such disapprobation is grounded, will, as naturally as any other, be
the opinion of the \emph{badness} of the object: such, at least, it must
be, in as far as the principle of utility is taken for the standard.

Now there are certain motives which, unless in a few particular cases,
have scarcely any other name to be expressed by but such a word as is
used only in a good sense. This is the case, for example, with the
motives of piety and honour. The consequence of this is, that if, in
speaking of such a motive, a man should have occasion to apply the
epithet bad to any actions which he mentions as apt to result from it,
he must appear to be guilty of a contradiction in terms. But the names
of motives which have scarcely any other name to be expressed by, but
such a word as is used only in a bad sense, are many more. 7 This is the
case, for example, with the motives of lust and avarice. And
accordingly, if in speaking of any such motive, a man should have
occasion to apply the epithets good or indifferent to any actions which
he mentions as apt to result from it, he must here also appear to be
guilty of a similar contradiction.

This perverse association of ideas cannot, it is evident, but throw
great difficulties in the way of the inquiry now before us. Confining
himself to the language most in use, a man can scarce avoid running, in
appearance, into perpetual contradictions. His propositions will appear,
on the one hand, repugnant to truth; and on the other hand, adverse to
utility. As paradoxes, they will excite contempt: as mischievous
paradoxes, indignation. For the truths he labours to convey, however
important, and however salutary, his reader is never the better: and he
himself is much the worse. To obviate this inconvenience, completely, he
has but this one unpleasant remedy; to lay aside the old phraseology and
invent a new one. Happy the man whose language is ductile enough to
permit him this resource. To palliate the inconvenience, where that
method of obviating it is impracticable, he has nothing left for it but
to enter into a long discussion, to state the whole matter at large, to
confess, that for the sake of promoting the purposes, he has violated
the established laws of language, and to throw himself upon the mercy of
his readers.\\

§3. Catalogue of motives corresponding to that of Pleasures and Pains.

XIV. From the pleasures of the senses, considered in the gross, results
the motive which, in a neutral sense, maybe termed physical desire: in a
bad sense, it is termed sensuality. Name used in a good sense it has
none. Of this, nothing can be determined, till it be considered
separately, with reference to the several species of pleasures to which
it corresponds.

XV. In particular, then, to the pleasures of the taste or palate
corresponds a motive, which in a neutral sense having received no name
that can serve to express it in all cases, can only be termed, by
circumlocution, the love of the pleasures of the palate. In particular
cases it is styled hunger: in others, thirst. The love of good cheer
expresses this motive, but seems to go beyond: intimating, that the
pleasure is to be partaken of in company, and involving a kind of
sympathy. In a bad sense, it is styled in some cases greediness,
voraciousness, gluttony: in others, principally when applied to
children, lickerishness. It may in some cases also be represented by the
word daintiness. Name used in a good sense it has none.

1. A boy, who does not want for victuals, steals a cake out of a
pastry-cook's shop, and eats it. In this case his motive will be
universally deemed a bad one: and if it be asked what it is, it may be
answered, perhaps, lickerishness.\\
2. A boy buys a cake out of a pastry-cook's shop, and eats it. In this
case his motive can scarcely be looked upon as either good or bad,
unless his master should be out of humour with him; and then perhaps he
may call it lickerishness, as before. In both cases, however, his motive
is the same. It is neither more nor less than the motive corresponding
to the pleasures of the palate.

XVI. To the pleasures of the sexual sense corresponds the motive which,
in a neutral sense, may be termed sexual desire. In a bad sense, it is
spoken of under the name of lasciviousness, and a variety of other names
of reprobation. Name used in a good sense it has none.

1. A man ravishes a virgin. In this case the motive is, without scruple,
termed by the name of lust, lasciviousness, and so forth; and is
universally looked upon as a bad one.\\
2. The same man, at another time, exercises the rights of marriage with
his wife. In this case the motive is accounted, perhaps, a good one, or
at least indifferent: and here people would scruple to call it by any of
those names. In both cases, however, the motive may be precisely the
same. In both cases it may be neither more nor less than sexual desire.

XVII. To the pleasures of curiosity corresponds the motive known by the
same name: and which may be otherwise called the love of novelty, or the
love of experiment; and, on particular occasions, sport, and sometimes
play.

1. A boy, in order to divert himself, reads an improving book: the
motive is accounted, perhaps, a good one: at any rate not a bad one.\\
2. He sets his top a spinning: the motive is deemed, at any rate, not a
bad one.\\
3. He sets loose a mad ox among a crowd; his motive is now, perhaps,
termed an abominable one. Yet in all three cases the motive may be the
very same: it may be neither more nor less than curiosity.

XVIII. As to the other pleasures of sense they are of too little
consequence to have given any separate denominations to the
corresponding motives.

XIX. To the pleasures of wealth corresponds the sort of motive which, in
a neutral sense, may be termed pecuniary interest: in a bad sense, it is
termed, in some cases, avarice, covetousness, rapacity, or lucre: in
other cases, niggardliness: in a good sense, but only in particular
cases, economy and frugality; and in some cases the word industry may be
applied to it: in a sense nearly indifferent, but rather bad than
otherwise, it is styled, though only in particular cases, parsimony.

1. For money you gratify a man's hatred, by putting his adversary to
death.\\
2. For money you plough his field for him.'' In the first case
your motive is termed lucre, and is accounted corrupt and abominable:
and in the second, for want of a proper appellation, it is styled
industry; and is looked upon as innocent at least, if not meritorious.
Yet the motive is in both cases precisely the same: it is neither more
nor less than pecuniary interest.

XX. The pleasures of skill are neither distinct enough, nor of
consequence enough, to have given any name to the corresponding motive.

XXI. To the pleasures of amity corresponds a motive which, in a neutral
sense, may be termed the desire of ingratiating one's self. In a bad
sense it is in certain cases styled servility: in a good sense it has no
name that is peculiar to it: in the cases in which it has been looked on
with a favourable eye, it has seldom been distinguished from the motive
of sympathy or benevolence, with which, in such cases, it is commonly
associated.

1. To acquire the affections of a woman before marriage, to preserve
them afterwards, you do every thing, that is consistent with other
duties, to make her happy: in this case your motive is looked upon as
laudable, though there is no name for it.\\
2. For the same purpose, you poison a woman with whom she is at enmity:
in this case your motive is looked upon as abominable, though still
there is no name for it.\\
3. To acquire or preserve the favour of a man who is richer or more
powerful than yourself, you make yourself subservient to his pleasures.
Let them even be lawful pleasures, if people choose to attribute your
behaviour to this motive, you will not get them to find any other name
for it than servility. Yet in all three cases the motive is the same: it
is neither more nor less than the desire of ingratiating yourself.

XXII. To the pleasures of the moral sanction, or, as they may otherwise
be called, the pleasures of a good name, corresponds a motive which, in
a neutral sense, has scarcely yet obtained any adequate appellative. It
may be styled, the love of reputation. It is nearly related to the
motive last preceding: being neither more nor less than the desire of
ingratiating one's self with, or, as in this case we should rather say,
of recommending one's self to, the world at large. In a good sense, it
is termed honour, or the sense of honour: or rather, the word honour is
introduced somehow or other upon the occasion of its being brought to
view: for in strictness the word honour is put rather to signify that
imaginary object, which a man is spoken of as possessing upon the
occasion of his obtaining a conspicuous share of the pleasures that are
in question. In particular cases, it is styled the love of glory. In a
bad sense, it is styled, in some cases, false honour; in others, pride;
in others, vanity. In a sense not decidedly bad, but rather bad than
otherwise, ambition. In an indifferent sense, in some cases, the love of
fame: in others, the sense of shame. And, as the pleasures belonging to
the moral sanction run undistinguishably into the pains derived from the
same source, it may also be styled, in some cases, the fear of
dishonour, the fear of disgrace, the fear of infamy, the fear of
ignominy, or the fear of shame.

1. You have received an affront from a man: according to the custom of
the country, in order, on the one hand, to save yourself from the shame
of being thought to bear it patiently; on the other hand, to obtain the
reputation of courage; you challenge him to fight with mortal weapons.
In this case your motive will by some people be accounted laudable, and
styled honour: by others it will be accounted blameable, and these, if
they call it honour, will prefix an epithet of improbation to it, and
call it false honour.

2. In order to obtain a post of rank and dignity, and thereby to
increase the respects paid you by the public, you bribe the electors who
are to confer it, or the judge before whom the title to it is in
dispute. In this case your motive is commonly accounted corrupt and
abominable, and is styled, perhaps, by some such name as dishonest or
corrupt ambition, as there is no single name for it.

3. In order to obtain the good-will of the public, you bestow a large
sum in works of private charity or public utility. In this case people
will be apt not to agree about your motive. Your enemies will put a bad
colour upon it, and call it ostentation: your friends, to save you from
this reproach, will choose to impute your conduct not to this motive but
to some other: such as that of charity (the denomination in this case
given to private sympathy) or that of public spirit.

4. A king, for the sake of gaining the admiration annexed to the name of
conqueror (we will suppose power and resentment out of the question)
engages his kingdom in a bloody war. His motive, by the multitude (whose
sympathy for millions is easily overborne by the pleasure which their
imagination finds in gaping at any novelty they observe in the conduct
of a single person) is deemed an admirable one. Men of feeling and
reflection, who disapprove of the dominion exercised by this motive on
this occasion, without always perceiving that it is the same motive
which in other instances meets with their approbation, deem it an
abominable one; and because the multitude, who are the manufacturers of
language, have not given them a simple name to call it by, they will
call it by some such compound name as the love of false glory or false
ambition. Yet in all four cases the motive is the same: it is neither
more nor less than the love of reputation.

XXIII. To the pleasures of power corresponds the motive which, in a
neutral sense, may be termed the love of power. People, who are out of
humour with it sometimes, call it the lust of power. In a good sense, it
is scarcely provided with a name. In certain cases this motive, as well
as the love of reputation, are confounded under the same name, ambition.
This is not to be wondered at, considering the intimate connexion there
is between the two motives in many cases: since it commonly happens,
that the same object which affords the one sort of pleasure, affords the
other sort at the same time: for instance, offices, which are at once
posts of honour and places of trust: and since at any rate reputation is
the road to power.

1. If, in order to gain a place in administration, you poison the man
who occupies it.\\
2. If, in the same view, you propose a salutary plan for the advancement
of the public welfare; your motive is in both cases the same. Yet in the
first case it is accounted criminal and abominable: in the second case
allowable, and even laudable.

XXIV. To the pleasures as well as to the pains of the religious sanction
corresponds a motive which has, strictly speaking, no perfectly neutral
name applicable to all cases, unless the s, word religion be admitted in
this character: though the word religion, strictly speaking, seems to
mean not so much the motive itself, as a kind of fictitious personage,
by whom the motive is supposed to be created, or an assemblage of acts,
supposed to be dictated by that personage: nor does it seem to be
completely settled into a neutral sense. In the same sense it is also,
in some cases, styled religious zeal: in other cases, the fear of God.
The love of God, though commonly contrasted with the fear of God, does
not come strictly under this head. It coincides properly with a motive
of a different denomination; viz., a kind of sympathy or good-will,
which has the Deity for its object. In a good sense, it is styled
devotion, piety, and pious zeal. In a bad sense, it is styled, in some
cases, superstition, or superstitious zeal: in other cases, fanaticism,
or fanatic zeal: in a sense not decidedly bad, because not appropriated
to this motive, enthusiasm, or enthusiastic zeal.

1. In order to obtain the favour of the Supreme Being, a man
assassinates his lawful sovereign. In this case the motive is now almost
universally looked upon as abominable, and is termed fanaticism:
formerly it was by great numbers accounted laudable, and was by them
called pious zeal.\\
2. In the same view, a man lashes himself with thongs. In this case, in
yonder house, the motive is accounted laudable, and is called pious
zeal: in the next house it is deemed contemptible, and called
superstition.\\
3. In the same view, a man eats a piece of bread (or at least what to
external appearance is a piece of bread) with certain ceremonies. In
this case, in yonder house, his motive is looked upon as laudable, and
is styled piety and devotion: in the next house it is deemed abominable,
and styled superstition, as before: perhaps even it is absurdly styled
impiety.\\
4. In the same view, a man holds a cow by the tail while he is dying. On
the Thames the motive would in this case be deemed contemptible, and
called superstition. On the Ganges it is deemed meritorious, and called
piety.\\
5. In the same view, a man bestows a large sum in works of charity, or
public utility. In this case the motive is styled laudable, by those at
least to whom the works in question appear to come under this
description: and by these at least it would be styled piety. Yet in all
these cases the motive is precisely the same: it is neither more nor
less than the motive belonging to the religious sanction.

XXV. To the pleasures of sympathy corresponds the motive which, in a
neutral sense, is termed good-will. The word sympathy may also be used
on this occasion: though the sense of it seems to be rather more
extensive. In a good sense, it is styled benevolence: and in certain
cases, philanthropy; and, in a figurative way, brotherly love; in
others, humanity; in others, charity; in others, pity and compassion; in
others, mercy; in others, gratitude; in others, tenderness; in others,
patriotism; in others, public spirit. Love is also employed in this as
in so many other senses. In a bad sense, it has no name applicable to it
in all cases: in particular cases it is styled partiality. The word
zeal, with certain epithets prefixed to it, might also be employed
sometimes on this occasion, though the sense of it be more extensive;
applying sometimes to ill as well as to good will. It is thus we speak
of party zeal, national zeal, and public zeal. The word attachment is
also used with the like epithets: we also say family-attachment. The
French expression, \emph{esprit de corps,} for which as yet there seems
to be scarcely any name in English, might be rendered, in some cases,
though rather inadequately, by the terms corporation spirit, corporation
attachment, or corporation zeal.

1. A man who has set a town on fire is apprehended and committed: out of
regard or compassion for him, you help him to break prison. In this case
the generality of people will probably scarcely know whether to condemn
your motive or to applaud it: those who condemn your conduct, will be
disposed rather to impute it to some other motive: if they style it
benevolence or compassion, they will be for prefixing an epithet, and
calling it false benevolence or false compassion.\\
2. The man is taken again, and is put upon his trial: to save him you
swear falsely in his favour. People, who would not call your motive a
bad one before, will perhaps call it so now.

3. A man is at law with you about an estate: he has no right to it: the
judge knows this, yet, having an esteem or affection for your adversary,
adjudges it to him. In this case the motive is by every body deemed
abominable, and is termed injustice and partiality.

4. You detect a statesman in receiving bribes: out of regard to the
public interest, you give information of it, and prosecute him. In this
case, by all who acknowledge your conduct to have originated from this
motive, your motive will be deemed a laudable one, and styled public
spirit. But his friends and adherents will not choose to account for
your conduct in any such manner: they will rather attribute it to party
enmity.

5. You find a man on the point of starving: you relieve him; and save
his life. In this case your motive will by every body be accounted
laudable, and it will be termed compassion, pity, charity, benevolence.
Yet in all these cases the motive is the same: it is neither more nor
less than the motive of good-will.

XXVI. To the pleasures of malevolence, or antipathy, corresponds the
motive which, in a neutral sense, is termed antipathy or displeasure:
and, in particular cases, dislike, aversion, abhorrence, and
indignation: in a neutral sense, or perhaps a sense leaning a little to
the bad side, ill-will: and, in particular cases, anger, wrath, and
enmity. In a bad sense it is styled, in different cases, wrath, spleen,
ill-humour, hatred, malice, rancour, rage, fury, cruelty, tyranny, envy,
jealousy, revenge, misanthropy, and by other names, which it is hardly
worth while to endeavour to collect. Like good-will, it is used with
epithets expressive of the persons who are the objects of the affection.
Hence we hear of party enmity, party rage, and so forth. In a good sense
there seems to be no single name for it. In compound expressions it may
be spoken of in such a sense, by epithets, such as \emph{just} and
\emph{laudable,} prefixed to words that are used in a neutral or nearly
neutral sense.

1. You rob a man: he prosecutes you, and gets you punished: out of
resentment you set upon him, and hang him with your own hands. In this
case your motive will universally be deemed detestable, and will be
called malice, cruelty, revenge, and so forth.

2. A man has stolen a little money from you: out of resentment you
prosecute him, and get him hanged by course of law. In this case people
will probably be a little divided in their opinions about your motive:
your friends will deem it a laudable one, and call it a just or laudable
resentment: your enemies will perhaps be disposed to deem it blameable,
and call it cruelty, malice, revenge, and so forth: to obviate which,
your friends will try perhaps to change the motive, and call it public
spirit.

3. A man has murdered your father: out of resentment you prosecute him,
and get him put to death in course of law. In this case your motive will
be universally deemed a laudable one, and styled, as before, a just or
laudable resentment: and your friends, in order to bring forward the
more amiable principle from which the malevolent one, which was your
immediate motive, took its rise, will be for keeping the latter out of
sight, speaking of the former only, under some such name as filial
piety. Yet in all these cases the motive is the same: it is neither more
nor less than the motive of ill-will.

XXVII. To the several sorts of pains, or at least to all such of them as
are conceived to subsist in an intense degree, and to death, which, as
far as we can perceive, is the termination of all the pleasures, as well
as all the pains we are acquainted with, corresponds the motive, which
in a neutral sense is styled, in general, self-preservation: the desire
of preserving one's self from the pain or evil in question. Now in many
instances the desire of pleasure, and the sense of pain, run into one
another undistinguishably. Self-preservation, therefore, where the
degree of the pain which it corresponds to is but slight will scarcely
be distinguishable, by any precise line, from the motives corresponding
to the several sorts of pleasures. Thus in the case of the pains of
hunger and thirst: physical want will in many cases be scarcely
distinguishable from physical desire. In some cases it is styled, still
in a neutral sense, self-defence. Between the pleasures and the pains of
the moral and religious sanctions, and consequently of the motives that
correspond to them, as likewise between the pleasures of amity, and the
pains of enmity, this want of boundaries has already been taken notice
of. The case is the same between the pleasures of wealth, and the pains
of privation corresponding to those pleasures. There are many cases,
therefore, in which it will be difficult to distinguish the motive of
self-preservation from pecuniary interest, from the desire of
ingratiating one's self, from the love of reputation, and from religious
hope: in which cases, those more specific and explicit names will
naturally be preferred to this general and inexplicit one. There are
also a multitude of compound names, which either are already in use, or
might be devised, to distinguish the specific branches of the motive of
self-preservation from those several motives of a pleasurable origin:
such as the fear of poverty, the fear of losing such or such a man's
regard, the fear of shame, and the fear of God. Moreover, to the evil of
death corresponds, in a neutral sense, the love of life; in a bad sense,
cowardice: which corresponds also to the pains of the senses, at least
when considered as subsisting in an acute degree. There seems to be no
name for the love of life that has a good sense; unless it be the vague
and general name of prudence.

1. To save yourself from being hanged, pilloried, imprisoned, or fined,
you poison the only person who can give evidence against you. In this
case your motive will universally be styled abominable: but as the term
self-preservation has no bad sense, people will not care to make this
use of it: they will be apt rather to change the motive, and call it
malice.

2. A woman, having been just delivered of an illegitimate child, in
order to save herself from shame, destroys the child, or abandons it. In
this case, also, people will call the motive a bad one, and, not caring
to speak of it under a neutral name, they will be apt to change the
motive, and call it by some such name as cruelty.

3. To save the expense of a halfpenny, you suffer a man, whom you could
preserve at that expense, to perish with want, before your eyes. In this
case your motive will be universally deemed an abominable one; and, to
avoid calling it by so indulgent a name as self-preservation, people
will be apt to call it avarice and niggardliness, with which indeed in
this case it indistinguishably coincides: for the sake of finding a more
reproachful appellation, they will be apt likewise to change the motive,
and term it cruelty.

4. To put an end to the pain of hunger, you steal a loaf of bread. In
this case your motive will scarcely, perhaps, be deemed a very bad one;
and, in order to express more indulgence for it, people will be apt to
find a stronger name for it than self-preservation, terming it
\emph{necessity.}

5. To save yourself from drowning, you beat off an innocent man who has
got hold of the same plank. In this case your motive will in general be
deemed neither good nor bad, and it will be termed self-preservation, or
necessity, or the love of life.

6. To save your life from a gang of robbers, you kill them in the
conflict. In this case the motive may, perhaps, be deemed rather
laudable than otherwise, and, besides self-preservation, is styled also
self-defence.

7. A soldier is sent out upon a party against a weaker party of the
enemy: before he gets up with them, to save his life, he runs away. In
this case the motive will universally be deemed a contemptible one, and
will be called cowardice. Yet in all these various cases, the motive is
still the same. It is neither more nor less than self-preservation.

XXVIII. In particular, to the pains of exertion corresponds the motive,
which, in a neutral sense, may be termed the love of ease, or by a
longer circumlocution, the desire of avoiding trouble. In a bad sense,
it is termed indolence. It seems to have no name that carries with it a
good sense.

1. To save the trouble of taking care of it, a parent leaves his child
to perish. In this case the motive will be deemed an abominable one,
and, because indolence will seem too mild a name for it, the motive
will, perhaps, be changed, and spoken of under some such term as
cruelty.

2. To save yourself from an illegal slavery, you make your escape. In
this case the motive will be deemed certainly not a bad one: and,
because indolence, or even the love of ease, will be thought too
unfavourable a name for it, it will, perhaps, be styled the love of
liberty.

3. A mechanic, in order to save his labour, makes an improvement in his
machinery. In this case, people will look upon his motive as a good one;
and finding no name for it that carries a good sense, they will be
disposed to keep the motive out of sight: they will speak rather of his
ingenuity, than of the motive which was the means of his manifesting
that quality. Yet in all these cases the motive is the same: it is
neither more nor less than the love of ease.

XXIX. It appears then that there is no such thing as any sort of motive
which is a bead one in itself: nor, consequently, any such thing as a
sort of motive, which in itself is exclusively a good one. And as to
their effects, it appears too that these are sometimes bad, at other
times either indifferent or good: and this appears to be the case with
every sort of motive. \emph{If any sort of motive then is either good or
bad on the score} of its effects, this is the case only on individual
occasions, and with individual motives; and this is the case with one
sort of motive as well as with another. \emph{If any sort of motive then
can, in consideration of its effects, be termed with any propriety a bad
one,} it can only be with reference to the balance of all the effects it
may have had of both kinds within a given period, that is, of its most
usual tendency.

XXX. What then? (it will be said) are not lust, cruelty, avarice, bad
motives? Is there so much as any one individual e occasion, in which
motives like these can be otherwise than bad? No, certainly: and yet the
proposition, that there is no one \emph{sort} of motive but what will on
many occasions be a good one, is nevertheless true. The fact is, that
these are names which, if properly applied, are never applied but in the
cases where the motives they signify happen to be bad. The names of
those motives, considered apart from their effects, are sexual desire,
displeasure, and pecuniary interest. To sexual desire, when the effects
of it are looked upon as bad, is given the name of lust. Now lust is
always a bad motive. Why? Because if the case be such, that the effects
of the motive are not bad, it does not go, or at least ought not to go,
by the name of lust. The case is, then, that when I say, ``Lust is a bad
motive,'' it is a proposition that merely concerns the import of the
word lust; and which would be false if transferred to the other word
used for the same motive, sexual desire. Hence we see the emptiness of
all those rhapsodies of commonplace morality, which consist in the
taking of such names as lust, cruelty, and avarice, and branding them
with marks of reprobation: applied to the \emph{thing,} they are false;
applied to the \emph{name,} they are true indeed, but nugatory. Would
you do a real service to mankind, show them the cases in which sexual
desire \emph{merits} the name of lust; displeasure, that of cruelty; and
pecuniary interest, that of avarice.

XXXI. If it were necessary to apply such denominations as good, bad, and
indifferent to motives, they might be classed in the following manner,
in consideration of the most frequent complexion of their effects. In
the class of good motives might begs placed the articles of,\\
1. Good-will.\\
2. Love of reputation.\\
3. Desire of amity. And,\\
4. Religion.\\
In the class of bad motives,\\
5. Displeasure.\\
In the class of neutral or indifferent motives,\\
6. Physical desire.\\
7. Pecuniary interest.\\
8. Love of power.\\
9. Self-preservation; as including the fear of the pains of the senses,
the love of ease, and the love of life.

XXXII. This method of arrangement, however, cannot but be imperfect; and
the nomenclature belonging to it is in danger of being fallacious. For
by what method of investigation can a man be assured, that with regard
to the motives ranked under the name of good, the good effects they have
had, from the beginning of the world, have, in each of the four species
comprised under this name, been superior to the bad? still more
difficulty would a man find in assuring himself, that with regard to
those which are ranked under the name of neutral or indifferent, the
effects they have had have exactly balanced each other, the value of the
good being neither greater nor less than that of the bad. It is to be
considered, that the interests of the person himself can no more be left
out of the estimate, than those of the rest of the community. For what
would become of the species, if it were not for the motives of hunger
and thirst, sexual desire, the fear of pain, and the love of life? Nor
in the actual constitution of human nature is the motive of displeasure
less necessary, perhaps, than any of the others: although a system, in
which the business of life might be carried on without it, might
possibly be conceived. It seems, therefore, that they could scarcely,
without great danger of mistakes, be distinguished in this manner even
with reference to each other.

XXXIII. The only way, it should seem, in which a motive can with safety
and propriety be styled good or bad, is with reference to its effects in
each individual instance; and principally from the intention it gives
birth to: from which arise, as will be shown hereafter, the most
material part of its effects. A motive is good, when the intention it
gives birth to is a good one; bad, when the intention is a bad one: and
an intention is good or bad, according to the material consequences that
are the objects of it. So far is it from the goodness of the intention's
being to be known only from the species of the motive. But from one and
the same motive, as we have seen, may result intentions of every sort of
complexion whatsoever. This circumstance, therefore, can afford no clue
for the arrangement of the several sorts of motives.

XXXIV. A more commodious method, therefore, it should seem, would be to
distribute them according to the influence which they appear to have on
the interests of the other members of the community, laying those of the
party himself out of the question: to wit, according to the tendency
which they appear to have to unite, or disunite, his interests and
theirs. On this plan they may be distinguished into \emph{social,
dissocial,} and \emph{self-regarding.} In the social class may be
reckoned,\\
1. Good-will.\\
2. Love of reputation.\\
3. Desire of amity.\\
4. Religion. In the dissocial may be placed,\\
5. Displeasure. In the self-regarding class,\\
6. Physical desire.\\
7. Pecuniary interest.\\
8. Love of power.\\
9. Self-preservation; as including the fear of the pains of the senses,
the love of ease, and the love of life.

XXXV. With respect to the motives that have been termed social, if any
farther distinction should be of use, to that of good-will alone may be
applied the epithet of \emph{purely-social;} while the love of
reputation, the desire of amity, and the motive of religion, may
together be comprised under the division of \emph{semi-social:} the
social tendency being much more constant and unequivocal in the former
than in any of the three latter. Indeed these last, social as they may
be termed, are self-regarding at the same time.\\

§4. Order of pre-eminence among motives.

XXXVI. Of all these sorts of motives, good-will is that of which the
dictates, taken in a general view, are surest of coinciding with those
of the principle of utility. For the dictates of utility are neither nor
less than the dictates of the most extensive 8 and enlightened (that is
\emph{well-advised)} benevolence. The dictates of the other motives may
be conformable to those of utility, or repugnant, as it may happen.

XXXVII. In this, however, it is taken for granted, that in the case in
question the dictates of benevolence are not contradicted by those of a
more extensive, that is enlarged, benevolence. Now when the dictates of
benevolence, as respecting the interests of a certain set of persons,
are repugnant to the dictates of the same motive, as respecting the more
important (or valuable) interests of another set of persons, the former
dictates, it is evident, are repealed, as it were, by the latter: and a
man, were he to be governed by the former, could scarcely, with
propriety, be said to be governed by the dictates of benevolence. On
this account were the motives on both sides sure to be alike present to
a man's mind, the case of such a repugnancy would hardly be worth
distinguishing, since the partial benevolence might be considered as
swallowed up in the more extensive: if the former prevailed, and
governed the action, it must be considered as not owing its birth to
benevolence, but to some other motive: if the latter prevailed, the
former might be considered as having no effect. But the case is, that a
partial benevolence may govern the action, without entering into any
direct competition with the more extensive benevolence, which would
forbid it; because the interests of the less numerous assemblage of
persons may be present to a man's mind, at a time when those of the more
numerous are either not present, or, if present, make no impression. It
is in this way that the dictates of this motive may be repugnant to
utility, yet still be the dictates of benevolence. What makes those of
private benevolence conformable upon the whole to the principle of
utility, is, that in general they stand unopposed by those of public: if
they are repugnant to them, it is only by accident. What makes them the
more conformable, is, that in a civilized society, in most of the cases
in which they would of themselves be apt to run counter to those of
public benevolence, they find themselves opposed by stronger motives of
the self-regarding class, which are played off against them by the laws;
and that it is only in cases where they stand unopposed by the other
more salutary dictates, that they are left free. An act of injustice or
cruelty, committed by a man for the sake of his father or his son, is
punished, and with reason, as much as if it were committed for his own.

XXXVIII. After good-will, the motive of which the dictates seem to have
the next best chance for coinciding with those of utility, is that of
the love of reputation. There is but one circumstance which prevents the
dictates of this motive from coinciding in all cases with those of the
former. This is, that men in their likings and dislikings, in the
dispositions they manifest to annex to any mode of conduct their
approbation or their disapprobation, and in consequence to the person
who appears to practice it, their good or their ill will, do not govern
themselves exclusively by the principle of utility. sometimes it is the
principle of asceticism they are guided by: sometimes the principle of
sympathy and antipathy. There is another circumstance, which diminishes,
not their conformity to the principle of utility, but only their
efficacy in comparison with the dictates of the motive of benevolence.
The dictates of this motive will operate as strongly in secret as in
public: whether it appears likely that the conduct which they recommend
will be known or not: those of the love of reputation will coincide with
those of benevolence only in proportion as a man's conduct seems likely
to be known. This circumstance, however, does not make so much
difference as at first sight might appear. Acts, in proportion as they
are material, are apt to become known: and in point of reputation, the
slightest suspicion often serves for proof. Besides, if an act be a
disreputable one, it is not any assurance a man can have of the secrecy
of the particular act in question, that will of course surmount the
objections he may have against engaging in it. Though the act in
question should remain secret, it will go towards forming a habit, which
may give birth to other acts, that may not meet with the same good
fortune. There is no human being, perhaps, who is at years of
discretion, on whom considerations of this sort have not some weight:
and they have the more weight upon a man, in proportion to the strength
of his intellectual powers, and the firmness of his mind. Add to this,
the influence which habit itself, when once formed, has in restraining a
man from acts towards which, from the view of the disrepute annexed to
them, as well as from any other cause, he has contracted an aversion.
The influence of habit, in such cases, is a matter of fact, which,
though not readily accounted for, is acknowledged and indubitable.

XXXIX. After the dictates of the love of reputation come, as it should
seem, those of the desire of amity. The former are disposed to coincide
with those of utility, inasmuch as they are disposed to coincide with
those of benevolence. Now those of the desire of amity are apt also to
coincide, in a certain sort, with those of benevolence. But the sort of
benevolence with the dictates of which the love of reputation coincides,
is the more extensive; that with which those of the desire of amity
coincide, the less extensive. Those of the love of amity have still,
however, the advantage of those of the self-regarding motives. The
former, at one period or other of his life, dispose a man to contribute
to the happiness of a considerable number of persons: the latter, from
the beginning of life the end of it, confine themselves to the care of
that single individual. The dictates of the desire of amity, it is
plain, will approach nearer to a coincidence with those of the love of
reputation, and thence with those of utility, in proportion,
\emph{cæteris paribas,} to the number of the persons whose amity a man
has occasion to desire: and hence it is, for example, that an English
member of parliament, with all his own weaknesses, and all the follies
of the people whose amity he has to cultivate, is probably, in general,
a better character than the secretary of a visier at Constantinople, or
of a naib in Indostan.

XL. The dictates of religion are, under the infinite diversity of
religions, so extremely variable, that it is difficult to know what
general account to give of them, or in what rank to place the motive
they belong to. Upon the mention of religion, people's first thoughts
turn naturally to the religion they themselves profess. This is a great
source of miscalculation, and has a tendency to place this sort of
motive in a higher rank than it deserves. The dictates of religion would
coincide, in all cases, with those of utility, were the Being, who is
the object of religion, universally supposed to be as benevolent as he
is supposed to be wise and powerful; and were the notions entertained of
his benevolence, at the same time, as correct as those which are
entertained of his wisdom and his power. Unhappily, however, neither of
these is the case. He is universally supposed to be all-powerful: for by
the Deity, what else does any man mean than the Being, whatever he be,
by whom every thing is done. And as to knowledge, by the same rule that
he should know one thing he should know another. These notions seem to
be as correct, for all material purposes, as they are universal. But
among the votaries of religion (of which number the multifarious
fraternity of Christians is but a small part) there seem to be but few
(I will not say how few) who are real believers in his benevolence. They
call him benevolent in words, but they do not mean that he is so in
reality. They do not mean, that he is benevolent as man is conceived to
be benevolent: they do not mean that he is benevolent in the only sense
in which benevolence has a meaning. For if they did, they would
recognize that the dictates of religion could be neither more nor less
than the dictates of utility: not a tittle different: not a tittle less
or more. But the case is, that on a thousand occasions they turn their
backs on the principle of utility. They go astray after the strange
principles its antagonists: sometimes it is the principle of asceticism:
sometimes the principle of sympathy and antipathy. Accordingly, the idea
they bear in their minds, on such occasions, is but too often the idea
of malevolence; to which idea, stripping it of its own proper name, they
bestow the specious appellation of the social motive. The dictates of
religion, in short, are no other than the dictates of that principle
which has been already mentioned under the name of the theological
principle. These, as has been observed, are just as it may happen,
according to the biases of the person in question, copies of the
dictates of one or other of the three original principles: sometimes,
indeed, of the dictates of utility: but frequently of those of
asceticism, or those of sympathy and antipathy. In this respect they are
only on a par with the dictates of the love of reputation: in another
they are below it. The dictates of religion are in all places intermixed
more or less with dictates unconformable to those of utility, deduced
from tests, well or ill interpreted, of the writings held for sacred by
each sect: unconformable, by imposing practices sometimes inconvenient
to a man's self, sometimes pernicious to the rest of the community. The
sufferings of uncalled martyrs, the calamities of holy wars and
religious persecutions, the mischiefs of intolerant laws, (objects which
can here only be glanced at, not detailed) are so many additional
mischiefs over and above the number of those which were ever brought
into the world by the love of reputation. On the other hand, it is
manifest, that with respect to the power of operating in secret, the
dictates of religion have the same advantage over those of the love of
reputation, and the desire of amity, as is possessed by the dictates of
benevolence.

XLI. Happily, the dictates of religion seem to approach nearer and
nearer to a coincidence with those of utility every day. But why?
Because the dictates of the moral sanction do so: and those coincide
with or are influenced by these. Men of the worst religions, influenced
by the voice and practice of the surrounding world, borrow continually a
new and a new leaf out of the book of utility: and with these, in order
not to break with their religion, they endeavour, sometimes with
violence enough, to patch together and adorn the repositories of their
faith.

XLII. As to the self-regarding and dissocial motives, the order that
takes place among these, and the preceding one, in point of
extra-regarding influence, is too evident to need insisting on. As to
the order that takes place among the motives, of the self-regarding
class, considered in comparison with one another, there seems to be no
difference which on this occasion would be worth mentioning. With
respect to the dissocial motive, it makes a difference (with regard to
its extra-regarding effects) from which of two sources it originates;
whether from self-regarding or from social considerations. The
displeasure you conceive against a man may be founded either on some act
which offends you in the first instance, or on an act which offends you
no otherwise than because you look upon it as being prejudicial to some
other party on whose behalf you interest yourself: which other party may
be of course either a determinate individual, or any assemblage of
individuals, determinate or indeterminate. It is obvious enough, that a
motive, though in itself dissocial, may, by issuing from a social
origin, possess a social tendency; and that its tendency, in this case,
is likely to be the more social, the more enlarged the description is of
the persons whose interests you espouse. Displeasure, venting itself
against a man, on account of a mischief supposed to be done by him to
the public, may be more social in its effects than any good-will, the
exertions of which are confined to an individual.

§ 5. Conflict among motives .

XLIII. When a man has it in contemplation to engage in any action, he is
frequently acted upon at the same time by the force of divers motives:
one motive, or set of motives, acting in one direction; another motive,
or set of motives, acting as it were in an opposite direction. The
motives on one side disposing him to engage in the action: those on the
other, disposing him not to engage in it. Now, any motive, the influence
of which tends to dispose him to engage in the action in question, may
be termed an \emph{impelling motive:} any motive, the influence of which
tends to dispose him not to engage in it, a \emph{restraining} motive.
But these appellations may of course be interchanged, according as the
act is of the positive kind, or the negative.

XLIV. It has been shown, that there is no sort of motive but may give
birth to any sort of action. It follows, therefore, that there are no
two motives but may come to be opposed to one another. Where the
tendency of the act is bad, the most common case is for it to have been
dictated by a motive either of the self-regarding, or of the dissocial
class. In such case the motive of benevolence has commonly been acting,
though ineffectually, in the character of a restraining motive.

XLV. An example may be of use, to show the variety of contending
motives, by which a man may be acted upon at the same time. Crillon, a
Catholic (at a time when it was generally thought meritorious among
Catholics to extirpate Protestants), was ordered by his king, Charles
IX. of France, to fall privately upon Coligny, a Protestant, and
assassinate him: his answer was, ``Excuse me, Sire; but I'll fight him
with all my heart.'' Here, then, were all the three forces above
mentioned, including that of the political sanction, acting upon him at
once. By the political sanction, or at least so much of the force of it
as such a mandate, from such a sovereign, issued on such an occasion,
might be supposed to carry with it, he was enjoined to put Coligny to
death in the way of assassination: by the religious sanction, that is,
by the dictates of religious zeal, he was enjoined to put him to death
in any way: by the moral sanction, or in other words, by the dictates of
honour, that is, of the love of reputation, he was permitted (which
permission, when coupled with the mandates of his sovereign, operated,
he conceived, as an injunction) to fight the adversary upon equal terms:
by the dictates of enlarged benevolence (supposing the mandate to be
unjustifiable) he\\
was enjoined not to attempt his life in any way, but to remain at peace
with him: supposing the mandate to be unjustifiable, by the dictates of
private benevolence he was enjoined not to meddle with him at any rate.
Among this confusion of repugnant dictates, Crillon, it seems, gave the
preference, in the first place, to those of honour: in the next place,
to those of benevolence. He would have fought, had his offer been
accepted; as it was not, he remained at peace.

Here a multitude of questions might arise. Supposing the dictates of the
political sanction to follow the mandate of the sovereign, of what kind
were the motives which they afforded him for compliance? The answer is,
of the self-regarding kind at any rate: inasmuch as, by the supposition,
it was in the power of the sovereign to punish him for non-compliance,
or reward him for compliance. Did they afford him the motive of religion
(I mean independently of the circumstance of heresy above mentioned) the
answer is, Yes, if his notion was, that it was God's pleasure he should
comply with them; No, if it was not. Did they afford him the motive of
the love of reputation? Yes, if it was his notion that the world would
expect and require that he should comply with them: No, if it was not.
Did they afford him that of benevolence? Yes, if it was his notion that
the community would upon the whole be the better for his complying with
them: No, if it was not. But did the dictates of the political sanction,
in the case in question, actually follow the mandates of the sovereign:
in other words, was such a mandate legal? This we see is a mere question
of local jurisprudence, altogether foreign to the present purpose.

XLVI. What is here said about the goodness and badness of motives, is
far from being a mere matter of words. There will be occasion to make
use of it hereafter for various important purposes. I shall have need of
it for the sake of dissipating various prejudices, which are of
disservice to the community, sometimes by cherishing the flame of civil
dissensions, at other times, by obstructing the course of justice. It
will be shown, that in the case of many offences, the consideration of
the motive is a most material one: for that in the first place it makes
a very material difference in the magnitude of the mischief: in the next
place, that it is easy to be ascertained; and thence may be made a
ground for a difference in the demand for punishment: but that in other
cases it is altogether incapable of being ascertained; and that, were it
capable of being ever so well ascertained, good or bad, it could make no
difference in the demand for punishment: that in all cases, the motive
that may happen to govern a prosecutor, is a consideration totally
immaterial: whence maybe seen the mischievousness of the prejudice that
is so apt to be entertained against informers; and the consequence it is
of that the judge, in particular, should be proof against the influence
of such delusions.\\
Lastly, The subject of motives is one with which it is necessary to be
acquainted, in order to pass a judgment on any means that may be
proposed for combating offenses in their source.\\
But before the theoretical foundation for these practical observations
can be completely laid, it is necessary we should say something on the
subject of \emph{disposition:} which, accordingly, will furnish matter
for the ensuing chapter.

\chapter{Human Dispositions in General}

I. In the foregoing chapter it has been shown at large. that goodness or
badness can not, with any propriety, be predicated of motives. Is there
nothing then about a man that may properly be termed good or bad, when,
on such or such an occasion; he suffers himself to be governed by such
or such a motive. Yes, certainly: his \emph{disposition.} Now
disposition is a kind of fictitious entity, feigned for the convenience
of discourse, in order to express what there is supposed to be
\emph{permanent} in a man's frame of mind, where, on such or such an
occasion, he has been influenced by sued or such a motive, to engage in
an act, which, as it appeared to him, was of such or such a tendency.

II. It is with disposition as with every thing else: it will be good or
bad according to its effects: according to the effects it has in
augmenting or diminishing the happiness of the community. A man's
disposition may accordingly be considered in two points of view:
according to the influence it has, either, 1. on his own happiness: or,
2. on the happiness of others. Viewed in both these lights together, or
in either of them indiscriminately, it may be termed, on the one hand,
good; on the other, bad; or, in flagrant cases, depraved. Viewed in the
former of these lights, it has scarcely any peculiar name, which has as
yet been appropriated to it. It might be termed, though but,
inexpressively, frail or infirm, on the one hand: sound or firm, on the
other. Viewed in the other light, it might be termed beneficent, or
meritorious, on the one hand: pernicious or mischievous, on the other.
Now of that branch of a man's disposition, the effects of which regard
in the first instance only himself, there needs not much to be said
here. To reform it when bad, is the business rather of the moralist than
the legislator: nor is it susceptible of those various modifications
which make so material difference in the effects of the other. Again,
with respect to that part of it, the effects whereof regard others in
the first instance, it is only in as far as it is of a mischievous
nature that the penal branch of law has any immediate concern with it:
in as far as it may be of a beneficent nature, it belongs to a hitherto
but little cultivated, and as yet unnamed branch of law, which might be
styled the remuneratory.

III. A man then is said to be of a mischievous disposition, when, by the
influence of no matter what motives, he is \emph{presumed} to be more
apt to engage, or form intentions of engaging, in acts which are
\emph{apparently} of a pernicious tendency, than in such as are
apparently of a beneficial tendency: of a meritorious or beneficent
disposition in the opposite case.

IV. I say presumed: for, by the supposition, all that appears is one
single action, attended with one single train of circumstances: but from
that degree of consistency and uniformity which experience has shown to
be observable in the different actions of the same person, the probable
existence (past or future) of a number of acts of a similar nature, is
naturally and justly inferred from the observation of one single one.
Under such circumstances, such as the motive proves to be in one
instance, such is the disposition to be presumed to be in others.

V. I say \emph{apparently} mischievous: that is, apparently with regard
to him: such as to him appear to possess that tendency: for from the
mere event, independent of what to him it appears beforehand likely to
be, nothing can be inferred on either side. If to him it appears likely
to be mischievous, in such case, though in the upshot it should prove
innocent, or even beneficial, it makes no difference; there is not the
less reason for presuming his disposition to be a bad one: if to him it
appears likely to be beneficial or innocent, in such case, though in the
upshot it should prove pernicious, there is not the more reason on that
account for presuming his disposition to be a good one. And here we see
the importance of the circumstances of intentionality, consciousness,
unconsciousness, and mis-supposal.

VI. The truth of these positions depends upon two others, both of them
sufficiently verified by experience: The one is, that in the ordinary
course of things the consequences of actions commonly turn out
conformable to intentions. A man who sets up a butcher's shop, and deals
in beef, when he intends to knock down an ox, commonly does knock down
an ox; though by some unlucky accident he may chance to miss his blow
and knock down a man: he who sets up a grocer's shop, and deals sugar,
when he intends to sell sugar, commonly does sell sugar: though by some
unlucky accident he may chance to sell arsenic in the room of it.

VII. The other is, that a man who entertains intentions of doing
mischief at one time is apt to entertain the like intentions of another.

VIII. There are two circumstances upon which the nature of the
disposition, as indicated by any act, is liable to depend:\\
1. The apparent tendency of the act:\\
2. The nature of the motive which gave birth to it. This dependency is
subject to different rules, according to the nature of the motive. In
stating them, I suppose all along the apparent tendency of the act to
be, as it commonly is, the same as the real.

IX. 1. Where the tendency of the act is \emph{good,} and the motive is
of the \emph{self-regarding} kind. In this case the motive affords no
inference on either side. It affords no indication of a good
disposition: but neither does it afford any indication of a bad one.
\textless{} BR\textgreater{}A baker sells his bread to a hungry man who
asks for it. This, we see, is one of those acts of which, in ordinary
cases, the tendency is unquestionably good. The baker's motive is the
ordinary commercial motive of pecuniary interest. It is plain, that
there is nothing in the transaction, thus stated, that can afford the
least ground for presuming that the baker is a better or a worse man
than any of his neighbours.

X. 2. Where the tendency of the act is bad, and the motive, as before,
is of the self-regarding kind. In this case the disposition indicated is
a mischievous one.\\
A man steals bread out of a baker's shop: this is one of those of which
the tendency will readily be acknowledged to be bad. Why, and in what
respects it is so, will be stated farther on. His motive, we will say,
is that of pecuniary interest; the desire of getting the value of the
bread for nothing. His disposition, accordingly, appears to be a bad
one: for every one will allow a thievish disposition to be a bad one.

XI. 3. Where the tendency of the act is \emph{good,} and the motive is
the purely social one of \emph{good-will.} In this case the disposition
indicated is a beneficent one.

A baker gives a poor man a loaf of bread. His motive is compassion; a
name given to the motive of benevolence, in particular cases of its
operation. The disposition indicated by the baker, in this case, is such
as every man will be ready enough to acknowledge to be a good one.

XII. 4. Where the tendency of the act is \emph{bad,} and the motive is
the purely social one of good-will. Even in this case the disposition
which the motive indicates is dubious: it may be a mischievous or a
meritorious one, as it happens; according as the mischievousness of the
act is more or less apparent.

XIII. It may be thought, that a case of this sort cannot exist; and that
to suppose it, is a contradiction in terms. For the act is one, which,
by the supposition, the agent knows to be a mischievous one. How then
can it be, that good-will, that is, the desire of doing good, could have
been the motive that led him into it? To reconcile this, we must advert
to the distinction between enlarged benevolence and confined. The motive
that led him into it, was that of confined benevolence. Had he followed
the dictates of enlarged benevolence, he would not have done what he
did. Now, although he followed the dictates of that branch of
benevolence, which in any single instance of its exertion is
mischievous, when opposed to the other, yet, as the cases which call for
the exertion of the former are, beyond comparison, more numerous than
those which call for the exertion of the latter, the disposition
indicated by him, in following the impulse of the former, will often be
such as in a man, of the common run of men, may be allowed to be a good
one upon the whole.

XIV. A man with a numerous family of children, on the point of starving,
goes into a baker's shop, steals a loaf, divides it all among the
children, reserving none of it for himself. It will be hard to infer
that that man's disposition is a mischievous one upon the whole. Alter
the case, give him but one child, and that hungry perhaps, but in no
imminent danger of starving: and now let the man set fire to a house
full of people, for the sake of stealing money out of it to buy the
bread with. The disposition here indicated will hardly be looked upon as
a good one.

XV. Another case will appear more difficult to decide than either.
Ravaillac assassinated one of the best and wisest of sovereigns, at a
time when a good and wise sovereign, a blessing at all times so valuable
to a state, was particularly precious: and that to the inhabitants of a
populous and extensive empire. He is taken, and doomed to the most
excruciating tortures. His son, well persuaded of his being a sincere
penitent, and that mankind, in case of his being at large, would have
nothing more to fear from him, effectuates his escape. Is this then a
sign of a good disposition in the son, or of a bad one? Perhaps some
will answer, of a bad one; for, besides the interest which the nation
has in the sufferings of such a criminal, on the score of the example,
the future good behaviour of such a criminal is more than any one can
have sufficient ground to be persuaded of.

XVI. Well then, let Ravaillac, the son, not facilitate his father's
escape; but content himself with conveying poison to him, that at the
price of an easier death he may escape his torments. The decision will
now, perhaps, be more difficult. The act is a wrong one, let it be
allowed, and such as ought by all means to be punished: but is the
disposition manifested by it a bad one? Because the young man breaks the
laws in this one instance, is it probable, that if let alone, he would
break the laws in ordinary instances, for the satisfaction of any
inordinate desires of his own? The answer of most men would probably be
in the negative.

XVII. 5. Where the tendency of the act is \emph{good,} and the motive is
a semi-social one, the \emph{love of reputation.} In this case the
disposition indicated is a good one.\\
In a time of scarcity, a baker, for the sake of gaining the esteem of
the neighbourhood, distributes bread \emph{gratis} among the industrious
poor. Let this be taken for granted: and let it be allowed to be a
matter of uncertainty, whether he had any real feeling for the
sufferings of those whom he has relieved, or no. His disposition, for
all that, cannot, with any pretence of reason, be termed otherwise than
a good and beneficent one. It can only be in consequence of some very
idle prejudice, if it receives a different name.

XVIII. 6. Where the tendency of the act is \emph{bad,} and the motive,
as before, is a semi-social one, the love of reputation. In this case,
the disposition which it indicates is more or less good or bad: in the
first place, according as the tendency of the act is more or less
mischievous: in the next place according as the dictates of the moral
sanction, in the society in question, approach more or less to a
coincidence with those of utility. It does not seem probable, that in
any nation, which is in a state of tolerable civilization, in short, in
any nation in which such rules as these can come to be consulted, the
dictates of the moral sanction will so far recede from a coincidence
with those of utility (that is, of enlightened benevolence) that the
disposition indicated in this case can be otherwise than a good one upon
the whole.

XIX. An Indian receives an injury, real or imaginary, from an Indian of
another tribe. He revenges it upon the person of his antagonist with the
most excruciating torments: the case being, that cruelties inflicted on
such an occasion, gain him reputation in his own tribe. The disposition
manifested in such a case can never be deemed a good one, among a people
ever so few degrees advanced, in point of civilization, above the
Indians.

XX. A nobleman (to come back to Europe) contracts a debt with a poor
tradesman. The same nobleman, presently afterwards, contracts a debt, to
the same amount, to another nobleman, at play. He is unable to pay both:
he pays the whole debt to the companion of his amusements, and no part
of it to the tradesman. The disposition manifested in this case can
scarcely be termed otherwise than a bad one. It is certainly, however,
not so bad as if he had paid neither. The principle of love of
reputation, or (as it is called in the case of this partial application
of it) honour, is here opposed to the worthier principle of benevolence,
and gets the better of it. But it gets the better also of the
self-regarding principle of pecuniary interest. The disposition,
therefore, which it indicates, although not so good a one as that in
which the principle of benevolence predominates, is better than one in
which the principle of self interest predominates. He would be the
better for having more benevolence: but would he be the better for
having no honour? This seems to admit of great dispute.

XXI. 7. Where the tendency of the act is \emph{good,} and the motive is
the semi-social one of \emph{religion.} In this case, the disposition
indicated by it (considered with respect to the influence of it on the
man's conduct towards others) is manifestly a beneficent and meritorious
one.\\
A baker distributes bread \emph{gratis} among the industrious poor. It
is not that he feels for their distresses: nor is it for the sake of
gaining reputation among his neighbours. It is for the sake of gaining
the favour of the Deity: to whom, he takes for granted, such conduct
will be acceptable. The disposition manifested by such conduct is
plainly what every man would call a good one.

XXII. 8. Where the tendency of the act is \emph{bad,} and the motive is
that of religion, as before. In this case the disposition is dubious. It
is good or bad, and more or less good or bad, in the first place, as the
tendency of the act is more or less mischievous; in the next place,
according as the religious tenets of the person in question approach
more or less to a coincidence with the dictates of utility.

XXIII. It should seem from history, that even in nations in a tolerable
state of civilization in other respects, the dictates of religion have
been found so far to recede from a coincidence with those of utility; in
other words, from those of enlightened benevolence; that the disposition
indicated in this case may even be a bad one upon the whole. This
however is no objection to the inference which it affords of a good
disposition in those countries (such as perhaps are most of the
countries of Europe at present) in which its dictates respecting the
conduct of a man towards other men approach very nearly to a coincidence
with those of utility. The dictates of religion, in their application to
the conduct of a man in what concerns himself alone, seem in most
European nations to savour a good deal of the ascetic principle: but the
obedience to such mistaken dictates indicates not any such disposition
as is likely to break out into acts of pernicious tendency with respect
to others. Instances in which the dictates of religion lead a man into
acts which are pernicious in this latter view, seem at present to be but
rare: unless it be acts of persecution, or impolitic measures on the
part of government, where the law itself is either the principal actor
or an accomplice in the mischief. Ravaillac, instigated by no other
motive than this, gave his country one of the most fatal stabs that a
country ever received from a single hand: but happily the Ravaillacs are
but rare. They have been more frequent, however, in France than in any
other country during the same period: and it is remarkable, that in
every instance it is this motive that has produced them. When they do
appear, however, nobody, I suppose, but such as themselves, will be for
terming a disposition, such as they manifest, a good one. It seems
hardly to be denied, but that they are just so much the worse for their
notions of religion; and that had they been left to the sole guidance of
benevolence, and the love of reputation, without any religion at all, it
would have been but so much the better for mankind. One may say nearly
the same thing, perhaps, of those persons who, without any particular
obligation, have taken an active part in the execution of laws made for
the punishment of those who have the misfortune to differ with the
magistrate in matters of religion, much more of the legislator himself,
who has put it in their power. If Louis XIV had had no religion, France
would not have lost 800,000 of its most valuable subjects. The same
thing may be said of the authors of the wars called holy ones; whether
waged against persons called Infidels or persons branded with the still
more odious name of Heretics. In Denmark, not a great many years ago, a
sect is said to have arisen, who, by a strange perversion of reason,
took it into their heads, that, by leading to repentance, murder, or any
other horrid crime, might be made the road to heaven. It should all
along, however, be observed, that instances of this latter kind were
always rare: and that in almost all the countries of Europe, instances
of the former kind, though once abundantly frequent, have for some time
ceased. In certain countries, however, persecution at home, or (what
produces a degree of restraint, which is one part of the mischiefs of
persecution) I mean the \emph{disposition} to persecute, whensoever
occasion happens, is not yet at an end: insomuch that if there is no
\emph{actual} persecution, it is only because there are no heretics; and
if there are no heretics, it is only because there are no thinkers.

XXIV. 9. Where the tendency of the act is \emph{good,} and the motive
(as before) is the dissocial one of ill-will. In this case the motive
seems not to afford any indication on either side. It is no indication
of a good disposition; but neither is it any indication of a bad one.\\
You have detected a baker in selling short weight: you prosecute him for
the cheat. It is not for the sake of gain that you engaged in the
prosecution; for there is nothing to be got by it: it is not from public
spirit: it is not for the sake of reputation; for there is no reputation
to be got by it: it is not in the view of pleasing the Deity: it is
merely on account of a quarrel you have with the man you prosecute. From
the transaction, as thus stated, there does not seem to be any thing to
be said either in favour of your disposition or against it. The tendency
of the act is good: but you would not have engaged in it, had it not
been from a motive which there seems no particular reason to conclude
will ever prompt you to engage in an act of the same kind again. Your
motive is of that sort which may, with least impropriety, be termed a
bad one: but the act is of that sort, which, were it engaged in ever so
often, could never have any evil tendency; nor indeed any other tendency
than a good one. By the supposition, the motive it happened to be
dictated by was that of ill-will: but the act itself is of such a nature
as to have wanted nothing but sufficient discernment on your part in
order to have been dictated by the most enlarged benevolence. Now, from
a man's having suffered himself to be induced to gratify his resentment
by means of an act of which the tendency is good, it by no means follows
that he would be ready on another occasion, through the influence of the
same sort of motive, to engage in any act of which the tendency is a bad
one. The motive that impelled you was a dissocial one: but what social
motive could there have been to restrain you ? None, but what might have
been outweighed by a more enlarged motive of the same kind. Now, because
the dissocial motive prevailed when it stood alone, it by no means
follows that it would prevail when it had a social one to combat it.

XXV. 10. Where the tendency of the act is \emph{bad,} and the motive is
the dissocial one of malevolence. In this case these disposition it
indicates is of course a mischievous one.\\
The man who stole the bread from the baker, as before, did it with no
other view than merely to impoverish and afflict him: accordingly, when
he had got the bread, he did not eat, or sell it; but destroyed it. That
the disposition, evidenced by such a transaction, is a bad one, is what
every body must perceive immediately.

XXVI. Thus much with respect to the circumstances from which the
mischievousness or meritoriousness of a man's disposition is to be
inferred in the gross: we come now to the \emph{measure} of that
mischievousness or meritoriousness, as resulting from those
circumstances. Now with meritorious acts and dispositions we have no
direct concern in the present work. All that penal law is concerned to
do, is to measure the depravity of the disposition where the act is
mischievous. To this object, therefore, we shall here confine ourselves.

XXVII. It is evident, that the nature of a man's disposition must depend
upon the nature of the motives he is apt to be influenced by: in other
words, upon the degree of his sensibility to the force of such and such
motives. For his disposition is, as it were, the sum of his intentions:
the disposition he is of during a certain period, the sum or result of
his intentions during that period, If, of the acts he has been intending
to engage in during the supposed period, those which are apparently of a
mischievous tendency, bear a large proportion to those which appear to
him to be of the contrary tendency, his disposition will be of the
mischievous cast: if but a small proportion, of the innocent or upright.

XXVIII. Now intentions, like every thing else, are produced by the
things that are their causes: and the causes of intentions are motives.
If, on any occasion, a man forms either a good or a bad intention, it
must be by the influence of some motive.

XXIX. When the act, which a motive prompts a man to engage in, is of a
mischievous nature, it may, for distinction's sake, be termed a
\emph{seducing} or corrupting motive: in which case also any motive
which, in opposition to the former, acts in the character of a
restraining motive, may be styled a \emph{tutelary,} preservatory, or
preserving motive.

XXX. Tutelary motives may again be distinguished into \emph{standing} or
constant, and \emph{occasional.} By standing tutelary motives, I mean
such as act with more or less force in all, or at least in most cases,
tending to restrain a man from \emph{any} mischievous acts he may be
prompted to engage in; and that with a force which depends upon the
general nature of the act, rather than upon any accidental circumstance
with which any individual act of that sort may happen to be accompanied.
By occasional tutelary motives, I mean such motives as may chance to act
in this direction or not, according to the nature of the act, and of the
particular occasion on which the engaging in it is brought into
contemplation.

XXXI. Now it has been shown, that there is no sort of motive by which a
man may not be prompted to engage in acts that are of a mischievous
nature; that is, which may not come to act in the capacity of a seducing
motive. It has been shown, on the other hand, that there are some
motives which are remarkably less likely to operate in this way than
others. It has also been shown, that the least likely of all is that of
benevolence or good-will: the most common tendency of which, it has been
shown, is to act in the character of a tutelary motive. It has also been
shown, that even when by accident it acts in one way in the character of
a seducing motive, still in another way it acts in the opposite
character of a tutelary one. The motive of good-will, in as far as it
respects the interests of one set of persons, may prompt a man to engage
in acts which are productive of mischief to another and more extensive
set: but this is only because his good-will is imperfect and confined:
not taking into contemplation the interests of all the persons whose
interests are at stake. The same motive, were the affection it issued
from more enlarged, would operate effectually, in the character of a
constraining motive, against that very act to which, by the supposition,
it gives birth. This same sort of motive may therefore, without any real
contradiction or deviation from truth, be ranked in the number of
standing tutelary motives, notwithstanding the occasions in which it may
act at the same time in the character of a seducing one.

XXXII. The same observation, nearly, may be applied to the semi-social
motive of love of reputation. The force of this, like that of the
former, is liable to be divided against itself. As in the case of
good-will, the interests of some of the persons, who may be the objects
of that sentiment, are liable to be at variance with those of others: so
in the case of love of reputation, the sentiments of some of the
persons, whose good opinion is desired, may be at variance with the
sentiments of other persons of that number. Now in the case of an act,
which is really of a mischievous nature, it can scarcely happen that
there shall be no persons whatever who will look upon it with an eye of
disapprobation. It can scarcely ever happen, therefore, that an act
really mischievous shall not have some part at least, if not the whole,
of the force of this motive to oppose it; nor, therefore, that this
motive should not act with some degree of force in the character of a
tutelary motive. This, therefore, may be set down as another article in
the catalogue of standing tutelary motives.

XXXIII. The same observation may be applied to the desire of amity,
though not in altogether equal measure. For, notwithstanding the
mischievousness of an act, it may happen, without much difficulty, that
all the persons for whose amity a man entertains any particular present
desire which is accompanied with expectation, may concur in regarding it
with an eye rather of approbation than the contrary. This is but too apt
to be the case among such fraternities as those of thieves, smugglers,
and many other denominations of offenders. This, however, is not
constantly, nor indeed most commonly the case: insomuch, that the desire
of amity may still be regarded, upon the whole, as a tutelary motive,
were it only from the closeness of its connexion with the love of
reputation. And it may be ranked among standing tutelary motives, since,
where it does apply, the force with which it acts, depends not upon the
occasional circumstances of the act which it opposes, but upon
principles as general as those upon which depend the action of the other
semi-social motives.

XXXIV. The motive of religion is not altogether in the same case with
the three former. The force of it is not, like theirs, liable to be
divided against itself. I mean in the civilized nations of modern times,
among whom the notion of the unity of the Godhead is universal. In times
of classical antiquity it was otherwise. If a man got Venus on his side,
Pallas was on the other: if Æolus was for him, Neptune was against him.
Æneas, with all his piety, had but a partial interest at the court of
heaven. That matter stands upon a different footing nowadays. In any
given person, the force of religion, whatever it be, is now all of it on
one side. It may balance, indeed, on which side it shall declare itself:
and it may declare itself, as we have seen already in but too many
instances, on the wrong as well as on the right. It has been, at least
till lately, perhaps is still, accustomed so much to declare itself on
the wrong side, and that in such material instances, that on that
account it seemed not proper to place it, in point of social tendency,
on a level altogether with the motive of benevolence. Where it does act,
however, as it does in by far the greatest number of cases, in
opposition to the ordinary seducing motives, it acts, like the motive of
benevolence, in an uniform manner, not depending upon the particular
circumstances that may attend the commission of the act; but tending to
oppose it, merely on account of its mischievousness; and therefore, with
equal force, in whatsoever circumstances it may be proposed to be
committed. This, therefore, may also be added to the catalogue of
standing tutelary motives.

XXXV. As to the motives which may operate occasionally (in the character
of tutelary motives, these, it has been already intimated, are of
various sorts, and various degrees of strength in various offenses:
depending not only upon the nature of the offence, but upon the
accidental circumstances in which the idea of engaging in it may come in
contemplation. Nor is there any sort of motive which may not come to
operate in this character; as may be easily conceived. A thief, for
instance, may be prevented from engaging in a projected scheme of
house-breaking, by sitting too long over his bottle (love of the
pleasures of the palate), by a visit from his doxy, by the occasion he
may have to go elsewhere, in order to receive his dividend of a former
booty (pecuniary interest); and so on.

XXXVI. There are some motives, however, which seem more apt to act in
this character than others; especially as things are now constituted,
now that the law has every where opposed to the force of the principal
seducing motives, artificial tutelary motives of its own creation. Of
the motives here meant it will be necessary to take a general view. They
seem to be reducible to two heads; viz.,\\
1. The love of ease; a motive put into action by the prospect of the
trouble of the attempt; that is, the trouble which it may be necessary
to bestow, in overcoming the physical difficulties that may accompany
it.\\
2. Self-preservation, as opposed to the dangers to which a man may be
exposed in the prosecution of it.

XXXVII. These dangers may be either,\\
1. Of a purely physical nature: or,\\
2. Dangers resulting from moral agency; in other words, from the conduct
of any such persons to whom the act, if known, may be expected to prove
obnoxious.\\
But moral agency supposes knowledge with respect to the circumstances
that are to have the effect of external motives in giving birth to it.
Now the obtaining such knowledge, with respect to the commission of any
obnoxious act, on the part of any persons who may be disposed to make
the agent suffer for it, is called \emph{detection;} and the agent
concerning whom such knowledge is obtained, is said to be detected. The
dangers, therefore, which may threaten an offender from this quarter,
depend, whatever they may be, on the event of his detection; any may,
therefore, be all of them comprised under the article of the
\emph{danger of detection.}

XXXVIII. The danger depending upon detection may be divided again into
two branches:\\
1. That which may result from any opposition that may be made to the
enterprise by persons on the spot; that is, at the very time the
enterprise is carrying on:\\
2. That which respects the legal punishment, or to other suffering, that
may await at a distance upon the issue of the enterprise.

XXXIX. It may be worth calling to mind on this occasion, that among the
tutelary motives, which have been styled constant ones, there are two of
which the force depends (though not so entirely as the force of the
occasional ones which have been or just mentioned, yet in a great
measure) upon the circumstance of detection. These, it may be
remembered, are, the love of reputation, and the desire of amity. In
proportion, therefore, as the chance of being detected appears greater,
these motives will apply with the greater force: with the less force, as
it appears less. This is not the case with the two other standing
tutelary motives, that of benevolence, and that of religion.

XL. We are now in a condition to determine, with some degree of
precision, what is to be understood by the \emph{strength of a
temptation} , and what indication it may give of the degree of
mischievousness in a man's disposition in the case of any offence. When
a man is prompted to engage in any mischievous act, we will say, for
shortness, in an offense, the strength of the temptation depends upon
the ratio between the force of the seducing motives on the one hand, and
such of the occasional tutelary ones, as the circumstances of the case
call forth into action, on the other. The temptation, then, may be said
to be strong, when the pleasure or advantage to be got from the crime is
such as in the eyes of the offender must appear great in comparison of
the trouble and danger that appear to him to accompany the enterprise:
slight or weak, when that pleasure or advantage is such as must appear
small in comparison of such trouble and such danger. It is plain the
strength of the temptation depends not upon the force of the impelling
(that is of the seducing) motives altogether: for let the opportunity be
more favourable, that is, let the trouble, or any branch of the danger,
be made less than before, it will be acknowledged, that the temptation
is made so much the stronger: and on the other hand, let the opportunity
become less favourable, or, in other words, let the trouble, or any
branch of the danger, be made greater than before, the temptation will
be so much the weaker.\\
Now, after taking account of such tutelary motives as have been styled
occasional, the only tutelary motives that can remain are those which
have been termed standing ones. But those which have been termed the
standing tutelary motives, are the same that we have been styling
social. It follows, therefore, that the strength of the temptation, in
any case, after deducting the force of the social motives, is as the sum
of the forces of the seducing, to the sum of the forces of the
occasional tutelary motives.

XLI. It remains to be inquired, what indication concerning the
mischievousness or depravity of a man's disposition is afforded by the
strength of the temptation, in the case where any offense happens to
have been committed. It appears, then, that the weaker the temptation
is, by which a man has been overcome, the more depraved and mischievous
it shows his disposition to have been. For the goodness of his
disposition is measured by the degree of his sensibility to the action
of the social motives: in other words, by the strength of the influence
which those motives have over him: now, the less considerable the force
is by which their influence on him has been overcome, the more
convincing is the proof that has been given of the weakness of that
influence. Again, The degree of a man's sensibility to the force of the
social motives being given, it is plain that the force with which those
motives tend to restrain him from engaging in any mischievous
enterprise, will be as the apparent mischievousness of such enterprise,
that is, as the degree of mischief with which it appears to him likely
to be attended. In other words, the less mischievous the offence appears
to him to be, the less averse he will be, as far as he is guided by
social considerations, to engage in it; the more mischievous, the more
averse. If then the nature of the offense is such as must appear to him
highly mischievous, and yet he engages in it notwithstanding, it shows,
that the degree of his sensibility to the force of the social motives is
but slight; and consequently that his disposition is proportionably
depraved. Moreover, the less the strength of the temptation was; the
more pernicious and depraved does it show his disposition to have been.
For the less the strength of the temptation was, the less was the force
which the influence of those motives had to overcome: the clearer
therefore is the proof that has been given of the weakness of that
influence.

XLII. From what has been said, it seems, that, for judging of the
indication that is afforded concerning the depravity of a man's
disposition by the strength of the temptation, compared with the
mischievousness of the enterprise, the following rules may be laid down:

Rule 1. \emph{The strength of the temptation being given, the
mischievousness} \emph{of the disposition manifested by the enterprise,
is as. the apparent mischievousness of the act.}\\
Thus, it would show a more depraved disposition, to murder a man for a
reward of a guinea, or falsely to charge him with a robbery for the same
reward, than to obtain the same sum from him by simple theft: the
trouble he would have to take, and the risk he would have to run, being
supposed to stand on the same footing in the one case as in the other.

Rule 2. \emph{The apparent mischievousness of the act being given, a
man's disposition is the more depraved, the slighter the temptation is
by which he has been overcome.}\\
Thus, it shows a more depraved and dangerous disposition, if a man kill
another out of mere sport, as the Emperor of Morocco, Muley Mahomet, is
said to have done great numbers, than out of revenge, as Sylla and
Marius did thousands, or in the view of self-preservation, as Augustus
killed many, or even for lucre, as the same Emperor is said to have
killed some. And the effects of such a depravity, on that part of the
public which is apprised of it, run in the same proportion. From
Augustus, some persons only had to fear, under some particular
circumstances. From Muley Mahomet, every man had to fear at all times.

Rule 3. \emph{The apparent mischievousness of the act being given, the
evidence which it affords of the depravity of a man's disposition is the
less conclusive, the stronger the temptation is by which he has been
overcome.}\\
Thus, if a poor man, who is ready to die with hunger, steal a loaf of
bread, it is a less explicit sign of depravity, than if a rich man were
to commit a theft to the same amount. It will be observed, that in this
rule all that is said is, that the evidence of depravity is in this case
the less conclusive: it is not said that the depravity is positively the
less. For in this case it is possible, for any thing that appears to the
contrary, that the theft might have been committed, even had the
temptation been not so strong. In this case, the alleviating
circumstance is only a matter of presumption; in the former, the
aggravating circumstance is a matter of certainty.

Rule 4. \emph{Where the motive is of the dissocial kind, the apparent
mischievousness of the act, and the strength of the temptation, being
given, the depravity is as the degree of deliberation with which it is
accompanied.}\\
For in every man, be his disposition ever so depraved, the social
motives are those which, wherever the self-regarding ones stand neuter,
regulate and determine the general tenor of his life. If the dissocial
motives are put in action, it is only in particular circumstances, and
on particular occasions; the gentle but constant force of the social
motives being for a while subdued. The general and standing bias of
every man's nature is, therefore, towards that side to which the force
of the social motives would determine him to adhere. This being the
case, the force of the social motives tends continually to put an end to
that of the dissocial ones; as, in natural bodies, the force of friction
tends to put an end to that which is generated by impulse. Time, then,
which wears away the force of the dissocial motives, adds to that of the
social. The longer, therefore, a man continues, on a given occasion,
under the dominion of the dissocial motives, the more convincing is the
proof that has been given of his insensibility to the force of the
social ones.\\
Thus, it shows a worse disposition, where a man lays a deliberate plan
for beating his antagonist, and beats him accordingly, than if he were
to beat him upon the spot, in consequence of a sudden quarrel: and worse
again, if, after having had him a long while together in his power, he
beats him at intervals, and at his leisure.

XLIII. The depravity of disposition, indicated by an act, is a material
consideration in several respects. Any mark of extraordinary depravity,
by adding to the terror already inspired by the crime, and by holding up
the offender as a person from whom there may be more mischief to be
apprehended in future, adds in that way to the demand for punishment. By
indicating a general want of sensibility on the part of the offender, it
may add in another way also to the demand for punishment. The article of
disposition is of the more importance, inasmuch as, in measuring out the
quantum of punishment, the principle of sympathy and antipathy is apt to
look at nothing else. A man who punishes because he hates, and only
because he hates, such a man, when he does not find any thing odious in
the disposition, is not for punishing at all; and when he does, he is
not for carrying the punishment further than his hatred carries him.
Hence the aversion we find so frequently expressed against the maxim,
that the punishment must rise with the strength of the temptation; a
maxim, the contrary of which, as we shall see, would be as cruel to
offenders themselves, as it would be subversive of the purposes of
punishment.

\chapter{Of the Consequences of a Mischievous Act}

§1. Shapes in which the mischief of an act may show itself

I. Hitherto we have been speaking of the various articles or objects on
which the consequences or tendency of an act may depend: of the bare
\emph{act} itself: of the \emph{circumstances} it may have been, or may
have been supposed to be, accompanied with: of the \emph{consciousness}
a man may have had with respect to any such circumstances: of the
\emph{intentions} that may have preceded the act: of the \emph{motives}
that may have given birth to those intentions: and of the
\emph{disposition} that may have been indicated by the connexion between
such intentions and such motives. We now come to speak of
\emph{consequences} or tendency: an article which forms the concluding
link in all this chain of causes and effects, involving in it the
materiality of the whole. Now, such part of this tendency as is of a
mischievous nature, is all that we have any direct concern with; to
that, therefore, we shall here confine ourselves.

II. The tendency of an act is mischievous when the consequences of it
are mischievous; that is to say, either the certain consequences or the
probable. The consequences, how many and whatsoever they may be, of an
act, of which the tendency is mischievous, may, such of them as are
mischievous, be conceived to constitute one aggregate body, which may be
termed the mischief of the act.

III. This mischief may frequently be distinguished, as it were, into two
shares or parcels: the one containing what may be called the primary
mischief; the other, what may be called the secondary. That share may be
termed the \emph{primary,} which it sustained by an assignable
individual, or a multitude of assignable individuals. That share may be
termed the \emph{secondary,} which, taking its origin from the former,
extends itself either over the whole community, or over some other
multitude of unassignable individuals.

IV. The primary mischief of an act may again be distinguished into two
branches: 1. The \emph{original:} and, 2. The \emph{derivative.} By the
original branch, I mean that which alights upon and is confined to any
person who is a sufferer in the first instance, and on his own account:
the person, for instance, who is beaten, robbed, or murdered. By the
derivative branch, I mean any share of mischief which may befall any
other assignable persons in consequence of his being a sufferer, and no
otherwise. These persons must, of course, be persons who in some way or
other are connected with him. Now the ways in which one person may be
connected with another, have been already seen: they may be connected in
the way of \emph{interest} (meaning self-regarding interest) or merely
in the way of \emph{sympathy.} And again, persons connected with a given
person, in the way of interest, may be connected with him either by
affording \emph{support} to him, or by deriving it from him.

V. The secondary mischief, again, may frequently be seen to consist of
two other shares or parcels: the first consisting of \emph{pain;} the
other of \emph{danger.} The pain which it produces is a pain of
apprehension: a pain grounded on the apprehension of suffering such
mischiefs or inconveniences, whatever they may be, as it is the nature
of the primary mischief to produce. It may be styled, in one word, the
\emph{alarm.} The danger is the \emph{chance,} whatever it may be, which
the multitude it concerns may in consequence of the primary mischief
stand exposed to, of suffering such mischiefs or inconveniences. For
danger is nothing but the chance of pain, or, what comes to the same
thing, of loss of pleasure.

VI. An example may serve to make this clear. A man attacks you on the
road, and robs you. You suffer a pain on the occasion of losing so much
money: you also suffered a pain at the thoughts of the personal
ill-treatment you apprehended he might give you, in case of your not
happening to satisfy his demands. These together constitute the original
branch of the primary mischief, resulting from the act of robbery. A
creditor of yours, who expected you to pay him with part of that money,
and a son of yours, who expected you to have given him another part, are
in consequence disappointed. You are obliged to have recourse to bounty
of your father, to make good part of the deficiency. These mischiefs
together make up the derivative branch. The report of this robbery
circulates from hand to hand, and spreads itself in the neighbourhood.
It finds its way into the newspapers, and is propagated over the whole
country. Various people, on this occasion, call to mind the danger which
they and their friends, as it appears from this example, stand exposed
to in travelling; especially such as may have occasion to travel the
same road. On this occasion they naturally feel a certain degree of
pain: slighter or heavier, according to the degree of ill-treatment they
may understand you to have received; the frequency of the occasion each
person may have to travel in that same road, or its neighbourhood; the
vicinity of each person to the spot; his personal courage; the quantity
of money he may have occasion to carry about with him; and a variety of
other circumstances. This constitutes the first part of the secondary
mischief, resulting from the act of robbery; viz., the alarm. But people
of one description or other, not only are disposed to conceive
themselves to incur a chance of being robbed, in consequence of the
robbery committed upon you, but (as will be shown presently) they do
really incur such a chance. And it is this chance which constitutes the
remaining part of the secondary mischief of the act of robbery; viz.,
the danger.

VII. Let us see what this chance amounts to; and whence it comes. How is
it, for instance, that one robbery can contribute to produce another? In
the first place, it is certain that: it cannot create any direct motive.
A motive must be the prospect of some pleasure, or other advantage, to
be enjoyed in future: but the robbery in question is past: nor would it
furnish any such prospect were it to come: for it is not one robbery
that will furnish pleasure to him who may be about to commit another
robbery. The consideration that is to operate upon a man, as a motive or
inducement to commit a robbery, must be the idea of the pleasure he
expects to derive from the fruits of that very robbery: but this
pleasure exists independently of any other robbery.

VIII. The means, then, by which one robbery tends, as it should seem, to
produce another robbery, are two.\\
1. By suggesting to a person exposed to the temptation, the idea of
committing such another robbery (accompanied, perhaps, with the belief
of its facility). In this case the influence it exerts applies itself,
in the first place, to the understanding.\\
2. By weakening the force of the tutelary motives which tend to restrain
him from such an action, and thereby adding to the strength of the
temptation. In this case the influence applies itself to the will. These
forces are,\\
1. The motive of benevolence, which acts as a branch of the physical
sanction\\
2. The motive of self-preservation, as against the punishment that may
stand provided by the political sanction.\\
3. The fear of shame; a motive belonging to the moral sanction.\\
4. The fear of the divine displeasure; a motive belonging to the
religious sanction. On the first and last of these forces it has,
perhaps, no influence worth insisting on: but it has on the other two.

IX. The way in which a past robbery may weaken the force with which the
\emph{political} sanction tends to prevent a future robbery, may be thus
conceived. The way in which this sanction tends to prevent a robbery, is
by denouncing some particular kind of punishment against any who shall
be guilty of it: the \emph{real} value of which punishment will of
course be diminished by the \emph{real} uncertainty: as also, if there
be any difference, the \emph{apparent} value by the \emph{apparent}
uncertainty. Now this uncertainty is proportionably increased by every
instance in which a man is known to commit the offense, without
undergoing the punishment. This, of course, will be the case with every
offense for a certain time; in short, until the punishment allotted to
it takes place. If punishment takes place at last, this branch of the
mischief of the offense is then at last, but not till then, put a stop
to.

X. The way in which a past robbery may weaken the force with which the
\emph{moral} sanction tends to prevent a future robbery, may be thus
conceived. The way in which the moral sanction tends to prevent a
robbery, is by holding forth the indignation of mankind as ready to fall
upon him who shall be guilty of it. Now this indignation will be the
more formidable, according to the number of those who join in it: it
will be the less so, the fewer they are who join in it. But there cannot
be a stronger way of showing that a man does not join in whatever
indignation may be entertained against a practice, than the engaging in
it himself. It shows not only that he himself feels no indignation
against it, but that it seems to him there is no sufficient reason for
apprehending what indignation may be felt against it by others.
Accordingly, where robberies are frequent, and unpunished, robberies are
committed without shame. It was thus amongst the Grecians formerly. It
is thus among the Arabs still.

XI. In whichever way then a past offense tends to pave the way for the
commission of a future Hence, whether by suggesting the idea of
committing it, or by adding to the strength of the temptation, in both
cases it may be said to operate by the force or \emph{influence of
example.}

XII. The two branches of the secondary mischief of an act, the alarm and
the danger, must not be confounded: though intimately connected, they
are perfectly distinct: either may subsist without the other. The
neighbourhood may be alarmed with the report of a robbery, when, in
fact, no robbery either has been committed or is in a way to be
committed: a neighbourhood may be on the point of being disturbed by
robberies, without knowing any thing of the matter. Accordingly, we
shall soon perceive, that some acts produce alarm without danger:
others, danger without alarm.

XIII. As well the danger as the alarm may again be divided, each of
them, into two branches: the first, consisting of so much of the alarm
or danger as may be apt to result from the future behaviour of the same
agent: the second, consisting of so much as may be apt to result from
the behaviour of other persons: such others, to wit, as may come to
engage in acts of the same sort and tendency.

XIV. The distinction between the primary and the secondary consequences
of an act must be carefully attended to. It is so just, that the latter
may often be of a directly opposite nature to be the former. In some
cases, where the primary consequences of the act are attended with a
mischief, the secondary consequences be may be beneficial, and that to
such a degree, as even greatly to outweigh the mischief of the primary.
This is the case, for instance, with all acts of punishment, when
properly applied. Of these, the primary mischief being never intended to
fall but upon such persons as may happen to have committed some act
which it is expedient to prevent, the secondary mischief, that is, the
alarm and the danger, extends no farther than to such persons as are
under temptation to commit it: in which case, in as far as it tends to
restrain them from committing such acts, it is of a beneficial nature.

XV. Thus much with regard to acts that produce positive pain, and that
immediately. This case, by reason of its simplicity, seemed the fittest
to take the lead. But acts may produce mischief in various other ways;
which, together with those already specified, may all be comprised by
the following abridged analysis.\\
Mischief may admit of a division in any one of three points of view.\\
1. According to its own \emph{nature.}\\
2. According to its \emph{cause.}\\
3. According to the person, or other party, who is the \emph{object} of
it\\
1. With regard to its nature, it may be either \emph{simple} or
\emph{complex}\\
2: when simple, it may either be \emph{positive} or \emph{negative:}
positive, consisting of actual pain: negative, consisting of the loss of
pleasure.

Whether simple or complex, and whether positive or negative, it may be
either \emph{certain} or \emph{contingent.} When it is negative, it
consists of the loss of some benefit or advantage: this benefit may be
material in both or either of two ways:\\
1. By affording actual pleasure: or,\\
2. By averting pain or \emph{danger,} which is the chance of pain: that
is, by affording \emph{security.} In as far, then, as the benefit which
a mischief tends to avert, is productive of security, the tendency of
such mischief is to produce \emph{insecurity.}\\
2. With regard to its \emph{cause,} mischief may be produced either by
one \emph{single} action, or not without the \emph{concurrence} of other
actions: if not without the concurrence of other actions, these others
may be the actions either of the \emph{same person,} or of \emph{other}
persons: in either case, they may be either acts of the \emph{same kind}
as that in question, or of \emph{other} kinds.\\
3. Lastly, with regard to the party who is the \emph{object} of the
mischief, or, in other words, who is in a way to be affected by it, such
party maybe either an \emph{assignable} individual, or assemblage of
individuals, or else a multitude of \emph{unassignable} individuals.
When the object is an assignable individual, this individual may either
be the person \emph{himself} who is the author of the mischief, or some
\emph{other} person. When the individuals who are the objects of it, are
an unassignable multitude, this multitude may be either the \emph{whole}
political community or state, or some \emph{subordinate} division of it.
Now when the object of the mischief is the author himself, it may be
styled \emph{self-regarding:} when any other party is the object,
\emph{extra-regarding:} when such other party is an individual, it may
be styled \emph{private:} when a subordinate branch of the community,
\emph{semi-public:} when the whole community, \emph{public.} Here, for
the present, we must stop. To pursue the subject through its inferior
distinctions, will be the business of the chapter which exhibits the
division of offenses.

The cases which have been already illustrated, are those in which the
primary mischief is not necessarily otherwise than a simple one, and
that positive: present, and therefore certain: producible by a single
action, without any necessity of the concurrence of any other action,
either on the part of the same agent, or of others; and having for its
object an assignable individual, or, by accident an assemblage of
assignable individuals: extra-regarding therefore, and private. This
primary mischief is accompanied by a secondary: the first branch of
which is sometimes contingent and sometimes certain, the other never
otherwise than contingent: both extra-regarding and semi-public: in
other respects, pretty much upon a par with the primary mischief: except
that the first branch, viz., the alarm, though inferior in magnitude to
the primary, is, in point of extent, and therefore, upon the whole, in
point of magnitude, much superior.

XVI. Two instances more will be sufficient to illustrate the most
material of the modifications above exhibited.\\
A man drinks a certain quantity of liquor, and intoxicates himself. The
intoxication in this particular instance does him no sort of harm: or,
what comes to the same thing, none that is perceptible. But it is
probable, and indeed next to certain, that a given number of acts of the
same kind would do him a very considerable degree of harm: more or less
according to his constitution and other circumstances: for this is no
more than what experience manifests every day. It is also certain, that
one act of this sort, by one means or other, tends considerably to
increase the disposition a man may be in to practise other acts of the
same sort: for this also is verified by experience. This, therefore, is
one instance where the mischief producible by the act is contingent in
other words, in which the tendency of the act is no otherwise
mischievous than in virtue of its producing a \emph{chance} of mischief.
This chance depends upon the concurrence of other acts of the same kind;
and those such as must be practiced by the same person. The object of
the mischief is that very person himself who is the author of it, and he
only, unless by accident. The mischief is therefore private and
self-regarding. As to its secondary mischief, alarm, it produces none:
it produces indeed a certain quantity of danger by the influence of
example: but it is not often that this danger will amount to a quantity
worth regarding.

XVII. Again. A man omits paying his share to a public tax. This we see
is an act of the negative kind. Is this then to be placed upon the list
of mischievous acts? Yes, certainly. Upon what grounds? Upon the
following. To defend the community against its external as well as its
internal adversaries are tasks, not to mention others of a less
indispensable nature which cannot be fulfilled but at a considerable
expense. But whence is the money for defraying this expense to come? It
can be obtained in no other manner than by contributions to be collected
from individuals; in a word, by taxes. The produce then of these taxes
is to be looked upon as a kind of \emph{benefit} which it is necessary
the governing part of the community should receive for the use of the
whole. This produce, before it can be applied to its destination,
requires that there should be certain persons commissioned to receive
and to apply it. Now if these persons, had they received it, would have
applied it to its proper destination, it would have been a benefit: the
not putting them in a way to receive it, is then a mischief. But it is
possible, that if received, it might not have been applied to its proper
destination; or that the services, in consideration of which it was
bestowed, might not have been performed. It is possible, that the
under-officer, who collected the produce of the tax, might not have paid
it over to his principal: it is possible that the principal might not
have forwarded it on according to its farther destination; to the judge,
for instance, who is to protect the community against its clandestine
enemies from within, or the soldier, who is to protect it against its
open enemies from without: it is possible that the judge, or the
soldier, had they received it, would not however have been induced by it
to fulfil their respective duties: it is possible, that the judge would
not have sat for the punishment of criminals, and the decision of
controversies: it is possible that the soldier would not have drawn his
sword in the defense of the community. These, together with an infinity
of other intermediate acts, which for the sake of brevity I pass over,
form a connected chain of duties, the discharge of which is necessary to
the preservation of the community. They must every one of them be
discharged, ere the benefit to which they are contributory can be
produced. If they are all discharged, in that case the benefit subsists,
and any act, by tending to intercept that benefit, may produce a
mischief. But if any of them are not, the benefit fails: it fails of
itself: it would not have subsisted, although the act in question (the
act of non-payment) had not been committed. The benefit is therefore
contingent; and, accordingly, upon a certain supposition, the act which
consists in the averting of it is not a mischievous one. But this
supposition, in any tolerably-ordered government, will rarely indeed be
verified. In the very worst ordered government that exists, the greatest
part of the duties that are levied are paid over according to their
destination: and, with regard to any particular sum, that is attempted
to be levied upon any particular person upon any particular occasion, it
is therefore manifest, that, unless it be certain that it will not be so
disposed of, the act of withholding it is a mischievous one.\\
The act of payment, when referable to any particular sum, especially if
it be a small one, might also have failed of proving beneficial on
another ground: and, consequently, the act of nonpayment, of proving
mischievous. It is possible that the same services, precisely, might
have been rendered without the money as with it. If, then, speaking of
any small limited sum, such as the greatest which any one person is
called upon to pay at a time, a man were to say, that the non-payment of
it would be attended with mischievous consequences; this would be far
from certain: but what comes to the same thing as if it were, it is
perfectly certain when applied to the whole. It is certain, that if all
of a sudden the payment of all taxes was to cease, there would no longer
be anything effectual done, either for the maintenance of justice, or
for the defence of the community against its foreign adversaries: that
therefore the weak would presently be oppressed and injured in all
manner of ways, by the strong at home, and both together overwhelmed by
oppressors abroad. Upon the whole, therefore, it is manifest, that in
this case, though the mischief is remote and contingent, though in its
first appearance it consists of nothing more than the interception of a
\emph{benefit,} and though the individuals, in whose favour that benefit
would have been reduced into the explicit form of pleasure or security,
are altogether unassignable, yet the mischievous tendency of the act is
not on all these accounts the less indisputable. The mischief, in point
of \emph{intensity} and \emph{duration,} is indeed unknown: it is
\emph{uncertain:} it is \emph{remote.} But in point of \emph{extent} it
is immense; and in point of \emph{fecundity,} pregnant to a degree that
baffles calculation.

XVIII. It may now be time to observe, that it is only in the case where
the mischief is extra-regarding, and has an assignable person or persons
for its object, that so much of the secondary branch of it as consists
in \emph{alarm} can have place. When the individuals it affects are
uncertain, and altogether out of sight, no alarm can be produced: as
there is nobody whose sufferings you can see, there is nobody whose
sufferings you can be alarmed at. No alarm, for instance, is produced by
nonpayment to a tax. If at any distant and uncertain period of time such
offence should chance to be productive of any kind of alarm, it would
appear to proceed, as indeed immediately it would proceed, from a very
different cause. It might be immediately referable, for example, to the
act of a legislator, who should deem it necessary to lay on a new tax,
in order to make up for the deficiency occasioned in the produce of the
old one. Or it might be referable to the act of an enemy, who, under
favour of a deficiency thus created in the fund allotted for defense,
might invade the country, and exact from it much heavier contributions
than those which had been thus withholden from the sovereign.\\
As to any alarm which such an offence might raise among the few who
might chance to regard the matter with the eyes of statesmen, it is of
too slight and uncertain a nature to be worth taking into the account.

§2. How intentionality, etc;. may influence the mischief of an act.

XIX. We have seen the nature of the secondary mischief, which is apt to
be reflected, as it were, from the primary, in the cases where the
individuals who are the objects of the mischief are assignable. It is
now time to examine into the circumstances upon which the production of
such secondary mischief depends. These circumstances are no others than
the four articles which have formed the subjects of the four last
preceding chapters: viz.,\\
1. The intentionality,\\
2. The consciousness.\\
3. The motive.\\
4. The disposition.\\
It is to be observed all along, that it is only the \emph{danger} that
is immediately governed by the \emph{real} state of the mind in respect
to those articles: it is by the \emph{apparent} state of it that the
\emph{alarm} is governed. It is governed by the real only in as far as
the apparent happens, as in most cases it may be expected to do, to
quadrate with the real. The different influences of the articles of
intentionality and consciousness may be represented in the several cases
following.

XX. Case 1. Where the act is so completely unintentional, as to be
altogether \emph{involuntary.} In this case it is attended with no
secondary mischief at all.\\
A bricklayer is at work upon a house: a passenger is walking in the
street below. A fellow-workman comes and gives the bricklayer a violent
push, in consequence of which he falls upon the passenger, and hurts
him. It is plain there is nothing in this event that can give other
people, who may happen to be in the street, the least reason to
apprehend any thing in future on the part of the man who fell, whatever
there may be with regard to the man who pushed him.

XXI. Case 2. Where the act, though not unintentional, is
\emph{unadvised,} insomuch that the mischievous part of the consequences
is unintentional, but the unadvisedness is attended with
\emph{heedlessness.} In this case the act is attended with some small
degree of secondary mischief, in proportion to the degree of
heedlessness.\\
A groom being on horseback, and riding through a frequented street,
turns a corner at a full pace, and rides over a passenger, who happens
to be going by. It is plain, by this behaviour of the groom, some degree
of alarm may be produced, less or greater, according to the degree of
heedlessness betrayed by him: according to the quickness of his pace,
the fullness of the street, and so forth. He has done mischief, it may
be said, by his carelessness, already: who knows but that on other
occasions the like cause may produce the like effect.

XXII. Case 3. Where the act is \emph{misadvised} with respect to a
circumstance, which, had it existed, would \emph{fully} have excluded or
(what comes to the same thing) outweighed the primary mischief: and
there is no rashness in the case. In this case the act attended with no
secondary mischief at all.\\
It is needless to multiply examples any farther.

XXIII. Case 4. Where the act is misadvised with respect to a
circumstance which would have excluded or counterbalanced the primary
mischief in \emph{part}, but not entirely: and still there is no
rashness. In this case the set is attended with some degree of secondary
mischief, in proportion to that part of the primary which remains
unexcluded or uncounterbalanced.

XXIV. Case 5. Where the act is misadvised with respect to a
circumstance, which, had it existed, would have excluded or
counterbalanced the primary mischief entirely, or in part: and there is
a degree of \emph{rashness} in the supposal. In this case, the act is
also attended with a farther degree of secondary mischief, in proportion
to the degree of rashness.

XXV. Case 6. Where the consequences are \emph{completely} intentional,
and there is no mis-supposal in the case. In this case the secondary
mischief is at the highest.

XXVI. Thus much with regard to intentionality and consciousness. We now
come to consider in what manner the secondary mischief is affected by
the nature of the \emph{motive.}\\
Where an act is pernicious in its primary consequences, the secondary
mischief is not obliterated by the \emph{goodness} of the motive; though
the motive be of the best kind. For, notwithstanding the goodness of the
motive, an act of which the primary consequences are pernicious, is
produced by it in the instance in question, by the supposition. It may,
therefore, in other instances: although this is not so likely to happen
from a good motive as from a bad one.

XXVII. An act, which, though pernicious in its primary consequences, is
rendered in other respects beneficial upon the whole, by virtue of its
secondary consequences, is not changed back again, and rendered
pernicious upon the whole by the \emph{badness} of the motive: although
the motive be of the worst kind.

XXVIII. But when not only the primary consequences of an act are
pernicious, but, in other respects, the secondary likewise, the
secondary mischief may be \emph{aggravated} by the nature of the motive:
so much of that mischief, to wit, as respects the future behaviour of
the same person.

XXIX. It is not from the worst kind of motive, however, that the
secondary mischief of an act receives its greatest aggravation.

XXX. The aggravation which the secondary mischief of an act, in as far
as it respects the future behaviour of the same person, receives from
the nature of a motive in an individual case, is as the tendency of the
motive to produce, on the part of the same person, acts of the like bad
tendency with that of the act in question.

XXXI. The tendency of a motive to produce acts of the like kind, on the
part of any given person, is as the \emph{strength} and \emph{constancy}
of its influence on that person, as applied to the production of such
effects. P\textgreater{}XXXII. The tendency of a species of motive to
give birth to acts of any kind, among persons in general, is as the
\emph{strength, constancy,} and \emph{extensiveness} of its influence,
as applied to the production of such effects.

XXXIII. Now the motives, whereof the influence is at once most powerful,
most constant, and most extensive, are the motives of physical desire,
the love of wealth, the love of ease, the love of life, and the fear of
pain: all of them self-regarding motives. The motive of displeasure,
whatever it may be in point of strength and extensiveness, is not near
so constant in its influence (the case of mere antipathy excepted) as
any of the other three. A pernicious act, therefore, when committed
through vengeance. or otherwise through displeasure, is not near so
mischievous as the same pernicious act, when committed by force of any
one of those other motives.

XXXIV. As to the motive of religion, whatever it may sometimes prove to
be in point of strength and constancy, it is not in point of extent so
universal, especially in its application to acts of a mischievous
nature, as any of the three preceding motives. It may, however, be as
universal in a particular state, or in a particular district of a
particular state. It is liable indeed to be very irregular in its
operations. It is apt, however, to be frequently as powerful as the
motive of vengeance, or indeed any other motive whatsoever. It will
sometimes even be more powerful than any other motive. It is, at any
rate, much more constant. A pernicious act, therefore, when committed
through the motive of religion, is more mischievous than when committed
through the motive of ill-will.

XXXV. Lastly, The secondary mischief, to wit, so much of it as hath
respect to the future behaviour of the same person, is aggravated or
lessened by the apparent depravity or beneficence of his disposition:
and that in the proportion of such apparent depravity or beneficence.

XXXVI. The consequences we have hitherto been speaking of, are the
\emph{natural} consequences, of which the act, and the other articles we
have been considering, are the causes: consequences that result from the
behaviour of the individual, who is the offending agent, without the
interference of political authority. We now come to speak of
\emph{punishment:} which, in the sense in which it is here considered,
is an \emph{artificial} consequence, annexed by political authority to
an offensive act, in one instance, in the view of putting a stop to the
production of events similar to the obnoxious part of its natural
consequences, in other instances.

\chapter{Cases Unmeet for Punishment}

§ 1. General view of cases unmeet for punishment.\\
I. The general object which all laws have, or ought to have, in common,
is to augment the total happiness of the community; and therefore, in
the first place, to exclude, as far as may be, every thing that tends to
subtract from that happiness: in other words, to exclude mischief.

II. But all punishment is mischief: all punishment in itself is evil.
Upon the principle of utility, if it ought at all to be admitted, it
ought only to be admitted in as far as it promises to exclude some
greater evil.

III. It is plain, therefore, that in the following cases punishment
ought not to be inflicted.\\
7. Where it is \emph{groundless:} where there is no mischief for it to
prevent; the act not being mischievous upon the whole.\\
8. Where it must be \emph{inefficacious:} where it cannot act so as to
prevent the mischief.\\
9. Where it is \emph{unprofitable,} or too \emph{expensive:} where the
mischief it would produce would be greater than what it prevented.\\
10. Where it is \emph{needless:} where the mischief may be prevented, or
cease of itself, without it: that is, at a cheaper rate.\\

§ 2. Cases in which punishment is groundless.\\
These are,\\
IV. 1. Where there has never been any mischief: where no mischief has
been produced to any body by the act in question. Of this number are
those in which the act was such as might, on a some occasions, be
mischievous or disagreeable, but the person whose interest it concerns
gave his \emph{consent} to the performance of it. This consent, provided
it be free, and fairly obtained, is the best proof that can be produced,
that, to the person who gives it, no mischief, at least no immediate
mischief, upon the whole, is done. For no man can be so good a judge as
the man himself, what it is gives him pleasure or displeasure.

V. 2. Where the mischief was \emph{outweighed:} although a mischief was
produced by that act, yet the same act was necessary to the production
of a benefit which was of greater value than the mischief. This may be
the case with any thing that is done in the way of precaution against
instant calamity, as also with any thing that is done in the exercise of
the several sorts of powers necessary to be established in every
community, to wit, domestic, judicial, military, and supreme.

VI. 3. Where there is a certainty of an adequate compensation: and that
in all cases where the offense can be committed. This supposes two
things:\\
1. That the offence is such as admits of an adequate compensation:\\
2. That such a compensation is sure to be forthcoming.\\
Of these suppositions, the latter will be found to be a merely ideal
one: a supposition that cannot, in the universality here given to it, be
verified by fact. It cannot, therefore, in practice, be numbered amongst
the grounds of absolute impunity. It may, however, be admitted as a
ground for an abatement of that punishment, which other considerations,
standing by themselves, would seem to dictate.\\

§ 3. Cases in which punishment must be inefficacious\\
These are,\\
VII. 1. Where the penal provision is \emph{not established} until after
the act is done.\\
Such are the cases, 1. Of an \emph{ex-post-facto} law; where the
legislator himself appoints not a punishment till after the act is
done.\\
2. Of a sentence beyond the law; where the judge, of his own authority,
appoints a punishment which the legislator had not appointed.

VIII. 2. Where the penal provision, though established, is \emph{not
conveyed} to the notice of the person on whom it seems intended that it
should operate. Such is the case where the law has omitted to employ any
of the expedients which are necessary, to make sure that every person
whatsoever, who is within the reach of the law, be apprised of all the
cases whatsoever, in which (being in the station of life he is in) he
can be subjected to the penalties of the law.

IX. 3. Where the penal provision, though it were conveyed to a man's
notice, \emph{could produce no effect} on him, with respect to the
preventing him from engaging in any act of the \emph{sort} in question.
Such is the case,\\
1. In extreme \emph{infancy;} where a man has not yet attained that
state or disposition of mind in which the prospect of evils so distant
as those which are held forth by the law, has the effect of influencing
his conduct.\\
2. In \emph{insanity;} where the person, if he has attained to that
disposition, has since been deprived of it through the influence of some
permanent though unseen cause.\\
3. In \emph{intoxication;} where he has been a deprived of it by the
transient influence of a visible cause: such as the use of wine, or
opium, or other drugs, that act in this manner on the nervous system:
which condition is indeed neither more nor less than a temporary
insanity produced by an assignable cause.

X. 4. Where the penal provision (although, being conveyed to the party's
notice, it might very well prevent his engaging in acts of the sort in
question, provided he knew that it related to those acts) could not have
this effect, with regard to the \emph{individual} act he is about to
engage in: to wit, because he knows not that it is of the number of
those to which the penal provision relates. This may happen,\\
1. In the case of \emph{unintentionality;} where he intends not to
engage, and thereby knows not that he is about to engage, in the
\emph{act} in which eventually he is about to engage.\\
2. In the case of \emph{unconsciousness;} where, although he may know
that he is about to engage in the \emph{act} itself, yet, from not
knowing all the material \emph{circumstances} attending it, he knows not
of the \emph{tendency} it has to produce that mischief, in contemplation
of which it has been made penal in most instances\\
3. In the case of \emph{missupposal;} where, although he may know of the
tendency the act has to produce that degree of mischief, he supposes it,
though mistakenly, to be attended with some circumstance, or set of
circumstances, which, if it had been attended with, it would either not
have been productive of that mischief, or have been productive of such a
greater degree of good, as has determined the legislator in such a case
not to make it penal.

XI. 5. Where, though the penal clause might exercise a full and
prevailing influence, were it to act alone, yet by the
\emph{predominant} influence of some opposite cause upon the will, it
must necessarily be ineffectual; because the evil which he sets himself
about to undergo, in the case of his not engaging in the act, is so
great, that the evil denounced by the penal clause, in case of his
engaging in it, cannot appear greater. This may happen,\\
1. In the case of \emph{physical danger;} where the evil is such as
appears likely to be brought about by the unassisted powers of
\emph{nature.}\\
2. In the case of a threatened mischief; where it is such as appears
likely to be brought about through the intentional and conscious agency
of \emph{man.}

XII. 6. Where (though the penal clause may exert a full and prevailing
influence over the \emph{will} of the party) yet his \emph{physical
faculties} (owing to the predominant influence of some physical cause)
are not in a condition to follow the determination of the will: insomuch
that the act is absolutely \emph{involuntary.} Such is the case of
physical \emph{compulsion} or \emph{restraint,} by whatever means
brought about; where the man's hand, for instance, is pushed against
some object which his will disposes him \emph{not} to touch; or tied
down from touching some object which his will disposes him to touch.

§ 4. Cases where punishment is unprofitable.\\
These are,\\
XIII. 1. Where, on the one hand, the nature of the offense, on the other
hand, that of the punishment, are, in the \emph{ordinary state of
things,} such, that when compared together, the evil of the latter will
turn out to be greater than that of the former.

XIV. Now the evil of the punishment divides itself into four branches,
by which so many different sets of persons are affected.\\
1. The evil of \emph{coercion} or \emph{restraint:} or the pain which it
gives a man not to be able to do the act, whatever it be, which by the
apprehension of the punishment he is deterred from doing. This is felt
by those by whom the law is \emph{observed.}\\
2. The evil of \emph{apprehension:} or the pain which a man, who has
exposed himself to punishment, feels at the thoughts of undergoing it.
This is felt by those by whom the law has been \emph{broken,} and who
feel themselves in \emph{danger} of its being executed upon them.\\
3. The evil of \emph{sufferance:} or the pain which a man feels, in
virtue of the punishment itself, from the time when he begins to undergo
it. This is felt by those by whom the law is broken, and upon whom it
comes actually to be executed.\\
4. The pain of sympathy, and the other \emph{derivative} evils resulting
to the persons who are in \emph{connection} with the several classes of
original sufferers just mentioned. Now of these four lots of evil, the
first will be greater or less, according to the nature of the act from
which the party is restrained: the second and third according to the
nature of the punishment which stands annexed to that offence.

XV. On the other hand, as to the evil of the offense, this will also, of
course, be greater or less, according to the nature of each offense. The
proportion between the one evil and the other will therefore be
different in the case of each particular offence. The cases, therefore,
where punishment is unprofitable on this ground, can by no other means
be discovered, than by an examination of each particular offense; which
is what will be the business of the body of the work.

XVI. 2. Where, although in the \emph{ordinary state} of things, the evil
resulting from the punishment is not greater than the benefit which is
likely to result from the force with which it operates, during the same
space of time, towards the excluding the evil of the offenses, yet it
may have been rendered so by the influence of some \emph{occasional
circumstances.} In the number of these circumstances may be,\\
1. The multitude of delinquents at a particular juncture; being such as
would increase, beyond the ordinary measure, the \emph{quantum} of the
second and third lots, and thereby also of a part of the fourth lot, in
the evil of the punishment.\\
2. The extraordinary value of the services of some one delinquent; in
the case where the effect of the punishment would be to deprive the
community of the benefit of those services.\\
3. The displeasure of the \emph{people;} that is, of an indefinite
number of the members of the \emph{same} community, in cases where
(owing to of the influence of some occasional incident) they happen to
conceive, that the offense or the offender ought not to be punished at
all, or at least ought not to be punished in the way in question.\\
4. The displeasure of \emph{foreign powers;} that is, of the governing
body, or a considerable number of the members of some \emph{foreign}
community or communities, with which the community in question is
connected.\\

§ 5. Cases where punishment is needless. These are,\\
XVII. 1. Where the purpose of putting an end to the practice may be
attained as effectually at a cheaper rate: by instruction, is for
instance, as well as by terror: by informing the understanding, as well
as by exercising an immediate influence on the will. This seems to be
the case with respect to all those offenses which consist in the
disseminating pernicious principles in matters of \emph{duty;} of
whatever kind the duty be; whether political, or moral, or religious.
And this, whether such principles be disseminated \emph{under,} or even
\emph{without;} a sincere persuasion of their being beneficial. I say,
even \emph{without:} for though in such a case it is not instruction
that can prevent the writer from endeavouring to inculcate his
principles, yet it may the readers from adopting them: without which,
his endeavouring to inculcate them will do no harm. In such a case, the
sovereign will commonly have little need to take an active part: if it
be the interest of \emph{one} individual to inculcate principles that
are pernicious, it will as surely be the interest of \emph{other}
individuals to expose them. But if the sovereign must needs take a part
in the controversy, the pen is the proper weapon to combat error with,
not the sword.

\chapter{Of the Proportion between Punishments and Offences}

I. We have seen that the general object of all laws is to prevent
mischief; that is to say, when it is worth while; but that, where there
are no other means of doing this than punishment, there are four cases
in which it is \emph{not} worth while.

II. When it is worth while, there are four subordinate designs or
objects, which, in the course of his endeavours to compass, as far as
may be, that one general object, a legislator, whose views are governed
by the principle of utility, comes naturally to propose to himself.

III. 1. His first, most extensive, and most eligible object, is to
prevent, in as far as it is possible, and worth while, all sorts of
offenses whatsoever: in other words, so to manage, that no offense
whatsoever may be committed.

IV. 2. But if a man must needs commit an offense of some kind or other,
the next object is to induce him to commit an offense \emph{less}
mischievous, rather than one \emph{more} mischievous: in other words, to
choose always the \emph{least} mischievous, of two offenses that will
either of them suit his purpose.

V. 3. When a man has resolved upon a particular offense, the next object
is to dispose him to do \emph{no more} mischief than is \emph{necessary}
to his purpose: in other words, to do as little mischief as is
consistent with the benefit he has in view.

VI. 4. The last object is, whatever the mischief be, which it is
proposed to prevent, to prevent it at as \emph{cheap} a rate as
possible.

VII. Subservient to these four objects, or purposes, must be the rules
or canons by which the proportion of punishments to offenses is to be
governed.

VIII. Rule 1. The first object, it has been seen, is to prevent, in as
far as it is worth while, all sorts of offenses; therefore, \emph{The
value of the punishment must not less in any case than what is
sufficient to outweigh that of the profit of the offense.} If it be, the
offence (unless some other considerations, independent of the punishment
should intervene and operate efficaciously in the character of tutelary
motives) will be sure to be to committed notwithstanding: the whole lot
of punishment will be thrown away: it will be altogether
\emph{inefficacious.}

IX. The above rule has been often objected to, on account of its seeming
harshness: but this can only have happened for want of its being
properly understood. The strength of the temptation, \emph{cæteris
paribas,} is as the profit of the offense: the quantum of the punishment
must rise with the profit of the offense: \emph{cæteris paribas,} it
must therefore rise with the strength of the temptation. This there is
no disputing. True it is, that the stronger the temptation, the less
conclusive is the indication which the act of delinquency affords of the
depravity of the offender's disposition. So far then as the absence of
any aggravation, arising from extraordinary depravity of disposition,
may operate, or at the utmost, so far as the presence of a ground of
extenuation, resulting from the innocence or beneficence of the
offender's disposition, can operate, the strength of the temptation may
operate in abatement of the demand for punishment. But it can never
operate so far as to indicate the propriety of making the punishment
ineffectual, which it is sure to be when brought below the level of the
apparent profit of the offense.\\
The partial benevolence which should prevail for the reduction of it
below this level, would counteract as well those purposes which such a
motive would actually have in view, as those more extensive purposes
which benevolence ought to have in view; it would be cruelty not only to
the public, but the very persons in whose behalf in pleads: in its
effects, I mean, however opposite in its intention. Cruelty to the
public, that is cruelty to the innocent, by suffering them, for wnat of
an adequate protection, to lie exposed to the mischief of the offense:
cruelty even the offender himself, by punishing him to no purpose, and
without the chance of compassing that beneficial end, by which alone the
introduction of the evil of punishment is to be justified.

X. Rule 2. But whether a given offence shall be prevented in a given
degree by a given quantity of punishment, is never any thing better than
a chance; for the purchasing of which, whatever punishment is employed,
is so much expended into advance. However, for the sake of giving it the
better chance of outweighing the profit of the offence, \emph{The
greater the mischief of the offense, the greater is the expense which it
may be worth while to be at, in the way of punishment.}

Where two offences come in competition, the punishment for the greater
offence must be sufficient to induce a man to prefer the less.

XII. Rule 4. When a man has resolved upon a particular offense, the next
object is, to induce him to do no more mischief than what is necessary
for his purpose: therefore\\
\emph{The punishment should be adjusted in such manner to each
particular offence, that for every part of the mischief there may be a
motive to restrain the offender frown giving birth to it.}

XIII. Rule 5. The last object is, whatever mischief is guarded against,
to guard against it at as cheap a rate as possible: therefore\\
\emph{The punishment ought in no case to be more than what is necessary
to bring it into conformity with the rules here given.}

XIV. Rule 6. It is further to be observed, that owing to the different
manners and degrees in which persons under different circumstances are
affected by the same exciting cause, a punishment which is the same in
name will not always either really produce, or even so much as appear to
others to produce, in two different persons the same degree of pain:
therefore\\
\emph{That the quantity actually indicted on each individual offender
nay correspond to the quantity intended for similar offenders in
general, the several circumstances influencing sensibility ought always
to be taken into account.}

XV. Of the above rules of proportion, the first four, we may perceive
serve to mark out limits on the side of diminution; the limits
\emph{below} which a punishment ought not to be \emph{diminished:} the
fifth the limits on the side of increase; the limits \emph{above} which
it ought not to be \emph{increased.} The five first are calculated to
serve as guides to the legislator: the sixth is calculated in some
measure, indeed, to the same purpose; but principally for guiding the
judge in his endeavors to conform, on both sides, to the intentions of
the legislator.

XVI. Let us look back a little. The first rule, in order to render it
more conveniently applicable to practice, may need perhaps to be a
little more particularly unfolded. It is to be observed, then, that for
the sake of accuracy, it was necessary, instead of the word
\emph{quantity} to make use of the less perspicuous term \emph{value.}
For the word \emph{quantity} will not properly include and the
circumstances either of certainty or proximity: circumstances which, in
estimating the value of a lot of pain or pleasure, must always be taken
into the account. Now, on the one hand, a lot of punishment is a lot of
pain; on the other hand, the profit of an offense is a lot of pleasure,
or what is equivalent to it. But the profit of the offense \emph{is}
commonly more \emph{certain} than the punishment, or, what comes to the
same thing, \emph{appears} so at least to the offender. It is at any
rate commonly more \emph{immediate.} It follows, therefore, that, in
order to maintain its superiority over the profit of the offense, the
punishment must have its value made up in some other way, in proportion
to that whereby it falls short in the two points of \emph{certainty} and
\emph{proximity.} Now there is no other way in which it can receive any
addition to its \emph{value,} but by receiving an addition in point of
\emph{magnitude.} Wherever then the value of the punishment falls short,
either in point of \emph{certainty,} or of \emph{proximity,} of that of
the profit of the offence, it must receive a proportionable addition in
point of \emph{magnitude.}

XVII. Yet farther. To make sure of giving the value of the punishment
the superiority over that of the offence, it may be of necessary, in
some cases, to take into account the profit not only of the
\emph{individual} offence to which the punishment is to be annexed, but
also of such \emph{other} offences of the \emph{same sort} as the
offender is likely to have already committed without detection. This
random mode of calculation, severe as it is, it will be impossible to
avoid having recourse to, in certain cases: in such, to wit, in which
the profit is pecuniary, the chance of detection very small, and the
obnoxious act of such a nature as indicates a habit: for example, in the
case of frauds against the coin. If it be \emph{not} recurred to, the
practice of committing the offence will be sure to be, upon the balance
of the account, a gainful practice. That being the case, the legislator
will be absolutely sure of \emph{not} being able to suppress it, and the
whole punishment that is bestowed upon it will be thrown away. In a word
(to keep to the same expressions we set out with) that whole quantity of
punishment will be \emph{inefficacious} .

XVIII. Rule 7. These things being considered, the three following rules
may be laid down by way of supplement and explanation to Rule 1.\\
\emph{To enable the value of the punishment to outweigh that of the
profit} \emph{of the offense, it must be increased, in point of
magnitude, in proportion as it falls short in point of certainty.}

XIX. Rule 8. \emph{Punishment must be further increased in point of
magnitude, in proportion as it falls short in point of proximity.}

XX. Rule 9. \emph{Where the act is conclusively indicative of a habit,
such an increase must be given to the punishment as may enable it to
outweigh the profit not only of the individual offence, but of such
other like offenses as are likely to have been committed with impunity
by the same offender.}\\
XXI. There may be a few other circumstances or considerations which may
influence, in some small degree, the demand for punishment: but as the
propriety of these is either not so demonstrable, or not so constant, or
the application of them not so determinate, as that of the foregoing, it
may be doubted whether they be worth putting on a level with the others.

XXII. Rule 10. \emph{When a punishment, which in point of quality is
particularly well calculated to answer its intention cannot exist in
less than a certain quantity, it may sometimes be of use, for the sake
of employing it, to stretch a little beyond that quantity which, on
other accounts, would be strictly necessary.}

XXIII. Rule 11. \emph{In particular, this may sometimes be the case,
where the punishment proposed is of such a nature as to be particularly
well calculated to answer the purpose of a moral lesson.}

XXIV. Rule 12. The tendency of the above considerations is to dictate an
augmentation in the punishment: the following rule operates in the way
of diminution. There are certain cases (it has been seen) in which, by
the influence of accidental circumstances, punishment may be rendered
unprofitable in the whole: in the same cases it may chance to be
rendered unprofitable as to a part only. Accordingly, \emph{In adjusting
the quantum of punishment, the circumstances; by which all punishment
may be rendered unprofitable, ought to be attended to.}

\emph{Among provisions designed to perfect the proportion between
punishments and offences, if any occur, which, by their own particular
good effects, would not make up for the harm they would do by adding to
the intricacy of the Code, they should be omitted.}

XXVI. It may be remembered, that the political sanction, being that to
which the sort of punishment belongs, which in this chapter is all along
in view, is but one of four sanctions, which may all of them contribute
their share towards producing the same effects. It maybe expected,
therefore, that in adjusting the quantity of political punishment,
allowance should be made for the assistance it may meet with from those
other controlling powers. True it is, that from each of these several
sources a very powerful assistance may sometimes be derived. But the
case is, that (setting aside the moral sanction, in the case where the
force of it is expressly adopted into and modified by the political) the
force of those other powers is never determinate enough to be depended
upon. It can never be reduced, like political punishment, into exact
lots, nor meted out in number, quantity, and value. The legislator is
therefore obliged to provide the full complement of punishment, as if he
were sure of not receiving any assistance whatever from any of those
quarters. If\\
he does, so much the better: but lest he should not, it is necessary he
should, at all events, make that provision which depends upon himself.\\
XXVII. It may be of use, in this place, to recapitulate the several
circumstances, which, in establishing the proportion betwixt punishments
and offenses, are to be attended to. These seem to be as follows:\\
I. \emph{On the part of the offence:}\\
1. The profit of the offense;\\
2. The mischief of the offense;\\
3. The profit and mischief of other greater or lesser offences, of
different sorts, which the offender may have to choose out of;\\
4. The profit and mischief of other offenses, of the same sort, which
the same offender may probably have been guilty of already.

II. \emph{On the part of the punishment:}\\
5. The magnitude of the punishment: composed of its intensity and
duration;\\
6. The deficiency of the punishment in point of certainty;\\
7. The deficiency of the punishment in point of proximity;\\
8. The quality of the punishment;\\
9. The accidental advantage in point of quality of a punishment, not
strictly needed in point of quantity;\\
10. The use of a punishment of a particular quality, in the character of
a moral lesson.

III. \emph{On the part of the offender:}\\
11. The responsibility of the class of persons in a way to offend;\\
12. The sensibility of each particular offender\\
13. The particular merits or useful qualities of any particular
offender, in case of a punishment which might deprive the community of
the benefit of them;\\
14. The multitude of offenders on any particular occasion.

IV. \emph{On the part of the public,} at any particular conjuncture:\\
15. The inclinations of the people, for or against any quantity or mode
of punishment;\\
16. The inclinations of foreign powers.

V. \emph{On the part of the law:} that is, of the public for a
continuance:\\
17. The necessity of making small sacrifices, in point of
proportionality, for the sake of simplicity.

XXVIII. There are some, perhaps, who, at first sight, may look upon the
nicety employed in the adjustment of such rules, as so much labour lost:
for gross ignorance, they will say, never, troubles itself about laws,
and passion does not calculate. But, the evil of ignorance admits of
cure: and as to the proposition that passion does not calculate, this,
like most of these very general and oracular propositions, is not true.
When matters of such importance as pain and pleasure are at stake, and
these in the highest degree (the only matters, in short, that can be of
importance) who is there that does not calculate? Men calculate, some
with less exactness, indeed, some with more: but all men calculate. I
would not say, that even a madman does not calculate. Passion
calculates, more or less, in every man: in different men, according to
the warmth or coolness of their dispositions: according to the firmness
or irritability of their minds: according to the nature of the motives
by which they are acted upon. Happily, of all passions, that is the most
given to calculation, from the excesses of which, by reason of its
strength, constancy, and universality, society has most to apprehend: I
mean that which corresponds to the motive of pecuniary interest: so that
these niceties, if such they are to be called, have the best chance of
being efficacious, where efficacy is of the most importance.

\chapter{Of the Properties to be Given to a Lot of
Punishment}

I. It has been shown what the rules are, which ought to be observed in
adjusting the proportion between the punishments and the offense. The
properties to be given to a lot of punishment, in every instance, will
of course be such as it stands in need of, in order to be capable of
being applied, in conformity to those rules: the quality will be
regulated by the quantity.

II. The first of those rules, we may remember, was, that the quantity of
punishment must not be less, in any case, than what is sufficient to
outweigh the profit of the offence: since, as often as it is less, the
whole lot (unless by accident the deficiency should be supplied from
some of the other sanctions) is thrown away: it is \emph{inefficacious.}
The fifth was, that the punishment ought in no case to be more than what
is required by the several other rules: since, if it be, all that is
above that quantity is \emph{needless.} The fourth was, that the
punishment should be adjusted in such manner to each individual offence,
that every part of the mischief of that offence may have a penalty (that
is, a tutelary motive) to encounter it: otherwise, with respect to so
much of the offense as has not a penalty to correspond to it, it is as
if there were no punishment in the case. Now to none of those rules can
a lot of punishment be conformable, unless, for every variation in point
of quantity, in the mischief of the species of offense to which it is
annexed, such lot of punishment admits of a correspondent variation. To
prove this, let the profit of the offence admit of a multitude of
degrees. Suppose it, then, at any one of these degrees: if the
punishment be less than what is suitable to that degree, it will be
\emph{inefficacious;} it will be so much thrown away: if it be more, as
far as the difference extends, it will be \emph{needless;} it will
therefore be thrown away also in that case.\\
The first property, therefore, that ought to be given to a lot of
punishment, is that of being variable in point of quantity, in
conformity to every variation which can take place in either the profit
or mischief of the offense. This property might, perhaps, be termed, in
a single word, \emph{variability.}

III. A second property, intimately connected with the former, may be
styled \emph{equability.} It will avail but little, that a mode of
punishment (proper in all other respects) has been established by the
legislator; and that capable of being screwed up or let down to any
degree that can be required; if, after all, whatever degree of it be
pitched upon, that same degree shall be liable, according to
circumstances, to produce a very heavy degree of pain, or a very slight
one, or even none at all. In this case, as in the former, if
circumstances happen one way, there will be a great deal of pain
produced which will be \emph{needless:} if the other way, there will be
no pain at all applied, or none that will be \emph{efficacious.} A
punishment, when liable to this irregularity, may be styled an unequable
one: when free from it, an equable one. The quantity of pain produced by
the punishment will, it is true, depend in a considerable degree upon
circumstances distinct from the nature of the punishment itself: upon
the condition which the offender is in, with respect to the
circumstances by which a man's sensibility is liable to be influenced.
But the influence of these very circumstances will in many cases be
reciprocally influenced by the nature of the punishment: in other words,
the pain which is produced by any mode of punishment, will be the joint
effect of the punishment which is applied to him, and the circumstances
in which he is exposed to it. Now there are some punishments, of which
the effect may be liable to undergo a greater alteration by the
influence of such foreign circumstances, than the effect of other
punishments is liable to undergo. So far, then, as this is the case,
equability or unequability may be regarded as properties belonging to
the punishment itself.

IV. An example of a mode of punishment which is apt to be unequable, is
that of \emph{banishment,} when the \emph{locus a quo} (or place the
party is banished from) is some determinate place appointed by the law,
which perhaps the offender cares not whether he ever see or no. This is
also the case with \emph{pecuniary,} or \emph{quasi-pecuniary}
punishment, when it respects some particular species of property, which
the offender may have been possessed of, or not, as it may happen. All
these punishments may be split down into parcels, and measured out with
the utmost nicety: being divisible by time, at least, if by nothing
else. They are not, therefore, any of them defective in point of
variability: and yet, in many cases, this defect in point of equability
may make them as unfit for use as if they were.

V. The third rule of proportion was, that where two offenses come in
competition, the punishment for the greater offenses must be sufficient
to induce a man to prefer the less. Now, to be sufficient for this
purpose, it must be evidently and uniformly greater: greater, not in the
eyes of some men only, but of all men who are liable to be in a
situation to take their choice between the two offenses; that is, in
effect, of all mankind. In other words, the two punishments must be
perfectly \emph{commensurable.} Hence arises a third property, which may
be termed \emph{commensurability:} to wit, with reference to other
punishments.

VI. But punishments of different kinds are in very few instances
uniformly greater one than another; especially when the lowest degrees
of that which is ordinarily the greater, are compared with the highest
degrees of that which is ordinarily the less: in other words,
punishments of different kinds are in few instances uniformly
commensurable. The only certain and universal means of making two lots
of punishment perfectly commensurable, is by making the lesser an
ingredient in the composition of the greater. This may be done in either
of two ways.\\
1. By adding to the lesser punishment another quantity of punishment of
the same kind.\\
2. By adding to it another quantity of a different kind. The latter mode
is not less certain than the former: for though one cannot always be
absolutely sure, that to the same person a given punishment will appear
greater than another given punishment; yet one may be always absolutely
sure, that any given punishment, so as it does but come into
contemplation, will appear greater than none at all.

VII. Again: Punishment cannot act any farther than in as far as the idea
of it, and of its connection with the offense, is present in the mind.
The idea of it, if not present, cannot act at all; and then the
punishment itself must be \emph{inefficacious.} Now, to be present, it
must be remembered, and to be remembered it must have been learnt. But
of all punishments that can be imagined, there are none of which the
connection with the offense is either so easily learnt, or so
efficaciously remembered, as those of which the idea is already in part
associated with some part of the idea of the offense: which is the case
when the one and the other have some circumstance that belongs to them
in common. When this is the case with a punishment and an offense, the
punishment is said to bear an \emph{analogy} to, or to be
\emph{characteristic} of, the offence. \emph{Characteristicalness} is,
therefore, a fourth property, which on this account ought to be given,
whenever it can conveniently be given, to a lot of punishment. VIII. It
is obvious, that the effect of this contrivance will be the greater, as
the analogy is the closer. The analogy will be the closer, the more
\emph{material} that circumstance is, which is in common. Now the most
material circumstance that can belong to an offense and a punishment in
common, is the hurt or damage which they produce. The closest analogy,
therefore, that can subsist between an offense and the punishment
annexed to it, is that which subsists between them when the hurt or
damage they produce is of the same nature: in other words, that which is
constituted by the circumstance of identity in point of damage.
Accordingly, the mode of punishment, which of all others bears the
closest analogy to the offense, is that which in the proper and exact
sense of the word is termed \emph{retaliation.} Retaliation, therefore,
in the few cases in which it is practicable, and not too expensive, will
have one great advantage over every other mode of punishment.

IX. Again: It is the idea only of the punishment (or, in other words,
the \emph{apparent} punishment) that really acts upon the mind; the
punishment itself (the \emph{real} punishment) acts not any farther than
as giving rise to that idea. It is the apparent punishment, therefore,
that does all the service, I mean in the way of example, which is the
principal object. It is the real punishment that does all the mischief.
Now the ordinary and obvious way of increasing the magnitude of the
apparent punishment, is by increasing the magnitude of the real. The
apparent magnitude, however, may to a certain degree be increased by
other less expensive means: whenever, therefore, at the same time that
these less expensive means would have answered that purpose, an
additional real punishment is employed, this additional real punishment
is \emph{needless.} As to these less expensive means, they consist,\\
1. In the choice of a particular mode of punishment, a punishment of a
particular quality, independent of the quantity.\\
2. In a particular set of \emph{solemnities} distinct from the
punishment itself, and accompanying the execution of it.

X. A mode of punishment, according as the appearance of it bears a
greater proportion to the reality, may be said to be the more
\emph{exemplary.} Now as to what concerns the choice of the punishment
itself, there is not any means by which a given quantity of punishment
can be rendered more exemplary, than by choosing it of such a sort as
shall bear an \emph{analogy} to the offense. Hence another reason for
rendering the punishment analogous to, or in other words characteristic
of, the offense.

XI. Punishment, it is still to be remembered, is in itself an expense:
it is in itself an evil. Accordingly the fifth rule of proportion is,
not to produce more of it than what is demanded by the other rules. But
this is the case as often as any particle of pain is produced, which
contributes nothing to the effect proposed. Now if any mode of
punishment is more apt than another to produce any such superfluous and
needless pain, it may be styled \emph{unfrugal;} if less, it may be
styled \emph{frugal. Frugality,} therefore, is a sixth property to be
wished for in a mode of punishment.

XII. The perfection of frugality, in a mode of punishment, is where not
only no superfluous pain is produced on the part of the person punished,
but even that same operation, by which he is subjected to pain, is made
to answer the purpose of producing pleasure on the part of some other
person. Understand a profit or stock of pleasure of the self-regarding
kind: for a pleasure of the dissocial kind is produced almost of course,
on the part of all persons in whose breasts the offence has excited the
sentiment of ill-will. Now this is the case with pecuniary punishment,
as also with such punishments of the \emph{quasi-pecuniary} kind as
consist in the subtraction of such a species of possession as is
transferable from one party to another. The pleasure, indeed, produced
by such an operation, is not in general equal to the pain: it may,
however, be so in particular circumstances, as where he, from whom the
thing is taken, is very rich, and he, to whom it is given, very poor:
and, be it what it will, it is always so much more than can be produced
by any other mode of punishment.

XIII. The properties of exemplarity and frugality seem to pursue the
same immediate end, though by different courses. Both are occupied in
diminishing the ratio of the real suffering to the apparent: but
exemplarity tends to increase the apparent; frugality to reduce the
real.

XIV. Thus much concerning the properties to be given to punishments in
general, to whatsoever offenses they are to be applied. Those which
follow are of less importance, either as referring only to certain
offenses in particular, or depending upon the influence of transitory
and local circumstances.\\
In the first place, the four distinct ends into which the main and
general end of punishment is divisible, may give rise to so many
distinct properties, according as any particular mode of punishment
appear to be more particularly adapted to the compassing of one or of
another of those ends. To that of \emph{example,} as being the principal
one, a particular property has already been adapted. There remains the
three inferior ones of \emph{reformation, disablement,} and
\emph{compensation.}

XV. A seventh property, therefore, to be wished for in a mode of
punishment, is that of \emph{subserviency to reformation,} or
\emph{reforming tendency.} Now any punishment is subservient to
reformation in proportion to its \emph{quantity:} since the greater the
punishment a man has experienced, the stronger is the tendency it has to
create in him an aversion towards the offense which was the cause of it:
and that with respect to all offenses alike. But there are certain
punishments which, with regard to certain offenses, have a particular
tendency to produce that effect by reason of their \emph{quality:} and
where this is the case, the punishments in question, as applied to the
offenses in question, will \emph{pro tanto} have the advantage over all
others. This influence will depend upon the nature of the motive which
is the cause of the offence: the punishment most subservient to
reformation will be the sort of punishment that is best calculated to
invalidate the force of that motive.

XVI. Thus, in offenses originating from the motive of ill-will, that
punishment has the strongest reforming tendency, which is best
calculatedto weaken the force of the irascible affections. And more
particularly, in that sort of offense which consists in an obstinate
refusal, on the part of the offender, to do something which is lawfully
required of him, and in which the obstinacy is in great measure kept up
by his resentment against those who have an interest in forcing him to
compliance, the most efficacious punishment seems to be that of
confinement to spare diet.

XVII. Thus, also, in offenses which owe their birth to the joint
influence of indolence and pecuniary interest, that punishment seems to
possess the strongest reforming tendency, which is best calculated to
weaken the force of the former of those dispositions. And more
particularly, in the cases of theft, embezzlement, and every species of
defraudment, the mode of punishment best adapted to this purpose seems,
in most cases, to be that of penal labour.

XVIII. An eighth property to be given to a lot of punishment in certain
cases, is that of \emph{efficacy with respect to disablement,} or, as it
might be styled more briefly, \emph{disabling efficacy.} This is a
property which may be given in perfection to a lot of punishment; and
that with much greater certainty than the property of subserviency to
reformation. The inconvenience is, that this property is apt, in
general, to run counter to that of frugality: there being, in most
cases, no certain way of disabling a man from doing mischief, without,
at the same time, disabling him, in a great measure, from doing good,
either to himself or others. The mischief therefore of the offense must
be so great as to demand a very considerable lot of punishment, for the
purpose of example, before it can warrant the application of a
punishment equal to that which is necessary for the purpose of
disablement.

XIX. The punishment, of which the efficacy in this way is the greatest,
is evidently that of death. In this case the efficacy of it is certain .
This accordingly is the punishment peculiarly adapted to those cases in
which the name of the offender, so long as he lives, may be sufficient
to keep a whole nation in a flame. This will now and then be the case
with competitors for the sovereignty, and leaders of the factions in
civil wars: though, when applied to offenses of so questionable a
nature, in which the question concerning criminality turns more upon
success than any thing else; an infliction of this sort may seem more to
savour of hostility than punishment. At the same time this punishment,
it is evident, is in an eminent degree \emph{unfrugal;} which forms one
among the many objections there are against the use of it, in any but
very extraordinary cases.

XX. In ordinary cases the purpose maybe sufficiently answered by one or
other of the various kinds of confinement and banishment: of which,
imprisonment is the most strict and efficacious. For when an offense is
so circumstanced that it cannot be committed but in a certain place, as
is the case, for the most part, with offenses against the person, all
the law has to do, in order to disable the offender from committing it,
is to prevent his being in that place. In any of the offenses which
consist in the breach or the abuse of any kind of trust, the purpose may
be compassed at a still cheaper rate, merely by forfeiture of the trust:
and in general, in any of those offenses which can only be committed
under favour of some relation in which the offender stands with
reference to any person, or sets of persons, merely by forfeiture of
that relation: that is, of the right of continuing to reap the
advantages belonging to it. This is the case, for instance, with any of
those offences which consist in an abuse of the privileges of marriage,
or of the liberty of carrying on any lucrative or other occupation.

XXI. The ninth property is that of \emph{subserviency to compensation.}
This property of punishment, if it be \emph{vindictive} compensation
that is in view, will, with little variation, be in proportion to the
quantity: if \emph{lucrative,} it is the peculiar and characteristic
property of pecuniary punishment.

XXII. In the rear of all these properties may be introduced that of
\emph{popularity;} a very fleeting and indeterminate kind of property,
which may belong to a lot of punishment one moment, and be lost by it
the next. By popularity is meant the property of being acceptable, or
rather not unacceptable, to the bulk of the people, among whom it is
proposed to be established. In strictness of speech, it should rather be
called \emph{absence of unpopularity:} for it cannot be expected, in
regard to such a matter as punishment, that any species or lot of it
should be positively acceptable and grateful to the people: it is
sufficient, for the most part, if they have no decided aversion to the
thoughts of it. Now the property of characteristicalness, above noticed,
seems to go as far towards conciliating the approbation of the people to
a mode of punishment, as any; insomuch that popularity may be regarded
as a kind of secondary quality, depending upon that of
characteristicalness. The use of inserting this property in the
catalogue, is chiefly to make it serve by way of memento to the
legislator not to introduce, without a cogent necessity, any mode or lot
of punishment, towards which he happens to perceive any violent aversion
entertained by the body of the people.

XXIII. The effects of unpopularity in a mode of punishment are analogous
to those of unfrugality. The unnecessary pain which denominates a
punishment unfrugal, is most apt to be that which is produced on the
part of the offender. A portion of superfluous pain is in like manner
produced when the punishment is unpopular: but in this case it is
produced on the part of persons altogether innocent, the people at
large. This is already one mischief; and another is, the weakness which
it is apt to introduce into the law. When the people are satisfied with
the law, they voluntarily lend their assistance in the execution: when
they are dissatisfied, they will naturally withhold that assistance; it
is well if they do not take a positive part in raising impediments. This
contributes greatly to the uncertainty of the punishment; by which, in
the first instance, the frequency of the offense receives an increase.
In process of time that deficiency, as usual, is apt to draw on an
increase in magnitude: an addition of a certain quantity which otherwise
would be \emph{needless.}

XXIV. This property, it is to be observed, necessarily supposes, on the
part of the people, some prejudice or other, which it is the business of
the legislator to endeavour to correct. For if the aversion to the
punishment in question were grounded on, the principle of utility, the
punishment would be such as, on other accounts, ought not to be
employed: in which case its popularity or unpopularity would never be
worth drawing into question. It is properly therefore a property not so
much of the punishment as of the people: a disposition to entertain an
unreasonable dislike against an object which merits their approbation.
It is the sign also of another property, to wit. indolence or weakness,
on the part of the legislator: in suffering the people for the want of
some instruction, which ought to be and might be given them, to quarrel
with their own interest. Be this as it may, so long as any such
dissatisfaction subsists, it behoves the legislator to have an eye to
it, as much as if it were ever so well grounded. Every nation is liable
to have its prejudices and its caprices which it is the business of the
legislator to look out for, to study, and to cure.

XXV. The eleventh and last of all the properties that seem to be
requisite in a lot of punishment, is that of \emph{remissibility.} The
general presumption is, that when punishment is applied, punishment is
needful: that it ought to be applied, and therefore cannot want to be
\emph{remitted.} But in very particular, and those always very
deplorable cases, it may by accident happen otherwise. It may happen
that punishment shall have been inflicted, where, according to the
intention of the law itself, it ought not to have been inflicted: that
is, where the sufferer is innocent of the offense. At the time of the
sentence passed he appeared guilty: but since then, accident has brought
his innocence to light. This being the case, so much of the destined
punishment as he has suffered already, there is no help for. The
business is then to free him from as much as is yet to come. But is
there any yet to come? There is very little chance of there being any,
unless it be so much as consists of \emph{chronical} punishment: such as
imprisonment, banishment, penal labour, and the like. So much as
consists of \emph{acute} punishment, to wit where the penal process
itself is over presently, however permanent the punishment may be in its
effects, may be considered as irremissible. This is the case, for
example, with whipping, branding, mutilation, and capital punishment.
The most perfectly irremissible of any is capital punishment. For though
other punishments cannot, when they are over, be remitted, they may be
compensated for; and although the unfortunate victim cannot be put into
the same condition, yet possibly means may be found of putting him into
as good a condition, as he would have been in if he had never suffered.
This may in general be done very effectually where the punishment has
been no other than pecuniary.\\
There is another case in which the property of remissibility may appear
to be of use: this is, where, although the offender has been justly
punished, yet on account of some good behaviour of his, displayed at a
time subsequent to that of the commencement of the punishment, it may
seem expedient to remit a part of it. But this it can scarcely be, if
the proportion of the punishment is, in other respects, what it ought to
be. The purpose of example is the more important object, in comparison
of that of reformation. It is not very likely, that less punishment
should be required for the former purpose than for the latter. For it
must be rather an extraordinary case, if a punishment, which is
sufficient to deter a man who has only thought of it for a few moments,
should not be sufficient to deter a man who has been feeling it all the
time. Whatever, then, is required for the purpose of example, must abide
at all events: it is not any reformation on the part of the offender,
that can warrant the remitting of any part of it: if it could, a man
would have nothing to do but to reform immediately, and so free himself
from the greatest part of that punishment which was deemed necessary. In
order, then, to warrant the remitting of any part of a punishment upon
this ground, it must first be supposed that the punishment at first
appointed was more than was necessary for the purpose of example, and
consequently that a part of it was \emph{needless} upon the whole. This,
indeed, is apt enough to be the case, under the imperfect systems that
are as yet on foot: and therefore, during the continuance of those
systems, the property of remissibility may, on this second ground
likewise, as well as on the former, be deemed a useful one. But this
would not be the case in any new-constructed system, in which the rules
of proportion above laid down should be observed. In such a system,
therefore, the utility of this property would rest solely on the former
ground.

XXVI. Upon taking a survey of the various possible modes of punishment,
it will appear evidently, that there is not any one of them that
possesses all the above properties in perfection. To do the best that
can be done in the way of punishment, it will therefore be necessary,
upon most occasions, to compound them, and make them into complex lots,
each consisting of a number of different modes of punishment put
together: the nature and proportions of the constituent parts of each
lot being different, according to the nature of the offence which it is
designed to combat.

XXVII. It may not be amiss to bring together, and exhibit in one view,
the eleven properties above established. They are as follows:\\
Two of them are concerned in establishing a proper proportion between a
single offense and its punishment; viz.,\\
1. Variability.\\
2. Equability.\\
One, in establishing a proportion, between more offences than one, and
more punishments than one; viz.,\\
3. Commensurability.

A fourth contributes to place the punishment in that situation in which
alone it can be efficacious; and at the same time to be bestowing on it
the two farther properties of exemplarity and popularity; viz.,\\
4. Characteristicalness.

Two others are concerned in excluding all useless punishment; the one
indirectly, by heightening the efficacy of what is useful; the other in
a direct way; viz.,\\
5. Exemplarity.\\
6. Frugality.

Three others contribute severally to the three inferior ends of
punishment; viz.,\\
7. Subserviency to reformation.\\
8. Efficacy in disabling.\\
9. Subserviency to compensation.

Another property tends to exclude a collateral mischief, which a
particular mode of punishment is liable accidentally to produce; viz.,\\
10. Popularity.

The remaining property tends to palliate a mischief, which all
punishment, as such is liable accidentally to produce; viz.,\\
11. Remissibility.

The properties of commensurability, characteristicalness, exemplarity,
subserviency to reformation, and efficacy in disabling, are more
particularly calculated to augment the profit which is to be made by
punishment: frugality, subserviency to compensation, popularity, and
remissibility, to diminish the expense: variability and equability are
alike subservient to both those purposes.

XXVIII. We now come to take a general survey of the system of offences:
that is, of such acts to which, on account of the mischievous
consequences they have a natural tendency to produce, and in the view of
putting a stop to those consequences, it may be proper to annex a
certain artificial consequence, consisting of punishment, to be
inflicted on the authors of such acts according to the principles just
established.

\chapter{Division of Offenses}

§1. Classes of Offences

I. It is necessary, at the outset, to make a distinction between such
acts as \emph{are} or \emph{may} be, and such as \emph{ought} to be
offences.\\
Any act may be an offence, which they whom the community of are in the
habit of obeying shall be pleased to make one: that to is, any act which
they shall be pleased to prohibit or to punish. But, upon the principle
of utility, such acts alone \emph{ought} to be made offences, as the
good of the community requires should be made so.

II. The good of the community cannot require, that any act should be
made an offence, which is not liable. in some way or other, to be
detrimental to the community. For in the case of such an act, all
punishment is \emph{groundless.}

III. But if the whole assemblage of any number of individuals be
considered as constituting an imaginary compound \emph{body,} a
community or political state; any act that is detrimental to any one or
more of those \emph{members} is, as to so much of its effects,
detrimental to the \emph{state.}

IV. An act cannot be detrimental to a \emph{state,} but by being
detrimental to some one or more of the \emph{individuals} that compose
it. But these individuals may either be \emph{assignable} or
\emph{unassignable.}

V. When there is any assignable individual to whom an offence is
detrimental, that person may either be a person \emph{other} than the
offender, or the offender \emph{himself.}

VI. Offences that are detrimental, in the first instance, to assignable
persons other than the offender, may be termed by one common name,
\emph{offences against individuals.} And of these may be composed the
1st class of offences. To contrast them with offences of the 2nd and 4th
classes, it may also sometimes be convenient to style them
\emph{private} offences. To contrast them at the same time with offences
of the 3rd class, they may be styled \emph{private extra-regarding}
offences.

VII. When it appears, in general, that there are persons to whom the act
in question may be detrimental, but such persons cannot be individually
assigned, the circle within which it appears that they may be found, is
either of less extent than that which comprises the whole community, or
not. If of less, the persons comprised within this lesser circle may be
considered for this purpose as composing a body of themselves; comprised
within, but distinguishable from, the greater body of the whole
community. The circumstance that constitutes the union between the
members of this lesser body, may be either their residence within a
particular place, or, in short, any other less explicit principle of
union, which may serve to distinguish them from the remaining members of
the community. In the first case, the act may be styled an \emph{offence
against a neighbourhood:} in the second, an offence against a particular
\emph{class} of persons in the community. Offenses, then, against a
class or neighbourhood, may, together, constitute the 2nd class of
offences. To contrast them with private offences on the one hand, and
public on the other, they may also be styled \emph{semi-public}
offences.

VIII. Offences, which in the first instance are detrimental to the
offender himself, and to no one else, unless it be by their being
detrimental to himself, may serve to compose a third class. To contrast
them the better with offences of the first, second, and fourth classes,
all which are of a \emph{transitive} nature, they might be styled
\emph{intransitive} offences; but still better, \emph{self-regarding.}

IX. The fourth class may be composed of such acts as ought to be made
offences, on account of the distant mischief which they threaten to
bring upon an unassignable indefinite multitude of the whole number of
individuals, of which the community is composed: although no particular
individual should appear more likely to be a sufferer by them than
another. These may be called \emph{public} offences, or offences against
the \emph{state.}

X. A fifth class, or appendix, may be composed of such acts as,
according to the circumstances in which they are committed, or and more
particularly according to the purposes to which they are applied, may be
detrimental in any one of the ways in which the act of one man can be
detrimental to another. These may to be termed \emph{multiform,} or
\emph{heterogeneous} offences. Offences that are in this case may be
reduced to two great heads:\\
1. Offences by \emph{falsehood:} and\\
2. Offenses against \emph{trust.}

§2. Divisions and sub-divisions

XI. Let us see by what method these classes may be farther subdivided.\\
First, then, with regard to offences against individuals.\\
In the present period of existence, a man's being and wellbeing, his
happiness and his security; in a word, his pleasures and his immunity
from pains, are all dependent, more or less, in the first place, upon
his \emph{own person;} in the next place, upon the \emph{exterior
objects} that surround him. These objects are either \emph{things,} or
other \emph{persons.} Under one or other of these classes must evidently
be comprised every sort of exterior object, by means of which his
interest can be affected. If then, by means of any offence, a man should
on any occasion become a sufferer, it must be in one or other of two
ways:\\
1. \emph{absolutely,} to wit, immediately in his own person; in which
case the offence may be said to be an offence against his person: or,\\
2. \emph{relatively,} by reason of some \emph{material} relation which
the before mentioned exterior objects may happen to bear, in the way of
\emph{causality} to his happiness.\\
Now in as far as a man is in a way to derive either happiness or
security from any object which belongs to the class of \emph{things,}
such thing is said to be his \emph{property,} or at least he is said to
have a \emph{property} or an \emph{interest} therein: an offence,
therefore, which tends to lessen the facility he might otherwise have of
deriving happiness or security from an object which belongs to the class
of things may be styled an offence against his property. With regard to
persons, in as far as, from objects of this class, a man is in a way to
derive happiness or security, it is in virtue of their \emph{services:}
in virtue of some services, which, by one sort of inducement or another,
they may be disposed to render him. Now, then, take any man, by way of
example, and the disposition, whatever it may be, which he may be in to
render you service, either has no other connection to give birth or
support to it, than the general one which binds him to the whole
species, or it has some other connection more particular. In the latter
case, such a connection may be spoken of as constituting, in your
favour, a kind of fictitious or incorporeal object of property, which is
styled your \emph{condition.} An offence, therefore, the tendency of
which is to lessen the facility you might otherwise have of deriving
happiness from the services of a person thus specially connected with
you, may be styled an offence against your condition in life, or simply
against your condition. Conditions in life must evidently be as various
as the relations by which they are constituted. This will be seen more
particularly farther on. In the mean time those of husband, wife,
parent, child, master, servant, citizen of such or such a city,
natural-born subject of such or such a country, may answer the purpose
of examples.

Where there is no such particular connection, or (what comes to the same
thing) where the disposition, whatever it may be, which a man is in to
render you service, is not considered as depending upon such connection,
but simply upon the good-will he bears to you; in such case, in order to
express what chance you have of deriving a benefit from his services, a
kind of fictitious object of property is spoken of, as being constituted
in your favour, and is called your \emph{reputation.} An offence,
therefore, the tendency of which is to lessen the facility you might
otherwise have had of deriving happiness or security from the services
of persons at large, whether connected with you or not by any special
tie, may be styled an offence against your \emph{reputation.} It
appears, therefore, that if by any offence an individual becomes a
sufferer, it must be in one or other of the four points above mentioned;
viz., his person, his property, his condition in life, or his
reputation. These sources of distinction, then, may serve to form so
many subordinate divisions. If any offences should be found to affect a
person in more than one of these points at the same time, such offences
may respectively be put under so many separate divisions; and such
compound divisions may be subjoined to the preceding simple ones The
several divisions (simple and compound together) which are hereinafter
established, stand as follows:\\
1. Offenses against person.\\
2. Offenses against reputation.\\
3. Offenses against property.\\
4. Offenses against condition.\\
5. Offenses against person and property together.\\
6. Offenses against person and reputation together.

XII. Next with regard to semi-public offences. Pain, considered with
reference to the time of the act from which it is liable to issue, must,
it is evident, be either present, past, or future. In as far as it is
either present or past, it cannot be the result of any act which comes
under the description of a semi-public offence: for if it be present or
past, the individuals who experience, or who have experienced, it are
\emph{assignable.} There remains that sort of mischief, which, if it
ever come to exist at all, is as yet but future: mischief, thus
circumstanced, takes the name of \emph{danger.} Now, then, when by means
of the act of any person a whole neighbourhood, or other class of
persons, are exposed to danger, this danger must either be
\emph{intentional} on his part, or \emph{unintentional.} If
unintentional, such danger, when it is converted into actual mischief,
takes the name of a \emph{calamity:} offences, productive of such
danger, may be styled \emph{semi-public offences operating through
calamity;} or, more briefly, \emph{offences through calamity.} If the
danger be intentional, insomuch that it might be produced, and might
convert itself into actual mischief, without the concurrence of any
calamity, it may be said to originate in \emph{mere delinquency:}
offences, then, which, without the concurrence of any calamity, tend to
produce such danger as disturbs the security of a local, or other
subordinate class of persons, may be styled \emph{semi-public offences
operating merely by delinquency,} or more briefly, \emph{offences of
mere delinquency.} '

XIII. With regard to any farther sub-divisions, offences through
calamity will depend upon the nature of the several calamities to which
man, and the several things that are of use to him, stand exposed. These
will be considered in another place.

XIV. Semi-public offences of mere delinquency will follow the method of
division applied to offences against individuals. It will easily be
conceived, that whatever pain or inconvenience any given individual may
be made to suffer, to the danger of that pain or inconvenience may any
number of individuals, assignable or not assignable, be exposed. Now
there are four points or articles, as we have seen, in respect to which
an individual may be made to suffer pain or inconvenience. If then, with
respect to any one of them, the connection of causes and effects is
such, that to the danger of suffering in that article a number of
persons, who individually are not assignable, may, by the delinquency of
one person, be exposed, such article will form a ground of distinction
on which a particular sub-division of semi-public offences may be
established: if, with respect to any such article, no such effect can
take place, that ground of distinction will lie for the present
unoccupied: ready, however, upon any change of circumstances, or in the
manner of viewing the subject, to receive a correspondent subdivision of
offences, if ever it should seem necessary that any such offences should
be created.

XV. We come next to self-regarding offences; or, more properly, to acts
productive in the first instance of no other than a self-regarding
mischief: acts which, if in any instance it be thought fit to constitute
them offences, will come under the denomination of offences against
one's self. This class will not for the present give us much trouble.
For it is evident, that in whatever points a man is vulnerable by the
hand of another, in the same points may he be conceived to be vulnerable
by his own. Whatever divisions therefore will serve for the first class,
the same will serve for this. As to the questions, What acts are
productive of a mischief of this stamp? and, among such as \emph{are,}
which it may, and which it may not, be \emph{worth while} to treat upon
the footing of offences? these are points, the latter of which at least
is too unsettled, and too open to controversy, to be laid down with that
degree of confidence which is implied in the exhibition of properties
which are made use of as the groundwork of an arrangement. Properties
for this purpose ought to be such as show themselves at first glance,
and appear to belong to the subject beyond dispute.

XVI. Public offences may be distributed under eleven divisions.\\
1. Offences against \emph{external} security.\\
2. Offences against \emph{justice.}\\
3. Offences against the \emph{preventive} branch of the police.\\
4. Offences against the public \emph{force.}\\
5. Offences against the \emph{positive} increase of the national
\emph{felicity.}\\
6. Offences against the public \emph{wealth.}\\
7. Offences against \emph{population.}\\
8. Offences against the \emph{national wealth.}\\
9. Offences against the \emph{sovereignty.}\\
10. Offences against \emph{religion.}\\
11. Offences against the national \emph{interest} in general. The way in
which these several sorts of offences connect with one another, and with
the interest of the public, that is, of an unassignable multitude of the
individuals of which that body is composed, may be thus conceived.

XVII. Mischief by which the interest of the public as above defined may
be affected, must, if produced at all, be produced either by
means\textgreater{}of an influence exerted on the operations of
government, or by other means, without the exertion of such influence.
To begin with the latter case: mischief, be it what it will, and let it
happen to whom it will, must be produced either by the unassisted powers
of the agent in question, or by the instrumentality of some other
agents. In the latter case, these agents will be either persons or
things. Persons again must be either not members of the community in
question, or members. Mischief produced by the instrumentality of
persons, may accordingly be produced by the instrumentality either of
\emph{external} or of \emph{internal} adversaries. Now when it is
produced by the agent's own unassisted powers, or by the instrumentality
of internal adversaries, or only by the instrumentality of things, it is
seldom that it can show itself in any other shape (setting aside any
influence it may exert in the operations of government) than either that
of an offence against assignable individuals, or that of an offence
against a local or other subordinate class of persons. If there should
be a way in which mischief can be produced, by any of these means, to
individuals altogether unassignable, it will scarcely be found
conspicuous or important enough to occupy a title by itself: it may
accordingly be referred to the miscellaneous head of \emph{offences
against the national interest in general.} The only mischief, of any
considerable account, which can be made to impend indiscriminately over
the whole number of members in the community, is that complex kind of
mischief which results from a state of war, and is produced by the
instrumentality of external adversaries; by their being provoked, for
instance, or invited, or encouraged to invasion. In this way may a man
very well bring down a mischief, and that a very heavy one, upon the
whole community in general, and that without taking a part in any of the
injuries which came in consequence to be offered to particular
individuals.\\
Next with regard to the mischief which an offence may bring upon the
public by its influence on the operations of the government. This it may
occasion either,\\
1. In a more immediate way, by its influence on those \emph{operations}
themselves:\\
2. In a more remote way, by its influence on the \emph{instruments} by
or by the help of which those operations should be performed: or\\
3. In a more remote way still, by its influence on the \emph{sources}
from whence such instruments are to be derived.\\
First then, as to the operations of government, the tendency of these,
in as far as it is conformable to what on the principle of utility it
ought to be, is in every case either to avert mischief from the
community, or to make an addition to the sum of positive good. Now
mischief, we have seen, must come either from external adversaries, from
internal adversaries, or from calamities. With regard to mischief from
external adversaries, there requires no further division. As to mischief
from internal adversaries, the expedients employed for averting it may
be distinguished into such as may be applied \emph{before} the discovery
of any mischievous design in particular, and such as cannot be employed
but in consequence of the discovery of some such design: the former of
these are commonly referred to a branch which may be styled the
\emph{preventive} branch of the \emph{police:} the latter to that of
justice.\\
Secondly, As to the \emph{instruments} which government, whether in the
averting of evil or in the producing of positive good, can have to work
with, these must be either \emph{persons} or \emph{things.} Those which
are destined to the particular function of guarding against mischief
from adversaries in general, but more particularly from external
adversaries, may be distinguished from the rest under the collective
appellation of the \emph{public military force,} and, for conciseness'
sake, the \emph{military force.} The rest may be characterized by the
collective appellation~ of the \emph{public wealth.}\\
Thirdly, with regard to the sources or funds from whence these
instruments, howsoever applied, must be derived, such of them as come
under the denomination of \emph{persons} must be taken out of the whole
number of persons that are in the community, that is, out of the total
\emph{population} of the state: so that the greater the population, the
greater may \emph{cæteris paribus} be this branch of the public wealth;
and the less, the less. In like manner, such as come under the
denomination of \emph{things} may be, and most of them commonly are,
taken out of the sum total of those things which are the separate
properties of the several members of the community: the sum of which
properties may be termed \emph{the national wealth} so that the greater
the national wealth, the greater \emph{cæteris paribus} may be this
remaining branch of the public wealth; and the less, the less. It is
here to be observed, that if the influence exerted on any occasion by
any individual over the operations of the government be pernicious, it
must be in one or other of two ways:\\
1. By causing, or tending to cause, operations \emph{not} to be
performed which \emph{ought} to be performed; in other words, by
\emph{impeding} the operations of government. Or,\\
2. By causing operations to \emph{be} performed which ought \emph{not}
to be performed; in other words, by \emph{misdirecting} them.\\
Lastly, to the total assemblage of the persons by whom the several
political operations above mentioned come to be performed, we set out
with applying the collective appellation of the \emph{government.} Among
these persons there \emph{commonly} is some one person, or body of
persons whose office it is to assign and distribute to the rest their
several departments, to determine the conduct to be pursued by each in
the performance of the particular set of operations that belongs to him,
and even upon occasion to exercise his function in his stead. Where
there is any such person, or body of persons, \emph{he} or \emph{it}
may, according as the turn of the phrase requires, be termed \emph{the
sovereign,} or the \emph{sovereignty.} Now it is evident, that to impede
or misdirect the operations of the sovereign, as here described, may be
to impede or misdirect the operations of the several departments of
government as described above.

From this analysis, by which the connection between the several
above-mentioned heads of offences is exhibited, we may now collect a
definition for each article. By \emph{offences against external
security,} we may understand such offences whereof the tendency is to
bring upon the public a mischief resulting from the hostilities of
foreign adversaries. By \emph{offences against justice,} such offences
whereof the tendency is to impede or misdirect the operations of that
power which is employed in the business of guarding the public against
the mischiefs resulting from the delinquency of internal adversaries, as
far as it is to be done by expedients, which do not come to be applied
in any case till \emph{after} the discovery of some particular design of
the sort of those which they are calculated to prevent. By
\emph{offences against the preventive branch of the police,} such
offences whereof the tendency is to impede or misdirect the operations
of that power which is employed in guarding against mischiefs resulting
from the delinquency of internal adversaries, by expedients that come to
be applied \emph{beforehand;} or of that which is employed in guarding
against the mischiefs that might be occasioned by physical calamities.
By \emph{offences against the public force,} such offences whereof the
tendency is to impede or misdirect the operations of that power which.
destined to guard the public from the mischiefs which may result from
the hostility of foreign adversaries, and, in case of necessity, in the
capacity of ministers of justice, from mischiefs of the number of those
which result from the delinquency of internal adversaries. By
\emph{offences against the increase of the national felicity,} such
offences whereof the tendency is to impede or misapply the operations of
those powers that are employed in the conducting of various
establishments, which are calculated to make, in so many different ways,
a positive addition to the stock of public happiness.\\
By \emph{offences against the public wealth,} such offences whereof the
tendency is to diminish the amount or misdirect the application of the
money, and other articles of wealth, which the government reserves as a
fund, out of which the stock of instruments employed in the service
above mentioned may be kept up. By \emph{offences against population,}
such offences whereof the tendency is to diminish the numbers or impair
the political value of the sum total of the members of the community. By
\emph{offences against the national wealth,} such offences whereof the
tendency is to diminish the quantity, or impair the value, of the things
which compose the separate properties or estates of the several members
of the community.

XVIII. In this deduction, it may be asked, what place is left for
religion. This we shall see presently. For combating the various kinds
of offences above enumerated, that is, for combating all the offences
(those not excepted which we are now about considering) which it is in
man's nature to commit, the state has two great engines,
\emph{punishment} and \emph{reward:} punishment, to be applied to all,
and upon all ordinary occasions: reward, to be applied to a few, for
particular purposes, and upon extraordinary occasions. But whether or no
a man has done the act which renders him an object meet for punishment
or reward, the eyes of those, whosoever they be, to whom the management
of these engines is entrusted cannot always see, nor, where it is
punishment that is to be administered, can their hands be always sure to
reach him. To supply these deficiencies in point of power, it is thought
necessary, or at least \emph{useful} (without which the \emph{truth} of
the doctrine would be nothing to the purpose), to inculcate into the
minds of the people the belief of the existence of a power applicable to
the same purposes, and not liable to the same deficiencies: the power of
a supreme invisible being, to whom a disposition of contributing to the
same ends to which the several institutions already mentioned are
calculated to contribute, must for this purpose be ascribed. It is of
course expected that this power will, at one time or other, be employed
in the promoting of those ends: and to keep up and strengthen this
expectation among men, is spoken of as being the employment of a kind of
allegorical personage, feigned, as before, for convenience of discourse,
and styled \emph{religion.} To diminish, then, or misapply the influence
of religion, is \emph{pro tanto} to diminish or misapply what power the
state has of combating with effect any of the before enumerated kinds of
offences; that is, all kinds of offences whatsoever. Acts that appear to
have this tendency may be styled \emph{offences against religion.} Of
these then may be composed the tenth division of the class of offences
against the state,

XIX. If there be any acts which appear liable to affect the state in any
one or more of the above ways, by operating in prejudice of the external
security of the state, or of its internal security; of the public force;
of the increase of the national felicity; of the public wealth; of the
rational population; of the national wealth; of the sovereignty; or of
religion; at the same time that it is not clear in which of all these
ways they will affect it most, nor but that, according to contingencies,
they may affect it in one of these ways only or in another; such acts
may be collected together under a miscellaneous division by themselves,
and styled \emph{offences against the national interest in general.} Of
these then may be composed the eleventh and last division of the class
of offences against the state.

XX. We come now to class the fifth: consisting of \emph{multiform}
offences. These, as has been already intimated, are either. offences by
\emph{falsehood,} or offences concerning \emph{trust.} Under the head of
offences by falsehood, may be comprehended,\\
1. Simple falsehoods.\\
2. Forgery.\\
3. Personation.\\
4. Perjury.\\
Let us observe in what particulars these four kinds of falsehood agree,
and in what they differ.

XXI. Offences by falsehood, however diversified in other particulars,
have this in common, that they consist in some abuse of the faculty of
discourse, or rather, as we shall see hereafter, of the faculty of
influencing the sentiment of belief in other men, whether by discourse
or otherwise. The use of discourse is to influence belief, and that in
such manner as to give other men to understand that things are as they
are really. Falsehoods, of whatever kind they be, agree in this: that
they give men to understand that things are otherwise than as in reality
they are.

XXII. Personation, forgery, and perjury, are each of them distinguished
from other modes of uttering falsehood by certain special circumstances.
When a falsehood is not accompanied by any of those circumstances, it
maybe styled simple falsehood. These circumstances are,\\
1. The \emph{form} in which the falsehood is uttered.\\
2. The circumstance of its relating or not to the identity of the
\emph{person} of him who utters it.\\
3. The solemnity of the \emph{occasion} on which it is uttered. The
particular application of these distinctive characters may more
commodiously be reserved for another place.

XXIII. We come now to the sub-divisions of offences by falsehood. These
will bring us back into the regular track of analysis, pursued, without
deviation, through the four preceding classes.\\
By whatever means a mischief is brought about, whether falsehood be or
be not of the number, the individuals liable to be affected by it must
either be assignable or unassignable. If assignable, there are but four
material articles in respect to which they can be affected: to wit,
their persons, their properties, their reputations, and their conditions
in life. The case is the same, if, though unassignable, they are
comprisable in any class subordinate to that which is composed of the
whole number of members of the state. If the falsehood tend to the
detriment of the whole state, it can only be by operating in one or
other of the characters, which every act that is an offence against the
state must assume; viz., that of an offence against external Security,
against justice, against the preventive branch of the police, against
the public force, against the increase of the national felicity, against
the public wealth, against the national population, against the national
wealth, against the sovereignty of the state, or against its religion.

XXIV. It is the common property, then, of the offences that belong to
this division, to run over the same ground that is occupied by those of
the preceding classes. But some of them, as we shall see, are apt, on
various occasions, to drop or change the names which bring them under
this division: this is chiefly the ease with regard to simple
falsehoods. Others retain their names unchanged; and even thereby
supersede the names which would otherwise belong to the offences which
they denominate: this is chiefly the case with regard to personation,
forgery, and perjury. When this circumstance then, the circumstance of
falsehood, intervenes, in some cases the name which takes the lead is
that which indicates the offence by its effect; in other cases, it is
that which indicates the expedient or instrument as it were by the help
of which the offence is committed. Falsehood, take it by itself,
consider it as not being accompanied by any other material
circumstances, nor therefore productive of any material effects, can
never, upon the principle of utility, constitute any offence at all.
Combined with other circumstances, there is scarce any sort of
pernicious effect which it may not be instrumental in producing. It is
therefore rather in compliance with the laws of language, than in
consideration of the nature of the things themselves, that falsehoods
are made separate mention of under the name and in the character of
distinct offences. All this would appear plain enough, if it were now a
time for entering into particulars: but that is what cannot be done,
consistently with any principle of order or convenience, until the
inferior divisions of those other classes shall have been previously
exhibited.

XXV. We come now to offences against trust. A trust is, where there is
any particular act which one party, in the exercise of some
\emph{power,} or some \emph{right,} which is conferred on him, is bound
to perform for the benefit of another. Or, more fully, thus: A party is
said to be invested with a trust, when, being invested with a
\emph{power,} or with a \emph{right,} there is a certain behaviour
which, in the exercise of that power, or of that right, he is bound to
maintain for the benefit of some other party. In such case, the party
first mentioned is styled a trustee: for the other party, no name has
ever yet been found: for want of a name, there seems to be no other
resource than to give a new and more extensive sense to the word
\emph{beneficiary,} or to say at length \emph{the party to be
benefited.} \emph{\\
}The trustee is also said to have a trust \emph{conferred} or
\emph{imposed} upon him, to be \emph{invested} with a trust, to have had
a trust given him to execute, to perform, to discharge, or to fulfil.
The party to be benefited, is said to have a trust established or
created in his favour: and so on through a ariety of other phrases.

XXVI. Now it may occur, that a \emph{trust} is oftentimes spoken of as a
species of \emph{condition:} that a trust is also spoken of as a species
of \emph{property:} and that a condition itself is also spoken of same
light. It may be thought, therefore, that in the first class, the
division of offences against condition should have been included under
that of the offences against property: and that at any rate, so much of
the fifth class now before us as contains offences against trust, should
have been included under one or other of those two divisions of the
first class. But upon examination it will appear, that no one of these
divisions could with convenience, nor even perhaps with propriety, have
been included under either of the other two. It will appear at the same
time, that there is an intimate connection subsisting amongst them all:
insomuch that of the lists of the offences to which they are
respectively exposed, any one may serve in great measure as a model for
any other. There are certain offences to which all trusts as such are
exposed: to all these offences every sort of condition will be found
exposed: at the same time that particular species of the offences
against trust will, upon their application to particular conditions,
receive different particular denominations. It will appear also, that of
the two groups of offences into which the list of those against trust
will be found naturally to divide itself, there is one, and but one, to
which property, taken in its proper and more confined sense, stands
exposed: and that these, in their application to the subject of
property, will be found susceptible of distinct modifications, to which
the usage of language, and the occasion there is for distinguishing them
in point of treatment, make it necessary to find names.

XXVI. In the first place, as there are, or at least may be (as we shall
see) conditions which are not trusts, so there are trusts of which the
idea would not be readily and naturally understood to be included under
the word \emph{condition:} add to which, that of those conditions which
do include a trust, the greater number include other ingredients along
with it: so that the idea of a condition, if on the one hand it
stretches beyond the idea of a trust, does on the other hand fall short
of it. Of the several sorts of trusts, by far the most important are
those in which it is the public that stands in the relation of
\emph{beneficiary.} Now these trusts, it should seem, would hardly
present themselves at first view upon the mention of the word
\emph{condition.} At any rate, what is more material, the most important
of the offences against these kinds of trust would not seem to be
included under the denomination of offences against condition. The
offences which by this latter appellation would be brought to view,
would be such only as seemed to affect the interests of an individual:
of him, for example, who is considered as being invested with that
condition. But in offences against public trust, it is the influence
they have on the interests of the public that constitutes by much the
most material part of their pernicious tendency: the influence they have
on the interests of any individual, the only part of their influence
which would be readily brought to view by the appellation of offences
against condition, is comparatively as nothing. The word trust directs
the attention at once to the interests of that party for whom the person
in question is trustee: which party, upon the addition of the epithet
public, is immediately understood to be the body composed of the whole
assemblage, or an indefinite portion of the whole assemblage of the
members of the state. The idea presented by the words \emph{public
trust} is clear and unambiguous: it is but an obscure and ambiguous garb
that that idea could be expressed in by the words \emph{public
condition.} It appears, therefore, that the principal part of the
offences, included under the denomination of offences against trust,
could not, commodiously at least have been included under the head of
offences against condition.

XXVI. It is evident enough, that for the same reasons neither could they
have been included under the head of offences against property. It would
have appeared preposterous, and would have argued a total inattention to
the leading principle of the whole work, the principle of utility, to
have taken the most mischievous and alarming part of the offences to
which the public stands exposed, and forced them into the list of
offences against the property of an individual: of that individual, to
wit, who in that case would be considered as having in him the property
of that public trust, which by the offences in question is affected.\\
Nor would it have been less improper to have included conditions, all of
them, under the head of property: and thereby the whole catalogue of
offences against condition, under the catalogue of offences against
property. True it is, that there are offences against condition, which
perhaps with equal propriety, and without any change in their nature,
might be considered in the light of offences against property: so
extensive and so vague are the ideas that are wont to be annexed to both
these objects. But there are other offences which though with
unquestionable propriety they might be referred to the head of offences
against condition, could not, without the utmost violence done to
language, be forced under the appellation of offences against property.
Property, considered with respect to the proprietor, implies invariably
a benefit, and nothing else: whatever obligations or burthens may, by
accident, stand annexed to it, yet in itself it can never be otherwise
than beneficial. On the part of the proprietor, it is created not by any
commands that are laid on him, but by his being left free to do with
such or such an article as he likes. The obligations it is created by,
are in every instance laid upon other people. On the other hand, as to
conditions, there are several which are of a mixed nature, importing as
well a burthen to him who stands invested with them as a benefit: which
indeed is the case with those conditions which we hear most of under
that name, and which make the greatest figure.\\
There are even conditions which import nothing but burthen, without any
spark of benefit. Accordingly, when between two parties there is such a
relation, that one of them stands in the place of an object of property
with respect to the other; the word \emph{property} is applied only on
one side; but the word \emph{condition} is applied alike to both: it is
but one of them that is said on that account to be possessed of
property; but both of them are alike spoken of as being possessed of or
being invested with a condition: it is the master alone that is
considered as possessing a property, of which the servant, in virtue of
the services he is bound to render, is the object: but the servant, not
less than the master, is spoken of as possessing or being invested with
a condition.\\
The case is, that if a man's condition is ever spoken of as constituting
an article of his \emph{property,} it is in the same loose and
indefinite sense of the word in which almost every other offence that
could be imagined might be reckoned into the list of offences against
property. If the language indeed were in every instance, in which it
made use of the phrase, \emph{object of property,} perspicuous enough to
point out under that appellation the material and really existent body,
the \emph{person} or the \emph{thing} in which those acts terminate, by
the performance of which the property is said to be \emph{enjoyed;} if,
in short, in the import given to the phrase \emph{object of property,}
it made no other use of it than the putting it to signify what is now
called a corporeal \emph{object,} this difficulty and this confusion
would not have occurred. But the import of the phrase \emph{object of
property,} and in consequence the import of the word \emph{property,}
has been made to take a much wider range. In almost every case in which
the law does any thing for a man's benefit or advantage, men are apt to
speak of it, on some occasion or other, as conferring on him a sort of
property. At the same time, for one reason or other, it has in several
cases been not practicable, or not agreeable, to bring to view, under
the appellation of \emph{the object of his property,} the thing in which
the acts, by the performance of which the property is said to be
enjoyed, have their termination, or the person in whom they have their
commencement. Yet something which could be spoken of under that
appellation was absolutely requisite. The expedient then has been to
create, as it were, on every occasion, an ideal being, and to assign to
a man this ideal being for the object of his property: and these are the
sort of objects to which men of science, in taking a view of the
operations of the law in this behalf, came, in process of time, to give
the name of \emph{incorporeal.} Now of these incorporeal objects of
property the variety is prodigious. Fictitious entities of this kind
have been fabricated almost out of every thing: not \emph{conditions}
only (that of a trustee included), but even \emph{reputation} have been
of the number. Even \emph{liberty} has been considered in this same
point of view: and though on so many occasions it is contrasted with
\emph{property,} yet on other occasions, being reckoned into the
catalogue of possessions, it seems to have been considered as a branch
of property. Some of these applications of the words \emph{property,
object of property} (the last, for instance), are looked upon, indeed,
as more figurative, and less proper than the rest: but since the truth
is, that where the immediate object is incorporeal, they are all of them
improper, it is scarce practicable any where to draw the line.\\
Notwithstanding all this latitude, yet, among the relations in virtue of
which you are said to be possessed of a condition, there is one at least
which can scarcely, by the most forced construction, be said to render
any other man, or any other thing, the object of your property. This is
the right of persevering in a certain course of action; for instance, in
the exercising of a certain trade. Now to confer on you this right, in a
certain degree at least, the law has nothing more to do than barely to
abstain from forbidding you to exercise it. Were it to go farther, and,
for the sake of enabling you to exercise your trade to the greater
advantage, prohibit others from exercising the like, then, indeed,
persons might be found, who in a certain sense, and by a construction
rather forced than otherwise, might be spoken of as being the objects of
your property: viz., by being made to render you that sort of negative
service which consists in the forbearing to do those acts which would
lessen the profits of your trade. But the ordinary right of exercising
any such trade or profession, as is not the object of a monopoly,
imports no such thing; and yet, by possessing this right, a man is said
to possess a condition: and by forfeiting it, to forfeit his
condition.\\
After all, it will be seen, that there must be cases in which, according
to the usage of language, the same offence may, with more or less
appearance of propriety, be referred to the head of offences against
condition, or that of offences against property, indifferently. In such
cases the following rule may serve for drawing the line. Wherever, in
virtue of your possessing a property, or being the object of a property
possessed by another you are characterised, according to the usage of
language, by a particular name, such as master, servant, husband, wife,
steward, agent, attorney, or the like, there the word \emph{condition}
may be employed in exclusion of the word \emph{property:} and an offence
in which, in virtue of your bearing such relation, you are concerned,
either in the capacity of an offender, or in that of a party injured,
may be referred to the head of offences against condition, and not to
that of offences against property. To give an example: Being bound, in
the capacity of land steward to a certain person, to oversee the
repairing of a certain bridge, you forbear to do so: in this case, as
the services you are bound to render are of the number of those which
give occasion to the party, from whom they are due, to be spoken of
under a certain generical name, viz., that of land steward, the offence
of withholding them may be referred to the class of offences against
condition. But suppose that, without being engaged in that general and
miscellaneous course of service, which with reference to a particular
person would denominate you his land steward, you were bound, whether by
usage or by contract, to render him that single sort of service, you
stand aggregated (for that of architect, mason, or the like, is not here
in question), the offence you commit by withholding such service cannot
with propriety be referred to the class of offences against condition:
it can only therefore be referred to the class of offences against
property.\\
By way of further distinction, it may be remarked, that where a man, in
virtue of his being bound to render, or of others being bound to render
him, certain services, is spoken of as possessing a condition, the
assemblage of services is generally so considerable, in point of
duration, as to constitute a course of considerable length, so as on a
variety of occasions to come to be varied and repeated: and in most
cases, when the condition is not of a domestic nature, sometimes for the
benefit of one person, sometimes for that of another. Services which
come to be rendered to a particular person on a particular occasion,
especially if they be of short duration, have seldom the effect of
occasioning either party to be spoken of as being invested with a
condition. The particular occasional services which one man may come, by
contract or otherwise, to be bound to render to another, are innumerably
various: but the number of conditions which have names may be counted,
and are, comparatively, but few.

XXVI. If after all, notwithstanding the rule here given for separating
conditions from articles of property, any object should present itself
which should appear to be referable, with equal propriety, to either
head, the inconvenience would not be material; since in such cases, as
will be seen a little farther on, whichever appellation were adopted,
the list of the offences, to which the object stands exposed, would be
substantially the same.\\
These difficulties being cleared up, we now proceed to exhibit an
analytical view of the several possible offences against trust.

XXVII. Offences against trust may be distinguished, in the first place,
into such as concern the existence of the trust in the hands of such or
such a person, and such as concern the exercise of the functions that
belong to it. First then, with regard to such as relate to its
existence. An offence of this description, like one of any other
description, if an offence it ought to be, must to some person or other
import a prejudice. This prejudice maybe distinguished into two
branches:\\
1. That which may fall on such persons as are or should be invested with
the trust:\\
2. That which may fall on the persons for whose sake it is or should be
instituted, or on other persons at large. To begin with the former of
these branches. Let any trust be conceived. The consequences which it is
in the nature of it to be productive of to the possessor, must, in as
far as they are \emph{material,} be either of an advantageous or of a
disadvantageous nature: in as far as they are advantageous, the trust
may be considered as a \emph{benefit} or privilege: in as far as they
are disadvantageous, it may be considered as a \emph{burthen.} To
consider it then upon the footing of a benefit. The trust either is of
the number of those which ought by law to subsist; that is, which the
legislator meant should be established; or is not. If it is, the
possession which at any time you may be deprived of, with respect to it,
must at that time be either present or to come: if to come (in which
case it maybe regarded either as certain or as contingent), the
investitive event, or event from whence your possession of it should
have taken its commencement, was either an event in the production of
which the will of the offender should have been instrumental, or any
other event at large: in the former case, the offence may be termed
\emph{wrongful non-investment of trust:} in the latter case,
\emph{wrongful interception of trust.} If at the time of the offence
whereby you are deprived of it, you were already in possession of it,
the offence may be styled \emph{wrongful divestment of trust.} In any of
these cases, the effect of the offence is either to put somebody else
into the trust, or not: if not, it is wrongful divestment, wrongful
interception, or wrongful divestment, and nothing more: if it be, the
person put in possession is either the wrong-doer himself, in which case
it may be styled \emph{usurpation of trust;} or some other person, in
which case it may be styled \emph{wrongful investment,} or attribution,
\emph{of trust.} If the trust in question is \emph{not} of the number of
those which ought to subsist, it depends upon the manner in which one
man deprives another of it, whether such deprivation shall or shall not
be an offence, and, accordingly, whether non-investment, interception,
or divestment, shall or shall not be wrongful. But the putting any body
into it must at any rate be an offence: and this offence may be either
usurpation or wrongful investment, as before.\\
In the next place, to consider it upon the footing of a burthen. In this
point of view, if no other interest than that of the persons liable to
be invested with it were considered, it is what ought not, upon the
principle of utility, to subsist: if it ought, it can only be for the
sake of the persons in whose favour it is established. If then it ought
not on any account to subsist, neither non-investment, interception, nor
divestment, can be wrongful with relation to the persons first
mentioned, whatever they may be on any other account, in respect of the
manner in which they happen to be performed: for usurpation, though not
likely to be committed, there is the same room as before: so likewise is
there for wrongful investment; which, in as far as the trust is
considered as a burthen, may be styled wrongful imposition of trust. If
the trust, being still of the burthensome kind, is of the number of
those which \emph{ought} to subsist, any offence that can be committed,
with relation to the existence of it, must consist either in causing a
person to \emph{be} in possession of it, who ought \emph{not} to be, or
in causing a person \emph{not} to be in possession of it who
\emph{ought} to be: in the former case, it must be either usurpation or
wrongful divestment, as before: in the latter case, the person who is
caused to be not in possession, is either the wrong-doer himself, or
some other: if the wrong-doer himself, either at the time of the offence
he was in possession of it, or he was not: if he was, it may be termed
\emph{wrongful abdication of trust;} if not, \emph{wrongful
detrectation} or \emph{non-assumption:} if the person, whom the offence
causes not to be in the trust, is any other person, the offence must be
either wrongful divestment, wrongful non-investment, or wrongful
interception, as before: in any of which cases to consider the trust in
the light of a burthen, it might also be styled \emph{wrongful exemption
from trust.}\\
\emph{}Lastly, with regard to the prejudice which the persons for whose
benefit the trust is instituted, or any other persons whose interests
may come to be affected by its existing or not existing in such or such
hands, are liable to sustain. Upon examination it will appear, that by
every sort of offence whereby the persons who are or should be in
possession of it are liable, in that respect, to sustain a prejudice,
the persons now in question are also liable to sustain a prejudice. The
prejudice, in this case, is evidently of a very different nature from
what it was of in the other: but the same general names will be
applicable in this case as in that. If the beneficiaries, or persons
whose interests are at stake upon the exercise of the trust, or any of
them, are liable to sustain a prejudice, resulting from the quality of
the person by whom it may be filled, such prejudice must result from the
one or the other of two causes:\\
1. From a person's having the possession of it who ought not to have it:
or\\
2. From a person's not having it who ought: whether it be a benefit or
burthen to the possessor, is a circumstance that to this purpose makes
no difference. In the first of these cases the offences from which the
prejudice takes its rise are those of usurpation of trust, wrongful
attribution of trust, and wrongful imposition of trust: in the latter,
wrongful non-investment of trust, wrongful interception of trust,
wrongful divestment of trust, wrongful abdication of trust, and wrongful
detrectation of trust.

So much for the offences which concern the existence or possession of a
trust: those with concern the exercise of the functions that belong to
it may be thus conceived. You are in possession of a trust: the time
then for your acting in it must, on any given occasion, (neglecting, for
simplicity's sake, the then present instant) be either past or yet to
come. If past, your conduct on that occasion must have been either
conformable to the purposes for which the trust was instituted, or
uncomformable: if comformable, there has been no mischief in the case:
if unconformable, the fault has been either in yourself alone, or in
some other person, or in both: in as far as it has lain in yourself, it
has consisted either in your \emph{not} doing something which you ought
to do, in which case it may be styled \emph{negative breach of trust};
or in your \emph{doing} something which you ought \emph{not} to do: if
in the doing something which you ought not to do, the party to whom the
prejudice has accrued is either the same for whose benefit the trust was
instituted, or in some other party at large: in the former of these
cases, the offence may be styled \emph{positive breach of trust.}
Supposing the time for your acting in the trust to be yet to come, the
effect of any act which tends to render it actually and eventually
unconformable, or to produce a chance of its being so. IN the former of
these cases, it can do no otherwise than take one or other of the shapes
that have just been mentioned. In the latter case, the blame must lie
either in yourself alone, or in some other person, or in both together,
as before. If in another person, the acts whereby he may tend to render
your conduct unconformable, must be exercised either on yourself, or on
other objects at large. If exercised on yourself, the influence they
possess must either be such as operates immediately on your body, or
such as operates immediately on your mind. In the latter case, again,
the tendency of them must be to deprive you either of the knowledge, or
of the power, or of the inclination, which would be necessary to your
maintaining such a conduct as shall be conformable to the purposes in
question. If they be such, of which the tendency is to deprive you of
the inclination in question, it must be by applying to your will the
force of some \emph{seducing} motive.\\
Lastly, This motive must be either of the \emph{coercive,} or of the
\emph{alluring} kind; in other words, it must present itself either in
the shape of a mischief or of an advantage. Now in none of all the cases
that have been mentioned, except the last, does the offence receive any
new denomination; according to the event it is either a disturbance of
trust, or an abortive attempt to be guilty of that offence. In this last
it is termed \emph{bribery;} and it is that particular species of it
which may be termed \emph{active} bribery, or \emph{bribe-giving.} In
this case, to consider the matter on your part, either you accept of the
bribe, or you do not: if not, and you do not afterwards commit, or go
about to commit, either a breach or an abuse of trust, there is no
offence, on your part, in the case: if you do accept it, whether you
eventually do or do not commit the breach or the abuse which it is the
bribe-giver's intention you should commit, you at any rate commit an
offence which is also termed bribery: and which, for distinction sake,
may be termed \emph{passive} bribery, or \emph{bribe-taking.} As to any
farther distinctions, they will depend upon the nature of the particular
sort of trust in question, and therefore belong not to the present
place.

And thus we have thirteen sub-divisions of offences against trust:
viz.,\\
1. Wrongful non-investment of trust.\\
2. Wrongful interception of trust.\\
3. Wrongful divestment of trust.\\
4. Usurpation of trust.\\
5. Wrongful investment or attribution of trust.\\
6. Wrongful abdication of trust.\\
7. Wrongful detrectation of trust.\\
8. Wrongful imposition of trust.\\
9. Negative breach of trust.\\
10. Positive breach of trust.\\
11. Abuse of trust.\\
12. Disturbance of trust.\\
13. Bribery.

XXVIII. From what has been said, it appears that there cannot be any
other offences, on the part of a trustee, by which a \emph{beneficiary}
can receive on any particular occasion any assignable specific
prejudice. One sort of acts, however, there are by which a trustee may
be put in some \emph{danger} of receiving a prejudice, although neither
the nature of the prejudice, nor the occasion on which he is in danger
of receiving it, should be assignable. These can be no other than such
acts, whatever they may be, as dispose the trustee to be acted upon by a
given bribe with greater effect than any with which he could otherwise
be acted upon: or in other words, which place him in such circumstances
as have a tendency to increase the quantum of his sensibility to the
action of any motive of the sort in question. Of these acts, there seem
to be no others, that will admit of a description applicable to all
places and times alike, than acts of \emph{prodigality} on the part of
the trustee. But in acts of this nature the prejudice to the
\emph{beneficiary} is contingent only and unliquidated; while the
prejudice to the trustee himself is certain and liquidated. If therefore
on any occasion it should be found advisable to treat it on the footing
of an offence, it will find its place more naturally in the class of
self-regarding ones.

XXIX. As to the subdivisions of offences against trust, these are
perfectly analogous to those of offences by falsehood. The trust may be
private, semi-public, or public: it may concern property, person,
reputation, or condition; or any two or more of those articles at a
time: as will be more particularly explained in another place. Here too
the offence, in running over the ground occupied by the three prior
classes, will in some instances change its name, while in others it will
not.

XXX. Lastly, if it be asked, What sort of relation there subsists
between falsehoods on one hand, and offences concerning trust on the
other hand; the answer is, they are altogether disparate. Falsehood is a
circumstance that may enter into the composition of any sort of offence,
those concerning trust, as well as any other: in some as an accidental,
in others as an essential instrument. Breach or abuse of trust are
circumstances which, in the character of accidental concomitants, may
enter into the composition of any other offences (those against
falsehood included) besides those to which they respectively give
name.\\

§3. Genera of Class I

XXXI. Returning now to class the first, let us pursue the distribution a
step farther, and branch out the several divisions of that class, as
above exhibited, into their respective \emph{genera,} that is, into such
minuter divisions as are capable of being characterised by denominations
of which a great part are already current among the people. In this
place the analysis must stop. To apply it in the same regular form to
any of the other classes seems scarcely practicable: to semi-public, as
also to public offences, on account of the interference of local
circumstances: to self-regarding ones, on account of the necessity it
would create of deciding prematurely upon points which may appear liable
to controversy: to offences by falsehood, and offences against trust, on
account of the dependence there is between this class and the three
former. What remains to be done in this way, with reference to these
four classes, will require discussion, and will therefore be introduced
with more propriety in the body of the work, than in a preliminary part,
of which the business is only to draw outlines.

XXXII. An act, by which the happiness of an individual is disturbed, is
either \emph{simple} in its effects or \emph{complex.} It may be styled
simple in its effects, when it affects him in one only of the articles
or points in which his interest, as we have seen, is liable to be
affected: complex, when it affects him in several of those points at
once. Such as are simple in their effects must of course be first
considered.

XXXIII. In a simple way, that is in one way at a time, a man's happiness
is liable to be disturbed either\\
1. By actions referring to his own person itself; or\\
2. By actions referring to such external objects on which his happiness
is more or less dependent. As to his own person, it is composed of two
different parts, or reputed parts, his body and his mind. Acts which
exert a pernicious influence on his person, whether it be on the
corporeal or on the mental part of it, will operate thereon either
immediately, and without affecting his will, or mediately, through the
intervention of that faculty: viz., by means of the influence which they
cause his will to exercise over his body. If with the intervention of
his will, it must be by \emph{mental coercion:} that is, by causing him
to \emph{will} to maintain, and thence actually to maintain, a certain
conduct which it is disagreeable, or in any other way pernicious, to him
to maintain. This conduct may either be positive or negative: when
positive, the coercion is styled \emph{compulsion} or \emph{constraint:}
when negative, \emph{restraint.} Now the way in which the coercion is
disagreeable to him, may be by producing either pain of body, or only
pain of mind. If pain of body is produced by it, the offence will come
as well under this as under other denominations, which we shall come to
presently. Moreover, the conduct which a man, by means of the coercion,
is forced to maintain, will be determined either specifically and
originally, by the determination of the particular acts themselves which
he is forced to perform or to abstain from, or generally and
incidentally, by means of his being forced to be or not to be in such or
such a place. But if he is prevented from being in one place, he is
confined thereby to another. For the whole surface of the earth, like
the surface of any greater or lesser body, may be conceived to be
divided into two, as well as into any other number of parts or spots. If
the spot then, which he is confined to, be smaller than the spot which
he is excluded from, his condition may be called \emph{confinement:} if
larger, \emph{banishment.} Whether an act, the effect of which is to
exert a pernicious influence on the person of him who suffers by it
operates with or without the intervention of an act of his will, the
mischief it produces will either be \emph{mortal} or \emph{not mortal.}
If not mortal, it will either be \emph{reparable,} that is temporary, or
\emph{irreparable,} that is perpetual. If reparable, the mischievous act
may be termed a \emph{simple corporal injury;} if irreparable, an
\emph{irreparable corporal injury.} Lastly, a pain that a man
experiences in his mind will either be a pain of actual
\emph{sufferance,} or a pain of \emph{apprehension.} If a pain of
apprehension, either the offender himself is represented as intending to
bear a part in the production of it, or he is not. In the former case
the offence may be styled \emph{menacement:} in the latter case, as also
where the pain is a pain of actual sufferance, a \emph{simple mental
injury.}

And thus we have nine genera or kinds of personal injuries; which, when
ranged in the order most commodious for examination, will stand as
follows; viz.,\\
1. Simple corporal injuries.\\
2. Irreparable corporal injuries.\\
3. Simple injurious restrainment.\\
4. Simple injurious compulsion.\\
5. Wrongful confinement.\\
6. Wrongful banishment.\\
7. Wrongful homicide.\\
8. Wrongful menacement.\\
9. Simple mental injuries.

XXXIV. We come now to offences against reputation merely. These require
but few distinctions. In point of reputation there is but one way of
suffering, which is by losing a portion of the good-will of others. Now,
in respect of the good-will which others bear you, you may be a loser in
either of two ways:\\
1. By the manner in which you are thought to behave \emph{yourself;}
and\\
2. By the manner in which \emph{others} behave, or are thought to
behave, towards you.\\
To cause people to think that you yourself have so behaved, as to have
been guilty of any of those acts which cause a man to possess less than
he did before of the good-will of the community, is what may be styled
\emph{defamation.}\\
But such is the constitution of human nature, and such the force of
prejudice, that a man merely by manifesting his own want of good-will
towards you, though ever so unjust in itself, and ever so unlawfully
expressed, may in a manner force others to withdraw from you a part of
theirs. When he does this by words, or by such actions as have no other
effect than in as far as they stand in the place of words, the offence
may be styled \emph{vilification.} When it is done by such actions as,
besides their having this effect, are injuries to the person, the
offence may be styled a \emph{personal insult:} if it has got the length
of reaching the body, a \emph{corporal insult:} if it stopped short
before it reached that length, it may be styled \emph{insulting
menacement.} And thus we have two \emph{genera} or kinds of offences
against reputation merely; to wit,\\
1. Defamation: and,\\
2. Vilification, or Revilement.\\
As to corporal insults, and insulting menacement, they belong to the
compound title of offences against person and reputation both together.

XXXV. If the property of one man suffers by the delinquency of another,
such property either was in trust with the offender, or it was not: if
it was in trust, the offence is a breach of trust, and of whatever
nature it may be in other respects, may be styled \emph{dissipation in
breach of trust,} or \emph{dissipation of property in trust.} This is a
particular case: the opposite one is the more common: in such case the
several ways in which property may, by possibility, become the object of
an offence, may be thus conceived. Offences against property, of
whatever kind it be, may be distinguished, as hath been already
intimated, into such as concern the legal possession of it, or right to
it, and such as concern only the enjoyment of it, or, what is the same
thing, the exercise of that right. Under the former of these heads come,
as hath been already intimated, the several offences of \emph{wrongful
non-investment, wrongful interception, wrongful divestment, usurpation,}
and \emph{wrongful attribution.} When in the commission of any of these
offences a falsehood has served as an instrument, and that, as it is
commonly called, a \emph{wilful,} or as it might more properly be
termed, an \emph{advised} one, the epithet \emph{fraudulent} may be
prefixed to the name of the offence, or substituted in the room of the
word \emph{wrongful.} The circumstance of fraudulency then may serve to
characterise a particular species, comprisable under each of those
generic heads: in like manner the circumstance of \emph{force,} of which
more a little farther on, may serve to characterize another. With
respect to wrongful interception in particular, the \emph{investitive
event} by which the title to the thing in question should have accrued
to you, and for want of which such title is, through the delinquency of
the offender, as it were, \emph{intercepted,} is either an act of his
own, expressing it as his will, that you should be considered by the law
as the person who is legally in possession of it, or it is any other
event at large: in the former case, if the thing, of which you should
have been put into possession, is a sum of money to a certain amount,
the offence is that which has received the name of \emph{insolvency;}
which branch of delinquency, in consideration of the importance and
extent of it, may be treated on the footing of a distinct genus of
itself.\\
Next, with regard to such of the offences against property as concern
only the enjoyment of the object in question. This object must be either
a service, or set of services, which should have been rendered by some
\emph{person,} or else an article belonging to the class of
\emph{things.} In the former ease, the offence may be styled
\emph{wrongful withholding of services.} In the latter case it may admit
of farther modifications, which may be thus conceived: When any object
which you have had the physical occupation or enjoyment of, ceases, in
any degree, in consequence of the act of another man, and without any
change made in so much of that power as depends upon the intrinsic
physical condition of your person, to be subject to that power; this
cessation is either owing to change in the intrinsic condition of the
thing itself, or in its exterior situation with\\
respect to you, that is, to its being situated out of your reach. In the
former case, the nature of the change is either such as to put it out of
your power to make any use of it at all, in which case the thing is said
to be \emph{destroyed,} and the offence whereby it is so treated may be
termed \emph{wrongful destruction:} or such only as to render the uses
it is capable of being put to of less value than before, in which case
it is said to be \emph{damaged,} or to have sustained damage, and the
offence may be termed \emph{wrongful endamagement.} Moreover, in as far
as the value which a thing is of to you is considered as being liable to
be in some degree impaired, by any act on the part of any other person
exercised upon that thing, although on a given occasion no perceptible
damage should ensue, the exercise of any such act is commonly treated on
the footing of an offence, which may be termed \emph{wrongful using} or
\emph{occupation.}\\
If the cause of the thing's failing in its capacity of being of use to
you, lies in the exterior situation of it with relation to you, the
offence may be styled \emph{wrongful detainment.} Wrongful detainment,
or detention, during any given period of time, may either be accompanied
with the intention of detaining the thing for ever (that is for an
indefinite time), or not: if it be, and if it be accompanied at the same
time with the intention of not being amenable to law for what is done,
it seems to answer to the idea commonly annexed to the word
\emph{embezzlement,} an offence which is commonly accompanied with
breach of trust. In the case of wrongful occupation, the physical
faculty of occupying may have been obtained with or without the
assistance or consent of the proprietor, or other person appearing to
have a right to afford such assistance or consent. If without such
assistance or consent, and the occupation be accompanied with the
intention of detaining the thing for ever, together with the intention
of not being amenable to law for what is done, the offence seems to
answer to the idea commonly annexed to the word \emph{theft} or
\emph{stealing.} If in the same circumstances a force is put upon the
body of any person who uses, or appears to be disposed to use, any
endeavours to prevent the act, this seems to be one of the cases in
which the offence is generally understood to come under the name of
\emph{robbery.}\\
If the physical faculty in question was obtained with the assistance or
consent of a proprietor or other person above spoken of, and still the
occupation of the thing is an offence, it may have been either because
the assistance or consent was not fairly or because it was not freely
obtained. If not \emph{fairly} obtained, it was obtained by falsehood,
which, if \emph{advised,} is in such a case termed \emph{fraud:} and the
offence, if accompanied with the intention of not being amenable to law,
may be termed \emph{fraudulent obtainment} or \emph{defraudment.} If not
\emph{freely} obtained, it was obtained by \emph{force:} to wit, either
by a force put upon the body, which has been already mentioned, or by a
force put upon the mind. If by a force put upon the mind, or in other
words, by the application of coercive motives, it must be by producing
the apprehension of some evil: which evil, if the act is an offence,
must be some evil to which on the occasion in question the one person
has no right to expose the other. This is one case in which, if the
offence be accompanied with the intention of detaining the thing for
ever, whether it be or be not accompanied with the intention of not
being amenable to law, it seems to agree with the idea of what is
commonly meant by \emph{extortion.} Now the part a man takes in exposing
another to the evil in question, must be either a positive or a negative
part. In the former case, again, the evil must either be present or
distant. In the case then where the assistance or consent is obtained by
a force put upon the body, or where, if by a force put upon the mind,
the part taken in the exposing a man to the apprehension of the evil is
positive, the evil present, and the object of it his person, and if at
any rate the extortion, thus applied, be accompanied with the intention
of not being amenable to law, it seems to agree with the remaining case
of what goes under the name of \emph{robbery.}\\
As to dissipation in breach of trust, this, when productive of a
pecuniary profit to the trustee, seems to be one species of what is
commonly meant by \emph{peculation.} Another, and the only remaining
one, seems to consist in acts of occupation exercised by the trustee
upon the things which are the objects of the fiduciary property, for his
own benefit, and to the damage of the beneficiary. As to robbery, this
offence, by the manner in which the assistance or consent is obtained,
becomes an offence against property and person at the same time.
Dissipation in breach of trust, and peculation, may perhaps be more
commodiously treated of under the head of offences against trust. After
these exceptions, we have thirteen genera or principal kinds of offences
against property, which, when ranged in the order most commodious for
examination, may stand as follows, viz.,\\
I. Wrongful non-investment of property.\\
2. Wrongful interception of property.\\
3. Wrongful divestment of property.\\
4. Usurpation of\\
5. Wrongful investment of property.\\
6. Wrongful withholding of services.\\
7. Wrongful destruction or endamagement.\\
8. Wrongful occupation.\\
9. Wrongful detainment.\\
10. Embezzlement.\\
11. Theft.\\
12. Defraudment.\\
13. Extortion.

We proceed now to consider offences which are complex in their effects.
Regularly, indeed, we should come to offences against condition; but it
will be more convenient to speak first of offences by which a man's
interest is affected in two of the preceding points at once.

XXXVI. First, then, with regard to offences which affects person and
reputation together. When any man, by a mode of treatment which affects
the person, injures the reputation of another, his end and purpose must
have been either his own immediate pleasure, or that sort of reflected
pleasure, which in certain circumstances may be reaped from the
suffering of another. Now the only immediate pleasure worth regarding,
which any one can reap from the person of another, and which at the same
time is capable of affecting the reputation of the latter, is the
pleasure of the sexual appetites. This pleasure, then, if reaped at all,
must have been reaped either against the consent of the party, or with
consent. If with consent, the consent must have been obtained either
freely and fairly both, or freely but not fairly, or else not even
freely; in which case the fairness is out of the question. If the
consent be altogether wanting, the offence is called \emph{rape:} if not
fairly obtained, \emph{seduction} simply: if not freely, it may be
called \emph{forcible seduction.} In any case, either the offence has
gone the length of consummation, or has stopped short of that period; if
it has gone that length, it takes one or other of the names just
mentioned: if not, it may be included alike in all cases under the
denomination of a \emph{simple lascivious injury.} Lastly, to take the
case where a man injuring you in your reputation, by proceedings that
regard your person, does it for the sake of that sort of pleasure which
will sometimes result from the contemplation of another's pain. Under
these circumstances either the offence has actually gone the length of a
corporal injury, or it has rested in menacement: in the first case it
may be styled a \emph{corporal insult;} in the other, it may come under
the name of \emph{insulting menacement.} And thus we have six genera, or
kind of offences, against person and reputation together; which, when
ranged in the order most commodious for consideration, will stand
thus:\\
1. Corporal insults.\\
2. Insulting menacement.\\
3. Seduction.\\
4. Rape.\\
5. Forcible seduction.\\
6. Simple lascivious injuries.

XXXVII. Secondly, with respect to those which affect person and property
together. That a force put upon the person of a man may be among the
means by which the title to property may be unlawfully taken away or
acquired, has been already stated. A force of this sort then is a
circumstance which may accompany the offences of wrongful interception,
wrongful divestment, usurpation, and wrongful investment. But in these
cases the intervention of this circumstance does not happen to have
given any new denomination to the offence. In all or any of these cases,
however, by prefixing the epithet \emph{forcible,} we may have so many
names of offences, which may either be considered as constituting so
many species of the genera belonging to the division of offences against
property, or as so many genera belonging to the division now before us.
Among the offences that concern the enjoyment of the thing, the case is
the same with wrongful destruction and wrongful endamagement; as also
with wrongful occupation and wrongful detainment. As to the offence of
wrongful occupation, it is only in the case where the thing occupied
belongs to the class of immovables, that, when accompanied by the kind
of force in question, has obtained a particular name which is in common
use: in this case it is called \emph{forcible entry:} forcible
detainment, as applied also to immovables, but only to immovables, has
obtained, among lawyers at least, the name of \emph{forcible detainer.}
And thus we may distinguish ten genera, or kinds of offences, against
person and property together, which, omitting for conciseness' sake the
epithet \emph{wrongful,} will stand thus:\\
1. Forcible interception of property.\\
2. Forcible divestment of property.\\
3. Forcible usurpation.\\
4. Forcible investment.\\
5. Forcible destruction or endamagement.\\
6. Forcible occupation of movables.\\
7. Forcible entry.\\
8. Forcible detainment of movables.\\
9. Forcible detainment of immovables.\\
10. Robbery.

XXXVIII. We come now to offences against \emph{condition.} A man's
condition or station in life is constituted by the legal relation he
bears to the persons who are about him; that is, as we have already had
occasion to show, by \emph{duties,} which, by being imposed on one side,
give birth to \emph{rights} or \emph{powers} on the other. These
relations, it is evident, may be almost infinitely diversified. Some
means, however, may be found of circumscribing the field within which
the varieties of them are displayed. In the first place, they must
either be such as are capable of displaying themselves within the circle
of a private family, or such as require a larger space. The conditions
constituted by the former sort of relations may be styled
\emph{domestic:} those constituted by the latter, \emph{civil.}

XXXIX. As to domestic conditions, the legal relations by which they are
constituted may be distinguished into\\
1. Such as are superadded to relations purely natural: and\\
2. Such as, without any such natural basis, subsist purely by
institution.\\
By relations purely natural, I mean those which may be said to subsist
between certain persons in virtue of the concern which they themselves,
or certain other persons, have had in the process which is necessary to
the continuance of the species. These relations may be distinguished, in
the first place, into contiguous and uncontiguous. The uncontiguous
subsist through the intervention of such as are contiguous. The
contiguous may be distinguished, in the first place, into
\emph{connubial,} and \emph{post-connubial.} Those which may be termed
connubial are two:\\
1. That which the male bears towards the female:\\
2. That which the female bears to the male. The post-connubial are
either \emph{productive} or \emph{derivative.} The productive is that
which the male and female above-mentioned bear each of them towards the
children who are the immediate fruit of their union; this is termed the
relation of \emph{parentality.} Now as the parents must be, so the
children may be, of different sexes. Accordingly the relation of
parentality may be distinguished into four species:\\
1. That which a father bears to his son: this is termed
\emph{paternity.}\\
2. That which a father bears to his daughter: this also is termed
paternity.\\
3. That which a mother bears to her son: this is called
\emph{maternity.}\\
4. That which a mother bears to her daughter: this also is termed
maternity. Uncontiguous natural relations may be distinguished into
\emph{immediate} and \emph{remote.} Such as are immediate, are what one
person bears to another in consequence of their bearing each of them one
simple relation to some third person. Thus the paternal grandfather is
related to the paternal grandson by means of the two different kinds,
which together they bear to the father: the brother on the father's
side, to the brother, by means of the two relations of the same kind,
which together they bear to the father. In the same manner we might
proceed to find places in the system for the infinitely diversified
relations which result from the combinations that may be formed by
mixing together the several sorts of relationships by \emph{ascent,}
relationships by \emph{descent, collateral} relationships, and
relationships by \emph{affinity:} which latter, when the union between
the two parties through whom the affinity takes place is sanctioned by
matrimonial solemnities, are termed relationships by \emph{marriage.}
But this, as it would be a most intricate and tedious task, so happily
is it, for the present purpose, an unnecessary one. The only natural
relations to which it will be necessary to pay any particular attention,
are those which, when sanctioned by law, give birth to the conditions of
husband and wife, the two relations comprised under the head of
parentality, and the corresponding relations comprised under the head of
filiality or filiation.\\
What then are the relations of a legal kind which can be superinduced
upon the above-mentioned natural relations? They must be such as it is
the nature of law to give birth to and establish. But the relations
which subsist purely by institution exhaust, as we shall see, the whole
stock of relationships which it is in the nature of the law to give
birth to and establish. The relations then which can be superinduced
upon those which are purely natural, cannot be in themselves any other
than what are of the number of those which subsist purely by
institution: so that all the difference there can be between a legal
relation of the one sort, and a legal relation of the other sort, is,
that in the former case the circumstance which gave birth to the natural
relation serves as a mark to indicate where the legal relation is to
fix: in the latter case, the place where the legal relation is to attach
is determined not by that circumstance but by some other. From these
considerations it will appear manifestly enough, that for treating of
the several sorts of conditions, as well natural as purely conventional,
in the most commodious order, it will be necessary to give the
precedence to the latter. Proceeding throughout upon the same principle,
we shall all along give the priority, not to those which are first by
nature, but to those which are most simple in point of description.
There is no other way of avoiding perpetual anticipations and
repetitions.

XL. We come now to consider the domestic or family relations, which are
purely of legal institution. It is to these in effect, that both kinds
of domestic conditions, considered as the work of law, are indebted for
their origin. When the law, no matter for what purpose, takes upon
itself to operate, in a matter in which it has not operated before, it
can only be by imposing \emph{obligation.} Now when a legal obligation
is imposed on any man, there are but two ways in which it can in the
first instance be enforced. The one is by giving the power of enforcing
it to the party in whose favour it is imposed: the other is by reserving
that power to certain third persons, who, in virtue of their possessing
it, are styled ministers of justice. In the first case, the party
favoured is said to possess not only a \emph{right} as against the party
obliged, but also a \emph{power} over him: in the second case, a
\emph{right} only, uncorroborated by power. In the first case, the party
favoured may be styled a \emph{superior,} and as they are both members
of the same family, a \emph{domestic superior,} with reference to the
party obliged: who, in the same case, may be styled a \emph{domestic
inferior,} with reference to the party favoured. Now in point of
possibility. it is evident, that domestic conditions, or a kind of
fictitious possession analogous to domestic conditions, might have been
looked upon as constituted, as well by rights alone, without powers on
either side, as by powers. But in point of utility it does not seem
expedient: and in point of fact, probably owing to the invariable
perception which men must have had of the inexpediency, no such
conditions seem ever to have been constituted by such feeble bands. Of
the legal relationships then, which are capable of being made to subsist
within the circle of a family, there remain those only in which the
obligation is enforced by power. Now then, wherever any such power is
conferred, the end or purpose for which it was conferred (unless the
legislator can be supposed to act without a motive) must have been the
producing of a benefit to somebody: in other words, it must have been
conferred for the \emph{sake} of somebody. The person then, for whose
sake it is conferred, must either be one of the two parties just
mentioned, or a third party: if one of these two, it must be either the
superior or the inferior. If the superior, such superior is commonly
called a \emph{master;} and the inferior is termed his \emph{servant:}
and the power may be termed a \emph{beneficial} one. If it be for the
sake of the inferior that the power is established, the superior is
termed a \emph{guardian;} and the inferior his \emph{ward:} and the
power, being thereby coupled with a trust, may be termed a
\emph{fiduciary} one. If for the sake of a third party, the superior may
be termed a \emph{superintendent;} and the inferior his
\emph{subordinate.} This third party will either be an assignable
individual or set of individuals, or a set of unassignable individuals.
In this latter case the trust is either a public or a semi-public one:
and the condition which it constitutes is not of the domestic, but of
the civil kind. In the former case, this third party or
\emph{principal,} as he may be termed, either has a beneficial power
over the superintendent, or he has not: if he has, the superintendent is
his servant, and consequently so also is the subordinate: if not, the
superintendent is the master of the subordinate; and all the advantage
which the principal has over his superintendent, it that of possessing a
set of rights, uncorroborated by power; and therefore, as we have seen,
not fit to constitute a condition of the domestic kind. But be the
condition what it may which is constituted by these rights, of what
nature can the obligations be, to which the superintendent is capable of
being subjected by means of them? They are neither more nor less than
those which a man is capable of being subjected to by powers. It
follows, therefore, that the functions of a principal and his
superintendent coincide with those of a master and his servant; and
consequently that the offences relative to the two former conditions
will coincide with the offences relative to the two latter.

XLI. Offences to which the condition of a master, like any other kind of
condition, is exposed, may, as hath been already intimated be
distinguished into such as concern the existence a of the condition
itself, and such as concern the performance of the functions of it,
while subsisting.\\
First then, with regard to such as affect its existence. It is obvious
enough that the services of one man may be a benefit to another: the
condition of a master may therefore be a beneficial one. It stands
exposed, therefore, to the offences of \emph{wrongful non-investment,
wrongful interception, usurpation, wrongful investment,} and
\emph{wrongful divestment.} But how should it stand exposed to the
offences of \emph{wrongful abdication, wrongful detrectation,} and
\emph{wrongful imposition?} Certainly it cannot of itself; for services,
when a man has the power of exacting them or not, as he thinks fit, can
never be a burthen. But if to the powers, by which the condition of a
master is constituted, the law thinks fit to annex any obligation on the
part of the master; for instance, that of affording maintenance, or
giving wages, to the servant, or paying money to anybody else; it is
evident that in virtue of such obligation the condition \emph{may}
become a burthen. In this case, however, the condition possessed by the
master will not properly speaking, be the pure and simple condition of a
master: it will be a kind of complex object, resolvable into the
beneficial condition of a master, and the burthensome obligation which
is annexed to it. Still however, if the nature of the obligation lies
within a narrow compass, and does not, in the manner of that which
constitutes a trust, interfere with the exercise of those powers by
which the condition of the superior is constituted, the latter,
notwithstanding this foreign mixture, will still retain the name of
mastership. In this case therefore, but not otherwise, the condition of
a master may stand exposed to the offences of \emph{wrongful abdication,
wrongful detrectation,} and \emph{wrongful imposition.} Next as to the
behaviour of persons with reference to this condition, while considered
as subsisting. In virtue of its being a benefit, it is exposed to
\emph{disturbance.} This disturbance will either be the offence of a
stranger, or the offence of the servant himself. Where it is the offence
of a stranger, and is committed by taking the person of the servant, in
circumstances in which the taking of an object belonging to the class of
things would be an act of theft, or (what is scarcely worth
distinguishing from theft) an act of embezzlement: it may be termed
\emph{servant-stealing.} Where it is the offence of the servant himself,
it is styled \emph{breach of duty.} Now the most flagrant species of
breach of duty, and that which includes indeed every other, is that
which consists in the servant's withdrawing himself from the place in
which the duty should be performed.\\
This species of breach of duty is termed \emph{elopement.} Again, in
virtue of the power belonging to this condition, it is liable, on the
part of the master to \emph{abuse.} But this power is not coupled with a
trust. The condition of a master is therefore not exposed to any offence
which is analogous to breach of trust. Lastly, on account of its being
exposed to abuse, it may be conceived to stand, in point of possibility,
exposed to \emph{bribery.} But considering how few, and how
insignificant, the persons are who are liable to be subject to the power
here in question, this is an offence which, on account of the want of
temptation, there will seldom be any example of in practice. We may
therefore reckon thirteen sorts of offences to which the condition of a
master is exposed; viz.,\\
1. Wrongful non-investment of mastership.\\
2. Wrongful interception of mastership.\\
3. Wrongful divestment of mastership.\\
4. Usurpation of mastership.\\
5. Wrongful investment of mastership.\\
6. Wrongful abdication of mastership.\\
7. Wrongful detrectation of mastership.\\
8. Wrongful imposition of mastership.\\
9. Abuse of mastership.\\
10. Disturbance of mastership.\\
11. Breach of duty in servants.\\
12. Elopement of servants.\\
13. Servant-stealing.

XLII. As to the \emph{power} by which the condition of a master is
constituted, this may be either \emph{limited} or \emph{unlimited.} When
it is altogether unlimited, the condition of the servant is styled
\emph{pure slavery.} But as the rules of language are as far as can be
conceived from being steady on this head, the term slavery is commonly
made use of wherever the limitations prescribed to the power of the
master are looked upon as inconsiderable. Whenever any such limitation
is prescribed, a kind of fictitious entity is thereby created, and, in
quality of an incorporeal object of possession, is bestowed upon the
servant: this object is of the class of those which are called
\emph{rights:} and in the present case is termed, in a more particular
manner, a \emph{liberty;} and sometimes a \emph{privilege,} an
\emph{immunity,} or an \emph{exemption.} Now those limitations on the
one hand, and these liberties on the other, may, it is evident, be as
various as the acts (positive or negative) which the master may or may
not have the power of obliging the servant to submit to or to perform.
Correspondent then to the infinitude of these liberties, is the
infinitude of the modifications which the condition of mastership (or,
as it is more common to say in such a case, that of servitude) admits
of. These modifications, it is evident, may, in different countries, be
infinitely diversified. Indifferent countries, therefore, the offences
characterised by the above names will, if specifically considered, admit
of very different descriptions. If there be a spot upon the earth so
wretched as to exhibit the spectacle of pure and absolutely unlimited
slavery, on that spot there will be no such thing as any abuse of
mastership; which means neither more nor less than that no abuse of
mastership will there be treated on the footing of an offence. As to the
question, Whether any, and what, modes of servitude ought to\\
be established or kept on foot? this is a question, the solution of
which belongs to the civil branch of the art of legislation.

XLIII. Next, with regard to the offences that may concern the condition
of a servant. It might seem at first sight, that a condition of this
kind could not have a spark of benefit belonging to it: that it could
not be attended with any other consequences than such as rendered it a
mere burthen. But a burthen itself may be a benefit, in comparison of a
greater burthen. Conceive a man's situation then to be such, that he
must, at any rate, be in a state of pure slavery. Still may it be
material to him, and highly material, who the person is whom he has for
his master. A state of slavery then, under one master, may be a
beneficial state to him, in comparison with a state of slavery under
another master. The condition of a servant then is exposed to the
several offences to which a condition, in virtue of its being a
beneficial one, is exposed. More than this, where the power of the
master is limited, and the limitations annexed to it, and thence the
liberties of the servant, are considerable, the servitude may even be
positively eligible. For amongst those limitations may be such as are
sufficient to enable the servant to possess property of his own: being
capable then of possessing property of his own, he may be capable of
receiving it from his master: in short, he may receive wages, or other
emoluments, from his master; and the benefit resulting from these wages
may be so considerable as to outweigh the burthen of the servitude, and,
by that means, render that condition more beneficial upon the whole, and
more eligible, than that of one who is not in any respect under the
control of any such person as a master. Accordingly, by these means the
condition of the servant may be so eligible, that his entrance into it,
and his continuance in it, may have been altogether the result of his
own choice. That the nature of the two conditions may be the more
clearly understood, it may be of use to show the sort of correspondency
there is between the offences which affect the existence of the one, and
those which affect the existence of the other. That this correspondency
cannot but be very intimate is obvious at first sight. It is not,
however, that a given offence in the former catalogue coincides with an
offence of the same name in the latter catalogue: usurpation of
servantship with usurpation of mastership, for example. But the case is,
that an offence of one denomination in the one catalogue coincides with
an offence of a different denomination in the other catalogue. Nor is
the coincidence constant and certain: but liable to contingencies, as we
shall see. First, then, wrongful non-investment of the condition of a
servant, if it be the offence of one who should have been the master,
coincides with wrongful detrectation of mastership: if it be the offence
of a third person, it involves in it non-investment of mastership,
which, provided the mastership be in the eyes of him who should have
been master a beneficial thing, but not otherwise, is wrongful.\\
2. Wrongful interception of the condition of a servant, if it be the
offence of him who should have been master, coincides with wrongful
detrectation of mastership: if it be the offence of a third person, and
the mastership be a beneficial thing, it involves in it wrongful
interception of mastership.\\
3. Wrongful divestment of servantship, if it be the offence of the
master, but not otherwise, coincides with wrongful abdication of
mastership: if it be the offence of a stranger, it involves in it
divestment of mastership, which, in as far as the mastership is a
beneficial thing, is wrongful.\\
4. Usurpation of servantship coincides necessarily with wrongful
imposition of mastership: it will be apt to involve in it wrongful
divestment of mastership: but this only in the case where the usurper,
previously to the usurpation, was in a state of servitude under some
other master.\\
5. Wrongful investment of servantship (the servantship being considered
as a beneficial thing) coincides with imposition of mastership; which,
if in the eyes of the pretended master the mastership should chance to
be a burthen, will be wrongful.\\
6. Wrongful abdication of servantship coincides with wrongful divestment
of mastership.\\
7. Wrongful detrectation of servantship, with wrongful non-investment of
mastership.\\
8. Wrongful imposition of servantship, if it be the offence of the
pretended master, coincides with usurpation of mastership: if it be the
offence of a stranger, it involves in it imposition of mastership,
which, if in the eyes of the pretended master the mastership should be a
burthen, will be wrongful.\\
As to abuse of mastership, disturbance of mastership, breach of duty in
servants, elopement of servants, and servant-stealing, these are
offences which, without any change of denomination, bear equal relation
to both conditions. And thus we may reckon thirteen sorts of offences to
which the condition of a servant stands exposed: viz.,\\
1. Wrongful non-investment of servantship.\\
2. Wrongful interception of servantship.\\
3. Wrongful divestment of servantship.\\
4. Usurpation of servantship.\\
5. Wrongful investment of servantship.\\
6. Wrongful abdication of servantship.\\
7. Wrongful detrectation of servantship.\\
8. Wrongful imposition of servantship.\\
9. Abuse of mastership.\\
10. Disturbance of mastership.\\
11. Breach of duty in servants.\\
12. Elopement of servants.\\
13. Servant-stealing.

XLIV. We now come to the offences to which the condition of of a
guardian is exposed. A guardian is one who is invested with power over
another, living within the compass of the same family, and called a
ward; the power being to be exercised for the benefit of the ward. Now
then, what are the cases in which it can be for the benefit of one man,
that another, living within the compass of the same family, should
exercise power over him? Consider either of the parties by himself, and
suppose him, in point of understanding, to be on a level with the other,
it seems evident enough that no such cases can ever exist. To the
production of happiness on the part of any given person (in like manner
as to the production of any other effect which is the result of human
agency) three things it is necessary should concur: knowledge,
inclination, and physical power. Now as there is no man who is so sure
of being \emph{inclined,} on all occasions, to promote your happiness as
you yourself are, so neither is there any man who upon the whole can
have had so good opportunities as you must have had of \emph{knowing}
what is most conducive to that purpose. For who should know so well as
you do what it is that gives you pain or pleasure? Moreover, as to
power, it is manifest that no superiority in this respect, on the part
of a stranger, could, for a constancy, make up for so great a deficiency
as he must lie under in respect of two such material points as knowledge
and inclination. If then there be a case where it can be for the
advantage of one man to be under the power of another, it must be on
account of some palpable and very considerable deficiency, on the part
of the former, in point of intellects, or (which is the same thing in
other words) in point of knowledge or understanding. Now there are two
cases in which such palpable deficiency is known to take place. These
are,\\
1. Where a man's intellect is not yet arrived at that state in which it
is capable of directing his own inclination in the pursuit of happiness:
this is the case of \emph{infancy.}\\
2. Where by some particular known or unknown circumstance his intellect
has either never arrived at that state, or having arrived at it has
fallen from it: which is the case of \emph{insanity.} \emph{\\
}By what means then is it to be ascertained whether a man's intellect is
in that state or no? For exhibiting the quantity of sensible heat in a
human body we have a very tolerable sort of instrument, the thermometer;
but for exhibiting the quantity of intelligence, we have no such
instrument. It is evident, therefore, that the line which separates the
quantity of intelligence which is sufficient for the purposes of
self-government from that which is not sufficient, must be, in a great
measure, arbitrary. Where the insufficiency is the result of want of
age, the sufficient quantity of intelligence, be it what it may, does
not accrue to all at the same period of their lives. It becomes
therefore necessary for legislators to cut the gordian knot, and fix
upon a particular period, at which and not before, truly or not, every
person whatever shall be deemed, as far as depends upon age, to be in
possession of this sufficient quantity. In this case then a line is
drawn which may be the same for every man, and in the description of
which, such as it is, whatever persons are concerned may be certain of
agreeing: the circumstance of time affording a mark by which the line in
question may be traced with the utmost degree of nicety. On the other
hand, where the insufficiency is the result of insanity, there is not
even this resource: so that here the legislator has no other expedient
than to appoint some particular person or persons to give a particular
determination of the question, in every instance in which it occurs,
according to his or their particular and arbitrary discretion. Arbitrary
enough it must be at any rate, since the only way in which it can be
exercised is by considering whether the share of intelligence possessed
by the individual in question does or does not come up to the loose and
indeterminate idea which persons so appointed may chance to entertain
with respect to the quantity which is deemed sufficient.

XLV. The line then being drawn, or supposed to be so, it is expedient to
a man who cannot, with safety to himself, be left in his own power, that
he should be placed in the power of another. How long then should he
remain so? Just so long as his inability is supposed to continue: that
is, in the case of infancy, till he arrives at that period at which the
law deems him to be of full age: in the case of insanity, till he be of
sound mind and understanding. Now it is evident, that this period, in
the case of infancy, may not arrive for a considerable time: and in the
case of insanity, perhaps never. The duration of the power belonging to
this trust must therefore, in the one case, be very considerable; in the
other case, indefinite.

XLVI. The next point to consider, is what \emph{may} be the extent of
it? for as to what \emph{ought} to be, that is a matter to be settled,
not in a general analytical sketch, but in a particular and
circumstantial dissertation. By possibility, then, this power may
possess any extent that can be imagined: it may extend to any acts
which, physically speaking, it may be in the power of the ward to
perform himself, or be the object of if exercised by the guardian.
Conceive the power, for a moment, to stand upon this footing: the
condition of the ward stands now exactly upon a footing with pure
slavery. Add the obligation by which the power is turned into a trust:
the limits of the power are now very considerably narrowed. What then is
the purport of this obligation? Of what nature is the course of conduct
it prescribes? It is such a course of conduct as shall be best
calculated for procuring to the ward the greatest quantity of happiness
which his faculties, and the circumstances he is in, will admit of:
saving always, in the first place, the regard which the guardian is
permitted to show to his own happiness; and, in the second place, that
which he is obliged, as well as permitted, to show to that of other men.
This is, in fact, no other than that course of conduct which the ward,
did he but know how, ought, in point of \emph{prudence,} to maintain of
himself: so that the business of the former is to govern the latter
precisely in the manner in which this latter ought to govern himself.
Now to instruct each individual in what manner to govern his own conduct
in the details of life, is the particular business of private ethics: to
instruct individuals in what manner to govern the conduct of those whose
happiness, during nonage, is committed to their charge, is the business
of the art of private education. The details, therefore, of the rules to
be given for that purpose, any more than the acts which are capable of
being committed in violation of those rules, belong not to the art of
legislation: since, as will be seen more particularly hereafter, such
details could not, with any chance of advantage, be provided for by the
legislator. Some general outlines might indeed be drawn by his
authority: and, in point of fact, some are in every civilized state. But
such regulations, it is evident, must be liable to great variation: in
the first place, according to the infinite diversity of civil conditions
which a man may stand invested with in any given state: in the next
place, according to the diversity of local circumstances that may
influence the nature of the conditions which may chance to be
established in different states. On this account, the offences which
would be constituted by such regulations could not be comprised under
any concise and settled denominations, capable of a permanent and
extensive application. No place, therefore, can be allotted to them
here.

XLVII. By what has been said, we are the better prepared for taking an
account of the offences to which the condition in question stands
exposed. Guardianship being a private trust, is of course exposed to
those offences, and no others, by which a private trust is liable to be
affected. Some of them, however, on account of the special quality of
the trust, will admit of some further particularity of description.\\
In the first place, breach of this species of trust may be termed
\emph{mismanagement} of guardianship: in the second place, of whatever
nature the duties are which are capable of being annexed to this
condition, it must often happen, that in order to fulfil them, it is
necessary the guardian should be at a certain particular place.
Mismanagement of guardianship, when it consists in the not being, on the
occasion in question, at the place in question, may be termed
\emph{desertion} of guardianship.\\
Thirdly, It is manifest enough, that the object which the guardian ought
to propose to himself, in the exercise of the powers to which those
duties are annexed, is to procure for the ward the greatest quantity of
happiness which can be procured for him, consistently with the regard
which is due to the other interests that have been mentioned: for this
is the object which the ward would have proposed to himself, and might
and ought to have been allowed to propose to himself, had he been
capable of governing his own conduct. Now, in order to procure this
happiness, it is necessary that he should possess a certain power over
the objects on the use of which such happiness depends. These objects
are either the person of the ward himself, or other objects that are
extraneous to him. These other objects are either things or persons. As
to \emph{things,} then, objects of this class, insofar as a man's
happiness depends upon the use of them, are styled his \emph{property.}
The case is the same with the services of any \emph{persons} over whom
he may happen to possess a beneficial power, or to whose services he may
happen to possess a beneficial right. Now when property of any kind,
which is in trust, suffers by the delinquency of him with whom it is in
trust, such offence, of whatever nature it is in other respects, may be
styled \emph{dissipation} in breach of trust: and if it be attended with
a profit to the trustee, it may be styled \emph{peculation.}\\
Fourthly, For one person to exercise a power of any kind over another,
it is necessary that the latter should either perform certain acts, upon
being commanded so to do by the former, or at least should suffer
certain acts to be exercised upon himself. In this respect a ward must
stand upon the footing of a servant: and the condition of a ward must,
in this respect, stand exposed to the same offences to which that of a
servant stands exposed: that is, on the part of a stranger, to
\emph{disturbance,} which, in particular circumstances, will amount to
\emph{theft:} on the part of the ward, to \emph{breach of duty:} which,
in particular circumstances, maybe effected by \emph{elopement.}\\
Fifthly, There does not seem to be any offence concerning guardianship
that corresponds to \emph{abuse of trust:} I mean in the sense to which
the last-mentioned denomination has been here confined. The reason is,
that guardianship, being a trust of a private nature, does not, as such,
confer upon the trustee any power, either over the persons or over the
property of any party, other than the \emph{beneficiary} himself. If by
accident it confers on the trustee a power over any persons whose
services constitute a part of the property of the beneficiary, the
trustee becomes thereby, in certain respects, the master of such
servants.\\
Sixthly, Bribery also is a sort of offence to which, in this case, there
is not commonly much temptation. It is an offence, however, which by
possibility is capable of taking this direction: and must therefore be
aggregated to the number of the offences to which the condition of a
guardian stands exposed. And thus we have in all seventeen of these
offences: viz.,\\
1. Wrongful non-investment of guardianship.\\
2. Wrongful interception of guardianship.\\
3. Wrongful divestment of guardianship.\\
4 Usurpation of guardianship.\\
5. Wrongful investment of guardianship.\\
6. Wrongful abdication of guardianship.\\
7. Detrectation of guardianship.\\
8. Wrongful imposition of guardianship.\\
9. Mismanagement of guardianship.\\
10. Desertion of guardianship.\\
11. Dissipation in prejudice of wardship.\\
12. Peculation in prejudice of wardship.\\
13. Disturbance of guardianship.\\
14. Breach of duty to guardians.\\
15. Elopement from guardians.\\
16. Ward-stealing.\\
17. Bribery in prejudice of wardship.

XLVIII. Next, with regard to offences to which the condition of wardship
is exposed. Those which first affect the existence of the condition
itself are as follows:\\
1. Wrongful non-investment of the condition of a ward. This, if it be
the offence of one who should have been guardian, coincides with
wrongful detrectation of guardianship: if it be the offence of a third
person, it involves in it non-investment of guardianship, which,
provided the guardianship is, in the eyes of him who should have been
guardian, a desirable thing, is wrongful.\\
2. Wrongful interception of wardship. This, if it be the offence of him
who should have been guardian, coincides with wrongful detrectation of
guardianship: if it be the offence of a third person, it involves in it
interception of guardianship, which, provided the guardianship is, in
the eyes of him who should have been guardian, a desirable thing, is
wrongful.\\
3. Wrongful divestment of wardship. This, if it be the offence of the
guardian, but not otherwise, coincides with wrongful abdication of
guardianship: if it be the offence of a third person, it involves in it
divestment of guardianship, which, if the guardianship is, in the eyes
of the guardian, a desirable thing, is wrongful.\\
4. Usurpation of the condition of a ward: an offence not very likely to
be committed. This coincides at any rate with wrongful imposition of
guardianship; and if the usurper were already under the guardianship of
another guardian, it will involve in it wrongful divestment of such
guardianship.\\
5. Wrongful investment of wardship (the wardship being considered as a
beneficial thing): this coincides with imposition of guardianship,
which, if in the eyes of the pretended guardian the guardianship should
be a burthen, will be wrongful.\\
6. Wrongful abdication of wardship. This coincides with wrongful
divestment of guardianship.\\
7. Wrongful detrectation of wardship. This coincides with wrongful
interception of guardianship.\\
8. Wrongful imposition of wardship. This, if the offender be the
pretended guardian, coincides with usurpation of guardianship: if a
stranger, it involves in it wrongful imposition of guardianship. As to
such of the offences relative to this condition, as concern the
consequences of it while subsisting, they are of such a nature that,
without any change of denomination, they belong equally to the condition
of a guardian and that of a ward. We may therefore reckon seventeen
sorts of offences relative to the condition of a ward:\\
1. Wrongful non-investment of wardship.\\
2. Wrongful interception of wardship.\\
3. Wrongful divestment of wardship.\\
4. Usurpation of wardship.\\
5. Wrongful investment of wardship.\\
6. Wrongful abdication of wardship.\\
7. Wrongful detrectation of wardship.\\
8. Wrongful interception of wardship.\\
9. Mismanagement of guardianship.\\
10. Desertion of guardianship.\\
11. Dissipation in prejudice of wardship.\\
12. Peculation in prejudice of wardship.\\
13. Disturbance of guardianship.\\
14. Breach of duty to guardians.\\
15. Elopement from guardians.\\
16. Ward-stealing.\\
17. Bribery in prejudice of wardship.

XLIX. We come now to the offences to which the condition or of a parent
stands exposed: and first, with regard to those by which the very
existence of the condition is affected. On this occasion, in order to
see the more clearly into the subject, it will be necessary to
distinguish between the natural relationship, and the legal relationship
which is superinduced as it were upon the natural one. The natural one
being constituted by a particular event, which, either on account of its
being already past, or on some other account, is equally out of the
power of the law neither is, nor can be made, the subject of an offence.
\emph{Is} a man your father? It is not any offence of mine that can make
you not his son. Is he \emph{not} your father? It is not any offence of
mine that can render him so. But although he does in fact bear that
relation to you, I, by an offence of mine, may perhaps so manage
matters, that he shall not be \emph{thought} to bear it: which, with
respect to any legal advantages which either he or you could derive from
such relationship, will be the same thing as if he did not. In the
capacity of a witness, I may cause the judges to believe that he is not
your father, and to decree accordingly: or, in the capacity of a judge,
I may myself decree him not to be your father. Leaving then the purely
natural relationship as an object equally out of the reach of justice
and injustice, the legal condition, it is evident, will stand exposed to
the same offences, neither more nor less, as every other condition, that
is capable of being either beneficial or burthensome, stands exposed to.
Next, with regard to the exercise of the functions belonging to this
condition, considered as still subsisting. In parentality there must be
two persons concerned, the father and the mother. The condition of a
parent includes, therefore, two conditions; that of a father, and that
of a mother, with respect to such or such a child. Now it is evident,
that between these two parties, whatever beneficiary powers, and other
rights, as also whatever obligations, are annexed to the condition of a
parent, may be shared in any proportions that can be imagined. But if in
these several objects of legal creation, each of these two parties have
severally a share, and if the interests of all these parties are in any
degree provided for, it is evident that each of the parents will stand,
with relation to the child, in two several capacities: that of a master,
and that of a guardian. The condition of a parent then, in as far as it
is the work of law, may be considered as a complex condition, compounded
of that of a guardian, and that of a master. To the parent then, in
quality of guardian, results a set of duties, involving, as necessary to
the discharge of them, certain powers: to the child, in the character of
a ward, a set of rights corresponding to the parent's duties, and a set
of duties corresponding to his powers. To the parent again, in quality
of master, a set of beneficiary powers, without any other necessary
limitation (so long as they last) than what is annexed to them by the
duties incumbent on him in quality of a guardian: to the child, in the
character of a servant, a set of duties corresponding to the parent's
beneficiary powers, and without any other necessary limitation (so long
as they last) than what is annexed to them by the rights which belong to
the child in his capacity of ward. The condition of a parent will
therefore be exposed to all the offences to which either that of a
guardian or that of a master are exposed: and, as each of the parents
will partake, more or less, of both those characters, the offences to
which the two conditions are exposed may be nominally, as they will be
substantially, the same. Taking them then all together, the offences to
which the condition of a parent is exposed will stand as follows:\\
1. Wrongful non-investment of parentality.\\
2. Wrongful interception of parentality.\\
3. Wrongful divestment of parentality.\\
4. Usurpation of parentality.\\
5. Wrongful investment of parentality.\\
6. Wrongful abdication of parentality.\\
7. Wrongful detrectation of parentality.\\
8. Wrongful imposition of parentality.\\
9. Mismanagement of parental guardianship.\\
10. Desertion of parental guardianship.\\
11. Dissipation in prejudice of filial wardship.\\
12. Peculation in prejudice of filial wardship.\\
13. Abuse of parental powers.\\
14. Disturbance of parental guardianship.\\
15. Breach of duty to parents.\\
16. Elopement from parents.\\
17. Child-stealing.\\
18. Bribery in prejudice of filial wardship.

L. Next with regard to the offences to which the \emph{filial}
condition, the condition of a son or daughter, stands exposed. The
principles to be pursued in the investigation of offences of this
description have already been sufficiently developed. It will be
sufficient, therefore, to enumerate them without further discussion. The
only peculiarities by which offences relative to the condition in
question stand distinguished from the offences relative to all the
preceding conditions, depend upon this one circumstance; viz., that it
is certain every one must have had a father and a mother: at the same
time that it is not certain that every one must have had a master, a
servant, a guardian, or a ward. It will be observed all along, that
where a person, from whom, if alive, the benefit would be taken, or on
whom the burthen would be imposed, be dead, so much of the mischief is
extinct along with the object of the offence. There still, however,
remains so much of the mischief as depends upon the advantage or
disadvantage which might accrue to persons related, or supposed to be
related, in the several remoter degrees, to him in question. The
catalogue then of these offences stands as follows:\\
1. Wrongful non-investment of filiation. This, if it be the offence of
him or her who should have been recognized as the parent, coincides with
wrongful detrectation of parentality: if it be the offence of a third
person, it involves in it non-investment of parentality, which, provided
the parentality is, in the eyes of him or her who should have been
recognised as the parent, a desirable thing, is wrongful.\\
2. Wrongful interception of filiation. This, if it be the offence of him
or her who should have been recognised as the parent, coincides with
wrongful detrectation of parentality: if it be the offence of a third
person, it involves in it interception of parentality, which, provided
the parentality is, in the eyes of him or her who should have been
recognized as parent, a desirable thing, is wrongful.\\
3. Wrongful divestment of filiation. This, if it be the offence of him
or her who should be recognized as parent, coincides with wrongful
abdication of parentality: if it be the offence of a third person, it
involves in it divestment of parentality; to wit, of paternity, or of
maternity, or of both; which, if the parentality is, in the eyes of him
or her who should be recognized as parent, a desirable thing, are
respectively wrongful.\\
4. Usurpation of filiation. This coincides with wrongful imposition of
parentality; to wit, either of paternity, or of maternity, or of both:
and necessarily involves in it divestment of parentality, which, if the
parentality thus divested were, in the eyes of him or her who are thus
divested of it, a desirable thing, is wrongful.\\
5. Wrongful investment of filiation: (the filiation being considered as
a beneficial thing.) This coincides with imposition of parentality,
which, if in the eyes of the pretended father or mother the parentality
should be an undesirable thing, will be wrongful.\\
6. Wrongful abdication of filiation. This necessarily coincides with
wrongful divestment of parentality; it also is apt to involve in it
wrongful imposition of parentality; though not necessarily either to the
advantage or to the prejudice of any certain person. For if a man,
supposed at first to be your son, appears afterwards not to be yours, it
is certain indeed that he is the son of some other man, but it may not
appear who that other man is.\\
7. Wrongful detrectation of filiation. This coincides with wrongful
noninvestment or wrongful interception of parentality.\\
8. Wrongful imposition of filiation. This, if it be the offence of the
pretended parent, coincides necessarily with usurpation of parentality:
if it be the offence of a third person, it necessarily involves
imposition of parentality; as also divestment of parentality: either or
both of which, according to the circumstance above mentioned, may or may
not be wrongful.\\
9. Mismanagement of parental guardianship.\\
10. Desertion of parental guardianship.\\
11. Dissipation in prejudice of filial wardship.\\
12. Peculation in prejudice of filial wardship.\\
13. Abuse of parental power.\\
14. Disturbance of parental guardianship.\\
15. Breach of duty to parents.\\
16. Elopement from parents.\\
17. Child-stealing.\\
18. Bribery in prejudice of parental guardianship.

LI. We shall now be able to apply ourselves with some advantage to the
examination of the several offences to which the marital condition, or
condition of a husband, stands exposed. A husband is a man, between whom
and a certain woman, who in this case is called his wife, there subsists
a legal obligation for the purpose of their living together, and in
particular for the purpose of a sexual intercourse to be carried on
between them This obligation will naturally be considered in four points
of view:\\
1. In respect of its commencement.\\
2. In respect of the placing of it.\\
3. In respect of the nature of it.\\
4. In respect of its duration.\\
First then, it is evident, that in point of possibility, one method of
commencement is as conceivable as another: the time of its commencement
might have been marked by one sort of event (by one sort of
\emph{signal,} as it may here be called) as well as by another. But in
practice the signal has usually been, as in point of utility it ought
constantly to be, a contract entered into by the parties: that is, a set
of signs, pitched upon by the law, as expressive of their \emph{mutual
consent,} to take upon them this condition.\\
Secondly, and thirdly, with regard to the placing of the obligations
which are the result of the contract, it is evident that they must rest
either solely on one side, or mutually on both. On the first
supposition, the condition is not to be distinguished from pure slavery.
In this case, either the wife must be the slave of the husband, or the
husband of the wife. The first of these suppositions has perhaps never
been exemplified; the opposing influence of physical causes being too
universal to have ever been surmounted: the latter seems to have been
exemplified but too often; perhaps among the first Romans; at any rate,
in many barbarous nations.\\
Thirdly, with regard to the nature of the obligations. If they are not
suffered to rest all on one side, certain rights are thereby given to
the other. There must, therefore, be rights on both sides. Now, where
there are mutual rights possessed by two persons, as against each other,
either there are powers annexed to those rights, or not. But the persons
in question are, by the supposition, to live together: in which case we
have shown, that it is not only expedient, but in a manner necessary,
that on one side there should be powers. Now it is only on one side that
powers can be: for suppose them on both sides, and they destroy one
another. The question is then, In which of the parties these powers
shall be lodged? we have shown, that on the principle of utility they
ought to be lodged in the husband. The powers then which subsist being
lodged in the husband, the next question is, Shall the interest of one
party only, or of both, be consulted in the exercise of them? it is
evident, that on the principle of utility the interests of both ought
alike to be consulted: since in two persons taken together, more
happiness is producible than in one. This being the case, it is
manifest, that the legal relation which the husband will bear to the
wife will be a complex one: compounded of that of master and that of
guardian.

LII. The offences then to which the condition of a husband of will be
exposed, will be the sum of those to which the two conditions of master
and guardian are exposed. Thus far the condition of a husband, with
respect to the general outlines of it, stands upon the same footing as
that of a parent. But there are certain reciprocal services, which being
the main subject of the matrimonial contract, constitute the essence of
the two matrimonial relations, and which neither a master nor guardian,
as such, nor a parent, at any rate, have usually been permitted to
receive. These must of course have been distinguished from the
indiscriminate train of services at large which the husband in his
character of master is empowered to exact, and of those which in his
character of guardian he is bound to render. Being thus distinguished,
the offences relative to the two conditions have, in many instances, in
as far as they have reference to these peculiar services, acquired
particular denominations.\\
In the first place, with regard to the contract, from the celebration of
which the legal condition dates its existence. It is obvious that in
point of possibility, this contract might, on the part of either sex,
subsist with respect to several persons of the other sex at the same
time: the husband might have any number of wives: the wife might have
any number of husbands: the husband might enter into the contract with a
number of wives at the same time: or, if with only one at a time, he
might reserve to himself a right of engaging in a similar contract with
any number, or with only such or such a number of other women
afterwards, during the continuance of each former contract. This latter
accordingly is the footing upon which, as is well known, marriage is and
has been established in many extensive countries: particularly in all
those which profess the Mahometan religion. In point of possibility, it
is evident that the like liberty might be reserved on the part of the
wife: though in point of practice no examples of such an arrangement
seem ever to have occurred. Which of all these arrangements is in point
of utility the most expedient, is a question which would require too
much discussion to answer in the course of an analytical process like
the present, and which belongs indeed to the civil branch of
legislation, rather than to the penal. In Christian countries, the
solemnization of any such contract is made to exclude the solemnization
of any subsequent one during the continuance of a former: and the
solemnization of any such subsequent contract is accordingly treated as
an offence, under the name of \emph{Polygamy.} \emph{\\
}Polygamy then is at any rate, on the part of the man, a particular
modification of that offence which may be styled usurpation of the
condition of a husband. As to its other effects, they will be different,
according as it was the man only, or the woman only, or both, that were
in a state of matrimony at the time of the commission of the offence. If
the man only, then his offence involves in it \emph{pro tanto} that of
wrongful divestment of the condition of a wife, in prejudice of his
prior wife. If the woman only, then it involves in it \emph{pro tanto}
that of wrongful divestment of the condition of a husband, in prejudice
of her prior husband. If both were already married, it of course
involves both the wrongful divestments which have just been mentioned.
And on the other hand also, the converse of all this may be observed
with regard to polygamy on the part of the woman.\\
Secondly, As the engaging not to enter into any subsequent engagement of
the like kind during the continuance of the first, is one of the
conditions on which the law lends its sanction to the first; so another
is, the inserting as one of the articles of this engagement an
undertaking not to render to, or accept from, any other person the
services which form the characteristic object of it: the rendering or
acceptance of any such services is accordingly treated as an offence,
under the name of \emph{adultery:} under which name is also comprised
the offence of the stranger, who, in the commission of the above
offence, is the necessary accomplice.\\
Thirdly, Disturbing either of the parties to this engagement, in the
possession of these characteristic services, may, in like manner, be
distinguished from the offence of disturbing them in the enjoyment of
the miscellaneous advantages derivable from the same condition; and on
whichever side the blame rests, whether that of the party, or that of a
third person, may be termed \emph{wrongful withholding of connubial
services.} And thus we have one-and-twenty sorts of offences to which,
as the law stands at present in Christian countries, the condition of a
husband stands exposed: viz.,\\
1. Wrongful non-investment of the condition of a husband.\\
2. Wrongful interception of the condition of a husband.\\
3. Wrongful divestment of the condition of a husband.\\
4. Usurpation of the condition of a husband.\\
5. Polygamy.\\
6. Wrongful investment of the condition of a husband.\\
7. Wrongful abdication of the condition of a husband.\\
8. Wrongful detrectation of the condition of a husband.\\
9. Wrongful imposition of the condition of a husband.\\
10. Mismanagement of marital guardianship.\\
11. Desertion of marital guardianship.\\
12. Dissipation in prejudice of matrimonial wardship.\\
13. Peculation in prejudice of matrimonial wardship.\\
14. Abuse of marital power.\\
15. Disturbance of marital guardianship.\\
16. Wrongful withholding of connubial services.\\
17. Adultery.\\
18. Breach of duty to husbands.\\
19. Elopement from husbands.\\
20. Wife-stealing.\\
21. Bribery in prejudice of marital guardianship.

LIII. Next with regard to the offences to which the condition of a wife
stands exposed. From the patterns that have been exhibited already, the
coincidences and associations that take place between the offences that
concern the existence of this condition and those which concern the
existence of the condition of a husband, may easily enough be
apprehended without farther repetitions. The catalogue of those now
under consideration will be precisely the same in every article as the
catalogue last exhibited.

LIV. Thus much for the several sorts of offences relative to the several
sorts of domestic conditions: those which are constituted by such
natural relations as are contiguous being included. There remain those
which are uncontiguous: of which, after so much as has been said of the
others, it will naturally be expected that some notice should be taken.
These, however, do not afford any of that matter which is necessary to
constitute a condition. In point of fact, no power seems ever to be
annexed to any of them. A grandfather, perhaps, may be called by the law
to take upon him the guardianship of his orphan grandson: but then the
power he has belongs to him not as grandfather, but as guardian. In
point of possibility, indeed, power might be annexed to these relations,
just as it might to any other. But still no new sort of domestic
condition would result from it: since it has been shown that there can
be no others, that, being constituted by power, shall be distinct from
those which have been already mentioned. Such as they are, however, they
have this in common with the before-mentioned relations, that they are
capable of importing either benefit or burthen: they therefore stand
exposed to the several offences whereby those or any other relations are
liable to be affected in point of existence. It might be expected,
therefore, that in virtue of these offences, they should be added to the
list of the relations which are liable to be objects of delinquency. But
the fact is, that they already stand included in it: and although not
expressly named, yet as effectually as if they were. On the one hand, it
is only by affecting such or such a contiguous relation that any offence
affecting uncontiguous relations can take place. On the other hand,
neither can any offence affecting the existence of the contiguous
relations be committed, without affecting the existence of an indefinite
multitude of such as are uncontiguous. A false witness comes, and causes
it to be believed that you are the son of a woman, who, in truth, is not
your mother. What follows? An endless tribe of other false
persuasions'' that you are the grandson of the father and of the
mother of this supposed mother: that you are the son of some husband of
hers, or, at least, of some man with whom she has cohabited: the
grandson of his father and his mother; and so on: the brother of their
other children, if they have any: the brother-in-law of the husbands and
wives of those children, if married: the uncle of the children of those
children: and so on.'' On the other hand, that you are not the
son of your real mother, nor of your real father: that you are not the
grandson of either of your real grandfathers or grandmothers; and so on
without end: all which persuasions result from, and are included in, the
one original false persuasion of your being the son of this your
pretended mother.\\
It should seem, therefore, at first sight, that none of the offences
against these uncontiguous relations could ever come expressly into
question: for by the same rule that one ought, so it might seem ought a
thousand others: the offences against the uncontiguous being merged as
it were in those which affect the contiguous relations. So far, however,
is this from being the case, that in speaking of an offence of this
stamp, it is not uncommon to hear a great deal said of this or that
uncontiguous relationship which it affects, at the same time that no
notice at all shall\\
be taken of any of those which are contiguous. How happens this?
Because, to the uncontiguous relation are annexed perhaps certain
remarkable advantages or disadvantages, while to all the intermediate
relations none shall be annexed which are in comparison worth noticing.
Suppose Antony or Lepidus to have contested the relationship of Octavius
(afterwards Augustus) to Caius Julius Cæsar. How could it have been
done? It could only have been by contesting, either Octavius's being the
son of Atia, or Atia's being the daughter of Julia, or Julia's being the
daughter of Lucius Julius Cæsar, or Lucius Julius Cæsar's being the
father of Caius. But to have been the son of Atia, or the grandson of
Julia, or the great grandson of Lucius Julius Cæsar, was, in
comparison, of small importance. Those intervening relationships were,
comparatively speaking, of no other use to him than in virtue of their
being so many necessary links in the genealogical chain which connected
him with the sovereign of the empire.\\
As to the advantages and disadvantages which may happen to be annexed to
any of those uncontiguous relationships, we have seen already that no
powers over the correlative person, nor any corresponding obligations,
are of the number. Of what nature then can they be? They are, in truth,
no other than what are the result either of local and accidental
institutions, or of some spontaneous bias that has been taken by the
moral sanction. It would, therefore, be to little purpose to attempt
tracing them out \emph{a priori} by any exhaustive process: all that can
be done is, to pick up and lay together some of the principal articles
in each catalogue by way of specimen. The advantages which a given
relationship is apt to impart, seem to be referable chiefly to the
following heads:\\
1. Chance of succession to the property, or a part of the property, of
the correlative person.\\
2. Chance of pecuniary support, to be yielded by the correlative person,
either by appointment of law, or by spontaneous donation.\\
3. Accession of legal rank; including any legal privileges which may
happen to be annexed to it: such as capacity of holding such and such
beneficial offices; exemption from such and such burthensome
obligations; for instance, paying taxes, serving burthensome offices..\\
4. Accession of rank by courtesy; including the sort of reputation which
is customarily and spontaneously annexed to distinguished birth and
family alliance: whereon may depend the chance of advancement in the way
of marriage, or in a thousand other ways less obvious. The disadvantages
which a given relation is liable to impart, seem to be referable chiefly
to the following heads:\\
1. Chance of being obliged, either by law, or by force of the moral
sanction, to yield pecuniary support to the correlative party.\\
2. Loss of legal rank: including the legal disabilities, as well as the
burthensome obligations, which the law is apt to annex, sometimes with
injustice enough, to the lower stations.\\
3. Loss of rank by courtesy: including the loss of the advantages
annexed by custom to such rank.\\
4. Incapacity of contracting matrimony with the correlative person,
where the supposed consanguinity or affinity lies within the prohibited
degrees.

LV. We come now to civil conditions: these, it may well be imagined, may
be infinitely various: as various as the acts which a man may be either
commanded or allowed, whether for his own benefit, or that of others, to
abstain from or to perform. As many different denominations as there are
of persons distinguished with a view to such commands and allowances
(those denominations only excepted which relate to the conditions above
spoken of under the name of domestic ones) so many civil conditions one
might enumerate. Means however, more or less explicit, may be found out
of circumscribing their infinitude.\\
What the materials are, if so they may be called, of which conditions,
or any other kind of legal possession, can be made up, we have already
seen: beneficial powers, fiduciary powers, beneficial rights, fiduciary
rights, relative duties, absolute duties. But as many conditions as
import a power or right of the fiduciary kind, as possessed by the
person whose condition is in question, belong to the head of trusts. The
catalogue of the offences to which these conditions are exposed,
coincides therefore exactly with the catalogue of offences against
trust: under which head they have been considered in a general point of
view under the head of offences against trust: and such of them as are
of a domestic nature, in a more particular manner in the character of
offences against the several domestic conditions. Conditions constituted
by such duties of the relative kind, as have for their counterparts
trusts constituted by fiduciary powers, as well as rights on the side of
the correlative party, and those of a private nature, have also been
already discussed under the appellation of domestic conditions. The same
observation maybe applied to the conditions constituted by such powers
of the beneficial kind over persons as are of a private nature: as also
to the subordinate correlative conditions constituted by the duties
corresponding to those rights and powers. As to absolute duties, there
is no instance of a condition thus created, of which the institution is
upon the principle of utility to be justified; unless the several
religious conditions of the monastic kind should be allowed of as
examples. There remain, as the only materials out of which the
conditions which yet remain to be considered can be composed, conditions
constituted by beneficial powers over things; conditions constituted by
beneficial rights to things (that is, rights to powers over things) or
by rights to those rights, and so on; conditions constituted by rights
to services; and conditions constituted by the duties corresponding to
those respective rights. Out of these are to be taken those of which the
materials are the ingredients of the several modifications of property,
the several conditions of proprietorship. These are the conditions, if
such for a moment they may be styled, which having but here and there
any specific names, are not commonly considered on the footing of
conditions: so that the acts which, if such conditions were recognised
might be considered as offences against those conditions, are not wont
to be considered in any other light than that of offences against
property.

Now the case is, as hath been already intimated, that of these civil
conditions, those which are wont to be considered under that name, are
not distinguished by any uniform and explicit line from those of which
the materials are wont to be carried to the head of property: a set of
rights shall, in one instance, be considered as constituting an article
of property rather than a condition: while, in another instance, a set
of rights of the same stamp is considered as constituting rather a
condition than an article of property. This will probably be found to be
the case in all languages: and toe usage is different again in one
language from what it is in another. From these causes it seems to be
impracticable to subject the class of civil conditions to any exhaustive
method: so that for making a complete collection of them there seems to
be no other expedient than that of searching the language through for
them, and taking them as they come. To exemplify this observation, it
may be of use to lay open the structure as it were of two or three of
the principal sorts or classes of conditions, comparing them with two or
three articles of property which appear to be nearly of the same
complexion: by this means the nature and generation, if one may so call
it, of both these classes of ideal objects may be the more clearly
understood.\\
The several sorts of civil conditions that are not fiduciary may all, or
at least the greater part of them, be comprehended under the head of
\emph{rank,} or that of \emph{profession;} the latter word being taken
in its most extensive sense, so as to include not only what are called
the liberal professions, but those also which are exercised by the
several sorts of traders, artists, manufacturers, and other persons of
whatsoever station, who are in the way of making a profit by their
labour. Among ranks then, as well as professions, let us, for the sake
of perspicuity, take for examples such articles as stand the clearest
from any mixture of either fiduciary or beneficial power. The rank of
knighthood is constituted, how? by prohibiting all other persons from
performing certain acts, the performance of which is the symbol of the
order, at the same time that the knight in question, and his companions,
are permitted: for instance, to wear a ribbon of a certain colour in a
certain manner: to call himself by a certain title: to use an armorial
seal with a certain mark on it. By laying all persons but the knight
under this prohibition, the law subjects them to a set of duties: and
since from the discharge of these duties a benefit results to the person
in whose favour they are created, to wit, the benefit of enjoying such a
share of extraordinary reputation and respect as men are wont to yield
to a person thus distinguished, to discharge them is to render him a
service: and the duty being a duty of the negative class, a duty
consisting in the performance of certain acts of the negative kind, the
service is what may be called \emph{a service of forbearance.} It
appears then, that to generate this condition there must be two sorts of
services: that which is the immediate cause of it, a service of the
negative kind, to be rendered by the community at large: that which is
the cause again of this service, a service of the positive kind, to be
rendered by the law.\\
The condition of a professional man stands upon a narrower footing. To
constitute this condition there needs nothing more than a permission
given him on the part of the legislator to perform those acts, in the
performance of which consists the exercise of his profession: to give or
sell his advice or assistance in matters of law or physic: to give or
sell his services as employed in the executing or overseeing of a
manufacture or piece of work of such or such a kind: to sell a commodity
of such or such a sort. Here then we see there is but one sort of
service requisite; a service which may be merely of the negative kind,
to be rendered by the law: the service of permitting him to exercise his
profession: a service which, if there has been no prohibition laid on
before, is rendered by simply forbearing to prohibit him.\\
Now the ideal objects, which in the cases above specified are said to be
conferred upon a man by the services that are respectively in question,
are in both cases not articles of property but conditions. By such a
behaviour on the part of the law, as shall be the reverse of that
whereby they were respectively produced, a man may be made to forfeit
them: and what he is then said to forfeit is in neither case his
property; but in one case, his rank or dignity: in the other case, his
trade or his profession: and in both cases, his condition.\\
Other cases there are again in which the law, by a process of the same
sort with that by which it constituted the former of the two
above-mentioned conditions, confers on him an ideal object, which the
laws of language have placed under the head of property. The law permits
a man to sell books: that is, all sorts of books in general. Thus far
all that it has done is to invest him with a condition: and this
condition he would equally possess, although everybody else in the world
were to sell books likewise. Let the law now take an active part in his
favour, and prohibit all other persons from selling books of a certain
description, he remaining at liberty to sell them as before. It
therefore confers on him a sort of exclusive privilege or monopoly,
which is called a \emph{copyright.} But by investing him with this
right, it is not said to invest him with any new sort of condition: what
it invests him with is spoken of as an article of property; to wit, of
that sort of property which is termed incorporeal: and so on in the case
of an engraving, a mechanical engine, a medicine; or, in short, of a
saleable article of any other sort. Yet when it gave him an exclusive
right of wearing a particular sort of ribbon, the object which it was
then considered as conferring on him was not an article of property but
a condition.\\
By forbearing to subject you to certain disadvantages, to which it
subjects an alien, the law confers on you the condition of a
natural-born subject: by subjecting him to them, it imposes on him the
condition of an alien: by conferring on you certain privileges or
rights, which it denies to a \emph{roturier,} the law confers on you the
condition of a \emph{gentilhomme;} by forbearing to confer on him those
privileges, it imposes on him the condition of a \emph{roturier.} The
rights, out of which the two advantageous conditions here exemplified
are both of them as it were composed, have for their counterpart a sort
of services of forbearance, rendered, as we have seen, not by private
individuals, but by the law itself. As to the duties which it creates in
rendering you these services, they are to be considered as duties
imposed by the legislator on the ministers of justice.\\
It may be observed, with regard to the greater part of the conditions
here comprised under the general appellation of \emph{civil,} that the
relations corresponding to those by which they are respectively
constituted, are not provided with appellatives. The relation which has
a name, is that which is borne by the party favoured to the party bound:
that which is borne by the party bound to the party favoured has not
any. This is a circumstance that may help to distinguish them from those
conditions which we have termed domestic. In the domestic conditions, if
on the one side the party \emph{to} whom the power is given is called a
master; on the other side, the party \emph{over} whom that power is
given, the party who is the object of that power, is termed a servant.
In the civil conditions this is not the case. On the one side, a man, in
virtue of certain services of forbearance, which the rest of the
community are bound to render him, is denominated a knight of such or
such an order: but on the other side, these services do not bestow any
particular denomination on the persons from whom such services are due.
Another man, in virtue of the legislator's rendering that sort of
negative service which consists in the not prohibiting him from
exercising a trade, invests him at his option with the condition of a
trader: it accordingly denominates him a farmer, a baker, a weaver, and
so on: but the ministers of the law do not, in virtue of their rendering
the man this sort of negative service, acquire for themselves any
particular name. Suppose even that the trade you have the right of
exercising happens to be the object of a monopoly, and that the
legislator, besides rendering you himself those services which you
derive from the permission he bestows on you, obliges other persons to
render you those farther services which you receive from their
forbearing to follow the same trade; yet neither do they, in virtue of
their being thus bound, acquire any particular name.\\
After what has been said of the nature of the several sorts of civil
conditions that have names, the offences to which they are exposed may,
without much difficulty, be imagined. Taken by itself, every condition
which is thus constituted by a permission granted to the possessor, is
of course of a beneficial nature: it is, therefore, exposed to all those
offences to which the possession of a benefit is exposed. But either on
account of a man's being obliged to persevere when once engaged in it,
or on account of such other obligations as may stand annexed to the
possession of it, or on account of the comparative degree of disrepute
which may stand annexed to it by the moral sanction, it may by accident
be a burthen: it is on this account liable to stand exposed to the
offences to which, as hath been seen, every thing that partakes of the
nature of a burthen stands exposed. As to any offences which may concern
the exercise of the functions belonging to it, if it happens to have any
duties annexed to it, such as those, for instance, which are constituted
by regulations touching the exercise of a trade, it will stand exposed
to so many breaches of duty; and lastly, whatsoever are the functions
belonging to it, it will stand exposed at any rate to
\emph{disturbance.} \emph{\\
}In the forming however of the catalogue of these offences, exactness is
of the less consequence, inasmuch as an act, if it should happen not to
be comprised in this catalogue, and yet is in any respect of a
pernicious nature, will be sure to be found in some other division of
the system of offences: if a baker sells bad bread for the price of
good, it is a kind of fraud upon the buyer; and perhaps an injury of the
simple corporal kind done to the health of an individual, or a
neighbourhood: if a clothier sells bad cloth for good at home, it is a
fraud; if to foreigners abroad, it may, over and above the fraud put
upon the foreign purchaser, have pernicious effects perhaps in the
prosperity of the trade at home, and become thereby an offence against
the national wealth. So again with regard to \emph{disturbance:} if a
man be disturbed in the exercise of his trade, the offence will probably
be a wrongful \emph{interception of the profit} he might be presumed to
have been in a way to make by it: and were it even to appear in any case
that a man exercised a trade, or what is less unlikely, a liberal
profession, without having profit in his view, the offence will still be
reducible to the head of \emph{simple injurious restrainment,} or
\emph{simple injurious compulsion.}\\

§ 4. Advantages of the present method\\
LVI. A few words, for the purpose of giving a general view of the method
of division here pursued, and of the advantages which it possesses, may
have their use. The whole system of offences, we may observe, is
branched out into five classes. In the three first, the subordinate
divisions are taken from the same source; viz., from the consideration
of the different points, in respect whereof the interest of an
individual is exposed to suffer. By this uniformity, a considerable
degree of light seems to be thrown upon the whole system; particularly
upon the offences that come under the third class: objects which have
never hitherto been brought into any sort of order. With regard to the
fourth class, in settling the precedence between its several subordinate
divisions, it seemed most natural and satisfactory to place those first,
the connection whereof with the welfare of individuals seemed most
obvious and immediate. The mischievous effects of those offences, which
tend in an immediate way to deprive individuals of the protection
provided for them against the attacks of one another, and of those which
tend to bring down upon them the attacks of foreign assailants, seem
alike obvious and palpable. The mischievous quality of such as tend to
weaken the force that is provided to combat those attacks, but
particularly the latter, though evident enough, is one link farther off
in the chain of causes and effects. The ill effects of such offences as
are of disservice only by diminishing the particular fund from whence
that force is to be extracted, such effects, I say, though indisputable,
are still more distant and out of sight. The same thing may be observed
with regard to such as are mischievous only by affecting the universal
fund. Offences against the sovereignty in general would not be
mischievous, if offences of the several descriptions preceding were not
mischievous. Nor in a temporal view are offences against religion
mischievous, except in as far as, by removing, or weakening, or
misapplying one of the three great incentives to virtue, and checks to
vice, they tend to open the door to the several mischiefs, which it is
the nature of all those other offences to produce. As to the fifth
class, this, as hath already been observed, exhibits, at first view, an
irregularity, which however seems to be unavoidable. But this
irregularity is presently corrected, when the analysis returns back, as
it does after a step or two, into the path from which the tyranny of
language had forced it a while to deviate.

It was necessary that it should have two purposes in view: the one, to
exhibit, upon a scale more or less minute, a systematical enumeration of
the several possible modifications of delinquency, denominated or
undenominated; the other, to find places in the list for such names of
offences as were in current use: for the first purpose, nature was to
set the law; for the other, custom. Had the nature of the things
themselves been the only guide, every such difference in the manner of
perpetration, and such only, should have served as a ground for a
different denomination, as was attended with a difference in point of
effect. This however of itself would never have been sufficient; for as
on one hand the new language, which it would have been necessary to
invent, would have been uncouth, and in a manner unintelligible: so on
the other hand the names, which were before in current use, and which,
in spite of all systems, good or bad, must have remained in current use,
would have continued unexplained. To have adhered exclusively to the
current language, would have been as bad on the other side; for in that
case the catalogue of offences, when compared to that of the mischiefs
that are capable of being produced, would have been altogether broken
and uncomplete.\\
To reconcile these two objects, in as far as they seemed to be
reconcilable, the following course has therefore been pursued. The
logical whole, constituted by the sum total of possible offences, has
been bisected in as many different directions as were necessary, and the
process in each direction carried down to that stage at which the
particular ideas thus divided found names in current use in readiness to
receive them. At that period I have stopped; leaving any minuter
distinctions to be enumerated in the body of the work, as so many
species of the genus characterised by such or such a name. If in the
course of any such process I came to a mode of conduct which, though it
required to be taken notice of, and perhaps had actually been taken
notice of, under all laws, in the character of an offence, had hitherto
been expressed under different laws, by different circumlocutions,
without ever having received any name capable of occupying the place of
a substantive in a sentence, I have frequently ventured so far as to
fabricate a new name for it, such an one as the idiom of the language,
and the acquaintance I happened to have with it, would admit of. These
names consisting in most instances, and that unavoidably, of two or
three words brought together, in a language too which admits not, like
the German and the Greek, of their being melted into one, can never be
upon a par, in point of commodiousness, with those univocal appellatives
which make part of the established stock.\\
In the choice of names in current use, care has been taken to avoid all
such as have been grounded on local distinctions, ill founded perhaps in
the nation in which they received their birth, and at any rate not
applicable to the circumstances of other countries.\\
The analysis, as far as it goes, is as applicable to the legal concerns
of one country as of another: and where, if it had descended into
further details, it would have ceased to be so, there I have taken care
always to stop: and thence it is that it has come to be so much more
particular in the class of offences against individuals, than in any of
the other classes. One use then of this arrangement, if it should be
found to have been properly conducted, will be its serving to point out
in what it is that the legal interests of all countries agree, and in
what it is that they are liable to differ: how far a rule that is proper
for one, will serve, and how far it will not serve, for another. That
the legal interests of different ages and countries have nothing in
common, and that they have every thing, are suppositions equally distant
from the truth.

LVII. A natural method, such as it hath been here attempted to exhibit,
seems to possess four capital advantages; not to mention others of
inferior note. In the first place, it affords such assistance to the
apprehension and to the memory, as those faculties would in vain look
for in any technical arrangement. That arrangement of the objects of any
science may, it should seem, be termed a \emph{natural} one, which takes
such properties to characterise them by, as men in general are, by the
common constitution of man's nature, independently of any accidental
impressions they may have received from the influence of any local or
other particular causes, accustomed to attend to: such, in a word, as
\emph{naturally,} that is readily and at first sight, engage, and firmly
fix, the attention of any one to whom they have once been pointed out.
Now by what other means should an object engage or fix a man's
attention, unless by interesting him? and what circumstance belonging to
any action can be more interesting, or rather what other circumstance
belonging to it can be at all interesting to him, than that of the
influence it promises to have on his own happiness, and the happiness of
those who are about him? By what other mark then should he more easily
find the place which any offence occupies in the system, or by what
other clue should he more readily recall it?

LVIII. In the next place, it not only gives at first glance a general
intimation of the nature of each division of offences, in as far as that
nature is determined by some one characteristic property, but it gives
room for a number of general propositions to be formed concerning the
particular offences that come under that division, in such manner as to
exhibit a variety of other properties that may belong to them in common.
It gives room therefore, for the framing of a number of propositions
concerning them, which, though very general, because predicated of a
great number of articles, shall be as generally true.

LIX. In the third place, it is so contrived, that the very place which
any offence is made to occupy, suggests the reason of its being put
there. It serves to indicate not only that such and such acts are made
offences, but \emph{why} they \emph{ought} to be. By this means, while
it addresses itself to the understanding, it recommends itself in some
measure to the affections. By the intimation it gives of the nature and
tendency of each obnoxious act, it accounts for, and in some measure
vindicates, the treatment which it may be thought proper to bestow upon
that act in the way of punishment. To the subject then it is a kind of
perpetual apology: showing the necessity of every defalcation, which,
for the security and prosperity of each individual, it is requisite to
make from the liberty of every other. To the legislator it is a kind of
perpetual lesson: serving at once as a corrective to his prejudices, and
as a check upon his passions. Is there a mischief which has escaped him?
in a natural arrangement, if at the same time an exhaustive one, he
cannot fail to find it. Is he tempted ever to force innocence within the
pale of guilt? the difficulty of finding a place for it advertises him
of his error. Such are the uses of a map of universal delinquency, laid
down upon the principle of utility: such the advantages, which the
legislator as well as the subject may derive from it. Abide by it, and
every thing that is arbitrary in legislation vanishes. An
evil-intentioned or prejudiced legislator durst not look it in the face.
He would proscribe it, and with reason: it would be a satire on his
laws.

LX. In the fourth place, a natural arrangement, governed as it is by a
principle which is recognised by all men, will serve alike for the
jurisprudence of all nations. In a system of proposed law, framed in
pursuance of such a method, the language will serve as a glossary by
which all systems of positive law might be explained, while the matter
serves as a standard by which they might be tried. Thus illustrated, the
practice of every nation might be a lesson to every other: and mankind
might carry on a mutual interchange of experiences and improvements as
easily in this as in every other walk of science. If any one of these
objects should in any degree be attained, the labour of this analysis,
severe as it has been, will not have been thrown away.\\

§ 5. Characters of the five classes

LXI. It has been mentioned as an advantage possessed by this method, and
not possessed by any other, that the objects comprised under it are cast
into groups, to which a variety of propositions may be applied in
common. A collection of these propositions, as applied to the several
classes, may be considered as exhibiting the distinctive characters of
each class. So many of these propositions as can be applied to the
offences belonging to any given class, so many properties are they found
to have in common: so many of these common properties as may
respectively be attributed to them, so many properties may be set down
to serve as \emph{characters} of the class. A collection of these
characters it may here be proper to exhibit. The more of them we can
bring together, the more clearly and fully will the nature of the
several classes, and of the offences they are composed of, be
understood.

LXII. Characters of Class 1; composed of PRIVATE offences, or offences
against assignable \emph{individuals.}\\
1. When arrived at their last stage (the stage of \emph{consumation)}
they produce, all of them, a primary mischief as well as a secondary.\\
2. The individuals whom they affect in the first instance (that is, by
their primary mischief) are constantly \emph{assignable.} This extends
to all; to \emph{attempts} and \emph{preparations,} as well as to such
as have arrived at the stage of consummation.\\
3. Consequently they admit of \emph{compensation:} in which they differ
from the offences of all the other classes, as such.\\
4. They admit also of \emph{retaliation;} in which also they differ from
the offences of all the other classes.\\
5. There is always some person who has a natural and peculiar interest
to prosecute them. In this they differ from self-regarding offences:
also from semi-public and public ones; except in as far as the two
latter may chance to involve a private mischief.\\
6. The mischief they produce is obvious: more so than that of
semi-public offences: and still more so than that of self-regarding
ones, or even public.\\
7. They are every where, and must ever be, obnoxious to the censure of
the world: more so than semi-public offences as such; and still more so
than public ones.\\
8. They are more \emph{constantly} obnoxious to the censure of the world
than self-regarding offences: and would be so universally, were it not
for the influence of the two false principles; the principle of
asceticism, and the principle of antipathy.\\
9. They are less apt than semi-public and public offences to require
different descriptions in different states and countries: in which
respect they are much upon a par with self-regarding ones.\\
10. By certain circumstances of aggravation, they are liable to be
transformed into semi-public offences; and by certain others, into
public.\\
11. There can be no ground for punishing them, until they can be proved
to have occasioned, or to be about to occasion some particular mischief
to some particular individual. In this they differ from semi-public
offences, and from public.\\
12. In slight cases, \emph{compensation} given to the individual
affected by them may be a sufficient ground for remitting punishment:
for if the primary mischief has not been sufficient to produce any
alarm, the whole of the mischief may be cured by compensation. In this
also they differ from semi-public offences, and from public ones.

LXIII. Characters of Class 2; composed of SEMI-PUBLIC offences, or
offences affecting a whole subordinate \emph{class} of persons.\\
1. As such, they produce no primary mischief. The mischief they produce
consists of one or other or both branches of the secondary mischief
produced by offences against individuals, without the primary.\\
2. In as far as they are to be considered as belonging to this class,
the persons whom they affect in the first instance are not individually
assignable.\\
3. They are apt, however, to involve or terminate in some primary
mischief of the first order; which when they do, they advance into the
first class, and become private offences.\\
4. They admit not, as such, of compensation.\\
5. Nor of retaliation\\
6. As such, there is never any one particular individual whose exclusive
interest it is to prosecute them: a circle of persons may, however,
always be marked out, within which may be found some who have a greater
interest to prosecute than any who are out of that circle have.\\
7. The mischief they produce is in general pretty obvious: not so much
so indeed as that of private offences, but more so upon the whole than
that of self-regarding and public ones.\\
8. They are rather less obnoxious to the censure of the world than
private offences; but they are more so than public ones: they would also
be more so than self-regarding ones, were it not for the influence of
the two false principles, the principle of sympathy and antipathy, and
that of asceticism.\\
9. They are more apt than private and self-regarding offences to require
different descriptions in different countries: but less so than public
ones.\\
10. There may be ground for punishing them before they have been proved
to have occasioned, or to be about to occasion, mischief to any
particular individual; which is not the case with private offences.\\
11. In no cases can satisfaction given to any particular individual
affected by them be a sufficient ground for remitting punishment: for by
such satisfaction it is but a part of the mischief of them that is
cured. In this they differ from private offences; but agree with public.

LXIV. Characters of Class 3; consisting of SELF REGARDING offences:
offences against \emph{one's self.} \emph{\\
}1. In individual instances it will often be questionable, whether they
are productive of any primary mischief at all: secondary, they produce
none.\\
2. They affect not any other individuals, assignable or not assignable,
except in as far as they affect the offender himself; unless by
possibility in particular cases; and in a very slight and distant manner
the whole state.\\
3. They admit not, therefore, of \emph{compensation,}\\
4. Nor of \emph{retaliation.}\\
5. No person has naturally any peculiar interest to prosecute them:
except in as far as in virtue of some \emph{connection} he may have with
the offender, either in point of \emph{sympathy} or of \emph{interest,}
a mischief of the \emph{derivative} kind may happen to devolve upon
him.\\
6. The mischief they produce is apt to be unobvious and in general more
questionable than that of any of the other classes.\\
7. They are however apt, many of them, to be more obnoxious to the
censure of the world than public offences; owing to the influence of the
two false principles; the principle of asceticism, and the principle of
antipathy. Some of them more even than semi-public, or even than private
offence.\\
8. They are less apt than offences of any other class to require
different descriptions in different states and countries,\\
9. Among the inducements to punish them, antipathy against the offender
is apt to have a greater share than sympathy for the public.\\
10. The best plea for punishing them is founded on a faint probability
there may be of their being productive of a mischief, which, if real,
will place them in the class of public ones: chiefly in those divisions
of it which are composed of offences against population, and offences
against the national wealth.

LXV. Characters of Class 4; consisting of PUBLIC offences, offences
against the \emph{state} in general.\\
1. As such, they produce not any primary mischief; and the secondary
mischief they produce, which consists frequently of danger without
alarm, though great in \emph{value,} is in \emph{specie} very
indeterminate.\\
2. The individuals whom they affect, in the first instance, are
constantly unassignable; except in as far as by accident they happen to
involve or terminate in such or such offences against individuals.\\
3. Consequently they admit not of compensation.\\
4. Nor of retaliation.\\
5. Nor is there any person who has naturally any particular interest to
prosecute them; except in as far as they appear to affect the power, or
in any other manner the private interest, of some person in authority.\\
6. The mischief they produce, as such, is comparatively unobvious; much
more so than that of private offences, and more so likewise, than that
of semi-public ones.\\
7. They are, as such, much less obnoxious to the censure of the world,
than private offences; less even than semi-public, or even than
self-regarding offences; unless in particular cases, through sympathy to
certain persons in authority, whose private interests they may appear to
affect.\\
8. They are more apt than any of the other classes to admit of different
descriptions, in different states and countries.\\
9. They are constituted, in many cases, by some circumstances of
aggravation superadded to a private offence: and therefore, in these
cases, involve the mischief and exhibit the other characters belonging
to both classes. They are however, even in such cases, properly enough
ranked in the 4th class, inasmuch as the mischief they produce in virtue
of the properties which aggregate them to that class, eclipses and
swallows up that which they produce in virtue of those properties which
aggregate them to the 1st.\\
10. There may be sufficient ground for punishing them, without their
being proved to have occasioned, or to be about to occasion, any
particular mischief to any particular individual. In this they differ
from private offences, but agree with semi-public ones. Here, as in
semi-public offences, the \emph{extent} of the mischief makes up for the
\emph{uncertainty} of it.\\
11. In no case can satisfaction, given to any particular individual
affected by them, be a sufficient ground for remitting punishment. In
this they differ from private offences; but agree with semi-public.

LXVI. Characters of Class 5, or appendix: composed of MULTIFORM or
ANOMALOUS offences; and containing offences by FALSEHOOD, and offences
concerning TRUST.\\
1. Taken collectively, in the parcels marked out by their popular
appellations, they are incapable of being aggregated to any systematical
method of distribution, grounded upon the mischief of the offence.\\
2. They may, however, be thrown into sub-divisions, which may be
aggregated to such a method of distribution.\\
3. These sub-divisions will naturally and readily rank under the
divisions of the several preceding classes of this system.\\
4. Each of the two great divisions of this class spreads itself in that
manner over all the preceding classes.\\
5. In some acts of this class, the distinguishing circumstance which
constitutes the essential character of the offence, will in some
instances enter necessarily, in the character of a criminative
circumstance, into the constitution of the offence; insomuch that,
without the intervention of this circumstance, no offence at all, of
that denomination, can be committed. In other instances, the offence may
subsist without it; and where it interferes, it comes in as an
accidental independent circumstance, capable of constituting a ground of
aggravation.

\chapter{Of the Limits of the Penal Branch of
Jurisprudence}

§1. Limits between Private Ethics and the Art of legislation\\
I. So much for the division of offenses in general. Now an offense is an
act prohibited, or (what comes to the same thing) an act of which the
contrary is commanded, by the law: and what is it that the law can be
employed in doing, besides prohibiting and commanding? It should seem
then, according to this view of the matter, that were we to have settled
what may be proper to be done with relation to offences, we should
thereby have settled every thing that may be proper to be done in the
way of law. Yet that branch which concerns the method of dealing with
offences, and which is termed sometimes the \emph{criminal,} sometimes
the \emph{penal,} branch, is universally understood to be but one out of
two branches which compose the whole subject of the art of legislation;
that which is termed the \emph{civil} being the other. Between these two
branches then, it is evident enough, there cannot but be a very intimate
connection; so intimate is it indeed, that the limits between them are
by no means easy to mark out. The case is the same in some degree
between the whole business of legislation (civil and penal branches
taken together) and that of private ethics. Of these several limits
however it will be in a manner necessary to exhibit some idea: lest, on
the one hand, we should seem to leave any part of the subject that
\emph{does} belong to as untouched, or, on the other hand, to deviate on
any side into a track which does not belong to us.\\
In the course of this enquiry, that part of it I mean which concerns the
limits between the civil and the penal branch of law, it will be
necessary to settle a number of points, of which the connection with the
main question might not at first sight be suspected. To ascertain what
sort of a thing \emph{a} law is; what the \emph{parts} are that are to
be found in it; what it must contain in order to be \emph{complete;}
what the connection is between that part of a body of laws which belongs
to the subject of \emph{procedure} and the rest of the law at
large:'' all these, it will be seen, are so many problems, which
must be solved before any satisfactory answer can be given to the main
question above mentioned.\\
Nor is this their only use: for it is evident enough, that the notion of
a complete law must first be fixed, before the legislator can in any
case know what it is he has to do, or when his work is done.

II. Ethics at large may be defined, the art of directing men's actions
to the production of the greatest possible quantity of happiness, on the
part of those whose interest is in view.

III. What then are the actions which it can be in a man's power to
direct? They must be either his own actions, or those of other agents.
Ethics, in as far as it is the art of directing a man's own actions, may
be styled the \emph{art of self-government,} or \emph{private ethics.}

IV. What other agents then are there, which, at the same time that they
are under the influence of man's direction, are susceptible of
happiness. They are of two sorts:\\
1. Other human beings who are styled persons.\\
2. Other animals, which, on account of their interests having been
neglected by the insensibility of the ancient jurists, stand degraded
into the class of \emph{things.} As to other human beings, the art of
directing their actions to the above end is what we mean, or at least
the only thing which, upon the principle of utility, we \emph{ought} to
mean, by the art of government: which, in as far as the measures it
displays itself in are of a permanent nature, is generally distinguished
by the name of \emph{legislation:} as it is by that of
\emph{administration,} when they are of a temporary nature, determined
by the occurrences of the day.

V. Now human creatures, considered with respect to the maturity of their
faculties, are either in an \emph{adult,} or in a \emph{non-adult}
state. The art of government, in as far as it concerns the direction of
the actions of persons in a non-adult state, may be termed the art of
\emph{education.} In as far as this business is entrusted with those
who, in virtue of some private relationship, are in the main the best
disposed to take upon them, and the best able to discharge, this office,
it may be termed the art of \emph{private education:} in as far as it is
exercised by those whose province it is to superintend the conduct of
the whole community, it may be termed the art of \emph{public
education.}

VI. As to ethics in general, a man's happiness will depend, in the first
place, upon such parts of his behaviour as none but himself are
interested in; in the next place, upon such parts of it as may affect
the happiness of those about him. In as far as his happiness depends
upon the first-mentioned part of his behaviour, it is said to depend
upon his \emph{duty to himself.} Ethics then, in as far as it is the art
of directing a man's actions in this respect, may be termed the art of
discharging one's duty to one's self: and the quality with which a man
manifests by the discharge of this branch of duty (if duty it is to be
called) is the of \emph{prudence}. In as far as his happiness, and that
of any other person or persons whose interests are considered, depends
upon such parts of his behaviour as may affect the interests of those
about, it may be said to depend on his \emph{duty to others ;}or, to use
a phrase now somewhat antiquated, his \emph{duty to his neighbour.}
Ethics then, in as far as it is the art of directing a man's actions in
this respect, may be termed the art of discharging one's duty to one's
neighbour. Now the happiness of one's neighbour may be consulted in two
ways:\\
1. In a negative way, by forbearing to diminish it.\\
2. In a positive way, by studying to increase it. A man's duty to his
neighbour is accordingly partly negative and partly positive: to
discharge the negative branch of it, is \emph{probity:} to discharge the
positive branch, \emph{beneficence.}

VII. It may here be asked, How it is that upon the principle of private
ethics, legislation and religion out of the question, a man's happiness
depends upon such parts of his conduct as affect, immediately at least,
the happiness of no one but himself: this is as much as to ask, What
motives (independent of such as legislation and religion may chance to
furnish) can one man have to consult the happiness of another by what
motives, or, which comes to the same thing, by what obligations, can he
be bound to obey the dictates of \emph{probity} and \emph{beneficence.}
In answer to this, it cannot but be admitted, that the only interests
which a man at all times and upon all occasions is sure to find
\emph{adequate} motives for consulting, are his own. Notwithstanding
this, there are no occasions in which a man has not some motives for
consulting the happiness of other men. In the first place, he has, on
all occasions, the purely social motive of sympathy or benevolence: in
the next place, he has, on most occasions, the semi-social motives of
love of amity and love of reputation. The motive of sympathy will act
upon him with more or less effect, according to the \emph{bias} of his
sensibility: the two other motives, according to a variety of
circumstances, principally according to the strength of his intellectual
powers, the firmness and steadiness of his mind, the quantum of his
moral sensibility, and the characters of the people he has to deal with.

VIII. Now private ethics has happiness for its end: and legislation can
have no other. Private ethics concerns every member, that is, the
happiness and the actions of every member, of any community that can be
proposed; and legislation can concern no more. Thus far, then, private
ethics and the art of legislation go hand in hand. The end they have, or
ought to have, in view, is of the same nature. The persons whose
happiness they ought to have in view, as also the persons whose conduct
they ought to be occupied in directing, are precisely the same. The very
acts they ought to be conversant about, are even in a great, measure the
same. Where then lies the difference? In that the acts which they ought
to be conversant about, though in a \emph{great measure,} are not
\emph{perfectly and throughout} the same. There is no case in which a
private man ought not to direct his own conduct to the production of his
own happiness, and of that of his fellow-creatures: but there are cases
in which the legislator ought not (in a direct way at least, and by
means of punishment applied immediately to particular \emph{individual}
acts) to attempt to direct the conduct of the several other members of
the community. Every act which promises to be beneficial upon the whole
to the community (himself included) each individual ought to perform of
himself: but it is not every such act that the legislator ought to
compel him to perform. Every act which promises to be pernicious upon
the whole to the community (himself included) each individual ought to
abstain from of him: but it is not every such act that the legislator
ought to compel him to abstain from.

IX. Where then is the line to be drawn?'' We shall not have far
to seek for it. The business is to give an idea of the cases in which
ethics ought, and in which legislation ought not (in a direct manner at
least) to interfere. If legislation interferes in a direct manner, it
must be by punishment. Now the cases in which punishment, meaning the
punishment of the political sanction, ought not to be inflicted, have
been already stated. 2. If then there be any of these cases in which,
although legislation ought not, private ethics does or ought to
interfere, these cases will serve to point out the limits between the
two arts or branches of science. These cases. it may be remembered, are
of four sorts:\\
1. Where punishment would be groundless.\\
2. Where it would be inefficacious.\\
3. Where it would be unprofitable.\\
4. Where it would be needless. Let us look over all these cases, and see
whether in any of them there is room for the interference of private
ethics, at the same time that there is none for the direct interference
of legislation.

X. 1. First then, as to the cases where punishment would be
\emph{groundless.} In these cases it is evident, that the restrictive
interference of ethics would be groundless too. It is because, upon the
whole, there is no evil in the act, that legislation ought not to
endeavour to prevent it. No more, for the same reason, ought private
ethics.

XI. 2. As to the cases in which punishment would be
\emph{inefficacious.} These, we may observe, may be divided into two
sets or classes. The first do not depend at all upon the natured of the
act: they turn only upon a defect in the timing of the punishment. The
punishment in question is no more than what, for any thing that appears,
ought to have been applied to the act in question. It ought, however, to
have been applied at a different time; viz., not till after it had been
properly denounced. These are the cases of an \emph{ex-post-facto} law;
of a judicial sentence beyond the law; and of a law not sufficiently
promulgated. The acts here in question then might, for anything that
appears, come properly under the department even of coercive
legislation: of course do they under that of private ethics. As to the
other set of cases, in which punishment would be inefficacious; neither
do these depend upon the nature of the act, that is, of the \emph{sort}
of act: they turn only upon some extraneous \emph{circumstances}, with
which an act of \emph{any} sort may chance to be accompanied. These,
however, are of such a nature as not only to exclude the application of
legal punishment, but in general to leave little room for the influence
of private ethics. These are the cases where the will could not be
deterred from any act, even by the extraordinary force of artificial
punishment: as in the cases of extreme infancy, insanity, and perfect
intoxication: of course, therefore, it could not by such slender and
precarious force as could be applied by private ethics. The case is in
this respect the same, under the circumstances of unintentionality with
respect to the event of the action, unconsciousness with regard to the
circumstances, and mis-supposal with regard to the existence of
circumstances which have not existed; as also where the force, even of
extraordinary punishment, is rendered inoperative by the superior force
of a physical danger or threatened mischief. It is evident, that in
these cases, if the thunders of the law prove impotent, the whispers of
simple morality can have but little influence.

XII. 3. As to the cases where punishment would be \emph{unprofitable.}
These are the cases which constitute the great field for the exclusive
interference of private ethics. When a punishment is unprofitable, or in
other words too expensive, it is because the evil of the punishment
exceeds that of the offence. Now the evil of the punishment, we may
remember, is distinguishable into four branches:\\
1. The evil of coercion, including constraint or restraint, according as
the act commanded is of the positive kind or the negative.\\
2. The evil of apprehension.\\
3. The evil of sufferance.\\
4. The derivative evils resulting to persons in \emph{connection} with
those by whom the three above-mentioned original evils are sustained.
Now with respect to those original evils, the persons who lie exposed to
them may be two very different sets of persons. In the first place,
persons who may have actually committed, or been prompted to commit, the
acts really meant to be prohibited. In the next place, persons who may
have performed, or been prompted to perform, such other acts as they
fear may be in danger of being involved in the punishment designed only
for the former. But of these two sets of acts, it is the former only
that are pernicious: it is, therefore, the former only that it can be
the business of private ethics to endeavour to prevent. The latter being
by the supposition not mischievous, to prevent them is what it can no
more be the business of ethics to endeavour at, than of legislation. It
remains to show how it may happen, that there should be acts really
pernicious, which, although they may very properly come under the
censure of private ethics, may yet be no fit objects for the legislator
to control.

XIII. Punishment then, as applied to delinquency, may be unprofitable in
both or either of two ways:\\
1. By the expense it would amount to, even supposing the application of
it to be confined altogether to delinquency:\\
2. By the danger there may be of its involving the innocent in the fate
designed only for the guilty.\\
First then, with regard to the cases in which the expense of the
punishment, as applied to the guilty, would outweigh the profit to be
made by it. These cases, it is evident, depend upon a certain proportion
between the evil of the punishment and the evil of the offence. Now were
the offence of such a nature, that a punishment which, in point of
\emph{magnitude,} should but just exceed the profit of it, would be
sufficient to prevent it, it might be rather difficult perhaps to find
an instance in which such punishment would clearly appear to be
unprofitable. But the fact is, there are many cases in which a
punishment, in order to have any chance of being efficacious, must, in
point of magnitude, be raised a great deal above that level. Thus it is,
wherever the danger of detection is, or, what comes to the same thing,
is likely to appear to be, so small, as to make the punishment appear in
a high degree uncertain. In this case it is necessary, as has been
shown, if punishment be at all applied, to raise it in point of
magnitude as much as it falls short in point of certainty. It is
evident, however, that all this can be but guesswork: and that the
effect of such a proportion will be rendered precarious, by a variety of
circumstances: by the want of sufficient promulgation on the part of the
laws: by the particular circumstances of the temptation: and by the
circumstances influencing the sensibility of the several individuals who
are exposed to it. Let the \emph{seducing} motives be strong, the
offence then will at any rate be frequently committed. Now and then
indeed, owing to a coincidence of circumstances more or less
extraordinary, it will be detected, and by that means punished. But for
the purpose of example, which is the principal one, an act of
punishment, considered in itself, is of no use: what use it can be of,
depends altogether upon the expectation it raises of similar punishment,
in future cases of similar delinquency. But this future punishment, it
is evident, must always depend upon detection. If then the want of
detection is such as must in general (especially to eyes fascinated by
the force of the seducing motives) appear too improbable to be reckoned
upon, the punishment, though it should be inflicted, may come to be of
no use. Here then will be two opposite evils running on at the same
time, yet neither of them reducing the quantum of the other: the evil of
the disease and the evil of the painful and inefficacious remedy. It
seems to be partly owing to some such considerations, that fornication,
for example, or the illicit commerce between the sexes, has commonly
either gone altogether unpunished, or been punished in a degree inferior
to that in which, on other accounts, legislators might have been
disposed to punish it.

XIV. Secondly, with regard to the cases in which political punishment,
as applied to delinquency, may be unprofitable, in virtue of the danger
there may be of its involving the innocent in the fate designed only for
the guilty. Whence should this danger then arise? From the difficulty
there may be of fixing the idea of the guilty action: that is. of
subjecting it to such a definition as shall be clear and precise enough
to guard effectively against misapplication. This difficulty may arise
from either of two sources: the one permanent, to wit, the nature of the
\emph{actions} themselves: the other occasional, I mean the qualities of
the \emph{men} who may have to deal with those actions in the way of
government. In as far as it arises from the latter of these sources, it
may depend partly upon the use which the \emph{legislator} may be
\emph{able} to make of language; partly upon the use which, according to
the apprehension of the legislators the \emph{judge} may be
\emph{disposed} to make of it. As far as legislation is concerned, it
will depend upon the degree of perfecting to which the arts of language
may have been carried, in the first place, in the nation in general; in
the next place. by the legislator in particular. It is to a sense of
this difficulty, as it should seem, that we may attribute the caution
with which most legislators have abstained from subjecting to censure,
on the part of the law, such actions as come under the notion of
rudeness, for example, or treachery, or ingratitude. The attempt to
bring acts of so vague and questionable a nature under the control of
law, will argue either a very immature age, in which the difficulties
which give birth to that danger are not descried; or a very enlightened
age, in which they are overcome.

XV. For the sake of obtaining the clearer idea of the limits between the
art of legislation and private ethics, it may now be time to call to
mind the distinctions above established with regard to ethics in
general. The degree in which private ethics stands in need of the
assistance of legislation is different in the three branches of duty
above distinguished. Of the rules of moral duty, those which seem to
stand least in need of the assistance of legislation are the rules of
\emph{prudence.} It can only be through some defect on the part of the
understanding, if a man be ever deficient in point of duty to himself.
If he does wrong, there is nothing else that it can be owing to but
either some \emph{inadvertence} or some \emph{mis-supposal} with regard
to the circumstances on which his happiness depends. It is a standing
topic of complaint, that a man knows too little of himself. Be it so:
but is it so certain that the legislator must know more? It is plain,
that of individuals the legislator can know nothing: concerning those
points of conduct which depend upon the particular circumstances of each
individual, it is plain, therefore, that he can determine nothing to
advantage. It is only with respect to those broad lines of conduct in
which all persons, or very large and permanent descriptions of persons,
may be in a way to engage, that he can have any pretense for
interfering; and even here the propriety of his interference will, in
most instances, lie very open to dispute. At any rate, he must never
expect to produce a perfect compliance by the mere force of the sanction
of which he is himself the author. All he can hope to do, is to increase
the efficacy of private ethics, by giving strength and direction to the
influence of the moral sanction. With what chance of success, for
example, would a legislator go about to extirpate drunkenness and
fornication by dint of legal punishment? Not all the tortures which
ingenuity could invent would compass it: and, before he had made any
progress worth regarding, such a mass of evil would be produced by the
punishment, as would exceed, a thousandfold, the utmost possible
mischief of the offence.. The great difficulty would be in the procuring
evidence; an object which could not be attempted, with any probability
of success, without spreading dismay through every family, tearing the
bonds of sympathy asunder, and rooting out the influence of all the
social motives. All that he can do then, against offences of this
nature, with any prospect of advantage, in the way of direct
legislation, is to subject them, in cases of notoriety, to a slight
censure, so as thereby to cover them with a slight shade of artificial
disrepute.

XVI. It may be observed, that with regard to this branch of duty,
legislators have, in general, been disposed to carry their interference
full as far as is expedient. The great difficulty here is, to persuade
them to confine themselves within bounds. A thousand little passions and
prejudices have led them to narrow the liberty of the subject in this
line, in cases in which the punishment is either attended with no profit
at all, or with none that will make up for the expense.

XVII. The mischief of this sort of interference is more particularly
conspicuous in the article of religion. The reasoning, in this case, is
of the following stamp. There are certain errors, in matters of belief,
to which all mankind are prone: and for these errors in judgment, it is
the determination of a Being of infinite benevolence, to punish them
with an infinity of torments. But from these errors the legislator
himself is necessarily free: for the men, who happen to be at hand for
him to consult with, being men perfectly enlightened, unfettered, and
unbiased, have such advantages over all the rest of the world, that when
they sit down to enquire out the truth relative to points so plain and
so familiar as those in question, they cannot fail to find it. This
being the case, when the sovereign sees his people ready to plunge
headlong into an abyss of fire, shall he not stretch out a hand to save
them? Such, for example, seems to have been the train of reasoning, and
such the motives, which led Lewis the XIVth into those coercive measures
which he took for the conversion of heretics and the confirmation of
true believers. The groundwork, pure sympathy and loving-kindness: the
superstructure, all the miseries which the most determined malevolence
could have devised.\\
But of this more fully in another place.

XVIII. The rules of \emph{probity} are those, which in point of
expediency stand most in need of assistance on the part of the
legislator, and in which, in point of fact, his interference has been
most extensive. There are few cases in which it \emph{would} be
expedient to punish a man for hurting \emph{himself:} but there are few
cases, if any, in which it would not be expedient to punish a man for
injuring his neighbour. With regard to that branch of probity which is
opposed to offences against property, private ethics depends in a manner
for its very existence upon legislation. Legislation must first
determine what things are to be regarded as each man's property, before
the general rules of ethics, on this head, can have any particular
application. The case is the same with regard to offences against the
state. Without legislation there would be no such thing as a
\emph{state:} no particular persons invested with powers to be exercised
for the benefit of the rest. It is plain, therefore, that in this branch
the interference of the legislator cannot any where be dispensed with.
We must first know what are the dictates of legislation, before we can
know what are the dictates of private ethics.

XIX. As to the rules of beneficence, these, as far as concerns matters
of detail, must necessarily be abandoned in great measure to the
jurisdiction of private ethics. In many cases the beneficial quality of
the act depends essentially upon the disposition of the agent; that is,
upon the motive by which he appears to have been prompted to perform it:
upon their belonging to the head of sympathy, love of amity, or love of
reputation; and not to any head of self-regarding motives. brought into
play by the force of political constraint: in a word, upon their being
such as denominate his conduct \emph{free} and \emph{voluntary,}
according to one of the many senses given to those ambiguous
expressions. The limits of the law on this head seem, however, to be
capable of being extended a good deal farther than they seem ever to
have been extended hitherto. In particular, in cases where the person is
in danger, why should it not be made the duty of every man to save
another from mischief, when it can be done without prejudicing himself,
as well as to abstain from bringing it on him? This accordingly is the
idea pursued in the body of the work.

XX. To conclude this section, let us recapitulate and bring to a point
the difference between private ethics. considered as an art or science,
on the one hand, and that branch of jurisprudence which contains the art
or science of legislation, on the other. Private ethics teaches how each
man may dispose himself to pursue the course most conducive to his own
happiness, by means of such motives as offer of themselves: the art of
legislation (which may be considered as one branch of the science of
jurisprudence) teaches how a multitude of men, composing a community,
may be disposed to pursue that course which upon the whole is the most
conducive to the happiness of the whole community, by means of motives
to be applied by the legislator.\\
We come now to exhibit the limits between penal and civil jurisprudence.
For this purpose it may be of use to give a distinct though summary view
of the principal branches into which jurisprudence, considered in its
utmost extent, is wont to be divided.\\

§ 2. Jurisprudence, its branches\\
XXI. Jurisprudence is a fictitious entity: nor can any meaning be found
for the word, but by placing it in company with some word that shall be
significative of a real entity. To know what is meant by jurisprudence,
we must know, for example, what is meant by a book of jurisprudence. A
book of jurisprudence can have but one or the other of two objects:\\
1. To ascertain what the \emph{law} is:\\
2. to ascertain what it ought to be.\\
In the former case it may be styled a book of \emph{expository}
jurisprudence; in the latter, a book of \emph{censorial} jurisprudence:
or, in other words, a book on\\
the \emph{art of legislation.}

XXII. A book of expository jurisprudence, is either \emph{authoritative}
or \emph{unauthoritative.} It is styled authoritative, when it is
composed by him who, by representing the state of the law to be so and
so, causeth it so to be; that is, of the legislator himself:
unauthoritative, when it is the work of any other person at large.

XXIII. Now \emph{law,} or \emph{the law,} taken indefinitely, is an
abstract and collective term; which, when it means any thing, can mean
neither more nor less than the sum total of a number of individual laws
taken together. It follows, that of whatever other modifications the
subject of a book of jurisprudence is susceptible, they must all of them
be taken from some circumstance or other of which such individual laws,
or the assemblages into which they may be sorted, are susceptible. The
circumstances that have given rise to the principal branches of
jurisprudence we are wont to hear of, seem to be as follows:\\
1. The \emph{extent} of the laws in question in point of dominion.\\
2. The \emph{political quality} of the persons whose conduct they
undertake to regulate.\\
3. The \emph{time} of their being in force.\\
4. The manner in which they are \emph{expressed.}\\
5. The concern which they have with the article of \emph{punishment}

XXIV. In the first place, in point of extent, what is delivered
concerning the laws in question, may have reference either to the laws
of such or such a nation or nations in particular, or to the laws of all
nations whatsoever: in the first case, the book may be said to relate to
\emph{local,} in the other, to \emph{universal jurisprudence.} \emph{\\
}Now of the infinite variety of nations there are upon the earth, there
are no two which agree exactly in their laws: certainly not in the
whole: perhaps not even in any single article: and let them agree today,
they would disagree tomorrow. This is evident enough with regard to the
\emph{substance} of the laws: and it would be still more extraordinary
if they agreed in point of \emph{form;} that is, if they were conceived
in precisely the same strings of words. What is more, as the languages
of nations are commonly different, as well as their laws, it is seldom
that, strictly speaking, they have so much as a single \emph{word} in
common. However, among the words that are appropriated to the subject of
law, there are some that in all languages are pretty exactly
correspondent to one another: which comes to the same thing nearly as if
they were the same. Of this stamp, for example, are those which
correspond to the words \emph{power, right, obligation, liberty,} and
many others.\\
It follows, that if there are any books which can, properly speaking, be
styled books of universal jurisprudence, they must be looked for within
very narrow limits. Among such as are expository, there can be none that
are authoritative: nor even, as far as the \emph{substance} of the laws
is concerned, any that are unauthoritative. To be susceptible of an
universal application, all that a book of the expository kind can have
to treat of, is the import of words: to be, strictly speaking,
universal, it must confine itself to terminology. Accordingly the
definitions which there has been occasion here and there to intersperse
in the course of the present work, and particularly the definition
hereafter given of the word \emph{law,} may be considered as matter
belonging to the head of universal jurisprudence. Thus far in strictness
of speech: though in point of usage, where a man, in laying down what he
apprehends to be the law, extends his views to a few of the nations with
which his own is most connected, it is common enough to consider what he
writes as relating to universal jurisprudence.\\
It is in the censorial line that there is the greatest room for
disquisitions that apply to the circumstances of all nations alike: and
in this line what regards the substance of the laws in question is as
susceptible of an universal application, as what regards the words. That
the laws of all nations, or even of any two nations, should coincide in
all points, would be as ineligible as it is impossible: some leading
points, however, there seem to be, in respect of which the laws of all
civilized nations might, without inconvenience, be the same. To mark out
some of these points will, as far as it goes, be the business of the
body of this work.

XXV. In the second place, with regard to the \emph{political quality} of
the persons whose conduct is the object of the law. These may, on any
given occasion, be considered either as members of the same state, or as
members of different states: in the first ease, the law may be referred
to the head of \emph{internal,} in the second case, to that of
\emph{international} jurisprudence.\\
Now as to any transactions which may take place between individuals who
are subjects of different states, these are regulated by the internal
laws, and decided upon by the internal tribunals, of the one or the
other of those states: the case is the same where the sovereign of the
one has any immediate transactions with a private member of the other:
the sovereign reducing himself, \emph{pro re natâ,} to the condition of
a private person, as often as he submits his cause to either tribunal;
whether by claiming a benefit, or defending himself against a burthen.
There remain then the mutual transactions between sovereigns, as such,
for the subject of that branch of jurisprudence which may be properly
and exclusively termed \emph{international.} \emph{\\
}With what degree of propriety rules for the conduct of persons of this
description can come under the appellation of \emph{laws,} is a question
that must rest till the nature of the thing called \emph{a law} shall
have been more particularly unfolded.\\
It is evident enough, that international jurisprudence may, as well as
internal, be censorial as well as expository, unauthoritative as well as
authoritative.

XXVI. Internal jurisprudence, again, may either concern all the members
of a state indiscriminately, or such of them only as are connected in
the way of residence, or otherwise, with a particular district.
Jurisprudence is accordingly sometimes distinguished into
\emph{national} and \emph{provincial.} But as the epithet
\emph{provincial} is hardly applicable to districts so small as many of
those which have laws of their own are wont to be, such as towns,
parishes, and manors; the term \emph{local} (where universal
jurisprudence is plainly out of the question) or the term
\emph{particular,} though this latter is not very characteristic, might
either of them be more commodious.

XXVII. Thirdly, with respect to \emph{time.} In a work of the expository
kind, the laws that are in question may either be such as are still in
force at the time when the book is writing, or such as have ceased to be
in force. In the latter case the subject of it might be termed
\emph{ancient;} in the former, \emph{present} or \emph{living}
jurisprudence: that is, if the substantive \emph{jurisprudence,} and no
other, must at any rate be employed, and that with an epithet in both
cases. But the truth is, that a book of the former kind is rather a book
of history than a book of jurisprudence; and, if the word
\emph{jurisprudence} be expressive of the subject, it is only with some
such words as \emph{history} or \emph{antiquities} prefixed. And as the
laws which are any where in question are supposed, if nothing appears to
the contrary, to be those which are in force, no such epithet as that of
\emph{present} or \emph{living} commonly appears.\\
Where a book is so circumstanced, that the laws which form the subject
of it, though in force at the time of its being written, are in force no
longer, that book is neither a book of living jurisprudence, nor a book
on the history of jurisprudence: it is no longer the former, and it
never was the latter. It is evident that, owing to the changes which
from time to time must take place, in a greater or less degree, in every
body of laws, every book of jurisprudence, which is of an expository
nature, must in the course of a few years, come to partake more or less
of this condition.\\
The most common and most useful object of a history of jurisprudence, is
to exhibit the circumstances that have attended the establishment of
laws actually in force. But the exposition of the dead laws which have
been superseded, is inseparably interwoven with that of the living ones
which have superseded them. The great use of both these branches of
\emph{science,} is to furnish examples for the \emph{art} of
legislation.

XXVIII. Fourthly, in point of \emph{expression,} the laws in question
may subsist either in the form of \emph{statute} or in that of
\emph{customary} law. As to the difference between these two branches
(which respects only the article of form or expression) it cannot
properly be made appear till some progress has been made in the
definition of a law.

XXIX. Lastly, The most intricate distinction of all, and that which
comes most frequently on the carpet, is that which is made between the
\emph{civil} branch of jurisprudence and the \emph{penal,} which latter
is wont, in certain circumstances, to receive the name of
\emph{criminal.}\\
What is a penal code of laws? What a civil code? Of what nature are
their contents? Is it that there are two sorts of laws, the one penal
the other civil, so that the laws in a penal code are all penal laws,
while the laws in a civil code are all civil laws? Or is it, that in
every law there is some matter which is of a penal nature, and which
therefore belongs to the penal code; and at the same time other matter
which is of a civil nature, and which therefore belongs to the civil
code? Or is it, that some laws belong to one code or the other
exclusively, while others are divided between the two? To answer these
questions in any manner that shall be tolerably satisfactory, it will be
necessary to ascertain what \emph{a law} is; meaning one entire but
single law: and what are the parts into which a law, as such, is capable
of being distinguished: or, in other words, to ascertain what the
properties are that are to be found in every object which can with
propriety receive the appellation of a law. This then will be the
business of the third and fourth sections: what concerns the import of
the word \emph{criminal,} as applied to law, will be discussed
separately in the fifth.

\textbf{Notes}

1. For example.'' It is worse to lose time than simply not to
gain.'' A loss falls the lighter by being divided.'' The
suffering, of a person hurt in gratification of enmity, is greater than
the gratification produced by the same course.'' These, and a
few others which he will have occasion to exhibit at the head of another
publication, having the same claim to the appellation of axioms, as
those given by mathematicians under that name; since, referring to
universal experience as their immediate basis, they are incapable of
demonstration, and require only to be developed and illustrated, in
order to be recognised as incontestable.\\
2. \emph{A Fragment on Government,} etc., reprinted 1822.\\
3. Such as obligation, right, power, possession, title, exemption,
immunity, franchise, privilege, nullity, validity, and the like.\\
4. See ch. xvi. {[}Division{]}, par. 42, 44.\\
5. Hume's \emph{History.\\
}6. Hume's \emph{History.\\
}7. For the reason, see chap. xi. {[}Dispositions{]}, par. xvii. note.\\
8. See ch. iv. and ch. vi. par. xxi.\\
9. See B. I. tit. {[}Offences against Religion{]}\\
10. Here ends the original work, in the state into which it was brought
in 5 November, 1780. What follows is now added in January, 1789.

The third, fourth, and fifth sections intended, as expressed in the
text, to have been added to this chapter, will not here, nor now be
given; because to give them in a manner tolerably complete and
satisfactory, might require a considerable volume. This volume will form
a work of itself, closing the series of works mentioned in the
preface.\\
What follows here may serve to give a slight intimation of the nature of
the task, which such a work will have to achieve: it will at the same
time furnish, not any thing like a satisfactory answer to the questions
mentioned in the text, but a slight and general indication of the course
to be taken for giving them such an answer.

II. What is a law? What the parts of a law? The subject of these
questions it is to be observed, is the \emph{logical,} the \emph{ideal,}
the \emph{intellectual} whole not the \emph{physical} one: the
\emph{law,} and not the \emph{statute.} An enquiry, directed to the
latter sort of object, could neither admit of difficulty nor afford
instruction. In this sense whatever is given for law by the person or
persons recognized as possessing the power of making laws, is
\emph{law.} The \emph{Metamorphoses} of Ovid, if thus given, would be
law. So much as was embraced by one and the same act of authentication,
so much as received the touch of the sceptre at one stroke, is
\emph{one} law: a whole law, and nothing more. A statute of George II
made to substitute an \emph{or} instead of an \emph{and} in a former
statute is a complete law; a statute containing an entire body of laws,
perfect in all its parts, would not be more so. By the word \emph{law}
then, as often as it occurs in the succeeding pages is meant that ideal
object, of which the part, the whole, or the multiple, or an assemblage
of parts, wholes, and multiples mixed together, is exhibited by a
statute; not the statute which exhibits them.

III. Every law, when complete, is either of a \emph{coercive} or an
\emph{uncoercive} nature. A coercive law is a \emph{command.} An
uncoercive, or rather a \emph{discoercive,} law is the
\emph{revocation,} in whole or in part, of a coercive law.

IV. What has been termed a \emph{declaratory} law, sofar as it stands
distinguished from either a coercive or a discoercive law, is not
properly speaking a law. It is not the expression of an act of the will
exercised at the time: it is a mere notification of the existence of a
law, either of the coercive or the discoercive kind, as already
subsisting: of the existence of some document expressive of some act of
the will, exercised, not at the time, but at some former period. If it
does any thing more than give information of this fact, viz., of the
prior existence of a law of either the coercive or the discoercive kind,
it ceases \emph{pro tanto} to be what is meant by a declaratory law, and
assuming either the coercive or the discoercive quality.

V. Every coercive law creates an \emph{offence,} that is, converts an
act of some sort, or other into an offence. It is only by so doing that
it can \emph{impose obligation,} that it can \emph{produce coercion.}

VI. A law confining itself to the creation of an offence, and a law
commanding a punishment to be administered in case of the commision of
such an offense, are two distinct laws, not parts (as they seem to have
been generally accounted hitherto) of one and the same law. The acts
they command are altogether different; the persons they are addressed to
are altogether different. Instance, \emph{Let no man steal;} and,
\emph{Let the judge cause whoever is convicted of stealing to be
hanged.}\\
They might be styled, the former, a \emph{simply imperative} law; the
other a \emph{punitory:} but the punitory, if it commands the punishment
to be inflicted, and does not merely permit it, is as truly
\emph{imperative} as the other: only it is punitory besides, which the
other is not.

VII. A law of the discoercive kind, considered in itself, can have no
punitory law belonging to it: to receive the assistance and support of a
punitory in law, it must flrst receive that of a simply imperative or
coercive law, and it is to this latter that the punitory law will attach
itself, and not to the discoercive one. Example, discoercive law.
\emph{The sheriff has power to hang all such as the judge, proceeding in
due course of law, shall order him to hang.} Example of a coercive law,
made in support of the above discoereive one. \emph{Let no man hinder
the sheriff from hanging such as the judge, proceeding in due course of
law, shall order him to hang.} Example of a punitory law, made in
support of the above coercive one. \emph{Let the judge cause to be
imprisoned whosoever attempts to hinder the sheriff from hanging one,
whom the judge, proceeding in due course of law, has ordered him to
hang.}

VIII. But though a simply imperative law, and the punitory law attached
to it, are so far distinct laws, that the former contains nothing of the
latter, and the latter, in its direct tenor, contains nothing of the
former; yet by \emph{implication,} and that a necessary one, the
punitory does involve and include the import of the simply imperative
law to which it is appended. To say to the judge \emph{Cause to be
hanged whoever in due form of law is convicted of stealing,} is, though
not a direct, yet as intelligible~ a way of intimating to men in general
that they must not steal, as to say to them directly, \emph{Do not
steal:} and one sees, how much more likely to be efficacious.

IX. It should seem then, that, wherever a simply imperative law is to
have a punitory one appended to it, the former might be spared
altogether: in, which case, saving the exception (which naturally should
seem not likely to be a frequent one) of a law capable of answering its
purpose without such an appendage, there should be no occasion in the
whole body of the law for any other than punitory, or in other words
than penal, laws. And this, perhaps, would be the case, were it not for
the necessity of a large quantity of matter of the \emph{expository}
kind, of which we come now to speak.

X. It will happen in the instance of many, probably of most, possibly of
all commands endued with the force of a public law, that, in the
expression, given to such a command it shall be necessary to have
recourse to terms too complex in their signification to exhibit the
requisite ideas, without the assistance of a greater or less quantity of
matter of an expository nature. Such terms, like the symbols used in
algebraical notation, are rather substitutes and indexes to the terms
capable of themselves of exhibiting the ideas in question, than the real
and immediate representatives of those ideas. Take for instance the law,
\emph{Thou shalt not steal.} Such a command, were it to rest there,
could never sufficiently answer the purpose of a law. A word of so vague
and unexplicit a meaning cannot otherwise perform this office, than by
giving a general intimation of a variety of propositions, each
requiring, to convey it to the apprehension, a more particular and ample
assemblage of terms. Stealing, for example (according to a definition
not accurate enough for use, but sufficiently so for the present
purpose), is \emph{the taking of a thing which is another's, by one who
has no} TITLE \emph{so to do,} and \emph{is conscious of his having
none.} Even after this exposition, supposing it a correct one, can the
law beregarded as completely expressed? Certainly not. For what is meant
by \emph{a man's having a} TITLE \emph{to take a thing?} To be complete,
the law must have exhibited, amongst a multitude of other things, two
catalogues: the one of events to which it has given the quality of
\emph{conferring title} in such a case; the other of the events to which
it has given the quality of \emph{taking it away.} What follows? That
for a man to have \emph{stolen,} for a man to \emph{have had no title to
what he took,} either no one of the articles contained in the first of
those lists must have happened in his favour, or if there has, some one
of the number of those contained in the second must have happened to his
prejudice.

XI. Such then is the nature of a general law, that while the imperative
part of it, the \emph{punctum saliens} as it may be termed, of this
artificial body, shall not take up above two or three words, its
expository appendage, without which that imperative part could not
rightly perform its office, may occupy a considerable volume. But this
may equally be the case with a private order given in a family. Take for
instance one from a bookseller to his foreman. \emph{Remove, from this
shop to my new one, my whole stock, according to this printed
catalogue.'' Remove, from this shop to my new one, my whole
stock,} is the imperative matter of this order; the catalogue referred
to contains the expository appendage.

XII. The same mass of expository matter may serve in common for, may
appertain in common to, many commands, many masses of imperative matter.
Thus, amongst other things, the catalogue of \emph{collative} and
\emph{ablative} events, with respect to \emph{titles} above spoken of
(see No. X of this note), will belong in common to all or most of the
laws constitutive of the various offences against property. Thus, in
mathematical diagrams, one and the same base shall serve for a whole
cluster of triangles.

XIII. Such expository matter, being of a complexion so different from
the imperative it would be no wonder if the connection of the former
with the latter should escape the observation: which. indeed, is perhaps
pretty generally the case. And so long as any mass of legislative matter
presents itself, which is not itself imperative or the contrary, or of
which the connection with matter of one of those two descriptions is not
apprehended, so long and so far the truth of the proposition, \emph{That
every law is a command or its opposite,} may remain unsuspected, or
appear questionable; so long also may the incompleteness of the greater
part of those masses of legislative matter, which wear the complexion of
complete laws upon the face of them, also the method to be taken for
rendering them really complete, remain undiscovered.

XIV. A circumstance, that will naturally contribute to increase the
difficulty of the discovery, is the great variety of ways in which the
imperation of a law maybe conveyed'' the great variety of forms
which the imperative part of a law may indiscriminately assume: some
more directly, some less directly expressive of the imperative quality.
\emph{Thou shalt not steal. Let so man steal. Whoso stealeth shall be
punished so and so. If any man steal, he shall be punished so and so.
Stealing is where a man does so and so; the punishment for stealing is
so and so. To judges} so and so named, and so and so constituted,
\emph{belong the cognizance of such and such offences;} viz.,
\emph{stealing''} and so on. These are but part of a multitude
of forms of words, in any of which the command by which stealing is
prohibited might equally be couched: and it is manifest to what a
degree, in some of them, the imperative quality is clouded and concealed
from ordinary apprehension.

XV. After this explanation, a general proposition or two, that may be
laid down, may help to afford some little insight into the structure and
contents of a complete body of laws.'' So many different sorts
of offences created, so many different laws of the \emph{coercive} kind:
so many \emph{exceptions} taken out of the descriptions of those
offences, so many laws of the \emph{discoercive} kind.\\
To class \emph{offences,} as hath been attempted to be done in the
preceding chapter, is therefore to class laws: to exhibit a complete
catalogue of all the offences created by law, including the whole mass
of expository matter necessary for fixing and exhibiting the import of
the terms contained in the several laws, by which those offences are
respectively created, would be to exhibit a complete collection of the
laws in force: in a word a complete body of law; a \emph{pannomion,}if
so it might be termed.

XVI. From the obscurity in which the limits of a \emph{law,} and the
distinction betwixt a law of the civil or simply imperative kind and a
punitory law, of are naturally involved, results the obscurity of the
limits betwixt a civil and a penal \emph{code,} betwixt a civil branch
of the law and the penal.\\
The question, \emph{What parts of the total mass of legislative matter
belong to the civil branch, and what to the penal?} supposes that divers
political states, or at least that some one such state, are to be found,
having as well a civil code as a penal code, each of them complete in
its kind, and marked out by certain limits. But no one such state has
ever yet existed.\\
To put a question to which a true answer can be given, we must
substitute to the foregoing question some such a one as that which
follows:

Suppose two masses of legislative matter to be drawn up at this time of
day, the one under the name of a civil code, the other of a penal code,
each meant to be complete in its kind'' in what general way, is
it natural to suppose, that the different sorts of matter, as above
distinguished, would be distributed between them?\\
To this question the following answer seems likely to come as near as
any other to the truth. The \emph{civil} code would not consist of a
collection of civil laws, each complete in itself, as well as clear of
all penal ones: Neither would the \emph{penal} code (since we have seen
that it \emph{could} not) consist of a collection of punitive laws, each
not only complete in itself, but clear of all civil ones. But

XVII. The civil code would consist chiefly of mere masses of expository
matter. The imperative matter, to which those masses of expository
matter respectively appertained, would be found'' not in that
same code'' not in the civil code'' nor in a pure state,
free from all admixture of punitory laws; but in the penal
code'' in a state of combination'' involved, in manner
as above explained, in so many correspondent punitory laws.

XVIII. The penal code then would consist principally of punitive laws,
involving the imperative matter of the whole number of civil laws: along
with which would probably also be found various masses of expository
matter, appertaining not to the civil, but to the punitory laws. The
body of penal law enacted by the Empress-Queen Maria Theresa, agrees
pretty well with this account.\\
XIX. The mass of legislative matter published in French as well as
German under the auspices of Frederic II. of Prussia, by the name of
Code Frederic, but never established with force of law, appears, for
example, to be almost wholly composed of masses of expository matter,
the relation of which to any imperative matter appears to have been but
very imperfectly apprehended.

XX. In that enormous mass of confusion and inconsistency, the ancient
Roman, or, as it is termed by way of eminence, the civil law, the
imperative matter, and even all traces of the imperative character, seem
at last to have been smothered in the expository. \emph{Esto} had been
the language of primaeval simplicity: \emph{esto} had been the language
of the twelve tables. By the time of Justinian (so thick was the
darkness raised by clouds of commentators) the penal law had been
crammed into an odd corner of the civil'' the whole catalogue of
offences, and even of crimes, lay buried under a heap of
\emph{obligations'' will} was hid in \emph{opinion''}
and the original \emph{esto} had transformed itself into \emph{videtur,}
in the mouths even of the most despotic sovereigns. P\textgreater{}XXI.
Among the barbarous nations that grew up out of the ruins of the Roman
Empire, Law, emerging from under the mountain of expository rubbish,
reassumed for a while the language of command: and then she had
simplicity at least, if nothing else, to recommend her.

XXII. Besides the civil and the penal, every complete body of law must
contain a third branch, the \emph{constitutional.}\\
The constitutional branch is chiefly employed in conferring, on
particular classes of persons, \emph{powers,} to be exercised for the
good of the whole society, or of considerable parts of it, and
prescribing \emph{duties} to the persons invested with those powers.\\
The powers are principally constituted, in the first instance, by
discoercive or permissive laws operating as exceptions to certain laws
of the coercive or imperative kind. Instance: \emph{A tax-gatherer, as
such, may, on such and such an occasion, take such and such things,
without any other}\\
The duties are created by imperative laws, addressed to the persons on
whom the powers are conferred. Instance: \emph{On such and such an
occasion, such and such a tax-gatherer shall take such and such things.
Such and such a judge shall, in such and such a case, cause persons so
and so offending to be hanged.}\\
The parts which perform the function of indicating who the individuals
are, who, in every case, shall be considered as belonging to those
classes, have neither a permissive complexion, nor an imperative.\\
They are so many masses of expository matter, appertaining in common to
all laws, into the texture of which, the names of those classes of
persons have occasion to be inserted. Instance; imperative
matter:'' \emph{Let the judge cause whoever, in due course of
law, is convicted of stealing, to be hanged.} Nature of the expository
matter:'' Who is the person meant by the word \emph{judge?} He
who has been \emph{invested} with that office in such a manner: and in
respect of whom no \emph{event} has happened, of the number of those, to
which the effect is given, of reducing him to the condition of one
\emph{divested} of that office.

XXIII. Thus it is, that one and the same law, one and the same command,
will have its matter divided, not only between two great codes, or main
branches of the whole body of the laws, the civil and the penal; but
amongst three such branches, the civil, the penal and the
constitutional.

XXIV. In countries, where a great part of the law exists in no other
shape, than that of which in England is called \emph{common} law but
might be more expressively termed \emph{judiciary,} there must be a
great multitude of laws, the import of which cannot be sufficiently made
out for practice, without referring to this common law, for more or less
of the expository matter belonging to them. Thus in England the
exposition of the word \emph{title,} that basis of the of whole fabric
of the laws of property, is nowhere else to be found. And, as
uncertainty is of the very essence of every particle of law so
denominated (for the instant it is clothed in a certain authoritative
form of words it changes its nature, and passes over to the other
denomination) hence it is that a great part of the laws in being in such
countries remain uncertain and incomplete. What are those countries? To
this hour, every one on the surface of the globe.

XXV. Had the science of architecture no fixed nomenclature belonging to
it'' were there no settled names for distinguishing the
different sorts of buildings nor the different parts of the same
building from each other'' what would it be? It would be what
the science of legislation, considered with respect to its \emph{form,}
remains at present. Were there no architects who could distinguish a
dwelling-house from a barn, or a side-wall from a ceiling, what would
architects be? They would be what all legislators are at present.

XXVI. From this very slight and imperfect sketch, may be collected not
an answer to the questions in the text but an intimation, and that but
an imperfect one, of the course to be taken for giving such an answer;
and, at any rate, some idea of the difficulty, as well as of the
necessity, of the, task.\\
If it were thought necessary to recur to experience for proofs of this
difficulty, and this necessity, they need not be long wanting.

Take, for instance, so many well-meant endeavours on the part of popular
bodies, and so many well-meant recommendations in ingenious books, to
restrain supreme representative assemblies from making laws in such and
such cases, or to such and such an effect. Such laws, to answer the
intended purpose, require a perfect mastery in the science of law
considered in respect of its form'' in the sort of anatomy
spoken of in the preface to this work: but a perfect, or even a moderate
insight into that science, would prevent their being couched in those
loose and inadequate terms, in which they may be observed so frequently
to be conceived; as a perfect acquaintance with the dictates of utility
on that head would, in many, if not in most, of those instances,
discounsel the attempt. Keep to the letter, and in attempting to prevent
the making of bad laws, you will find them prohibiting the making of the
most necessary laws, perhaps even of all laws: quit the letter, and they
express no more than if each man were to say, \emph{Your laws shall
become \emph{ipso facto} void, as often as they contain any thing which
is not to my mind.}

Of such unhappy attempts, examples may be met with in the legislation of
many nations: but in none more frequently than in that newly-created
nation, one of the most enlightened, if not the most enlightened, at
this day on the globe.

XXVII. Take for instance the \emph{Declaration of Rights,} enacted by
the State of North Carolina, in convention, in or about the month of
September, 1788, and said to be copied, with a small exception, from one
in like manner enacted by the State of Virginia.

The following, to go no farther, is the first and fundamental article:\\

\begin{quote}
``That there are certain natural rights, of which men, when they form a
social compact, cannot deprive or divest their posterity, among which
are the enjoyment of life and liberty, with the means of acquiring,
possessing and protecting property, and pursuing and obtaining happiness
and safety.''
\end{quote}

Not to dwell on the oversight of confining to posterity the benefit of
the rights thus declared, what follows? That'' as against those
whom the protection, thus meant to be afforded, includes'' every
law, or other order, divesting a man of \emph{the enjoyment of life or
liberty,} is void.

Therefore this is the case, amongst others, with every coercive law.

Therefore, as against the persons thus protected, every order, for
example, to pay money on the score of taxation, or of debt from
individual to, individual, or otherwise, is void: for the effect of it,
if complied with, is to "deprive and \emph{divest him", pro tanto,} of
the enjoyment of liberty, viz., the liberty of paying or not paying as
he thinks proper: not to mention the species opposed to imprisonment, in
the event of such a mode of coercion's being resorted to: likewise of
property, which is itself a \emph{"means of acquiring, possessing and
protecting property, and of pursuing and obtaining happiness and
saftey.}

Therefore also, as against such persons, every order to attack an armed
enemy, in time of war, is also void: for, the necessary effect of such
an order is to "deprive some of them \emph{of the enjoyment of life."}

The above-mentioned consequences may suffice for examples, amongst an
endless train of similar ones.

Leaning on his elbow, in an attitude of profound and solemn meditation,
"What \emph{a multitude of things there are} " (exclaimed the
dancing-master Marcel) "in \emph{a minuet!}'' "May we now
add?'' \emph{and in a law.}
\end{document}