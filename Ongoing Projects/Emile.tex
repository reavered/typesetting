\documentclass[12pt]{report}
\usepackage[12pt]{moresize}
\usepackage[utf8]{inputenc}
\usepackage[english]{babel}
\usepackage[top=2.5cm, bottom=2.5cm, left=2.5cm, right=2.5cm]{geometry}
\usepackage{ebgaramond}

%=======SECTION HEADERS=========%
\usepackage{titlesec}
\usepackage{titletoc}

%========QUOTES=========%
\usepackage{epigraph}
\usepackage[autostyle, english = american]{csquotes}
\newcommand{\aquote}[2]{\normalsize \begin{displayquote} #1\end{displayquote} \hfill \textit{#2}}

%=======PARAGRAPH FORMATTING=========%
\setlength{\parindent}{0pt} %no paragraph indents
\setlength{\parskip}{1em}   %single space between paragraphs

\renewcommand{\chaptermark}[1]{\markboth{\MakeUppercase{Book \thechapter}}{}} %Book format- heading

%=======CHAPTER FORMATTING=========%
\renewcommand\thesection{{\arabic{section}}}   %section numbering style

\titleformat
{\chapter} 
[display]
{\fontfamily{ppl}\Huge} 
{Book \thechapter} 
{\leftmargin}{}[]

\newcommand{\mychapter}[2]{
\setcounter{chapter}{#1}
    \setcounter{section}{0}
    \chapter*{#2}
    \addcontentsline{toc}{chapter}{#2}
}

%=======SECTION HEADER SPACING=========%
\titlespacing{\chapter}{0mm}{-2em}{0em}
\titlespacing{\section}{0mm}{3mm}{2mm}

%=======TITLE PAGE=========%
\title{\HUGE\bfseries{Emile, or On Education}}
\author{\Large by Jean-Jacques Rousseau}
\date{\vspace{-4mm}Translated by Barbara Foxley}

%=======FOOTNOTES=========%
\renewcommand{\thefootnote}{[\arabic{footnote}]}
\setlength{\skip\footins}{1cm}
\usepackage[]{footmisc}
\renewcommand{\footnotemargin}{3mm} %Setting left margin
\renewcommand{\footnotelayout}{\hspace{2mm}} %spacing between the footnote number and the text of footnote

\usepackage{hyperref}
\hypersetup{bookmarksnumbered} %Bookmarks are numbered in the ToC when converted to PDF or EPUB

\titlecontents{chapter}% formatting-toc-chapters
    [0pt]% <left-indent>
    {}% <above-code>
    {\bfseries Book\ \thecontentslabel}% <numbered-entry-format>
    {}% <numberless-entry-format>
    {\bfseries\hfill\contentspage}% <filler-page-format>
\titlecontents{section}%formatting-toc-sections
    [3.8em] 
    {\vspace{-3mm}}
    {\contentslabel{2.3em}}
    {}
    {\titlerule*[1pc]{.}\contentspage}
\begin{document}

\begin{titlepage}
    \maketitle
\end{titlepage}

%=======TABLE OF CONTENTS=========%
\renewcommand*\contentsname{\vspace{-1cm} Table of Contents}
\tableofcontents

%=======MAIN DOCUMENT=========%
\mychapter{1}{Author's Preface}

This collection of scattered thoughts and observations has little order
or continuity; it was begun to give pleasure to a good mother who thinks
for herself. My first idea was to write a tract a few pages long, but I
was carried away by my subject, and before I knew what I was doing my
tract had become a kind of book, too large indeed for the matter
contained in it, but too small for the subject of which it treats. For a
long time I hesitated whether to publish it or not, and I have often
felt, when at work upon it, that it is one thing to publish a few
pamphlets and another to write a book. After vain attempts to improve
it, I have decided that it is my duty to publish it as it stands. I
consider that public attention requires to be directed to this subject,
and even if my own ideas are mistaken, my time will not have been wasted
if I stir up others to form right ideas. A solitary who casts his
writings before the public without any one to advertise them, without
any party ready to defend them, one who does not even know what is
thought and said about those writings, is at least free from one
anxiety---if he is mistaken, no one will take his errors for gospel.

I shall say very little about the value of a good education, nor shall I
stop to prove that the customary method of education is bad; this has
been done again and again, and I do not wish to fill my book with things
which everyone knows. I will merely state that, go as far back as you
will, you will find a continual outcry against the established method,
but no attempt to suggest a better. The literature and science of our
day tend rather to destroy than to build up. We find fault after the
manner of a master; to suggest, we must adopt another style, a style
less in accordance with the pride of the philosopher. In spite of all
those books, whose only aim, so they say, is public utility, the most
useful of all arts, the art of training men, is still neglected. Even
after Locke's book was written the subject remained almost untouched,
and I fear that my book will leave it pretty much as it found it.

We know nothing of childhood; and with our mistaken notions the further
we advance the further we go astray. The wisest writers devote
themselves to what a man ought to know, without asking what a child is
capable of learning. They are always looking for the man in the child,
without considering what he is before he becomes a man. It is to this
study that I have chiefly devoted myself, so that if my method is
fanciful and unsound, my observations may still be of service. I may be
greatly mistaken as to what ought to be done, but I think I have clearly
perceived the material which is to be worked upon. Begin thus by making
a more careful study of your scholars, for it is clear that you know
nothing about them; yet if you read this book with that end in view, I
think you will find that it is not entirely useless.

With regard to what will be called the systematic portion of the book,
which is nothing more than the course of nature, it is here that the
reader will probably go wrong, and no doubt I shall be attacked on this
side, and perhaps my critics may be right. You will tell me, ``This is
not so much a treatise on education as the visions of a dreamer with
regard to education.'' What can I do? I have not written about other
people's ideas of education, but about my own. My thoughts are not those
of others; this reproach has been brought against me again and again.
But is it within my power to furnish myself with other eyes, or to adopt
other ideas? It is within my power to refuse to be wedded to my own
opinions and to refuse to think myself wiser than others. I cannot
change my mind; I can distrust myself. This is all I can do, and this I
have done. If I sometimes adopt a confident tone, it is not to impress
the reader, it is to make my meaning plain to him. Why should I profess
to suggest as doubtful that which is not a matter of doubt to myself? I
say just what I think.

When I freely express my opinion, I have so little idea of claiming
authority that I always give my reasons, so that you may weigh and judge
them for yourselves; but though I would not obstinately defend my ideas,
I think it my duty to put them forward; for the principles with regard
to which I differ from other writers are not matters of indifference; we
must know whether they are true or false, for on them depends the
happiness or the misery of mankind. People are always telling me to make
PRACTICABLE suggestions. You might as well tell me to suggest what
people are doing already, or at least to suggest improvements which may
be incorporated with the wrong methods at present in use. There are
matters with regard to which such a suggestion is far more chimerical
than my own, for in such a connection the good is corrupted and the bad
is none the better for it. I would rather follow exactly the established
method than adopt a better method by halves. There would be fewer
contradictions in the man; he cannot aim at one and the same time at two
different objects. Fathers and mothers, what you desire that you can do.
May I count on your goodwill?

There are two things to be considered with regard to any scheme. In the
first place, ``Is it good in itself'' In the second, ``Can it be easily
put into practice?''

With regard to the first of these it is enough that the scheme should be
intelligible and feasible in itself, that what is good in it should be
adapted to the nature of things, in this case, for example, that the
proposed method of education should be suitable to man and adapted to
the human heart.

The second consideration depends upon certain given conditions in
particular cases; these conditions are accidental and therefore
variable; they may vary indefinitely. Thus one kind of education would
be possible in Switzerland and not in France; another would be adapted
to the middle classes but not to the nobility. The scheme can be carried
out, with more or less success, according to a multitude of
circumstances, and its results can only be determined by its special
application to one country or another, to this class or that. Now all
these particular applications are not essential to my subject, and they
form no part of my scheme. It is enough for me that, wherever men are
born into the world, my suggestions with regard to them may be carried
out, and when you have made them what I would have them be, you have
done what is best for them and best for other people. If I fail to
fulfil this promise, no doubt I am to blame; but if I fulfil my promise,
it is your own fault if you ask anything more of me, for I have promised
you nothing more.

\mychapter{2}{Book I}

God makes all things good; man meddles with them and they become evil.
He forces one soil to yield the products of another, one tree to bear
another's fruit. He confuses and confounds time, place, and natural
conditions. He mutilates his dog, his horse, and his slave. He destroys
and defaces all things; he loves all that is deformed and monstrous; he
will have nothing as nature made it, not even man himself, who must
learn his paces like a saddle-horse, and be shaped to his master's taste
like the trees in his garden. Yet things would be worse without this
education, and mankind cannot be made by halves. Under existing
conditions a man left to himself from birth would be more of a monster
than the rest. Prejudice, authority, necessity, example, all the social
conditions into which we are plunged, would stifle nature in him and put
nothing in her place. She would be like a sapling chance sown in the
midst of the highway, bent hither and thither and soon crushed by the
passers-by.

Tender, anxious mother, {[}Footnote: The earliest education is most
important and it undoubtedly is woman's work. If the author of nature
had meant to assign it to men he would have given them milk to feed the
child. Address your treatises on education to the women, for not only
are they able to watch over it more closely than men, not only is their
influence always predominant in education, its success concerns them
more nearly, for most widows are at the mercy of their children, who
show them very plainly whether their education was good or bad. The
laws, always more concerned about property than about people, since
their object is not virtue but peace, the laws give too little authority
to the mother. Yet her position is more certain than that of the father,
her duties are more trying; the right ordering of the family depends
more upon her, and she is usually fonder of her children. There are
occasions when a son may be excused for lack of respect for his father,
but if a child could be so unnatural as to fail in respect for the
mother who bore him and nursed him at her breast, who for so many years
devoted herself to his care, such a monstrous wretch should be smothered
at once as unworthy to live. You say mothers spoil their children, and
no doubt that is wrong, but it is worse to deprave them as you do. The
mother wants her child to be happy now. She is right, and if her method
is wrong, she must be taught a better. Ambition, avarice, tyranny, the
mistaken foresight of fathers, their neglect, their harshness, are a
hundredfold more harmful to the child than the blind affection of the
mother. Moreover, I must explain what I mean by a mother and that
explanation follows.{]} I appeal to you. You can remove this young tree
from the highway and shield it from the crushing force of social
conventions. Tend and water it ere it dies. One day its fruit will
reward your care. From the outset raise a wall round your child's soul;
another may sketch the plan, you alone should carry it into execution.

Plants are fashioned by cultivation, man by education. If a man were
born tall and strong, his size and strength would be of no good to him
till he had learnt to use them; they would even harm him by preventing
others from coming to his aid; {[}Footnote: Like them in externals, but
without speech and without the ideas which are expressed by speech, he
would be unable to make his wants known, while there would be nothing in
his appearance to suggest that he needed their help.{]} left to himself
he would die of want before he knew his needs. We lament the
helplessness of infancy; we fail to perceive that the race would have
perished had not man begun by being a child.

We are born weak, we need strength; helpless, we need aid; foolish, we
need reason. All that we lack at birth, all that we need when we come to
man's estate, is the gift of education.

This education comes to us from nature, from men, or from things. The
inner growth of our organs and faculties is the education of nature, the
use we learn to make of this growth is the education of men, what we
gain by our experience of our surroundings is the education of things.

Thus we are each taught by three masters. If their teaching conflicts,
the scholar is ill-educated and will never be at peace with himself; if
their teaching agrees, he goes straight to his goal, he lives at peace
with himself, he is well-educated.

Now of these three factors in education nature is wholly beyond our
control, things are only partly in our power; the education of men is
the only one controlled by us; and even here our power is largely
illusory, for who can hope to direct every word and deed of all with
whom the child has to do.

Viewed as an art, the success of education is almost impossible, since
the essential conditions of success are beyond our control. Our efforts
may bring us within sight of the goal, but fortune must favour us if we
are to reach it.

What is this goal? As we have just shown, it is the goal of nature.
Since all three modes of education must work together, the two that we
can control must follow the lead of that which is beyond our control.
Perhaps this word Nature has too vague a meaning. Let us try to define
it.

Nature, we are told, is merely habit. What does that mean? Are there not
habits formed under compulsion, habits which never stifle nature? Such,
for example, are the habits of plants trained horizontally. The plant
keeps its artificial shape, but the sap has not changed its course, and
any new growth the plant may make will be vertical. It is the same with
a man's disposition; while the conditions remain the same, habits, even
the least natural of them, hold good; but change the conditions, habits
vanish, nature reasserts herself. Education itself is but habit, for are
there not people who forget or lose their education and others who keep
it? Whence comes this difference? If the term nature is to be restricted
to habits conformable to nature we need say no more.

We are born sensitive and from our birth onwards we are affected in
various ways by our environment. As soon as we become conscious of our
sensations we tend to seek or shun the things that cause them, at first
because they are pleasant or unpleasant, then because they suit us or
not, and at last because of judgments formed by means of the ideas of
happiness and goodness which reason gives us. These tendencies gain
strength and permanence with the growth of reason, but hindered by our
habits they are more or less warped by our prejudices. Before this
change they are what I call Nature within us.

Everything should therefore be brought into harmony with these natural
tendencies, and that might well be if our three modes of education
merely differed from one another; but what can be done when they
conflict, when instead of training man for himself you try to train him
for others? Harmony becomes impossible. Forced to combat either nature
or society, you must make your choice between the man and the citizen,
you cannot train both.

The smaller social group, firmly united in itself and dwelling apart
from others, tends to withdraw itself from the larger society. Every
patriot hates foreigners; they are only men, and nothing to
him.{[}Footnote: Thus the wars of republics are more cruel than those of
monarchies. But if the wars of kings are less cruel, their peace is
terrible; better be their foe than their subject.{]} This defect is
inevitable, but of little importance. The great thing is to be kind to
our neighbours. Among strangers the Spartan was selfish, grasping, and
unjust, but unselfishness, justice, and harmony ruled his home life.
Distrust those cosmopolitans who search out remote duties in their books
and neglect those that lie nearest. Such philosophers will love the
Tartars to avoid loving their neighbour.

The natural man lives for himself; he is the unit, the whole, dependent
only on himself and on his like. The citizen is but the numerator of a
fraction, whose value depends on its denominator; his value depends upon
the whole, that is, on the community. Good social institutions are those
best fitted to make a man unnatural, to exchange his independence for
dependence, to merge the unit in the group, so that he no longer regards
himself as one, but as a part of the whole, and is only conscious of the
common life. A citizen of Rome was neither Caius nor Lucius, he was a
Roman; he ever loved his country better than his life. The captive
Regulus professed himself a Carthaginian; as a foreigner he refused to
take his seat in the Senate except at his master's bidding. He scorned
the attempt to save his life. He had his will, and returned in triumph
to a cruel death. There is no great likeness between Regulus and the men
of our own day.

The Spartan Pedaretes presented himself for admission to the council of
the Three Hundred and was rejected; he went away rejoicing that there
were three hundred Spartans better than himself. I suppose he was in
earnest; there is no reason to doubt it. That was a citizen.

A Spartan mother had five sons with the army. A Helot arrived; trembling
she asked his news. ``Your five sons are slain.'' ``Vile slave, was that
what I asked thee?'' ``We have won the victory.'' She hastened to the
temple to render thanks to the gods. That was a citizen.

He who would preserve the supremacy of natural feelings in social life
knows not what he asks. Ever at war with himself, hesitating between his
wishes and his duties, he will be neither a man nor a citizen. He will
be of no use to himself nor to others. He will be a man of our day, a
Frenchman, an Englishman, one of the great middle class.

To be something, to be himself, and always at one with himself, a man
must act as he speaks, must know what course he ought to take, and must
follow that course with vigour and persistence. When I meet this miracle
it will be time enough to decide whether he is a man or a citizen, or
how he contrives to be both.

Two conflicting types of educational systems spring from these
conflicting aims. One is public and common to many, the other private
and domestic.

If you wish to know what is meant by public education, read Plato's
Republic. Those who merely judge books by their titles take this for a
treatise on politics, but it is the finest treatise on education ever
written.

In popular estimation the Platonic Institute stands for all that is
fanciful and unreal. For my own part I should have thought the system of
Lycurgus far more impracticable had he merely committed it to writing.
Plato only sought to purge man's heart; Lycurgus turned it from its
natural course.

The public institute does not and cannot exist, for there is neither
country nor patriot. The very words should be struck out of our
language. The reason does not concern us at present, so that though I
know it I refrain from stating it.

I do not consider our ridiculous colleges {[}Footnote: There are
teachers dear to me in many schools and especially in the University of
Paris, men for whom I have a great respect, men whom I believe to be
quite capable of instructing young people, if they were not compelled to
follow the established custom. I exhort one of them to publish the
scheme of reform which he has thought out. Perhaps people would at
length seek to cure the evil if they realised that there was a
remedy.{]} as public institutes, nor do I include under this head a
fashionable education, for this education facing two ways at once
achieves nothing. It is only fit to turn out hypocrites, always
professing to live for others, while thinking of themselves alone. These
professions, however, deceive no one, for every one has his share in
them; they are so much labour wasted.

Our inner conflicts are caused by these contradictions. Drawn this way
by nature and that way by man, compelled to yield to both forces, we
make a compromise and reach neither goal. We go through life, struggling
and hesitating, and die before we have found peace, useless alike to
ourselves and to others.

There remains the education of the home or of nature; but how will a man
live with others if he is educated for himself alone? If the twofold
aims could be resolved into one by removing the man's
self-contradictions, one great obstacle to his happiness would be gone.
To judge of this you must see the man full-grown; you must have noted
his inclinations, watched his progress, followed his steps; in a word
you must really know a natural man. When you have read this work, I
think you will have made some progress in this inquiry.

What must be done to train this exceptional man! We can do much, but the
chief thing is to prevent anything being done. To sail against the wind
we merely follow one tack and another; to keep our position in a stormy
sea we must cast anchor. Beware, young pilot, lest your boat slip its
cable or drag its anchor before you know it.

In the social order where each has his own place a man must be educated
for it. If such a one leave his own station he is fit for nothing else.
His education is only useful when fate agrees with his parents' choice;
if not, education harms the scholar, if only by the prejudices it has
created. In Egypt, where the son was compelled to adopt his father's
calling, education had at least a settled aim; where social grades
remain fixed, but the men who form them are constantly changing, no one
knows whether he is not harming his son by educating him for his own
class.

In the natural order men are all equal and their common calling is that
of manhood, so that a well-educated man cannot fail to do well in that
calling and those related to it. It matters little to me whether my
pupil is intended for the army, the church, or the law. Before his
parents chose a calling for him nature called him to be a man. Life is
the trade I would teach him. When he leaves me, I grant you, he will be
neither a magistrate, a soldier, nor a priest; he will be a man. All
that becomes a man he will learn as quickly as another. In vain will
fate change his station, he will always be in his right place.
``Occupavi te, fortuna, atque cepi; omnes-que aditus tuos interclusi, ut
ad me aspirare non posses.'' The real object of our study is man and his
environment. To my mind those of us who can best endure the good and
evil of life are the best educated; hence it follows that true education
consists less in precept than in practice. We begin to learn when we
begin to live; our education begins with ourselves, our first teacher is
our nurse. The ancients used the word ``Education'' in a different
sense, it meant ``Nurture.'' ``Educit obstetrix,'' says Varro. ``Educat
nutrix, instituit paedagogus, docet magister.'' Thus, education,
discipline, and instruction are three things as different in their
purpose as the dame, the usher, and the teacher. But these distinctions
are undesirable and the child should only follow one guide.

We must therefore look at the general rather than the particular, and
consider our scholar as man in the abstract, man exposed to all the
changes and chances of mortal life. If men were born attached to the
soil of our country, if one season lasted all the year round, if every
man's fortune were so firmly grasped that he could never lose it, then
the established method of education would have certain advantages; the
child brought up to his own calling would never leave it, he could never
have to face the difficulties of any other condition. But when we
consider the fleeting nature of human affairs, the restless and uneasy
spirit of our times, when every generation overturns the work of its
predecessor, can we conceive a more senseless plan than to educate a
child as if he would never leave his room, as if he would always have
his servants about him? If the wretched creature takes a single step up
or down he is lost. This is not teaching him to bear pain; it is
training him to feel it.

People think only of preserving their child's life; this is not enough,
he must be taught to preserve his own life when he is a man, to bear the
buffets of fortune, to brave wealth and poverty, to live at need among
the snows of Iceland or on the scorching rocks of Malta. In vain you
guard against death; he must needs die; and even if you do not kill him
with your precautions, they are mistaken. Teach him to live rather than
to avoid death: life is not breath, but action, the use of our senses,
our mind, our faculties, every part of ourselves which makes us
conscious of our being. Life consists less in length of days than in the
keen sense of living. A man maybe buried at a hundred and may never have
lived at all. He would have fared better had he died young.

Our wisdom is slavish prejudice, our customs consist in control,
constraint, compulsion. Civilised man is born and dies a slave. The
infant is bound up in swaddling clothes, the corpse is nailed down in
his coffin. All his life long man is imprisoned by our institutions.

I am told that many midwives profess to improve the shape of the
infant's head by rubbing, and they are allowed to do it. Our heads are
not good enough as God made them, they must be moulded outside by the
nurse and inside by the philosopher. The Caribs are better off than we
are. The child has hardly left the mother's womb, it has hardly begun to
move and stretch its limbs, when it is deprived of its freedom. It is
wrapped in swaddling bands, laid down with its head fixed, its legs
stretched out, and its arms by its sides; it is wound round with linen
and bandages of all sorts so that it cannot move. It is fortunate if it
has room to breathe, and it is laid on its side so that water which
should flow from its mouth can escape, for it is not free to turn its
head on one side for this purpose.

The new-born child requires to stir and stretch his limbs to free them
from the stiffness resulting from being curled up so long. His limbs are
stretched indeed, but he is not allowed to move them. Even the head is
confined by a cap. One would think they were afraid the child should
look as if it were alive.

Thus the internal impulses which should lead to growth find an
insurmountable obstacle in the way of the necessary movements. The child
exhausts his strength in vain struggles, or he gains strength very
slowly. He was freer and less constrained in the womb; he has gained
nothing by birth.

The inaction, the constraint to which the child's limbs are subjected
can only check the circulation of the blood and humours; it can only
hinder the child's growth in size and strength, and injure its
constitution. Where these absurd precautions are absent, all the men are
tall, strong, and well-made. Where children are swaddled, the country
swarms with the hump-backed, the lame, the bow-legged, the rickety, and
every kind of deformity. In our fear lest the body should become
deformed by free movement, we hasten to deform it by putting it in a
press. We make our children helpless lest they should hurt themselves.

Is not such a cruel bondage certain to affect both health and temper?
Their first feeling is one of pain and suffering; they find every
necessary movement hampered; more miserable than a galley slave, in vain
they struggle, they become angry, they cry. Their first words you say
are tears. That is so. From birth you are always checking them, your
first gifts are fetters, your first treatment, torture. Their voice
alone is free; why should they not raise it in complaint? They cry
because you are hurting them; if you were swaddled you would cry louder
still.

What is the origin of this senseless and unnatural custom? Since mothers
have despised their first duty and refused to nurse their own children,
they have had to be entrusted to hired nurses. Finding themselves the
mothers of a stranger's children, without the ties of nature, they have
merely tried to save themselves trouble. A child unswaddled would need
constant watching; well swaddled it is cast into a corner and its cries
are unheeded. So long as the nurse's negligence escapes notice, so long
as the nursling does not break its arms or legs, what matter if it dies
or becomes a weakling for life. Its limbs are kept safe at the expense
of its body, and if anything goes wrong it is not the nurse's fault.

These gentle mothers, having got rid of their babies, devote themselves
gaily to the pleasures of the town. Do they know how their children are
being treated in the villages? If the nurse is at all busy, the child is
hung up on a nail like a bundle of clothes and is left crucified while
the nurse goes leisurely about her business. Children have been found in
this position purple in the face, their tightly bandaged chest forbade
the circulation of the blood, and it went to the head; so the sufferer
was considered very quiet because he had not strength to cry. How long a
child might survive under such conditions I do not know, but it could
not be long. That, I fancy, is one of the chief advantages of swaddling
clothes.

It is maintained that unswaddled infants would assume faulty positions
and make movements which might injure the proper development of their
limbs. That is one of the empty arguments of our false wisdom which has
never been confirmed by experience. Out of all the crowds of children
who grow up with the full use of their limbs among nations wiser than
ourselves, you never find one who hurts himself or maims himself; their
movements are too feeble to be dangerous, and when they assume an
injurious position, pain warns them to change it.

We have not yet decided to swaddle our kittens and puppies; are they any
the worse for this neglect? Children are heavier, I admit, but they are
also weaker. They can scarcely move, how could they hurt themselves! If
you lay them on their backs, they will lie there till they die, like the
turtle, unable to turn itself over. Not content with having ceased to
suckle their children, women no longer wish to do it; with the natural
result motherhood becomes a burden; means are found to avoid it. They
will destroy their work to begin it over again, and they thus turn to
the injury of the race the charm which was given them for its increase.
This practice, with other causes of depopulation, forbodes the coming
fate of Europe. Her arts and sciences, her philosophy and morals, will
shortly reduce her to a desert. She will be the home of wild beasts, and
her inhabitants will hardly have changed for the worse.

I have sometimes watched the tricks of young wives who pretend that they
wish to nurse their own children. They take care to be dissuaded from
this whim. They contrive that husbands, doctors, and especially mothers
should intervene. If a husband should let his wife nurse her own baby it
would be the ruin of him; they would make him out a murderer who wanted
to be rid of her. A prudent husband must sacrifice paternal affection to
domestic peace. Fortunately for you there are women in the country
districts more continent than your wives. You are still more fortunate
if the time thus gained is not intended for another than yourself.

There can be no doubt about a wife's duty, but, considering the contempt
in which it is held, it is doubtful whether it is not just as good for
the child to be suckled by a stranger. This is a question for the
doctors to settle, and in my opinion they have settled it according to
the women's wishes, {[}Footnote: The league between the women and the
doctors has always struck me as one of the oddest things in Paris. The
doctors' reputation depends on the women, and by means of the doctors
the women get their own way. It is easy to see what qualifications a
doctor requires in Paris if he is to become celebrated.{]} and for my
own part I think it is better that the child should suck the breast of a
healthy nurse rather than of a petted mother, if he has any further evil
to fear from her who has given him birth.

Ought the question, however, to be considered only from the
physiological point of view? Does not the child need a mother's care as
much as her milk? Other women, or even other animals, may give him the
milk she denies him, but there is no substitute for a mother's love.

The woman who nurses another's child in place of her own is a bad
mother; how can she be a good nurse? She may become one in time; use
will overcome nature, but the child may perish a hundred times before
his nurse has developed a mother's affection for him.

And this affection when developed has its drawbacks, which should make
any feeling woman afraid to put her child out to nurse. Is she prepared
to divide her mother's rights, or rather to abdicate them in favour of a
stranger; to see her child loving another more than herself; to feel
that the affection he retains for his own mother is a favour, while his
love for his foster-mother is a duty; for is not some affection due
where there has been a mother's care?

To remove this difficulty, children are taught to look down on their
nurses, to treat them as mere servants. When their task is completed the
child is withdrawn or the nurse is dismissed. Her visits to her
foster-child are discouraged by a cold reception. After a few years the
child never sees her again. The mother expects to take her place, and to
repair by her cruelty the results of her own neglect. But she is greatly
mistaken; she is making an ungrateful foster-child, not an affectionate
son; she is teaching him ingratitude, and she is preparing him to
despise at a later day the mother who bore him, as he now despises his
nurse.

How emphatically would I speak if it were not so hopeless to keep
struggling in vain on behalf of a real reform. More depends on this than
you realise. Would you restore all men to their primal duties, begin
with the mothers; the results will surprise you. Every evil follows in
the train of this first sin; the whole moral order is disturbed, nature
is quenched in every breast, the home becomes gloomy, the spectacle of a
young family no longer stirs the husband's love and the stranger's
reverence. The mother whose children are out of sight wins scanty
esteem; there is no home life, the ties of nature are not strengthened
by those of habit; fathers, mothers, children, brothers, and sisters
cease to exist. They are almost strangers; how should they love one
another? Each thinks of himself first. When the home is a gloomy
solitude pleasure will be sought elsewhere.

But when mothers deign to nurse their own children, then will be a
reform in morals; natural feeling will revive in every heart; there will
be no lack of citizens for the state; this first step by itself will
restore mutual affection. The charms of home are the best antidote to
vice. The noisy play of children, which we thought so trying, becomes a
delight; mother and father rely more on each other and grow dearer to
one another; the marriage tie is strengthened. In the cheerful home life
the mother finds her sweetest duties and the father his pleasantest
recreation. Thus the cure of this one evil would work a wide-spread
reformation; nature would regain her rights. When women become good
mothers, men will be good husbands and fathers.

My words are vain! When we are sick of worldly pleasures we do not
return to the pleasures of the home. Women have ceased to be mothers,
they do not and will not return to their duty. Could they do it if they
would? The contrary custom is firmly established; each would have to
overcome the opposition of her neighbours, leagued together against the
example which some have never given and others do not desire to follow.

Yet there are still a few young women of good natural disposition who
refuse to be the slaves of fashion and rebel against the clamour of
other women, who fulfil the sweet task imposed on them by nature. Would
that the reward in store for them might draw others to follow their
example. My conclusion is based upon plain reason, and upon facts I have
never seen disputed; and I venture to promise these worthy mothers the
firm and steadfast affection of their husbands and the truly filial love
of their children and the respect of all the world. Child-birth will be
easy and will leave no ill-results, their health will be strong and
vigorous, and they will see their daughters follow their example, and
find that example quoted as a pattern to others.

No mother, no child; their duties are reciprocal, and when ill done by
the one they will be neglected by the other. The child should love his
mother before he knows what he owes her. If the voice of instinct is not
strengthened by habit it soon dies, the heart is still-born. From the
outset we have strayed from the path of nature.

There is another by-way which may tempt our feet from the path of
nature. The mother may lavish excessive care on her child instead of
neglecting him; she may make an idol of him; she may develop and
increase his weakness to prevent him feeling it; she wards off every
painful experience in the hope of withdrawing him from the power of
nature, and fails to realise that for every trifling ill from which she
preserves him the future holds in store many accidents and dangers, and
that it is a cruel kindness to prolong the child's weakness when the
grown man must bear fatigue.

Thetis, so the story goes, plunged her son in the waters of Styx to make
him invulnerable. The truth of this allegory is apparent. The cruel
mothers I speak of do otherwise; they plunge their children into
softness, and they are preparing suffering for them, they open the way
to every kind of ill, which their children will not fail to experience
after they grow up.

Fix your eyes on nature, follow the path traced by her. She keeps
children at work, she hardens them by all kinds of difficulties, she
soon teaches them the meaning of pain and grief. They cut their teeth
and are feverish, sharp colics bring on convulsions, they are choked by
fits of coughing and tormented by worms, evil humours corrupt the blood,
germs of various kinds ferment in it, causing dangerous eruptions.
Sickness and danger play the chief part in infancy. One half of the
children who are born die before their eighth year. The child who has
overcome hardships has gained strength, and as soon as he can use his
life he holds it more securely.

This is nature's law; why contradict it? Do you not see that in your
efforts to improve upon her handiwork you are destroying it; her cares
are wasted? To do from without what she does within is according to you
to increase the danger twofold. On the contrary, it is the way to avert
it; experience shows that children delicately nurtured are more likely
to die. Provided we do not overdo it, there is less risk in using their
strength than in sparing it. Accustom them therefore to the hardships
they will have to face; train them to endure extremes of temperature,
climate, and condition, hunger, thirst, and weariness. Dip them in the
waters of Styx. Before bodily habits become fixed you may teach what
habits you will without any risk, but once habits are established any
change is fraught with peril. A child will bear changes which a man
cannot bear, the muscles of the one are soft and flexible, they take
whatever direction you give them without any effort; the muscles of the
grown man are harder and they only change their accustomed mode of
action when subjected to violence. So we can make a child strong without
risking his life or health, and even if there were some risk, it should
not be taken into consideration. Since human life is full of dangers,
can we do better than face them at a time when they can do the least
harm?

A child's worth increases with his years. To his personal value must be
added the cost of the care bestowed upon him. For himself there is not
only loss of life, but the consciousness of death. We must therefore
think most of his future in our efforts for his preservation. He must be
protected against the ills of youth before he reaches them: for if the
value of life increases until the child reaches an age when he can be
useful, what madness to spare some suffering in infancy only to multiply
his pain when he reaches the age of reason. Is that what our master
teaches us?

Man is born to suffer; pain is the means of his preservation. His
childhood is happy, knowing only pain of body. These bodily sufferings
are much less cruel, much less painful, than other forms of suffering,
and they rarely lead to self-destruction. It is not the twinges of gout
which make a man kill himself, it is mental suffering that leads to
despair. We pity the sufferings of childhood; we should pity ourselves;
our worst sorrows are of our own making.

The new-born infant cries, his early days are spent in crying. He is
alternately petted and shaken by way of soothing him; sometimes he is
threatened, sometimes beaten, to keep him quiet. We do what he wants or
we make him do what we want, we submit to his whims or subject him to
our own. There is no middle course; he must rule or obey. Thus his
earliest ideas are those of the tyrant or the slave. He commands before
he can speak, he obeys before he can act, and sometimes he is punished
for faults before he is aware of them, or rather before they are
committed. Thus early are the seeds of evil passions sown in his young
heart. At a later day these are attributed to nature, and when we have
taken pains to make him bad we lament his badness.

In this way the child passes six or seven years in the hands of women,
the victim of his own caprices or theirs, and after they have taught him
all sorts of things, when they have burdened his memory with words he
cannot understand, or things which are of no use to him, when nature has
been stifled by the passions they have implanted in him, this sham
article is sent to a tutor. The tutor completes the development of the
germs of artificiality which he finds already well grown, he teaches him
everything except self-knowledge and self-control, the arts of life and
happiness. When at length this infant slave and tyrant, crammed with
knowledge but empty of sense, feeble alike in mind and body, is flung
upon the world, and his helplessness, his pride, and his other vices are
displayed, we begin to lament the wretchedness and perversity of
mankind. We are wrong; this is the creature of our fantasy; the natural
man is cast in another mould.

Would you keep him as nature made him? Watch over him from his birth.
Take possession of him as soon as he comes into the world and keep him
till he is a man; you will never succeed otherwise. The real nurse is
the mother and the real teacher is the father. Let them agree in the
ordering of their duties as well as in their method, let the child pass
from one to the other. He will be better educated by a sensible though
ignorant father than by the cleverest master in the world. For zeal will
atone for lack of knowledge, rather than knowledge for lack of zeal. But
the duties of public and private business! Duty indeed! Does a father's
duty come last. {[}Footnote: When we read in Plutarch that Cato the
Censor, who ruled Rome with such glory, brought up his own sons from the
cradle, and so carefully that he left everything to be present when
their nurse, that is to say their mother, bathed them; when we read in
Suetonius that Augustus, the master of the world which he had conquered
and which he himself governed, himself taught his grandsons to write, to
swim, to understand the beginnings of science, and that he always had
them with him, we cannot help smiling at the little people of those days
who amused themselves with such follies, and who were too ignorant, no
doubt, to attend to the great affairs of the great people of our own
time.{]} It is not surprising that the man whose wife despises the duty
of suckling her child should despise its education. There is no more
charming picture than that of family life; but when one feature is
wanting the whole is marred. If the mother is too delicate to nurse her
child, the father will be too busy to teach him. Their children,
scattered about in schools, convents, and colleges, will find the home
of their affections elsewhere, or rather they will form the habit of
caring for nothing. Brothers and sisters will scarcely know each other;
when they are together in company they will behave as strangers. When
there is no confidence between relations, when the family society ceases
to give savour to life, its place is soon usurped by vice. Is there any
man so stupid that he cannot see how all this hangs together?

A father has done but a third of his task when he begets children and
provides a living for them. He owes men to humanity, citizens to the
state. A man who can pay this threefold debt and neglect to do so is
guilty, more guilty, perhaps, if he pays it in part than when he
neglects it entirely. He has no right to be a father if he cannot fulfil
a father's duties. Poverty, pressure of business, mistaken social
prejudices, none of these can excuse a man from his duty, which is to
support and educate his own children. If a man of any natural feeling
neglects these sacred duties he will repent it with bitter tears and
will never be comforted.

But what does this rich man do, this father of a family, compelled, so
he says, to neglect his children? He pays another man to perform those
duties which are his alone. Mercenary man! do you expect to purchase a
second father for your child? Do not deceive yourself; it is not even a
master you have hired for him, it is a flunkey, who will soon train such
another as himself.

There is much discussion as to the characteristics of a good tutor. My
first requirement, and it implies a good many more, is that he should
not take up his task for reward. There are callings so great that they
cannot be undertaken for money without showing our unfitness for them;
such callings are those of the soldier and the teacher.

``But who must train my child?'' ``I have just told you, you should do
it yourself.'' ``I cannot.'' ``You cannot! Then find a friend. I see no
other course.''

A tutor! What a noble soul! Indeed for the training of a man one must
either be a father or more than man. It is this duty you would calmly
hand over to a hireling!

The more you think of it the harder you will find it. The tutor must
have been trained for his pupil, his servants must have been trained for
their master, so that all who come near him may have received the
impression which is to be transmitted to him. We must pass from
education to education, I know not how far. How can a child be well
educated by one who has not been well educated himself!

Can such a one be found? I know not. In this age of degradation who
knows the height of virtue to which man's soul may attain? But let us
assume that this prodigy has been discovered. We shall learn what he
should be from the consideration of his duties. I fancy the father who
realises the value of a good tutor will contrive to do without one, for
it will be harder to find one than to become such a tutor himself; he
need search no further, nature herself having done half the work.

Some one whose rank alone is known to me suggested that I should educate
his son. He did me a great honour, no doubt, but far from regretting my
refusal, he ought to congratulate himself on my prudence. Had the offer
been accepted, and had I been mistaken in my method, there would have
been an education ruined; had I succeeded, things would have been
worse---his son would have renounced his title and refused to be a
prince.

I feel too deeply the importance of a tutor's duties and my own
unfitness, ever to accept such a post, whoever offered it, and even the
claims of friendship would be only an additional motive for my refusal.
Few, I think, will be tempted to make me such an offer when they have
read this book, and I beg any one who would do so to spare his pains. I
have had enough experience of the task to convince myself of my own
unfitness, and my circumstances would make it impossible, even if my
talents were such as to fit me for it. I have thought it my duty to make
this public declaration to those who apparently refuse to do me the
honour of believing in the sincerity of my determination. If I am unable
to undertake the more useful task, I will at least venture to attempt
the easier one; I will follow the example of my predecessors and take
up, not the task, but my pen; and instead of doing the right thing I
will try to say it.

I know that in such an undertaking the author, who ranges at will among
theoretical systems, utters many fine precepts impossible to practise,
and even when he says what is practicable it remains undone for want of
details and examples as to its application.

I have therefore decided to take an imaginary pupil, to assume on my own
part the age, health, knowledge, and talents required for the work of
his education, to guide him from birth to manhood, when he needs no
guide but himself. This method seems to me useful for an author who
fears lest he may stray from the practical to the visionary; for as soon
as he departs from common practice he has only to try his method on his
pupil; he will soon know, or the reader will know for him, whether he is
following the development of the child and the natural growth of the
human heart.

This is what I have tried to do. Lest my book should be unduly bulky, I
have been content to state those principles the truth of which is
self-evident. But as to the rules which call for proof, I have applied
them to Emile or to others, and I have shown, in very great detail, how
my theories may be put into practice. Such at least is my plan; the
reader must decide whether I have succeeded. At first I have said little
about Emile, for my earliest maxims of education, though very different
from those generally accepted, are so plain that it is hard for a man of
sense to refuse to accept them, but as I advance, my scholar, educated
after another fashion than yours, is no longer an ordinary child, he
needs a special system. Then he appears upon the scene more frequently,
and towards the end I never lose sight of him for a moment, until,
whatever he may say, he needs me no longer.

I pass over the qualities required in a good tutor; I take them for
granted, and assume that I am endowed with them. As you read this book
you will see how generous I have been to myself.

I will only remark that, contrary to the received opinion, a child's
tutor should be young, as young indeed as a man may well be who is also
wise. Were it possible, he should become a child himself, that he may be
the companion of his pupil and win his confidence by sharing his games.
Childhood and age have too little in common for the formation of a
really firm affection. Children sometimes flatter old men; they never
love them.

People seek a tutor who has already educated one pupil. This is too
much; one man can only educate one pupil; if two were essential to
success, what right would he have to undertake the first? With more
experience you may know better what to do, but you are less capable of
doing it; once this task has been well done, you will know too much of
its difficulties to attempt it a second time---if ill done, the first
attempt augurs badly for the second.

It is one thing to follow a young man about for four years, another to
be his guide for five-and-twenty. You find a tutor for your son when he
is already formed; I want one for him before he is born. Your man may
change his pupil every five years; mine will never have but one pupil.
You distinguish between the teacher and the tutor. Another piece of
folly! Do you make any distinction between the pupil and the scholar?
There is only one science for children to learn---the duties of man.
This science is one, and, whatever Xenophon may say of the education of
the Persians, it is indivisible. Besides, I prefer to call the man who
has this knowledge master rather than teacher, since it is a question of
guidance rather than instruction. He must not give precepts, he must let
the scholar find them out for himself.

If the master is to be so carefully chosen, he may well choose his
pupil, above all when he proposes to set a pattern for others. This
choice cannot depend on the child's genius or character, as I adopt him
before he is born, and they are only known when my task is finished. If
I had my choice I would take a child of ordinary mind, such as I assume
in my pupil. It is ordinary people who have to be educated, and their
education alone can serve as a pattern for the education of their
fellows. The others find their way alone.

The birthplace is not a matter of indifference in the education of man;
it is only in temperate climes that he comes to his full growth. The
disadvantages of extremes are easily seen. A man is not planted in one
place like a tree, to stay there the rest of his life, and to pass from
one extreme to another you must travel twice as far as he who starts
half-way.

If the inhabitant of a temperate climate passes in turn through both
extremes his advantage is plain, for although he may be changed as much
as he who goes from one extreme to the other, he only removes half-way
from his natural condition. A Frenchman can live in New Guinea or in
Lapland, but a negro cannot live in Tornea nor a Samoyed in Benin. It
seems also as if the brain were less perfectly organised in the two
extremes. Neither the negroes nor the Laps are as wise as Europeans. So
if I want my pupil to be a citizen of the world I will choose him in the
temperate zone, in France for example, rather than elsewhere.

In the north with its barren soil men devour much food, in the fertile
south they eat little. This produces another difference: the one is
industrious, the other contemplative. Society shows us, in one and the
same spot, a similar difference between rich and poor. The one dwells in
a fertile land, the other in a barren land.

The poor man has no need of education. The education of his own station
in life is forced upon him, he can have no other; the education received
by the rich man from his own station is least fitted for himself and for
society. Moreover, a natural education should fit a man for any
position. Now it is more unreasonable to train a poor man for wealth
than a rich man for poverty, for in proportion to their numbers more
rich men are ruined and fewer poor men become rich. Let us choose our
scholar among the rich; we shall at least have made another man; the
poor may come to manhood without our help.

For the same reason I should not be sorry if Emile came of a good
family. He will be another victim snatched from prejudice.

Emile is an orphan. No matter whether he has father or mother, having
undertaken their duties I am invested with their rights. He must honour
his parents, but he must obey me. That is my first and only condition.

I must add that there is just one other point arising out of this; we
must never be separated except by mutual consent. This clause is
essential, and I would have tutor and scholar so inseparable that they
should regard their fate as one. If once they perceive the time of their
separation drawing near, the time which must make them strangers to one
another, they become strangers then and there; each makes his own little
world, and both of them being busy in thought with the time when they
will no longer be together, they remain together against their will. The
disciple regards his master as the badge and scourge of childhood, the
master regards his scholar as a heavy burden which he longs to be rid
of. Both are looking forward to the time when they will part, and as
there is never any real affection between them, there will be scant
vigilance on the one hand, and on the other scant obedience.

But when they consider they must always live together, they must needs
love one another, and in this way they really learn to love one another.
The pupil is not ashamed to follow as a child the friend who will be
with him in manhood; the tutor takes an interest in the efforts whose
fruits he will enjoy, and the virtues he is cultivating in his pupil
form a store laid up for his old age.

This agreement made beforehand assumes a normal birth, a strong,
well-made, healthy child. A father has no choice, and should have no
preference within the limits of the family God has given him; all his
children are his alike, the same care and affection is due to all.
Crippled or well-made, weak or strong, each of them is a trust for which
he is responsible to the Giver, and nature is a party to the marriage
contract along with husband and wife.

But if you undertake a duty not imposed upon you by nature, you must
secure beforehand the means for its fulfilment, unless you would
undertake duties you cannot fulfil. If you take the care of a sickly,
unhealthy child, you are a sick nurse, not a tutor. To preserve a
useless life you are wasting the time which should be spent in
increasing its value, you risk the sight of a despairing mother
reproaching you for the death of her child, who ought to have died long
ago.

I would not undertake the care of a feeble, sickly child, should he live
to four score years. I want no pupil who is useless alike to himself and
others, one whose sole business is to keep himself alive, one whose body
is always a hindrance to the training of his mind. If I vainly lavish my
care upon him, what can I do but double the loss to society by robbing
it of two men, instead of one? Let another tend this weakling for me; I
am quite willing, I approve his charity, but I myself have no gift for
such a task; I could never teach the art of living to one who needs all
his strength to keep himself alive.

The body must be strong enough to obey the mind; a good servant must be
strong. I know that intemperance stimulates the passions; in course of
time it also destroys the body; fasting and penance often produce the
same results in an opposite way. The weaker the body, the more imperious
its demands; the stronger it is, the better it obeys. All sensual
passions find their home in effeminate bodies; the less satisfaction
they can get the keener their sting.

A feeble body makes a feeble mind. Hence the influence of physic, an art
which does more harm to man than all the evils it professes to cure. I
do not know what the doctors cure us of, but I know this: they infect us
with very deadly diseases, cowardice, timidity, credulity, the fear of
death. What matter if they make the dead walk, we have no need of
corpses; they fail to give us men, and it is men we need.

Medicine is all the fashion in these days, and very naturally. It is the
amusement of the idle and unemployed, who do not know what to do with
their time, and so spend it in taking care of themselves. If by ill-luck
they had happened to be born immortal, they would have been the most
miserable of men; a life they could not lose would be of no value to
them. Such men must have doctors to threaten and flatter them, to give
them the only pleasure they can enjoy, the pleasure of not being dead.

I will say no more at present as to the uselessness of medicine. My aim
is to consider its bearings on morals. Still I cannot refrain from
saying that men employ the same sophism about medicine as they do about
the search for truth. They assume that the patient is cured and that the
seeker after truth finds it. They fail to see that against one life
saved by the doctors you must set a hundred slain, and against the value
of one truth discovered the errors which creep in with it. The science
which instructs and the medicine which heals are no doubt excellent, but
the science which misleads us and the medicine which kills us are evil.
Teach us to know them apart. That is the real difficulty. If we were
content to be ignorant of truth we should not be the dupes of falsehood;
if we did not want to be cured in spite of nature, we should not be
killed by the doctors. We should do well to steer clear of both, and we
should evidently be the gainers. I do not deny that medicine is useful
to some men; I assert that it is fatal to mankind.

You will tell me, as usual, that the doctors are to blame, that medicine
herself is infallible. Well and good, then give us the medicine without
the doctor, for when we have both, the blunders of the artist are a
hundredfold greater than our hopes from the art. This lying art,
invented rather for the ills of the mind than of the body, is useless to
both alike; it does less to cure us of our diseases than to fill us with
alarm. It does less to ward off death than to make us dread its
approach. It exhausts life rather than prolongs it; should it even
prolong life it would only be to the prejudice of the race, since it
makes us set its precautions before society and our fears before our
duties. It is the knowledge of danger that makes us afraid. If we
thought ourselves invulnerable we should know no fear. The poet armed
Achilles against danger and so robbed him of the merit of courage; on
such terms any man would be an Achilles.

Would you find a really brave man? Seek him where there are no doctors,
where the results of disease are unknown, and where death is little
thought of. By nature a man bears pain bravely and dies in peace. It is
the doctors with their rules, the philosophers with their precepts, the
priests with their exhortations, who debase the heart and make us afraid
to die.

Give me a pupil who has no need of these, or I will have nothing to do
with him. No one else shall spoil my work, I will educate him myself or
not at all. That wise man, Locke, who had devoted part of his life to
the study of medicine, advises us to give no drugs to the child, whether
as a precaution, or on account of slight ailments. I will go farther,
and will declare that, as I never call in a doctor for myself, I will
never send for one for Emile, unless his life is clearly in danger, when
the doctor can but kill him.

I know the doctor will make capital out of my delay. If the child dies,
he was called in too late; if he recovers, it is his doing. So be it;
let the doctor boast, but do not call him in except in extremity.

As the child does not know how to be cured, he knows how to be ill. The
one art takes the place of the other and is often more successful; it is
the art of nature. When a beast is ill, it keeps quiet and suffers in
silence; but we see fewer sickly animals than sick men. How many men
have been slain by impatience, fear, anxiety, and above all by medicine,
men whom disease would have spared, and time alone have cured. I shall
be told that animals, who live according to nature, are less liable to
disease than ourselves. Well, that way of living is just what I mean to
teach my pupil; he should profit by it in the same way.

Hygiene is the only useful part of medicine, and hygiene is rather a
virtue than a science. Temperance and industry are man's true remedies;
work sharpens his appetite and temperance teaches him to control it.

To learn what system is most beneficial you have only to study those
races remarkable for health, strength, and length of days. If common
observation shows us that medicine neither increases health nor prolongs
life, it follows that this useless art is worse than useless, since it
wastes time, men, and things on what is pure loss. Not only must we
deduct the time spent, not in using life, but preserving it, but if this
time is spent in tormenting ourselves it is worse than wasted, it is so
much to the bad, and to reckon fairly a corresponding share must be
deducted from what remains to us. A man who lives ten years for himself
and others without the help of doctors lives more for himself and others
than one who spends thirty years as their victim. I have tried both, so
I think I have a better right than most to draw my own conclusions.

For these reasons I decline to take any but a strong and healthy pupil,
and these are my principles for keeping him in health. I will not stop
to prove at length the value of manual labour and bodily exercise for
strengthening the health and constitution; no one denies it. Nearly all
the instances of long life are to be found among the men who have taken
most exercise, who have endured fatigue and labour. {[}Footnote: I
cannot help quoting the following passage from an English newspaper, as
it throws much light on my opinions: ``A certain Patrick O'Neil, born in
1647, has just married his seventh wife in 1760. In the seventeenth year
of Charles II. he served in the dragoons and in other regiments up to
1740, when he took his discharge. He served in all the campaigns of
William III. and Marlborough. This man has never drunk anything but
small beer; he has always lived on vegetables, and has never eaten meat
except on few occasions when he made a feast for his relations. He has
always been accustomed to rise with the sun and go to bed at sunset
unless prevented by his military duties. He is now in his 130th year; he
is healthy, his hearing is good, and he walks with the help of a stick.
In spite of his great age he is never idle, and every Sunday he goes to
his parish church accompanied by his children, grandchildren, and great
grandchildren.''{]} Neither will I enter into details as to the care I
shall take for this alone. It will be clear that it forms such an
essential part of my practice that it is enough to get hold of the idea
without further explanation.

When our life begins our needs begin too. The new-born infant must have
a nurse. If his mother will do her duty, so much the better; her
instructions will be given her in writing, but this advantage has its
drawbacks, it removes the tutor from his charge. But it is to be hoped
that the child's own interests, and her respect for the person to whom
she is about to confide so precious a treasure, will induce the mother
to follow the master's wishes, and whatever she does you may be sure she
will do better than another. If we must have a strange nurse, make a
good choice to begin with.

It is one of the misfortunes of the rich to be cheated on all sides;
what wonder they think ill of mankind! It is riches that corrupt men,
and the rich are rightly the first to feel the defects of the only tool
they know. Everything is ill-done for them, except what they do
themselves, and they do next to nothing. When a nurse must be selected
the choice is left to the doctor. What happens? The best nurse is the
one who offers the highest bribe. I shall not consult the doctor about
Emile's nurse, I shall take care to choose her myself. I may not argue
about it so elegantly as the surgeon, but I shall be more reliable, I
shall be less deceived by my zeal than the doctor by his greed.

There is no mystery about this choice; its rules are well known, but I
think we ought probably to pay more attention to the age of the milk as
well as its quality. The first milk is watery, it must be almost an
aperient, to purge the remains of the meconium curdled in the bowels of
the new-born child. Little by little the milk thickens and supplies more
solid food as the child is able to digest it. It is surely not without
cause that nature changes the milk in the female of every species
according to the age of the offspring.

Thus a new-born child requires a nurse who has recently become mother.
There is, I know, a difficulty here, but as soon as we leave the path of
nature there are difficulties in the way of all well-doing. The wrong
course is the only right one under the circumstances, so we take it.

The nurse must be healthy alike in disposition and in body. The violence
of the passions as well as the humours may spoil her milk. Moreover, to
consider the body only is to keep only half our aim in view. The milk
may be good and the nurse bad; a good character is as necessary as a
good constitution. If you choose a vicious person, I do not say her
foster-child will acquire her vices, but he will suffer for them. Ought
she not to bestow on him day by day, along with her milk, a care which
calls for zeal, patience, gentleness, and cleanliness. If she is
intemperate and greedy her milk will soon be spoilt; if she is careless
and hasty what will become of a poor little wretch left to her mercy,
and unable either to protect himself or to complain. The wicked are
never good for anything.

The choice is all the more important because her foster-child should
have no other guardian, just as he should have no teacher but his tutor.
This was the custom of the ancients, who talked less but acted more
wisely than we. The nurse never left her foster-daughter; this is why
the nurse is the confidante in most of their plays. A child who passes
through many hands in turn, can never be well brought up.

At every change he makes a secret comparison, which continually tends to
lessen his respect for those who control him, and with it their
authority over him. If once he thinks there are grown-up people with no
more sense than children the authority of age is destroyed and his
education is ruined. A child should know no betters but its father and
mother, or failing them its foster-mother and its tutor, and even this
is one too many, but this division is inevitable, and the best that can
be done in the way of remedy is that the man and woman who control him
shall be so well agreed with regard to him that they seem like one.

The nurse must live rather more comfortably, she must have rather more
substantial food, but her whole way of living must not be altered, for a
sudden change, even a change for the better, is dangerous to health, and
since her usual way of life has made her healthy and strong, why change
it?

Country women eat less meat and more vegetables than towns-women, and
this vegetarian diet seems favourable rather than otherwise to
themselves and their children. When they take nurslings from the upper
classes they eat meat and broth with the idea that they will form better
chyle and supply more milk. I do not hold with this at all, and
experience is on my side, for we do not find children fed in this way
less liable to colic and worms.

That need not surprise us, for decaying animal matter swarms with worms,
but this is not the case with vegetable matter. {[}Footnote: Women eat
bread, vegetables, and dairy produce; female dogs and cats do the same;
the she-wolves eat grass. This supplies vegetable juices to their milk.
There are still those species which are unable to eat anything but
flesh, if such there are, which I very much doubt.{]} Milk, although
manufactured in the body of an animal, is a vegetable substance; this is
shown by analysis; it readily turns acid, and far from showing traces of
any volatile alkali like animal matter, it gives a neutral salt like
plants.

The milk of herbivorous creatures is sweeter and more wholesome than the
milk of the carnivorous; formed of a substance similar to its own, it
keeps its goodness and becomes less liable to putrifaction. If quantity
is considered, it is well known that farinaceous foods produce more
blood than meat, so they ought to yield more milk. If a child were not
weaned too soon, and if it were fed on vegetarian food, and its
foster-mother were a vegetarian, I do not think it would be troubled
with worms.

Milk derived from vegetable foods may perhaps be more liable to go sour,
but I am far from considering sour milk an unwholesome food; whole
nations have no other food and are none the worse, and all the array of
absorbents seems to me mere humbug. There are constitutions which do not
thrive on milk, others can take it without absorbents. People are afraid
of the milk separating or curdling; that is absurd, for we know that
milk always curdles in the stomach. This is how it becomes sufficiently
solid to nourish children and young animals; if it did not curdle it
would merely pass away without feeding them. {[}Footnote: Although the
juices which nourish us are liquid, they must be extracted from solids.
A hard-working man who ate nothing but soup would soon waste away. He
would be far better fed on milk, just because it curdles.{]} In vain you
dilute milk and use absorbents; whoever swallows milk digests cheese,
this rule is without exception; rennet is made from a calf's stomach.

Instead of changing the nurse's usual diet, I think it would be enough
to give food in larger quantities and better of its kind. It is not the
nature of the food that makes a vegetable diet indigestible, but the
flavouring that makes it unwholesome. Reform your cookery, use neither
butter nor oil for frying. Butter, salt, and milk should never be
cooked. Let your vegetables be cooked in water and only seasoned when
they come to table. The vegetable diet, far from disturbing the nurse,
will give her a plentiful supply of milk. {[}Footnote: Those who wish to
study a full account of the advantages and disadvantages of the
Pythagorean regime, may consult the works of Dr. Cocchi and his opponent
Dr. Bianchi on this important subject.{]} If a vegetable diet is best
for the child, how can meat food be best for his nurse? The things are
contradictory.

Fresh air affects children's constitutions, particularly in early years.
It enters every pore of a soft and tender skin, it has a powerful effect
on their young bodies. Its effects can never be destroyed. So I should
not agree with those who take a country woman from her village and shut
her up in one room in a town and her nursling with her. I would rather
send him to breathe the fresh air of the country than the foul air of
the town. He will take his new mother's position, will live in her
cottage, where his tutor will follow him. The reader will bear in mind
that this tutor is not a paid servant, but the father's friend. But if
this friend cannot be found, if this transfer is not easy, if none of my
advice can be followed, you will say to me, ``What shall I do instead?''
I have told you already---``Do what you are doing;'' no advice is needed
there.

Men are not made to be crowded together in ant-hills, but scattered over
the earth to till it. The more they are massed together, the more
corrupt they become. Disease and vice are the sure results of
over-crowded cities. Of all creatures man is least fitted to live in
herds. Huddled together like sheep, men would very soon die. Man's
breath is fatal to his fellows. This is literally as well as
figuratively true.

Men are devoured by our towns. In a few generations the race dies out or
becomes degenerate; it needs renewal, and it is always renewed from the
country. Send your children to renew themselves, so to speak, send them
to regain in the open fields the strength lost in the foul air of our
crowded cities. Women hurry home that their children may be born in the
town; they ought to do just the opposite, especially those who mean to
nurse their own children. They would lose less than they think, and in
more natural surroundings the pleasures associated by nature with
maternal duties would soon destroy the taste for other delights.

The new-born infant is first bathed in warm water to which a little wine
is usually added. I think the wine might be dispensed with. As nature
does not produce fermented liquors, it is not likely that they are of
much value to her creatures.

In the same way it is unnecessary to take the precaution of heating the
water; in fact among many races the new-born infants are bathed with no
more ado in rivers or in the sea. Our children, made tender before birth
by the softness of their parents, come into the world with a
constitution already enfeebled, which cannot be at once exposed to all
the trials required to restore it to health. Little by little they must
be restored to their natural vigour. Begin then by following this
custom, and leave it off gradually. Wash your children often, their
dirty ways show the need of this. If they are only wiped their skin is
injured; but as they grow stronger gradually reduce the heat of the
water, till at last you bathe them winter and summer in cold, even in
ice-cold water. To avoid risk this change must be slow, gradual, and
imperceptible, so you may use the thermometer for exact measurements.

This habit of the bath, once established, should never be broken off, it
must be kept up all through life. I value it not only on grounds of
cleanliness and present health, but also as a wholesome means of making
the muscles supple, and accustoming them to bear without risk or effort
extremes of heat and cold. As he gets older I would have the child
trained to bathe occasionally in hot water of every bearable degree, and
often in every degree of cold water. Now water being a denser fluid
touches us at more points than air, so that, having learnt to bear all
the variations of temperature in water, we shall scarcely feel this of
the air. {[}Footnote: Children in towns are stifled by being kept
indoors and too much wrapped up. Those who control them have still to
learn that fresh air, far from doing them harm, will make them strong,
while hot air will make them weak, will give rise to fevers, and will
eventually kill them.{]}

When the child draws its first breath do not confine it in tight
wrappings. No cap, no bandages, nor swaddling clothes. Loose and flowing
flannel wrappers, which leave its limbs free and are not too heavy to
check his movements, not too warm to prevent his feeling the air.
{[}Footnote: I say ``cradle'' using the common word for want of a
better, though I am convinced that it is never necessary and often
harmful to rock children in the cradle.{]} Put him in a big cradle, well
padded, where he can move easily and safely. As he begins to grow
stronger, let him crawl about the room; let him develop and stretch his
tiny limbs; you will see him gain strength from day to day. Compare him
with a well swaddled child of the same age and you will be surprised at
their different rates of progress. {[}Footnote: The ancient Peruvians
wrapped their children in loose swaddling bands, leaving the arms quite
free. Later they placed them unswaddled in a hole in the ground, lined
with cloths, so that the lower part of the body was in the hole, and
their arms were free and they could move the head and bend the body at
will without falling or hurting themselves. When they began to walk they
were enticed to come to the breast. The little negroes are often in a
position much more difficult for sucking. They cling to the mother's
hip, and cling so tightly that the mother's arm is often not needed to
support them. They clasp the breast with their hand and continue sucking
while their mother goes on with her ordinary work. These children begin
to walk at two months, or rather to crawl. Later on they can run on all
fours almost as well as on their feet.---Buffon. M. Buffon might also
have quoted the example of England, where the senseless and barbarous
swaddling clothes have become almost obsolete. Cf. La Longue Voyage de
Siam, Le Beau Voyage de Canada, etc.{]}

You must expect great opposition from the nurses, who find a half
strangled baby needs much less watching. Besides his dirtyness is more
perceptible in an open garment; he must be attended to more frequently.
Indeed, custom is an unanswerable argument in some lands and among all
classes of people.

Do not argue with the nurses; give your orders, see them carried out,
and spare no pains to make the attention you prescribe easy in practice.
Why not take your share in it? With ordinary nurslings, where the body
alone is thought of, nothing matters so long as the child lives and does
not actually die, but with us, when education begins with life, the
new-born child is already a disciple, not of his tutor, but of nature.
The tutor merely studies under this master, and sees that his orders are
not evaded. He watches over the infant, he observes it, he looks for the
first feeble glimmering of intelligence, as the Moslem looks for the
moment of the moon's rising in her first quarter.

We are born capable of learning, but knowing nothing, perceiving
nothing. The mind, bound up within imperfect and half grown organs, is
not even aware of its own existence. The movements and cries of the
new-born child are purely reflex, without knowledge or will.

Suppose a child born with the size and strength of manhood, entering
upon life full grown like Pallas from the brain of Jupiter; such a
child-man would be a perfect idiot, an automaton, a statue without
motion and almost without feeling; he would see and hear nothing, he
would recognise no one, he could not turn his eyes towards what he
wanted to see; not only would he perceive no external object, he would
not even be aware of sensation through the several sense-organs. His eye
would not perceive colour, his ear sounds, his body would be unaware of
contact with neighbouring bodies, he would not even know he had a body,
what his hands handled would be in his brain alone; all his sensations
would be united in one place, they would exist only in the common
``sensorium,'' he would have only one idea, that of self, to which he
would refer all his sensations; and this idea, or rather this feeling,
would be the only thing in which he excelled an ordinary child.

This man, full grown at birth, would also be unable to stand on his
feet, he would need a long time to learn how to keep his balance;
perhaps he would not even be able to try to do it, and you would see the
big strong body left in one place like a stone, or creeping and crawling
like a young puppy.

He would feel the discomfort of bodily needs without knowing what was
the matter and without knowing how to provide for these needs. There is
no immediate connection between the muscles of the stomach and those of
the arms and legs to make him take a step towards food, or stretch a
hand to seize it, even were he surrounded with it; and as his body would
be full grown and his limbs well developed he would be without the
perpetual restlessness and movement of childhood, so that he might die
of hunger without stirring to seek food. However little you may have
thought about the order and development of our knowledge, you cannot
deny that such a one would be in the state of almost primitive ignorance
and stupidity natural to man before he has learnt anything from
experience or from his fellows.

We know then, or we may know, the point of departure from which we each
start towards the usual level of understanding; but who knows the other
extreme? Each progresses more or less according to his genius, his
taste, his needs, his talents, his zeal, and his opportunities for using
them. No philosopher, so far as I know, has dared to say to man, ``Thus
far shalt thou go and no further.'' We know not what nature allows us to
be, none of us has measured the possible difference between man and man.
Is there a mind so dead that this thought has never kindled it, that has
never said in his pride, ``How much have I already done, how much more
may I achieve? Why should I lag behind my fellows?''

As I said before, man's education begins at birth; before he can speak
or understand he is learning. Experience precedes instruction; when he
recognises his nurse he has learnt much. The knowledge of the most
ignorant man would surprise us if we had followed his course from birth
to the present time. If all human knowledge were divided into two parts,
one common to all, the other peculiar to the learned, the latter would
seem very small compared with the former. But we scarcely heed this
general experience, because it is acquired before the age of reason.
Moreover, knowledge only attracts attention by its rarity, as in
algebraic equations common factors count for nothing. Even animals learn
much. They have senses and must learn to use them; they have needs, they
must learn to satisfy them; they must learn to eat, walk, or fly.
Quadrupeds which can stand on their feet from the first cannot walk for
all that; from their first attempts it is clear that they lack
confidence. Canaries who escape from their cage are unable to fly,
having never used their wings. Living and feeling creatures are always
learning. If plants could walk they would need senses and knowledge,
else their species would die out. The child's first mental experiences
are purely affective, he is only aware of pleasure and pain; it takes
him a long time to acquire the definite sensations which show him things
outside himself, but before these things present and withdraw
themselves, so to speak, from his sight, taking size and shape for him,
the recurrence of emotional experiences is beginning to subject the
child to the rule of habit. You see his eyes constantly follow the
light, and if the light comes from the side the eyes turn towards it, so
that one must be careful to turn his head towards the light lest he
should squint. He must also be accustomed from the first to the dark, or
he will cry if he misses the light. Food and sleep, too, exactly
measured, become necessary at regular intervals, and soon desire is no
longer the effect of need, but of habit, or rather habit adds a fresh
need to those of nature. You must be on your guard against this.

The only habit the child should be allowed to contract is that of having
no habits; let him be carried on either arm, let him be accustomed to
offer either hand, to use one or other indifferently; let him not want
to eat, sleep, or do anything at fixed hours, nor be unable to be left
alone by day or night. Prepare the way for his control of his liberty
and the use of his strength by leaving his body its natural habit, by
making him capable of lasting self-control, of doing all that he wills
when his will is formed.

As soon as the child begins to take notice, what is shown him must be
carefully chosen. The natural man is interested in all new things. He
feels so feeble that he fears the unknown: the habit of seeing fresh
things without ill effects destroys this fear. Children brought up in
clean houses where there are no spiders are afraid of spiders, and this
fear often lasts through life. I never saw peasants, man, woman, or
child, afraid of spiders.

Since the mere choice of things shown him may make the child timid or
brave, why should not his education begin before he can speak or
understand? I would have him accustomed to see fresh things, ugly,
repulsive, and strange beasts, but little by little, and far off till he
is used to them, and till having seen others handle them he handles them
himself. If in childhood he sees toads, snakes, and crayfish, he will
not be afraid of any animal when he is grown up. Those who are
continually seeing terrible things think nothing of them.

All children are afraid of masks. I begin by showing Emile a mask with a
pleasant face, then some one puts this mask before his face; I begin to
laugh, they all laugh too, and the child with them. By degrees I
accustom him to less pleasing masks, and at last hideous ones. If I have
arranged my stages skilfully, far from being afraid of the last mask, he
will laugh at it as he did at the first. After that I am not afraid of
people frightening him with masks.

When Hector bids farewell to Andromache, the young Astyanax, startled by
the nodding plumes on the helmet, does not know his father; he flings
himself weeping upon his nurse's bosom and wins from his mother a smile
mingled with tears. What must be done to stay this terror? Just what
Hector did; put the helmet on the ground and caress the child. In a
calmer moment one would do more; one would go up to the helmet, play
with the plumes, let the child feel them; at last the nurse would take
the helmet and place it laughingly on her own head, if indeed a woman's
hand dare touch the armour of Hector.

If Emile must get used to the sound of a gun, I first fire a pistol with
a small charge. He is delighted with this sudden flash, this sort of
lightning; I repeat the process with more powder; gradually I add a
small charge without a wad, then a larger; in the end I accustom him to
the sound of a gun, to fireworks, cannon, and the most terrible
explosions.

I have observed that children are rarely afraid of thunder unless the
peals are really terrible and actually hurt the ear, otherwise this fear
only comes to them when they know that thunder sometimes hurts or kills.
When reason begins to cause fear, let use reassure them. By slow and
careful stages man and child learn to fear nothing.

In the dawn of life, when memory and imagination have not begun to
function, the child only attends to what affects its senses. His sense
experiences are the raw material of thought; they should, therefore, be
presented to him in fitting order, so that memory may at a future time
present them in the same order to his understanding; but as he only
attends to his sensations it is enough, at first, to show him clearly
the connection between these sensations and the things which cause them.
He wants to touch and handle everything; do not check these movements
which teach him invaluable lessons. Thus he learns to perceive the heat,
cold, hardness, softness, weight, or lightness of bodies, to judge their
size and shape and all their physical properties, by looking, feeling,
{[}Footnote: Of all the senses that of smell is the latest to develop in
children up to two or three years of age they appear to be insensible of
pleasant or unpleasant odours; in this respect they are as indifferent
or rather as insensible as many animals.{]} listening, and, above all,
by comparing sight and touch, by judging with the eye what sensation
they would cause to his hand.

It is only by movement that we learn the difference between self and not
self; it is only by our own movements that we gain the idea of space.
The child has not this idea, so he stretches out his hand to seize the
object within his reach or that which is a hundred paces from him. You
take this as a sign of tyranny, an attempt to bid the thing draw near,
or to bid you bring it. Nothing of the kind, it is merely that the
object first seen in his brain, then before his eyes, now seems close to
his arms, and he has no idea of space beyond his reach. Be careful,
therefore, to take him about, to move him from place to place, and to
let him perceive the change in his surroundings, so as to teach him to
judge of distances.

When he begins to perceive distances then you must change your plan, and
only carry him when you please, not when he pleases; for as soon as he
is no longer deceived by his senses, there is another motive for his
effort. This change is remarkable and calls for explanation.

The discomfort caused by real needs is shown by signs, when the help of
others is required. Hence the cries of children; they often cry; it must
be so. Since they are only conscious of feelings, when those feelings
are pleasant they enjoy them in silence; when they are painful they say
so in their own way and demand relief. Now when they are awake they can
scarcely be in a state of indifference, either they are asleep or else
they are feeling something.

All our languages are the result of art. It has long been a subject of
inquiry whether there ever was a natural language common to all; no
doubt there is, and it is the language of children before they begin to
speak. This language is inarticulate, but it has tone, stress, and
meaning. The use of our own language has led us to neglect it so far as
to forget it altogether. Let us study children and we shall soon learn
it afresh from them. Nurses can teach us this language; they understand
all their nurslings say to them, they answer them, and keep up long
conversations with them; and though they use words, these words are
quite useless. It is not the hearing of the word, but its accompanying
intonation that is understood.

To the language of intonation is added the no less forcible language of
gesture. The child uses, not its weak hands, but its face. The amount of
expression in these undeveloped faces is extraordinary; their features
change from one moment to another with incredible speed. You see smiles,
desires, terror, come and go like lightning; every time the face seems
different. The muscles of the face are undoubtedly more mobile than our
own. On the other hand the eyes are almost expressionless. Such must be
the sort of signs they use at an age when their only needs are those of
the body. Grimaces are the sign of sensation, the glance expresses
sentiment.

As man's first state is one of want and weakness, his first sounds are
cries and tears. The child feels his needs and cannot satisfy them, he
begs for help by his cries. Is he hungry or thirsty? there are tears; is
he too cold or too hot? more tears; he needs movement and is kept quiet,
more tears; he wants to sleep and is disturbed, he weeps. The less
comfortable he is, the more he demands change. He has only one language
because he has, so to say, only one kind of discomfort. In the imperfect
state of his sense organs he does not distinguish their several
impressions; all ills produce one feeling of sorrow.

These tears, which you think so little worthy of your attention, give
rise to the first relation between man and his environment; here is
forged the first link in the long chain of social order.

When the child cries he is uneasy, he feels some need which he cannot
satisfy; you watch him, seek this need, find it, and satisfy it. If you
can neither find it nor satisfy it, the tears continue and become
tiresome. The child is petted to quiet him, he is rocked or sung to
sleep; if he is obstinate, the nurse becomes impatient and threatens
him; cruel nurses sometimes strike him. What strange lessons for him at
his first entrance into life!

I shall never forget seeing one of these troublesome crying children
thus beaten by his nurse. He was silent at once. I thought he was
frightened, and said to myself, ``This will be a servile being from whom
nothing can be got but by harshness.'' I was wrong, the poor wretch was
choking with rage, he could not breathe, he was black in the face. A
moment later there were bitter cries, every sign of the anger, rage, and
despair of this age was in his tones. I thought he would die. Had I
doubted the innate sense of justice and injustice in man's heart, this
one instance would have convinced me. I am sure that a drop of boiling
liquid falling by chance on that child's hand would have hurt him less
than that blow, slight in itself, but clearly given with the intention
of hurting him.

This tendency to anger, vexation, and rage needs great care. Boerhaave
thinks that most of the diseases of children are of the nature of
convulsions, because the head being larger in proportion and the nervous
system more extensive than in adults, they are more liable to nervous
irritation. Take the greatest care to remove from them any servants who
tease, annoy, or vex them. They are a hundredfold more dangerous and
more fatal than fresh air and changing seasons. When children only
experience resistance in things and never in the will of man, they do
not become rebellious or passionate, and their health is better. This is
one reason why the children of the poor, who are freer and more
independent, are generally less frail and weakly, more vigorous than
those who are supposed to be better brought up by being constantly
thwarted; but you must always remember that it is one thing to refrain
from thwarting them, but quite another to obey them. The child's first
tears are prayers, beware lest they become commands; he begins by asking
for aid, he ends by demanding service. Thus from his own weakness, the
source of his first consciousness of dependence, springs the later idea
of rule and tyranny; but as this idea is aroused rather by his needs
than by our services, we begin to see moral results whose causes are not
in nature; thus we see how important it is, even at the earliest age, to
discern the secret meaning of the gesture or cry.

When the child tries to seize something without speaking, he thinks he
can reach the object, for he does not rightly judge its distance; when
he cries and stretches out his hands he no longer misjudges the
distance, he bids the object approach, or orders you to bring it to him.
In the first case bring it to him slowly; in the second do not even seem
to hear his cries. The more he cries the less you should heed him. He
must learn in good time not to give commands to men, for he is not their
master, nor to things, for they cannot hear him. Thus when the child
wants something you mean to give him, it is better to carry him to it
rather than to bring the thing to him. From this he will draw a
conclusion suited to his age, and there is no other way of suggesting it
to him.

The Abbe Saint-Pierre calls men big children; one might also call
children little men. These statements are true, but they require
explanation. But when Hobbes calls the wicked a strong child, his
statement is contradicted by facts. All wickedness comes from weakness.
The child is only naughty because he is weak; make him strong and he
will be good; if we could do everything we should never do wrong. Of all
the attributes of the Almighty, goodness is that which it would be
hardest to dissociate from our conception of Him. All nations who have
acknowledged a good and an evil power, have always regarded the evil as
inferior to the good; otherwise their opinion would have been absurd.
Compare this with the creed of the Savoyard clergyman later on in this
book.

Reason alone teaches us to know good and evil. Therefore conscience,
which makes us love the one and hate the other, though it is independent
of reason, cannot develop without it. Before the age of reason we do
good or ill without knowing it, and there is no morality in our actions,
although there is sometimes in our feeling with regard to other people's
actions in relation to ourselves. A child wants to overturn everything
he sees. He breaks and smashes everything he can reach; he seizes a bird
as he seizes a stone, and strangles it without knowing what he is about.

Why so? In the first place philosophy will account for this by inbred
sin, man's pride, love of power, selfishness, spite; perhaps it will say
in addition to this that the child's consciousness of his own weakness
makes him eager to use his strength, to convince himself of it. But
watch that broken down old man reduced in the downward course of life to
the weakness of a child; not only is he quiet and peaceful, he would
have all about him quiet and peaceful too; the least change disturbs and
troubles him, he would like to see universal calm. How is it possible
that similar feebleness and similar passions should produce such
different effects in age and in infancy, if the original cause were not
different? And where can we find this difference in cause except in the
bodily condition of the two. The active principle, common to both, is
growing in one case and declining in the other; it is being formed in
the one and destroyed in the other; one is moving towards life, the
other towards death. The failing activity of the old man is centred in
his heart, the child's overflowing activity spreads abroad. He feels, if
we may say so, strong enough to give life to all about him. To make or
to destroy, it is all one to him; change is what he seeks, and all
change involves action. If he seems to enjoy destructive activity it is
only that it takes time to make things and very little time to break
them, so that the work of destruction accords better with his eagerness.

While the Author of nature has given children this activity, He takes
care that it shall do little harm by giving them small power to use it.
But as soon as they can think of people as tools to be used, they use
them to carry out their wishes and to supplement their own weakness.
This is how they become tiresome, masterful, imperious, naughty, and
unmanageable; a development which does not spring from a natural love of
power, but one which has been taught them, for it does not need much
experience to realise how pleasant it is to set others to work and to
move the world by a word.

As the child grows it gains strength and becomes less restless and
unquiet and more independent. Soul and body become better balanced and
nature no longer asks for more movement than is required for
self-preservation. But the love of power does not die with the need that
aroused it; power arouses and flatters self-love, and habit strengthens
it; thus caprice follows upon need, and the first seeds of prejudice and
obstinacy are sown.

FIRST MAXIM.---Far from being too strong, children are not strong enough
for all the claims of nature. Give them full use of such strength as
they have; they will not abuse it.

SECOND MAXIM.---Help them and supply the experience and strength they
lack whenever the need is of the body.

THIRD MAXIM.---In the help you give them confine yourself to what is
really needful, without granting anything to caprice or unreason; for
they will not be tormented by caprice if you do not call it into
existence, seeing it is no part of nature.

FOURTH MAXIM---Study carefully their speech and gestures, so that at an
age when they are incapable of deceit you may discriminate between those
desires which come from nature and those which spring from perversity.

The spirit of these rules is to give children more real liberty and less
power, to let them do more for themselves and demand less of others; so
that by teaching them from the first to confine their wishes within the
limits of their powers they will scarcely feel the want of whatever is
not in their power.

This is another very important reason for leaving children's limbs and
bodies perfectly free, only taking care that they do not fall, and
keeping anything that might hurt them out of their way.

The child whose body and arms are free will certainly cry much less than
a child tied up in swaddling clothes. He who knows only bodily needs,
only cries when in pain; and this is a great advantage, for then we know
exactly when he needs help, and if possible we should not delay our help
for an instant. But if you cannot relieve his pain, stay where you are
and do not flatter him by way of soothing him; your caresses will not
cure his colic, but he will remember what he must do to win them; and if
he once finds out how to gain your attention at will, he is your master;
the whole education is spoilt.

Their movements being less constrained, children will cry less; less
wearied with their tears, people will not take so much trouble to check
them. With fewer threats and promises, they will be less timid and less
obstinate, and will remain more nearly in their natural state. Ruptures
are produced less by letting children cry than by the means taken to
stop them, and my evidence for this is the fact that the most neglected
children are less liable to them than others. I am very far from wishing
that they should be neglected; on the contrary, it is of the utmost
importance that their wants should be anticipated, so that they need not
proclaim their wants by crying. But neither would I have unwise care
bestowed on them. Why should they think it wrong to cry when they find
they can get so much by it? When they have learned the value of their
silence they take good care not to waste it. In the end they will so
exaggerate its importance that no one will be able to pay its price;
then worn out with crying they become exhausted, and are at length
silent.

Prolonged crying on the part of a child neither swaddled nor out of
health, a child who lacks nothing, is merely the result of habit or
obstinacy. Such tears are no longer the work of nature, but the work of
the child's nurse, who could not resist its importunity and so has
increased it, without considering that while she quiets the child to-day
she is teaching him to cry louder to-morrow.

Moreover, when caprice or obstinacy is the cause of their tears, there
is a sure way of stopping them by distracting their attention by some
pleasant or conspicuous object which makes them forget that they want to
cry. Most nurses excel in this art, and rightly used it is very useful;
but it is of the utmost importance that the child should not perceive
that you mean to distract his attention, and that he should be amused
without suspecting you are thinking about him; now this is what most
nurses cannot do.

Most children are weaned too soon. The time to wean them is when they
cut their teeth. This generally causes pain and suffering. At this time
the child instinctively carries everything he gets hold of to his mouth
to chew it. To help forward this process he is given as a plaything some
hard object such as ivory or a wolf's tooth. I think this is a mistake.
Hard bodies applied to the gums do not soften them; far from it, they
make the process of cutting the teeth more difficult and painful. Let us
always take instinct as our guide; we never see puppies practising their
budding teeth on pebbles, iron, or bones, but on wood, leather, rags,
soft materials which yield to their jaws, and on which the tooth leaves
its mark.

We can do nothing simply, not even for our children. Toys of silver,
gold, coral, cut crystal, rattles of every price and kind; what vain and
useless appliances. Away with them all! Let us have no corals or
rattles; a small branch of a tree with its leaves and fruit, a stick of
liquorice which he may suck and chew, will amuse him as well as these
splendid trifles, and they will have this advantage at least, he will
not be brought up to luxury from his birth.

It is admitted that pap is not a very wholesome food. Boiled milk and
uncooked flour cause gravel and do not suit the stomach. In pap the
flour is less thoroughly cooked than in bread and it has not fermented.
I think bread and milk or rice-cream are better. If you will have pap,
the flour should be lightly cooked beforehand. In my own country they
make a very pleasant and wholesome soup from flour thus heated.
Meat-broth or soup is not a very suitable food and should be used as
little as possible. The child must first get used to chewing his food;
this is the right way to bring the teeth through, and when the child
begins to swallow, the saliva mixed with the food helps digestion.

I would have them first chew dried fruit or crusts. I should give them
as playthings little bits of dry bread or biscuits, like the Piedmont
bread, known in the country as ``grisses.'' By dint of softening this
bread in the mouth some of it is eventually swallowed the teeth come
through of themselves, and the child is weaned almost imperceptibly.
Peasants have usually very good digestions, and they are weaned with no
more ado.

From the very first children hear spoken language; we speak to them
before they can understand or even imitate spoken sounds. The vocal
organs are still stiff, and only gradually lend themselves to the
reproduction of the sounds heard; it is even doubtful whether these
sounds are heard distinctly as we hear them. The nurse may amuse the
child with songs and with very merry and varied intonation, but I object
to her bewildering the child with a multitude of vain words of which it
understands nothing but her tone of voice. I would have the first words
he hears few in number, distinctly and often repeated, while the words
themselves should be related to things which can first be shown to the
child. That fatal facility in the use of words we do not understand
begins earlier than we think. In the schoolroom the scholar listens to
the verbiage of his master as he listened in the cradle to the babble of
his nurse. I think it would be a very useful education to leave him in
ignorance of both.

All sorts of ideas crowd in upon us when we try to consider the
development of speech and the child's first words. Whatever we do they
all learn to talk in the same way, and all philosophical speculations
are utterly useless.

To begin with, they have, so to say, a grammar of their own, whose rules
and syntax are more general than our own; if you attend carefully you
will be surprised to find how exactly they follow certain analogies,
very much mistaken if you like, but very regular; these forms are only
objectionable because of their harshness or because they are not
recognised by custom. I have just heard a child severely scolded by his
father for saying, ``Mon pere, irai-je-t-y?'' Now we see that this child
was following the analogy more closely than our grammarians, for as they
say to him, ``Vas-y,'' why should he not say, ``Irai-je-t-y?'' Notice
too the skilful way in which he avoids the hiatus in irai-je-y or
y-irai-je? Is it the poor child's fault that we have so unskilfully
deprived the phrase of this determinative adverb ``y,'' because we did
not know what to do with it? It is an intolerable piece of pedantry and
most superfluous attention to detail to make a point of correcting all
children's little sins against the customary expression, for they always
cure themselves with time. Always speak correctly before them, let them
never be so happy with any one as with you, and be sure that their
speech will be imperceptibly modelled upon yours without any correction
on your part.

But a much greater evil, and one far less easy to guard against, is that
they are urged to speak too much, as if people were afraid they would
not learn to talk of themselves. This indiscreet zeal produces an effect
directly opposite to what is meant. They speak later and more
confusedly; the extreme attention paid to everything they say makes it
unnecessary for them to speak distinctly, and as they will scarcely open
their mouths, many of them contract a vicious pronunciation and a
confused speech, which last all their life and make them almost
unintelligible.

I have lived much among peasants, and I never knew one of them lisp, man
or woman, boy or girl. Why is this? Are their speech organs differently
made from our own? No, but they are differently used. There is a hillock
facing my window on which the children of the place assemble for their
games. Although they are far enough away, I can distinguish perfectly
what they say, and often get good notes for this book. Every day my ear
deceives me as to their age. I hear the voices of children of ten; I
look and see the height and features of children of three or four. This
experience is not confined to me; the townspeople who come to see me,
and whom I consult on this point, all fall into the same mistake.

This results from the fact that, up to five or six, children in town,
brought up in a room and under the care of a nursery governess, do not
need to speak above a whisper to make themselves heard. As soon as their
lips move people take pains to make out what they mean; they are taught
words which they repeat inaccurately, and by paying great attention to
them the people who are always with them rather guess what they meant to
say than what they said.

It is quite a different matter in the country. A peasant woman is not
always with her child; he is obliged to learn to say very clearly and
loudly what he wants, if he is to make himself understood. Children
scattered about the fields at a distance from their fathers, mothers and
other children, gain practice in making themselves heard at a distance,
and in adapting the loudness of the voice to the distance which
separates them from those to whom they want to speak. This is the real
way to learn pronunciation, not by stammering out a few vowels into the
ear of an attentive governess. So when you question a peasant child, he
may be too shy to answer, but what he says he says distinctly, while the
nurse must serve as interpreter for the town child; without her one can
understand nothing of what he is muttering between his teeth.
{[}Footnote: There are exceptions to this; and often those children who
at first are most difficult to hear, become the noisiest when they begin
to raise their voices. But if I were to enter into all these details I
should never make an end; every sensible reader ought to see that defect
and excess, caused by the same abuse, are both corrected by my method. I
regard the two maxims as inseparable---always enough---never too much.
When the first is well established, the latter necessarily follows on
it.{]}

As they grow older, the boys are supposed to be cured of this fault at
college, the girls in the convent schools; and indeed both usually speak
more clearly than children brought up entirely at home. But they are
prevented from acquiring as clear a pronunciation as the peasants in
this way---they are required to learn all sorts of things by heart, and
to repeat aloud what they have learnt; for when they are studying they
get into the way of gabbling and pronouncing carelessly and ill; it is
still worse when they repeat their lessons; they cannot find the right
words, they drag out their syllables. This is only possible when the
memory hesitates, the tongue does not stammer of itself. Thus they
acquire or continue habits of bad pronunciation. Later on you will see
that Emile does not acquire such habits or at least not from this cause.

I grant you uneducated people and villagers often fall into the opposite
extreme. They almost always speak too loud; their pronunciation is too
exact, and leads to rough and coarse articulation; their accent is too
pronounced, they choose their expressions badly, etc.

But, to begin with, this extreme strikes me as much less dangerous than
the other, for the first law of speech is to make oneself understood,
and the chief fault is to fail to be understood. To pride ourselves on
having no accent is to pride ourselves on ridding our phrases of
strength and elegance. Emphasis is the soul of speech, it gives it its
feeling and truth. Emphasis deceives less than words; perhaps that is
why well-educated people are so afraid of it. From the custom of saying
everything in the same tone has arisen that of poking fun at people
without their knowing it. When emphasis is proscribed, its place is
taken by all sorts of ridiculous, affected, and ephemeral
pronunciations, such as one observes especially among the young people
about court. It is this affectation of speech and manner which makes
Frenchmen disagreeable and repulsive to other nations on first
acquaintance. Emphasis is found, not in their speech, but in their
bearing. That is not the way to make themselves attractive.

All these little faults of speech, which you are so afraid the children
will acquire, are mere trifles; they may be prevented or corrected with
the greatest ease, but the faults which are taught them when you make
them speak in a low, indistinct, and timid voice, when you are always
criticising their tone and finding fault with their words, are never
cured. A man who has only learnt to speak in society of fine ladies
could not make himself heard at the head of his troops, and would make
little impression on the rabble in a riot. First teach the child to
speak to men; he will be able to speak to the women when required.

Brought up in all the rustic simplicity of the country, your children
will gain a more sonorous voice; they will not acquire the hesitating
stammer of town children, neither will they acquire the expressions nor
the tone of the villagers, or if they do they will easily lose them;
their master being with them from their earliest years, and more and
more in their society the older they grow, will be able to prevent or
efface by speaking correctly himself the impression of the peasants'
talk. Emile will speak the purest French I know, but he will speak it
more distinctly and with a better articulation than myself.

The child who is trying to speak should hear nothing but words he can
understand, nor should he say words he cannot articulate; his efforts
lead him to repeat the same syllable as if he were practising its clear
pronunciation. When he begins to stammer, do not try to understand him.
To expect to be always listened to is a form of tyranny which is not
good for the child. See carefully to his real needs, and let him try to
make you understand the rest. Still less should you hurry him into
speech; he will learn to talk when he feels the want of it.

It has indeed been remarked that those who begin to speak very late
never speak so distinctly as others; but it is not because they talked
late that they are hesitating; on the contrary, they began to talk late
because they hesitate; if not, why did they begin to talk so late? Have
they less need of speech, have they been less urged to it? On the
contrary, the anxiety aroused with the first suspicion of this
backwardness leads people to tease them much more to begin to talk than
those who articulated earlier; and this mistaken zeal may do much to
make their speech confused, when with less haste they might have had
time to bring it to greater perfection.

Children who are forced to speak too soon have no time to learn either
to pronounce correctly or to understand what they are made to say; while
left to themselves they first practise the easiest syllables, and then,
adding to them little by little some meaning which their gestures
explain, they teach you their own words before they learn yours. By this
means they do not acquire your words till they have understood them.
Being in no hurry to use them, they begin by carefully observing the
sense in which you use them, and when they are sure of them they adopt
them.

The worst evil resulting from the precocious use of speech by young
children is that we not only fail to understand the first words they
use, we misunderstand them without knowing it; so that while they seem
to answer us correctly, they fail to understand us and we them. This is
the most frequent cause of our surprise at children's sayings; we
attribute to them ideas which they did not attach to their words. This
lack of attention on our part to the real meaning which words have for
children seems to me the cause of their earliest misconceptions; and
these misconceptions, even when corrected, colour their whole course of
thought for the rest of their life. I shall have several opportunities
of illustrating these by examples later on.

Let the child's vocabulary, therefore, be limited; it is very
undesirable that he should have more words than ideas, that he should be
able to say more than he thinks. One of the reasons why peasants are
generally shrewder than townsfolk is, I think, that their vocabulary is
smaller. They have few ideas, but those few are thoroughly grasped.

The infant is progressing in several ways at once; he is learning to
talk, eat, and walk about the same time. This is really the first phase
of his life. Up till now, he was little more than he was before birth;
he had neither feeling nor thought, he was barely capable of sensation;
he was unconscious of his own existence.

``Vivit, et est vitae nescius ipse suae.''---Ovid.

\mychapter{3}{Book II}

We have now reached the second phase of life; infancy, strictly
so-called, is over; for the words infans and puer are not synonymous.
The latter includes the former, which means literally ``one who cannot
speak;'' thus Valerius speaks of puerum infantem. But I shall continue
to use the word child (French enfant) according to the custom of our
language till an age for which there is another term.

When children begin to talk they cry less. This progress is quite
natural; one language supplants another. As soon as they can say ``It
hurts me,'' why should they cry, unless the pain is too sharp for words?
If they still cry, those about them are to blame. When once Emile has
said, ``It hurts me,'' it will take a very sharp pain to make him cry.

If the child is delicate and sensitive, if by nature he begins to cry
for nothing, I let him cry in vain and soon check his tears at their
source. So long as he cries I will not go near him; I come at once when
he leaves off crying. He will soon be quiet when he wants to call me, or
rather he will utter a single cry. Children learn the meaning of signs
by their effects; they have no other meaning for them. However much a
child hurts himself when he is alone, he rarely cries, unless he expects
to be heard.

Should he fall or bump his head, or make his nose bleed, or cut his
fingers, I shall show no alarm, nor shall I make any fuss over him; I
shall take no notice, at any rate at first. The harm is done; he must
bear it; all my zeal could only frighten him more and make him more
nervous. Indeed it is not the blow but the fear of it which distresses
us when we are hurt. I shall spare him this suffering at least, for he
will certainly regard the injury as he sees me regard it; if he finds
that I hasten anxiously to him, if I pity him or comfort him, he will
think he is badly hurt. If he finds I take no notice, he will soon
recover himself, and will think the wound is healed when it ceases to
hurt. This is the time for his first lesson in courage, and by bearing
slight ills without fear we gradually learn to bear greater.

I shall not take pains to prevent Emile hurting himself; far from it, I
should be vexed if he never hurt himself, if he grew up unacquainted
with pain. To bear pain is his first and most useful lesson. It seems as
if children were small and weak on purpose to teach them these valuable
lessons without danger. The child has such a little way to fall he will
not break his leg; if he knocks himself with a stick he will not break
his arm; if he seizes a sharp knife he will not grasp it tight enough to
make a deep wound. So far as I know, no child, left to himself, has ever
been known to kill or maim itself, or even to do itself any serious
harm, unless it has been foolishly left on a high place, or alone near
the fire, or within reach of dangerous weapons. What is there to be said
for all the paraphernalia with which the child is surrounded to shield
him on every side so that he grows up at the mercy of pain, with neither
courage nor experience, so that he thinks he is killed by a pin-prick
and faints at the sight of blood?

With our foolish and pedantic methods we are always preventing children
from learning what they could learn much better by themselves, while we
neglect what we alone can teach them. Can anything be sillier than the
pains taken to teach them to walk, as if there were any one who was
unable to walk when he grows up through his nurse's neglect? How many we
see walking badly all their life because they were ill taught?

Emile shall have no head-pads, no go-carts, no leading-strings; or at
least as soon as he can put one foot before another he shall only be
supported along pavements, and he shall be taken quickly across them.
{[}Footnote: There is nothing so absurd and hesitating as the gait of
those who have been kept too long in leading-strings when they were
little. This is one of the observations which are considered trivial
because they are true.{]} Instead of keeping him mewed up in a stuffy
room, take him out into a meadow every day; let him run about, let him
struggle and fall again and again, the oftener the better; he will learn
all the sooner to pick himself up. The delights of liberty will make up
for many bruises. My pupil will hurt himself oftener than yours, but he
will always be merry; your pupils may receive fewer injuries, but they
are always thwarted, constrained, and sad. I doubt whether they are any
better off.

As their strength increases, children have also less need for tears.
They can do more for themselves, they need the help of others less
frequently. With strength comes the sense to use it. It is with this
second phase that the real personal life has its beginning; it is then
that the child becomes conscious of himself. During every moment of his
life memory calls up the feeling of self; he becomes really one person,
always the same, and therefore capable of joy or sorrow. Hence we must
begin to consider him as a moral being.

Although we know approximately the limits of human life and our chances
of attaining those limits, nothing is more uncertain than the length of
the life of any one of us. Very few reach old age. The chief risks occur
at the beginning of life; the shorter our past life, the less we must
hope to live. Of all the children who are born scarcely one half reach
adolescence, and it is very likely your pupil will not live to be a man.

What is to be thought, therefore, of that cruel education which
sacrifices the present to an uncertain future, that burdens a child with
all sorts of restrictions and begins by making him miserable, in order
to prepare him for some far-off happiness which he may never enjoy? Even
if I considered that education wise in its aims, how could I view
without indignation those poor wretches subjected to an intolerable
slavery and condemned like galley-slaves to endless toil, with no
certainty that they will gain anything by it? The age of harmless mirth
is spent in tears, punishments, threats, and slavery. You torment the
poor thing for his good; you fail to see that you are calling Death to
snatch him from these gloomy surroundings. Who can say how many children
fall victims to the excessive care of their fathers and mothers? They
are happy to escape from this cruelty; this is all that they gain from
the ills they are forced to endure: they die without regretting, having
known nothing of life but its sorrows.

Men, be kind to your fellow-men; this is your first duty, kind to every
age and station, kind to all that is not foreign to humanity. What
wisdom can you find that is greater than kindness? Love childhood,
indulge its sports, its pleasures, its delightful instincts. Who has not
sometimes regretted that age when laughter was ever on the lips, and
when the heart was ever at peace? Why rob these innocents of the joys
which pass so quickly, of that precious gift which they cannot abuse?
Why fill with bitterness the fleeting days of early childhood, days
which will no more return for them than for you? Fathers, can you tell
when death will call your children to him? Do not lay up sorrow for
yourselves by robbing them of the short span which nature has allotted
to them. As soon as they are aware of the joy of life, let them rejoice
in it, go that whenever God calls them they may not die without having
tasted the joy of life.

How people will cry out against me! I hear from afar the shouts of that
false wisdom which is ever dragging us onwards, counting the present as
nothing, and pursuing without a pause a future which flies as we pursue,
that false wisdom which removes us from our place and never brings us to
any other.

Now is the time, you say, to correct his evil tendencies; we must
increase suffering in childhood, when it is less keenly felt, to lessen
it in manhood. But how do you know that you can carry out all these fine
schemes; how do you know that all this fine teaching with which you
overwhelm the feeble mind of the child will not do him more harm than
good in the future? How do you know that you can spare him anything by
the vexations you heap upon him now? Why inflict on him more ills than
befit his present condition unless you are quite sure that these present
ills will save him future ill? And what proof can you give me that those
evil tendencies you profess to cure are not the result of your foolish
precautions rather than of nature? What a poor sort of foresight, to
make a child wretched in the present with the more or less doubtful hope
of making him happy at some future day. If such blundering thinkers fail
to distinguish between liberty and licence, between a merry child and a
spoilt darling, let them learn to discriminate.

Let us not forget what befits our present state in the pursuit of vain
fancies. Mankind has its place in the sequence of things; childhood has
its place in the sequence of human life; the man must be treated as a
man and the child as a child. Give each his place, and keep him there.
Control human passions according to man's nature; that is all we can do
for his welfare. The rest depends on external forces, which are beyond
our control.

Absolute good and evil are unknown to us. In this life they are blended
together; we never enjoy any perfectly pure feeling, nor do we remain
for more than a moment in the same state. The feelings of our minds,
like the changes in our bodies, are in a continual flux. Good and ill
are common to all, but in varying proportions. The happiest is he who
suffers least; the most miserable is he who enjoys least. Ever more
sorrow than joy---this is the lot of all of us. Man's happiness in this
world is but a negative state; it must be reckoned by the fewness of his
ills.

Every feeling of hardship is inseparable from the desire to escape from
it; every idea of pleasure from the desire to enjoy it. All desire
implies a want, and all wants are painful; hence our wretchedness
consists in the disproportion between our desires and our powers. A
conscious being whose powers were equal to his desires would be
perfectly happy.

What then is human wisdom? Where is the path of true happiness? The mere
limitation of our desires is not enough, for if they were less than our
powers, part of our faculties would be idle, and we should not enjoy our
whole being; neither is the mere extension of our powers enough, for if
our desires were also increased we should only be the more miserable.
True happiness consists in decreasing the difference between our desires
and our powers, in establishing a perfect equilibrium between the power
and the will. Then only, when all its forces are employed, will the soul
be at rest and man will find himself in his true position.

In this condition, nature, who does everything for the best, has placed
him from the first. To begin with, she gives him only such desires as
are necessary for self-preservation and such powers as are sufficient
for their satisfaction. All the rest she has stored in his mind as a
sort of reserve, to be drawn upon at need. It is only in this primitive
condition that we find the equilibrium between desire and power, and
then alone man is not unhappy. As soon as his potential powers of mind
begin to function, imagination, more powerful than all the rest, awakes,
and precedes all the rest. It is imagination which enlarges the bounds
of possibility for us, whether for good or ill, and therefore stimulates
and feeds desires by the hope of satisfying them. But the object which
seemed within our grasp flies quicker than we can follow; when we think
we have grasped it, it transforms itself and is again far ahead of us.
We no longer perceive the country we have traversed, and we think
nothing of it; that which lies before us becomes vaster and stretches
still before us. Thus we exhaust our strength, yet never reach our goal,
and the nearer we are to pleasure, the further we are from happiness.

On the other hand, the more nearly a man's condition approximates to
this state of nature the less difference is there between his desires
and his powers, and happiness is therefore less remote. Lacking
everything, he is never less miserable; for misery consists, not in the
lack of things, but in the needs which they inspire.

The world of reality has its bounds, the world of imagination is
boundless; as we cannot enlarge the one, let us restrict the other; for
all the sufferings which really make us miserable arise from the
difference between the real and the imaginary. Health, strength, and a
good conscience excepted, all the good things of life are a matter of
opinion; except bodily suffering and remorse, all our woes are
imaginary. You will tell me this is a commonplace; I admit it, but its
practical application is no commonplace, and it is with practice only
that we are now concerned.

What do you mean when you say, ``Man is weak''? The term weak implies a
relation, a relation of the creature to whom it is applied. An insect or
a worm whose strength exceeds its needs is strong; an elephant, a lion,
a conqueror, a hero, a god himself, whose needs exceed his strength is
weak. The rebellious angel who fought against his own nature was weaker
than the happy mortal who is living at peace according to nature. When
man is content to be himself he is strong indeed; when he strives to be
more than man he is weak indeed. But do not imagine that you can
increase your strength by increasing your powers. Not so; if your pride
increases more rapidly your strength is diminished. Let us measure the
extent of our sphere and remain in its centre like the spider in its
web; we shall have strength sufficient for our needs, we shall have no
cause to lament our weakness, for we shall never be aware of it.

The other animals possess only such powers as are required for
self-preservation; man alone has more. Is it not very strange that this
superfluity should make him miserable? In every land a man's labour
yields more than a bare living. If he were wise enough to disregard this
surplus he would always have enough, for he would never have too much.
``Great needs,'' said Favorin, ``spring from great wealth; and often the
best way of getting what we want is to get rid of what we have.'' By
striving to increase our happiness we change it into wretchedness. If a
man were content to live, he would live happy; and he would therefore be
good, for what would he have to gain by vice?

If we were immortal we should all be miserable; no doubt it is hard to
die, but it is sweet to think that we shall not live for ever, and that
a better life will put an end to the sorrows of this world. If we had
the offer of immortality here below, who would accept the sorrowful
gift? {[}Footnote: You understand I am speaking of those who think, and
not of the crowd.{]} What resources, what hopes, what consolation would
be left against the cruelties of fate and man's injustice? The ignorant
man never looks before; he knows little of the value of life and does
not fear to lose it; the wise man sees things of greater worth and
prefers them to it. Half knowledge and sham wisdom set us thinking about
death and what lies beyond it; and they thus create the worst of our
ills. The wise man bears life's ills all the better because he knows he
must die. Life would be too dearly bought did we not know that sooner or
later death will end it.

Our moral ills are the result of prejudice, crime alone excepted, and
that depends on ourselves; our bodily ills either put an end to
themselves or to us. Time or death will cure them, but the less we know
how to bear it, the greater is our pain, and we suffer more in our
efforts to cure our diseases than if we endured them. Live according to
nature; be patient, get rid of the doctors; you will not escape death,
but you will only die once, while the doctors make you die daily through
your diseased imagination; their lying art, instead of prolonging your
days, robs you of all delight in them. I am always asking what real good
this art has done to mankind. True, the doctors cure some who would have
died, but they kill millions who would have lived. If you are wise you
will decline to take part in this lottery when the odds are so great
against you. Suffer, die, or get better; but whatever you do, live while
you are alive.

Human institutions are one mass of folly and contradiction. As our life
loses its value we set a higher price upon it. The old regret life more
than the young; they do not want to lose all they have spent in
preparing for its enjoyment. At sixty it is cruel to die when one has
not begun to live. Man is credited with a strong desire for
self-preservation, and this desire exists; but we fail to perceive that
this desire, as felt by us, is largely the work of man. In a natural
state man is only eager to preserve his life while he has the means for
its preservation; when self-preservation is no longer possible, he
resigns himself to his fate and dies without vain torments. Nature
teaches us the first law of resignation. Savages, like wild beasts, make
very little struggle against death, and meet it almost without a murmur.
When this natural law is overthrown reason establishes another, but few
discern it, and man's resignation is never so complete as nature's.

Prudence! Prudence which is ever bidding us look forward into the
future, a future which in many cases we shall never reach; here is the
real source of all our troubles! How mad it is for so short-lived a
creature as man to look forward into a future to which he rarely
attains, while he neglects the present which is his? This madness is all
the more fatal since it increases with years, and the old, always timid,
prudent, and miserly, prefer to do without necessaries to-day that they
may have luxuries at a hundred. Thus we grasp everything, we cling to
everything; we are anxious about time, place, people, things, all that
is and will be; we ourselves are but the least part of ourselves. We
spread ourselves, so to speak, over the whole world, and all this vast
expanse becomes sensitive. No wonder our woes increase when we may be
wounded on every side. How many princes make themselves miserable for
the loss of lands they never saw, and how many merchants lament in Paris
over some misfortune in the Indies!

Is it nature that carries men so far from their real selves? Is it her
will that each should learn his fate from others and even be the last to
learn it; so that a man dies happy or miserable before he knows what he
is about. There is a healthy, cheerful, strong, and vigorous man; it
does me good to see him; his eyes tell of content and well-being; he is
the picture of happiness. A letter comes by post; the happy man glances
at it, it is addressed to him, he opens it and reads it. In a moment he
is changed, he turns pale and falls into a swoon. When he comes to
himself he weeps, laments, and groans, he tears his hair, and his
shrieks re-echo through the air. You would say he was in convulsions.
Fool, what harm has this bit of paper done you? What limb has it torn
away? What crime has it made you commit? What change has it wrought in
you to reduce you to this state of misery?

Had the letter miscarried, had some kindly hand thrown it into the fire,
it strikes me that the fate of this mortal, at once happy and unhappy,
would have offered us a strange problem. His misfortunes, you say, were
real enough. Granted; but he did not feel them. What of that? His
happiness was imaginary. I admit it; health, wealth, a contented spirit,
are mere dreams. We no longer live in our own place, we live outside it.
What does it profit us to live in such fear of death, when all that
makes life worth living is our own?

Oh, man! live your own life and you will no longer be wretched. Keep to
your appointed place in the order of nature and nothing can tear you
from it. Do not kick against the stern law of necessity, nor waste in
vain resistance the strength bestowed on you by heaven, not to prolong
or extend your existence, but to preserve it so far and so long as
heaven pleases. Your freedom and your power extend as far and no further
than your natural strength; anything more is but slavery, deceit, and
trickery. Power itself is servile when it depends upon public opinion;
for you are dependent on the prejudices of others when you rule them by
means of those prejudices. To lead them as you will, they must be led as
they will. They have only to change their way of thinking and you are
forced to change your course of action. Those who approach you need only
contrive to sway the opinions of those you rule, or of the favourite by
whom you are ruled, or those of your own family or theirs. Had you the
genius of Themistocles, {[}Footnote: ``You see that little boy,'' said
Themistocles to his friends, ``the fate of Greece is in his hands, for
he rules his mother and his mother rules me, I rule the Athenians and
the Athenians rule the Greeks.'' What petty creatures we should often
find controlling great empires if we traced the course of power from the
prince to those who secretly put that power in motion.{]} viziers,
courtiers, priests, soldiers, servants, babblers, the very children
themselves, would lead you like a child in the midst of your legions.
Whatever you do, your actual authority can never extend beyond your own
powers. As soon as you are obliged to see with another's eyes you must
will what he wills. You say with pride, ``My people are my subjects.''
Granted, but what are you? The subject of your ministers. And your
ministers, what are they? The subjects of their clerks, their
mistresses, the servants of their servants. Grasp all, usurp all, and
then pour out your silver with both hands; set up your batteries, raise
the gallows and the wheel; make laws, issue proclamations, multiply your
spies, your soldiers, your hangmen, your prisons, and your chains. Poor
little men, what good does it do you? You will be no better served, you
will be none the less robbed and deceived, you will be no nearer
absolute power. You will say continually, ``It is our will,'' and you
will continually do the will of others.

There is only one man who gets his own way---he who can get it
single-handed; therefore freedom, not power, is the greatest good. That
man is truly free who desires what he is able to perform, and does what
he desires. This is my fundamental maxim. Apply it to childhood, and all
the rules of education spring from it.

Society has enfeebled man, not merely by robbing him of the right to his
own strength, but still more by making his strength insufficient for his
needs. This is why his desires increase in proportion to his weakness;
and this is why the child is weaker than the man. If a man is strong and
a child is weak it is not because the strength of the one is absolutely
greater than the strength of the other, but because the one can
naturally provide for himself and the other cannot. Thus the man will
have more desires and the child more caprices, a word which means, I
take it, desires which are not true needs, desires which can only be
satisfied with the help of others.

I have already given the reason for this state of weakness. Parental
affection is nature's provision against it; but parental affection may
be carried to excess, it may be wanting, or it may be ill applied.
Parents who live under our ordinary social conditions bring their child
into these conditions too soon. By increasing his needs they do not
relieve his weakness; they rather increase it. They further increase it
by demanding of him what nature does not demand, by subjecting to their
will what little strength he has to further his own wishes, by making
slaves of themselves or of him instead of recognising that mutual
dependence which should result from his weakness or their affection.

The wise man can keep his own place; but the child who does not know
what his place is, is unable to keep it. There are a thousand ways out
of it, and it is the business of those who have charge of the child to
keep him in his place, and this is no easy task. He should be neither
beast nor man, but a child. He must feel his weakness, but not suffer
through it; he must be dependent, but he must not obey; he must ask, not
command. He is only subject to others because of his needs, and because
they see better than he what he really needs, what may help or hinder
his existence. No one, not even his father, has the right to bid the
child do what is of no use to him.

When our natural tendencies have not been interfered with by human
prejudice and human institutions, the happiness alike of children and of
men consists in the enjoyment of their liberty. But the child's liberty
is restricted by his lack of strength. He who does as he likes is happy
provided he is self-sufficing; it is so with the man who is living in a
state of nature. He who does what he likes is not happy if his desires
exceed his strength; it is so with a child in like conditions. Even in a
state of nature children only enjoy an imperfect liberty, like that
enjoyed by men in social life. Each of us, unable to dispense with the
help of others, becomes so far weak and wretched. We were meant to be
men, laws and customs thrust us back into infancy. The rich and great,
the very kings themselves are but children; they see that we are ready
to relieve their misery; this makes them childishly vain, and they are
quite proud of the care bestowed on them, a care which they would never
get if they were grown men.

These are weighty considerations, and they provide a solution for all
the conflicting problems of our social system. There are two kinds of
dependence: dependence on things, which is the work of nature; and
dependence on men, which is the work of society. Dependence on things,
being non-moral, does no injury to liberty and begets no vices;
dependence on men, being out of order, {[}Footnote: In my PRINCIPLES OF
POLITICAL LAW it is proved that no private will can be ordered in the
social system.{]} gives rise to every kind of vice, and through this
master and slave become mutually depraved. If there is any cure for this
social evil, it is to be found in the substitution of law for the
individual; in arming the general will with a real strength beyond the
power of any individual will. If the laws of nations, like the laws of
nature, could never be broken by any human power, dependence on men
would become dependence on things; all the advantages of a state of
nature would be combined with all the advantages of social life in the
commonwealth. The liberty which preserves a man from vice would be
united with the morality which raises him to virtue.

Keep the child dependent on things only. By this course of education you
will have followed the order of nature. Let his unreasonable wishes meet
with physical obstacles only, or the punishment which results from his
own actions, lessons which will be recalled when the same circumstances
occur again. It is enough to prevent him from wrong doing without
forbidding him to do wrong. Experience or lack of power should take the
place of law. Give him, not what he wants, but what he needs. Let there
be no question of obedience for him or tyranny for you. Supply the
strength he lacks just so far as is required for freedom, not for power,
so that he may receive your services with a sort of shame, and look
forward to the time when he may dispense with them and may achieve the
honour of self-help.

Nature provides for the child's growth in her own fashion, and this
should never be thwarted. Do not make him sit still when he wants to run
about, nor run when he wants to be quiet. If we did not spoil our
children's wills by our blunders their desires would be free from
caprice. Let them run, jump, and shout to their heart's content. All
their own activities are instincts of the body for its growth in
strength; but you should regard with suspicion those wishes which they
cannot carry out for themselves, those which others must carry out for
them. Then you must distinguish carefully between natural and artificial
needs, between the needs of budding caprice and the needs which spring
from the overflowing life just described.

I have already told you what you ought to do when a child cries for this
thing or that. I will only add that as soon as he has words to ask for
what he wants and accompanies his demands with tears, either to get his
own way quicker or to over-ride a refusal, he should never have his way.
If his words were prompted by a real need you should recognise it and
satisfy it at once; but to yield to his tears is to encourage him to
cry, to teach him to doubt your kindness, and to think that you are
influenced more by his importunity than your own good-will. If he does
not think you kind he will soon think you unkind; if he thinks you weak
he will soon become obstinate; what you mean to give must be given at
once. Be chary of refusing, but, having refused, do not change your
mind.

Above all, beware of teaching the child empty phrases of politeness,
which serve as spells to subdue those around him to his will, and to get
him what he wants at once. The artificial education of the rich never
fails to make them politely imperious, by teaching them the words to use
so that no one will dare to resist them. Their children have neither the
tone nor the manner of suppliants; they are as haughty or even more
haughty in their entreaties than in their commands, as though they were
more certain to be obeyed. You see at once that ``If you please'' means
``It pleases me,'' and ``I beg'' means ``I command.'' What a fine sort
of politeness which only succeeds in changing the meaning of words so
that every word is a command! For my own part, I would rather Emile were
rude than haughty, that he should say ``Do this'' as a request, rather
than ``Please'' as a command. What concerns me is his meaning, not his
words.

There is such a thing as excessive severity as well as excessive
indulgence, and both alike should be avoided. If you let children suffer
you risk their health and life; you make them miserable now; if you take
too much pains to spare them every kind of uneasiness you are laying up
much misery for them in the future; you are making them delicate and
over-sensitive; you are taking them out of their place among men, a
place to which they must sooner or later return, in spite of all your
pains. You will say I am falling into the same mistake as those bad
fathers whom I blamed for sacrificing the present happiness of their
children to a future which may never be theirs.

Not so; for the liberty I give my pupil makes up for the slight
hardships to which he is exposed. I see little fellows playing in the
snow, stiff and blue with cold, scarcely able to stir a finger. They
could go and warm themselves if they chose, but they do not choose; if
you forced them to come in they would feel the harshness of constraint a
hundredfold more than the sharpness of the cold. Then what becomes of
your grievance? Shall I make your child miserable by exposing him to
hardships which he is perfectly ready to endure? I secure his present
good by leaving him his freedom, and his future good by arming him
against the evils he will have to bear. If he had his choice, would he
hesitate for a moment between you and me?

Do you think any man can find true happiness elsewhere than in his
natural state; and when you try to spare him all suffering, are you not
taking him out of his natural state? Indeed I maintain that to enjoy
great happiness he must experience slight ills; such is his nature. Too
much bodily prosperity corrupts the morals. A man who knew nothing of
suffering would be incapable of tenderness towards his fellow-creatures
and ignorant of the joys of pity; he would be hard-hearted, unsocial, a
very monster among men.

Do you know the surest way to make your child miserable? Let him have
everything he wants; for as his wants increase in proportion to the ease
with which they are satisfied, you will be compelled, sooner or later,
to refuse his demands, and this unlooked-for refusal will hurt him more
than the lack of what he wants. He will want your stick first, then your
watch, the bird that flies, or the star that shines above him. He will
want all he sets eyes on, and unless you were God himself, how could you
satisfy him?

Man naturally considers all that he can get as his own. In this sense
Hobbes' theory is true to a certain extent: Multiply both our wishes and
the means of satisfying them, and each will be master of all. Thus the
child, who has only to ask and have, thinks himself the master of the
universe; he considers all men as his slaves; and when you are at last
compelled to refuse, he takes your refusal as an act of rebellion, for
he thinks he has only to command. All the reasons you give him, while he
is still too young to reason, are so many pretences in his eyes; they
seem to him only unkindness; the sense of injustice embitters his
disposition; he hates every one. Though he has never felt grateful for
kindness, he resents all opposition.

How should I suppose that such a child can ever be happy? He is the
slave of anger, a prey to the fiercest passions. Happy! He is a tyrant,
at once the basest of slaves and the most wretched of creatures. I have
known children brought up like this who expected you to knock the house
down, to give them the weather-cock on the steeple, to stop a regiment
on the march so that they might listen to the band; when they could not
get their way they screamed and cried and would pay no attention to any
one. In vain everybody strove to please them; as their desires were
stimulated by the ease with which they got their own way, they set their
hearts on impossibilities, and found themselves face to face with
opposition and difficulty, pain and grief. Scolding, sulking, or in a
rage, they wept and cried all day. Were they really so greatly favoured?
Weakness, combined with love of power, produces nothing but folly and
suffering. One spoilt child beats the table; another whips the sea. They
may beat and whip long enough before they find contentment.

If their childhood is made wretched by these notions of power and
tyranny, what of their manhood, when their relations with their
fellow-men begin to grow and multiply? They are used to find everything
give way to them; what a painful surprise to enter society and meet with
opposition on every side, to be crushed beneath the weight of a universe
which they expected to move at will. Their insolent manners, their
childish vanity, only draw down upon them mortification, scorn, and
mockery; they swallow insults like water; sharp experience soon teaches
them that they have realised neither their position nor their strength.
As they cannot do everything, they think they can do nothing. They are
daunted by unexpected obstacles, degraded by the scorn of men; they
become base, cowardly, and deceitful, and fall as far below their true
level as they formerly soared above it.

Let us come back to the primitive law. Nature has made children helpless
and in need of affection; did she make them to be obeyed and feared? Has
she given them an imposing manner, a stern eye, a loud and threatening
voice with which to make themselves feared? I understand how the roaring
of the lion strikes terror into the other beasts, so that they tremble
when they behold his terrible mane, but of all unseemly, hateful, and
ridiculous sights, was there ever anything like a body of statesmen in
their robes of office with their chief at their head bowing down before
a swaddled babe, addressing him in pompous phrases, while he cries and
slavers in reply?

If we consider childhood itself, is there anything so weak and wretched
as a child, anything so utterly at the mercy of those about it, so
dependent on their pity, their care, and their affection? Does it not
seem as if his gentle face and touching appearance were intended to
interest every one on behalf of his weakness and to make them eager to
help him? And what is there more offensive, more unsuitable, than the
sight of a sulky or imperious child, who commands those about him, and
impudently assumes the tones of a master towards those without whom he
would perish?

On the other hand, do you not see how children are fettered by the
weakness of infancy? Do you not see how cruel it is to increase this
servitude by obedience to our caprices, by depriving them of such
liberty as they have? a liberty which they can scarcely abuse, a liberty
the loss of which will do so little good to them or us. If there is
nothing more ridiculous than a haughty child, there is nothing that
claims our pity like a timid child. With the age of reason the child
becomes the slave of the community; then why forestall this by slavery
in the home? Let this brief hour of life be free from a yoke which
nature has not laid upon it; leave the child the use of his natural
liberty, which, for a time at least, secures him from the vices of the
slave. Bring me those harsh masters, and those fathers who are the
slaves of their children, bring them both with their frivolous
objections, and before they boast of their own methods let them for once
learn the method of nature.

I return to practical matters. I have already said your child must not
get what he asks, but what he needs; {[}Footnote: We must recognise that
pain is often necessary, pleasure is sometimes needed. So there is only
one of the child's desires which should never be complied with, the
desire for power. Hence, whenever they ask for anything we must pay
special attention to their motive in asking. As far as possible give
them everything they ask for, provided it can really give them pleasure;
refuse everything they demand from mere caprice or love of power.{]} he
must never act from obedience, but from necessity.

The very words OBEY and COMMAND will be excluded from his vocabulary,
still more those of DUTY and OBLIGATION; but the words strength,
necessity, weakness, and constraint must have a large place in it.
Before the age of reason it is impossible to form any idea of moral
beings or social relations; so avoid, as far as may be, the use of words
which express these ideas, lest the child at an early age should attach
wrong ideas to them, ideas which you cannot or will not destroy when he
is older. The first mistaken idea he gets into his head is the germ of
error and vice; it is the first step that needs watching. Act in such a
way that while he only notices external objects his ideas are confined
to sensations; let him only see the physical world around him. If not,
you may be sure that either he will pay no heed to you at all, or he
will form fantastic ideas of the moral world of which you prate, ideas
which you will never efface as long as he lives.

``Reason with children'' was Locke's chief maxim; it is in the height of
fashion at present, and I hardly think it is justified by its results;
those children who have been constantly reasoned with strike me as
exceedingly silly. Of all man's faculties, reason, which is, so to
speak, compounded of all the rest, is the last and choicest growth, and
it is this you would use for the child's early training. To make a man
reasonable is the coping stone of a good education, and yet you profess
to train a child through his reason! You begin at the wrong end, you
make the end the means. If children understood reason they would not
need education, but by talking to them from their earliest age in a
language they do not understand you accustom them to be satisfied with
words, to question all that is said to them, to think themselves as wise
as their teachers; you train them to be argumentative and rebellious;
and whatever you think you gain from motives of reason, you really gain
from greediness, fear, or vanity with which you are obliged to reinforce
your reasoning.

Most of the moral lessons which are and can be given to children may be
reduced to this formula; Master. You must not do that.

Child. Why not?

Master. Because it is wrong.

Child. Wrong! What is wrong?

Master. What is forbidden you.

Child. Why is it wrong to do what is forbidden?

Master. You will be punished for disobedience.

Child. I will do it when no one is looking.

Master. We shall watch you.

Child. I will hide.

Master. We shall ask you what you were doing.

Child. I shall tell a lie.

Master. You must not tell lies.

Child. Why must not I tell lies?

Master. Because it is wrong, etc.

That is the inevitable circle. Go beyond it, and the child will not
understand you. What sort of use is there in such teaching? I should
greatly like to know what you would substitute for this dialogue. It
would have puzzled Locke himself. It is no part of a child's business to
know right and wrong, to perceive the reason for a man's duties.

Nature would have them children before they are men. If we try to invert
this order we shall produce a forced fruit immature and flavourless,
fruit which will be rotten before it is ripe; we shall have young
doctors and old children. Childhood has its own ways of seeing,
thinking, and feeling; nothing is more foolish than to try and
substitute our ways; and I should no more expect judgment in a
ten-year-old child than I should expect him to be five feet high.
Indeed, what use would reason be to him at that age? It is the curb of
strength, and the child does not need the curb.

When you try to persuade your scholars of the duty of obedience, you add
to this so-called persuasion compulsion and threats, or still worse,
flattery and bribes. Attracted by selfishness or constrained by force,
they pretend to be convinced by reason. They see as soon as you do that
obedience is to their advantage and disobedience to their disadvantage.
But as you only demand disagreeable things of them, and as it is always
disagreeable to do another's will, they hide themselves so that they may
do as they please, persuaded that they are doing no wrong so long as
they are not found out, but ready, if found out, to own themselves in
the wrong for fear of worse evils. The reason for duty is beyond their
age, and there is not a man in the world who could make them really
aware of it; but the fear of punishment, the hope of forgiveness,
importunity, the difficulty of answering, wrings from them as many
confessions as you want; and you think you have convinced them when you
have only wearied or frightened them.

What does it all come to? In the first place, by imposing on them a duty
which they fail to recognise, you make them disinclined to submit to
your tyranny, and you turn away their love; you teach them deceit,
falsehood, and lying as a way to gain rewards or escape punishment; then
by accustoming them to conceal a secret motive under the cloak of an
apparent one, you yourself put into their hands the means of deceiving
you, of depriving you of a knowledge of their real character, of
answering you and others with empty words whenever they have the chance.
Laws, you say, though binding on conscience, exercise the same
constraint over grown-up men. That is so, but what are these men but
children spoilt by education? This is just what you should avoid. Use
force with children and reasoning with men; this is the natural order;
the wise man needs no laws.

Treat your scholar according to his age. Put him in his place from the
first, and keep him in it, so that he no longer tries to leave it. Then
before he knows what goodness is, he will be practising its chief
lesson. Give him no orders at all, absolutely none. Do not even let him
think that you claim any authority over him. Let him only know that he
is weak and you are strong, that his condition and yours puts him at
your mercy; let this be perceived, learned, and felt. Let him early find
upon his proud neck, the heavy yoke which nature has imposed upon us,
the heavy yoke of necessity, under which every finite being must bow.
Let him find this necessity in things, not in the caprices {[}Footnote:
You may be sure the child will regard as caprice any will which opposes
his own or any will which he does not understand. Now the child does not
understand anything which interferes with his own fancies.{]} of man;
let the curb be force, not authority. If there is something he should
not do, do not forbid him, but prevent him without explanation or
reasoning; what you give him, give it at his first word without prayers
or entreaties, above all without conditions. Give willingly, refuse
unwillingly, but let your refusal be irrevocable; let no entreaties move
you; let your ``No,'' once uttered, be a wall of brass, against which
the child may exhaust his strength some five or six times, but in the
end he will try no more to overthrow it.

Thus you will make him patient, equable, calm, and resigned, even when
he does not get all he wants; for it is in man's nature to bear
patiently with the nature of things, but not with the ill-will of
another. A child never rebels against, ``There is none left,'' unless he
thinks the reply is false. Moreover, there is no middle course; you must
either make no demands on him at all, or else you must fashion him to
perfect obedience. The worst education of all is to leave him hesitating
between his own will and yours, constantly disputing whether you or he
is master; I would rather a hundred times that he were master.

It is very strange that ever since people began to think about education
they should have hit upon no other way of guiding children than
emulation, jealousy, envy, vanity, greediness, base cowardice, all the
most dangerous passions, passions ever ready to ferment, ever prepared
to corrupt the soul even before the body is full-grown. With every piece
of precocious instruction which you try to force into their minds you
plant a vice in the depths of their hearts; foolish teachers think they
are doing wonders when they are making their scholars wicked in order to
teach them what goodness is, and then they tell us seriously, ``Such is
man.'' Yes, such is man, as you have made him. Every means has been
tried except one, the very one which might succeed---well-regulated
liberty. Do not undertake to bring up a child if you cannot guide him
merely by the laws of what can or cannot be. The limits of the possible
and the impossible are alike unknown to him, so they can be extended or
contracted around him at your will. Without a murmur he is restrained,
urged on, held back, by the hands of necessity alone; he is made
adaptable and teachable by the mere force of things, without any chance
for vice to spring up in him; for passions do not arise so long as they
have accomplished nothing.

Give your scholar no verbal lessons; he should be taught by experience
alone; never punish him, for he does not know what it is to do wrong;
never make him say, ``Forgive me,'' for he does not know how to do you
wrong. Wholly unmoral in his actions, he can do nothing morally wrong,
and he deserves neither punishment nor reproof.

Already I see the frightened reader comparing this child with those of
our time; he is mistaken. The perpetual restraint imposed upon your
scholars stimulates their activity; the more subdued they are in your
presence, the more boisterous they are as soon as they are out of your
sight. They must make amends to themselves in some way or other for the
harsh constraint to which you subject them. Two schoolboys from the town
will do more damage in the country than all the children of the village.
Shut up a young gentleman and a young peasant in a room; the former will
have upset and smashed everything before the latter has stirred from his
place. Why is that, unless that the one hastens to misuse a moment's
licence, while the other, always sure of freedom, does not use it
rashly. And yet the village children, often flattered or constrained,
are still very far from the state in which I would have them kept.

Let us lay it down as an incontrovertible rule that the first impulses
of nature are always right; there is no original sin in the human heart,
the how and why of the entrance of every vice can be traced. The only
natural passion is self-love or selfishness taken in a wider sense. This
selfishness is good in itself and in relation to ourselves; and as the
child has no necessary relations to other people he is naturally
indifferent to them; his self-love only becomes good or bad by the use
made of it and the relations established by its means. Until the time is
ripe for the appearance of reason, that guide of selfishness, the main
thing is that the child shall do nothing because you are watching him or
listening to him; in a word, nothing because of other people, but only
what nature asks of him; then he will never do wrong.

I do not mean to say that he will never do any mischief, never hurt
himself, never break a costly ornament if you leave it within his reach.
He might do much damage without doing wrong, since wrong-doing depends
on the harmful intention which will never be his. If once he meant to do
harm, his whole education would be ruined; he would be almost hopelessly
bad.

Greed considers some things wrong which are not wrong in the eyes of
reason. When you leave free scope to a child's heedlessness, you must
put anything he could spoil out of his way, and leave nothing fragile or
costly within his reach. Let the room be furnished with plain and solid
furniture; no mirrors, china, or useless ornaments. My pupil Emile, who
is brought up in the country, shall have a room just like a peasant's.
Why take such pains to adorn it when he will be so little in it? I am
mistaken, however; he will ornament it for himself, and we shall soon
see how.

But if, in spite of your precautions, the child contrives to do some
damage, if he breaks some useful article, do not punish him for your
carelessness, do not even scold him; let him hear no word of reproval,
do not even let him see that he has vexed you; behave just as if the
thing had come to pieces of itself; you may consider you have done great
things if you have managed to hold your tongue.

May I venture at this point to state the greatest, the most important,
the most useful rule of education? It is: Do not save time, but lose it.
I hope that every-day readers will excuse my paradoxes; you cannot avoid
paradox if you think for yourself, and whatever you may say I would
rather fall into paradox than into prejudice. The most dangerous period
in human life lies between birth and the age of twelve. It is the time
when errors and vices spring up, while as yet there is no means to
destroy them; when the means of destruction are ready, the roots have
gone too deep to be pulled up. If the infant sprang at one bound from
its mother's breast to the age of reason, the present type of education
would be quite suitable, but its natural growth calls for quite a
different training. The mind should be left undisturbed till its
faculties have developed; for while it is blind it cannot see the torch
you offer it, nor can it follow through the vast expanse of ideas a path
so faintly traced by reason that the best eyes can scarcely follow it.

Therefore the education of the earliest years should be merely negative.
It consists, not in teaching virtue or truth, but in preserving the
heart from vice and from the spirit of error. If only you could let well
alone, and get others to follow your example; if you could bring your
scholar to the age of twelve strong and healthy, but unable to tell his
right hand from his left, the eyes of his understanding would be open to
reason as soon as you began to teach him. Free from prejudices and free
from habits, there would be nothing in him to counteract the effects of
your labours. In your hands he would soon become the wisest of men; by
doing nothing to begin with, you would end with a prodigy of education.

Reverse the usual practice and you will almost always do right. Fathers
and teachers who want to make the child, not a child but a man of
learning, think it never too soon to scold, correct, reprove, threaten,
bribe, teach, and reason. Do better than they; be reasonable, and do not
reason with your pupil, more especially do not try to make him approve
what he dislikes; for if reason is always connected with disagreeable
matters, you make it distasteful to him, you discredit it at an early
age in a mind not yet ready to understand it. Exercise his body, his
limbs, his senses, his strength, but keep his mind idle as long as you
can. Distrust all opinions which appear before the judgment to
discriminate between them. Restrain and ward off strange impressions;
and to prevent the birth of evil do not hasten to do well, for goodness
is only possible when enlightened by reason. Regard all delays as so
much time gained; you have achieved much, you approach the boundary
without loss. Leave childhood to ripen in your children. In a word,
beware of giving anything they need to-day if it can be deferred without
danger to to-morrow.

There is another point to be considered which confirms the suitability
of this method: it is the child's individual bent, which must be
thoroughly known before we can choose the fittest moral training. Every
mind has its own form, in accordance with which it must be controlled;
and the success of the pains taken depends largely on the fact that he
is controlled in this way and no other. Oh, wise man, take time to
observe nature; watch your scholar well before you say a word to him;
first leave the germ of his character free to show itself, do not
constrain him in anything, the better to see him as he really is. Do you
think this time of liberty is wasted? On the contrary, your scholar will
be the better employed, for this is the way you yourself will learn not
to lose a single moment when time is of more value. If, however, you
begin to act before you know what to do, you act at random; you may make
mistakes, and must retrace your steps; your haste to reach your goal
will only take you further from it. Do not imitate the miser who loses
much lest he should lose a little. Sacrifice a little time in early
childhood, and it will be repaid you with usury when your scholar is
older. The wise physician does not hastily give prescriptions at first
sight, but he studies the constitution of the sick man before he
prescribes anything; the treatment is begun later, but the patient is
cured, while the hasty doctor kills him.

But where shall we find a place for our child so as to bring him up as a
senseless being, an automaton? Shall we keep him in the moon, or on a
desert island? Shall we remove him from human society? Will he not
always have around him the sight and the pattern of the passions of
other people? Will he never see children of his own age? Will he not see
his parents, his neighbours, his nurse, his governess, his man-servant,
his tutor himself, who after all will not be an angel? Here we have a
real and serious objection. But did I tell you that an education
according to nature would be an easy task? Oh, men! is it my fault that
you have made all good things difficult? I admit that I am aware of
these difficulties; perhaps they are insuperable; but nevertheless it is
certain that we do to some extent avoid them by trying to do so. I am
showing what we should try to attain, I do not say we can attain it, but
I do say that whoever comes nearest to it is nearest to success.

Remember you must be a man yourself before you try to train a man; you
yourself must set the pattern he shall copy. While the child is still
unconscious there is time to prepare his surroundings, so that nothing
shall strike his eye but what is fit for his sight. Gain the respect of
every one, begin to win their hearts, so that they may try to please
you. You will not be master of the child if you cannot control every one
about him; and this authority will never suffice unless it rests upon
respect for your goodness. There is no question of squandering one's
means and giving money right and left; I never knew money win love. You
must neither be harsh nor niggardly, nor must you merely pity misery
when you can relieve it; but in vain will you open your purse if you do
not open your heart along with it, the hearts of others will always be
closed to you. You must give your own time, attention, affection, your
very self; for whatever you do, people always perceive that your money
is not you. There are proofs of kindly interest which produce more
results and are really more useful than any gift; how many of the sick
and wretched have more need of comfort than of charity; how many of the
oppressed need protection rather than money? Reconcile those who are at
strife, prevent lawsuits; incline children to duty, fathers to kindness;
promote happy marriages; prevent annoyances; freely use the credit of
your pupil's parents on behalf of the weak who cannot obtain justice,
the weak who are oppressed by the strong. Be just, human, kindly. Do not
give alms alone, give charity; works of mercy do more than money for the
relief of suffering; love others and they will love you; serve them and
they will serve you; be their brother and they will be your children.

This is one reason why I want to bring up Emile in the country, far from
those miserable lacqueys, the most degraded of men except their masters;
far from the vile morals of the town, whose gilded surface makes them
seductive and contagious to children; while the vices of peasants,
unadorned and in their naked grossness, are more fitted to repel than to
seduce, when there is no motive for imitating them.

In the village a tutor will have much more control over the things he
wishes to show the child; his reputation, his words, his example, will
have a weight they would never have in the town; he is of use to every
one, so every one is eager to oblige him, to win his esteem, to appeal
before the disciple what the master would have him be; if vice is not
corrected, public scandal is at least avoided, which is all that our
present purpose requires.

Cease to blame others for your own faults; children are corrupted less
by what they see than by your own teaching. With your endless preaching,
moralising, and pedantry, for one idea you give your scholars, believing
it to be good, you give them twenty more which are good for nothing; you
are full of what is going on in your own minds, and you fail to see the
effect you produce on theirs. In the continual flow of words with which
you overwhelm them, do you think there is none which they get hold of in
a wrong sense? Do you suppose they do not make their own comments on
your long-winded explanations, that they do not find material for the
construction of a system they can understand---one which they will use
against you when they get the chance?

Listen to a little fellow who has just been under instruction; let him
chatter freely, ask questions, and talk at his ease, and you will be
surprised to find the strange forms your arguments have assumed in his
mind; he confuses everything, and turns everything topsy-turvy; you are
vexed and grieved by his unforeseen objections; he reduces you to be
silent yourself or to silence him: and what can he think of silence in
one who is so fond of talking? If ever he gains this advantage and is
aware of it, farewell education; from that moment all is lost; he is no
longer trying to learn, he is trying to refute you.

Zealous teachers, be simple, sensible, and reticent; be in no hurry to
act unless to prevent the actions of others. Again and again I say,
reject, if it may be, a good lesson for fear of giving a bad one. Beware
of playing the tempter in this world, which nature intended as an
earthly paradise for men, and do not attempt to give the innocent child
the knowledge of good and evil; since you cannot prevent the child
learning by what he sees outside himself, restrict your own efforts to
impressing those examples on his mind in the form best suited for him.

The explosive passions produce a great effect upon the child when he
sees them; their outward expression is very marked; he is struck by this
and his attention is arrested. Anger especially is so noisy in its rage
that it is impossible not to perceive it if you are within reach. You
need not ask yourself whether this is an opportunity for a pedagogue to
frame a fine disquisition. What! no fine disquisition, nothing, not a
word! Let the child come to you; impressed by what he has seen, he will
not fail to ask you questions. The answer is easy; it is drawn from the
very things which have appealed to his senses. He sees a flushed face,
flashing eyes, a threatening gesture, he hears cries; everything shows
that the body is ill at ease. Tell him plainly, without affectation or
mystery, ``This poor man is ill, he is in a fever.'' You may take the
opportunity of giving him in a few words some idea of disease and its
effects; for that too belongs to nature, and is one of the bonds of
necessity which he must recognise. By means of this idea, which is not
false in itself, may he not early acquire a certain aversion to giving
way to excessive passions, which he regards as diseases; and do you not
think that such a notion, given at the right moment, will produce a more
wholesome effect than the most tedious sermon? But consider the after
effects of this idea; you have authority, if ever you find it necessary,
to treat the rebellious child as a sick child; to keep him in his room,
in bed if need be, to diet him, to make him afraid of his growing vices,
to make him hate and dread them without ever regarding as a punishment
the strict measures you will perhaps have to use for his recovery. If it
happens that you yourself in a moment's heat depart from the calm and
self-control which you should aim at, do not try to conceal your fault,
but tell him frankly, with a gentle reproach, ``My dear, you have hurt
me.''

Moreover, it is a matter of great importance that no notice should be
taken in his presence of the quaint sayings which result from the
simplicity of the ideas in which he is brought up, nor should they be
quoted in a way he can understand. A foolish laugh may destroy six
months' work and do irreparable damage for life. I cannot repeat too
often that to control the child one must often control oneself.

I picture my little Emile at the height of a dispute between two
neighbours going up to the fiercest of them and saying in a tone of
pity, ``You are ill, I am very sorry for you.'' This speech will no
doubt have its effect on the spectators and perhaps on the disputants.
Without laughter, scolding, or praise I should take him away, willing or
no, before he could see this result, or at least before he could think
about it; and I should make haste to turn his thoughts to other things,
so that he would soon forget all about it.

I do not propose to enter into every detail, but only to explain general
rules and to give illustrations in cases of difficulty. I think it is
impossible to train a child up to the age of twelve in the midst of
society, without giving him some idea of the relations between one man
and another, and of the morality of human actions. It is enough to delay
the development of these ideas as long as possible, and when they can no
longer be avoided to limit them to present needs, so that he may neither
think himself master of everything nor do harm to others without knowing
or caring. There are calm and gentle characters which can be led a long
way in their first innocence without any danger; but there are also
stormy dispositions whose passions develop early; you must hasten to
make men of them lest you should have to keep them in chains.

Our first duties are to ourselves; our first feelings are centred on
self; all our instincts are at first directed to our own preservation
and our own welfare. Thus the first notion of justice springs not from
what we owe to others, but from what is due to us. Here is another error
in popular methods of education. If you talk to children of their
duties, and not of their rights, you are beginning at the wrong end, and
telling them what they cannot understand, what cannot be of any interest
to them.

If I had to train a child such as I have just described, I should say to
myself, ``A child never attacks people, {[}Footnote: A child should
never be allowed to play with grown-up people as if they were his
inferiors, nor even as if they were only his equals. If he ventured to
strike any one in earnest, were it only the footman, were it the hangman
himself, let the sufferer return his blows with interest, so that he
will not want to do it again. I have seen silly women inciting children
to rebellion, encouraging them to hit people, allowing themselves to be
beaten, and laughing at the harmless blows, never thinking that those
blows were in intention the blows of a murderer, and that the child who
desires to beat people now will desire to kill them when he is grown
up.{]} only things; and he soon learns by experience to respect those
older and stronger than himself. Things, however, do not defend
themselves. Therefore the first idea he needs is not that of liberty but
of property, and that he may get this idea he must have something of his
own.'' It is useless to enumerate his clothes, furniture, and
playthings; although he uses these he knows not how or why he has come
by them. To tell him they were given him is little better, for giving
implies having; so here is property before his own, and it is the
principle of property that you want to teach him; moreover, giving is a
convention, and the child as yet has no idea of conventions. I hope my
reader will note, in this and many other cases, how people think they
have taught children thoroughly, when they have only thrust on them
words which have no intelligible meaning to them. {[}Footnote: This is
why most children want to take back what they have given, and cry if
they cannot get it. They do not do this when once they know what a gift
is; only they are more careful about giving things away.{]}

We must therefore go back to the origin of property, for that is where
the first idea of it must begin. The child, living in the country, will
have got some idea of field work; eyes and leisure suffice for that, and
he will have both. In every age, and especially in childhood, we want to
create, to copy, to produce, to give all the signs of power and
activity. He will hardly have seen the gardener at work twice, sowing,
planting, and growing vegetables, before he will want to garden himself.

According to the principles I have already laid down, I shall not thwart
him; on the contrary, I shall approve of his plan, share his hobby, and
work with him, not for his pleasure but my own; at least, so he thinks;
I shall be his under-gardener, and dig the ground for him till his arms
are strong enough to do it; he will take possession of it by planting a
bean, and this is surely a more sacred possession, and one more worthy
of respect, than that of Nunes Balboa, who took possession of South
America in the name of the King of Spain, by planting his banner on the
coast of the Southern Sea.

We water the beans every day, we watch them coming up with the greatest
delight. Day by day I increase this delight by saying, ``Those belong to
you.'' To explain what that word ``belong'' means, I show him how he has
given his time, his labour, and his trouble, his very self to it; that
in this ground there is a part of himself which he can claim against all
the world, as he could withdraw his arm from the hand of another man who
wanted to keep it against his will.

One fine day he hurries up with his watering-can in his hand. What a
scene of woe! Alas! all the beans are pulled up, the soil is dug over,
you can scarcely find the place. Oh! what has become of my labour, my
work, the beloved fruits of my care and effort? Who has stolen my
property! Who has taken my beans? The young heart revolts; the first
feeling of injustice brings its sorrow and bitterness; tears come in
torrents, the unhappy child fills the air with cries and groans, I share
his sorrow and anger; we look around us, we make inquiries. At last we
discover that the gardener did it. We send for him.

But we are greatly mistaken. The gardener, hearing our complaint, begins
to complain louder than we:

What, gentlemen, was it you who spoilt my work! I had sown some Maltese
melons; the seed was given me as something quite out of the common, and
I meant to give you a treat when they were ripe; but you have planted
your miserable beans and destroyed my melons, which were coming up so
nicely, and I can never get any more. You have behaved very badly to me
and you have deprived yourselves of the pleasure of eating most
delicious melons.

JEAN JACQUES. My poor Robert, you must forgive us. You had given your
labour and your pains to it. I see we were wrong to spoil your work, but
we will send to Malta for some more seed for you, and we will never dig
the ground again without finding out if some one else has been
beforehand with us.

ROBERT. Well, gentlemen, you need not trouble yourselves, for there is
no more waste ground. I dig what my father tilled; every one does the
same, and all the land you see has been occupied time out of mind.

EMILE. Mr. Robert, do people often lose the seed of Maltese melons?

ROBERT. No indeed, sir; we do not often find such silly little gentlemen
as you. No one meddles with his neighbour's garden; every one respects
other people's work so that his own may be safe.

EMILE. But I have not got a garden.

ROBERT. I don't care; if you spoil mine I won't let you walk in it, for
you see I do not mean to lose my labour.

JEAN JACQUES. Could not we suggest an arrangement with this kind Robert?
Let him give my young friend and myself a corner of his garden to
cultivate, on condition that he has half the crop.

ROBERT. You may have it free. But remember I shall dig up your beans if
you touch my melons.

In this attempt to show how a child may be taught certain primitive
ideas we see how the notion of property goes back naturally to the right
of the first occupier to the results of his work. That is plain and
simple, and quite within the child's grasp. From that to the rights of
property and exchange there is but a step, after which you must stop
short.

You also see that an explanation which I can give in writing in a couple
of pages may take a year in practice, for in the course of moral ideas
we cannot advance too slowly, nor plant each step too firmly. Young
teacher, pray consider this example, and remember that your lessons
should always be in deeds rather than words, for children soon forget
what they say or what is said to them, but not what they have done nor
what has been done to them.

Such teaching should be given, as I have said, sooner or later, as the
scholar's disposition, gentle or turbulent, requires it. The way of
using it is unmistakable; but to omit no matter of importance in a
difficult business let us take another example.

Your ill-tempered child destroys everything he touches. Do not vex
yourself; put anything he can spoil out of his reach. He breaks the
things he is using; do not be in a hurry to give him more; let him feel
the want of them. He breaks the windows of his room; let the wind blow
upon him night and day, and do not be afraid of his catching cold; it is
better to catch cold than to be reckless. Never complain of the
inconvenience he causes you, but let him feel it first. At last you will
have the windows mended without saying anything. He breaks them again;
then change your plan; tell him dryly and without anger, ``The windows
are mine, I took pains to have them put in, and I mean to keep them
safe.'' Then you will shut him up in a dark place without a window. At
this unexpected proceeding he cries and howls; no one heeds. Soon he
gets tired and changes his tone; he laments and sighs; a servant
appears, the rebel begs to be let out. Without seeking any excuse for
refusing, the servant merely says, ``I, too, have windows to keep,'' and
goes away. At last, when the child has been there several hours, long
enough to get very tired of it, long enough to make an impression on his
memory, some one suggests to him that he should offer to make terms with
you, so that you may set him free and he will never break windows again.
That is just what he wants. He will send and ask you to come and see
him; you will come, he will suggest his plan, and you will agree to it
at once, saying, ``That is a very good idea; it will suit us both; why
didn't you think of it sooner?'' Then without asking for any affirmation
or confirmation of his promise, you will embrace him joyfully and take
him back at once to his own room, considering this agreement as sacred
as if he had confirmed it by a formal oath. What idea do you think he
will form from these proceedings, as to the fulfilment of a promise and
its usefulness? If I am not greatly mistaken, there is not a child upon
earth, unless he is utterly spoilt already, who could resist this
treatment, or one who would ever dream of breaking windows again on
purpose. Follow out the whole train of thought. The naughty little
fellow hardly thought when he was making a hole for his beans that he
was hewing out a cell in which his own knowledge would soon imprison
him. {[}Footnote: Moreover if the duty of keeping his word were not
established in the child's mind by its own utility, the child's growing
consciousness would soon impress it on him as a law of conscience, as an
innate principle, only requiring suitable experiences for its
development. This first outline is not sketched by man, it is engraved
on the heart by the author of all justice. Take away the primitive law
of contract and the obligation imposed by contract and there is nothing
left of human society but vanity and empty show. He who only keeps his
word because it is to his own profit is hardly more pledged than if he
had given no promise at all. This principle is of the utmost importance,
and deserves to be thoroughly studied, for man is now beginning to be at
war with himself.{]}

We are now in the world of morals, the door to vice is open. Deceit and
falsehood are born along with conventions and duties. As soon as we can
do what we ought not to do, we try to hide what we ought not to have
done. As soon as self-interest makes us give a promise, a greater
interest may make us break it; it is merely a question of doing it with
impunity; we naturally take refuge in concealment and falsehood. As we
have not been able to prevent vice, we must punish it. The sorrows of
life begin with its mistakes.

I have already said enough to show that children should never receive
punishment merely as such; it should always come as the natural
consequence of their fault. Thus you will not exclaim against their
falsehood, you will not exactly punish them for lying, but you will
arrange that all the ill effects of lying, such as not being believed
when we speak the truth, or being accused of what we have not done in
spite of our protests, shall fall on their heads when they have told a
lie. But let us explain what lying means to the child.

There are two kinds of lies; one concerns an accomplished fact, the
other concerns a future duty. The first occurs when we falsely deny or
assert that we did or did not do something, or, to put it in general
terms, when we knowingly say what is contrary to facts. The other occurs
when we promise what we do not mean to perform, or, in general terms,
when we profess an intention which we do not really mean to carry out.
These two kinds of lie are sometimes found in combination, {[}Footnote:
Thus the guilty person, accused of some evil deed, defends himself by
asserting that he is a good man. His statement is false in itself and
false in its application to the matter in hand.{]} but their differences
are my present business.

He who feels the need of help from others, he who is constantly
experiencing their kindness, has nothing to gain by deceiving them; it
is plainly to his advantage that they should see things as they are,
lest they should mistake his interests. It is therefore plain that lying
with regard to actual facts is not natural to children, but lying is
made necessary by the law of obedience; since obedience is disagreeable,
children disobey as far as they can in secret, and the present good of
avoiding punishment or reproof outweighs the remoter good of speaking
the truth. Under a free and natural education why should your child lie?
What has he to conceal from you? You do not thwart him, you do not
punish him, you demand nothing from him. Why should he not tell
everything to you as simply as to his little playmate? He cannot see
anything more risky in the one course than in the other.

The lie concerning duty is even less natural, since promises to do or
refrain from doing are conventional agreements which are outside the
state of nature and detract from our liberty. Moreover, all promises
made by children are in themselves void; when they pledge themselves
they do not know what they are doing, for their narrow vision cannot
look beyond the present. A child can hardly lie when he makes a promise;
for he is only thinking how he can get out of the present difficulty,
any means which has not an immediate result is the same to him; when he
promises for the future he promises nothing, and his imagination is as
yet incapable of projecting him into the future while he lives in the
present. If he could escape a whipping or get a packet of sweets by
promising to throw himself out of the window to-morrow, he would promise
on the spot. This is why the law disregards all promises made by minors,
and when fathers and teachers are stricter and demand that promises
shall be kept, it is only when the promise refers to something the child
ought to do even if he had made no promise.

The child cannot lie when he makes a promise, for he does not know what
he is doing when he makes his promise. The case is different when he
breaks his promise, which is a sort of retrospective falsehood; for he
clearly remembers making the promise, but he fails to see the importance
of keeping it. Unable to look into the future, he cannot foresee the
results of things, and when he breaks his promises he does nothing
contrary to his stage of reasoning.

Children's lies are therefore entirely the work of their teachers, and
to teach them to speak the truth is nothing less than to teach them the
art of lying. In your zeal to rule, control, and teach them, you never
find sufficient means at your disposal. You wish to gain fresh influence
over their minds by baseless maxims, by unreasonable precepts; and you
would rather they knew their lessons and told lies, than leave them
ignorant and truthful.

We, who only give our scholars lessons in practice, who prefer to have
them good rather than clever, never demand the truth lest they should
conceal it, and never claim any promise lest they should be tempted to
break it. If some mischief has been done in my absence and I do not know
who did it, I shall take care not to accuse Emile, nor to say, ``Did you
do it?'' {[}Footnote: Nothing could be more indiscreet than such a
question, especially if the child is guilty. Then if he thinks you know
what he has done, he will think you are setting a trap for him, and this
idea can only set him against you. If he thinks you do not know, he will
say to himself, ``Why should I make my fault known?'' And here we have
the first temptation to falsehood as the direct result of your foolish
question.{]} For in so doing what should I do but teach him to deny it?
If his difficult temperament compels me to make some agreement with him,
I will take good care that the suggestion always comes from him, never
from me; that when he undertakes anything he has always a present and
effective interest in fulfilling his promise, and if he ever fails this
lie will bring down on him all the unpleasant consequences which he sees
arising from the natural order of things, and not from his tutor's
vengeance. But far from having recourse to such cruel measures, I feel
almost certain that Emile will not know for many years what it is to
lie, and that when he does find out, he will be astonished and unable to
understand what can be the use of it. It is quite clear that the less I
make his welfare dependent on the will or the opinions of others, the
less is it to his interest to lie.

When we are in no hurry to teach there is no hurry to demand, and we can
take our time, so as to demand nothing except under fitting conditions.
Then the child is training himself, in so far as he is not being spoilt.
But when a fool of a tutor, who does not know how to set about his
business, is always making his pupil promise first this and then that,
without discrimination, choice, or proportion, the child is puzzled and
overburdened with all these promises, and neglects, forgets or even
scorns them, and considering them as so many empty phrases he makes a
game of making and breaking promises. Would you have him keep his
promise faithfully, be moderate in your claims upon him.

The detailed treatment I have just given to lying may be applied in many
respects to all the other duties imposed upon children, whereby these
duties are made not only hateful but impracticable. For the sake of a
show of preaching virtue you make them love every vice; you instil these
vices by forbidding them. Would you have them pious, you take them to
church till they are sick of it; you teach them to gabble prayers until
they long for the happy time when they will not have to pray to God. To
teach them charity you make them give alms as if you scorned to give
yourself. It is not the child, but the master, who should give; however
much he loves his pupil he should vie with him for this honour; he
should make him think that he is too young to deserve it. Alms-giving is
the deed of a man who can measure the worth of his gift and the needs of
his fellow-men. The child, who knows nothing of these, can have no merit
in giving; he gives without charity, without kindness; he is almost
ashamed to give, for, to judge by your practice and his own, he thinks
it is only children who give, and that there is no need for charity when
we are grown up.

Observe that the only things children are set to give are things of
which they do not know the value, bits of metal carried in their pockets
for which they have no further use. A child would rather give a hundred
coins than one cake. But get this prodigal giver to distribute what is
dear to him, his toys, his sweets, his own lunch, and we shall soon see
if you have made him really generous.

People try yet another way; they soon restore what he gave to the child,
so that he gets used to giving everything which he knows will come back
to him. I have scarcely seen generosity in children except of these two
types, giving what is of no use to them, or what they expect to get back
again. ``Arrange things,'' says Locke, ``so that experience may convince
them that the most generous giver gets the biggest share.'' That is to
make the child superficially generous but really greedy. He adds that
``children will thus form the habit of liberality.'' Yes, a usurer's
liberality, which expects cent. per cent. But when it is a question of
real giving, good-bye to the habit; when they do not get things back,
they will not give. It is the habit of the mind, not of the hands, that
needs watching. All the other virtues taught to children are like this,
and to preach these baseless virtues you waste their youth in sorrow.
What a sensible sort of education!

Teachers, have done with these shams; be good and kind; let your example
sink into your scholars' memories till they are old enough to take it to
heart. Rather than hasten to demand deeds of charity from my pupil I
prefer to perform such deeds in his presence, even depriving him of the
means of imitating me, as an honour beyond his years; for it is of the
utmost importance that he should not regard a man's duties as merely
those of a child. If when he sees me help the poor he asks me about it,
and it is time to reply to his questions, {[}Footnote: It must be
understood that I do not answer his questions when he wants; that would
be to subject myself to his will and to place myself in the most
dangerous state of dependence that ever a tutor was in.{]} I shall say,
``My dear boy, the rich only exist, through the good-will of the poor,
so they have promised to feed those who have not enough to live on,
either in goods or labour.'' ``Then you promised to do this?''
``Certainly; I am only master of the wealth that passes through my hands
on the condition attached to its ownership.''

After this talk (and we have seen how a child may be brought to
understand it) another than Emile would be tempted to imitate me and
behave like a rich man; in such a case I should at least take care that
it was done without ostentation; I would rather he robbed me of my
privilege and hid himself to give. It is a fraud suitable to his age,
and the only one I could forgive in him.

I know that all these imitative virtues are only the virtues of a
monkey, and that a good action is only morally good when it is done as
such and not because of others. But at an age when the heart does not
yet feel anything, you must make children copy the deeds you wish to
grow into habits, until they can do them with understanding and for the
love of what is good. Man imitates, as do the beasts. The love of
imitating is well regulated by nature; in society it becomes a vice. The
monkey imitates man, whom he fears, and not the other beasts, which he
scorns; he thinks what is done by his betters must be good. Among
ourselves, our harlequins imitate all that is good to degrade it and
bring it into ridicule; knowing their owners' baseness they try to equal
what is better than they are, or they strive to imitate what they
admire, and their bad taste appears in their choice of models, they
would rather deceive others or win applause for their own talents than
become wiser or better. Imitation has its roots in our desire to escape
from ourselves. If I succeed in my undertaking, Emile will certainly
have no such wish. So we must dispense with any seeming good that might
arise from it.

Examine your rules of education; you will find them all topsy-turvy,
especially in all that concerns virtue and morals. The only moral lesson
which is suited for a child---the most important lesson for every time
of life---is this: ``Never hurt anybody.'' The very rule of well-doing,
if not subordinated to this rule, is dangerous, false, and
contradictory. Who is there who does no good? Every one does some good,
the wicked as well as the righteous; he makes one happy at the cost of
the misery of a hundred, and hence spring all our misfortunes. The
noblest virtues are negative, they are also the most difficult, for they
make little show, and do not even make room for that pleasure so dear to
the heart of man, the thought that some one is pleased with us. If there
be a man who does no harm to his neighbours, what good must he have
accomplished! What a bold heart, what a strong character it needs! It is
not in talking about this maxim, but in trying to practise it, that we
discover both its greatness and its difficulty. {[}Footnote: The precept
``Never hurt anybody,'' implies the greatest possible independence of
human society; for in the social state one man's good is another man's
evil. This relation is part of the nature of things; it is inevitable.
You may apply this test to man in society and to the hermit to discover
which is best. A distinguished author says, ``None but the wicked can
live alone.'' I say, ``None but the good can live alone.'' This
proposition, if less sententious, is truer and more logical than the
other. If the wicked were alone, what evil would he do? It is among his
fellows that he lays his snares for others. If they wish to apply this
argument to the man of property, my answer is to be found in the passage
to which this note is appended.{]}

This will give you some slight idea of the precautions I would have you
take in giving children instruction which cannot always be refused
without risk to themselves or others, or the far greater risk of the
formation of bad habits, which would be difficult to correct later on;
but be sure this necessity will not often arise with children who are
properly brought up, for they cannot possibly become rebellious,
spiteful, untruthful, or greedy, unless the seeds of these vices are
sown in their hearts. What I have just said applies therefore rather to
the exception than the rule. But the oftener children have the
opportunity of quitting their proper condition, and contracting the
vices of men, the oftener will these exceptions arise. Those who are
brought up in the world must receive more precocious instruction than
those who are brought up in retirement. So this solitary education would
be preferable, even if it did nothing more than leave childhood time to
ripen.

There is quite another class of exceptions: those so gifted by nature
that they rise above the level of their age. As there are men who never
get beyond infancy, so there are others who are never, so to speak,
children, they are men almost from birth. The difficulty is that these
cases are very rare, very difficult to distinguish; while every mother,
who knows that a child may be a prodigy, is convinced that her child is
that one. They go further; they mistake the common signs of growth for
marks of exceptional talent. Liveliness, sharp sayings, romping, amusing
simplicity, these are the characteristic marks of this age, and show
that the child is a child indeed. Is it strange that a child who is
encouraged to chatter and allowed to say anything, who is restrained
neither by consideration nor convention, should chance to say something
clever? Were he never to hit the mark, his case would be stranger than
that of the astrologer who, among a thousand errors, occasionally
predicts the truth. ``They lie so often,'' said Henry IV., ``that at
last they say what is true.'' If you want to say something clever, you
have only to talk long enough. May Providence watch over those fine folk
who have no other claim to social distinction.

The finest thoughts may spring from a child's brain, or rather the best
words may drop from his lips, just as diamonds of great worth may fall
into his hands, while neither the thoughts nor the diamonds are his own;
at that age neither can be really his. The child's sayings do not mean
to him what they mean to us, the ideas he attaches to them are
different. His ideas, if indeed he has any ideas at all, have neither
order nor connection; there is nothing sure, nothing certain, in his
thoughts. Examine your so-called prodigy. Now and again you will
discover in him extreme activity of mind and extraordinary clearness of
thought. More often this same mind will seem slack and spiritless, as if
wrapped in mist. Sometimes he goes before you, sometimes he will not
stir. One moment you would call him a genius, another a fool. You would
be mistaken in both; he is a child, an eaglet who soars aloft for a
moment, only to drop back into the nest.

Treat him, therefore, according to his age, in spite of appearances, and
beware of exhausting his strength by over-much exercise. If the young
brain grows warm and begins to bubble, let it work freely, but do not
heat it any further, lest it lose its goodness, and when the first gases
have been given off, collect and compress the rest so that in after
years they may turn to life-giving heat and real energy. If not, your
time and your pains will be wasted, you will destroy your own work, and
after foolishly intoxicating yourself with these heady fumes, you will
have nothing left but an insipid and worthless wine.

Silly children grow into ordinary men. I know no generalisation more
certain than this. It is the most difficult thing in the world to
distinguish between genuine stupidity, and that apparent and deceitful
stupidity which is the sign of a strong character. At first sight it
seems strange that the two extremes should have the same outward signs;
and yet it may well be so, for at an age when man has as yet no true
ideas, the whole difference between the genius and the rest consists in
this: the latter only take in false ideas, while the former, finding
nothing but false ideas, receives no ideas at all. In this he resembles
the fool; the one is fit for nothing, the other finds nothing fit for
him. The only way of distinguishing between them depends upon chance,
which may offer the genius some idea which he can understand, while the
fool is always the same. As a child, the young Cato was taken for an
idiot by his parents; he was obstinate and silent, and that was all they
perceived in him; it was only in Sulla's ante-chamber that his uncle
discovered what was in him. Had he never found his way there, he might
have passed for a fool till he reached the age of reason. Had Caesar
never lived, perhaps this same Cato, who discerned his fatal genius, and
foretold his great schemes, would have passed for a dreamer all his
days. Those who judge children hastily are apt to be mistaken; they are
often more childish than the child himself. I knew a middle-aged man,
{[}Footnote: The Abbe de Condillac{]} whose friendship I esteemed an
honour, who was reckoned a fool by his family. All at once he made his
name as a philosopher, and I have no doubt posterity will give him a
high place among the greatest thinkers and the profoundest
metaphysicians of his day.

Hold childhood in reverence, and do not be in any hurry to judge it for
good or ill. Leave exceptional cases to show themselves, let their
qualities be tested and confirmed, before special methods are adopted.
Give nature time to work before you take over her business, lest you
interfere with her dealings. You assert that you know the value of time
and are afraid to waste it. You fail to perceive that it is a greater
waste of time to use it ill than to do nothing, and that a child ill
taught is further from virtue than a child who has learnt nothing at
all. You are afraid to see him spending his early years doing nothing.
What! is it nothing to be happy, nothing to run and jump all day? He
will never be so busy again all his life long. Plato, in his Republic,
which is considered so stern, teaches the children only through
festivals, games, songs, and amusements. It seems as if he had
accomplished his purpose when he had taught them to be happy; and
Seneca, speaking of the Roman lads in olden days, says, ``They were
always on their feet, they were never taught anything which kept them
sitting.'' Were they any the worse for it in manhood? Do not be afraid,
therefore, of this so-called idleness. What would you think of a man who
refused to sleep lest he should waste part of his life? You would say,
``He is mad; he is not enjoying his life, he is robbing himself of part
of it; to avoid sleep he is hastening his death.'' Remember that these
two cases are alike, and that childhood is the sleep of reason.

The apparent ease with which children learn is their ruin. You fail to
see that this very facility proves that they are not learning. Their
shining, polished brain reflects, as in a mirror, the things you show
them, but nothing sinks in. The child remembers the words and the ideas
are reflected back; his hearers understand them, but to him they are
meaningless.

Although memory and reason are wholly different faculties, the one does
not really develop apart from the other. Before the age of reason the
child receives images, not ideas; and there is this difference between
them: images are merely the pictures of external objects, while ideas
are notions about those objects determined by their relations. An image
when it is recalled may exist by itself in the mind, but every idea
implies other ideas. When we image we merely perceive, when we reason we
compare. Our sensations are merely passive, our notions or ideas spring
from an active principle which judges. The proof of this will be given
later.

I maintain, therefore, that as children are incapable of judging, they
have no true memory. They retain sounds, form, sensation, but rarely
ideas, and still more rarely relations. You tell me they acquire some
rudiments of geometry, and you think you prove your case; not so, it is
mine you prove; you show that far from being able to reason themselves,
children are unable to retain the reasoning of others; for if you follow
the method of these little geometricians you will see they only retain
the exact impression of the figure and the terms of the demonstration.
They cannot meet the slightest new objection; if the figure is reversed
they can do nothing. All their knowledge is on the sensation-level,
nothing has penetrated to their understanding. Their memory is little
better than their other powers, for they always have to learn over
again, when they are grown up, what they learnt as children.

I am far from thinking, however, that children have no sort of reason.
{[}Footnote: I have noticed again and again that it is impossible in
writing a lengthy work to use the same words always in the same sense.
There is no language rich enough to supply terms and expressions
sufficient for the modifications of our ideas. The method of defining
every term and constantly substituting the definition for the term
defined looks well, but it is impracticable. For how can we escape from
our vicious circle? Definitions would be all very well if we did not use
words in the making of them. In spite of this I am convinced that even
in our poor language we can make our meaning clear, not by always using
words in the same sense, but by taking care hat every time we use a word
the sense in which we use it is sufficiently indicated by the sense of
the context, so that each sentence in which the word occurs acts as a
sort of definition. Sometimes I say children are incapable of reasoning.
Sometimes I say they reason cleverly. I must admit that my words are
often contradictory, but I do not think there is any contradiction in my
ideas.{]} On the contrary, I think they reason very well with regard to
things that affect their actual and sensible well-being. But people are
mistaken as to the extent of their information, and they attribute to
them knowledge they do not possess, and make them reason about things
they cannot understand. Another mistake is to try to turn their
attention to matters which do not concern them in the least, such as
their future interest, their happiness when they are grown up, the
opinion people will have of them when they are men---terms which are
absolutely meaningless when addressed to creatures who are entirely
without foresight. But all the forced studies of these poor little
wretches are directed towards matters utterly remote from their minds.
You may judge how much attention they can give to them.

The pedagogues, who make a great display of the teaching they give their
pupils, are paid to say just the opposite; yet their actions show that
they think just as I do. For what do they teach? Words! words! words!
Among the various sciences they boast of teaching their scholars, they
take good care never to choose those which might be really useful to
them, for then they would be compelled to deal with things and would
fail utterly; the sciences they choose are those we seem to know when we
know their technical terms---heraldry, geography, chronology, languages,
etc., studies so remote from man, and even more remote from the child,
that it is a wonder if he can ever make any use of any part of them.

You will be surprised to find that I reckon the study of languages among
the useless lumber of education; but you must remember that I am
speaking of the studies of the earliest years, and whatever you may say,
I do not believe any child under twelve or fifteen ever really acquired
two languages.

If the study of languages were merely the study of words, that is, of
the symbols by which language expresses itself, then this might be a
suitable study for children; but languages, as they change the symbols,
also modify the ideas which the symbols express. Minds are formed by
language, thoughts take their colour from its ideas. Reason alone is
common to all. Every language has its own form, a difference which may
be partly cause and partly effect of differences in national character;
this conjecture appears to be confirmed by the fact that in every nation
under the sun speech follows the changes of manners, and is preserved or
altered along with them.

By use the child acquires one of these different forms, and it is the
only language he retains till the age of reason. To acquire two
languages he must be able to compare their ideas, and how can he compare
ideas he can barely understand? Everything may have a thousand meanings
to him, but each idea can only have one form, so he can only learn one
language. You assure me he learns several languages; I deny it. I have
seen those little prodigies who are supposed to speak half a dozen
languages. I have heard them speak first in German, then in Latin,
French, or Italian; true, they used half a dozen different vocabularies,
but they always spoke German. In a word, you may give children as many
synonyms as you like; it is not their language but their words that you
change; they will never have but one language.

To conceal their deficiencies teachers choose the dead languages, in
which we have no longer any judges whose authority is beyond dispute.
The familiar use of these tongues disappeared long ago, so they are
content to imitate what they find in books, and they call that talking.
If the master's Greek and Latin is such poor stuff, what about the
children? They have scarcely learnt their primer by heart, without
understanding a word of it, when they are set to translate a French
speech into Latin words; then when they are more advanced they piece
together a few phrases of Cicero for prose or a few lines of Vergil for
verse. Then they think they can speak Latin, and who will contradict
them?

In any study whatsoever the symbols are of no value without the idea of
the things symbolised. Yet the education of the child in confined to
those symbols, while no one ever succeeds in making him understand the
thing signified. You think you are teaching him what the world is like;
he is only learning the map; he is taught the names of towns, countries,
rivers, which have no existence for him except on the paper before him.
I remember seeing a geography somewhere which began with: ``What is the
world?''---``A sphere of cardboard.'' That is the child's geography. I
maintain that after two years' work with the globe and cosmography,
there is not a single ten-year-old child who could find his way from
Paris to Saint Denis by the help of the rules he has learnt. I maintain
that not one of these children could find his way by the map about the
paths on his father's estate without getting lost. These are the young
doctors who can tell us the position of Pekin, Ispahan, Mexico, and
every country in the world.

You tell me the child must be employed on studies which only need eyes.
That may be; but if there are any such studies, they are unknown to me.

It is a still more ridiculous error to set them to study history, which
is considered within their grasp because it is merely a collection of
facts. But what is meant by this word ``fact''? Do you think the
relations which determine the facts of history are so easy to grasp that
the corresponding ideas are easily developed in the child's mind! Do you
think that a real knowledge of events can exist apart from the knowledge
of their causes and effects, and that history has so little relation to
words that the one can be learnt without the other? If you perceive
nothing in a man's actions beyond merely physical and external
movements, what do you learn from history? Absolutely nothing; while
this study, robbed of all that makes it interesting, gives you neither
pleasure nor information. If you want to judge actions by their moral
bearings, try to make these moral bearings intelligible to your
scholars. You will soon find out if they are old enough to learn
history.

Remember, reader, that he who speaks to you is neither a scholar nor a
philosopher, but a plain man and a lover of truth; a man who is pledged
to no one party or system, a hermit, who mixes little with other men,
and has less opportunity of imbibing their prejudices, and more time to
reflect on the things that strike him in his intercourse with them. My
arguments are based less on theories than on facts, and I think I can
find no better way to bring the facts home to you than by quoting
continually some example from the observations which suggested my
arguments.

I had gone to spend a few days in the country with a worthy mother of a
family who took great pains with her children and their education. One
morning I was present while the eldest boy had his lessons. His tutor,
who had taken great pains to teach him ancient history, began upon the
story of Alexander and lighted on the well-known anecdote of Philip the
Doctor. There is a picture of it, and the story is well worth study. The
tutor, worthy man, made several reflections which I did not like with
regard to Alexander's courage, but I did not argue with him lest I
should lower him in the eyes of his pupil. At dinner they did not fail
to get the little fellow talking, French fashion. The eager spirit of a
child of his age, and the confident expectation of applause, made him
say a number of silly things, and among them from time to time there
were things to the point, and these made people forget the rest. At last
came the story of Philip the Doctor. He told it very distinctly and
prettily. After the usual meed of praise, demanded by his mother and
expected by the child himself, they discussed what he had said. Most of
them blamed Alexander's rashness, some of them, following the tutor's
example, praised his resolution, which showed me that none of those
present really saw the beauty of the story. ``For my own part,'' I said,
``if there was any courage or any steadfastness at all in Alexander's
conduct I think it was only a piece of bravado.'' Then every one agreed
that it was a piece of bravado. I was getting angry, and would have
replied, when a lady sitting beside me, who had not hitherto spoken,
bent towards me and whispered in my ear. ``Jean Jacques,'' said she,
``say no more, they will never understand you.'' I looked at her, I
recognised the wisdom of her advice, and I held my tongue.

Several things made me suspect that our young professor had not in the
least understood the story he told so prettily. After dinner I took his
hand in mine and we went for a walk in the park. When I had questioned
him quietly, I discovered that he admired the vaunted courage of
Alexander more than any one. But in what do you suppose he thought this
courage consisted? Merely in swallowing a disagreeable drink at a single
draught without hesitation and without any signs of dislike. Not a
fortnight before the poor child had been made to take some medicine
which he could hardly swallow, and the taste of it was still in his
mouth. Death, and death by poisoning, were for him only disagreeable
sensations, and senna was his only idea of poison. I must admit,
however, that Alexander's resolution had made a great impression on his
young mind, and he was determined that next time he had to take medicine
he would be an Alexander. Without entering upon explanations which were
clearly beyond his grasp, I confirmed him in his praiseworthy intention,
and returned home smiling to myself over the great wisdom of parents and
teachers who expect to teach history to children.

Such words as king, emperor, war, conquest, law, and revolution are
easily put into their mouths; but when it is a question of attaching
clear ideas to these words the explanations are very different from our
talk with Robert the gardener.

I feel sure some readers dissatisfied with that ``Say no more, Jean
Jacques,'' will ask what I really saw to admire in the conduct of
Alexander. Poor things! if you need telling, how can you comprehend it?
Alexander believed in virtue, he staked his head, he staked his own life
on that faith, his great soul was fitted to hold such a faith. To
swallow that draught was to make a noble profession of the faith that
was in him. Never did mortal man recite a finer creed. If there is an
Alexander in our own days, show me such deeds.

If children have no knowledge of words, there is no study that is
suitable for them. If they have no real ideas they have no real memory,
for I do not call that a memory which only recalls sensations. What is
the use of inscribing on their brains a list of symbols which mean
nothing to them? They will learn the symbols when they learn the things
signified; why give them the useless trouble of learning them twice
over? And yet what dangerous prejudices are you implanting when you
teach them to accept as knowledge words which have no meaning for them.
The first meaningless phrase, the first thing taken for granted on the
word of another person without seeing its use for himself, this is the
beginning of the ruin of the child's judgment. He may dazzle the eyes of
fools long enough before he recovers from such a loss. {[}Footnote: The
learning of most philosophers is like the learning of children. Vast
erudition results less in the multitude of ideas than in a multitude of
images. Dates, names, places, all objects isolated or unconnected with
ideas are merely retained in the memory for symbols, and we rarely
recall any of these without seeing the right or left page of the book in
which we read it, or the form in which we first saw it. Most science was
of this kind till recently. The science of our times is another matter;
study and observation are things of the past; we dream and the dreams of
a bad night are given to us as philosophy. You will say I too am a
dreamer; I admit it, but I do what the others fail to do, I give my
dreams as dreams, and leave the reader to discover whether there is
anything in them which may prove useful to those who are awake.{]}

No, if nature has given the child this plasticity of brain which fits
him to receive every kind of impression, it was not that you should
imprint on it the names and dates of kings, the jargon of heraldry, the
globe and geography, all those words without present meaning or future
use for the child, which flood of words overwhelms his sad and barren
childhood. But by means of this plasticity all the ideas he can
understand and use, all that concern his happiness and will some day
throw light upon his duties, should be traced at an early age in
indelible characters upon his brain, to guide him to live in such a way
as befits his nature and his powers.

Without the study of books, such a memory as the child may possess is
not left idle; everything he sees and hears makes an impression on him,
he keeps a record of men's sayings and doings, and his whole environment
is the book from which he unconsciously enriches his memory, till his
judgment is able to profit by it.

To select these objects, to take care to present him constantly with
those he may know, to conceal from him those he ought not to know, this
is the real way of training his early memory; and in this way you must
try to provide him with a storehouse of knowledge which will serve for
his education in youth and his conduct throughout life. True, this
method does not produce infant prodigies, nor will it reflect glory upon
their tutors and governesses, but it produces men, strong,
right-thinking men, vigorous both in mind and body, men who do not win
admiration as children, but honour as men.

Emile will not learn anything by heart, not even fables, not even the
fables of La Fontaine, simple and delightful as they are, for the words
are no more the fable than the words of history are history. How can
people be so blind as to call fables the child's system of morals,
without considering that the child is not only amused by the apologue
but misled by it? He is attracted by what is false and he misses the
truth, and the means adopted to make the teaching pleasant prevent him
profiting by it. Men may be taught by fables; children require the naked
truth.

All children learn La Fontaine's fables, but not one of them understands
them. It is just as well that they do not understand, for the morality
of the fables is so mixed and so unsuitable for their age that it would
be more likely to incline them to vice than to virtue. ``More
paradoxes!'' you exclaim. Paradoxes they may be; but let us see if there
is not some truth in them.

I maintain that the child does not understand the fables he is taught,
for however you try to explain them, the teaching you wish to extract
from them demands ideas which he cannot grasp, while the poetical form
which makes it easier to remember makes it harder to understand, so that
clearness is sacrificed to facility. Without quoting the host of wholly
unintelligible and useless fables which are taught to children because
they happen to be in the same book as the others, let us keep to those
which the author seems to have written specially for children.

In the whole of La Fontaine's works I only know five or six fables
conspicuous for child-like simplicity; I will take the first of these as
an example, for it is one whose moral is most suitable for all ages, one
which children get hold of with the least difficulty, which they have
most pleasure in learning, one which for this very reason the author has
placed at the beginning of his book. If his object were really to
delight and instruct children, this fable is his masterpiece. Let us go
through it and examine it briefly.

\subparagraph{THE FOX AND THE CROW}\label{id00369}

\subparagraph{A FABLE}\label{id00370}

``Maitre corbeau, sur un arbre perche'' (Mr. Crow perched on a
tree).---``Mr.!'' what does that word really mean? What does it mean
before a proper noun? What is its meaning here? What is a crow? What is
``un arbre perche''? We do not say ``on a tree perched,'' but perched on
a tree. So we must speak of poetical inversions, we must distinguish
between prose and verse.

``Tenait dans son bec un fromage'' (Held a cheese in his beak)---What
sort of a cheese? Swiss, Brie, or Dutch? If the child has never seen
crows, what is the good of talking about them? If he has seen crows will
he believe that they can hold a cheese in their beak? Your illustrations
should always be taken from nature.

``Maitre renard, par l'odeur alleche'' (Mr. Fox, attracted by the
smell).---Another Master! But the title suits the fox,---who is master
of all the tricks of his trade. You must explain what a fox is, and
distinguish between the real fox and the conventional fox of the fables.

``Alleche.'' The word is obsolete; you will have to explain it. You will
say it is only used in verse. Perhaps the child will ask why people talk
differently in verse. How will you answer that question?

``Alleche, par l'odeur d'un fromage.'' The cheese was held in his beak
by a crow perched on a tree; it must indeed have smelt strong if the
fox, in his thicket or his earth, could smell it. This is the way you
train your pupil in that spirit of right judgment, which rejects all but
reasonable arguments, and is able to distinguish between truth and
falsehood in other tales.

``Lui tient a peu pres ce langage'' (Spoke to him after this
fashion).---``Ce langage.'' So foxes talk, do they! They talk like
crows! Mind what you are about, oh, wise tutor; weigh your answer before
you give it, it is more important than you suspect.

``Eh! Bonjour, Monsieur le Corbeau!'' (``Good-day, Mr. Crow!'')---Mr.!
The child sees this title laughed to scorn before he knows it is a title
of honour. Those who say ``Monsieur du Corbeau'' will find their work
cut out for them to explain that ``du.''

``Que vous etes joli! Que vous me semblez beau!'' (``How handsome you
are, how beautiful in my eyes!'')---Mere padding. The child, finding the
same thing repeated twice over in different words, is learning to speak
carelessly. If you say this redundance is a device of the author, a part
of the fox's scheme to make his praise seem all the greater by his flow
of words, that is a valid excuse for me, but not for my pupil.

``Sans mentir, si votre ramage'' (``Without lying, if your
song'').---``Without lying.'' So people do tell lies sometimes. What
will the child think of you if you tell him the fox only says ``Sans
mentir'' because he is lying?

``Se rapporte a votre plumage'' (``Answered to your fine
feathers'').---``Answered!'' What does that mean? Try to make the child
compare qualities so different as those of song and plumage; you will
see how much he understands.

``Vous seriez le phenix des hotes de ces bois!'' (``You would be the
phoenix of all the inhabitants of this wood!'')---The phoenix! What is a
phoenix? All of a sudden we are floundering in the lies of
antiquity---we are on the edge of mythology.

``The inhabitants of this wood.'' What figurative language! The
flatterer adopts the grand style to add dignity to his speech, to make
it more attractive. Will the child understand this cunning? Does he
know, how could he possibly know, what is meant by grand style and
simple style?

``A ces mots le corbeau ne se sent pas de joie'' (At these words, the
crow is beside himself with delight).---To realise the full force of
this proverbial expression we must have experienced very strong feeling.

``Et, pour montrer sa belle voix'' (And, to show his fine
voice).---Remember that the child, to understand this line and the whole
fable, must know what is meant by the crow's fine voice.

``Il ouvre un large bec, laisse tomber sa proie'' (He opens his wide
beak and drops his prey).---This is a splendid line; its very sound
suggests a picture. I see the great big ugly gaping beak, I hear the
cheese crashing through the branches; but this kind of beauty is thrown
away upon children.

``Le renard s'en saisit, et dit, `Mon bon monsieur'\,'' (The fox catches
it, and says, ``My dear sir'').---So kindness is already folly. You
certainly waste no time in teaching your children.

``Apprenez que tout flatteur'' (``You must learn that every
flatterer'').---A general maxim. The child can make neither head nor
tail of it.

``Vit au depens de celui qui l'ecoute'' (``Lives at the expense of the
person who listens to his flattery'').---No child of ten ever understood
that.

``Ce lecon vaut bien un fromage, sans doute'' (``No doubt this lesson is
well worth a cheese'').---This is intelligible and its meaning is very
good. Yet there are few children who could compare a cheese and a
lesson, few who would not prefer the cheese. You will therefore have to
make them understand that this is said in mockery. What subtlety for a
child!

``Le corbeau, honteux et confus'' (The crow, ashamed and confused).---A
nothing pleonasm, and there is no excuse for it this time.

``Jura, mais un peu tard, qu'on ne l'y prendrait plus'' (Swore, but
rather too late, that he would not be caught in that way
again).---``Swore.'' What master will be such a fool as to try to
explain to a child the meaning of an oath?

What a host of details! but much more would be needed for the analysis
of all the ideas in this fable and their reduction to the simple and
elementary ideas of which each is composed. But who thinks this analysis
necessary to make himself intelligible to children? Who of us is
philosopher enough to be able to put himself in the child's place? Let
us now proceed to the moral.

Should we teach a six-year-old child that there are people who flatter
and lie for the sake of gain? One might perhaps teach them that there
are people who make fools of little boys and laugh at their foolish
vanity behind their backs. But the whole thing is spoilt by the cheese.
You are teaching them how to make another drop his cheese rather than
how to keep their own. This is my second paradox, and it is not less
weighty than the former one.

Watch children learning their fables and you will see that when they
have a chance of applying them they almost always use them exactly
contrary to the author's meaning; instead of being on their guard
against the fault which you would prevent or cure, they are disposed to
like the vice by which one takes advantage of another's defects. In the
above fable children laugh at the crow, but they all love the fox. In
the next fable you expect them to follow the example of the grasshopper.
Not so, they will choose the ant. They do not care to abase themselves,
they will always choose the principal part---this is the choice of
self-love, a very natural choice. But what a dreadful lesson for
children! There could be no monster more detestable than a harsh and
avaricious child, who realised what he was asked to give and what he
refused. The ant does more; she teaches him not merely to refuse but to
revile.

In all the fables where the lion plays a part, usually the chief part,
the child pretends to be the lion, and when he has to preside over some
distribution of good things, he takes care to keep everything for
himself; but when the lion is overthrown by the gnat, the child is the
gnat. He learns how to sting to death those whom he dare not attack
openly.

From the fable of the sleek dog and the starving wolf he learns a lesson
of licence rather than the lesson of moderation which you profess to
teach him. I shall never forget seeing a little girl weeping bitterly
over this tale, which had been told her as a lesson in obedience. The
poor child hated to be chained up; she felt the chain chafing her neck;
she was crying because she was not a wolf.

So from the first of these fables the child learns the basest flattery;
from the second, cruelty; from the third, injustice; from the fourth,
satire; from the fifth, insubordination. The last of these lessons is no
more suitable for your pupils than for mine, though he has no use for
it. What results do you expect to get from your teaching when it
contradicts itself! But perhaps the same system of morals which
furnishes me with objections against the fables supplies you with as
many reasons for keeping to them. Society requires a rule of morality in
our words; it also requires a rule of morality in our deeds; and these
two rules are quite different. The former is contained in the Catechism
and it is left there; the other is contained in La Fontaine's fables for
children and his tales for mothers. The same author does for both.

Let us make a bargain, M. de la Fontaine. For my own part, I undertake
to make your books my favourite study; I undertake to love you, and to
learn from your fables, for I hope I shall not mistake their meaning. As
to my pupil, permit me to prevent him studying any one of them till you
have convinced me that it is good for him to learn things three-fourths
of which are unintelligible to him, and until you can convince me that
in those fables he can understand he will never reverse the order and
imitate the villain instead of taking warning from his dupe.

When I thus get rid of children's lessons, I get rid of the chief cause
of their sorrows, namely their books. Reading is the curse of childhood,
yet it is almost the only occupation you can find for children. Emile,
at twelve years old, will hardly know what a book is. ``But,'' you say,
``he must, at least, know how to read.''

When reading is of use to him, I admit he must learn to read, but till
then he will only find it a nuisance.

If children are not to be required to do anything as a matter of
obedience, it follows that they will only learn what they perceive to be
of real and present value, either for use or enjoyment; what other
motive could they have for learning? The art of speaking to our absent
friends, of hearing their words; the art of letting them know at first
hand our feelings, our desires, and our longings, is an art whose
usefulness can be made plain at any age. How is it that this art, so
useful and pleasant in itself, has become a terror to children? Because
the child is compelled to acquire it against his will, and to use it for
purposes beyond his comprehension. A child has no great wish to perfect
himself in the use of an instrument of torture, but make it a means to
his pleasure, and soon you will not be able to keep him from it.

People make a great fuss about discovering the beat way to teach
children to read. They invent ``bureaux'' {[}Footnote: Translator's
note.---The ``bureau'' was a sort of case containing letters to be put
together to form words. It was a favourite device for the teaching of
reading and gave its name to a special method, called the bureau-method,
of learning to read.{]} and cards, they turn the nursery into a
printer's shop. Locke would have them taught to read by means of dice.
What a fine idea! And the pity of it! There is a better way than any of
those, and one which is generally overlooked---it consists in the desire
to learn. Arouse this desire in your scholar and have done with your
``bureaux'' and your dice---any method will serve.

Present interest, that is the motive power, the only motive power that
takes us far and safely. Sometimes Emile receives notes of invitation
from his father or mother, his relations or friends; he is invited to a
dinner, a walk, a boating expedition, to see some public entertainment.
These notes are short, clear, plain, and well written. Some one must
read them to him, and he cannot always find anybody when wanted; no more
consideration is shown to him than he himself showed to you yesterday.
Time passes, the chance is lost. The note is read to him at last, but it
is too late. Oh! if only he had known how to read! He receives other
notes, so short, so interesting, he would like to try to read them.
Sometimes he gets help, sometimes none. He does his best, and at last he
makes out half the note; it is something about going to-morrow to drink
cream---Where? With whom? He cannot tell---how hard he tries to make out
the rest! I do not think Emile will need a ``bureau.'' Shall I proceed
to the teaching of writing? No, I am ashamed to toy with these trifles
in a treatise on education.

I will just add a few words which contain a principle of great
importance. It is this---What we are in no hurry to get is usually
obtained with speed and certainty. I am pretty sure Emile will learn to
read and write before he is ten, just because I care very little whether
he can do so before he is fifteen; but I would rather he never learnt to
read at all, than that this art should be acquired at the price of all
that makes reading useful. What is the use of reading to him if he
always hates it? ``Id imprimis cavere oportebit, ne studia, qui amare
nondum potest, oderit, et amaritudinem semel perceptam etiam ultra rudes
annos reformidet.''---Quintil.

The more I urge my method of letting well alone, the more objections I
perceive against it. If your pupil learns nothing from you, he will
learn from others. If you do not instil truth he will learn falsehoods;
the prejudices you fear to teach him he will acquire from those about
him, they will find their way through every one of his senses; they will
either corrupt his reason before it is fully developed or his mind will
become torpid through inaction, and will become engrossed in material
things. If we do not form the habit of thinking as children, we shall
lose the power of thinking for the rest of our life.

I fancy I could easily answer that objection, but why should I answer
every objection? If my method itself answers your objections, it is
good; if not, it is good for nothing. I continue my explanation.

If, in accordance with the plan I have sketched, you follow rules which
are just the opposite of the established practice, if instead of taking
your scholar far afield, instead of wandering with him in distant
places, in far-off lands, in remote centuries, in the ends of the earth,
and in the very heavens themselves, you try to keep him to himself, to
his own concerns, you will then find him able to perceive, to remember,
and even to reason; this is nature's order. As the sentient being
becomes active his discernment develops along with his strength. Not
till his strength is in excess of what is needed for self-preservation,
is the speculative faculty developed, the faculty adapted for using this
superfluous strength for other purposes. Would you cultivate your
pupil's intelligence, cultivate the strength it is meant to control.
Give his body constant exercise, make it strong and healthy, in order to
make him good and wise; let him work, let him do things, let him run and
shout, let him be always on the go; make a man of him in strength, and
he will soon be a man in reason.

Of course by this method you will make him stupid if you are always
giving him directions, always saying come here, go there, stop, do this,
don't do that. If your head always guides his hands, his own mind will
become useless. But remember the conditions we laid down; if you are a
mere pedant it is not worth your while to read my book.

It is a lamentable mistake to imagine that bodily activity hinders the
working of the mind, as if these two kinds of activity ought not to
advance hand in hand, and as if the one were not intended to act as
guide to the other.

There are two classes of men who are constantly engaged in bodily
activity, peasants and savages, and certainly neither of these pays the
least attention to the cultivation of the mind. Peasants are rough,
coarse, and clumsy; savages are noted, not only for their keen senses,
but for great subtility of mind. Speaking generally, there is nothing
duller than a peasant or sharper than a savage. What is the cause of
this difference? The peasant has always done as he was told, what his
father did before him, what he himself has always done; he is the
creature of habit, he spends his life almost like an automaton on the
same tasks; habit and obedience have taken the place of reason.

The case of the savage is very different; he is tied to no one place, he
has no prescribed task, no superior to obey, he knows no law but his own
will; he is therefore forced to reason at every step he takes. He can
neither move nor walk without considering the consequences. Thus the
more his body is exercised, the more alert is his mind; his strength and
his reason increase together, and each helps to develop the other.

Oh, learned tutor, let us see which of our two scholars is most like the
savage and which is most like the peasant. Your scholar is subject to a
power which is continually giving him instruction; he acts only at the
word of command; he dare not eat when he is hungry, nor laugh when he is
merry, nor weep when he is sad, nor offer one hand rather than the
other, nor stir a foot unless he is told to do it; before long he will
not venture to breathe without orders. What would you have him think
about, when you do all the thinking for him? He rests securely on your
foresight, why should he think for himself? He knows you have undertaken
to take care of him, to secure his welfare, and he feels himself freed
from this responsibility. His judgment relies on yours; what you have
not forbidden that he does, knowing that he runs no risk. Why should he
learn the signs of rain? He knows you watch the clouds for him. Why
should he time his walk? He knows there is no fear of your letting him
miss his dinner hour. He eats till you tell him to stop, he stops when
you tell him to do so; he does not attend to the teaching of his own
stomach, but yours. In vain do you make his body soft by inaction; his
understanding does not become subtle. Far from it, you complete your
task of discrediting reason in his eyes, by making him use such
reasoning power as he has on the things which seem of least importance
to him. As he never finds his reason any use to him, he decides at last
that it is useless. If he reasons badly he will be found fault with;
nothing worse will happen to him; and he has been found fault with so
often that he pays no attention to it, such a common danger no longer
alarms him.

Yet you will find he has a mind. He is quick enough to chatter with the
women in the way I spoke of further back; but if he is in danger, if he
must come to a decision in difficult circumstances, you will find him a
hundredfold more stupid and silly than the son of the roughest labourer.

As for my pupil, or rather Nature's pupil, he has been trained from the
outset to be as self-reliant as possible, he has not formed the habit of
constantly seeking help from others, still less of displaying his stores
of learning. On the other hand, he exercises discrimination and
forethought, he reasons about everything that concerns himself. He does
not chatter, he acts. Not a word does he know of what is going on in the
world at large, but he knows very thoroughly what affects himself. As he
is always stirring he is compelled to notice many things, to recognise
many effects; he soon acquires a good deal of experience. Nature, not
man, is his schoolmaster, and he learns all the quicker because he is
not aware that he has any lesson to learn. So mind and body work
together. He is always carrying out his own ideas, not those of other
people, and thus he unites thought and action; as he grows in health and
strength he grows in wisdom and discernment. This is the way to attain
later on to what is generally considered incompatible, though most great
men have achieved it, strength of body and strength of mind, the reason
of the philosopher and the vigour of the athlete.

Young teacher, I am setting before you a difficult task, the art of
controlling without precepts, and doing everything without doing
anything at all. This art is, I confess, beyond your years, it is not
calculated to display your talents nor to make your value known to your
scholar's parents; but it is the only road to success. You will never
succeed in making wise men if you do not first make little imps of
mischief. This was the education of the Spartans; they were not taught
to stick to their books, they were set to steal their dinners. Were they
any the worse for it in after life? Ever ready for victory, they crushed
their foes in every kind of warfare, and the prating Athenians were as
much afraid of their words as of their blows.

When education is most carefully attended to, the teacher issues his
orders and thinks himself master, but it is the child who is really
master. He uses the tasks you set him to obtain what he wants from you,
and he can always make you pay for an hour's industry by a week's
complaisance. You must always be making bargains with him. These
bargains, suggested in your fashion, but carried out in his, always
follow the direction of his own fancies, especially when you are foolish
enough to make the condition some advantage he is almost sure to obtain,
whether he fulfils his part of the bargain or not. The child is usually
much quicker to read the master's thoughts than the master to read the
child's feelings. And that is as it should be, for all the sagacity
which the child would have devoted to self-preservation, had he been
left to himself, is now devoted to the rescue of his native freedom from
the chains of his tyrant; while the latter, who has no such pressing
need to understand the child, sometimes finds that it pays him better to
leave him in idleness or vanity.

Take the opposite course with your pupil; let him always think he is
master while you are really master. There is no subjection so complete
as that which preserves the forms of freedom; it is thus that the will
itself is taken captive. Is not this poor child, without knowledge,
strength, or wisdom, entirely at your mercy? Are you not master of his
whole environment so far as it affects him? Cannot you make of him what
you please? His work and play, his pleasure and pain, are they not,
unknown to him, under your control? No doubt he ought only to do what he
wants, but he ought to want to do nothing but what you want him to do.
He should never take a step you have not foreseen, nor utter a word you
could not foretell.

Then he can devote himself to the bodily exercises adapted to his age
without brutalising his mind; instead of developing his cunning to evade
an unwelcome control, you will then find him entirely occupied in
getting the best he can out of his environment with a view to his
present welfare, and you will be surprised by the subtlety of the means
he devises to get for himself such things as he can obtain, and to
really enjoy things without the aid of other people's ideas. You leave
him master of his own wishes, but you do not multiply his caprices. When
he only does what he wants, he will soon only do what he ought, and
although his body is constantly in motion, so far as his sensible and
present interests are concerned, you will find him developing all the
reason of which he is capable, far better and in a manner much better
fitted for him than in purely theoretical studies.

Thus when he does not find you continually thwarting him, when he no
longer distrusts you, no longer has anything to conceal from you, he
will neither tell you lies nor deceive you; he will show himself
fearlessly as he really is, and you can study him at your ease, and
surround him with all the lessons you would have him learn, without
awaking his suspicions.

Neither will he keep a curious and jealous eye on your own conduct, nor
take a secret delight in catching you at fault. It is a great thing to
avoid this. One of the child's first objects is, as I have said, to find
the weak spots in its rulers. Though this leads to spitefulness, it does
not arise from it, but from the desire to evade a disagreeable control.
Overburdened by the yoke laid upon him, he tries to shake it off, and
the faults he finds in his master give him a good opportunity for this.
Still the habit of spying out faults and delighting in them grows upon
people. Clearly we have stopped another of the springs of vice in
Emile's heart. Having nothing to gain from my faults, he will not be on
the watch for them, nor will he be tempted to look out for the faults of
others.

All these methods seem difficult because they are new to us, but they
ought not to be really difficult. I have a right to assume that you have
the knowledge required for the business you have chosen; that you know
the usual course of development of the human thought, that you can study
mankind and man, that you know beforehand the effect on your pupil's
will of the various objects suited to his age which you put before him.
You have the tools and the art to use them; are you not master of your
trade?

You speak of childish caprice; you are mistaken. Children's caprices are
never the work of nature, but of bad discipline; they have either obeyed
or given orders, and I have said again and again, they must do neither.
Your pupil will have the caprices you have taught him; it is fair you
should bear the punishment of your own faults. ``But how can I cure
them?'' do you say? That may still be done by better conduct on your own
part and great patience. I once undertook the charge of a child for a
few weeks; he was accustomed not only to have his own way, but to make
every one else do as he pleased; he was therefore capricious. The very
first day he wanted to get up at midnight, to try how far he could go
with me. When I was sound asleep he jumped out of bed, got his
dressing-gown, and waked me up. I got up and lighted the candle, which
was all he wanted. After a quarter of an hour he became sleepy and went
back to bed quite satisfied with his experiment. Two days later he
repeated it, with the same success and with no sign of impatience on my
part. When he kissed me as he lay down, I said to him very quietly, ``My
little dear, this is all very well, but do not try it again.'' His
curiosity was aroused by this, and the very next day he did not fail to
get up at the same time and woke me to see whether I should dare to
disobey him. I asked what he wanted, and he told me he could not sleep.
``So much the worse for you,'' I replied, and I lay quiet. He seemed
perplexed by this way of speaking. He felt his way to the flint and
steel and tried to strike a light. I could not help laughing when I
heard him strike his fingers. Convinced at last that he could not manage
it, he brought the steel to my bed; I told him I did not want it, and I
turned my back to him. Then he began to rush wildly about the room,
shouting, singing, making a great noise, knocking against chairs and
tables, but taking, however, good care not to hurt himself seriously,
but screaming loudly in the hope of alarming me. All this had no effect,
but I perceived that though he was prepared for scolding or anger, he
was quite unprepared for indifference.

However, he was determined to overcome my patience with his own
obstinacy, and he continued his racket so successfully that at last I
lost my temper. I foresaw that I should spoil the whole business by an
unseemly outburst of passion. I determined on another course. I got up
quietly, went to the tinder box, but could not find it; I asked him for
it, and he gave it me, delighted to have won the victory over me. I
struck a light, lighted the candle, took my young gentleman by the hand
and led him quietly into an adjoining dressing-room with the shutters
firmly fastened, and nothing he could break.

I left him there without a light; then locking him in I went back to my
bed without a word. What a noise there was! That was what I expected,
and took no notice. At last the noise ceased; I listened, heard him
settling down, and I was quite easy about him. Next morning I entered
the room at daybreak, and my little rebel was lying on a sofa enjoying a
sound and much needed sleep after his exertions.

The matter did not end there. His mother heard that the child had spent
a great part of the night out of bed. That spoilt the whole thing; her
child was as good as dead. Finding a good chance for revenge, he
pretended to be ill, not seeing that he would gain nothing by it. They
sent for the doctor. Unluckily for the mother, the doctor was a
practical joker, and to amuse himself with her terrors he did his best
to increase them. However, he whispered to me, ``Leave it to me, I
promise to cure the child of wanting to be ill for some time to come.''
As a matter of fact he prescribed bed and dieting, and the child was
handed over to the apothecary. I sighed to see the mother cheated on
every hand except by me, whom she hated because I did not deceive her.

After pretty severe reproaches, she told me her son was delicate, that
he was the sole heir of the family, his life must be preserved at all
costs, and she would not have him contradicted. In that I thoroughly
agreed with her, but what she meant by contradicting was not obeying him
in everything. I saw I should have to treat the mother as I had treated
the son. ``Madam,'' I said coldly, ``I do not know how to educate the
heir to a fortune, and what is more, I do not mean to study that art.
You can take that as settled.'' I was wanted for some days longer, and
the father smoothed things over. The mother wrote to the tutor to hasten
his return, and the child, finding he got nothing by disturbing my rest,
nor yet by being ill, decided at last to get better and to go to sleep.

You can form no idea of the number of similar caprices to which the
little tyrant had subjected his unlucky tutor; for his education was
carried on under his mother's eye, and she would not allow her son and
heir to be disobeyed in anything. Whenever he wanted to go out, you must
be ready to take him, or rather to follow him, and he always took good
care to choose the time when he knew his tutor was very busy. He wished
to exercise the same power over me and to avenge himself by day for
having to leave me in peace at night. I gladly agreed and began by
showing plainly how pleased I was to give him pleasure; after that when
it was a matter of curing him of his fancies I set about it differently.

In the first place, he must be shown that he was in the wrong. This was
not difficult; knowing that children think only of the present, I took
the easy advantage which foresight gives; I took care to provide him
with some indoor amusement of which he was very fond. Just when he was
most occupied with it, I went and suggested a short walk, and he sent me
away. I insisted, but he paid no attention. I had to give in, and he
took note of this sign of submission.

The next day it was my turn. As I expected, he got tired of his
occupation; I, however, pretended to be very busy. That was enough to
decide him. He came to drag me from my work, to take him at once for a
walk. I refused; he persisted. ``No,'' I said, ``when I did what you
wanted, you taught me how to get my own way; I shall not go out.''
``Very well,'' he replied eagerly, ``I shall go out by myself.'' ``As
you please,'' and I returned to my work.

He put on his things rather uneasily when he saw I did not follow his
example. When he was ready he came and made his bow; I bowed too; he
tried to frighten me with stories of the expeditions he was going to
make; to hear him talk you would think he was going to the world's end.
Quite unmoved, I wished him a pleasant journey. He became more and more
perplexed. However, he put a good face on it, and when he was ready to
go out he told his foot man to follow him. The footman, who had his
instructions, replied that he had no time, and that he was busy carrying
out my orders, and he must obey me first. For the moment the child was
taken aback. How could he think they would really let him go out alone,
him, who, in his own eyes, was the most important person in the world,
who thought that everything in heaven and earth was wrapped up in his
welfare? However, he was beginning to feel his weakness, he perceived
that he should find himself alone among people who knew nothing of him.
He saw beforehand the risks he would run; obstinacy alone sustained him;
very slowly and unwillingly he went downstairs. At last he went out into
the street, consoling himself a little for the harm that might happen to
himself, in the hope that I should be held responsible for it.

This was just what I expected. All was arranged beforehand, and as it
meant some sort of public scene I had got his father's consent. He had
scarcely gone a few steps, when he heard, first on this side then on
that, all sorts of remarks about himself. ``What a pretty little
gentleman, neighbour? Where is he going all alone? He will get lost! I
will ask him into our house.'' ``Take care you don't. Don't you see he
is a naughty little boy, who has been turned out of his own house
because he is good for nothing? You must not stop naughty boys; let him
go where he likes.'' ``Well, well; the good God take care of him. I
should be sorry if anything happened to him.'' A little further on he
met some young urchins of about his own age who teased him and made fun
of him. The further he got the more difficulties he found. Alone and
unprotected he was at the mercy of everybody, and he found to his great
surprise that his shoulder knot and his gold lace commanded no respect.

However, I had got a friend of mine, who was a stranger to him, to keep
an eye on him. Unnoticed by him, this friend followed him step by step,
and in due time he spoke to him. The role, like that of Sbrigani in
Pourceaugnac, required an intelligent actor, and it was played to
perfection. Without making the child fearful and timid by inspiring
excessive terror, he made him realise so thoroughly the folly of his
exploit that in half an hour's time he brought him home to me, ashamed
and humble, and afraid to look me in the face.

To put the finishing touch to his discomfiture, just as he was coming in
his father came down on his way out and met him on the stairs. He had to
explain where he had been, and why I was not with him. {[}Footnote: In a
case like this there is no danger in asking a child to tell the truth,
for he knows very well that it cannot be hid, and that if he ventured to
tell a lie he would be found out at once.{]} The poor child would gladly
have sunk into the earth. His father did not take the trouble to scold
him at length, but said with more severity than I should have expected,
``When you want to go out by yourself, you can do so, but I will not
have a rebel in my house, so when you go, take good care that you never
come back.''

As for me, I received him somewhat gravely, but without blame and
without mockery, and for fear he should find out we had been playing
with him, I declined to take him out walking that day. Next day I was
well pleased to find that he passed in triumph with me through the very
same people who had mocked him the previous day, when they met him out
by himself. You may be sure he never threatened to go out without me
again.

By these means and other like them I succeeded during the short time I
was with him in getting him to do everything I wanted without bidding
him or forbidding him to do anything, without preaching or exhortation,
without wearying him with unnecessary lessons. So he was pleased when I
spoke to him, but when I was silent he was frightened, for he knew there
was something amiss, and he always got his lesson from the thing itself.
But let us return to our subject.

The body is strengthened by this constant exercise under the guidance of
nature herself, and far from brutalising the mind, this exercise
develops in it the only kind of reason of which young children are
capable, the kind of reason most necessary at every age. It teaches us
how to use our strength, to perceive the relations between our own and
neighbouring bodies, to use the natural tools, which are within our
reach and adapted to our senses. Is there anything sillier than a child
brought up indoors under his mother's eye, who, in his ignorance of
weight and resistance, tries to uproot a tall tree or pick up a rock.
The first time I found myself outside Geneva I tried to catch a
galloping horse, and I threw stones at Mont Saleve, two leagues away; I
was the laughing stock of the whole village, and was supposed to be a
regular idiot. At eighteen we are taught in our natural philosophy the
use of the lever; every village boy of twelve knows how to use a lever
better than the cleverest mechanician in the academy. The lessons the
scholars learn from one another in the playground are worth a
hundredfold more than what they learn in the class-room.

Watch a cat when she comes into a room for the first time; she goes from
place to place, she sniffs about and examines everything, she is never
still for a moment; she is suspicious of everything till she has
examined it and found out what it is. It is the same with the child when
he begins to walk, and enters, so to speak, the room of the world around
him. The only difference is that, while both use sight, the child uses
his hands and the cat that subtle sense of smell which nature has
bestowed upon it. It is this instinct, rightly or wrongly educated,
which makes children skilful or clumsy, quick or slow, wise or foolish.

Man's primary natural goals are, therefore, to measure himself against
his environment, to discover in every object he sees those sensible
qualities which may concern himself, so his first study is a kind of
experimental physics for his own preservation. He is turned away from
this and sent to speculative studies before he has found his proper
place in the world. While his delicate and flexible limbs can adjust
themselves to the bodies upon which they are intended to act, while his
senses are keen and as yet free from illusions, then is the time to
exercise both limbs and senses in their proper business. It is the time
to learn to perceive the physical relations between ourselves and
things. Since everything that comes into the human mind enters through
the gates of sense, man's first reason is a reason of sense-experience.
It is this that serves as a foundation for the reason of the
intelligence; our first teachers in natural philosophy are our feet,
hands, and eyes. To substitute books for them does not teach us to
reason, it teaches us to use the reason of others rather than our own;
it teaches us to believe much and know little.

Before you can practise an art you must first get your tools; and if you
are to make good use of those tools, they must be fashioned sufficiently
strong to stand use. To learn to think we must therefore exercise our
limbs, our senses, and our bodily organs, which are the tools of the
intellect; and to get the best use out of these tools, the body which
supplies us with them must be strong and healthy. Not only is it quite a
mistake that true reason is developed apart from the body, but it is a
good bodily constitution which makes the workings of the mind easy and
correct.

While I am showing how the child's long period of leisure should be
spent, I am entering into details which may seem absurd. You will say,
``This is a strange sort of education, and it is subject to your own
criticism, for it only teaches what no one needs to learn. Why spend
your time in teaching what will come of itself without care or trouble?
Is there any child of twelve who is ignorant of all you wish to teach
your pupil, while he also knows what his master has taught him.''

Gentlemen, you are mistaken. I am teaching my pupil an art, the
acquirement of which demands much time and trouble, an art which your
scholars certainly do not possess; it is the art of being ignorant; for
the knowledge of any one who only thinks he knows, what he really does
know is a very small matter. You teach science; well and good; I am busy
fashioning the necessary tools for its acquisition. Once upon a time,
they say the Venetians were displaying the treasures of the Cathedral of
Saint Mark to the Spanish ambassador; the only comment he made was,
``Qui non c'e la radice.'' When I see a tutor showing off his pupil's
learning, I am always tempted to say the same to him.

Every one who has considered the manner of life among the ancients,
attributes the strength of body and mind by which they are distinguished
from the men of our own day to their gymnastic exercises. The stress
laid by Montaigne upon this opinion, shows that it had made a great
impression on him; he returns to it again and again. Speaking of a
child's education he says, ``To strengthen the mind you must harden the
muscles; by training the child to labour you train him to suffering; he
must be broken in to the hardships of gymnastic exercises to prepare him
for the hardships of dislocations, colics, and other bodily ills.'' The
philosopher Locke, the worthy Rollin, the learned Fleury, the pedant De
Crouzas, differing as they do so widely from one another, are agreed in
this one matter of sufficient bodily exercise for children. This is the
wisest of their precepts, and the one which is certain to be neglected.
I have already dwelt sufficiently on its importance, and as better
reasons and more sensible rules cannot be found than those in Locke's
book, I will content myself with referring to it, after taking the
liberty of adding a few remarks of my own.

The limbs of a growing child should be free to move easily in his
clothing; nothing should cramp their growth or movement; there should be
nothing tight, nothing fitting closely to the body, no belts of any
kind. The French style of dress, uncomfortable and unhealthy for a man,
is especially bad for children. The stagnant humours, whose circulation
is interrupted, putrify in a state of inaction, and this process
proceeds more rapidly in an inactive and sedentary life; they become
corrupt and give rise to scurvy; this disease, which is continually on
the increase among us, was almost unknown to the ancients, whose way of
dressing and living protected them from it. The hussar's dress, far from
correcting this fault, increases it, and compresses the whole of the
child's body, by way of dispensing with a few bands. The best plan is to
keep children in frocks as long as possible and then to provide them
with loose clothing, without trying to define the shape which is only
another way of deforming it. Their defects of body and mind may all be
traced to the same source, the desire to make men of them before their
time.

There are bright colours and dull; children like the bright colours
best, and they suit them better too. I see no reason why such natural
suitability should not be taken into consideration; but as soon as they
prefer a material because it is rich, their hearts are already given
over to luxury, to every caprice of fashion, and this taste is certainly
not their own. It is impossible to say how much education is influenced
by this choice of clothes, and the motives for this choice. Not only do
short-sighted mothers offer ornaments as rewards to their children, but
there are foolish tutors who threaten to make their pupils wear the
plainest and coarsest clothes as a punishment. ``If you do not do your
lessons better, if you do not take more care of your clothes, you shall
be dressed like that little peasant boy.'' This is like saying to them,
``Understand that clothes make the man.'' Is it to be wondered at that
our young people profit by such wise teaching, that they care for
nothing but dress, and that they only judge of merit by its outside.

If I had to bring such a spoilt child to his senses, I would take care
that his smartest clothes were the most uncomfortable, that he was
always cramped, constrained, and embarrassed in every way; freedom and
mirth should flee before his splendour. If he wanted to take part in the
games of children more simply dressed, they should cease their play and
run away. Before long I should make him so tired and sick of his
magnificence, such a slave to his gold-laced coat, that it would become
the plague of his life, and he would be less afraid to behold the
darkest dungeon than to see the preparations for his adornment. Before
the child is enslaved by our prejudices his first wish is always to be
free and comfortable. The plainest and most comfortable clothes, those
which leave him most liberty, are what he always likes best.

There are habits of body suited for an active life and others for a
sedentary life. The latter leaves the humours an equable and uniform
course, and the body should be protected from changes in temperature;
the former is constantly passing from action to rest, from heat to cold,
and the body should be inured to these changes. Hence people, engaged in
sedentary pursuits indoors, should always be warmly dressed, to keep
their bodies as nearly as possible at the same temperature at all times
and seasons. Those, however, who come and go in sun, wind, and rain, who
take much exercise, and spend most of their time out of doors, should
always be lightly clad, so as to get used to the changes in the air and
to every degree of temperature without suffering inconvenience. I would
advise both never to change their clothes with the changing seasons, and
that would be the invariable habit of my pupil Emile. By this I do not
mean that he should wear his winter clothes in summer like many people
of sedentary habits, but that he should wear his summer clothes in
winter like hard-working folk. Sir Isaac Newton always did this, and he
lived to be eighty.

Emile should wear little or nothing on his head all the year round. The
ancient Egyptians always went bareheaded; the Persians used to wear
heavy tiaras and still wear large turbans, which according to Chardin
are required by their climate. I have remarked elsewhere on the
difference observed by Herodotus on a battle-field between the skulls of
the Persians and those of the Egyptians. Since it is desirable that the
bones of the skull should grow harder and more substantial, less fragile
and porous, not only to protect the brain against injuries but against
colds, fever, and every influence of the air, you should therefore
accustom your children to go bare-headed winter and summer, day and
night. If you make them wear a night-cap to keep their hair clean and
tidy, let it be thin and transparent like the nets with which the
Basques cover their hair. I am aware that most mothers will be more
impressed by Chardin's observations than my arguments, and will think
that all climates are the climate of Persia, but I did not choose a
European pupil to turn him into an Asiatic.

Children are generally too much wrapped up, particularly in infancy.
They should be accustomed to cold rather than heat; great cold never
does them any harm, if they are exposed to it soon enough; but their
skin is still too soft and tender and leaves too free a course for
perspiration, so that they are inevitably exhausted by excessive heat.
It has been observed that infant mortality is greatest in August.
Moreover, it seems certain from a comparison of northern and southern
races that we become stronger by bearing extreme cold rather than
excessive heat. But as the child's body grows bigger and his muscles get
stronger, train him gradually to bear the rays of the sun. Little by
little you will harden him till he can face the burning heat of the
tropics without danger.

Locke, in the midst of the manly and sensible advice he gives us, falls
into inconsistencies one would hardly expect in such a careful thinker.
The same man who would have children take an ice-cold bath summer and
winter, will not let them drink cold water when they are hot, or lie on
damp grass. But he would never have their shoes water-tight; and why
should they let in more water when the child is hot than when he is
cold, and may we not draw the same inference with regard to the feet and
body that he draws with regard to the hands and feet and the body and
face? If he would have a man all face, why blame me if I would have him
all feet?

To prevent children drinking when they are hot, he says they should be
trained to eat a piece of bread first. It is a strange thing to make a
child eat because he is thirsty; I would as soon give him a drink when
he is hungry. You will never convince me that our first instincts are so
ill-regulated that we cannot satisfy them without endangering our lives.
Were that so, the man would have perished over and over again before he
had learned how to keep himself alive.

Whenever Emile is thirsty let him have a drink, and let him drink fresh
water just as it is, not even taking the chill off it in the depths of
winter and when he is bathed in perspiration. The only precaution I
advise is to take care what sort of water you give him. If the water
comes from a river, give it him just as it is; if it is spring-water let
it stand a little exposed to the air before he drinks it. In warm
weather rivers are warm; it is not so with springs, whose water has not
been in contact with the air. You must wait till the temperature of the
water is the same as that of the air. In winter, on the other hand,
spring water is safer than river water. It is, however, unusual and
unnatural to perspire greatly in winter, especially in the open air, for
the cold air constantly strikes the skin and drives the perspiration
inwards, and prevents the pores opening enough to give it passage. Now I
do not intend Emile to take his exercise by the fireside in winter, but
in the open air and among the ice. If he only gets warm with making and
throwing snowballs, let him drink when he is thirsty, and go on with his
game after drinking, and you need not be afraid of any ill effects. And
if any other exercise makes him perspire let him drink cold water even
in winter provided he is thirsty. Only take care to take him to get the
water some little distance away. In such cold as I am supposing, he
would have cooled down sufficiently when he got there to be able to
drink without danger. Above all, take care to conceal these precautions
from him. I would rather he were ill now and then, than always thinking
about his health.

Since children take such violent exercise they need a great deal of
sleep. The one makes up for the other, and this shows that both are
necessary. Night is the time set apart by nature for rest. It is an
established fact that sleep is quieter and calmer when the sun is below
the horizon, and that our senses are less calm when the air is warmed by
the rays of the sun. So it is certainly the healthiest plan to rise with
the sun and go to bed with the sun. Hence in our country man and all the
other animals with him want more sleep in winter than in summer. But
town life is so complex, so unnatural, so subject to chances and
changes, that it is not wise to accustom a man to such uniformity that
he cannot do without it. No doubt he must submit to rules; but the chief
rule is this---be able to break the rule if necessary. So do not be so
foolish as to soften your pupil by letting him always sleep his sleep
out. Leave him at first to the law of nature without any hindrance, but
never forget that under our conditions he must rise above this law; he
must be able to go to bed late and rise early, be awakened suddenly, or
sit up all night without ill effects. Begin early and proceed gently, a
step at a time, and the constitution adapts itself to the very
conditions which would destroy it if they were imposed for the first
time on the grown man.

In the next place he must be accustomed to sleep in an uncomfortable
bed, which is the best way to find no bed uncomfortable. Speaking
generally, a hard life, when once we have become used to it, increases
our pleasant experiences; an easy life prepares the way for innumerable
unpleasant experiences. Those who are too tenderly nurtured can only
sleep on down; those who are used to sleep on bare boards can find them
anywhere. There is no such thing as a hard bed for the man who falls
asleep at once.

The body is, so to speak, melted and dissolved in a soft bed where one
sinks into feathers and eider-down. The reins when too warmly covered
become inflamed. Stone and other diseases are often due to this, and it
invariably produces a delicate constitution, which is the seed-ground of
every ailment.

The best bed is that in which we get the best sleep. Emile and I will
prepare such a bed for ourselves during the daytime. We do not need
Persian slaves to make our beds; when we are digging the soil we are
turning our mattresses. I know that a healthy child may be made to sleep
or wake almost at will. When the child is put to bed and his nurse grows
weary of his chatter, she says to him, ``Go to sleep.'' That is much
like saying, ``Get well,'' when he is ill. The right way is to let him
get tired of himself. Talk so much that he is compelled to hold his
tongue, and he will soon be asleep. Here is at least one use for
sermons, and you may as well preach to him as rock his cradle; but if
you use this narcotic at night, do not use it by day.

I shall sometimes rouse Emile, not so much to prevent his sleeping too
much, as to accustom him to anything---even to waking with a start.
Moreover, I should be unfit for my business if I could not make him wake
himself, and get up, so to speak, at my will, without being called.

If he wakes too soon, I shall let him look forward to a tedious morning,
so that he will count as gain any time he can give to sleep. If he
sleeps too late I shall show him some favourite toy when he wakes. If I
want him to wake at a given hour I shall say, ``To-morrow at six I am
going fishing,'' or ``I shall take a walk to such and such a place.
Would you like to come too?'' He assents, and begs me to wake him. I
promise, or do not promise, as the case requires. If he wakes too late,
he finds me gone. There is something amiss if he does not soon learn to
wake himself.

Moreover, should it happen, though it rarely does, that a sluggish child
desires to stagnate in idleness, you must not give way to this tendency,
which might stupefy him entirely, but you must apply some stimulus to
wake him. You must understand that is no question of applying force, but
of arousing some appetite which leads to action, and such an appetite,
carefully selected on the lines laid down by nature, kills two birds
with one stone.

If one has any sort of skill, I can think of nothing for which a taste,
a very passion, cannot be aroused in children, and that without vanity,
emulation, or jealousy. Their keenness, their spirit of imitation, is
enough of itself; above all, there is their natural liveliness, of which
no teacher so far has contrived to take advantage. In every game, when
they are quite sure it is only play, they endure without complaint, or
even with laughter, hardships which they would not submit to otherwise
without floods of tears. The sports of the young savage involve long
fasting, blows, burns, and fatigue of every kind, a proof that even pain
has a charm of its own, which may remove its bitterness. It is not every
master, however, who knows how to season this dish, nor can every
scholar eat it without making faces. However, I must take care or I
shall be wandering off again after exceptions.

It is not to be endured that man should become the slave of pain,
disease, accident, the perils of life, or even death itself; the more
familiar he becomes with these ideas the sooner he will be cured of that
over-sensitiveness which adds to the pain by impatience in bearing it;
the sooner he becomes used to the sufferings which may overtake him, the
sooner he shall, as Montaigne has put it, rob those pains of the sting
of unfamiliarity, and so make his soul strong and invulnerable; his body
will be the coat of mail which stops all the darts which might otherwise
find a vital part. Even the approach of death, which is not death
itself, will scarcely be felt as such; he will not die, he will be, so
to speak, alive or dead and nothing more. Montaigne might say of him as
he did of a certain king of Morocco, ``No man ever prolonged his life so
far into death.'' A child serves his apprenticeship in courage and
endurance as well as in other virtues; but you cannot teach children
these virtues by name alone; they must learn them unconsciously through
experience.

But speaking of death, what steps shall I take with regard to my pupil
and the smallpox? Shall he be inoculated in infancy, or shall I wait
till he takes it in the natural course of things? The former plan is
more in accordance with our practice, for it preserves his life at a
time when it is of greater value, at the cost of some danger when his
life is of less worth; if indeed we can use the word danger with regard
to inoculation when properly performed.

But the other plan is more in accordance with our general
principles---to leave nature to take the precautions she delights in,
precautions she abandons whenever man interferes. The natural man is
always ready; let nature inoculate him herself, she will choose the
fitting occasion better than we.

Do not think I am finding fault with inoculation, for my reasons for
exempting my pupil from it do not in the least apply to yours. Your
training does not prepare them to escape catching smallpox as soon as
they are exposed to infection. If you let them take it anyhow, they will
probably die. I perceive that in different lands the resistance to
inoculation is in proportion to the need for it; and the reason is
plain. So I scarcely condescend to discuss this question with regard to
Emile. He will be inoculated or not according to time, place, and
circumstances; it is almost a matter of indifference, as far as he is
concerned. If it gives him smallpox, there will be the advantage of
knowing what to expect, knowing what the disease is; that is a good
thing, but if he catches it naturally it will have kept him out of the
doctor's hands, which is better.

An exclusive education, which merely tends to keep those who have
received it apart from the mass of mankind, always selects such teaching
as is costly rather than cheap, even when the latter is of more use.
Thus all carefully educated young men learn to ride, because it is
costly, but scarcely any of them learn to swim, as it costs nothing, and
an artisan can swim as well as any one. Yet without passing through the
riding school, the traveller learns to mount his horse, to stick on it,
and to ride well enough for practical purposes; but in the water if you
cannot swim you will drown, and we cannot swim unless we are taught.
Again, you are not forced to ride on pain of death, while no one is sure
of escaping such a common danger as drowning. Emile shall be as much at
home in the water as on land. Why should he not be able to live in every
element? If he could learn to fly, he should be an eagle; I would make
him a salamander, if he could bear the heat.

People are afraid lest the child should be drowned while he is learning
to swim; if he dies while he is learning, or if he dies because he has
not learnt, it will be your own fault. Foolhardiness is the result of
vanity; we are not rash when no one is looking. Emile will not be
foolhardy, though all the world were watching him. As the exercise does
not depend on its danger, he will learn to swim the Hellespont by
swimming, without any danger, a stream in his father's park; but he must
get used to danger too, so as not to be flustered by it. This is an
essential part of the apprenticeship I spoke of just now. Moreover, I
shall take care to proportion the danger to his strength, and I shall
always share it myself, so that I need scarcely fear any imprudence if I
take as much care for his life as for my own.

A child is smaller than a man; he has not the man's strength or reason,
but he sees and hears as well or nearly as well; his sense of taste is
very good, though he is less fastidious, and he distinguishes scents as
clearly though less sensuously. The senses are the first of our
faculties to mature; they are those most frequently overlooked or
neglected.

To train the senses it is not enough merely to use them; we must learn
to judge by their means, to learn to feel, so to speak; for we cannot
touch, see, or hear, except as we have been taught.

There is a mere natural and mechanical use of the senses which
strengthens the body without improving the judgment. It is all very well
to swim, run, jump, whip a top, throw stones; but have we nothing but
arms and legs? Have we not eyes and ears as well; and are not these
organs necessary for the use of the rest? Do not merely exercise the
strength, exercise all the senses by which it is guided; make the best
use of every one of them, and check the results of one by the other.
Measure, count, weigh, compare. Do not use force till you have estimated
the resistance; let the estimation of the effect always precede the
application of the means. Get the child interested in avoiding
insufficient or superfluous efforts. If in this way you train him to
calculate the effects of all his movements, and to correct his mistakes
by experience, is it not clear that the more he does the wiser he will
become?

Take the case of moving a heavy mass; if he takes too long a lever, he
will waste his strength; if it is too short, he will not have strength
enough; experience will teach him to use the very stick he needs. This
knowledge is not beyond his years. Take, for example, a load to be
carried; if he wants to carry as much as he can, and not to take up more
than he can carry, must he not calculate the weight by the appearance?
Does he know how to compare masses of like substance and different size,
or to choose between masses of the same size and different substances?
He must set to work to compare their specific weights. I have seen a
young man, very highly educated, who could not be convinced, till he had
tried it, that a bucket full of blocks of oak weighed less than the same
bucket full of water.

All our senses are not equally under our control. One of them, touch, is
always busy during our waking hours; it is spread over the whole surface
of the body, like a sentinel ever on the watch to warn us of anything
which may do us harm. Whether we will or not, we learn to use it first
of all by experience, by constant practice, and therefore we have less
need for special training for it. Yet we know that the blind have a
surer and more delicate sense of touch than we, for not being guided by
the one sense, they are forced to get from the touch what we get from
sight. Why, then, are not we trained to walk as they do in the dark, to
recognise what we touch, to distinguish things about us; in a word, to
do at night and in the dark what they do in the daytime without sight?
We are better off than they while the sun shines; in the dark it is
their turn to be our guide. We are blind half our time, with this
difference: the really blind always know what to do, while we are afraid
to stir in the dark. We have lights, you say. What always artificial
aids. Who can insure that they will always be at hand when required. I
had rather Emil's eyes were in his finger tips, than in the chandler's
shop.

If you are shut up in a building at night, clap your hands, you will
know from the sound whether the space is large or small, if you are in
the middle or in one corner. Half a foot from a wall the air, which is
refracted and does not circulate freely, produces a different effect on
your face. Stand still in one place and turn this way and that; a slight
draught will tell you if there is a door open. If you are on a boat you
will perceive from the way the air strikes your face not merely the
direction in which you are going, but whether the current is bearing you
slow or fast. These observations and many others like them can only be
properly made at night; however much attention we give to them by
daylight, we are always helped or hindered by sight, so that the results
escape us. Yet here we use neither hand nor stick. How much may be
learnt by touch, without ever touching anything!

I would have plenty of games in the dark! This suggestion is more
valuable than it seems at first sight. Men are naturally afraid of the
dark; so are some animals. {[}Footnote: This terror is very noticeable
during great eclipses of the sun.{]} Only a few men are freed from this
burden by knowledge, determination, and courage. I have seen thinkers,
unbelievers, philosophers, exceedingly brave by daylight, tremble like
women at the rustling of a leaf in the dark. This terror is put down to
nurses' tales; this is a mistake; it has a natural cause. What is this
cause? What makes the deaf suspicious and the lower classes
superstitious? Ignorance of the things about us, and of what is taking
place around us. {[}Footnote: Another cause has been well explained by a
philosopher, often quoted in this work, a philosopher to whose wide
views I am very greatly indebted.{]}

When under special conditions we cannot form a fair idea of distance,
when we can only judge things by the size of the angle or rather of the
image formed in our eyes, we cannot avoid being deceived as to the size
of these objects. Every one knows by experience how when we are
travelling at night we take a bush near at hand for a great tree at a
distance, and vice versa. In the same way, if the objects were of a
shape unknown to us, so that we could not tell their size in that way,
we should be equally mistaken with regard to it. If a fly flew quickly
past a few inches from our eyes, we should think it was a distant bird;
a horse standing still at a distance from us in the midst of open
country, in a position somewhat like that of a sheep, would be taken for
a large sheep, so long as we did not perceive that it was a horse; but
as soon as we recognise what it is, it seems as large as a horse, and we
at once correct our former judgment.

Whenever one finds oneself in unknown places at night where we cannot
judge of distance, and where we cannot recognise objects by their shape
on account of the darkness, we are in constant danger of forming
mistaken judgments as to the objects which present themselves to our
notice. Hence that terror, that kind of inward fear experienced by most
people on dark nights. This is foundation for the supposed appearances
of spectres, or gigantic and terrible forms which so many people profess
to have seen. They are generally told that they imagined these things,
yet they may really have seen them, and it is quite possible they really
saw what they say they did see; for it will always be the case that when
we can only estimate the size of an object by the angle it forms in the
eye, that object will swell and grow as we approach it; and if the
spectator thought it several feet high when it was thirty or forty feet
away, it will seem very large indeed when it is a few feet off; this
must indeed astonish and alarm the spectator until he touches it and
perceives what it is, for as soon as he perceives what it is, the object
which seemed so gigantic will suddenly shrink and assume its real size,
but if we run away or are afraid to approach, we shall certainly form no
other idea of the thing than the image formed in the eye, and we shall
have really seen a gigantic figure of alarming size and shape. There is,
therefore, a natural ground for the tendency to see ghosts, and these
appearances are not merely the creation of the imagination, as the men
of science would have us think.---Buffon, Nat. Hist.

In the text I have tried to show that they are always partly the
creation of the imagination, and with regard to the cause explained in
this quotation, it is clear that the habit of walking by night should
teach us to distinguish those appearances which similarity of form and
diversity of distance lend to the objects seen in the dark. For if the
air is light enough for us to see the outlines there must be more air
between us and them when they are further off, so that we ought to see
them less distinctly when further off, which should be enough, when we
are used to it, to prevent the error described by M. Buffon.
{[}Whichever explanation you prefer, my mode of procedure is still
efficacious, and experience entirely confirms it.{]} Accustomed to
perceive things from a distance and to calculate their effects, how can
I help supposing, when I cannot see, that there are hosts of creatures
and all sorts of movements all about me which may do me harm, and
against which I cannot protect myself? In vain do I know I am safe where
I am; I am never so sure of it as when I can actually see it, so that I
have always a cause for fear which did not exist in broad daylight. I
know, indeed, that a foreign body can scarcely act upon me without some
slight sound, and how intently I listen! At the least sound which I
cannot explain, the desire of self-preservation makes me picture
everything that would put me on my guard, and therefore everything most
calculated to alarm me.

I am just as uneasy if I hear no sound, for I might be taken unawares
without a sound. I must picture things as they were before, as they
ought to be; I must see what I do not see. Thus driven to exercise my
imagination, it soon becomes my master, and what I did to reassure
myself only alarms me more. I hear a noise, it is a robber; I hear
nothing, it is a ghost. The watchfulness inspired by the instinct of
self-preservation only makes me more afraid. Everything that ought to
reassure me exists only for my reason, and the voice of instinct is
louder than that of reason. What is the good of thinking there is
nothing to be afraid of, since in that case there is nothing we can do?

The cause indicates the cure. In everything habit overpowers
imagination; it is only aroused by what is new. It is no longer
imagination, but memory which is concerned with what we see every day,
and that is the reason of the maxim, ``Ab assuetis non fit passio,'' for
it is only at the flame of imagination that the passions are kindled.
Therefore do not argue with any one whom you want to cure of the fear of
darkness; take him often into dark places and be assured this practice
will be of more avail than all the arguments of philosophy. The tiler on
the roof does not know what it is to be dizzy, and those who are used to
the dark will not be afraid.

There is another advantage to be gained from our games in the dark. But
if these games are to be a success I cannot speak too strongly of the
need for gaiety. Nothing is so gloomy as the dark: do not shut your
child up in a dungeon, let him laugh when he goes, into a dark place,
let him laugh when he comes out, so that the thought of the game he is
leaving and the games he will play next may protect him from the
fantastic imagination which might lay hold on him.

There comes a stage in life beyond which we progress backwards. I feel I
have reached this stage. I am, so to speak, returning to a past career.
The approach of age makes us recall the happy days of our childhood. As
I grow old I become a child again, and I recall more readily what I did
at ten than at thirty. Reader, forgive me if I sometimes draw my
examples from my own experience. If this book is to be well written, I
must enjoy writing it.

I was living in the country with a pastor called M. Lambercier. My
companion was a cousin richer than myself, who was regarded as the heir
to some property, while I, far from my father, was but a poor orphan. My
big cousin Bernard was unusually timid, especially at night. I laughed
at his fears, till M. Lambercier was tired of my boasting, and
determined to put my courage to the proof. One autumn evening, when it
was very dark, he gave me the church key, and told me to go and fetch a
Bible he had left in the pulpit. To put me on my mettle he said
something which made it impossible for me to refuse.

I set out without a light; if I had had one, it would perhaps have been
even worse. I had to pass through the graveyard; I crossed it bravely,
for as long as I was in the open air I was never afraid of the dark.

As I opened the door I heard a sort of echo in the roof; it sounded like
voices and it began to shake my Roman courage. Having opened the door I
tried to enter, but when I had gone a few steps I stopped. At the sight
of the profound darkness in which the vast building lay I was seized
with terror and my hair stood on end. I turned, I went out through the
door, and took to my heels. In the yard I found a little dog, called
Sultan, whose caresses reassured me. Ashamed of my fears, I retraced my
steps, trying to take Sultan with me, but he refused to follow.
Hurriedly I opened the door and entered the church. I was hardly inside
when terror again got hold of me and so firmly that I lost my head, and
though the pulpit was on the right, as I very well knew, I sought it on
the left, and entangling myself among the benches I was completely lost.
Unable to find either pulpit or door, I fell into an indescribable state
of mind. At last I found the door and managed to get out of the church
and run away as I had done before, quite determined never to enter the
church again except in broad daylight.

I returned to the house; on the doorstep I heard M. Lambercier laughing,
laughing, as I supposed, at me. Ashamed to face his laughter, I was
hesitating to open the door, when I heard Miss Lambercier, who was
anxious about me, tell the maid to get the lantern, and M. Lambercier
got ready to come and look for me, escorted by my gallant cousin, who
would have got all the credit for the expedition. All at once my fears
departed, and left me merely surprised at my terror. I ran, I fairly
flew, to the church; without losing my way, without groping about, I
reached the pulpit, took the Bible, and ran down the steps. In three
strides I was out of the church, leaving the door open. Breathless, I
entered the room and threw the Bible on the table, frightened indeed,
but throbbing with pride that I had done it without the proposed
assistance.

You will ask if I am giving this anecdote as an example, and as an
illustration, of the mirth which I say should accompany these games. Not
so, but I give it as a proof that there is nothing so well calculated to
reassure any one who is afraid in the dark as to hear sounds of laughter
and talking in an adjoining room. Instead of playing alone with your
pupil in the evening, I would have you get together a number of merry
children; do not send them alone to begin with, but several together,
and do not venture to send any one quite alone, until you are quite
certain beforehand that he will not be too frightened.

I can picture nothing more amusing and more profitable than such games,
considering how little skill is required to organise them. In a large
room I should arrange a sort of labyrinth of tables, armchairs, chairs,
and screens. In the inextricable windings of this labyrinth I should
place some eight or ten sham boxes, and one real box almost exactly like
them, but well filled with sweets. I should describe clearly and briefly
the place where the right box would be found. I should give instructions
sufficient to enable people more attentive and less excitable than
children to find it. {[}Footnote: To practise them in attention, only
tell them things which it is clearly to their present interest that they
should understand thoroughly; above all be brief, never say a word more
than necessary. But neither let your speech be obscure nor of doubtful
meaning.{]} Then having made the little competitors draw lots, I should
send first one and then another till the right box was found. I should
increase the difficulty of the task in proportion to their skill.

Picture to yourself a youthful Hercules returning, box in hand, quite
proud of his expedition. The box is placed on the table and opened with
great ceremony. I can hear the bursts of laughter and the shouts of the
merry party when, instead of the looked-for sweets, he finds, neatly
arranged on moss or cotton-wool, a beetle, a snail, a bit of coal, a few
acorns, a turnip, or some such thing. Another time in a newly
whitewashed room, a toy or some small article of furniture would be hung
on the wall and the children would have to fetch it without touching the
wall. When the child who fetches it comes back, if he has failed ever so
little to fulfil the conditions, a dab of white on the brim of his cap,
the tip of his shoe, the flap of his coat or his sleeve, will betray his
lack of skill.

This is enough, or more than enough, to show the spirit of these games.
Do not read my book if you expect me to tell you everything.

What great advantages would be possessed by a man so educated, when
compared with others. His feet are accustomed to tread firmly in the
dark, and his hands to touch lightly; they will guide him safely in the
thickest darkness. His imagination is busy with the evening games of his
childhood, and will find it difficult to turn towards objects of alarm.
If he thinks he hears laughter, it will be the laughter of his former
playfellows, not of frenzied spirits; if he thinks there is a host of
people, it will not be the witches' sabbath, but the party in his
tutor's study. Night only recalls these cheerful memories, and it will
never alarm him; it will inspire delight rather than fear. He will be
ready for a military expedition at any hour, with or without his troop.
He will enter the camp of Saul, he will find his way, he will reach the
king's tent without waking any one, and he will return unobserved. Are
the steeds of Rhesus to be stolen, you may trust him. You will scarcely
find a Ulysses among men educated in any other fashion.

I have known people who tried to train the children not to fear the dark
by startling them. This is a very bad plan; its effects are just the
opposite of those desired, and it only makes children more timid.
Neither reason nor habit can secure us from the fear of a present danger
whose degree and kind are unknown, nor from the fear of surprises which
we have often experienced. Yet how will you make sure that you can
preserve your pupil from such accidents? I consider this the best advice
to give him beforehand. I should say to Emile, ``This is a matter of
self-defence, for the aggressor does not let you know whether he means
to hurt or frighten you, and as the advantage is on his side you cannot
even take refuge in flight. Therefore seize boldly anything, whether man
or beast, which takes you unawares in the dark. Grasp it, squeeze it
with all your might; if it struggles, strike, and do not spare your
blows; and whatever he may say or do, do not let him go till you know
just who he is. The event will probably prove that you had little to be
afraid of, but this way of treating practical jokers would naturally
prevent their trying it again.''

Although touch is the sense oftenest used, its discrimination remains,
as I have already pointed out, coarser and more imperfect than that of
any other sense, because we always use sight along with it; the eye
perceives the thing first, and the mind almost always judges without the
hand. On the other hand, discrimination by touch is the surest just
because of its limitations; for extending only as far as our hands can
reach, it corrects the hasty judgments of the other senses, which pounce
upon objects scarcely perceived, while what we learn by touch is learnt
thoroughly. Moreover, touch, when required, unites the force of our
muscles to the action of the nerves; we associate by simultaneous
sensations our ideas of temperature, size, and shape, to those of weight
and density. Thus touch is the sense which best teaches us the action of
foreign bodies upon ourselves, the sense which most directly supplies us
with the knowledge required for self-preservation.

As the trained touch takes the place of sight, why should it not, to
some extent, take the place of hearing, since sounds set up, in sonorous
bodies, vibrations perceptible by touch? By placing the hand on the body
of a 'cello one can distinguish without the use of eye or ear, merely by
the way in which the wood vibrates and trembles, whether the sound given
out is sharp or flat, whether it is drawn from the treble string or the
bass. If our touch were trained to note these differences, no doubt we
might in time become so sensitive as to hear a whole tune by means of
our fingers. But if we admit this, it is clear that one could easily
speak to the deaf by means of music; for tone and measure are no less
capable of regular combination than voice and articulation, so that they
might be used as the elements of speech.

There are exercises by which the sense of touch is blunted and deadened,
and others which sharpen it and make it delicate and discriminating. The
former, which employ much movement and force for the continued
impression of hard bodies, make the skin hard and thick, and deprive it
of its natural sensitiveness. The latter are those which give variety to
this feeling, by slight and repeated contact, so that the mind is
attentive to constantly recurring impressions, and readily learns to
discern their variations. This difference is clear in the use of musical
instruments. The harsh and painful touch of the 'cello, bass-viol, and
even of the violin, hardens the finger-tips, although it gives
flexibility to the fingers. The soft and smooth touch of the harpsichord
makes the fingers both flexible and sensitive. In this respect the
harpsichord is to be preferred.

The skin protects the rest of the body, so it is very important to
harden it to the effects of the air that it may be able to bear its
changes. With regard to this I may say I would not have the hand
roughened by too servile application to the same kind of work, nor
should the skin of the hand become hardened so as to lose its delicate
sense of touch which keeps the body informed of what is going on, and by
the kind of contact sometimes makes us shudder in different ways even in
the dark.

Why should my pupil be always compelled to wear the skin of an ox under
his foot? What harm would come of it if his own skin could serve him at
need as a sole. It is clear that a delicate skin could never be of any
use in this way, and may often do harm. The Genevese, aroused at
midnight by their enemies in the depth of winter, seized their guns
rather than their shoes. Who can tell whether the town would have
escaped capture if its citizens had not been able to go barefoot?

Let a man be always fore-armed against the unforeseen. Let Emile run
about barefoot all the year round, upstairs, downstairs, and in the
garden. Far from scolding him, I shall follow his example; only I shall
be careful to remove any broken glass. I shall soon proceed to speak of
work and manual occupations. Meanwhile, let him learn to perform every
exercise which encourages agility of body; let him learn to hold himself
easily and steadily in any position, let him practise jumping and
leaping, climbing trees and walls. Let him always find his balance, and
let his every movement and gesture be regulated by the laws of weight,
long before he learns to explain them by the science of statics. By the
way his foot is planted on the ground, and his body supported on his
leg, he ought to know if he is holding himself well or ill. An easy
carriage is always graceful, and the steadiest positions are the most
elegant. If I were a dancing master I would refuse to play the monkey
tricks of Marcel, which are only fit for the stage where they are
performed; but instead of keeping my pupil busy with fancy steps, I
would take him to the foot of a cliff. There I would show him how to
hold himself, how to carry his body and head, how to place first a foot
then a hand, to follow lightly the steep, toilsome, and rugged paths, to
leap from point to point, either up or down. He should emulate the
mountain-goat, not the ballet dancer.

As touch confines its operations to the man's immediate surroundings, so
sight extends its range beyond them; it is this which makes it
misleading; man sees half his horizon at a glance. In the midst of this
host of simultaneous impressions and the thoughts excited by them, how
can he fail now and then to make mistakes? Thus sight is the least
reliable of our senses, just because it has the widest range; it
functions long before our other senses, and its work is too hasty and on
too large a scale to be corrected by the rest. Moreover, the very
illusions of perspective are necessary if we are to arrive at a
knowledge of space and compare one part of space with another. Without
false appearances we should never see anything at a distance; without
the gradations of size and tone we could not judge of distance, or
rather distance would have no existence for us. If two trees, one of
which was a hundred paces from us and the other ten, looked equally
large and distinct, we should think they were side by side. If we
perceived the real dimensions of things, we should know nothing of
space; everything would seem close to our eyes.

The angle formed between any objects and our eye is the only means by
which our sight estimates their size and distance, and as this angle is
the simple effect of complex causes, the judgment we form does not
distinguish between the several causes; we are compelled to be
inaccurate. For how can I tell, by sight alone, whether the angle at
which an object appears to me smaller than another, indicates that it is
really smaller or that it is further off.

Here we must just reverse our former plan. Instead of simplifying the
sensation, always reinforce it and verify it by means of another sense.
Subject the eye to the hand, and, so to speak, restrain the
precipitation of the former sense by the slower and more reasoned pace
of the latter. For want of this sort of practice our sight measurements
are very imperfect. We cannot correctly, and at a glance, estimate
height, length, breadth, and distance; and the fact that engineers,
surveyors, architects, masons, and painters are generally quicker to see
and better able to estimate distances correctly, proves that the fault
is not in our eyes, but in our use of them. Their occupations give them
the training we lack, and they check the equivocal results of the angle
of vision by its accompanying experiences, which determine the relations
of the two causes of this angle for their eyes.

Children will always do anything that keeps them moving freely. There
are countless ways of rousing their interest in measuring, perceiving,
and estimating distance. There is a very tall cherry tree; how shall we
gather the cherries? Will the ladder in the barn be big enough? There is
a wide stream; how shall we get to the other side? Would one of the
wooden planks in the yard reach from bank to bank? From our windows we
want to fish in the moat; how many yards of line are required? I want to
make a swing between two trees; will two fathoms of cord be enough? They
tell me our room in the new house will be twenty-five feet square; do
you think it will be big enough for us? Will it be larger than this? We
are very hungry; here are two villages, which can we get to first for
our dinner?

An idle, lazy child was to be taught to run. He had no liking for this
or any other exercise, though he was intended for the army. Somehow or
other he had got it into his head that a man of his rank need know
nothing and do nothing---that his birth would serve as a substitute for
arms and legs, as well as for every kind of virtue. The skill of Chiron
himself would have failed to make a fleet-footed Achilles of this young
gentleman. The difficulty was increased by my determination to give him
no kind of orders. I had renounced all right to direct him by preaching,
promises, threats, emulation, or the desire to show off. How should I
make him want to run without saying anything? I might run myself, but he
might not follow my example, and this plan had other drawbacks.
Moreover, I must find some means of teaching him through this exercise,
so as to train mind and body to work together. This is how I, or rather
how the teacher who supplied me with this illustration, set about it.

When I took him a walk of an afternoon I sometimes put in my pocket a
couple of cakes, of a kind he was very fond of; we each ate one while we
were out, and we came back well pleased with our outing. One day he
noticed I had three cakes; he could have easily eaten six, so he ate his
cake quickly and asked for the other. ``No,'' said I, ``I could eat it
myself, or we might divide it, but I would rather see those two little
boys run a race for it.'' I called them to us, showed them the cake, and
suggested that they should race for it. They were delighted. The cake
was placed on a large stone which was to be the goal; the course was
marked out, we sat down, and at a given signal off flew the children!
The victor seized the cake and ate it without pity in the sight of the
spectators and of his defeated rival.

The sport was better than the cake; but the lesson did not take effect
all at once, and produced no result. I was not discouraged, nor did I
hurry; teaching is a trade at which one must be able to lose time and
save it. Our walks were continued, sometimes we took three cakes,
sometimes four, and from time to time there were one or two cakes for
the racers. If the prize was not great, neither was the ambition of the
competitors. The winner was praised and petted, and everything was done
with much ceremony. To give room to run and to add interest to the race
I marked out a longer course and admitted several fresh competitors.
Scarcely had they entered the lists than all the passers-by stopped to
watch. They were encouraged by shouting, cheering, and clapping. I
sometimes saw my little man trembling with excitement, jumping up and
shouting when one was about to reach or overtake another---to him these
were the Olympian games.

However, the competitors did not always play fair, they got in each
other's way, or knocked one another down, or put stones on the track.
That led us to separate them and make them start from different places
at equal distances from the goal. You will soon see the reason for this,
for I must describe this important affair at length.

Tired of seeing his favourite cakes devoured before his eyes, the young
lord began to suspect that there was some use in being a quick runner,
and seeing that he had two legs of his own, he began to practise running
on the quiet. I took care to see nothing, but I knew my stratagem had
taken effect. When he thought he was good enough (and I thought so too),
he pretended to tease me to give him the other cake. I refused; he
persisted, and at last he said angrily, ``Well, put it on the stone and
mark out the course, and we shall see.'' ``Very good,'' said I,
laughing, ``You will get a good appetite, but you will not get the
cake.'' Stung by my mockery, he took heart, won the prize, all the more
easily because I had marked out a very short course and taken care that
the best runner was out of the way. It will be evident that, after the
first step, I had no difficulty in keeping him in training. Soon he took
such a fancy for this form of exercise that without any favour he was
almost certain to beat the little peasant boys at running, however long
the course.

The advantage thus obtained led unexpectedly to another. So long as he
seldom won the prize, he ate it himself like his rivals, but as he got
used to victory he grew generous, and often shared it with the defeated.
That taught me a lesson in morals and I saw what was the real root of
generosity.

While I continued to mark out a different starting place for each
competitor, he did not notice that I had made the distances unequal, so
that one of them, having farther to run to reach the goal, was clearly
at a disadvantage. But though I left the choice to my pupil he did not
know how to take advantage of it. Without thinking of the distance, he
always chose the smoothest path, so that I could easily predict his
choice, and could almost make him win or lose the cake at my pleasure. I
had more than one end in view in this stratagem; but as my plan was to
get him to notice the difference himself, I tried to make him aware of
it. Though he was generally lazy and easy going, he was so eager in his
sports and trusted me so completely that I had great difficulty in
making him see that I was cheating him. When at last I managed to make
him see it in spite of his excitement, he was angry with me. ``What have
you to complain of?'' said I. ``In a gift which I propose to give of my
own free will am not I master of the conditions? Who makes you run? Did
I promise to make the courses equal? Is not the choice yours? Do not you
see that I am favouring you, and that the inequality you complain of is
all to your advantage, if you knew how to use it?'' That was plain to
him; and to choose he must observe more carefully. At first he wanted to
count the paces, but a child measures paces slowly and inaccurately;
moreover, I decided to have several races on one day; and the game
having become a sort of passion with the child, he was sorry to waste in
measuring the portion of time intended for running. Such delays are not
in accordance with a child's impatience; he tried therefore to see
better and to reckon the distance more accurately at sight. It was now
quite easy to extend and develop this power. At length, after some
months' practice, and the correction of his errors, I so trained his
power of judging at sight that I had only to place an imaginary cake on
any distant object and his glance was nearly as accurate as the
surveyor's chain.

Of all the senses, sight is that which we can least distinguish from the
judgments of the mind; as it takes a long time to learn to see. It takes
a long time to compare sight and touch, and to train the former sense to
give a true report of shape and distance. Without touch, without
progressive motion, the sharpest eyes in the world could give us no idea
of space. To the oyster the whole world must seem a point, and it would
seem nothing more to it even if it had a human mind. It is only by
walking, feeling, counting, measuring the dimensions of things, that we
learn to judge them rightly; but, on the other hand, if we were always
measuring, our senses would trust to the instrument and would never gain
confidence. Nor must the child pass abruptly from measurement to
judgment; he must continue to compare the parts when he could not
compare the whole; he must substitute his estimated aliquot parts for
exact aliquot parts, and instead of always applying the measure by hand
he must get used to applying it by eye alone. I would, however, have his
first estimates tested by measurement, so that he may correct his
errors, and if there is a false impression left upon the senses he may
correct it by a better judgment. The same natural standards of
measurement are in use almost everywhere, the man's foot, the extent of
his outstretched arms, his height. When the child wants to measure the
height of a room, his tutor may serve as a measuring rod; if he is
estimating the height of a steeple let him measure it by the house; if
he wants to know how many leagues of road there are, let him count the
hours spent in walking along it. Above all, do not do this for him; let
him do it himself.

One cannot learn to estimate the extent and size of bodies without at
the same time learning to know and even to copy their shape; for at
bottom this copying depends entirely on the laws of perspective, and one
cannot estimate distance without some feeling for these laws. All
children in the course of their endless imitation try to draw; and I
would have Emile cultivate this art; not so much for art's sake, as to
give him exactness of eye and flexibility of hand. Generally speaking,
it matters little whether he is acquainted with this or that occupation,
provided he gains clearness of sense---perception and the good bodily
habits which belong to the exercise in question. So I shall take good
care not to provide him with a drawing master, who would only set him to
copy copies and draw from drawings. Nature should be his only teacher,
and things his only models. He should have the real thing before his
eyes, not its copy on paper. Let him draw a house from a house, a tree
from a tree, a man from a man; so that he may train himself to observe
objects and their appearance accurately and not to take false and
conventional copies for truth. I would even train him to draw only from
objects actually before him and not from memory, so that, by repeated
observation, their exact form may be impressed on his imagination, for
fear lest he should substitute absurd and fantastic forms for the real
truth of things, and lose his sense of proportion and his taste for the
beauties of nature.

Of course I know that in this way he will make any number of daubs
before he produces anything recognisable, that it will be long before he
attains to the graceful outline and light touch of the draughtsman;
perhaps he will never have an eye for picturesque effect or a good taste
in drawing. On the other hand, he will certainly get a truer eye, a
surer hand, a knowledge of the real relations of form and size between
animals, plants, and natural objects, together with a quicker sense of
the effects of perspective. That is just what I wanted, and my purpose
is rather that he should know things than copy them. I would rather he
showed me a plant of acanthus even if he drew a capital with less
accuracy.

Moreover, in this occupation as in others, I do not intend my pupil to
play by himself; I mean to make it pleasanter for him by always sharing
it with him. He shall have no other rival; but mine will be a continual
rivalry, and there will be no risk attaching to it; it will give
interest to his pursuits without awaking jealousy between us. I shall
follow his example and take up a pencil; at first I shall use it as
unskilfully as he. I should be an Apelles if I did not set myself
daubing. To begin with, I shall draw a man such as lads draw on walls, a
line for each arm, another for each leg, with the fingers longer than
the arm. Long after, one or other of us will notice this lack of
proportion; we shall observe that the leg is thick, that this thickness
varies, that the length of the arm is proportionate to the body. In this
improvement I shall either go side by side with my pupil, or so little
in advance that he will always overtake me easily and sometimes get
ahead of me. We shall get brushes and paints, we shall try to copy the
colours of things and their whole appearance, not merely their shape. We
shall colour prints, we shall paint, we shall daub; but in all our
daubing we shall be searching out the secrets of nature, and whatever we
do shall be done under the eye of that master.

We badly needed ornaments for our room, and now we have them ready to
our hand. I will have our drawings framed and covered with good glass,
so that no one will touch them, and thus seeing them where we put them,
each of us has a motive for taking care of his own. I arrange them in
order round the room, each drawing repeated some twenty or thirty times,
thus showing the author's progress in each specimen, from the time when
the house is merely a rude square, till its front view, its side view,
its proportions, its light and shade are all exactly portrayed. These
graduations will certainly furnish us with pictures, a source of
interest to ourselves and of curiosity to others, which will spur us on
to further emulation. The first and roughest drawings I put in very
smart gilt frames to show them off; but as the copy becomes more
accurate and the drawing really good, I only give it a very plain dark
frame; it needs no other ornament than itself, and it would be a pity if
the frame distracted the attention which the picture itself deserves.
Thus we each aspire to a plain frame, and when we desire to pour scorn
on each other's drawings, we condemn them to a gilded frame. Some day
perhaps ``the gilt frame'' will become a proverb among us, and we shall
be surprised to find how many people show what they are really made of
by demanding a gilt frame.

I have said already that geometry is beyond the child's reach; but that
is our own fault. We fail to perceive that their method is not ours,
that what is for us the art of reasoning, should be for them the art of
seeing. Instead of teaching them our way, we should do better to adopt
theirs, for our way of learning geometry is quite as much a matter of
imagination as of reasoning. When a proposition is enunciated you must
imagine the proof; that is, you must discover on what proposition
already learnt it depends, and of all the possible deductions from that
proposition you must choose just the one required.

In this way the closest reasoner, if he is not inventive, may find
himself at a loss. What is the result? Instead of making us discover
proofs, they are dictated to us; instead of teaching us to reason, our
memory only is employed.

Draw accurate figures, combine them together, put them one upon another,
examine their relations, and you will discover the whole of elementary
geometry in passing from one observation to another, without a word of
definitions, problems, or any other form of demonstration but
super-position. I do not profess to teach Emile geometry; he will teach
me; I shall seek for relations, he will find them, for I shall seek in
such a fashion as to make him find. For instance, instead of using a
pair of compasses to draw a circle, I shall draw it with a pencil at the
end of bit of string attached to a pivot. After that, when I want to
compare the radii one with another, Emile will laugh at me and show me
that the same thread at full stretch cannot have given distances of
unequal length. If I wish to measure an angle of 60 degrees I describe
from the apex of the angle, not an arc, but a complete circle, for with
children nothing must be taken for granted. I find that the part of the
circle contained between the two lines of the angle is the sixth part of
a circle. Then I describe another and larger circle from the same
centre, and I find the second arc is again the sixth part of its circle.
I describe a third concentric circle with a similar result, and I
continue with more and more circles till Emile, shocked at my stupidity,
shows me that every arc, large or small, contained by the same angle
will always be the sixth part of its circle. Now we are ready to use the
protractor.

To prove that two adjacent angles are equal to two right angles people
describe a circle. On the contrary I would have Emile observe the fact
in a circle, and then I should say, ``If we took away the circle and
left the straight lines, would the angles have changed their size,
etc.?''

Exactness in the construction of figures is neglected; it is taken for
granted and stress is laid on the proof. With us, on the other hand,
there will be no question of proof. Our chief business will be to draw
very straight, accurate, and even lines, a perfect square, a really
round circle. To verify the exactness of a figure we will test it by
each of its sensible properties, and that will give us a chance to
discover fresh properties day by day. We will fold the two semi-circles
along the diameter, the two halves of the square by the diagonal; he
will compare our two figures to see who has got the edges to fit moat
exactly, i.e., who has done it best; we should argue whether this equal
division would always be possible in parallelograms, trapezes, etc. We
shall sometimes try to forecast the result of an experiment, to find
reasons, etc.

Geometry means to my scholar the successful use of the rule and compass;
he must not confuse it with drawing, in which these instruments are not
used. The rule and compass will be locked up, so that he will not get
into the way of messing about with them, but we may sometimes take our
figures with us when we go for a walk, and talk over what we have done,
or what we mean to do.

I shall never forget seeing a young man at Turin, who had learnt as a
child the relations of contours and surfaces by having to choose every
day isoperimetric cakes among cakes of every geometrical figure. The
greedy little fellow had exhausted the art of Archimedes to find which
were the biggest.

When the child flies a kite he is training eye and hand to accuracy;
when he whips a top, he is increasing his strength by using it, but
without learning anything. I have sometimes asked why children are not
given the same games of skill as men; tennis, mall, billiards, archery,
football, and musical instruments. I was told that some of these are
beyond their strength, that the child's senses are not sufficiently
developed for others. These do not strike me as valid reasons; a child
is not as tall as a man, but he wears the same sort of coat; I do not
want him to play with our cues at a billiard-table three feet high; I do
not want him knocking about among our games, nor carrying one of our
racquets in his little hand; but let him play in a room whose windows
have been protected; at first let him only use soft balls, let his first
racquets be of wood, then of parchment, and lastly of gut, according to
his progress. You prefer the kite because it is less tiring and there is
no danger. You are doubly wrong. Kite-flying is a sport for women, but
every woman will run away from a swift ball. Their white skins were not
meant to be hardened by blows and their faces were not made for bruises.
But we men are made for strength; do you think we can attain it without
hardship, and what defence shall we be able to make if we are attacked?
People always play carelessly in games where there is no danger. A
falling kite hurts nobody, but nothing makes the arm so supple as
protecting the head, nothing makes the sight so accurate as having to
guard the eye. To dash from one end of the room to another, to judge the
rebound of a ball before it touches the ground, to return it with
strength and accuracy, such games are not so much sports fit for a man,
as sports fit to make a man of him.

The child's limbs, you say, are too tender. They are not so strong as
those of a man, but they are more supple. His arm is weak, still it is
an arm, and it should be used with due consideration as we use other
tools. Children have no skill in the use of their hands. That is just
why I want them to acquire skill; a man with as little practice would be
just as clumsy. We can only learn the use of our limbs by using them. It
is only by long experience that we learn to make the best of ourselves,
and this experience is the real object of study to which we cannot apply
ourselves too early.

What is done can be done. Now there is nothing commoner than to find
nimble and skilful children whose limbs are as active as those of a man.
They may be seen at any fair, swinging, walking on their hands, jumping,
dancing on the tight rope. For many years past, troops of children have
attracted spectators to the ballets at the Italian Comedy House. Who is
there in Germany and Italy who has not heard of the famous pantomime
company of Nicolini? Has it ever occurred to any one that the movements
of these children were less finished, their postures less graceful,
their ears less true, their dancing more clumsy than those of grown-up
dancers? If at first the fingers are thick, short, and awkward, the
dimpled hands unable to grasp anything, does this prevent many children
from learning to read and write at an age when others cannot even hold a
pen or pencil? All Paris still recalls the little English girl of ten
who did wonders on the harpsichord. I once saw a little fellow of eight,
the son of a magistrate, who was set like a statuette on the table among
the dishes, to play on a fiddle almost as big as himself, and even
artists were surprised at his execution.

To my mind, these and many more examples prove that the supposed
incapacity of children for our games is imaginary, and that if they are
unsuccessful in some of them, it is for want of practice.

You will tell me that with regard to the body I am falling into the same
mistake of precocious development which I found fault with for the mind.
The cases are very different: in the one, progress is apparent only; in
the other it is real. I have shown that children have not the mental
development they appear to have, while they really do what they seem to
do. Besides, we must never forget that all this should be play, the easy
and voluntary control of the movements which nature demands of them, the
art of varying their games to make them pleasanter, without the least
bit of constraint to transform them into work; for what games do they
play in which I cannot find material for instruction for them? And even
if I could not do so, so long as they are amusing themselves harmlessly
and passing the time pleasantly, their progress in learning is not yet
of such great importance. But if one must be teaching them this or that
at every opportunity, it cannot be done without constraint, vexation, or
tedium.

What I have said about the use of the two senses whose use is most
constant and most important, may serve as an example of how to train the
rest. Sight and touch are applied to bodies at rest and bodies in
motion, but as hearing is only affected by vibrations of the air, only a
body in motion can make a noise or sound; if everything were at rest we
should never hear. At night, when we ourselves only move as we choose,
we have nothing to fear but moving bodies; hence we need a quick ear,
and power to judge from the sensations experienced whether the body
which causes them is large or small, far off or near, whether its
movements are gentle or violent. When once the air is set in motion, it
is subject to repercussions which produce echoes, these renew the
sensations and make us hear a loud or penetrating sound in another
quarter. If you put your ear to the ground you may hear the sound of
men's voices or horses' feet in a plain or valley much further off than
when you stand upright.

As we have made a comparison between sight and touch, it will be as well
to do the same for hearing, and to find out which of the two impressions
starting simultaneously from a given body first reaches the sense-organ.
When you see the flash of a cannon, you have still time to take cover;
but when you hear the sound it is too late, the ball is close to you.
One can reckon the distance of a thunderstorm by the interval between
the lightning and the thunder. Let the child learn all these facts, let
him learn those that are within his reach by experiment, and discover
the rest by induction; but I would far rather he knew nothing at all
about them, than that you should tell him.

In the voice we have an organ answering to hearing; we have no such
organ answering to sight, and we do not repeat colours as we repeat
sounds. This supplies an additional means of cultivating the ear by
practising the active and passive organs one with the other.

Man has three kinds of voice, the speaking or articulate voice, the
singing or melodious voice, and the pathetic or expressive voice, which
serves as the language of the passions, and gives life to song and
speech. The child has these three voices, just as the man has them, but
he does not know how to use them in combination. Like us, he laughs,
cries, laments, shrieks, and groans, but he does not know how to combine
these inflexions with speech or song. These three voices find their best
expression in perfect music. Children are incapable of such music, and
their singing lacks feeling. In the same way their spoken language lacks
expression; they shout, but they do not speak with emphasis, and there
is as little power in their voice as there is emphasis in their speech.
Our pupil's speech will be plainer and simpler still, for his passions
are still asleep, and will not blend their tones with his. Do not,
therefore, set him to recite tragedy or comedy, nor try to teach
declamation so-called. He will have too much sense to give voice to
things he cannot understand, or expression to feelings he has never
known.

Teach him to speak plainly and distinctly, to articulate clearly, to
pronounce correctly and without affectation, to perceive and imitate the
right accent in prose and verse, and always to speak loud enough to be
heard, but without speaking too loud---a common fault with
school-children. Let there be no waste in anything.

The same method applies to singing; make his voice smooth and true,
flexible and full, his ear alive to time and tune, but nothing more.
Descriptive and theatrical music is not suitable at his age------I would
rather he sang no words; if he must have words, I would try to compose
songs on purpose for him, songs interesting to a child, and as simple as
his own thoughts.

You may perhaps suppose that as I am in no hurry to teach Emile to read
and write, I shall not want to teach him to read music. Let us spare his
brain the strain of excessive attention, and let us be in no hurry to
turn his mind towards conventional signs. I grant you there seems to be
a difficulty here, for if at first sight the knowledge of notes seems no
more necessary for singing than the knowledge of letters for speaking,
there is really this difference between them: When we speak, we are
expressing our own thoughts; when we sing we are expressing the thoughts
of others. Now in order to express them we must read them.

But at first we can listen to them instead of reading them, and a song
is better learnt by ear than by eye. Moreover, to learn music thoroughly
we must make songs as well as sing them, and the two processes must be
studied together, or we shall never have any real knowledge of music.
First give your young musician practice in very regular, well-cadenced
phrases; then let him connect these phrases with the very simplest
modulations; then show him their relation one to another by correct
accent, which can be done by a fit choice of cadences and rests. On no
account give him anything unusual, or anything that requires pathos or
expression. A simple, tuneful air, always based on the common chords of
the key, with its bass so clearly indicated that it is easily felt and
accompanied, for to train his voice and ear he should always sing with
the harpsichord.

We articulate the notes we sing the better to distinguish them; hence
the custom of sol-faing with certain syllables. To tell the keys one
from another they must have names and fixed intervals; hence the names
of the intervals, and also the letters of the alphabet attached to the
keys of the clavier and the notes of the scale. C and A indicate fixed
sounds, invariable and always rendered by the same keys; Ut and La are
different. Ut is always the dominant of a major scale, or the
leading-note of a minor scale. La is always the dominant of a minor
scale or the sixth of a major scale. Thus the letters indicate fixed
terms in our system of music, and the syllables indicate terms
homologous to the similar relations in different keys. The letters show
the keys on the piano, and the syllables the degrees in the scale.
French musicians have made a strange muddle of this. They have confused
the meaning of the syllables with that of the letters, and while they
have unnecessarily given us two sets of symbols for the keys of the
piano, they have left none for the chords of the scales; so that Ut and
C are always the same for them; this is not and ought not to be; if so,
what is the use of C? Their method of sol-faing is, therefore, extremely
and needlessly difficult, neither does it give any clear idea to the
mind; since, by this method, Ut and Me, for example, may mean either a
major third, a minor third, an augmented third, or a diminished third.
What a strange thing that the country which produces the finest books
about music should be the very country where it is hardest to learn
music!

Let us adopt a simpler and clearer plan with our pupil; let him have
only two scales whose relations remain unchanged, and indicated by the
same symbols. Whether he sings or plays, let him learn to fix his scale
on one of the twelve tones which may serve as a base, and whether he
modulates in D, C, or G, let the close be always Ut or La, according to
the scale. In this way he will understand what you mean, and the
essential relations for correct singing and playing will always be
present in his mind; his execution will be better and his progress
quicker. There is nothing funnier than what the French call ``natural
sol-faing;'' it consists in removing the real meaning of things and
putting in their place other meanings which only distract us. There is
nothing more natural than sol-faing by transposition, when the scale is
transposed. But I have said enough, and more than enough, about music;
teach it as you please, so long as it is nothing but play.

We are now thoroughly acquainted with the condition of foreign bodies in
relation to our own, their weight, form, colour, density, size,
distance, temperature, stability, or motion. We have learnt which of
them to approach or avoid, how to set about overcoming their resistance
or to resist them so as to prevent ourselves from injury; but this is
not enough. Our own body is constantly wasting and as constantly
requires to be renewed. Although we have the power of changing other
substances into our own, our choice is not a matter of indifference.
Everything is not food for man, and what may be food for him is not all
equally suitable; it depends on his racial constitution, the country he
lives in, his individual temperament, and the way of living which his
condition demands.

If we had to wait till experience taught us to know and choose fit food
for ourselves, we should die of hunger or poison; but a kindly
providence which has made pleasure the means of self-preservation to
sentient beings teaches us through our palate what is suitable for our
stomach. In a state of nature there is no better doctor than a man's own
appetite, and no doubt in a state of nature man could find the most
palateable food the most wholesome.

Nor is this all. Our Maker provides, not only for those needs he has
created, but for those we create for ourselves; and it is to keep the
balance between our wants and our needs that he has caused our tastes to
change and vary with our way of living. The further we are from a state
of nature, the more we lose our natural tastes; or rather, habit becomes
a second nature, and so completely replaces our real nature, that we
have lost all knowledge of it.

From this it follows that the most natural tastes should be the
simplest, for those are more easily changed; but when they are sharpened
and stimulated by our fancies they assume a form which is incapable of
modification. The man who so far has not adapted himself to one country
can learn the ways of any country whatsoever; but the man who has
adopted the habits of one particular country can never shake them off.

This seems to be true of all our senses, especially of taste. Our first
food is milk; we only become accustomed by degrees to strong flavours;
at first we dislike them. Fruit, vegetables, herbs, and then fried meat
without salt or seasoning, formed the feasts of primitive man. When the
savage tastes wine for the first time, he makes a grimace and spits it
out; and even among ourselves a man who has not tasted fermented liquors
before twenty cannot get used to them; we should all be sober if we did
not have wine when we were children. Indeed, the simpler our tastes are,
the more general they are; made dishes are those most frequently
disliked. Did you ever meet with any one who disliked bread or water?
Here is the finger of nature, this then is our rule. Preserve the
child's primitive tastes as long as possible; let his food be plain and
simple, let strong flavours be unknown to his palate, and do not let his
diet be too uniform.

I am not asking, for the present, whether this way of living is
healthier or no; that is not what I have in view. It is enough for me to
know that my choice is more in accordance with nature, and that it can
be more readily adapted to other conditions. In my opinion, those who
say children should be accustomed to the food they will have when they
are grown up are mistaken. Why should their food be the same when their
way of living is so different? A man worn out by labour, anxiety, and
pain needs tasty foods to give fresh vigour to his brain; a child fresh
from his games, a child whose body is growing, needs plentiful food
which will supply more chyle. Moreover the grown man has already a
settled profession, occupation, and home, but who can tell what Fate
holds in store for the child? Let us not give him so fixed a bent in any
direction that he cannot change it if required without hardship. Do not
bring him up so that he would die of hunger in a foreign land if he does
not take a French cook about with him; do not let him say at some future
time that France is the only country where the food is fit to eat. By
the way, that is a strange way of praising one's country. On the other
hand, I myself should say that the French are the only people who do not
know what good food is, since they require such a special art to make
their dishes eatable.

Of all our different senses, we are usually most affected by taste. Thus
it concerns us more nearly to judge aright of what will actually become
part of ourselves, than of that which will merely form part of our
environment. Many things are matters of indifference to touch, hearing,
and sight; but taste is affected by almost everything. Moreover the
activity of this sense is wholly physical and material; of all the
senses, it alone makes no appeal to the imagination, or at least,
imagination plays a smaller part in its sensations; while imitation and
imagination often bring morality into the impressions of the other
senses. Thus, speaking generally, soft and pleasure-loving minds,
passionate and truly sensitive dispositions, which are easily stirred by
the other senses, are usually indifferent to this. From this very fact,
which apparently places taste below our other senses and makes our
inclination towards it the more despicable, I draw just the opposite
conclusion---that the best way to lead children is by the mouth.
Greediness is a better motive than vanity; for the former is a natural
appetite directly dependent on the senses, while the latter is the
outcome of convention, it is the slave of human caprice and liable to
every kind of abuse. Believe me the child will cease to care about his
food only too soon, and when his heart is too busy, his palate will be
idle. When he is grown up greediness will be expelled by a host of
stronger passions, while vanity will only be stimulated by them; for
this latter passion feeds upon the rest till at length they are all
swallowed up in it. I have sometimes studied those men who pay great
attention to good eating, men whose first waking thought is---What shall
we have to eat to-day? men who describe their dinner with as much detail
as Polybius describes a combat. I have found these so-called men were
only children of forty, without strength or vigour---fruges consumere
nati. Gluttony is the vice of feeble minds. The gourmand has his brains
in his palate, he can do nothing but eat; he is so stupid and incapable
that the table is the only place for him, and dishes are the only things
he knows anything about. Let us leave him to this business without
regret; it is better for him and for us.

It is a small mind that fears lest greediness should take root in the
child who is fit for something better. The child thinks of nothing but
his food, the youth pays no heed to it at all; every kind of food is
good, and we have other things to attend to. Yet I would not have you
use the low motive unwisely. I would not have you trust to dainties
rather than to the honour which is the reward of a good deed. But
childhood is, or ought to be, a time of play and merry sports, and I do
not see why the rewards of purely bodily exercises should not be
material and sensible rewards. If a little lad in Majorca sees a basket
on the tree-top and brings it down with his sling, is it not fair that
he should get something by this, and a good breakfast should repair the
strength spent in getting it. If a young Spartan, facing the risk of a
hundred stripes, slips skilfully into the kitchen, and steals a live fox
cub, carries it off in his garment, and is scratched, bitten till the
blood comes, and for shame lest he should be caught the child allows his
bowels to be torn out without a movement or a cry, is it not fair that
he should keep his spoils, that he should eat his prey after it has
eaten him? A good meal should never be a reward; but why should it not
be sometimes the result of efforts made to get it. Emile does not
consider the cake I put on the stone as a reward for good running; he
knows that the only way to get the cake is to get there first.

This does not contradict my previous rules about simple food; for to
tempt a child's appetite you need not stimulate it, you need only
satisfy it; and the commonest things will do this if you do not attempt
to refine children's taste. Their perpetual hunger, the result of their
need for growth, will be the best sauce. Fruit, milk, a piece of cake
just a little better than ordinary bread, and above all the art of
dispensing these things prudently, by these means you may lead a host of
children to the world's end, without on the one hand giving them a taste
for strong flavours, nor on the other hand letting them get tired of
their food.

The indifference of children towards meat is one proof that the taste
for meat is unnatural; their preference is for vegetable foods, such as
milk, pastry, fruit, etc. Beware of changing this natural taste and
making children flesh-eaters, if not for their health's sake, for the
sake of their character; for how can one explain away the fact that
great meat-eaters are usually fiercer and more cruel than other men;
this has been recognised at all times and in all places. The English are
noted for their cruelty {[}Footnote: I am aware that the English make a
boast of their humanity and of the kindly disposition of their race,
which they call ``good-natured people;'' but in vain do they proclaim
this fact; no one else says it of them.{]} while the Gaures are the
gentlest of men. {[}Footnote: The Banians, who abstain from flesh even
more completely than the Gaures, are almost as gentle as the Gaures
themselves, but as their morality is less pure and their form of worship
less reasonable they are not such good men.{]} All savages are cruel,
and it is not their customs that tend in this direction; their cruelty
is the result of their food. They go to war as to the chase, and treat
men as they would treat bears. Indeed in England butchers are not
allowed to give evidence in a court of law, no more can surgeons.
{[}Footnote: One of the English translators of my book has pointed out
my mistake, and both of them have corrected it. Butchers and surgeons
are allowed to give evidence in the law courts, but butchers may not
serve on juries in criminal cases, though surgeons are allowed to do
so.{]} Great criminals prepare themselves for murder by drinking blood.
Homer makes his flesh-eating Cyclops a terrible man, while his
Lotus-eaters are so delightful that those who went to trade with them
forgot even their own country to dwell among them.

``You ask me,'' said Plutarch, ``why Pythagoras abstained from eating the
flesh of beasts, but I ask you, what courage must have been needed by
the first man who raised to his lips the flesh of the slain, who broke
with his teeth the bones of a dying beast, who had dead bodies, corpses,
placed before him and swallowed down limbs which a few moments ago were
bleating, bellowing, walking, and seeing? How could his hand plunge the
knife into the heart of a sentient creature, how could his eyes look on
murder, how could he behold a poor helpless animal bled to death,
scorched, and dismembered? how can he bear the sight of this quivering
flesh? does not the very smell of it turn his stomach? is he not
repelled, disgusted, horror-struck, when he has to handle the blood from
these wounds, and to cleanse his fingers from the dark and viscous
bloodstains?
\begin{displayquote}
     ``The scorched skins wriggled upon the ground,
     The shrinking flesh bellowed upon the spit.
     Man cannot eat them without a shudder;
     He seems to hear their cries within his breast.
\end{displayquote}
``Thus must he have felt the first time he did despite to nature and made
this horrible meal; the first time he hungered for the living creature,
and desired to feed upon the beast which was still grazing; when he bade
them slay, dismember, and cut up the sheep which licked his hands. It is
those who began these cruel feasts, not those who abandon them, who
should cause surprise, and there were excuses for those primitive men,
excuses which we have not, and the absence of such excuses multiplies
our barbarity a hundredfold.

```Mortals, beloved of the gods,' says this primitive man, 'compare our
times with yours; see how happy you are, and how wretched were we. The
earth, newly formed, the air heavy with moisture, were not yet subjected
to the rule of the seasons. Three-fourths of the surface of the globe
was flooded by the ever-shifting channels of rivers uncertain of their
course, and covered with pools, lakes, and bottomless morasses. The
remaining quarter was covered with woods and barren forests. The earth
yielded no good fruit, we had no instruments of tillage, we did not even
know the use of them, and the time of harvest never came for those who
had sown nothing. Thus hunger was always in our midst. In winter, mosses
and the bark of trees were our common food. A few green roots of
dogs-bit or heather were a feast, and when men found beech-mast, nuts,
or acorns, they danced for joy round the beech or oak, to the sound of
some rude song, while they called the earth their mother and their
nurse. This was their only festival, their only sport; all the rest of
man's life was spent in sorrow, pain, and hunger.

``'At length, when the bare and naked earth no longer offered us any
food, we were compelled in self-defence to outrage nature, and to feed
upon our companions in distress, rather than perish with them. But you,
oh, cruel men! who forces you to shed blood? Behold the wealth of good
things about you, the fruits yielded by the earth, the wealth of field
and vineyard; the animals give their milk for your drink and their
fleece for your clothing. What more do you ask? What madness compels you
to commit such murders, when you have already more than you can eat or
drink? Why do you slander our mother earth, and accuse her of denying
you food? Why do you sin against Ceres, the inventor of the sacred laws,
and against the gracious Bacchus, the comforter of man, as if their
lavish gifts were not enough to preserve mankind? Have you the heart to
mingle their sweet fruits with the bones upon your table, to eat with
the milk the blood of the beasts which gave it? The lions and panthers,
wild beasts as you call them, are driven to follow their natural
instinct, and they kill other beasts that they may live. But, a
hundredfold fiercer than they, you fight against your instincts without
cause, and abandon yourselves to the most cruel pleasures. The animals
you eat are not those who devour others; you do not eat the carnivorous
beasts, you take them as your pattern. You only hunger for the sweet and
gentle creatures which harm no one, which follow you, serve you, and are
devoured by you as the reward of their service.

``\,`O unnatural murderer! if you persist in the assertion that nature
has made you to devour your fellow-creatures, beings of flesh and blood,
living and feeling like yourself, stifle if you can that horror with
which nature makes you regard these horrible feasts; slay the animals
yourself, slay them, I say, with your own hands, without knife or
mallet; tear them with your nails like the lion and the bear, take this
ox and rend him in pieces, plunge your claws into his hide; eat this
lamb while it is yet alive, devour its warm flesh, drink its soul with
its blood. You shudder! you dare not feel the living throbbing flesh
between your teeth? Ruthless man; you begin by slaying the animal and
then you devour it, as if to slay it twice. It is not enough. You turn
against the dead flesh, it revolts you, it must be transformed by fire,
boiled and roasted, seasoned and disguised with drugs; you must have
butchers, cooks, turnspits, men who will rid the murder of its horrors,
who will dress the dead bodies so that the taste deceived by these
disguises will not reject what is strange to it, and will feast on
corpses, the very sight of which would sicken you.'\,''

Although this quotation is irrelevant, I cannot resist the temptation to
transcribe it, and I think few of my readers will resent it.

In conclusion, whatever food you give your children, provided you
accustom them to nothing but plain and simple dishes, let them eat and
run and play as much as they want; you may be sure they will never eat
too much and will never have indigestion; but if you keep them hungry
half their time, when they do contrive to evade your vigilance, they
will take advantage of it as far as they can; they will eat till they
are sick, they will gorge themselves till they can eat no more. Our
appetite is only excessive because we try to impose on it rules other
than those of nature, opposing, controlling, prescribing, adding, or
substracting; the scales are always in our hands, but the scales are the
measure of our caprices not of our stomachs. I return to my usual
illustration; among peasants the cupboard and the apple-loft are always
left open, and indigestion is unknown alike to children and grown-up
people.

If, however, it happened that a child were too great an eater, though,
under my system, I think it is impossible, he is so easily distracted by
his favourite games that one might easily starve him without his knowing
it. How is it that teachers have failed to use such a safe and easy
weapon. Herodotus records that the Lydians, {[}Footnote: The ancient
historians are full of opinions which may be useful, even if the facts
which they present are false. But we do not know how to make any real
use of history. Criticism and erudition are our only care; as if it
mattered more that a statement were true or false than that we should be
able to get a useful lesson from it. A wise man should consider history
a tissue of fables whose morals are well adapted to the human heart.{]}
under the pressure of great scarcity, decided to invent sports and other
amusements with which to cheat their hunger, and they passed whole days
without thought of food. Your learned teachers may have read this
passage time after time without seeing how it might be applied to
children. One of these teachers will probably tell me that a child does
not like to leave his dinner for his lessons. You are right, sir---I was
not thinking of that sort of sport.

The sense of smell is to taste what sight is to touch; it goes before it
and gives it warning that it will be affected by this or that substance;
and it inclines it to seek or shun this experience according to the
impressions received beforehand. I have been told that savages receive
impressions quite different from ours, and that they have quite
different ideas with regard to pleasant or unpleasant odours. I can well
believe it. Odours alone are slight sensations; they affect the
imagination rather than the senses, and they work mainly through the
anticipations they arouse. This being so, and the tastes of savages
being so unlike the taste of civilised men, they should lead them to
form very different ideas with regard to flavours and therefore with
regard to the odours which announce them. A Tartar must enjoy the smell
of a haunch of putrid horseflesh, much as a sportsman enjoys a very high
partridge. Our idle sensations, such as the scents wafted from the
flower beds, must pass unnoticed among men who walk too much to care for
strolling in a garden, and do not work enough to find pleasure in
repose. Hungry men would find little pleasure in scents which did not
proclaim the approach of food.

Smell is the sense of the imagination; as it gives tone to the nerves it
must have a great effect on the brain; that is why it revives us for the
time, but eventually causes exhaustion. Its effects on love are pretty
generally recognised. The sweet perfumes of a dressing-room are not so
slight a snare as you may fancy them, and I hardly know whether to
congratulate or condole with that wise and somewhat insensible person
whose senses are never stirred by the scent of the flowers his mistress
wears in her bosom.

Hence the sense of smell should not be over-active in early childhood;
the imagination, as yet unstirred by changing passions, is scarcely
susceptible of emotion, and we have not enough experience to discern
beforehand from one sense the promise of another. This view is confirmed
by observation, and it is certain that the sense of smell is dull and
almost blunted in most children. Not that their sensations are less
acute than those of grown-up people, but that there is no idea
associated with them; they do not easily experience pleasure or pain,
and are not flattered or hurt as we are. Without going beyond my system,
and without recourse to comparative anatomy, I think we can easily see
why women are generally fonder of perfumes than men.

It is said that from early childhood the Redskins of Canada, train their
sense of smell to such a degree of subtlety that, although they have
dogs, they do not condescend to use them in hunting---they are their own
dogs. Indeed I believe that if children were trained to scent their
dinner as a dog scents game, their sense of smell might be nearly as
perfect; but I see no very real advantage to be derived from this sense,
except by teaching the child to observe the relation between smell and
taste. Nature has taken care to compel us to learn these relations. She
has made the exercise of the latter sense practically inseparable from
that of the former, by placing their organs close together, and by
providing, in the mouth, a direct pathway between them, so that we taste
nothing without smelling it too. Only I would not have these natural
relations disturbed in order to deceive the child, e.g.; to conceal the
taste of medicine with an aromatic odour, for the discord between the
senses is too great for deception, the more active sense overpowers the
other, the medicine is just as distasteful, and this disagreeable
association extends to every sensation experienced at the time; so the
slightest of these sensations recalls the rest to his imagination and a
very pleasant perfume is for him only a nasty smell; thus our foolish
precautions increase the sum total of his unpleasant sensations at the
cost of his pleasant sensations.

In the following books I have still to speak of the training of a sort
of sixth sense, called common-sense, not so much because it is common to
all men, but because it results from the well-regulated use of the other
five, and teaches the nature of things by the sum-total of their
external aspects. So this sixth sense has no special organ, it has its
seat in the brain, and its sensations which are purely internal are
called percepts or ideas. The number of these ideas is the measure of
our knowledge; exactness of thought depends on their clearness and
precision; the art of comparing them one with another is called human
reason. Thus what I call the reasoning of the senses, or the reasoning
of the child, consists in the formation of simple ideas through the
associated experience of several sensations; what I call the reasoning
of the intellect, consists in the formation, of complex ideas through
the association of several simple ideas.

If my method is indeed that of nature, and if I am not mistaken in the
application of that method, we have led our pupil through the region of
sensation to the bounds of the child's reasoning; the first step we take
beyond these bounds must be the step of a man. But before we make this
fresh advance, let us glance back for a moment at the path we have
hitherto followed. Every age, every station in life, has a perfection, a
ripeness, of its own. We have often heard the phrase ``a grown man;''
but we will consider ``a grown child.'' This will be a new experience
and none the less pleasing.

The life of finite creatures is so poor and narrow that the mere sight
of what is arouses no emotion. It is fancy which decks reality, and if
imagination does not lend its charm to that which touches our senses,
our barren pleasure is confined to the senses alone, while the heart
remains cold. The earth adorned with the treasures of autumn displays a
wealth of colour which the eye admires; but this admiration fails to
move us, it springs rather from thought than from feeling. In spring the
country is almost bare and leafless, the trees give no shade, the grass
has hardly begun to grow, yet the heart is touched by the sight. In this
new birth of nature, we feel the revival of our own life; the memories
of past pleasures surround us; tears of delight, those companions of
pleasure ever ready to accompany a pleasing sentiment, tremble on our
eyelids. Animated, lively, and delightful though the vintage may be, we
behold it without a tear.

And why is this? Because imagination adds to the sight of spring the
image of the seasons which are yet to come; the eye sees the tender
shoot, the mind's eye beholds its flowers, fruit, and foliage, and even
the mysteries they may conceal. It blends successive stages into one
moment's experience; we see things, not so much as they will be, but as
we would have them be, for imagination has only to take her choice. In
autumn, on the other hand, we only behold the present; if we wish to
look forward to spring, winter bars the way, and our shivering
imagination dies away among its frost and snow.

This is the source of the charm we find in beholding the beauties of
childhood, rather than the perfection of manhood. When do we really
delight in beholding a man? When the memory of his deeds leads us to
look back over his life and his youth is renewed in our eyes. If we are
reduced to viewing him as he is, or to picturing him as he will be in
old age, the thought of declining years destroys all our pleasure. There
is no pleasure in seeing a man hastening to his grave; the image of
death makes all hideous.

But when I think of a child of ten or twelve, strong, healthy,
well-grown for his age, only pleasant thoughts are called up, whether of
the present or the future. I see him keen, eager, and full of life, free
from gnawing cares and painful forebodings, absorbed in this present
state, and delighting in a fullness of life which seems to extend beyond
himself. I look forward to a time when he will use his daily increasing
sense, intelligence and vigour, those growing powers of which he
continually gives fresh proof. I watch the child with delight, I picture
to myself the man with even greater pleasure. His eager life seems to
stir my own pulses, I seem to live his life and in his vigour I renew my
own.

The hour strikes, the scene is changed. All of a sudden his eye grows
dim, his mirth has fled. Farewell mirth, farewell untrammelled sports in
which he delighted. A stern, angry man takes him by the hand, saying
gravely, ``Come with me, sir,'' and he is led away. As they are entering
the room, I catch a glimpse of books. Books, what dull food for a child
of his age! The poor child allows himself to be dragged away; he casts a
sorrowful look on all about him, and departs in silence, his eyes
swollen with the tears he dare not shed, and his heart bursting with the
sighs he dare not utter.

You who have no such cause for fear, you for whom no period of life is a
time of weariness and tedium, you who welcome days without care and
nights without impatience, you who only reckon time by your pleasures,
come, my happy kindly pupil, and console us for the departure of that
miserable creature. Come! Here he is and at his approach I feel a thrill
of delight which I see he shares. It is his friend, his comrade, who
meets him; when he sees me he knows very well that he will not be long
without amusement; we are never dependent on each other, but we are
always on good terms, and we are never so happy as when together.

His face, his bearing, his expression, speak of confidence and
contentment; health shines in his countenance, his firm step speaks of
strength; his colour, delicate but not sickly, has nothing of softness
or effeminacy. Sun and wind have already set the honourable stamp of
manhood on his countenance; his rounded muscles already begin to show
some signs of growing individuality; his eyes, as yet unlighted by the
flame of feeling, have at least all their native calm; They have not
been darkened by prolonged sorrow, nor are his cheeks furrowed by
ceaseless tears. Behold in his quick and certain movements the natural
vigour of his age and the confidence of independence. His manner is free
and open, but without a trace of insolence or vanity; his head which has
not been bent over books does not fall upon his breast; there is no need
to say, ``Hold your head up,'' he will neither hang his head for shame
or fear.

Make room for him, gentlemen, in your midst; question him boldly; have
no fear of importunity, chatter, or impertinent questions. You need not
be afraid that he will take possession of you and expect you to devote
yourself entirely to him, so that you cannot get rid of him.

Neither need you look for compliments from him; nor will he tell you
what I have taught him to say; expect nothing from him but the plain,
simple truth, without addition or ornament and without vanity. He will
tell you the wrong things he has done and thought as readily as the
right, without troubling himself in the least as to the effect of his
words upon you; he will use speech with all the simplicity of its first
beginnings.

We love to augur well of our children, and we are continually regretting
the flood of folly which overwhelms the hopes we would fain have rested
on some chance phrase. If my scholar rarely gives me cause for such
prophecies, neither will he give me cause for such regrets, for he never
says a useless word, and does not exhaust himself by chattering when he
knows there is no one to listen to him. His ideas are few but precise,
he knows nothing by rote but much by experience. If he reads our books
worse than other children, he reads far better in the book of nature;
his thoughts are not in his tongue but in his brain; he has less memory
and more judgment; he can only speak one language, but he understands
what he is saying, and if his speech is not so good as that of other
children his deeds are better.

He does not know the meaning of habit, routine, and custom; what he did
yesterday has no control over what he is doing to-day; he follows no
rule, submits to no authority, copies no pattern, and only acts or
speaks as he pleases. So do not expect set speeches or studied manners
from him, but just the faithful expression of his thoughts and the
conduct that springs from his inclinations. {[}Footnote: Habit owes its
charm to man's natural idleness, and this idleness grows upon us if
indulged; it is easier to do what we have already done, there is a
beaten path which is easily followed. Thus we may observe that habit is
very strong in the aged and in the indolent, and very weak in the young
and active. The rule of habit is only good for feeble hearts, and it
makes them more and more feeble day by day. The only useful habit for
children is to be accustomed to submit without difficulty to necessity,
and the only useful habit for man is to submit without difficulty to the
rule of reason. Every other habit is a vice.{]}

You will find he has a few moral ideas concerning his present state and
none concerning manhood; what use could he make of them, for the child
is not, as yet, an active member of society. Speak to him of freedom, of
property, or even of what is usually done; he may understand you so far;
he knows why his things are his own, and why other things are not his,
and nothing more. Speak to him of duty or obedience; he will not know
what you are talking about; bid him do something and he will pay no
attention; but say to him, ``If you will give me this pleasure, I will
repay it when required,'' and he will hasten to give you satisfaction,
for he asks nothing better than to extend his domain, to acquire rights
over you, which will, he knows, be respected. Maybe he is not sorry to
have a place of his own, to be reckoned of some account; but if he has
formed this latter idea, he has already left the realms of nature, and
you have failed to bar the gates of vanity.

For his own part, should he need help, he will ask it readily of the
first person he meets. He will ask it of a king as readily as of his
servant; all men are equals in his eyes. From his way of asking you will
see he knows you owe him nothing, that he is asking a favour. He knows
too that humanity moves you to grant this favour; his words are few and
simple. His voice, his look, his gesture are those of a being equally
familiar with compliance and refusal. It is neither the crawling,
servile submission of the slave, nor the imperious tone of the master,
it is a modest confidence in mankind; it is the noble and touching
gentleness of a creature, free, yet sensitive and feeble, who asks aid
of a being, free, but strong and kindly. If you grant his request he
will not thank you, but he will feel he has incurred a debt. If you
refuse he will neither complain nor insist; he knows it is useless; he
will not say, ``They refused to help me,'' but ``It was impossible,''
and as I have already said, we do not rebel against necessity when once
we have perceived it.

Leave him to himself and watch his actions without speaking, consider
what he is doing and how he sets about it. He does not require to
convince himself that he is free, so he never acts thoughtlessly and
merely to show that he can do what he likes; does he not know that he is
always his own master? He is quick, alert, and ready; his movements are
eager as befits his age, but you will not find one which has no end in
view. Whatever he wants, he will never attempt what is beyond his
powers, for he has learnt by experience what those powers are; his means
will always be adapted to the end in view, and he will rarely attempt
anything without the certainty of success; his eye is keen and true; he
will not be so stupid as to go and ask other people about what he sees;
he will examine it on his own account, and before he asks he will try
every means at his disposal to discover what he wants to know for
himself. If he lights upon some unexpected difficulty, he will be less
upset than others; if there is danger he will be less afraid. His
imagination is still asleep and nothing has been done to arouse it; he
only sees what is really there, and rates the danger at its true worth;
so he never loses his head. He does not rebel against necessity, her
hand is too heavy upon him; he has borne her yoke all his life long, he
is well used to it; he is always ready for anything.

Work or play are all one to him, his games are his work; he knows no
difference. He brings to everything the cheerfulness of interest, the
charm of freedom, and he snows the bent of his own mind and the extent
of his knowledge. Is there anything better worth seeing, anything more
touching or more delightful, than a pretty child, with merry, cheerful
glance, easy contented manner, open smiling countenance, playing at the
most important things, or working at the lightest amusements?

Would you now judge him by comparison? Set him among other children and
leave him to himself. You will soon see which has made most progress,
which comes nearer to the perfection of childhood. Among all the
children in the town there is none more skilful and none so strong.
Among young peasants he is their equal in strength and their superior in
skill. In everything within a child's grasp he judges, reasons, and
shows a forethought beyond the rest. Is it a matter of action, running,
jumping, or shifting things, raising weights or estimating distance,
inventing games, carrying off prizes; you might say, ``Nature obeys his
word,'' so easily does he bend all things to his will. He is made to
lead, to rule his fellows; talent and experience take the place of right
and authority. In any garb, under any name, he will still be first;
everywhere he will rule the rest, they will always feel his superiority,
he will be master without knowing it, and they will serve him unawares.

He has reached the perfection of childhood; he has lived the life of a
child; his progress has not been bought at the price of his happiness,
he has gained both. While he has acquired all the wisdom of a child, he
has been as free and happy as his health permits. If the Reaper Death
should cut him off and rob us of our hopes, we need not bewail alike his
life and death, we shall not have the added grief of knowing that we
caused him pain; we will say, ``His childhood, at least, was happy; we
have robbed him of nothing that nature gave him.''

The chief drawback to this early education is that it is only
appreciated by the wise; to vulgar eyes the child so carefully educated
is nothing but a rough little boy. A tutor thinks rather of the
advantage to himself than to his pupil; he makes a point of showing that
there has been no time wasted; he provides his pupil with goods which
can be readily displayed in the shop window, accomplishments which can
be shown off at will; no matter whether they are useful, provided they
are easily seen. Without choice or discrimination he loads his memory
with a pack of rubbish. If the child is to be examined he is set to
display his wares; he spreads them out, satisfies those who behold them,
packs up his bundle and goes his way. My pupil is poorer, he has no
bundle to display, he has only himself to show. Now neither child nor
man can be read at a glance. Where are the observers who can at once
discern the characteristics of this child? There are such people, but
they are few and far between; among a thousand fathers you will scarcely
find one.

Too many questions are tedious and revolting to most of us and
especially to children. After a few minutes their attention flags, they
cease to listen to your everlasting questions and reply at random. This
way of testing them is pedantic and useless; a chance word will often
show their sense and intelligence better than much talking, but take
care that the answer is neither a matter of chance nor yet learnt by
heart. A man must needs have a good judgment if he is to estimate the
judgment of a child.

I heard the late Lord Hyde tell the following story about one of his
friends. He had returned from Italy after a three years' absence, and
was anxious to test the progress of his son, a child of nine or ten. One
evening he took a walk with the child and his tutor across a level space
where the schoolboys were flying their kites. As they went, the father
said to his son, ``Where is the kite that casts this shadow?'' Without
hesitating and without glancing upwards the child replied, ``Over the
high road.'' ``And indeed,'' said Lord Hyde, ``the high road was between
us and the sun.'' At these words, the father kissed his child, and
having finished his examination he departed. The next day he sent the
tutor the papers settling an annuity on him in addition to his salary.

What a father! and what a promising child! The question is exactly
adapted to the child's age, the answer is perfectly simple; but see what
precision it implies in the child's judgment. Thus did the pupil of
Aristotle master the famous steed which no squire had ever been able to
tame.

\mychapter{4}{Book III}

The whole course of man's life up to adolescence is a period of
weakness; yet there comes a time during these early years when the
child's strength overtakes the demands upon it, when the growing
creature, though absolutely weak, is relatively strong. His needs are
not fully developed and his present strength is more than enough for
them. He would be a very feeble man, but he is a strong child.

What is the cause of man's weakness? It is to be found in the
disproportion between his strength and his desires. It is our passions
that make us weak, for our natural strength is not enough for their
satisfaction. To limit our desires comes to the same thing, therefore,
as to increase our strength. When we can do more than we want, we have
strength enough and to spare, we are really strong. This is the third
stage of childhood, the stage with which I am about to deal. I still
speak of childhood for want of a better word; for our scholar is
approaching adolescence, though he has not yet reached the age of
puberty.

About twelve or thirteen the child's strength increases far more rapidly
than his needs. The strongest and fiercest of the passions is still
unknown, his physical development is still imperfect and seems to await
the call of the will. He is scarcely aware of extremes of heat and cold
and braves them with impunity. He needs no coat, his blood is warm; no
spices, hunger is his sauce, no food comes amiss at this age; if he is
sleepy he stretches himself on the ground and goes to sleep; he finds
all he needs within his reach; he is not tormented by any imaginary
wants; he cares nothing what others think; his desires are not beyond
his grasp; not only is he self-sufficing, but for the first and last
time in his life he has more strength than he needs.

I know beforehand what you will say. You will not assert that the child
has more needs than I attribute to him, but you will deny his strength.
You forget that I am speaking of my own pupil, not of those puppets who
walk with difficulty from one room to another, who toil indoors and
carry bundles of paper. Manly strength, you say, appears only with
manhood; the vital spirits, distilled in their proper vessels and
spreading through the whole body, can alone make the muscles firm,
sensitive, tense, and springy, can alone cause real strength. This is
the philosophy of the study; I appeal to that of experience. In the
country districts, I see big lads hoeing, digging, guiding the plough,
filling the wine-cask, driving the cart, like their fathers; you would
take them for grown men if their voices did not betray them. Even in our
towns, iron-workers', tool makers', and blacksmiths' lads are almost as
strong as their masters and would be scarcely less skilful had their
training begun earlier. If there is a difference, and I do not deny that
there is, it is, I repeat, much less than the difference between the
stormy passions of the man and the few wants of the child. Moreover, it
is not merely a question of bodily strength, but more especially of
strength of mind, which reinforces and directs the bodily strength.

This interval in which the strength of the individual is in excess of
his wants is, as I have said, relatively though not absolutely the time
of greatest strength. It is the most precious time in his life; it comes
but once; it is very short, all too short, as you will see when you
consider the importance of using it aright.

He has, therefore, a surplus of strength and capacity which he will
never have again. What use shall he make of it? He will strive to use it
in tasks which will help at need. He will, so to speak, cast his present
surplus into the storehouse of the future; the vigorous child will make
provision for the feeble man; but he will not store his goods where
thieves may break in, nor in barns which are not his own. To store them
aright, they must be in the hands and the head, they must be stored
within himself. This is the time for work, instruction, and inquiry. And
note that this is no arbitrary choice of mine, it is the way of nature
herself.

Human intelligence is finite, and not only can no man know everything,
he cannot even acquire all the scanty knowledge of others. Since the
contrary of every false proposition is a truth, there are as many truths
as falsehoods. We must, therefore, choose what to teach as well as when
to teach it. Some of the information within our reach is false, some is
useless, some merely serves to puff up its possessor. The small store
which really contributes to our welfare alone deserves the study of a
wise man, and therefore of a child whom one would have wise. He must
know not merely what is, but what is useful.

From this small stock we must also deduct those truths which require a
full grown mind for their understanding, those which suppose a knowledge
of man's relations to his fellow-men---a knowledge which no child can
acquire; these things, although in themselves true, lead an
inexperienced mind into mistakes with regard to other matters.

We are now confined to a circle, small indeed compared with the whole of
human thought, but this circle is still a vast sphere when measured by
the child's mind. Dark places of the human understanding, what rash hand
shall dare to raise your veil? What pitfalls does our so-called science
prepare for the miserable child. Would you guide him along this
dangerous path and draw the veil from the face of nature? Stay your
hand. First make sure that neither he nor you will become dizzy. Beware
of the specious charms of error and the intoxicating fumes of pride.
Keep this truth ever before you---Ignorance never did any one any harm,
error alone is fatal, and we do not lose our way through ignorance but
through self-confidence.

His progress in geometry may serve as a test and a true measure of the
growth of his intelligence, but as soon as he can distinguish between
what is useful and what is useless, much skill and discretion are
required to lead him towards theoretical studies. For example, would you
have him find a mean proportional between two lines, contrive that he
should require to find a square equal to a given rectangle; if two mean
proportionals are required, you must first contrive to interest him in
the doubling of the cube. See how we are gradually approaching the moral
ideas which distinguish between good and evil. Hitherto we have known no
law but necessity, now we are considering what is useful; we shall soon
come to what is fitting and right.

Man's diverse powers are stirred by the same instinct. The bodily
activity, which seeks an outlet for its energies, is succeeded by the
mental activity which seeks for knowledge. Children are first restless,
then curious; and this curiosity, rightly directed, is the means of
development for the age with which we are dealing. Always distinguish
between natural and acquired tendencies. There is a zeal for learning
which has no other foundation than a wish to appear learned, and there
is another which springs from man's natural curiosity about all things
far or near which may affect himself. The innate desire for comfort and
the impossibility of its complete satisfaction impel him to the endless
search for fresh means of contributing to its satisfaction. This is the
first principle of curiosity; a principle natural to the human heart,
though its growth is proportional to the development of our feeling and
knowledge. If a man of science were left on a desert island with his
books and instruments and knowing that he must spend the rest of his
life there, he would scarcely trouble himself about the solar system,
the laws of attraction, or the differential calculus. He might never
even open a book again; but he would never rest till he had explored the
furthest corner of his island, however large it might be. Let us
therefore omit from our early studies such knowledge as has no natural
attraction for us, and confine ourselves to such things as instinct
impels us to study.

Our island is this earth; and the most striking object we behold is the
sun. As soon as we pass beyond our immediate surroundings, one or both
of these must meet our eye. Thus the philosophy of most savage races is
mainly directed to imaginary divisions of the earth or to the divinity
of the sun.

What a sudden change you will say. Just now we were concerned with what
touches ourselves, with our immediate environment, and all at once we
are exploring the round world and leaping to the bounds of the universe.
This change is the result of our growing strength and of the natural
bent of the mind. While we were weak and feeble, self-preservation
concentrated our attention on ourselves; now that we are strong and
powerful, the desire for a wider sphere carries us beyond ourselves as
far as our eyes can reach. But as the intellectual world is still
unknown to us, our thoughts are bounded by the visible horizon, and our
understanding only develops within the limits of our vision.

Let us transform our sensations into ideas, but do not let us jump all
at once from the objects of sense to objects of thought. The latter are
attained by means of the former. Let the senses be the only guide for
the first workings of reason. No book but the world, no teaching but
that of fact. The child who reads ceases to think, he only reads. He is
acquiring words not knowledge.

Teach your scholar to observe the phenomena of nature; you will soon
rouse his curiosity, but if you would have it grow, do not be in too
great a hurry to satisfy this curiosity. Put the problems before him and
let him solve them himself. Let him know nothing because you have told
him, but because he has learnt it for himself. Let him not be taught
science, let him discover it. If ever you substitute authority for
reason he will cease to reason; he will be a mere plaything of other
people's thoughts.

You wish to teach this child geography and you provide him with globes,
spheres, and maps. What elaborate preparations! What is the use of all
these symbols; why not begin by showing him the real thing so that he
may at least know what you are talking about?

One fine evening we are walking in a suitable place where the wide
horizon gives us a full view of the setting sun, and we note the objects
which mark the place where it sets. Next morning we return to the same
place for a breath of fresh air before sun-rise. We see the rays of
light which announce the sun's approach; the glow increases, the east
seems afire, and long before the sun appears the light leads us to
expect its return. Every moment you expect to see it. There it is at
last! A shining point appears like a flash of lightning and soon fills
the whole space; the veil of darkness rolls away, man perceives his
dwelling place in fresh beauty. During the night the grass has assumed a
fresher green; in the light of early dawn, and gilded by the first rays
of the sun, it seems covered with a shining network of dew reflecting
the light and colour. The birds raise their chorus of praise to greet
the Father of life, not one of them is mute; their gentle warbling is
softer than by day, it expresses the langour of a peaceful waking. All
these produce an impression of freshness which seems to reach the very
soul. It is a brief hour of enchantment which no man can resist; a sight
so grand, so fair, so delicious, that none can behold it unmoved.

Fired with this enthusiasm, the master wishes to impart it to the child.
He expects to rouse his emotion by drawing attention to his own. Mere
folly! The splendour of nature lives in man's heart; to be seen, it must
be felt. The child sees the objects themselves, but does not perceive
their relations, and cannot hear their harmony. It needs knowledge he
has not yet acquired, feelings he has not yet experienced, to receive
the complex impression which results from all these separate sensations.
If he has not wandered over arid plains, if his feet have not been
scorched by the burning sands of the desert, if he has not breathed the
hot and oppressive air reflected from the glowing rocks, how shall he
delight in the fresh air of a fine morning. The scent of flowers, the
beauty of foliage, the moistness of the dew, the soft turf beneath his
feet, how shall all these delight his senses. How shall the song of the
birds arouse voluptuous emotion if love and pleasure are still unknown
to him? How shall he behold with rapture the birth of this fair day, if
his imagination cannot paint the joys it may bring in its track? How can
he feel the beauty of nature, while the hand that formed it is unknown?

Never tell the child what he cannot understand: no descriptions, no
eloquence, no figures of speech, no poetry. The time has not come for
feeling or taste. Continue to be clear and cold; the time will come only
too soon when you must adopt another tone.

Brought up in the spirit of our maxims, accustomed to make his own tools
and not to appeal to others until he has tried and failed, he will
examine everything he sees carefully and in silence. He thinks rather
than questions. Be content, therefore, to show him things at a fit
season; then, when you see that his curiosity is thoroughly aroused, put
some brief question which will set him trying to discover the answer.

On the present occasion when you and he have carefully observed the
rising sun, when you have called his attention to the mountains and
other objects visible from the same spot, after he has chattered freely
about them, keep quiet for a few minutes as if lost in thought and then
say, ``I think the sun set over there last night; it rose here this
morning. How can that be?'' Say no more; if he asks questions, do not
answer them; talk of something else. Let him alone, and be sure he will
think about it.

To train a child to be really attentive so that he may be really
impressed by any truth of experience, he must spend anxious days before
he discovers that truth. If he does not learn enough in this way, there
is another way of drawing his attention to the matter. Turn the question
about. If he does not know how the sun gets from the place where it sets
to where it rises, he knows at least how it travels from sunrise to
sunset, his eyes teach him that. Use the second question to throw light
on the first; either your pupil is a regular dunce or the analogy is too
clear to be missed. This is his first lesson in cosmography.

As we always advance slowly from one sensible idea to another, and as we
give time enough to each for him to become really familiar with it
before we go on to another, and lastly as we never force our scholar's
attention, we are still a long way from a knowledge of the course of the
sun or the shape of the earth; but as all the apparent movements of the
celestial bodies depend on the same principle, and the first observation
leads on to all the rest, less effort is needed, though more time, to
proceed from the diurnal revolution to the calculation of eclipses, than
to get a thorough understanding of day and night.

Since the sun revolves round the earth it describes a circle, and every
circle must have a centre; that we know already. This centre is
invisible, it is in the middle of the earth, but we can mark out two
opposite points on the earth's surface which correspond to it. A skewer
passed through the three points and prolonged to the sky at either end
would represent the earth's axis and the sun's daily course. A round
teetotum revolving on its point represents the sky turning on its axis,
the two points of the teetotum are the two poles; the child will be
delighted to find one of them, and I show him the tail of the Little
bear. Here is a another game for the dark. Little by little we get to
know the stars, and from this comes a wish to know the planets and
observe the constellations.

We saw the sun rise at midsummer, we shall see it rise at Christmas or
some other fine winter's day; for you know we are no lie-a-beds and we
enjoy the cold. I take care to make this second observation in the same
place as the first, and if skilfully lead up to, one or other will
certainly exclaim, ``What a funny thing! The sun is not rising in the
same place; here are our landmarks, but it is rising over there. So
there is the summer east and the winter east, etc.'' Young teacher, you
are on the right track. These examples should show you how to teach the
sphere without any difficulty, taking the earth for the earth and the
sun for the sun.

As a general rule---never substitute the symbol for the thing signified,
unless it is impossible to show the thing itself; for the child's
attention is so taken up with the symbol that he will forget what it
signifies.

I consider the armillary sphere a clumsy disproportioned bit of
apparatus. The confused circles and the strange figures described on it
suggest witchcraft and frighten the child. The earth is too small, the
circles too large and too numerous, some of them, the colures, for
instance, are quite useless, and the thickness of the pasteboard gives
them an appearance of solidity so that they are taken for circular
masses having a real existence, and when you tell the child that these
are imaginary circles, he does not know what he is looking at and is
none the wiser.

We are unable to put ourselves in the child's place, we fail to enter
into his thoughts, we invest him with our own ideas, and while we are
following our own chain of reasoning, we merely fill his head with
errors and absurdities.

Should the method of studying science be analytic or synthetic? People
dispute over this question, but it is not always necessary to choose
between them. Sometimes the same experiments allow one to use both
analysis and synthesis, and thus to guide the child by the method of
instruction when he fancies he is only analysing. Then, by using both at
once, each method confirms the results of the other. Starting from
opposite ends, without thinking of following the same road, he will
unexpectedly reach their meeting place and this will be a delightful
surprise. For example, I would begin geography at both ends and add to
the study of the earth's revolution the measurement of its divisions,
beginning at home. While the child is studying the sphere and is thus
transported to the heavens, bring him back to the divisions of the globe
and show him his own home.

His geography will begin with the town he lives in and his father's
country house, then the places between them, the rivers near them, and
then the sun's aspect and how to find one's way by its aid. This is the
meeting place. Let him make his own map, a very simple map, at first
containing only two places; others may be added from time to time, as he
is able to estimate their distance and position. You see at once what a
good start we have given him by making his eye his compass.

No doubt he will require some guidance in spite of this, but very
little, and that little without his knowing it. If he goes wrong let him
alone, do not correct his mistakes; hold your tongue till he finds them
out for himself and corrects them, or at most arrange something, as
opportunity offers, which may show him his mistakes. If he never makes
mistakes he will never learn anything thoroughly. Moreover, what he
needs is not an exact knowledge of local topography, but how to find out
for himself. No matter whether he carries maps in his head provided he
understands what they mean, and has a clear idea of the art of making
them. See what a difference there is already between the knowledge of
your scholars and the ignorance of mine. They learn maps, he makes them.
Here are fresh ornaments for his room.

Remember that this is the essential point in my method---Do not teach
the child many things, but never to let him form inaccurate or confused
ideas. I care not if he knows nothing provided he is not mistaken, and I
only acquaint him with truths to guard him against the errors he might
put in their place. Reason and judgment come slowly, prejudices flock to
us in crowds, and from these he must be protected. But if you make
science itself your object, you embark on an unfathomable and shoreless
ocean, an ocean strewn with reefs from which you will never return. When
I see a man in love with knowledge, yielding to its charms and flitting
from one branch to another unable to stay his steps, he seems to me like
a child gathering shells on the sea-shore, now picking them up, then
throwing them aside for others which he sees beyond them, then taking
them again, till overwhelmed by their number and unable to choose
between them, he flings them all away and returns empty handed.

Time was long during early childhood; we only tried to pass our time for
fear of using it ill; now it is the other way; we have not time enough
for all that would be of use. The passions, remember, are drawing near,
and when they knock at the door your scholar will have no ear for
anything else. The peaceful age of intelligence is so short, it flies so
swiftly, there is so much to be done, that it is madness to try to make
your child learned. It is not your business to teach him the various
sciences, but to give him a taste for them and methods of learning them
when this taste is more mature. That is assuredly a fundamental
principle of all good education.

This is also the time to train him gradually to prolonged attention to a
given object; but this attention should never be the result of
constraint, but of interest or desire; you must be very careful that it
is not too much for his strength, and that it is not carried to the
point of tedium. Watch him, therefore, and whatever happens, stop before
he is tired, for it matters little what he learns; it does matter that
he should do nothing against his will.

If he asks questions let your answers be enough to whet his curiosity
but not enough to satisfy it; above all, when you find him talking at
random and overwhelming you with silly questions instead of asking for
information, at once refuse to answer; for it is clear that he no longer
cares about the matter in hand, but wants to make you a slave to his
questions. Consider his motives rather than his words. This warning,
which was scarcely needed before, becomes of supreme importance when the
child begins to reason.

There is a series of abstract truths by means of which all the sciences
are related to common principles and are developed each in its turn.
This relationship is the method of the philosophers. We are not
concerned with it at present. There is quite another method by which
every concrete example suggests another and always points to the next in
the series. This succession, which stimulates the curiosity and so
arouses the attention required by every object in turn, is the order
followed by most men, and it is the right order for all children. To
take our bearings so as to make our maps we must find meridians. Two
points of intersection between the equal shadows morning and evening
supply an excellent meridian for a thirteen-year-old astronomer. But
these meridians disappear, it takes time to trace them, and you are
obliged to work in one place. So much trouble and attention will at last
become irksome. We foresaw this and are ready for it.

Again I must enter into minute and detailed explanations. I hear my
readers murmur, but I am prepared to meet their disapproval; I will not
sacrifice the most important part of this book to your impatience. You
may think me as long-winded as you please; I have my own opinion as to
your complaints.

Long ago my pupil and I remarked that some substances such as amber,
glass, and wax, when well rubbed, attracted straws, while others did
not. We accidentally discover a substance which has a more unusual
property, that of attracting filings or other small particles of iron
from a distance and without rubbing. How much time do we devote to this
game to the exclusion of everything else! At last we discover that this
property is communicated to the iron itself, which is, so to speak,
endowed with life. We go to the fair one day {[}Footnote: I could not
help laughing when I read an elaborate criticism of this little tale by
M. de Formy. ``This conjuror,'' says he, ``who is afraid of a child's
competition and preaches to his tutor is the sort of person we meet with
in the world in which Emile and such as he are living.'' This witty M.
de Formy could not guess that this little scene was arranged beforehand,
and that the juggler was taught his part in it; indeed I did not state
this fact. But I have said again and again that I was not writing for
people who expected to be told everything.{]} and a conjuror has a wax
duck floating in a basin of water, and he makes it follow a bit of
bread. We are greatly surprised, but we do not call him a wizard, never
having heard of such persons. As we are continually observing effects
whose causes are unknown to us, we are in no hurry to make up our minds,
and we remain in ignorance till we find an opportunity of learning.

When we get home we discuss the duck till we try to imitate it. We take
a needle thoroughly magnetised, we imbed it in white wax, shaped as far
as possible like a duck, with the needle running through the body, so
that its eye forms the beak. We put the duck in water and put the end of
a key near its beak, and you will readily understand our delight when we
find that our duck follows the key just as the duck at the fair followed
the bit of bread. Another time we may note the direction assumed by the
duck when left in the basin; for the present we are wholly occupied with
our work and we want nothing more.

The same evening we return to the fair with some bread specially
prepared in our pockets, and as soon as the conjuror has performed his
trick, my little doctor, who can scarcely sit still, exclaims, ``The
trick is quite easy; I can do it myself.'' ``Do it then.'' He at once
takes the bread with a bit of iron hidden in it from his pocket; his
heart throbs as he approaches the table and holds out the bread, his
hand trembles with excitement. The duck approaches and follows his hand.
The child cries out and jumps for joy. The applause, the shouts of the
crowd, are too much for him, he is beside himself. The conjuror, though
disappointed, embraces him, congratulates him, begs the honour of his
company on the following day, and promises to collect a still greater
crowd to applaud his skill. My young scientist is very proud of himself
and is beginning to chatter, but I check him at once and take him home
overwhelmed with praise.

The child counts the minutes till to-morrow with absurd anxiety. He
invites every one he meets, he wants all mankind to behold his glory; he
can scarcely wait till the appointed hour. He hurries to the place; the
hall is full already; as he enters his young heart swells with pride.
Other tricks are to come first. The conjuror surpasses himself and does
the most surprising things. The child sees none of these; he wriggles,
perspires, and hardly breathes; the time is spent in fingering with a
trembling hand the bit of bread in his pocket. His turn comes at last;
the master announces it to the audience with all ceremony; he goes up
looking somewhat shamefaced and takes out his bit of bread. Oh fleeting
joys of human life! the duck, so tame yesterday, is quite wild to-day;
instead of offering its beak it turns tail and swims away; it avoids the
bread and the hand that holds it as carefully as it followed them
yesterday. After many vain attempts accompanied by derisive shouts from
the audience the child complains that he is being cheated, that is not
the same duck, and he defies the conjuror to attract it.

The conjuror, without further words, takes a bit of bread and offers it
to the duck, which at once follows it and comes to the hand which holds
it. The child takes the same bit of bread with no better success; the
duck mocks his efforts and swims round the basin. Overwhelmed with
confusion he abandons the attempt, ashamed to face the crowd any longer.
Then the conjuror takes the bit of bread the child brought with him and
uses it as successfully as his own. He takes out the bit of iron before
the audience---another laugh at our expense---then with this same bread
he attracts the duck as before. He repeats the experiment with a piece
of bread cut by a third person in full view of the audience. He does it
with his glove, with his finger-tip. Finally he goes into the middle of
the room and in the emphatic tones used by such persons he declares that
his duck will obey his voice as readily as his hand; he speaks and the
duck obeys; he bids him go to the right and he goes, to come back again
and he comes. The movement is as ready as the command. The growing
applause completes our discomfiture. We slip away unnoticed and shut
ourselves up in our room, without relating our successes to everybody as
we had expected.

Next day there is a knock at the door. When I open it there is the
conjuror, who makes a modest complaint with regard to our conduct. What
had he done that we should try to discredit his tricks and deprive him
of his livelihood? What is there so wonderful in attracting a duck that
we should purchase this honour at the price of an honest man's living?
``My word, gentlemen! had I any other trade by which I could earn a
living I would not pride myself on this. You may well believe that a man
who has spent his life at this miserable trade knows more about it than
you who only give your spare time to it. If I did not show you my best
tricks at first, it was because one must not be so foolish as to display
all one knows at once. I always take care to keep my best tricks for
emergencies; and I have plenty more to prevent young folks from
meddling. However, I have come, gentlemen, in all kindness, to show you
the trick that gave you so much trouble; I only beg you not to use it to
my hurt, and to be more discreet in future.'' He then shows us his
apparatus, and to our great surprise we find it is merely a strong
magnet in the hand of a boy concealed under the table. The man puts up
his things, and after we have offered our thanks and apologies, we try
to give him something. He refuses it. ``No, gentlemen,'' says he, ``I
owe you no gratitude and I will not accept your gift. I leave you in my
debt in spite of all, and that is my only revenge. Generosity may be
found among all sorts of people, and I earn my pay by doing my tricks
not by teaching them.''

As he is going he blames me out-right. ``I can make excuses for the
child,'' he says, ``he sinned in ignorance. But you, sir, should know
better. Why did you let him do it? As you are living together and you
are older than he, you should look after him and give him good advice.
Your experience should be his guide. When he is grown up he will
reproach, not only himself, but you, for the faults of his youth.''

When he is gone we are greatly downcast. I blame myself for my
easy-going ways. I promise the child that another time I will put his
interests first and warn him against faults before he falls into them,
for the time is coming when our relations will be changed, when the
severity of the master must give way to the friendliness of the comrade;
this change must come gradually, you must look ahead, and very far
ahead.

We go to the fair again the next day to see the trick whose secret we
know. We approach our Socrates, the conjuror, with profound respect, we
scarcely dare to look him in the face. He overwhelms us with politeness,
gives us the best places, and heaps coals of fire on our heads. He goes
through his performance as usual, but he lingers affectionately over the
duck, and often glances proudly in our direction. We are in the secret,
but we do not tell. If my pupil did but open his mouth he would be
worthy of death.

There is more meaning than you suspect in this detailed illustration.
How many lessons in one! How mortifying are the results of a first
impulse towards vanity! Young tutor, watch this first impulse carefully.
If you can use it to bring about shame and disgrace, you may be sure it
will not recur for many a day. What a fuss you will say. Just so; and
all to provide a compass which will enable us to dispense with a
meridian!

Having learnt that a magnet acts through other bodies, our next business
is to construct a bit of apparatus similar to that shown us. A bare
table, a shallow bowl placed on it and filled with water, a duck rather
better finished than the first, and so on. We often watch the thing and
at last we notice that the duck, when at rest, always turns the same
way. We follow up this observation; we examine the direction, we find
that it is from south to north. Enough! we have found our compass or its
equivalent; the study of physics is begun.

There are various regions of the earth, and these regions differ in
temperature. The variation is more evident as we approach the poles; all
bodies expand with heat and contract with cold; this is best measured in
liquids and best of all in spirits; hence the thermometer. The wind
strikes the face, then the air is a body, a fluid; we feel it though we
cannot see it. I invert a glass in water; the water will not fill it
unless you leave a passage for the escape of the air; so air is capable
of resistance. Plunge the glass further in the water; the water will
encroach on the air-space without filling it entirely; so air yields
somewhat to pressure. A ball filled with compressed air bounces better
than one filled with anything else; so air is elastic. Raise your arm
horizontally from the water when you are lying in your bath; you will
feel a terrible weight on it; so air is a heavy body. By establishing an
equilibrium between air and other fluids its weight can be measured,
hence the barometer, the siphon, the air-gun, and the air-pump. All the
laws of statics and hydrostatics are discovered by such rough
experiments. For none of these would I take the child into a physical
cabinet; I dislike that array of instruments and apparatus. The
scientific atmosphere destroys science. Either the child is frightened
by these instruments or his attention, which should be fixed on their
effects, is distracted by their appearance.

We shall make all our apparatus ourselves, and I would not make it
beforehand, but having caught a glimpse of the experiment by chance we
mean to invent step by step an instrument for its verification. I would
rather our apparatus was somewhat clumsy and imperfect, but our ideas
clear as to what the apparatus ought to be, and the results to be
obtained by means of it. For my first lesson in statics, instead of
fetching a balance, I lay a stick across the back of a chair, I measure
the two parts when it is balanced; add equal or unequal weights to
either end; by pulling or pushing it as required, I find at last that
equilibrium is the result of a reciprocal proportion between the amount
of the weights and the length of the levers. Thus my little physicist is
ready to rectify a balance before ever he sees one.

Undoubtedly the notions of things thus acquired for oneself are clearer
and much more convincing than those acquired from the teaching of
others; and not only is our reason not accustomed to a slavish
submission to authority, but we develop greater ingenuity in discovering
relations, connecting ideas and inventing apparatus, than when we merely
accept what is given us and allow our minds to be enfeebled by
indifference, like the body of a man whose servants always wait on him,
dress him and put on his shoes, whose horse carries him, till he loses
the use of his limbs. Boileau used to boast that he had taught Racine
the art of rhyming with difficulty. Among the many short cuts to
science, we badly need some one to teach us the art of learning with
difficulty.

The most obvious advantage of these slow and laborious inquiries is
this: the scholar, while engaged in speculative studies, is actively
using his body, gaining suppleness of limb, and training his hands to
labour so that he will be able to make them useful when he is a man. Too
much apparatus, designed to guide us in our experiments and to
supplement the exactness of our senses, makes us neglect to use those
senses. The theodolite makes it unnecessary to estimate the size of
angles; the eye which used to judge distances with much precision,
trusts to the chain for its measurements; the steel yard dispenses with
the need of judging weight by the hand as I used to do. The more
ingenious our apparatus, the coarser and more unskilful are our senses.
We surround ourselves with tools and fail to use those with which nature
has provided every one of us.

But when we devote to the making of these instruments the skill which
did instead of them, when for their construction we use the intelligence
which enabled us to dispense with them, this is gain not loss, we add
art to nature, we gain ingenuity without loss of skill. If instead of
making a child stick to his books I employ him in a workshop, his hands
work for the development of his mind. While he fancies himself a workman
he is becoming a philosopher. Moreover, this exercise has other
advantages of which I shall speak later; and you will see how, through
philosophy in sport, one may rise to the real duties of man.

I have said already that purely theoretical science is hardly suitable
for children, even for children approaching adolescence; but without
going far into theoretical physics, take care that all their experiments
are connected together by some chain of reasoning, so that they may
follow an orderly sequence in the mind, and may be recalled at need; for
it is very difficult to remember isolated facts or arguments, when there
is no cue for their recall.

In your inquiry into the laws of nature always begin with the commonest
and most conspicuous phenomena, and train your scholar not to accept
these phenomena as causes but as facts. I take a stone and pretend to
place it in the air; I open my hand, the stone falls. I see Emile
watching my action and I say, ``Why does this stone fall?''

What child will hesitate over this question? None, not even Emile,
unless I have taken great pains to teach him not to answer. Every one
will say, ``The stone falls because it is heavy.'' ``And what do you
mean by heavy?'' ``That which falls.'' ``So the stone falls because it
falls?'' Here is a poser for my little philosopher. This is his first
lesson in systematic physics, and whether he learns physics or no it is
a good lesson in common-sense.

As the child develops in intelligence other important considerations
require us to be still more careful in our choice of his occupations. As
soon as he has sufficient self-knowledge to understand what constitutes
his well-being, as soon as he can grasp such far-reaching relations as
to judge what is good for him and what is not, then he is able to
discern the difference between work and play, and to consider the latter
merely as relaxation. The objects of real utility may be introduced into
his studies and may lead him to more prolonged attention than he gave to
his games. The ever-recurring law of necessity soon teaches a man to do
what he does not like, so as to avert evils which he would dislike still
more. Such is the use of foresight, and this foresight, well or ill
used, is the source of all the wisdom or the wretchedness of mankind.

Every one desires happiness, but to secure it he must know what
happiness is. For the natural man happiness is as simple as his life; it
consists in the absence of pain; health, freedom, the necessaries of
life are its elements. The happiness of the moral man is another matter,
but it does not concern us at present. I cannot repeat too often that it
is only objects which can be perceived by the senses which can have any
interest for children, especially children whose vanity has not been
stimulated nor their minds corrupted by social conventions.

As soon as they foresee their needs before they feel them, their
intelligence has made a great step forward, they are beginning to know
the value of time. They must then be trained to devote this time to
useful purposes, but this usefulness should be such as they can readily
perceive and should be within the reach of their age and experience.
What concerns the moral order and the customs of society should not yet
be given them, for they are not in a condition to understand it. It is
folly to expect them to attend to things vaguely described as good for
them, when they do not know what this good is, things which they are
assured will be to their advantage when they are grown up, though for
the present they take no interest in this so-called advantage, which
they are unable to understand.

Let the child do nothing because he is told; nothing is good for him but
what he recognises as good. When you are always urging him beyond his
present understanding, you think you are exercising a foresight which
you really lack. To provide him with useless tools which he may never
require, you deprive him of man's most useful tool---common-sense. You
would have him docile as a child; he will be a credulous dupe when he
grows up. You are always saying, ``What I ask is for your good, though
you cannot understand it. What does it matter to me whether you do it or
not; my efforts are entirely on your account.'' All these fine speeches
with which you hope to make him good, are preparing the way, so that the
visionary, the tempter, the charlatan, the rascal, and every kind of
fool may catch him in his snare or draw him into his folly.

A man must know many things which seem useless to a child, but need the
child learn, or can he indeed learn, all that the man must know? Try to
teach the child what is of use to a child and you will find that it
takes all his time. Why urge him to the studies of an age he may never
reach, to the neglect of those studies which meet his present needs?
``But,'' you ask, ``will it not be too late to learn what he ought to
know when the time comes to use it?'' I cannot tell; but this I do know,
it is impossible to teach it sooner, for our real teachers are
experience and emotion, and man will never learn what befits a man
except under its own conditions. A child knows he must become a man; all
the ideas he may have as to man's estate are so many opportunities for
his instruction, but he should remain in complete ignorance of those
ideas which are beyond his grasp. My whole book is one continued
argument in support of this fundamental principle of education.

As soon as we have contrived to give our pupil an idea of the word
``Useful,'' we have got an additional means of controlling him, for this
word makes a great impression on him, provided that its meaning for him
is a meaning relative to his own age, and provided he clearly sees its
relation to his own well-being. This word makes no impression on your
scholars because you have taken no pains to give it a meaning they can
understand, and because other people always undertake to supply their
needs so that they never require to think for themselves, and do not
know what utility is.

``What is the use of that?'' In future this is the sacred formula, the
formula by which he and I test every action of our lives. This is the
question with which I invariably answer all his questions; it serves to
check the stream of foolish and tiresome questions with which children
weary those about them. These incessant questions produce no result, and
their object is rather to get a hold over you than to gain any real
advantage. A pupil, who has been really taught only to want to know what
is useful, questions like Socrates; he never asks a question without a
reason for it, for he knows he will be required to give his reason
before he gets an answer.

See what a powerful instrument I have put into your hands for use with
your pupil. As he does not know the reason for anything you can reduce
him to silence almost at will; and what advantages do your knowledge and
experience give you to show him the usefulness of what you suggest. For,
make no mistake about it, when you put this question to him, you are
teaching him to put it to you, and you must expect that whatever you
suggest to him in the future he will follow your own example and ask,
``What is the use of this?''

Perhaps this is the greatest of the tutor's difficulties. If you merely
try to put the child off when he asks a question, and if you give him a
single reason he is not able to understand, if he finds that you reason
according to your own ideas, not his, he will think what you tell him is
good for you but not for him; you will lose his confidence and all your
labour is thrown away. But what master will stop short and confess his
faults to his pupil? We all make it a rule never to own to the faults we
really have. Now I would make it a rule to admit even the faults I have
not, if I could not make my reasons clear to him; as my conduct will
always be intelligible to him, he will never doubt me and I shall gain
more credit by confessing my imaginary faults than those who conceal
their real defects.

In the first place do not forget that it is rarely your business to
suggest what he ought to learn; it is for him to want to learn, to seek
and to find it. You should put it within his reach, you should skilfully
awaken the desire and supply him with means for its satisfaction. So
your questions should be few and well-chosen, and as he will always have
more questions to put to you than you to him, you will always have the
advantage and will be able to ask all the oftener, ``What is the use of
that question?'' Moreover, as it matters little what he learns provided
he understands it and knows how to use it, as soon as you cannot give
him a suitable explanation give him none at all. Do not hesitate to say,
``I have no good answer to give you; I was wrong, let us drop the
subject.'' If your teaching was really ill-chosen there is no harm in
dropping it altogether; if it was not, with a little care you will soon
find an opportunity of making its use apparent to him.

I do not like verbal explanations. Young people pay little heed to them,
nor do they remember them. Things! Things! I cannot repeat it too often.
We lay too much stress upon words; we teachers babble, and our scholars
follow our example.

Suppose we are studying the course of the sun and the way to find our
bearings, when all at once Emile interrupts me with the question, ``What
is the use of that?'' what a fine lecture I might give, how many things
I might take occasion to teach him in reply to his question, especially
if there is any one there. I might speak of the advantages of travel,
the value of commerce, the special products of different lands and the
peculiar customs of different nations, the use of the calendar, the way
to reckon the seasons for agriculture, the art of navigation, how to
steer our course at sea, how to find our way without knowing exactly
where we are. Politics, natural history, astronomy, even morals and
international law are involved in my explanation, so as to give my pupil
some idea of all these sciences and a great wish to learn them. When I
have finished I shall have shown myself a regular pedant, I shall have
made a great display of learning, and not one single idea has he
understood. He is longing to ask me again, ``What is the use of taking
one's bearings?'' but he dare not for fear of vexing me. He finds it
pays best to pretend to listen to what he is forced to hear. This is the
practical result of our fine systems of education.

But Emile is educated in a simpler fashion. We take so much pains to
teach him a difficult idea that he will have heard nothing of all this.
At the first word he does not understand, he will run away, he will
prance about the room, and leave me to speechify by myself. Let us seek
a more commonplace explanation; my scientific learning is of no use to
him.

We were observing the position of the forest to the north of Montmorency
when he interrupted me with the usual question, ``What is the use of
that?'' ``You are right,'' I said. ``Let us take time to think it over,
and if we find it is no use we will drop it, for we only want useful
games.'' We find something else to do and geography is put aside for the
day.

Next morning I suggest a walk before breakfast; there is nothing he
would like better; children are always ready to run about, and he is a
good walker. We climb up to the forest, we wander through its clearings
and lose ourselves; we have no idea where we are, and when we want to
retrace our steps we cannot find the way. Time passes, we are hot and
hungry; hurrying vainly this way and that we find nothing but woods,
quarries, plains, not a landmark to guide us. Very hot, very tired, very
hungry, we only get further astray. At last we sit down to rest and to
consider our position. I assume that Emile has been educated like an
ordinary child. He does not think, he begins to cry; he has no idea we
are close to Montmorency, which is hidden from our view by a mere
thicket; but this thicket is a forest to him, a man of his size is
buried among bushes. After a few minutes' silence I begin
anxiously------

JEAN JACQUES. My dear Emile, what shall we do get out?

EMILE. I am sure I do not know. I am tired, I am hungry, I am thirsty. I
cannot go any further.

JEAN JACQUES. Do you suppose I am any better off? I would cry too if I
could make my breakfast off tears. Crying is no use, we must look about
us. Let us see your watch; what time is it?

EMILE. It is noon and I am so hungry!

JEAN JACQUES. Just so; it is noon and I am so hungry too.

EMILE. You must be very hungry indeed.

JEAN JACQUES. Unluckily my dinner won't come to find me. It is twelve
o'clock. This time yesterday we were observing the position of the
forest from Montmorency. If only we could see the position of
Montmorency from the forest.

EMILE. But yesterday we could see the forest, and here we cannot see the
town.

JEAN JACQUES. That is just it. If we could only find it without seeing
it.

EMILE. Oh! my dear friend!

JEAN JACQUES. Did not we say the forest was\ldots{}

EMILE. North of Montmorency.

JEAN JACQUES. Then Montmorency must lie\ldots{}

EMILE. South of the forest.

JEAN JACQUES. We know how to find the north at midday.

EMILE. Yes, by the direction of the shadows.

JEAN JACQUES. But the south?

EMILE. What shall we do?

JEAN JACQUES. The south is opposite the north.

EMILE. That is true; we need only find the opposite of the shadows. That
is the south! That is the south! Montmorency must be over there! Let us
look for it there!

JEAN JACQUES. Perhaps you are right; let us follow this path through the
wood.

EMILE. (Clapping his hands.) Oh, I can see Montmorency! there it is,
quite plain, just in front of us! Come to luncheon, come to dinner, make
haste! Astronomy is some use after all.

Be sure that he thinks this if he does not say it; no matter which,
provided I do not say it myself. He will certainly never forget this
day's lesson as long as he lives, while if I had only led him to think
of all this at home, my lecture would have been forgotten the next day.
Teach by doing whenever you can, and only fall back upon words when
doing is out of the question.

The reader will not expect me to have such a poor opinion of him as to
supply him with an example of every kind of study; but, whatever is
taught, I cannot too strongly urge the tutor to adapt his instances to
the capacity of his scholar; for once more I repeat the risk is not in
what he does not know, but in what he thinks he knows.

I remember how I once tried to give a child a taste for chemistry. After
showing him several metallic precipitates, I explained how ink was made.
I told him how its blackness was merely the result of fine particles of
iron separated from the vitriol and precipitated by an alkaline
solution. In the midst of my learned explanation the little rascal
pulled me up short with the question I myself had taught him. I was
greatly puzzled. After a few moments' thought I decided what to do. I
sent for some wine from the cellar of our landlord, and some very cheap
wine from a wine-merchant. I took a small {[}Footnote: Before giving any
explanation to a child a little bit of apparatus serves to fix his
attention.{]} flask of an alkaline solution, and placing two glasses
before me filled with the two sorts of wine, I said.

Food and drink are adulterated to make them seem better than they really
are. These adulterations deceive both the eye and the palate, but they
are unwholesome and make the adulterated article even worse than before
in spite of its fine appearance.

All sorts of drinks are adulterated, and wine more than others; for the
fraud is more difficult to detect, and more profitable to the fraudulent
person.

Sour wine is adulterated with litharge; litharge is a preparation of
lead. Lead in combination with acids forms a sweet salt which corrects
the harsh taste of the sour wine, but it is poisonous. So before we
drink wine of doubtful quality we should be able to tell if there is
lead in it. This is how I should do it.

Wine contains not merely an inflammable spirit as you have seen from the
brandy made from it; it also contains an acid as you know from the
vinegar made from it.

This acid has an affinity for metals, it combines with them and forms
salts, such as iron-rust, which is only iron dissolved by the acid in
air or water, or such as verdegris, which is only copper dissolved in
vinegar.

But this acid has a still greater affinity for alkalis than for metals,
so that when we add alkalis to the above-mentioned salts, the acid sets
free the metal with which it had combined, and combines with the alkali.

Then the metal, set free by the acid which held it in solution, is
precipitated and the liquid becomes opaque.

If then there is litharge in either of these glasses of wine, the acid
holds the litharge in solution. When I pour into it an alkaline
solution, the acid will be forced to set the lead free in order to
combine with the alkali. The lead, no longer held in solution, will
reappear, the liquor will become thick, and after a time the lead will
be deposited at the bottom of the glass.

If there is no lead {[}Footnote: The wine sold by retail dealers in
Paris is rarely free from lead, though some of it does not contain
litharge, for the counters are covered with lead and when the wine is
poured into the measures and some of it spilt upon the counter and the
measures left standing on the counter, some of the lead is always
dissolved. It is strange that so obvious and dangerous an abuse should
be tolerated by the police. But indeed well-to-do people, who rarely
drink these wines, are not likely to be poisoned by them.{]} nor other
metal in the wine the alkali will slowly {[}Footnote: The vegetable acid
is very gentle in its action. If it were a mineral acid and less
diluted, the combination would not take place without effervescence.{]}
combine with the acid, all will remain clear and there will be no
precipitate.

Then I poured my alkaline solution first into one glass and then into
the other. The wine from our own house remained clear and unclouded, the
other at once became turbid, and an hour later the lead might be plainly
seen, precipitated at the bottom of the glass.

``This,'' said I, ``is a pure natural wine and fit to drink; the other
is adulterated and poisonous. You wanted to know the use of knowing how
to make ink. If you can make ink you can find out what wines are
adulterated.''

I was very well pleased with my illustration, but I found it made little
impression on my pupil. When I had time to think about it I saw I had
been a fool, for not only was it impossible for a child of twelve to
follow my explanations, but the usefulness of the experiment did not
appeal to him; he had tasted both glasses of wine and found them both
good, so he attached no meaning to the word ``adulterated'' which I
thought I had explained so nicely. Indeed, the other words,
``unwholesome'' and ``poison,'' had no meaning whatever for him; he was
in the same condition as the boy who told the story of Philip and his
doctor. It is the condition of all children.

The relation of causes and effects whose connection is unknown to us,
good and ill of which we have no idea, the needs we have never felt,
have no existence for us. It is impossible to interest ourselves in them
sufficiently to make us do anything connected with them. At fifteen we
become aware of the happiness of a good man, as at thirty we become
aware of the glory of Paradise. If we had no clear idea of either we
should make no effort for their attainment; and even if we had a clear
idea of them, we should make little or no effort unless we desired them
and unless we felt we were made for them. It is easy to convince a child
that what you wish to teach him is useful, but it is useless to convince
if you cannot also persuade. Pure reason may lead us to approve or
censure, but it is feeling which leads to action, and how shall we care
about that which does not concern us?

Never show a child what he cannot see. Since mankind is almost unknown
to him, and since you cannot make a man of him, bring the man down to
the level of the child. While you are thinking what will be useful to
him when he is older, talk to him of what he knows he can use now.
Moreover, as soon as he begins to reason let there be no comparison with
other children, no rivalry, no competition, not even in running races. I
would far rather he did not learn anything than have him learn it
through jealousy or self-conceit. Year by year I shall just note the
progress he had made, I shall compare the results with those of the
following year, I shall say, ``You have grown so much; that is the ditch
you jumped, the weight you carried, the distance you flung a pebble, the
race you ran without stopping to take breath, etc.; let us see what you
can do now.''

In this way he is stimulated to further effort without jealousy. He
wants to excel himself as he ought to do; I see no reason why he should
not emulate his own performances.

I hate books; they only teach us to talk about things we know nothing
about. Hermes, they say, engraved the elements of science on pillars
lest a deluge should destroy them. Had he imprinted them on men's hearts
they would have been preserved by tradition. Well-trained minds are the
pillars on which human knowledge is most deeply engraved.

Is there no way of correlating so many lessons scattered through so many
books, no way of focussing them on some common object, easy to see,
interesting to follow, and stimulating even to a child? Could we but
discover a state in which all man's needs appear in such a way as to
appeal to the child's mind, a state in which the ways of providing for
these needs are as easily developed, the simple and stirring portrayal
of this state should form the earliest training of the child's
imagination.

Eager philosopher, I see your own imagination at work. Spare yourself
the trouble; this state is already known, it is described, with due
respect to you, far better than you could describe it, at least with
greater truth and simplicity. Since we must have books, there is one
book which, to my thinking, supplies the best treatise on an education
according to nature. This is the first book Emile will read; for a long
time it will form his whole library, and it will always retain an
honoured place. It will be the text to which all our talks about natural
science are but the commentary. It will serve to test our progress
towards a right judgment, and it will always be read with delight, so
long as our taste is unspoilt. What is this wonderful book? Is it
Aristotle? Pliny? Buffon? No; it is Robinson Crusoe.

Robinson Crusoe on his island, deprived of the help of his fellow-men,
without the means of carrying on the various arts, yet finding food,
preserving his life, and procuring a certain amount of comfort; this is
the thing to interest people of all ages, and it can be made attractive
to children in all sorts of ways. We shall thus make a reality of that
desert island which formerly served as an illustration. The condition, I
confess, is not that of a social being, nor is it in all probability
Emile's own condition, but he should use it as a standard of comparison
for all other conditions. The surest way to raise him above prejudice
and to base his judgments on the true relations of things, is to put him
in the place of a solitary man, and to judge all things as they would be
judged by such a man in relation to their own utility.

This novel, stripped of irrelevant matter, begins with Robinson's
shipwreck on his island, and ends with the coming of the ship which
bears him from it, and it will furnish Emile with material, both for
work and play, during the whole period we are considering. His head
should be full of it, he should always be busy with his castle, his
goats, his plantations. Let him learn in detail, not from books but from
things, all that is necessary in such a case. Let him think he is
Robinson himself; let him see himself clad in skins, wearing a tall cap,
a great cutlass, all the grotesque get-up of Robinson Crusoe, even to
the umbrella which he will scarcely need. He should anxiously consider
what steps to take; will this or that be wanting. He should examine his
hero's conduct; has he omitted nothing; is there nothing he could have
done better? He should carefully note his mistakes, so as not to fall
into them himself in similar circumstances, for you may be sure he will
plan out just such a settlement for himself. This is the genuine castle
in the air of this happy age, when the child knows no other happiness
but food and freedom.

What a motive will this infatuation supply in the hands of a skilful
teacher who has aroused it for the purpose of using it. The child who
wants to build a storehouse on his desert island will be more eager to
learn than the master to teach. He will want to know all sorts of useful
things and nothing else; you will need the curb as well as the spur.
Make haste, therefore, to establish him on his island while this is all
he needs to make him happy; for the day is at hand, when, if he must
still live on his island, he will not be content to live alone, when
even the companionship of Man Friday, who is almost disregarded now,
will not long suffice.

The exercise of the natural arts, which may be carried on by one man
alone, leads on to the industrial arts which call for the cooperation of
many hands. The former may be carried on by hermits, by savages, but the
others can only arise in a society, and they make society necessary. So
long as only bodily needs are recognised man is self-sufficing; with
superfluity comes the need for division and distribution of labour, for
though one man working alone can earn a man's living, one hundred men
working together can earn the living of two hundred. As soon as some men
are idle, others must work to make up for their idleness.

Your main object should be to keep out of your scholar's way all idea of
such social relations as he cannot understand, but when the development
of knowledge compels you to show him the mutual dependence of mankind,
instead of showing him its moral side, turn all his attention at first
towards industry and the mechanical arts which make men useful to one
another. While you take him from one workshop to another, let him try
his hand at every trade you show him, and do not let him leave it till
he has thoroughly learnt why everything is done, or at least everything
that has attracted his attention. With this aim you should take a share
in his work and set him an example. Be yourself the apprentice that he
may become a master; you may expect him to learn more in one hour's work
than he would retain after a whole day's explanation.

The value set by the general public on the various arts is in inverse
ratio to their real utility. They are even valued directly according to
their uselessness. This might be expected. The most useful arts are the
worst paid, for the number of workmen is regulated by the demand, and
the work which everybody requires must necessarily be paid at a rate
which puts it within the reach of the poor. On the other hand, those
great people who are called artists, not artisans, who labour only for
the rich and idle, put a fancy price on their trifles; and as the real
value of this vain labour is purely imaginary, the price itself adds to
their market value, and they are valued according to their costliness.
The rich think so much of these things, not because they are useful, but
because they are beyond the reach of the poor. Nolo habere bona, nisi
quibus populus inviderit.

What will become of your pupils if you let them acquire this foolish
prejudice, if you share it yourself? If, for instance, they see you show
more politeness in a jeweller's shop than in a locksmith's. What idea
will they form of the true worth of the arts and the real value of
things when they see, on the one hand, a fancy price and, on the other,
the price of real utility, and that the more a thing costs the less it
is worth? As soon as you let them get hold of these ideas, you may give
up all attempt at further education; in spite of you they will be like
all the other scholars---you have wasted fourteen years.

Emile, bent on furnishing his island, will look at things from another
point of view. Robinson would have thought more of a toolmaker's shop
than all Saide's trifles put together. He would have reckoned the
toolmaker a very worthy man, and Saide little more than a charlatan.

``My son will have to take the world as he finds it, he will not live
among the wise but among fools; he must therefore be acquainted with
their follies, since they must be led by this means. A real knowledge of
things may be a good thing in itself, but the knowledge of men and their
opinions is better, for in human society man is the chief tool of man,
and the wisest man is he who best knows the use of this tool. What is
the good of teaching children an imaginary system, just the opposite of
the established order of things, among which they will have to live?
First teach them wisdom, then show them the follies of mankind.''

These are the specious maxims by which fathers, who mistake them for
prudence, strive to make their children the slaves of the prejudices in
which they are educated, and the puppets of the senseless crowd, which
they hope to make subservient to their passions. How much must be known
before we attain to a knowledge of man. This is the final study of the
philosopher, and you expect to make it the first lesson of the child!
Before teaching him our sentiments, first teach him to judge of their
worth. Do you perceive folly when you mistake it for wisdom? To be wise
we must discern between good and evil. How can your child know men, when
he can neither judge of their judgments nor unravel their mistakes? It
is a misfortune to know what they think, without knowing whether their
thoughts are true or false. First teach him things as they really are,
afterwards you will teach him how they appear to us. He will then be
able to make a comparison between popular ideas and truth, and be able
to rise above the vulgar crowd; for you are unaware of the prejudices
you adopt, and you do not lead a nation when you are like it. But if you
begin to teach the opinions of other people before you teach how to
judge of their worth, of one thing you may be sure, your pupil will
adopt those opinions whatever you may do, and you will not succeed in
uprooting them. I am therefore convinced that to make a young man judge
rightly, you must form his judgment rather than teach him your own.

So far you see I have not spoken to my pupil about men; he would have
too much sense to listen to me. His relations to other people are as yet
not sufficiently apparent to him to enable him to judge others by
himself. The only person he knows is himself, and his knowledge of
himself is very imperfect. But if he forms few opinions about others,
those opinions are correct. He knows nothing of another's place, but he
knows his own and keeps to it. I have bound him with the strong cord of
necessity, instead of social laws, which are beyond his knowledge. He is
still little more than a body; let us treat him as such.

Every substance in nature and every work of man must be judged in
relation to his own use, his own safety, his own preservation, his own
comfort. Thus he should value iron far more than gold, and glass than
diamonds; in the same way he has far more respect for a shoemaker or a
mason than for a Lempereur, a Le Blanc, or all the jewellers in Europe.
In his eyes a confectioner is a really great man, and he would give the
whole academy of sciences for the smallest pastrycook in Lombard Street.
Goldsmiths, engravers, gilders, and embroiderers, he considers lazy
people, who play at quite useless games. He does not even think much of
a clockmaker. The happy child enjoys Time without being a slave to it;
he uses it, but he does not know its value. The freedom from passion
which makes every day alike to him, makes any means of measuring time
unnecessary. When I assumed that Emile had a watch, {[}Footnote: When
our hearts are abandoned to the sway of passion, then it is that we need
a measure of time. The wise man's watch is his equable temper and his
peaceful heart. He is always punctual, and he always knows the time.{]}
just as I assumed that he cried, it was a commonplace Emile that I chose
to serve my purpose and make myself understood. The real Emile, a child
so different from the rest, would not serve as an illustration for
anything.

There is an order no less natural and even more accurate, by which the
arts are valued according to bonds of necessity which connect them; the
highest class consists of the most independent, the lowest of those most
dependent on others. This classification, which suggests important
considerations on the order of society in general, is like the preceding
one in that it is subject to the same inversion in popular estimation,
so that the use of raw material is the work of the lowest and worst paid
trades, while the oftener the material changes hands, the more the work
rises in price and in honour. I do not ask whether industry is really
greater and more deserving of reward when engaged in the delicate arts
which give the final shape to these materials, than in the labour which
first gave them to man's use; but this I say, that in everything the art
which is most generally useful and necessary, is undoubtedly that which
most deserves esteem, and that art which requires the least help from
others, is more worthy of honour than those which are dependent on other
arts, since it is freer and more nearly independent. These are the true
laws of value in the arts; all others are arbitrary and dependent on
popular prejudice.

Agriculture is the earliest and most honourable of arts; metal work I
put next, then carpentry, and so on. This is the order in which the
child will put them, if he has not been spoilt by vulgar prejudices.
What valuable considerations Emile will derive from his Robinson in such
matters. What will he think when he sees the arts only brought to
perfection by sub-division, by the infinite multiplication of tools. He
will say, ``All those people are as silly as they are ingenious; one
would think they were afraid to use their eyes and their hands, they
invent so many tools instead. To carry on one trade they become the
slaves of many others; every single workman needs a whole town. My
friend and I try to gain skill; we only make tools we can take about
with us; these people, who are so proud of their talents in Paris, would
be no use at all on our island; they would have to become apprentices.''

Reader, do not stay to watch the bodily exercises and manual skill of
our pupil, but consider the bent we are giving to his childish
curiosity; consider his common-sense, his inventive spirit, his
foresight; consider what a head he will have on his shoulders. He will
want to know all about everything he sees or does, to learn the why and
the wherefore of it; from tool to tool he will go back to the first
beginning, taking nothing for granted; he will decline to learn anything
that requires previous knowledge which he has not acquired. If he sees a
spring made he will want to know how they got the steel from the mine;
if he sees the pieces of a chest put together, he will want to know how
the tree was out down; when at work he will say of each tool, ``If I had
not got this, how could I make one like it, or how could I get along
without it?''

It is, however, difficult to avoid another error. When the master is
very fond of certain occupations, he is apt to assume that the child
shares his tastes; beware lest you are carried away by the interest of
your work, while the child is bored by it, but is afraid to show it. The
child must come first, and you must devote yourself entirely to him.
Watch him, study him constantly, without his knowing it; consider his
feelings beforehand, and provide against those which are undesirable,
keep him occupied in such a way that he not only feels the usefulness of
the thing, but takes a pleasure in understanding the purpose which his
work will serve.

The solidarity of the arts consists in the exchange of industry, that of
commerce in the exchange of commodities, that of banks in the exchange
of money or securities. All these ideas hang together, and their
foundation has already been laid in early childhood with the help of
Robert the gardener. All we have now to do is to substitute general
ideas for particular, and to enlarge these ideas by means of numerous
examples, so as to make the child understand the game of business
itself, brought home to him by means of particular instances of natural
history with regard to the special products of each country, by
particular instances of the arts and sciences which concern navigation
and the difficulties of transport, greater or less in proportion to the
distance between places, the position of land, seas, rivers, etc.

There can be no society without exchange, no exchange without a common
standard of measurement, no common standard of measurement without
equality. Hence the first law of every society is some conventional
equality either in men or things.

Conventional equality between men, a very different thing from natural
equality, leads to the necessity for positive law, i.e., government and
kings. A child's political knowledge should be clear and restricted; he
should know nothing of government in general, beyond what concerns the
rights of property, of which he has already some idea.

Conventional equality between things has led to the invention of money,
for money is only one term in a comparison between the values of
different sorts of things; and in this sense money is the real bond of
society; but anything may be money; in former days it was cattle; shells
are used among many tribes at the present day; Sparta used iron; Sweden,
leather; while we use gold and silver.

Metals, being easier to carry, have generally been chosen as the middle
term of every exchange, and these metals have been made into coin to
save the trouble of continual weighing and measuring, for the stamp on
the coin is merely evidence that the coin is of given weight; and the
sole right of coining money is vested in the ruler because he alone has
the right to demand the recognition of his authority by the whole
nation.

The stupidest person can perceive the use of money when it is explained
in this way. It is difficult to make a direct comparison between various
things, for instance, between cloth and corn; but when we find a common
measure, in money, it is easy for the manufacturer and the farmer to
estimate the value of the goods they wish to exchange in terms of this
common measure. If a given quantity of cloth is worth a given some of
money, and a given quantity of corn is worth the same sum of money, then
the seller, receiving the corn in exchange for his cloth, makes a fair
bargain. Thus by means of money it becomes possible to compare the
values of goods of various kinds.

Be content with this, and do not touch upon the moral effects of this
institution. In everything you must show clearly the use before the
abuse. If you attempt to teach children how the sign has led to the
neglect of the thing signified, how money is the source of all the false
ideas of society, how countries rich in silver must be poor in
everything else, you will be treating these children as philosophers,
and not only as philosophers but as wise men, for you are professing to
teach them what very few philosophers have grasped.

What a wealth of interesting objects, towards which the curiosity of our
pupil may be directed without ever quitting the real and material
relations he can understand, and without permitting the formation of a
single idea beyond his grasp! The teacher's art consists in this: To
turn the child's attention from trivial details and to guide his
thoughts continually towards relations of importance which he will one
day need to know, that he may judge rightly of good and evil in human
society. The teacher must be able to adapt the conversation with which
he amuses his pupil to the turn already given to his mind. A problem
which another child would never heed will torment Emile half a year.

We are going to dine with wealthy people; when we get there everything
is ready for a feast, many guests, many servants, many dishes, dainty
and elegant china. There is something intoxicating in all these
preparations for pleasure and festivity when you are not used to them. I
see how they will affect my young pupil. While dinner is going on, while
course follows course, and conversation is loud around us, I whisper in
his ear, ``How many hands do you suppose the things on this table passed
through before they got here?'' What a crowd of ideas is called up by
these few words. In a moment the mists of excitement have rolled away.
He is thinking, considering, calculating, and anxious. The child is
philosophising, while philosophers, excited by wine or perhaps by female
society, are babbling like children. If he asks questions I decline to
answer and put him off to another day. He becomes impatient, he forgets
to eat and drink, he longs to get away from table and talk as he
pleases. What an object of curiosity, what a text for instruction.
Nothing has so far succeeded in corrupting his healthy reason; what will
he think of luxury when he finds that every quarter of the globe has
been ransacked, that some 2,000,000 men have laboured for years, that
many lives have perhaps been sacrificed, and all to furnish him with
fine clothes to be worn at midday and laid by in the wardrobe at night.

Be sure you observe what private conclusions he draws from all his
observations. If you have watched him less carefully than I suppose, his
thoughts may be tempted in another direction; he may consider himself a
person of great importance in the world, when he sees so much labour
concentrated on the preparation of his dinner. If you suspect his
thoughts will take this direction you can easily prevent it, or at any
rate promptly efface the false impression. As yet he can only
appropriate things by personal enjoyment, he can only judge of their
fitness or unfitness by their outward effects. Compare a plain rustic
meal, preceded by exercise, seasoned by hunger, freedom, and delight,
with this magnificent but tedious repast. This will suffice to make him
realise that he has got no real advantage from the splendour of the
feast, that his stomach was as well satisfied when he left the table of
the peasant, as when he left the table of the banker; from neither had
he gained anything he could really call his own.

Just fancy what a tutor might say to him on such an occasion. Consider
the two dinners and decide for yourself which gave you most pleasure,
which seemed the merriest, at which did you eat and drink most heartily,
which was the least tedious and required least change of courses? Yet
note the difference---this black bread you so enjoy is made from the
peasant's own harvest; his wine is dark in colour and of a common kind,
but wholesome and refreshing; it was made in his own vineyard; the cloth
is made of his own hemp, spun and woven in the winter by his wife and
daughters and the maid; no hands but theirs have touched the food. His
world is bounded by the nearest mill and the next market. How far did
you enjoy all that the produce of distant lands and the service of many
people had prepared for you at the other dinner? If you did not get a
better meal, what good did this wealth do you? how much of it was made
for you? Had you been the master of the house, the tutor might say, it
would have been of still less use to you; for the anxiety of displaying
your enjoyment before the eyes of others would have robbed you of it;
the pains would be yours, the pleasure theirs.

This may be a very fine speech, but it would be thrown away upon Emile,
as he cannot understand it, and he does not accept second-hand opinions.
Speak more simply to him. After these two experiences, say to him some
day, ``Where shall we have our dinner to-day? Where that mountain of
silver covered three quarters of the table and those beds of artificial
flowers on looking glass were served with the dessert, where those smart
ladies treated you as a toy and pretended you said what you did not
mean; or in that village two leagues away, with those good people who
were so pleased to see us and gave us such delicious cream?'' Emile will
not hesitate; he is not vain and he is no chatterbox; he cannot endure
constraint, and he does not care for fine dishes; but he is always ready
for a run in the country and is very fond of good fruit and vegetables,
sweet cream and kindly people. {[}Footnote: This taste, which I assume
my pupil to have acquired, is a natural result of his education.
Moreover, he has nothing foppish or affected about him, so that the
ladies take little notice of him and he is less petted than other
children; therefore he does not care for them, and is less spoilt by
their company; he is not yet of an age to feel its charm. I have taken
care not to teach him to kiss their hands, to pay them compliments, or
even to be more polite to them than to men. It is my constant rule to
ask nothing from him but what he can understand, and there is no good
reason why a child should treat one sex differently from the other.{]}
On our way, the thought will occur to him, ``All those people who
laboured to prepare that grand feast were either wasting their time or
they have no idea how to enjoy themselves.''

My example may be right for one child and wrong for the rest. If you
enter into their way of looking at things you will know how to vary your
instances as required; the choice depends on the study of the individual
temperament, and this study in turn depends on the opportunities which
occur to show this temperament. You will not suppose that, in the three
or four years at our disposal, even the most gifted child can get an
idea of all the arts and sciences, sufficient to enable him to study
them for himself when he is older; but by bringing before him what he
needs to know, we enable him to develop his own tastes, his own talents,
to take the first step towards the object which appeals to his
individuality and to show us the road we must open up to aid the work of
nature.

There is another advantage of these trains of limited but exact bits of
knowledge; he learns by their connection and interdependence how to rank
them in his own estimation and to be on his guard against those
prejudices, common to most men, which draw them towards the gifts they
themselves cultivate and away from those they have neglected. The man
who clearly sees the whole, sees where each part should be; the man who
sees one part clearly and knows it thoroughly may be a learned man, but
the former is a wise man, and you remember it is wisdom rather than
knowledge that we hope to acquire.

However that may be, my method does not depend on my examples; it
depends on the amount of a man's powers at different ages, and the
choice of occupations adapted to those powers. I think it would be easy
to find a method which appeared to give better results, but if it were
less suited to the type, sex, and age of the scholar, I doubt whether
the results would really be as good.

At the beginning of this second period we took advantage of the fact
that our strength was more than enough for our needs, to enable us to
get outside ourselves. We have ranged the heavens and measured the
earth; we have sought out the laws of nature; we have explored the whole
of our island. Now let us return to ourselves, let us unconsciously
approach our own dwelling. We are happy indeed if we do not find it
already occupied by the dreaded foe, who is preparing to seize it.

What remains to be done when we have observed all that lies around us?
We must turn to our own use all that we can get, we must increase our
comfort by means of our curiosity. Hitherto we have provided ourselves
with tools of all kinds, not knowing which we require. Perhaps those we
do not want will be useful to others, and perhaps we may need theirs.
Thus we discover the use of exchange; but for this we must know each
other's needs, what tools other people use, what they can offer in
exchange. Given ten men, each of them has ten different requirements. To
get what he needs for himself each must work at ten different trades;
but considering our different talents, one will do better at this trade,
another at that. Each of them, fitted for one thing, will work at all,
and will be badly served. Let us form these ten men into a society, and
let each devote himself to the trade for which he is best adapted, and
let him work at it for himself and for the rest. Each will reap the
advantage of the others' talents, just as if they were his own; by
practice each will perfect his own talent, and thus all the ten, well
provided for, will still have something to spare for others. This is the
plain foundation of all our institutions. It is not my aim to examine
its results here; I have done so in another book (Discours sur
l'inegalite).

According to this principle, any one who wanted to consider himself as
an isolated individual, self-sufficing and independent of others, could
only be utterly wretched. He could not even continue to exist, for
finding the whole earth appropriated by others while he had only
himself, how could he get the means of subsistence? When we leave the
state of nature we compel others to do the same; no one can remain in a
state of nature in spite of his fellow-creatures, and to try to remain
in it when it is no longer practicable, would really be to leave it, for
self-preservation is nature's first law.

Thus the idea of social relations is gradually developed in the child's
mind, before he can really be an active member of human society. Emile
sees that to get tools for his own use, other people must have theirs,
and that he can get in exchange what he needs and they possess. I easily
bring him to feel the need of such exchange and to take advantage of it.

``Sir, I must live,'' said a miserable writer of lampoons to the
minister who reproved him for his infamous trade. ``I do not see the
necessity,'' replied the great man coldly. This answer, excellent from
the minister, would have been barbarous and untrue in any other mouth.
Every man must live; this argument, which appeals to every one with more
or less force in proportion to his humanity, strikes me as unanswerable
when applied to oneself. Since our dislike of death is the strongest of
those aversions nature has implanted in us, it follows that everything
is permissible to the man who has no other means of living. The
principles, which teach the good man to count his life a little thing
and to sacrifice it at duty's call, are far removed from this primitive
simplicity. Happy are those nations where one can be good without
effort, and just without conscious virtue. If in this world there is any
condition so miserable that one cannot live without wrong-doing, where
the citizen is driven into evil, you should hang, not the criminal, but
those who drove him into crime.

As soon as Emile knows what life is, my first care will be to teach him
to preserve his life. Hitherto I have made no distinction of condition,
rank, station, or fortune; nor shall I distinguish between them in the
future, since man is the same in every station; the rich man's stomach
is no bigger than the poor man's, nor is his digestion any better; the
master's arm is neither longer nor stronger than the slave's; a great
man is no taller than one of the people, and indeed the natural needs
are the same to all, and the means of satisfying them should be equally
within the reach of all. Fit a man's education to his real self, not to
what is no part of him. Do you not see that in striving to fit him
merely for one station, you are unfitting him for anything else, so that
some caprice of Fortune may make your work really harmful to him? What
could be more absurd than a nobleman in rags, who carries with him into
his poverty the prejudices of his birth? What is more despicable than a
rich man fallen into poverty, who recalls the scorn with which he
himself regarded the poor, and feels that he has sunk to the lowest
depth of degradation? The one may become a professional thief, the other
a cringing servant, with this fine saying, ``I must live.''

You reckon on the present order of society, without considering that
this order is itself subject to inscrutable changes, and that you can
neither foresee nor provide against the revolution which may affect your
children. The great become small, the rich poor, the king a commoner.
Does fate strike so seldom that you can count on immunity from her
blows? The crisis is approaching, and we are on the edge of a
revolution. {[}Footnote: In my opinion it is impossible that the great
kingdoms of Europe should last much longer. Each of them has had its
period of splendour, after which it must inevitably decline. I have my
own opinions as to the special applications of this general statement,
but this is not the place to enter into details, and they are only too
evident to everybody.{]} Who can answer for your fate? What man has
made, man may destroy. Nature's characters alone are ineffaceable, and
nature makes neither the prince, the rich man, nor the nobleman. This
satrap whom you have educated for greatness, what will become of him in
his degradation? This farmer of the taxes who can only live on gold,
what will he do in poverty? This haughty fool who cannot use his own
hands, who prides himself on what is not really his, what will he do
when he is stripped of all? In that day, happy will he be who can give
up the rank which is no longer his, and be still a man in Fate's
despite. Let men praise as they will that conquered monarch who like a
madman would be buried beneath the fragments of his throne; I behold him
with scorn; to me he is merely a crown, and when that is gone he is
nothing. But he who loses his crown and lives without it, is more than a
king; from the rank of a king, which may be held by a coward, a villain,
or madman, he rises to the rank of a man, a position few can fill. Thus
he triumphs over Fortune, he dares to look her in the face; he depends
on himself alone, and when he has nothing left to show but himself he is
not a nonentity, he is somebody. Better a thousandfold the king of
Corinth a schoolmaster at Syracuse, than a wretched Tarquin, unable to
be anything but a king, or the heir of the ruler of three kingdoms, the
sport of all who would scorn his poverty, wandering from court to court
in search of help, and finding nothing but insults, for want of knowing
any trade but one which he can no longer practise.

The man and the citizen, whoever he may be, has no property to invest in
society but himself, all his other goods belong to society in spite of
himself, and when a man is rich, either he does not enjoy his wealth, or
the public enjoys it too; in the first case he robs others as well as
himself; in the second he gives them nothing. Thus his debt to society
is still unpaid, while he only pays with his property. ``But my father
was serving society while he was acquiring his wealth.'' Just so; he
paid his own debt, not yours. You owe more to others than if you had
been born with nothing, since you were born under favourable conditions.
It is not fair that what one man has done for society should pay
another's debt, for since every man owes all that he is, he can only pay
his own debt, and no father can transmit to his son any right to be of
no use to mankind. ``But,'' you say, ``this is just what he does when he
leaves me his wealth, the reward of his labour.'' The man who eats in
idleness what he has not himself earned, is a thief, and in my eyes, the
man who lives on an income paid him by the state for doing nothing,
differs little from a highwayman who lives on those who travel his way.
Outside the pale of society, the solitary, owing nothing to any man, may
live as he pleases, but in society either he lives at the cost of
others, or he owes them in labour the cost of his keep; there is no
exception to this rule. Man in society is bound to work; rich or poor,
weak or strong, every idler is a thief.

Now of all the pursuits by which a man may earn his living, the nearest
to a state of nature is manual labour; of all stations that of the
artisan is least dependent on Fortune. The artisan depends on his labour
alone, he is a free man while the ploughman is a slave; for the latter
depends on his field where the crops may be destroyed by others. An
enemy, a prince, a powerful neighbour, or a law-suit may deprive him of
his field; through this field he may be harassed in all sorts of ways.
But if the artisan is ill-treated his goods are soon packed and he takes
himself off. Yet agriculture is the earliest, the most honest of trades,
and more useful than all the rest, and therefore more honourable for
those who practise it. I do not say to Emile, ``Study agriculture,'' he
is already familiar with it. He is acquainted with every kind of rural
labour, it was his first occupation, and he returns to it continually.
So I say to him, ``Cultivate your father's lands, but if you lose this
inheritance, or if you have none to lose, what will you do? Learn a
trade.''

``A trade for my son! My son a working man! What are you thinking of,
sir?'' Madam, my thoughts are wiser than yours; you want to make him fit
for nothing but a lord, a marquis, or a prince; and some day he may be
less than nothing. I want to give him a rank which he cannot lose, a
rank which will always do him honour; I want to raise him to the status
of a man, and, whatever you may say, he will have fewer equals in that
rank than in your own.

The letter killeth, the spirit giveth life. Learning a trade matters
less than overcoming the prejudices he despises. You will never be
reduced to earning your livelihood; so much the worse for you. No
matter; work for honour, not for need: stoop to the position of a
working man, to rise above your own. To conquer Fortune and everything
else, begin by independence. To rule through public opinion, begin by
ruling over it.

Remember I demand no talent, only a trade, a genuine trade, a mere
mechanical art, in which the hands work harder than the head, a trade
which does not lead to fortune but makes you independent of her. In
households far removed from all danger of want I have known fathers
carry prudence to such a point as to provide their children not only
with ordinary teaching but with knowledge by means of which they could
get a living if anything happened. These far-sighted parents thought
they were doing a great thing. It is nothing, for the resources they
fancy they have secured depend on that very fortune of which they would
make their children independent; so that unless they found themselves in
circumstances fitted for the display of their talents, they would die of
hunger as if they had none.

As soon as it is a question of influence and intrigue you may as well
use these means to keep yourself in plenty, as to acquire, in the depths
of poverty, the means of returning to your former position. If you
cultivate the arts which depend on the artist's reputation, if you fit
yourself for posts which are only obtained by favour, how will that help
you when, rightly disgusted with the world, you scorn the steps by which
you must climb. You have studied politics and state-craft, so far so
good; but how will you use this knowledge, if you cannot gain the ear of
the ministers, the favourites, or the officials? if you have not the
secret of winning their favour, if they fail to find you a rogue to
their taste? You are an architect or a painter; well and good; but your
talents must be displayed. Do you suppose you can exhibit in the salon
without further ado? That is not the way to set about it. Lay aside the
rule and the pencil, take a cab and drive from door to door; there is
the road to fame. Now you must know that the doors of the great are
guarded by porters and flunkeys, who only understand one language, and
their ears are in their palms. If you wish to teach what you have
learned, geography, mathematics, languages, music, drawing, even to find
pupils, you must have friends who will sing your praises. Learning,
remember, gains more credit than skill, and with no trade but your own
none will believe in your skill. See how little you can depend on these
fine ``Resources,'' and how many other resources are required before you
can use what you have got. And what will become of you in your
degradation? Misfortune will make you worse rather than better. More
than ever the sport of public opinion, how will you rise above the
prejudices on which your fate depends? How will you despise the vices
and the baseness from which you get your living? You were dependent on
wealth, now you are dependent on the wealthy; you are still a slave and
a poor man into the bargain. Poverty without freedom, can a man sink
lower than this!

But if instead of this recondite learning adapted to feed the mind, not
the body, you have recourse, at need, to your hands and your handiwork,
there is no call for deceit, your trade is ready when required. Honour
and honesty will not stand in the way of your living. You need no longer
cringe and lie to the great, nor creep and crawl before rogues, a
despicable flatterer of both, a borrower or a thief, for there is little
to choose between them when you are penniless. Other people's opinions
are no concern of yours, you need not pay court to any one, there is no
fool to flatter, no flunkey to bribe, no woman to win over. Let rogues
conduct the affairs of state; in your lowly rank you can still be an
honest man and yet get a living. You walk into the first workshop of
your trade. ``Master, I want work.'' ``Comrade, take your place and
work.'' Before dinner-time you have earned your dinner. If you are sober
and industrious, before the week is out you will have earned your keep
for another week; you will have lived in freedom, health, truth,
industry, and righteousness. Time is not wasted when it brings these
returns.

Emile shall learn a trade. ``An honest trade, at least,'' you say. What
do you mean by honest? Is not every useful trade honest? I would not
make an embroiderer, a gilder, a polisher of him, like Locke's young
gentleman. Neither would I make him a musician, an actor, or an
author.{[}Footnote: You are an author yourself, you will reply. Yes, for
my sins; and my ill deeds, which I think I have fully expiated, are no
reason why others should be like me. I do not write to excuse my faults,
but to prevent my readers from copying them.{]} With the exception of
these and others like them, let him choose his own trade, I do not mean
to interfere with his choice. I would rather have him a shoemaker than a
poet, I would rather he paved streets than painted flowers on china.
``But,'' you will say, ``policemen, spies, and hangmen are useful
people.'' There would be no use for them if it were not for the
government. But let that pass. I was wrong. It is not enough to choose
an honest trade, it must be a trade which does not develop detestable
qualities in the mind, qualities incompatible with humanity. To return
to our original expression, ``Let us choose an honest trade,'' but let
us remember there can be no honesty without usefulness.

A famous writer of this century, whose books are full of great schemes
and narrow views, was under a vow, like the other priests of his
communion, not to take a wife. Finding himself more scrupulous than
others with regard to his neighbour's wife, he decided, so they say, to
employ pretty servants, and so did his best to repair the wrong done to
the race by his rash promise. He thought it the duty of a citizen to
breed children for the state, and he made his children artisans. As soon
as they were old enough they were taught whatever trade they chose; only
idle or useless trades were excluded, such as that of the wigmaker who
is never necessary, and may any day cease to be required, so long as
nature does not get tired of providing us with hair.

This spirit shall guide our choice of trade for Emile, or rather, not
our choice but his; for the maxims he has imbibed make him despise
useless things, and he will never be content to waste his time on vain
labours; his trade must be of use to Robinson on his island.

When we review with the child the productions of art and nature, when we
stimulate his curiosity and follow its lead, we have great opportunities
of studying his tastes and inclinations, and perceiving the first spark
of genius, if he has any decided talent in any direction. You must,
however, be on your guard against the common error which mistakes the
effects of environment for the ardour of genius, or imagines there is a
decided bent towards any one of the arts, when there is nothing more
than that spirit of emulation, common to men and monkeys, which impels
them instinctively to do what they see others doing, without knowing
why. The world is full of artisans, and still fuller of artists, who
have no native gift for their calling, into which they were driven in
early childhood, either through the conventional ideas of other people,
or because those about them were deceived by an appearance of zeal,
which would have led them to take to any other art they saw practised.
One hears a drum and fancies he is a general; another sees a building
and wants to be an architect. Every one is drawn towards the trade he
sees before him if he thinks it is held in honour.

I once knew a footman who watched his master drawing and painting and
took it into his head to become a designer and artist. He seized a
pencil which he only abandoned for a paint-brush, to which he stuck for
the rest of his days. Without teaching or rules of art he began to draw
everything he saw. Three whole years were devoted to these daubs, from
which nothing but his duties could stir him, nor was he discouraged by
the small progress resulting from his very mediocre talents. I have seen
him spend the whole of a broiling summer in a little ante-room towards
the south, a room where one was suffocated merely passing through it;
there he was, seated or rather nailed all day to his chair, before a
globe, drawing it again and again and yet again, with invincible
obstinacy till he had reproduced the rounded surface to his own
satisfaction. At last with his master's help and under the guidance of
an artist he got so far as to abandon his livery and live by his brush.
Perseverance does instead of talent up to a certain point; he got so
far, but no further. This honest lad's perseverance and ambition are
praiseworthy; he will always be respected for his industry and
steadfastness of purpose, but his paintings will always be third-rate.
Who would not have been deceived by his zeal and taken it for real
talent! There is all the difference in the world between a liking and an
aptitude. To make sure of real genius or real taste in a child calls for
more accurate observations than is generally suspected, for the child
displays his wishes not his capacity, and we judge by the former instead
of considering the latter. I wish some trustworthy person would give us
a treatise on the art of child-study. This art is well worth studying,
but neither parents nor teachers have mastered its elements.

Perhaps we are laying too much stress on the choice of a trade; as it is
a manual occupation, Emile's choice is no great matter, and his
apprenticeship is more than half accomplished already, through the
exercises which have hitherto occupied him. What would you have him do?
He is ready for anything. He can handle the spade and hoe, he can use
the lathe, hammer, plane, or file; he is already familiar with these
tools which are common to many trades. He only needs to acquire
sufficient skill in the use of any one of them to rival the speed, the
familiarity, and the diligence of good workmen, and he will have a great
advantage over them in suppleness of body and limb, so that he can
easily take any position and can continue any kind of movements without
effort. Moreover his senses are acute and well-practised, he knows the
principles of the various trades; to work like a master of his craft he
only needs experience, and experience comes with practice. To which of
these trades which are open to us will he give sufficient time to make
himself master of it? That is the whole question.

Give a man a trade befitting his sex, to a young man a trade befitting
his age. Sedentary indoor employments, which make the body tender and
effeminate, are neither pleasing nor suitable. No lad ever wanted to be
a tailor. It takes some art to attract a man to this woman's
work.{[}Footnote: There were no tailors among the ancients; men's
clothes were made at home by the women.{]} The same hand cannot hold the
needle and the sword. If I were king I would only allow needlework and
dressmaking to be done by women and cripples who are obliged to work at
such trades. If eunuchs were required I think the Easterns were very
foolish to make them on purpose. Why not take those provided by nature,
that crowd of base persons without natural feeling? There would be
enough and to spare. The weak, feeble, timid man is condemned by nature
to a sedentary life, he is fit to live among women or in their fashion.
Let him adopt one of their trades if he likes; and if there must be
eunuchs let them take those men who dishonour their sex by adopting
trades unworthy of it. Their choice proclaims a blunder on the part of
nature; correct it one way or other, you will do no harm.

An unhealthy trade I forbid to my pupil, but not a difficult or
dangerous one. He will exercise himself in strength and courage; such
trades are for men not women, who claim no share in them. Are not men
ashamed to poach upon the women's trades?

\aquote{``Luctantur paucae, comedunt coliphia paucae. \\
Vos lanam trahitis, calathisque peracta refertis\\
Vellera."}{Juven. Sat. II. V. 55.}

Women are not seen in shops in Italy, and to persons accustomed to the
streets of England and France nothing could look gloomier. When I saw
drapers selling ladies ribbons, pompons, net, and chenille, I thought
these delicate ornaments very absurd in the coarse hands fit to blow the
bellows and strike the anvil. I said to myself, ``In this country women
should set up as steel-polishers and armourers.'' Let each make and sell
the weapons of his or her own sex; knowledge is acquired through use.

I know I have said too much for my agreeable contemporaries, but I
sometimes let myself be carried away by my argument. If any one is
ashamed to be seen wearing a leathern apron or handling a plane, I think
him a mere slave of public opinion, ready to blush for what is right
when people poke fun at it. But let us yield to parents' prejudices so
long as they do not hurt the children. To honour trades we are not
obliged to practise every one of them, so long as we do not think them
beneath us. When the choice is ours and we are under no compulsion, why
not choose the pleasanter, more attractive and more suitable trade.
Metal work is useful, more useful, perhaps, than the rest, but unless
for some special reason Emile shall not be a blacksmith, a locksmith nor
an iron-worker. I do not want to see him a Cyclops at the forge. Neither
would I have him a mason, still less a shoemaker. All trades must be
carried on, but when the choice is ours, cleanliness should be taken
into account; this is not a matter of class prejudice, our senses are
our guides. In conclusion, I do not like those stupid trades in which
the workmen mechanically perform the same action without pause and
almost without mental effort. Weaving, stocking-knitting, stone-cutting;
why employ intelligent men on such work? it is merely one machine
employed on another.

All things considered, the trade I should choose for my pupil, among the
trades he likes, is that of a carpenter. It is clean and useful; it may
be carried on at home; it gives enough exercise; it calls for skill and
industry, and while fashioning articles for everyday use, there is scope
for elegance and taste. If your pupil's talents happened to take a
scientific turn, I should not blame you if you gave him a trade in
accordance with his tastes, for instance, he might learn to make
mathematical instruments, glasses, telescopes, etc.

When Emile learns his trade I shall learn it too. I am convinced he will
never learn anything thoroughly unless we learn it together. So we shall
both serve our apprenticeship, and we do not mean to be treated as
gentlemen, but as real apprentices who are not there for fun; why should
not we actually be apprenticed? Peter the Great was a ship's carpenter
and drummer to his own troops; was not that prince at least your equal
in birth and merit? You understand this is addressed not to Emile but to
you---to you, whoever you may be.

Unluckily we cannot spend the whole of our time at the workshop. We are
not only 'prentice-carpenters but 'prentice-men---a trade whose
apprenticeship is longer and more exacting than the rest. What shall we
do? Shall we take a master to teach us the use of the plane and engage
him by the hour like the dancing-master? In that case we should be not
apprentices but students, and our ambition is not merely to learn
carpentry but to be carpenters. Once or twice a week I think we should
spend the whole day at our master's; we should get up when he does, we
should be at our work before him, we should take our meals with him,
work under his orders, and after having had the honour of supping at his
table we may if we please return to sleep upon our own hard beds. This
is the way to learn several trades at once, to learn to do manual work
without neglecting our apprenticeship to life.

Let us do what is right without ostentation; let us not fall into vanity
through our efforts to resist it. To pride ourselves on our victory over
prejudice is to succumb to prejudice. It is said that in accordance with
an old custom of the Ottomans, the sultan is obliged to work with his
hands, and, as every one knows, the handiwork of a king is a
masterpiece. So he royally distributes his masterpieces among the great
lords of the Porte and the price paid is in accordance with the rank of
the workman. It is not this so-called abuse to which I object; on the
contrary, it is an advantage, and by compelling the lords to share with
him the spoils of the people it is so much the less necessary for the
prince to plunder the people himself. Despotism needs some such
relaxation, and without it that hateful rule could not last.

The real evil in such a custom is the idea it gives that poor man of his
own worth. Like King Midas he sees all things turn to gold at his touch,
but he does not see the ass' ears growing. Let us keep Emile's hands
from money lest he should become an ass, let him take the work but not
the wages. Never let his work be judged by any standard but that of the
work of a master. Let it be judged as work, not because it is his. If
anything is well done, I say, ``That is a good piece of work,'' but do
not ask who did it. If he is pleased and proud and says, ``I did it,''
answer indifferently, ``No matter who did it, it is well done.''

Good mother, be on your guard against the deceptions prepared for you.
If your son knows many things, distrust his knowledge; if he is unlucky
enough to be rich and educated in Paris he is ruined. As long as there
are clever artists he will have every talent, but apart from his masters
he will have none. In Paris a rich man knows everything, it is the poor
who are ignorant. Our capital is full of amateurs, especially women, who
do their work as M. Gillaume invents his colours. Among the men I know
three striking exceptions, among the women I know no exceptions, and I
doubt if there are any. In a general way a man becomes an artist and a
judge of art as he becomes a Doctor of Laws and a magistrate.

If then it is once admitted that it is a fine thing to have a trade,
your children would soon have one without learning it. They would become
postmasters like the councillors of Zurich. Let us have no such
ceremonies for Emile; let it be the real thing not the sham. Do not say
what he knows, let him learn in silence. Let him make his masterpiece,
but not be hailed as master; let him be a workman not in name but in
deed.

If I have made my meaning clear you ought to realise how bodily exercise
and manual work unconsciously arouse thought and reflexion in my pupil,
and counteract the idleness which might result from his indifference to
men's judgments, and his freedom from passion. He must work like a
peasant and think like a philosopher, if he is not to be as idle as a
savage. The great secret of education is to use exercise of mind and
body as relaxation one to the other.

But beware of anticipating teaching which demands more maturity of mind.
Emile will not long be a workman before he discovers those social
inequalities he had not previously observed. He will want to question me
in turn on the maxims I have given him, maxims he is able to understand.
When he derives everything from me, when he is so nearly in the position
of the poor, he will want to know why I am so far removed from it. All
of a sudden he may put scathing questions to me. ``You are rich, you
tell me, and I see you are. A rich man owes his work to the community
like the rest because he is a man. What are you doing for the
community?'' What would a fine tutor say to that? I do not know. He
would perhaps be foolish enough to talk to the child of the care he
bestows upon him. The workshop will get me out of the difficulty. ``My
dear Emile that is a very good question; I will undertake to answer for
myself, when you can answer for yourself to your own satisfaction.
Meanwhile I will take care to give what I can spare to you and to the
poor, and to make a table or a bench every week, so as not to be quite
useless.''

We have come back to ourselves. Having entered into possession of
himself, our child is now ready to cease to be a child. He is more than
ever conscious of the necessity which makes him dependent on things.
After exercising his body and his senses you have exercised his mind and
his judgment. Finally we have joined together the use of his limbs and
his faculties. We have made him a worker and a thinker; we have now to
make him loving and tender-hearted, to perfect reason through feeling.
But before we enter on this new order of things, let us cast an eye over
the stage we are leaving behind us, and perceive as clearly as we can
how far we have got.

At first our pupil had merely sensations, now he has ideas; he could
only feel, now he reasons. For from the comparison of many successive or
simultaneous sensations and the judgment arrived at with regard to them,
there springs a sort of mixed or complex sensation which I call an idea.

The way in which ideas are formed gives a character to the human mind.
The mind which derives its ideas from real relations is thorough; the
mind which relies on apparent relations is superficial. He who sees
relations as they are has an exact mind; he who fails to estimate them
aright has an inaccurate mind; he who concocts imaginary relations,
which have no real existence, is a madman; he who does not perceive any
relation at all is an imbecile. Clever men are distinguished from others
by their greater or less aptitude for the comparison of ideas and the
discovery of relations between them.

Simple ideas consist merely of sensations compared one with another.
Simple sensations involve judgments, as do the complex sensations which
I call simple ideas. In the sensation the judgment is purely passive; it
affirms that I feel what I feel. In the percept or idea the judgment is
active; it connects, compares, it discriminates between relations not
perceived by the senses. That is the whole difference; but it is a great
difference. Nature never deceives us; we deceive ourselves.

I see some one giving an ice-cream to an eight-year-old child; he does
not know what it is and puts the spoon in his mouth. Struck by the cold
he cries out, ``Oh, it burns!'' He feels a very keen sensation, and the
heat of the fire is the keenest sensation he knows, so he thinks that is
what he feels. Yet he is mistaken; cold hurts, but it does not burn; and
these two sensations are different, for persons with more experience do
not confuse them. So it is not the sensation that is wrong, but the
judgment formed with regard to it.

It is just the same with those who see a mirror or some optical
instrument for the first time, or enter a deep cellar in the depths of
winter or at midsummer, or dip a very hot or cold hand into tepid water,
or roll a little ball between two crossed fingers. If they are content
to say what they really feel, their judgment, being purely passive,
cannot go wrong; but when they judge according to appearances, their
judgment is active; it compares and establishes by induction relations
which are not really perceived. Then these inductions may or may not be
mistaken. Experience is required to correct or prevent error.

Show your pupil the clouds at night passing between himself and the
moon; he will think the moon is moving in the opposite direction and
that the clouds are stationary. He will think this through a hasty
induction, because he generally sees small objects moving and larger
ones at rest, and the clouds seems larger than the moon, whose distance
is beyond his reckoning. When he watches the shore from a moving boat he
falls into the opposite mistake and thinks the earth is moving because
he does not feel the motion of the boat and considers it along with the
sea or river as one motionless whole, of which the shore, which appears
to move, forms no part.

The first time a child sees a stick half immersed in water he thinks he
sees a broken stick; the sensation is true and would not cease to be
true even if he knew the reason of this appearance. So if you ask him
what he sees, he replies, ``A broken stick,'' for he is quite sure he is
experiencing this sensation. But when deceived by his judgment he goes
further and, after saying he sees a broken stick, he affirms that it
really is broken he says what is not true. Why? Because he becomes
active and judges no longer by observation but by induction, he affirms
what he does not perceive, i.e., that the judgment he receives through
one of his senses would be confirmed by another.

Since all our errors arise in our judgment, it is clear, that had we no
need for judgment, we should not need to learn; we should never be
liable to mistakes, we should be happier in our ignorance than we can be
in our knowledge. Who can deny that a vast number of things are known to
the learned, which the unlearned will never know? Are the learned any
nearer truth? Not so, the further they go the further they get from
truth, for their pride in their judgment increases faster than their
progress in knowledge, so that for every truth they acquire they draw a
hundred mistaken conclusions. Every one knows that the learned societies
of Europe are mere schools of falsehood, and there are assuredly more
mistaken notions in the Academy of Sciences than in a whole tribe of
American Indians.

The more we know, the more mistakes we make; therefore ignorance is the
only way to escape error. Form no judgments and you will never be
mistaken. This is the teaching both of nature and reason. We come into
direct contact with very few things, and these are very readily
perceived; the rest we regard with profound indifference. A savage will
not turn his head to watch the working of the finest machinery or all
the wonders of electricity. ``What does that matter to me?'' is the
common saying of the ignorant; it is the fittest phrase for the wise.

Unluckily this phrase will no longer serve our turn. Everything matters
to us, as we are dependent on everything, and our curiosity naturally
increases with our needs. This is why I attribute much curiosity to the
man of science and none to the savage. The latter needs no help from
anybody; the former requires every one, and admirers most of all.

You will tell me I am going beyond nature. I think not. She chooses her
instruments and orders them, not according to fancy, but necessity. Now
a man's needs vary with his circumstances. There is all the difference
in the world between a natural man living in a state of nature, and a
natural man living in society. Emile is no savage to be banished to the
desert, he is a savage who has to live in the town. He must know how to
get his living in a town, how to use its inhabitants, and how to live
among them, if not of them.

In the midst of so many new relations and dependent on them, he must
reason whether he wants to or no. Let us therefore teach him to reason
correctly.

The best way of learning to reason aright is that which tends to
simplify our experiences, or to enable us to dispense with them
altogether without falling into error. Hence it follows that we must
learn to confirm the experiences of each sense by itself, without
recourse to any other, though we have been in the habit of verifying the
experience of one sense by that of another. Then each of our sensations
will become an idea, and this idea will always correspond to the truth.
This is the sort of knowledge I have tried to accumulate during this
third phase of man's life.

This method of procedure demands a patience and circumspection which few
teachers possess; without them the scholar will never learn to reason.
For example, if you hasten to take the stick out of the water when the
child is deceived by its appearance, you may perhaps undeceive him, but
what have you taught him? Nothing more than he would soon have learnt
for himself. That is not the right thing to do. You have not got to
teach him truths so much as to show him how to set about discovering
them for himself. To teach him better you must not be in such a hurry to
correct his mistakes. Let us take Emile and myself as an illustration.

To begin with, any child educated in the usual way could not fail to
answer the second of my imaginary questions in the affirmative. He will
say, ``That is certainly a broken stick.'' I very much doubt whether
Emile will give the same reply. He sees no reason for knowing everything
or pretending to know it; he is never in a hurry to draw conclusions. He
only reasons from evidence and on this occasion he has not got the
evidence. He knows how appearances deceive us, if only through
perspective.

Moreover, he knows by experience that there is always a reason for my
slightest questions, though he may not see it at once; so he has not got
into the habit of giving silly answers; on the contrary, he is on his
guard, he considers things carefully and attentively before answering.
He never gives me an answer unless he is satisfied with it himself, and
he is hard to please. Lastly we neither of us take any pride in merely
knowing a thing, but only in avoiding mistakes. We should be more
ashamed to deceive ourselves with bad reasoning, than to find no
explanation at all. There is no phrase so appropriate to us, or so often
on our lips, as, ``I do not know;'' neither of us are ashamed to use it.
But whether he gives the silly answer or whether he avoids it by our
convenient phrase ``I do not know,'' my answer is the same. ``Let us
examine it.''

This stick immersed half way in the water is fixed in an upright
position. To know if it is broken, how many things must be done before
we take it out of the water or even touch it.

1. First we walk round it, and we see that the broken part follows us.
So it is only our eye that changes it; looks do not make things move.

2. We look straight down on that end of the stick which is above the
water, the stick is no longer bent, {[}Footnote: I have since found by
more exact experiment that this is not the case. Refraction acts in a
circle, and the stick appears larger at the end which is in the water,
but this makes no difference to the strength of the argument, and the
conclusion is correct.{]} the end near our eye exactly hides the other
end. Has our eye set the stick straight?

3. We stir the surface of the water; we see the stick break into several
pieces, it moves in zigzags and follows the ripples of the water. Can
the motion we gave the water suffice to break, soften, or melt the stick
like this?

4. We draw the water off, and little by little we see the stick
straightening itself as the water sinks. Is not this more than enough to
clear up the business and to discover refraction? So it is not true that
our eyes deceive us, for nothing more has been required to correct the
mistakes attributed to it.

Suppose the child were stupid enough not to perceive the result of these
experiments, then you must call touch to the help of sight. Instead of
taking the stick out of the water, leave it where it is and let the
child pass his hand along it from end to end; he will feel no angle,
therefore the stick is not broken.

You will tell me this is not mere judgment but formal reasoning. Just
so; but do not you see that as soon as the mind has got any ideas at
all, every judgment is a process of reasoning? So that as soon as we
compare one sensation with another, we are beginning to reason. The art
of judging and the art of reasoning are one and the same.

Emile will never learn dioptrics unless he learns with this stick. He
will not have dissected insects nor counted the spots on the sun; he
will not know what you mean by a microscope or a telescope. Your learned
pupils will laugh at his ignorance and rightly, I intend him to invent
these instruments before he uses them, and you will expect that to take
some time.

This is the spirit of my whole method at this stage. If the child rolls
a little ball between two crossed fingers and thinks he feels two balls,
I shall not let him look until he is convinced there is only one.

This explanation will suffice, I hope, to show plainly the progress made
by my pupil hitherto and the route followed by him. But perhaps the
number of things I have brought to his notice alarms you. I shall crush
his mind beneath this weight of knowledge. Not so, I am rather teaching
him to be ignorant of things than to know them. I am showing him the
path of science, easy indeed, but long, far-reaching and slow to follow.
I am taking him a few steps along this path, but I do not allow him to
go far.

Compelled to learn for himself, he uses his own reason not that of
others, for there must be no submission to authority if you would have
no submission to convention. Most of our errors are due to others more
than ourselves. This continual exercise should develop a vigour of mind
like that acquired by the body through labour and weariness. Another
advantage is that his progress is in proportion to his strength, neither
mind nor body carries more than it can bear. When the understanding lays
hold of things before they are stored in the memory, what is drawn from
that store is his own; while we are in danger of never finding anything
of our own in a memory over-burdened with undigested knowledge.

Emile knows little, but what he knows is really his own; he has no
half-knowledge. Among the few things he knows and knows thoroughly this
is the most valuable, that there are many things he does not know now
but may know some day, many more that other men know but he will never
know, and an infinite number which nobody will ever know. He is
large-minded, not through knowledge, but through the power of acquiring
it; he is open-minded, intelligent, ready for anything, and, as
Montaigne says, capable of learning if not learned. I am content if he
knows the ``Wherefore'' of his actions and the ``Why'' of his beliefs.
For once more my object is not to supply him with exact knowledge, but
the means of getting it when required, to teach him to value it at its
true worth, and to love truth above all things. By this method progress
is slow but sure, and we never need to retrace our steps.

Emile's knowledge is confined to nature and things. The very name of
history is unknown to him, along with metaphysics and morals. He knows
the essential relations between men and things, but nothing of the moral
relations between man and man. He has little power of generalisation, he
has no skill in abstraction. He perceives that certain qualities are
common to certain things, without reasoning about these qualities
themselves. He is acquainted with the abstract idea of space by the help
of his geometrical figures; he is acquainted with the abstract idea of
quantity by the help of his algebraical symbols. These figures and signs
are the supports on which these ideas may be said to rest, the supports
on which his senses repose. He does not attempt to know the nature of
things, but only to know things in so far as they affect himself. He
only judges what is outside himself in relation to himself, and his
judgment is exact and certain. Caprice and prejudice have no part in it.
He values most the things which are of use to himself, and as he never
departs from this standard of values, he owes nothing to prejudice.

Emile is industrious, temperate, patient, stedfast, and full of courage.
His imagination is still asleep, so he has no exaggerated ideas of
danger; the few ills he feels he knows how to endure in patience,
because he has not learnt to rebel against fate. As to death, he knows
not what it means; but accustomed as he is to submit without resistance
to the law of necessity, he will die, if die he must, without a groan
and without a struggle; that is as much as we can demand of nature, in
that hour which we all abhor. To live in freedom, and to be independent
of human affairs, is the best way to learn how to die.

In a word Emile is possessed of all that portion of virtue which
concerns himself. To acquire the social virtues he only needs a
knowledge of the relations which make those virtues necessary; he only
lacks knowledge which he is quite ready to receive.

He thinks not of others but of himself, and prefers that others should
do the same. He makes no claim upon them, and acknowledges no debt to
them. He is alone in the midst of human society, he depends on himself
alone, for he is all that a boy can be at his age. He has no errors, or
at least only such as are inevitable; he has no vices, or only those
from which no man can escape. His body is healthy, his limbs are supple,
his mind is accurate and unprejudiced, his heart is free and untroubled
by passion. Pride, the earliest and the most natural of passions, has
scarcely shown itself. Without disturbing the peace of others, he has
passed his life contented, happy, and free, so far as nature allows. Do
you think that the earlier years of a child, who has reached his
fifteenth year in this condition, have been wasted?

\mychapter{5}{Book IV}

How swiftly life passes here below! The first quarter of it is gone
before we know how to use it; the last quarter finds us incapable of
enjoying life. At first we do not know how to live; and when we know how
to live it is too late. In the interval between these two useless
extremes we waste three-fourths of our time sleeping, working,
sorrowing, enduring restraint and every kind of suffering. Life is
short, not so much because of the short time it lasts, but because we
are allowed scarcely any time to enjoy it. In vain is there a long
interval between the hour of death and that of birth; life is still too
short, if this interval is not well spent.

We are born, so to speak, twice over; born into existence, and born into
life; born a human being, and born a man. Those who regard woman as an
imperfect man are no doubt mistaken, but they have external resemblance
on their side. Up to the age of puberty children of both sexes have
little to distinguish them to the eye, the same face and form, the same
complexion and voice, everything is the same; girls are children and
boys are children; one name is enough for creatures so closely
resembling one another. Males whose development is arrested preserve
this resemblance all their lives; they are always big children; and
women who never lose this resemblance seem in many respects never to be
more than children.

But, speaking generally, man is not meant to remain a child. He leaves
childhood behind him at the time ordained by nature; and this critical
moment, short enough in itself, has far-reaching consequences.

As the roaring of the waves precedes the tempest, so the murmur of
rising passions announces this tumultuous change; a suppressed
excitement warns us of the approaching danger. A change of temper,
frequent outbreaks of anger, a perpetual stirring of the mind, make the
child almost ungovernable. He becomes deaf to the voice he used to obey;
he is a lion in a fever; he distrusts his keeper and refuses to be
controlled.

With the moral symptoms of a changing temper there are perceptible
changes in appearance. His countenance develops and takes the stamp of
his character; the soft and sparse down upon his cheeks becomes darker
and stiffer. His voice grows hoarse or rather he loses it altogether. He
is neither a child nor a man and cannot speak like either of them. His
eyes, those organs of the soul which till now were dumb, find speech and
meaning; a kindling fire illumines them, there is still a sacred
innocence in their ever brightening glance, but they have lost their
first meaningless expression; he is already aware that they can say too
much; he is beginning to learn to lower his eyes and blush, he is
becoming sensitive, though he does not know what it is that he feels; he
is uneasy without knowing why. All this may happen gradually and give
you time enough; but if his keenness becomes impatience, his eagerness
madness, if he is angry and sorry all in a moment, if he weeps without
cause, if in the presence of objects which are beginning to be a source
of danger his pulse quickens and his eyes sparkle, if he trembles when a
woman's hand touches his, if he is troubled or timid in her presence, O
Ulysses, wise Ulysses! have a care! The passages you closed with so much
pains are open; the winds are unloosed; keep your hand upon the helm or
all is lost.

This is the second birth I spoke of; then it is that man really enters
upon life; henceforth no human passion is a stranger to him. Our efforts
so far have been child's play, now they are of the greatest importance.
This period when education is usually finished is just the time to
begin; but to explain this new plan properly, let us take up our story
where we left it.

Our passions are the chief means of self-preservation; to try to destroy
them is therefore as absurd as it is useless; this would be to overcome
nature, to reshape God's handiwork. If God bade man annihilate the
passions he has given him, God would bid him be and not be; He would
contradict himself. He has never given such a foolish commandment, there
is nothing like it written on the heart of man, and what God will have a
man do, He does not leave to the words of another man. He speaks
Himself; His words are written in the secret heart.

Now I consider those who would prevent the birth of the passions almost
as foolish as those who would destroy them, and those who think this has
been my object hitherto are greatly mistaken.

But should we reason rightly, if from the fact that passions are natural
to man, we inferred that all the passions we feel in ourselves and
behold in others are natural? Their source, indeed, is natural; but they
have been swollen by a thousand other streams; they are a great river
which is constantly growing, one in which we can scarcely find a single
drop of the original stream. Our natural passions are few in number;
they are the means to freedom, they tend to self-preservation. All those
which enslave and destroy us have another source; nature does not bestow
them on us; we seize on them in her despite.

The origin of our passions, the root and spring of all the rest, the
only one which is born with man, which never leaves him as long as he
lives, is self-love; this passion is primitive, instinctive, it precedes
all the rest, which are in a sense only modifications of it. In this
sense, if you like, they are all natural. But most of these
modifications are the result of external influences, without which they
would never occur, and such modifications, far from being advantageous
to us, are harmful. They change the original purpose and work against
its end; then it is that man finds himself outside nature and at strife
with himself.

Self-love is always good, always in accordance with the order of nature.
The preservation of our own life is specially entrusted to each one of
us, and our first care is, and must be, to watch over our own life; and
how can we continually watch over it, if we do not take the greatest
interest in it?

Self-preservation requires, therefore, that we shall love ourselves; we
must love ourselves above everything, and it follows directly from this
that we love what contributes to our preservation. Every child becomes
fond of its nurse; Romulus must have loved the she-wolf who suckled him.
At first this attachment is quite unconscious; the individual is
attracted to that which contributes to his welfare and repelled by that
which is harmful; this is merely blind instinct. What transforms this
instinct into feeling, the liking into love, the aversion into hatred,
is the evident intention of helping or hurting us. We do not become
passionately attached to objects without feeling, which only follow the
direction given them; but those from which we expect benefit or injury
from their internal disposition, from their will, those we see acting
freely for or against us, inspire us with like feelings to those they
exhibit towards us. Something does us good, we seek after it; but we
love the person who does us good; something harms us and we shrink from
it, but we hate the person who tries to hurt us.

The child's first sentiment is self-love, his second, which is derived
from it, is love of those about him; for in his present state of
weakness he is only aware of people through the help and attention
received from them. At first his affection for his nurse and his
governess is mere habit. He seeks them because he needs them and because
he is happy when they are there; it is rather perception than kindly
feeling. It takes a long time to discover not merely that they are
useful to him, but that they desire to be useful to him, and then it is
that he begins to love them.

So a child is naturally disposed to kindly feeling because he sees that
every one about him is inclined to help him, and from this experience he
gets the habit of a kindly feeling towards his species; but with the
expansion of his relations, his needs, his dependence, active or
passive, the consciousness of his relations to others is awakened, and
leads to the sense of duties and preferences. Then the child becomes
masterful, jealous, deceitful, and vindictive. If he is not compelled to
obedience, when he does not see the usefulness of what he is told to do,
he attributes it to caprice, to an intention of tormenting him, and he
rebels. If people give in to him, as soon as anything opposes him he
regards it as rebellion, as a determination to resist him; he beats the
chair or table for disobeying him. Self-love, which concerns itself only
with ourselves, is content to satisfy our own needs; but selfishness,
which is always comparing self with others, is never satisfied and never
can be; for this feeling, which prefers ourselves to others, requires
that they should prefer us to themselves, which is impossible. Thus the
tender and gentle passions spring from self-love, while the hateful and
angry passions spring from selfishness. So it is the fewness of his
needs, the narrow limits within which he can compare himself with
others, that makes a man really good; what makes him really bad is a
multiplicity of needs and dependence on the opinions of others. It is
easy to see how we can apply this principle and guide every passion of
children and men towards good or evil. True, man cannot always live
alone, and it will be hard therefore to remain good; and this difficulty
will increase of necessity as his relations with others are extended.
For this reason, above all, the dangers of social life demand that the
necessary skill and care shall be devoted to guarding the human heart
against the depravity which springs from fresh needs.

Man's proper study is that of his relation to his environment. So long
as he only knows that environment through his physical nature, he should
study himself in relation to things; this is the business of his
childhood; when he begins to be aware of his moral nature, he should
study himself in relation to his fellow-men; this is the business of his
whole life, and we have now reached the time when that study should be
begun.

As soon as a man needs a companion he is no longer an isolated creature,
his heart is no longer alone. All his relations with his species, all
the affections of his heart, come into being along with this. His first
passion soon arouses the rest.

The direction of the instinct is uncertain. One sex is attracted by the
other; that is the impulse of nature. Choice, preferences, individual
likings, are the work of reason, prejudice, and habit; time and
knowledge are required to make us capable of love; we do not love
without reasoning or prefer without comparison. These judgments are none
the less real, although they are formed unconsciously. True love,
whatever you may say, will always be held in honour by mankind; for
although its impulses lead us astray, although it does not bar the door
of the heart to certain detestable qualities, although it even gives
rise to these, yet it always presupposes certain worthy characteristics,
without which we should be incapable of love. This choice, which is
supposed to be contrary to reason, really springs from reason. We say
Love is blind because his eyes are better than ours, and he perceives
relations which we cannot discern. All women would be alike to a man who
had no idea of virtue or beauty, and the first comer would always be the
most charming. Love does not spring from nature, far from it; it is the
curb and law of her desires; it is love that makes one sex indifferent
to the other, the loved one alone excepted.

We wish to inspire the preference we feel; love must be mutual. To be
loved we must be worthy of love; to be preferred we must be more worthy
than the rest, at least in the eyes of our beloved. Hence we begin to
look around among our fellows; we begin to compare ourselves with them,
there is emulation, rivalry, and jealousy. A heart full to overflowing
loves to make itself known; from the need of a mistress there soon
springs the need of a friend He who feels how sweet it is to be loved,
desires to be loved by everybody; and there could be no preferences if
there were not many that fail to find satisfaction. With love and
friendship there begin dissensions, enmity, and hatred. I behold
deference to other people's opinions enthroned among all these divers
passions, and foolish mortals, enslaved by her power, base their very
existence merely on what other people think.

Expand these ideas and you will see where we get that form of
selfishness which we call natural selfishness, and how selfishness
ceases to be a simple feeling and becomes pride in great minds, vanity
in little ones, and in both feeds continually at our neighbour's cost.
Passions of this kind, not having any germ in the child's heart, cannot
spring up in it of themselves; it is we who sow the seeds, and they
never take root unless by our fault. Not so with the young man; they
will find an entrance in spite of us. It is therefore time to change our
methods.

Let us begin with some considerations of importance with regard to the
critical stage under discussion. The change from childhood to puberty is
not so clearly determined by nature but that it varies according to
individual temperament and racial conditions. Everybody knows the
differences which have been observed with regard to this between hot and
cold countries, and every one sees that ardent temperaments mature
earlier than others; but we may be mistaken as to the causes, and we may
often attribute to physical causes what is really due to moral: this is
one of the commonest errors in the philosophy of our times. The teaching
of nature comes slowly; man's lessons are mostly premature. In the
former case, the senses kindle the imagination, in the latter the
imagination kindles the senses; it gives them a precocious activity
which cannot fail to enervate the individual and, in the long run, the
race. It is a more general and more trustworthy fact than that of
climatic influences, that puberty and sexual power is always more
precocious among educated and civilised races, than among the ignorant
and barbarous. {[}Footnote: ``In towns,'' says M. Buffon, ``and among
the well-to-do classes, children accustomed to plentiful and nourishing
food sooner reach this state; in the country and among the poor,
children are more backward, because of their poor and scanty food.'' I
admit the fact but not the explanation, for in the districts where the
food of the villagers is plentiful and good, as in the Valais and even
in some of the mountain districts of Italy, such as Friuli, the age of
puberty for both sexes is quite as much later than in the heart of the
towns, where, in order to gratify their vanity, people are often
extremely parsimonious in the matter of food, and where most people, in
the words of the proverb, have a velvet coat and an empty belly. It is
astonishing to find in these mountainous regions big lads as strong as a
man with shrill voices and smooth chins, and tall girls, well developed
in other respects, without any trace of the periodic functions of their
sex. This difference is, in my opinion, solely due to the fact that in
the simplicity of their manners the imagination remains calm and
peaceful, and does not stir the blood till much later, and thus their
temperament is much less precocious.{]} Children are preternaturally
quick to discern immoral habits under the cloak of decency with which
they are concealed. The prim speech imposed upon them, the lessons in
good behaviour, the veil of mystery you profess to hang before their
eyes, serve but to stimulate their curiosity. It is plain, from the way
you set about it, that they are meant to learn what you profess to
conceal; and of all you teach them this is most quickly assimilated.

Consult experience and you will find how far this foolish method hastens
the work of nature and ruins the character. This is one of the chief
causes of physical degeneration in our towns. The young people,
prematurely exhausted, remain small, puny, and misshapen, they grow old
instead of growing up, like a vine forced to bear fruit in spring, which
fades and dies before autumn.

To know how far a happy ignorance may prolong the innocence of children,
you must live among rude and simple people. It is a sight both touching
and amusing to see both sexes, left to the protection of their own
hearts, continuing the sports of childhood in the flower of youth and
beauty, showing by their very familiarity the purity of their pleasures.
When at length those delightful young people marry, they bestow on each
other the first fruits of their person, and are all the dearer
therefore. Swarms of strong and healthy children are the pledges of a
union which nothing can change, and the fruit of the virtue of their
early years.

If the age at which a man becomes conscious of his sex is deferred as
much by the effects of education as by the action of nature, it follows
that this age may be hastened or retarded according to the way in which
the child is brought up; and if the body gains or loses strength in
proportion as its development is accelerated or retarded, it also
follows that the more we try to retard it the stronger and more vigorous
will the young man be. I am still speaking of purely physical
consequences; you will soon see that this is not all.

From these considerations I arrive at the solution of the question so
often discussed---Should we enlighten children at an early period as to
the objects of their curiosity, or is it better to put them off with
decent shams? I think we need do neither. In the first place, this
curiosity will not arise unless we give it a chance. We must therefore
take care not to give it an opportunity. In the next place, questions
one is not obliged to answer do not compel us to deceive those who ask
them; it is better to bid the child hold his tongue than to tell him a
lie. He will not be greatly surprised at this treatment if you have
already accustomed him to it in matters of no importance. Lastly, if you
decide to answer his questions, let it be with the greatest plainness,
without mystery or confusion, without a smile. It is much less dangerous
to satisfy a child's curiosity than to stimulate it.

Let your answers be always grave, brief, decided, and without trace of
hesitation. I need not add that they should be true. We cannot teach
children the danger of telling lies to men without realising, on the
man's part, the danger of telling lies to children. A single untruth on
the part of the master will destroy the results of his education.

Complete ignorance with regard to certain matters is perhaps the best
thing for children; but let them learn very early what it is impossible
to conceal from them permanently. Either their curiosity must never be
aroused, or it must be satisfied before the age when it becomes a source
of danger. Your conduct towards your pupil in this respect depends
greatly on his individual circumstances, the society in which he moves,
the position in which he may find himself, etc. Nothing must be left to
chance; and if you are not sure of keeping him in ignorance of the
difference between the sexes till he is sixteen, take care you teach him
before he is ten.

I do not like people to be too fastidious in speaking with children, nor
should they go out of their way to avoid calling a spade a spade; they
are always found out if they do. Good manners in this respect are always
perfectly simple; but an imagination soiled by vice makes the ear
over-sensitive and compels us to be constantly refining our expressions.
Plain words do not matter; it is lascivious ideas which must be avoided.

Although modesty is natural to man, it is not natural to children.
Modesty only begins with the knowledge of evil; and how should children
without this knowledge of evil have the feeling which results from it?
To give them lessons in modesty and good conduct is to teach them that
there are things shameful and wicked, and to give them a secret wish to
know what these things are. Sooner or later they will find out, and the
first spark which touches the imagination will certainly hasten the
awakening of the senses. Blushes are the sign of guilt; true innocence
is ashamed of nothing.

Children have not the same desires as men; but they are subject like
them to the same disagreeable needs which offend the senses, and by this
means they may receive the same lessons in propriety. Follow the mind of
nature which has located in the same place the organs of secret
pleasures and those of disgusting needs; she teaches us the same
precautions at different ages, sometimes by means of one idea and
sometimes by another; to the man through modesty, to the child through
cleanliness.

I can only find one satisfactory way of preserving the child's
innocence, to surround him by those who respect and love him. Without
this all our efforts to keep him in ignorance fail sooner or later; a
smile, a wink, a careless gesture tells him all we sought to hide; it is
enough to teach him to perceive that there is something we want to hide
from him. The delicate phrases and expressions employed by persons of
politeness assume a knowledge which children ought not to possess, and
they are quite out of place with them, but when we truly respect the
child's innocence we easily find in talking to him the simple phrases
which befit him. There is a certain directness of speech which is
suitable and pleasing to innocence; this is the right tone to adopt in
order to turn the child from dangerous curiosity. By speaking simply to
him about everything you do not let him suspect there is anything left
unsaid. By connecting coarse words with the unpleasant ideas which
belong to them, you quench the first spark of imagination; you do not
forbid the child to say these words or to form these ideas; but without
his knowing it you make him unwilling to recall them. And how much
confusion is spared to those who speaking from the heart always say the
right thing, and say it as they themselves have felt it!

``Where do little children come from?'' This is an embarrassing
question, which occurs very naturally to children, one which foolishly
or wisely answered may decide their health and their morals for life.
The quickest way for a mother to escape from it without deceiving her
son is to tell him to hold his tongue. That will serve its turn if he
has always been accustomed to it in matters of no importance, and if he
does not suspect some mystery from this new way of speaking. But the
mother rarely stops there. ``It is the married people's secret,'' she
will say, ``little boys should not be so curious.'' That is all very
well so far as the mother is concerned, but she may be sure that the
little boy, piqued by her scornful manner, will not rest till he has
found out the married people's secret, which will very soon be the case.

Let me tell you a very different answer which I heard given to the same
question, one which made all the more impression on me, coming, as it
did, from a woman, modest in speech and behaviour, but one who was able
on occasion, for the welfare of her child and for the cause of virtue,
to cast aside the false fear of blame and the silly jests of the
foolish. Not long before the child had passed a small stone which had
torn the passage, but the trouble was over and forgotten. ``Mamma,''
said the eager child, ``where do little children come from?'' ``My
child,'' replied his mother without hesitation, ``women pass them with
pains that sometimes cost their life.'' Let fools laugh and silly people
be shocked; but let the wise inquire if it is possible to find a wiser
answer and one which would better serve its purpose.

In the first place the thought of a need of nature with which the child
is well acquainted turns his thoughts from the idea of a mysterious
process. The accompanying ideas of pain and death cover it with a veil
of sadness which deadens the imagination and suppresses curiosity;
everything leads the mind to the results, not the causes, of
child-birth. This is the information to which this answer leads. If the
repugnance inspired by this answer should permit the child to inquire
further, his thoughts are turned to the infirmities of human nature,
disgusting things, images of pain. What chance is there for any
stimulation of desire in such a conversation? And yet you see there is
no departure from truth, no need to deceive the scholar in order to
teach him.

Your children read; in the course of their reading they meet with things
they would never have known without reading. Are they students, their
imagination is stimulated and quickened in the silence of the study. Do
they move in the world of society, they hear a strange jargon, they see
conduct which makes a great impression on them; they have been told so
continually that they are men that in everything men do in their
presence they at once try to find how that will suit themselves; the
conduct of others must indeed serve as their pattern when the opinions
of others are their law. Servants, dependent on them, and therefore
anxious to please them, flatter them at the expense of their morals;
giggling governesses say things to the four-year-old child which the
most shameless woman would not dare to say to them at fifteen. They soon
forget what they said, but the child has not forgotten what he heard.
Loose conversation prepares the way for licentious conduct; the child is
debauched by the cunning lacquey, and the secret of the one guarantees
the secret of the other.

The child brought up in accordance with his age is alone. He knows no
attachment but that of habit, he loves his sister like his watch, and
his friend like his dog. He is unconscious of his sex and his species;
men and women are alike unknown; he does not connect their sayings and
doings with himself, he neither sees nor hears, or he pays no heed to
them; he is no more concerned with their talk than their actions; he has
nothing to do with it. This is no artificial error induced by our
method, it is the ignorance of nature. The time is at hand when that
same nature will take care to enlighten her pupil, and then only does
she make him capable of profiting by the lessons without danger. This is
our principle; the details of its rules are outside my subject; and the
means I suggest with regard to other matters will still serve to
illustrate this.

Do you wish to establish law and order among the rising passions,
prolong the period of their development, so that they may have time to
find their proper place as they arise. Then they are controlled by
nature herself, not by man; your task is merely to leave it in her
hands. If your pupil were alone, you would have nothing to do; but
everything about him enflames his imagination. He is swept along on the
torrent of conventional ideas; to rescue him you must urge him in the
opposite direction. Imagination must be curbed by feeling and reason
must silence the voice of conventionality. Sensibility is the source of
all the passions, imagination determines their course. Every creature
who is aware of his relations must be disturbed by changes in these
relations and when he imagines or fancies he imagines others better
adapted to his nature. It is the errors of the imagination which
transmute into vices the passions of finite beings, of angels even, if
indeed they have passions; for they must needs know the nature of every
creature to realise what relations are best adapted to themselves.

This is the sum of human wisdom with regard to the use of the passions.
First, to be conscious of the true relations of man both in the species
and the individual; second, to control all the affections in accordance
with these relations.

But is man in a position to control his affections according to such and
such relations? No doubt he is, if he is able to fix his imagination on
this or that object, or to form this or that habit. Moreover, we are not
so much concerned with what a man can do for himself, as with what we
can do for our pupil through our choice of the circumstances in which he
shall be placed. To show the means by which he may be kept in the path
of nature is to show plainly enough how he might stray from that path.

So long as his consciousness is confined to himself there is no morality
in his actions; it is only when it begins to extend beyond himself that
he forms first the sentiments and then the ideas of good and ill, which
make him indeed a man, and an integral part of his species. To begin
with we must therefore confine our observations to this point.

These observations are difficult to make, for we must reject the
examples before our eyes, and seek out those in which the successive
developments follow the order of nature.

A child sophisticated, polished, and civilised, who is only awaiting the
power to put into practice the precocious instruction he has received,
is never mistaken with regard to the time when this power is acquired.
Far from awaiting it, he accelerates it; he stirs his blood to a
premature ferment; he knows what should be the object of his desires
long before those desires are experienced. It is not nature which
stimulates him; it is he who forces the hand of nature; she has nothing
to teach him when he becomes a man; he was a man in thought long before
he was a man in reality.

The true course of nature is slower and more gradual. Little by little
the blood grows warmer, the faculties expand, the character is formed.
The wise workman who directs the process is careful to perfect every
tool before he puts it to use; the first desires are preceded by a long
period of unrest, they are deceived by a prolonged ignorance, they know
not what they want. The blood ferments and bubbles; overflowing vitality
seeks to extend its sphere. The eye grows brighter and surveys others,
we begin to be interested in those about us, we begin to feel that we
are not meant to live alone; thus the heart is thrown open to human
affection, and becomes capable of attachment.

The first sentiment of which the well-trained youth is capable is not
love but friendship. The first work of his rising imagination is to make
known to him his fellows; the species affects him before the sex. Here
is another advantage to be gained from prolonged innocence; you may take
advantage of his dawning sensibility to sow the first seeds of humanity
in the heart of the young adolescent. This advantage is all the greater
because this is the only time in his life when such efforts may be
really successful.

I have always observed that young men, corrupted in early youth and
addicted to women and debauchery, are inhuman and cruel; their
passionate temperament makes them impatient, vindictive, and angry;
their imagination fixed on one object only, refuses all others; mercy
and pity are alike unknown to them; they would have sacrificed father,
mother, the whole world, to the least of their pleasures. A young man,
on the other hand, brought up in happy innocence, is drawn by the first
stirrings of nature to the tender and affectionate passions; his warm
heart is touched by the sufferings of his fellow-creatures; he trembles
with delight when he meets his comrade, his arms can embrace tenderly,
his eyes can shed tears of pity; he learns to be sorry for offending
others through his shame at causing annoyance. If the eager warmth of
his blood makes him quick, hasty, and passionate, a moment later you see
all his natural kindness of heart in the eagerness of his repentance; he
weeps, he groans over the wound he has given; he would atone for the
blood he has shed with his own; his anger dies away, his pride abases
itself before the consciousness of his wrong-doing. Is he the injured
party, in the height of his fury an excuse, a word, disarms him; he
forgives the wrongs of others as whole-heartedly as he repairs his own.
Adolescence is not the age of hatred or vengeance; it is the age of
pity, mercy, and generosity. Yes, I maintain, and I am not afraid of the
testimony of experience, a youth of good birth, one who has preserved
his innocence up to the age of twenty, is at that age the best, the most
generous, the most loving, and the most lovable of men. You never heard
such a thing; I can well believe that philosophers such as you, brought
up among the corruption of the public schools, are unaware of it.

Man's weakness makes him sociable. Our common sufferings draw our hearts
to our fellow-creatures; we should have no duties to mankind if we were
not men. Every affection is a sign of insufficiency; if each of us had
no need of others, we should hardly think of associating with them. So
our frail happiness has its roots in our weakness. A really happy man is
a hermit; God only enjoys absolute happiness; but which of us has any
idea what that means? If any imperfect creature were self-sufficing,
what would he have to enjoy? To our thinking he would be wretched and
alone. I do not understand how one who has need of nothing could love
anything, nor do I understand how he who loves nothing can be happy.

Hence it follows that we are drawn towards our fellow-creatures less by
our feeling for their joys than for their sorrows; for in them we
discern more plainly a nature like our own, and a pledge of their
affection for us. If our common needs create a bond of interest our
common sufferings create a bond of affection. The sight of a happy man
arouses in others envy rather than love, we are ready to accuse him of
usurping a right which is not his, of seeking happiness for himself
alone, and our selfishness suffers an additional pang in the thought
that this man has no need of us. But who does not pity the wretch when
he beholds his sufferings? who would not deliver him from his woes if a
wish could do it? Imagination puts us more readily in the place of the
miserable man than of the happy man; we feel that the one condition
touches us more nearly than the other. Pity is sweet, because, when we
put ourselves in the place of one who suffers, we are aware,
nevertheless, of the pleasure of not suffering like him. Envy is bitter,
because the sight of a happy man, far from putting the envious in his
place, inspires him with regret that he is not there. The one seems to
exempt us from the pains he suffers, the other seems to deprive us of
the good things he enjoys.

Do you desire to stimulate and nourish the first stirrings of awakening
sensibility in the heart of a young man, do you desire to incline his
disposition towards kindly deed and thought, do not cause the seeds of
pride, vanity, and envy to spring up in him through the misleading
picture of the happiness of mankind; do not show him to begin with the
pomp of courts, the pride of palaces, the delights of pageants; do not
take him into society and into brilliant assemblies; do not show him the
outside of society till you have made him capable of estimating it at
its true worth. To show him the world before he is acquainted with men,
is not to train him, but to corrupt him; not to teach, but to mislead.

By nature men are neither kings, nobles, courtiers, nor millionaires.
All men are born poor and naked, all are liable to the sorrows of life,
its disappointments, its ills, its needs, its suffering of every kind;
and all are condemned at length to die. This is what it really means to
be a man, this is what no mortal can escape. Begin then with the study
of the essentials of humanity, that which really constitutes mankind.

At sixteen the adolescent knows what it is to suffer, for he himself has
suffered; but he scarcely realises that others suffer too; to see
without feeling is not knowledge, and as I have said again and again the
child who does not picture the feelings of others knows no ills but his
own; but when his imagination is kindled by the first beginnings of
growing sensibility, he begins to perceive himself in his
fellow-creatures, to be touched by their cries, to suffer in their
sufferings. It is at this time that the sorrowful picture of suffering
humanity should stir his heart with the first touch of pity he has ever
known.

If it is not easy to discover this opportunity in your scholars, whose
fault is it? You taught them so soon to play at feeling, you taught them
so early its language, that speaking continually in the same strain they
turn your lessons against yourself, and give you no chance of
discovering when they cease to lie, and begin to feel what they say. But
look at Emile; I have led him up to this age, and he has neither felt
nor pretended to feel. He has never said, ``I love you dearly,'' till he
knew what it was to love; he has never been taught what expression to
assume when he enters the room of his father, his mother, or his sick
tutor; he has not learnt the art of affecting a sorrow he does not feel.
He has never pretended to weep for the death of any one, for he does not
know what it is to die. There is the same insensibility in his heart as
in his manners. Indifferent, like every child, to every one but himself,
he takes no interest in any one; his only peculiarity is that he will
not pretend to take such an interest; he is less deceitful than others.

Emile having thought little about creatures of feeling will be a long
time before he knows what is meant by pain and death. Groans and cries
will begin to stir his compassion, he will turn away his eyes at the
sight of blood; the convulsions of a dying animal will cause him I know
not what anguish before he knows the source of these impulses. If he
were still stupid and barbarous he would not feel them; if he were more
learned he would recognise their source; he has compared ideas too
frequently already to be insensible, but not enough to know what he
feels.

So pity is born, the first relative sentiment which touches the human
heart according to the order of nature. To become sensitive and pitiful
the child must know that he has fellow-creatures who suffer as he has
suffered, who feel the pains he has felt, and others which he can form
some idea of, being capable of feeling them himself. Indeed, how can we
let ourselves be stirred by pity unless we go beyond ourselves, and
identify ourselves with the suffering animal, by leaving, so to speak,
our own nature and taking his. We only suffer so far as we suppose he
suffers; the suffering is not ours but his. So no one becomes sensitive
till his imagination is aroused and begins to carry him outside himself.

What should we do to stimulate and nourish this growing sensibility, to
direct it, and to follow its natural bent? Should we not present to the
young man objects on which the expansive force of his heart may take
effect, objects which dilate it, which extend it to other creatures,
which take him outside himself? should we not carefully remove
everything that narrows, concentrates, and strengthens the power of the
human self? that is to say, in other words, we should arouse in him
kindness, goodness, pity, and beneficence, all the gentle and attractive
passions which are naturally pleasing to man; those passions prevent the
growth of envy, covetousness, hatred, all the repulsive and cruel
passions which make our sensibility not merely a cipher but a minus
quantity, passions which are the curse of those who feel them.

I think I can sum up the whole of the preceding reflections in two or
three maxims, definite, straightforward, and easy to understand.

FIRST MAXIM.---It is not in human nature to put ourselves in the place
of those who are happier than ourselves, but only in the place of those
who can claim our pity.

If you find exceptions to this rule, they are more apparent than real.
Thus we do not put ourselves in the place of the rich or great when we
become fond of them; even when our affection is real, we only
appropriate to ourselves a part of their welfare. Sometimes we love the
rich man in the midst of misfortunes; but so long as he prospers he has
no real friend, except the man who is not deceived by appearances, who
pities rather than envies him in spite of his prosperity.

The happiness belonging to certain states of life appeals to us; take,
for instance, the life of a shepherd in the country. The charm of seeing
these good people so happy is not poisoned by envy; we are genuinely
interested in them. Why is this? Because we feel we can descend into
this state of peace and innocence and enjoy the same happiness; it is an
alternative which only calls up pleasant thoughts, so long as the wish
is as good as the deed. It is always pleasant to examine our stores, to
contemplate our own wealth, even when we do not mean to spend it.

From this we see that to incline a young man to humanity you must not
make him admire the brilliant lot of others; you must show him life in
its sorrowful aspects and arouse his fears. Thus it becomes clear that
he must force his own way to happiness, without interfering with the
happiness of others.

SECOND MAXIM.---We never pity another's woes unless we know we may
suffer in like manner ourselves.

``Non ignara mali, miseris succurrere disco.''---Virgil.

I know nothing go fine, so full of meaning, so touching, so true as
these words.

Why have kings no pity on their people? Because they never expect to be
ordinary men. Why are the rich so hard on the poor? Because they have no
fear of becoming poor. Why do the nobles look down upon the people?
Because a nobleman will never be one of the lower classes. Why are the
Turks generally kinder and more hospitable than ourselves? Because,
under their wholly arbitrary system of government, the rank and wealth
of individuals are always uncertain and precarious, so that they do not
regard poverty and degradation as conditions with which they have no
concern; to-morrow, any one may himself be in the same position as those
on whom he bestows alms to-day. This thought, which occurs again and
again in eastern romances, lends them a certain tenderness which is not
to be found in our pretentious and harsh morality.

So do not train your pupil to look down from the height of his glory
upon the sufferings of the unfortunate, the labours of the wretched, and
do not hope to teach him to pity them while he considers them as far
removed from himself. Make him thoroughly aware of the fact that the
fate of these unhappy persons may one day be his own, that his feet are
standing on the edge of the abyss, into which he may be plunged at any
moment by a thousand unexpected irresistible misfortunes. Teach him to
put no trust in birth, health, or riches; show him all the changes of
fortune; find him examples---there are only too many of them---in which
men of higher rank than himself have sunk below the condition of these
wretched ones. Whether by their own fault or another's is for the
present no concern of ours; does he indeed know the meaning of the word
fault? Never interfere with the order in which he acquires knowledge,
and teach him only through the means within his reach; it needs no great
learning to perceive that all the prudence of mankind cannot make
certain whether he will be alive or dead in an hour's time, whether
before nightfall he will not be grinding his teeth in the pangs of
nephritis, whether a month hence he will be rich or poor, whether in a
year's time he may not be rowing an Algerian galley under the lash of
the slave-driver. Above all do not teach him this, like his catechism,
in cold blood; let him see and feel the calamities which overtake men;
surprise and startle his imagination with the perils which lurk
continually about a man's path; let him see the pitfalls all about him,
and when he hears you speak of them, let him cling more closely to you
for fear lest he should fall. ``You will make him timid and cowardly,''
do you say? We shall see; let us make him kindly to begin with, that is
what matters most.

THIRD MAXIM.---The pity we feel for others is proportionate, not to the
amount of the evil, but to the feelings we attribute to the sufferers.

We only pity the wretched so far as we think they feel the need of pity.
The bodily effect of our sufferings is less than one would suppose; it
is memory that prolongs the pain, imagination which projects it into the
future, and makes us really to be pitied. This is, I think, one of the
reasons why we are more callous to the sufferings of animals than of
men, although a fellow-feeling ought to make us identify ourselves
equally with either. We scarcely pity the cart-horse in his shed, for we
do not suppose that while he is eating his hay he is thinking of the
blows he has received and the labours in store for him. Neither do we
pity the sheep grazing in the field, though we know it is about to be
slaughtered, for we believe it knows nothing of the fate in store for
it. In this way we also become callous to the fate of our fellow-men,
and the rich console themselves for the harm done by them to the poor,
by the assumption that the poor are too stupid to feel. I usually judge
of the value any one puts on the welfare of his fellow-creatures by what
he seems to think of them. We naturally think lightly of the happiness
of those we despise. It need not surprise you that politicians speak so
scornfully of the people, and philosophers profess to think mankind so
wicked.

The people are mankind; those who do not belong to the people are so few
in number that they are not worth counting. Man is the same in every
station of life; if that be so, those ranks to which most men belong
deserve most honour. All distinctions of rank fade away before the eyes
of a thoughtful person; he sees the same passions, the same feelings in
the noble and the guttersnipe; there is merely a slight difference in
speech, and more or less artificiality of tone; and if there is indeed
any essential difference between them, the disadvantage is all on the
side of those who are more sophisticated. The people show themselves as
they are, and they are not attractive; but the fashionable world is
compelled to adopt a disguise; we should be horrified if we saw it as it
really is.

There is, so our wiseacres tell us, the same amount of happiness and
sorrow in every station. This saying is as deadly in its effects as it
is incapable of proof; if all are equally happy why should I trouble
myself about any one? Let every one stay where he is; leave the slave to
be ill-treated, the sick man to suffer, and the wretched to perish; they
have nothing to gain by any change in their condition. You enumerate the
sorrows of the rich, and show the vanity of his empty pleasures; what
barefaced sophistry! The rich man's sufferings do not come from his
position, but from himself alone when he abuses it. He is not to be
pitied were he indeed more miserable than the poor, for his ills are of
his own making, and he could be happy if he chose. But the sufferings of
the poor man come from external things, from the hardships fate has
imposed upon him. No amount of habit can accustom him to the bodily ills
of fatigue, exhaustion, and hunger. Neither head nor heart can serve to
free him from the sufferings of his condition. How is Epictetus the
better for knowing beforehand that his master will break his leg for
him; does he do it any the less? He has to endure not only the pain
itself but the pains of anticipation. If the people were as wise as we
assume them to be stupid, how could they be other than they are? Observe
persons of this class; you will see that, with a different way of
speaking, they have as much intelligence and more common-sense than
yourself. Have respect then for your species; remember that it consists
essentially of the people, that if all the kings and all the
philosophers were removed they would scarcely be missed, and things
would go on none the worse. In a word, teach your pupil to love all men,
even those who fail to appreciate him; act in such way that he is not a
member of any class, but takes his place in all alike: speak in his
hearing of the human race with tenderness, and even with pity, but never
with scorn. You are a man; do not dishonour mankind.

It is by these ways and others like them---how different from the beaten
paths---that we must reach the heart of the young adolescent, and
stimulate in him the first impulses of nature; we must develop that
heart and open its doors to his fellow-creatures, and there must be as
little self-interest as possible mixed up with these impulses; above
all, no vanity, no emulation, no boasting, none of those sentiments
which force us to compare ourselves with others; for such comparisons
are never made without arousing some measure of hatred against those who
dispute our claim to the first place, were it only in our own
estimation. Then we must be either blind or angry, a bad man or a fool;
let us try to avoid this dilemma. Sooner or later these dangerous
passions will appear, so you tell me, in spite of us. I do not deny it.
There is a time and place for everything; I am only saying that we
should not help to arouse these passions.

This is the spirit of the method to be laid down. In this case examples
and illustrations are useless, for here we find the beginning of the
countless differences of character, and every example I gave would
possibly apply to only one case in a hundred thousand. It is at this age
that the clever teacher begins his real business, as a student and a
philosopher who knows how to probe the heart and strives to guide it
aright. While the young man has not learnt to pretend, while he does not
even know the meaning of pretence, you see by his look, his manner, his
gestures, the impression he has received from any object presented to
him; you read in his countenance every impulse of his heart; by watching
his expression you learn to protect his impulses and actually to control
them.

It has been commonly observed that blood, wounds, cries and groans, the
preparations for painful operations, and everything which directs the
senses towards things connected with suffering, are usually the first to
make an impression on all men. The idea of destruction, a more complex
matter, does not have so great an effect; the thought of death affects
us later and less forcibly, for no one knows from his own experience
what it is to die; you must have seen corpses to feel the agonies of the
dying. But when once this idea is established in the mind, there is no
spectacle more dreadful in our eyes, whether because of the idea of
complete destruction which it arouses through our senses, or because we
know that this moment must come for each one of us and we feel ourselves
all the more keenly affected by a situation from which we know there is
no escape.

These various impressions differ in manner and in degree, according to
the individual character of each one of us and his former habits, but
they are universal and no one is altogether free from them. There are
other impressions less universal and of a later growth, impressions most
suited to sensitive souls, such impressions as we receive from moral
suffering, inward grief, the sufferings of the mind, depression, and
sadness. There are men who can be touched by nothing but groans and
tears; the suppressed sobs of a heart labouring under sorrow would never
win a sigh; the sight of a downcast visage, a pale and gloomy
countenance, eyes which can weep no longer, would never draw a tear from
them. The sufferings of the mind are as nothing to them; they weigh
them, their own mind feels nothing; expect nothing from such persons but
inflexible severity, harshness, cruelty. They may be just and upright,
but not merciful, generous, or pitiful. They may, I say, be just, if a
man can indeed be just without being merciful.

But do not be in a hurry to judge young people by this standard, more
especially those who have been educated rightly, who have no idea of the
moral sufferings they have never had to endure; for once again they can
only pity the ills they know, and this apparent insensibility is soon
transformed into pity when they begin to feel that there are in human
life a thousand ills of which they know nothing. As for Emile, if in
childhood he was distinguished by simplicity and good sense, in his
youth he will show a warm and tender heart; for the reality of the
feelings depends to a great extent on the accuracy of the ideas.

But why call him hither? More than one reader will reproach me no doubt
for departing from my first intention and forgetting the lasting
happiness I promised my pupil. The sorrowful, the dying, such sights of
pain and woe, what happiness, what delight is this for a young heart on
the threshold of life? His gloomy tutor, who proposed to give him such a
pleasant education, only introduces him to life that he may suffer. This
is what they will say, but what care I? I promised to make him happy,
not to make him seem happy. Am I to blame if, deceived as usual by the
outward appearances, you take them for the reality?

Let us take two young men at the close of their early education, and let
them enter the world by opposite doors. The one mounts at once to
Olympus, and moves in the smartest society; he is taken to court, he is
presented in the houses of the great, of the rich, of the pretty women.
I assume that he is everywhere made much of, and I do not regard too
closely the effect of this reception on his reason; I assume it can
stand it. Pleasures fly before him, every day provides him with fresh
amusements; he flings himself into everything with an eagerness which
carries you away. You find him busy, eager, and curious; his first
wonder makes a great impression on you; you think him happy; but behold
the state of his heart; you think he is rejoicing, I think he suffers.

What does he see when first he opens his eyes? all sorts of so-called
pleasures, hitherto unknown. Most of these pleasures are only for a
moment within his reach, and seem to show themselves only to inspire
regret for their loss. Does he wander through a palace; you see by his
uneasy curiosity that he is asking why his father's house is not like
it. Every question shows you that he is comparing himself all the time
with the owner of this grand place. And all the mortification arising
from this comparison at once revolts and stimulates his vanity. If he
meets a young man better dressed than himself, I find him secretly
complaining of his parents' meanness. If he is better dressed than
another, he suffers because the latter is his superior in birth or in
intellect, and all his gold lace is put to shame by a plain cloth coat.
Does he shine unrivalled in some assembly, does he stand on tiptoe that
they may see him better, who is there who does not secretly desire to
humble the pride and vanity of the young fop? Everybody is in league
against him; the disquieting glances of a solemn man, the biting phrases
of some satirical person, do not fail to reach him, and if it were only
one man who despised him, the scorn of that one would poison in a moment
the applause of the rest.

Let us grant him everything, let us not grudge him charm and worth; let
him be well-made, witty, and attractive; the women will run after him;
but by pursuing him before he is in love with them, they will inspire
rage rather than love; he will have successes, but neither rapture nor
passion to enjoy them. As his desires are always anticipated; they never
have time to spring up among his pleasures, so he only feels the tedium
of restraint. Even before he knows it he is disgusted and satiated with
the sex formed to be the delight of his own; if he continues its pursuit
it is only through vanity, and even should he really be devoted to
women, he will not be the only brilliant, the only attractive young man,
nor will he always find his mistresses prodigies of fidelity.

I say nothing of the vexation, the deceit, the crimes, and the remorse
of all kinds, inseparable from such a life. We know that experience of
the world disgusts us with it; I am speaking only of the drawbacks
belonging to youthful illusions.

Hitherto the young man has lived in the bosom of his family and his
friends, and has been the sole object of their care; what a change to
enter all at once into a region where he counts for so little; to find
himself plunged into another sphere, he who has been so long the centre
of his own. What insults, what humiliation, must he endure, before he
loses among strangers the ideas of his own importance which have been
formed and nourished among his own people! As a child everything gave
way to him, everybody flocked to him; as a young man he must give place
to every one, or if he preserves ever so little of his former airs, what
harsh lessons will bring him to himself! Accustomed to get everything he
wants without any difficulty, his wants are many, and he feels continual
privations. He is tempted by everything that flatters him; what others
have, he must have too; he covets everything, he envies every one, he
would always be master. He is devoured by vanity, his young heart is
enflamed by unbridled passions, jealousy and hatred among the rest; all
these violent passions burst out at once; their sting rankles in him in
the busy world, they return with him at night, he comes back
dissatisfied with himself, with others; he falls asleep among a thousand
foolish schemes disturbed by a thousand fancies, and his pride shows him
even in his dreams those fancied pleasures; he is tormented by a desire
which will never be satisfied. So much for your pupil; let us turn to
mine.

If the first thing to make an impression on him is something sorrowful
his first return to himself is a feeling of pleasure. When he sees how
many ills he has escaped he thinks he is happier than he fancied. He
shares the suffering of his fellow-creatures, but he shares it of his
own free will and finds pleasure in it. He enjoys at once the pity he
feels for their woes and the joy of being exempt from them; he feels in
himself that state of vigour which projects us beyond ourselves, and
bids us carry elsewhere the superfluous activity of our well-being. To
pity another's woes we must indeed know them, but we need not feel them.
When we have suffered, when we are in fear of suffering, we pity those
who suffer; but when we suffer ourselves, we pity none but ourselves.
But if all of us, being subject ourselves to the ills of life, only
bestow upon others the sensibility we do not actually require for
ourselves, it follows that pity must be a very pleasant feeling, since
it speaks on our behalf; and, on the other hand, a hard-hearted man is
always unhappy, since the state of his heart leaves him no superfluous
sensibility to bestow on the sufferings of others.

We are too apt to judge of happiness by appearances; we suppose it is to
be found in the most unlikely places, we seek for it where it cannot
possibly be; mirth is a very doubtful indication of its presence. A
merry man is often a wretch who is trying to deceive others and distract
himself. The men who are jovial, friendly, and contented at their club
are almost always gloomy grumblers at home, and their servants have to
pay for the amusement they give among their friends. True contentment is
neither merry nor noisy; we are jealous of so sweet a sentiment, when we
enjoy it we think about it, we delight in it for fear it should escape
us. A really happy man says little and laughs little; he hugs his
happiness, so to speak, to his heart. Noisy games, violent delight,
conceal the disappointment of satiety. But melancholy is the friend of
pleasure; tears and pity attend our sweetest enjoyment, and great joys
call for tears rather than laughter.

If at first the number and variety of our amusements seem to contribute
to our happiness, if at first the even tenor of a quiet life seems
tedious, when we look at it more closely we discover that the
pleasantest habit of mind consists in a moderate enjoyment which leaves
little scope for desire and aversion. The unrest of passion causes
curiosity and fickleness; the emptiness of noisy pleasures causes
weariness. We never weary of our state when we know none more
delightful. Savages suffer less than other men from curiosity and from
tedium; everything is the same to them---themselves, not their
possessions---and they are never weary.

The man of the world almost always wears a mask. He is scarcely ever
himself and is almost a stranger to himself; he is ill at ease when he
is forced into his own company. Not what he is, but what he seems, is
all he cares for.

I cannot help picturing in the countenance of the young man I have just
spoken of an indefinable but unpleasant impertinence, smoothness, and
affectation, which is repulsive to a plain man, and in the countenance
of my own pupil a simple and interesting expression which indicates the
real contentment and the calm of his mind; an expression which inspires
respect and confidence, and seems only to await the establishment of
friendly relations to bestow his own confidence in return. It is thought
that the expression is merely the development of certain features
designed by nature. For my own part I think that over and above this
development a man's face is shaped, all unconsciously, by the frequent
and habitual influence of certain affections of the heart. These
affections are shown on the face, there is nothing more certain; and
when they become habitual, they must surely leave lasting traces. This
is why I think the expression shows the character, and that we can
sometimes read one another without seeking mysterious explanations in
powers we do not possess.

A child has only two distinct feelings, joy and sorrow; he laughs or he
cries; he knows no middle course, and he is constantly passing from one
extreme to the other. On account of these perpetual changes there is no
lasting impression on the face, and no expression; but when the child is
older and more sensitive, his feelings are keener or more permanent, and
these deeper impressions leave traces more difficult to erase; and the
habitual state of the feelings has an effect on the features which in
course of time becomes ineffaceable. Still it is not uncommon to meet
with men whose expression varies with their age. I have met with
several, and I have always found that those whom I could observe and
follow had also changed their habitual temper. This one observation
thoroughly confirmed would seem to me decisive, and it is not out of
place in a treatise on education, where it is a matter of importance,
that we should learn to judge the feelings of the heart by external
signs.

I do not know whether my young man will be any the less amiable for not
having learnt to copy conventional manners and to feign sentiments which
are not his own; that does not concern me at present, I only know he
will be more affectionate; and I find it difficult to believe that he,
who cares for nobody but himself, can so far disguise his true feelings
as to please as readily as he who finds fresh happiness for himself in
his affection for others. But with regard to this feeling of happiness,
I think I have said enough already for the guidance of any sensible
reader, and to show that I have not contradicted myself.

I return to my system, and I say, when the critical age approaches,
present to young people spectacles which restrain rather than excite
them; put off their dawning imagination with objects which, far from
inflaming their senses, put a check to their activity. Remove them from
great cities, where the flaunting attire and the boldness of the women
hasten and anticipate the teaching of nature, where everything presents
to their view pleasures of which they should know nothing till they are
of an age to choose for themselves. Bring them back to their early home,
where rural simplicity allows the passions of their age to develop more
slowly; or if their taste for the arts keeps them in town, guard them by
means of this very taste from a dangerous idleness. Choose carefully
their company, their occupations, and their pleasures; show them nothing
but modest and pathetic pictures which are touching but not seductive,
and nourish their sensibility without stimulating their senses. Remember
also, that the danger of excess is not confined to any one place, and
that immoderate passions always do irreparable damage. You need not make
your pupil a sick-nurse or a Brother of Pity; you need not distress him
by the perpetual sight of pain and suffering; you need not take him from
one hospital to another, from the gallows to the prison. He must be
softened, not hardened, by the sight of human misery. When we have seen
a sight it ceases to impress us, use is second nature, what is always
before our eyes no longer appeals to the imagination, and it is only
through the imagination that we can feel the sorrows of others; this is
why priests and doctors who are always beholding death and suffering
become so hardened. Let your pupil therefore know something of the lot
of man and the woes of his fellow-creatures, but let him not see them
too often. A single thing, carefully selected and shown at the right
time, will fill him with pity and set him thinking for a month. His
opinion about anything depends not so much on what he sees, but on how
it reacts on himself; and his lasting impression of any object depends
less on the object itself than on the point of view from which he
regards it. Thus by a sparing use of examples, lessons, and pictures,
you may blunt the sting of sense and delay nature while following her
own lead.

As he acquires knowledge, choose what ideas he shall attach to it; as
his passions awake, select scenes calculated to repress them. A veteran,
as distinguished for his character as for his courage, once told me that
in early youth his father, a sensible man but extremely pious, observed
that through his growing sensibility he was attracted by women, and
spared no pains to restrain him; but at last when, in spite of all his
care, his son was about to escape from his control, he decided to take
him to a hospital, and, without telling him what to expect, he
introduced him into a room where a number of wretched creatures were
expiating, under a terrible treatment, the vices which had brought them
into this plight. This hideous and revolting spectacle sickened the
young man. ``Miserable libertine,'' said his father vehemently,
``begone; follow your vile tastes; you will soon be only too glad to be
admitted to this ward, and a victim to the most shameful sufferings, you
will compel your father to thank God when you are dead.''

These few words, together with the striking spectacle he beheld, made an
impression on the young man which could never be effaced. Compelled by
his profession to pass his youth in garrison, he preferred to face all
the jests of his comrades rather than to share their evil ways. ``I have
been a man,'' he said to me, ``I have had my weaknesses, but even to the
present day the sight of a harlot inspires me with horror.'' Say little
to your pupil, but choose time, place, and people; then rely on concrete
examples for your teaching, and be sure it will take effect.

The way childhood is spent is no great matter; the evil which may find
its way is not irremediable, and the good which may spring up might come
later. But it is not so in those early years when a youth really begins
to live. This time is never long enough for what there is to be done,
and its importance demands unceasing attention; this is why I lay so
much stress on the art of prolonging it. One of the best rules of good
farming is to keep things back as much as possible. Let your progress
also be slow and sure; prevent the youth from becoming a man all at
once. While the body is growing the spirits destined to give vigour to
the blood and strength to the muscles are in process of formation and
elaboration. If you turn them into another channel, and permit that
strength which should have gone to the perfecting of one person to go to
the making of another, both remain in a state of weakness and the work
of nature is unfinished. The workings of the mind, in their turn, are
affected by this change, and the mind, as sickly as the body, functions
languidly and feebly. Length and strength of limb are not the same thing
as courage or genius, and I grant that strength of mind does not always
accompany strength of body, when the means of connection between the two
are otherwise faulty. But however well planned they may be, they will
always work feebly if for motive power they depend upon an exhausted,
impoverished supply of blood, deprived of the substance which gives
strength and elasticity to all the springs of the machinery. There is
generally more vigour of mind to be found among men whose early years
have been preserved from precocious vice, than among those whose evil
living has begun at the earliest opportunity; and this is no doubt the
reason why nations whose morals are pure are generally superior in sense
and courage to those whose morals are bad. The latter shine only through
I know not what small and trifling qualities, which they call wit,
sagacity, cunning; but those great and noble features of goodness and
reason, by which a man is distinguished and honoured through good deeds,
virtues, really useful efforts, are scarcely to be found except among
the nations whose morals are pure.

Teachers complain that the energy of this age makes their pupils unruly;
I see that it is so, but are not they themselves to blame? When once
they have let this energy flow through the channel of the senses, do
they not know that they cannot change its course? Will the long and
dreary sermons of the pedant efface from the mind of his scholar the
thoughts of pleasure when once they have found an entrance; will they
banish from his heart the desires by which it is tormented; will they
chill the heat of a passion whose meaning the scholar realises? Will not
the pupil be roused to anger by the obstacles opposed to the only kind
of happiness of which he has any notion? And in the harsh law imposed
upon him before he can understand it, what will he see but the caprice
and hatred of a man who is trying to torment him? Is it strange that he
rebels and hates you too?

I know very well that if one is easy-going one may be tolerated, and one
may keep up a show of authority. But I fail to see the use of an
authority over the pupil which is only maintained by fomenting the vices
it ought to repress; it is like attempting to soothe a fiery steed by
making it leap over a precipice.

Far from being a hindrance to education, this enthusiasm of adolescence
is its crown and coping-stone; this it is that gives you a hold on the
youth's heart when he is no longer weaker than you. His first affections
are the reins by which you control his movements; he was free, and now I
behold him in your power. So long as he loved nothing, he was
independent of everything but himself and his own necessities; as soon
as he loves, he is dependent on his affections. Thus the first ties
which unite him to his species are already formed. When you direct his
increasing sensibility in this direction, do not expect that it will at
once include all men, and that the word ``mankind'' will have any
meaning for him. Not so; this sensibility will at first confine itself
to those like himself, and these will not be strangers to him, but those
he knows, those whom habit has made dear to him or necessary to him,
those who are evidently thinking and feeling as he does, those whom he
perceives to be exposed to the pains he has endured, those who enjoy the
pleasures he has enjoyed; in a word, those who are so like himself that
he is the more disposed to self-love. It is only after long training,
after much consideration as to his own feelings and the feelings he
observes in others, that he will be able to generalise his individual
notions under the abstract idea of humanity, and add to his individual
affections those which may identify him with the race.

When he becomes capable of affection, he becomes aware of the affection
of others, {[}Footnote: Affection may be unrequited; not so friendship.
Friendship is a bargain, a contract like any other; though a bargain
more sacred than the rest. The word ``friend'' has no other correlation.
Any man who is not the friend of his friend is undoubtedly a rascal; for
one can only obtain friendship by giving it, or pretending to give
it.{]} and he is on the lookout for the signs of that affection. Do you
not see how you will acquire a fresh hold on him? What bands have you
bound about his heart while he was yet unaware of them! What will he
feel, when he beholds himself and sees what you have done for him; when
he can compare himself with other youths, and other tutors with you! I
say, ``When he sees it,'' but beware lest you tell him of it; if you
tell him he will not perceive it. If you claim his obedience in return
for the care bestowed upon him, he will think you have over-reached him;
he will see that while you profess to have cared for him without reward,
you meant to saddle him with a debt and to bind him to a bargain which
he never made. In vain you will add that what you demand is for his own
good; you demand it, and you demand it in virtue of what you have done
without his consent. When a man down on his luck accepts the shilling
which the sergeant professes to give him, and finds he has enlisted
without knowing what he was about, you protest against the injustice; is
it not still more unjust to demand from your pupil the price of care
which he has not even accepted!

Ingratitude would be rarer if kindness were less often the investment of
a usurer. We love those who have done us a kindness; what a natural
feeling! Ingratitude is not to be found in the heart of man, but
self-interest is there; those who are ungrateful for benefits received
are fewer than those who do a kindness for their own ends. If you sell
me your gifts, I will haggle over the price; but if you pretend to give,
in order to sell later on at your own price, you are guilty of fraud; it
is the free gift which is beyond price. The heart is a law to itself; if
you try to bind it, you lose it; give it its liberty, and you make it
your own.

When the fisherman baits his line, the fish come round him without
suspicion; but when they are caught on the hook concealed in the bait,
they feel the line tighten and they try to escape. Is the fisherman a
benefactor? Is the fish ungrateful? Do we find a man forgotten by his
benefactor, unmindful of that benefactor? On the contrary, he delights
to speak of him, he cannot think of him without emotion; if he gets a
chance of showing him, by some unexpected service, that he remembers
what he did for him, how delighted he is to satisfy his gratitude; what
a pleasure it is to earn the gratitude of his benefactor. How delightful
to say, ``It is my turn now.'' This is indeed the teaching of nature; a
good deed never caused ingratitude.

If therefore gratitude is a natural feeling, and you do not destroy its
effects by your blunders, be sure your pupil, as he begins to understand
the value of your care for him, will be grateful for it, provided you
have not put a price upon it; and this will give you an authority over
his heart which nothing can overthrow. But beware of losing this
advantage before it is really yours, beware of insisting on your own
importance. Boast of your services and they become intolerable; forget
them and they will not be forgotten. Until the time comes to treat him
as a man let there be no question of his duty to you, but his duty to
himself. Let him have his freedom if you would make him docile; hide
yourself so that he may seek you; raise his heart to the noble sentiment
of gratitude by only speaking of his own interest. Until he was able to
understand I would not have him told that what was done was for his
good; he would only have understood such words to mean that you were
dependent on him and he would merely have made you his servant. But now
that he is beginning to feel what love is, he also knows what a tender
affection may bind a man to what he loves; and in the zeal which keeps
you busy on his account, he now sees not the bonds of a slave, but the
affection of a friend. Now there is nothing which carries so much weight
with the human heart as the voice of friendship recognised as such, for
we know that it never speaks but for our good. We may think our friend
is mistaken, but we never believe he is deceiving us. We may reject his
advice now and then, but we never scorn it.

We have reached the moral order at last; we have just taken the second
step towards manhood. If this were the place for it, I would try to show
how the first impulses of the heart give rise to the first stirrings of
conscience, and how from the feelings of love and hatred spring the
first notions of good and evil. I would show that justice and kindness
are no mere abstract terms, no mere moral conceptions framed by the
understanding, but true affections of the heart enlightened by reason,
the natural outcome of our primitive affections; that by reason alone,
unaided by conscience, we cannot establish any natural law, and that all
natural right is a vain dream if it does not rest upon some instinctive
need of the human heart. {[}Footnote: The precept ``Do unto others as
you would have them do unto you'' has no true foundation but that of
conscience and feeling; for what valid reason is there why I, being
myself, should do what I would do if I were some one else, especially
when I am morally certain I never shall find myself in exactly the same
case; and who will answer for it that if I faithfully follow out this
maxim, I shall get others to follow it with regard to me? The wicked
takes advantage both of the uprightness of the just and of his own
injustice; he will gladly have everybody just but himself. This bargain,
whatever you may say, is not greatly to the advantage of the just. But
if the enthusiasm of an overflowing heart identifies me with my
fellow-creature, if I feel, so to speak, that I will not let him suffer
lest I should suffer too, I care for him because I care for myself, and
the reason of the precept is found in nature herself, which inspires me
with the desire for my own welfare wherever I may be. From this I
conclude that it is false to say that the precepts of natural law are
based on reason only; they have a firmer and more solid foundation. The
love of others springing from self-love, is the source of human justice.
The whole of morality is summed up in the gospel in this summary of the
law.{]} But I do not think it is my business at present to prepare
treatises on metaphysics and morals, nor courses of study of any kind
whatsoever; it is enough if I indicate the order and development of our
feelings and our knowledge in relation to our growth. Others will
perhaps work out what I have here merely indicated.

Hitherto my Emile has thought only of himself, so his first glance at
his equals leads him to compare himself with them; and the first feeling
excited by this comparison is the desire to be first. It is here that
self-love is transformed into selfishness, and this is the starting
point of all the passions which spring from selfishness. But to
determine whether the passions by which his life will be governed shall
be humane and gentle or harsh and cruel, whether they shall be the
passions of benevolence and pity or those of envy and covetousness, we
must know what he believes his place among men to be, and what sort of
obstacles he expects to have to overcome in order to attain to the
position he seeks.

To guide him in this inquiry, after we have shown him men by means of
the accidents common to the species, we must now show him them by means
of their differences. This is the time for estimating inequality natural
and civil, and for the scheme of the whole social order.

Society must be studied in the individual and the individual in society;
those who desire to treat politics and morals apart from one another
will never understand either. By confining ourselves at first to the
primitive relations, we see how men should be influenced by them and
what passions should spring from them; we see that it is in proportion
to the development of these passions that a man's relations with others
expand or contract. It is not so much strength of arm as moderation of
spirit which makes men free and independent. The man whose wants are few
is dependent on but few people, but those who constantly confound our
vain desires with our bodily needs, those who have made these needs the
basis of human society, are continually mistaking effects for causes,
and they have only confused themselves by their own reasoning.

Since it is impossible in the state of nature that the difference
between man and man should be great enough to make one dependent on
another, there is in fact in this state of nature an actual and
indestructible equality. In the civil state there is a vain and
chimerical equality of right; the means intended for its maintenance,
themselves serve to destroy it; and the power of the community, added to
the power of the strongest for the oppression of the weak, disturbs the
sort of equilibrium which nature has established between them.
{[}Footnote: The universal spirit of the laws of every country is always
to take the part of the strong against the weak, and the part of him who
has against him who has not; this defect is inevitable, and there is no
exception to it.{]} From this first contradiction spring all the other
contradictions between the real and the apparent, which are to be found
in the civil order. The many will always be sacrificed to the few, the
common weal to private interest; those specious words---justice and
subordination---will always serve as the tools of violence and the
weapons of injustice; hence it follows that the higher classes which
claim to be useful to the rest are really only seeking their own welfare
at the expense of others; from this we may judge how much consideration
is due to them according to right and justice. It remains to be seen if
the rank to which they have attained is more favourable to their own
happiness to know what opinion each one of us should form with regard to
his own lot. This is the study with which we are now concerned; but to
do it thoroughly we must begin with a knowledge of the human heart.

If it were only a question of showing young people man in his mask,
there would be no need to point him out, and he would always be before
their eyes; but since the mask is not the man, and since they must not
be led away by its specious appearance, when you paint men for your
scholar, paint them as they are, not that he may hate them, but that he
may pity them and have no wish to be like them. In my opinion that is
the most reasonable view a man can hold with regard to his fellow-men.

With this object in view we must take the opposite way from that
hitherto followed, and instruct the youth rather through the experience
of others than through his own. If men deceive him he will hate them;
but, if, while they treat him with respect, he sees them deceiving each
other, he will pity them. ``The spectacle of the world,'' said
Pythagoras, ``is like the Olympic games; some are buying and selling and
think only of their gains; others take an active part and strive for
glory; others, and these not the worst, are content to be lookers-on.''

I would have you so choose the company of a youth that he should think
well of those among whom he lives, and I would have you so teach him to
know the world that he should think ill of all that takes place in it.
Let him know that man is by nature good, let him feel it, let him judge
his neighbour by himself; but let him see how men are depraved and
perverted by society; let him find the source of all their vices in
their preconceived opinions; let him be disposed to respect the
individual, but to despise the multitude; let him see that all men wear
almost the same mask, but let him also know that some faces are fairer
than the mask that conceals them.

It must be admitted that this method has its drawbacks, and it is not
easy to carry it out; for if he becomes too soon engrossed in watching
other people, if you train him to mark too closely the actions of
others, you will make him spiteful and satirical, quick and decided in
his judgments of others; he will find a hateful pleasure in seeking bad
motives, and will fail to see the good even in that which is really
good. He will, at least, get used to the sight of vice, he will behold
the wicked without horror, just as we get used to seeing the wretched
without pity. Soon the perversity of mankind will be not so much a
warning as an excuse; he will say, ``Man is made so,'' and he will have
no wish to be different from the rest.

But if you wish to teach him theoretically to make him acquainted, not
only with the heart of man, but also with the application of the
external causes which turn our inclinations into vices; when you thus
transport him all at once from the objects of sense to the objects of
reason, you employ a system of metaphysics which he is not in a position
to understand; you fall back into the error, so carefully avoided
hitherto, of giving him lessons which are like lessons, of substituting
in his mind the experience and the authority of the master for his own
experience and the development of his own reason.

To remove these two obstacles at once, and to bring the human heart
within his reach without risk of spoiling his own, I would show him men
from afar, in other times or in other places, so that he may behold the
scene but cannot take part in it. This is the time for history; with its
help he will read the hearts of men without any lessons in philosophy;
with its help he will view them as a mere spectator, dispassionate and
without prejudice; he will view them as their judge, not as their
accomplice or their accuser.

To know men you must behold their actions. In society we hear them talk;
they show their words and hide their deeds; but in history the veil is
drawn aside, and they are judged by their deeds. Their sayings even help
us to understand them; for comparing what they say and what they do, we
see not only what they are but what they would appear; the more they
disguise themselves the more thoroughly they stand revealed.

Unluckily this study has its dangers, its drawbacks of several kinds. It
is difficult to adopt a point of view which will enable one to judge
one's fellow-creatures fairly. It is one of the chief defects of history
to paint men's evil deeds rather than their good ones; it is revolutions
and catastrophes that make history interesting; so long as a nation
grows and prospers quietly in the tranquillity of a peaceful government,
history says nothing; she only begins to speak of nations when, no
longer able to be self-sufficing, they interfere with their neighbours'
business, or allow their neighbours to interfere with their own; history
only makes them famous when they are on the downward path; all our
histories begin where they ought to end. We have very accurate accounts
of declining nations; what we lack is the history of those nations which
are multiplying; they are so happy and so good that history has nothing
to tell us of them; and we see indeed in our own times that the most
successful governments are least talked of. We only hear what is bad;
the good is scarcely mentioned. Only the wicked become famous, the good
are forgotten or laughed to scorn, and thus history, like philosophy, is
for ever slandering mankind.

Moreover, it is inevitable that the facts described in history should
not give an exact picture of what really happened; they are transformed
in the brain of the historian, they are moulded by his interests and
coloured by his prejudices. Who can place the reader precisely in a
position to see the event as it really happened? Ignorance or partiality
disguises everything. What a different impression may be given merely by
expanding or contracting the circumstances of the case without altering
a single historical incident. The same object may be seen from several
points of view, and it will hardly seem the same thing, yet there has
been no change except in the eye that beholds it. Do you indeed do
honour to truth when what you tell me is a genuine fact, but you make it
appear something quite different? A tree more or less, a rock to the
right or to the left, a cloud of dust raised by the wind, how often have
these decided the result of a battle without any one knowing it? Does
that prevent history from telling you the cause of defeat or victory
with as much assurance as if she had been on the spot? But what are the
facts to me, while I am ignorant of their causes, and what lessons can I
draw from an event, whose true cause is unknown to me? The historian
indeed gives me a reason, but he invents it; and criticism itself, of
which we hear so much, is only the art of guessing, the art of choosing
from among several lies, the lie that is most like truth.

Have you ever read Cleopatra or Cassandra or any books of the kind? The
author selects some well-known event, he then adapts it to his purpose,
adorns it with details of his own invention, with people who never
existed, with imaginary portraits; thus he piles fiction on fiction to
lend a charm to his story. I see little difference between such romances
and your histories, unless it is that the novelist draws more on his own
imagination, while the historian slavishly copies what another has
imagined; I will also admit, if you please, that the novelist has some
moral purpose good or bad, about which the historian scarcely concerns
himself.

You will tell me that accuracy in history is of less interest than a
true picture of men and manners; provided the human heart is truly
portrayed, it matters little that events should be accurately recorded;
for after all you say, what does it matter to us what happened two
thousand years ago? You are right if the portraits are indeed truly
given according to nature; but if the model is to be found for the most
part in the historian's imagination, are you not falling into the very
error you intended to avoid, and surrendering to the authority of the
historian what you would not yield to the authority of the teacher? If
my pupil is merely to see fancy pictures, I would rather draw them
myself; they will, at least, be better suited to him.

The worst historians for a youth are those who give their opinions.
Facts! Facts! and let him decide for himself; this is how he will learn
to know mankind. If he is always directed by the opinion of the author,
he is only seeing through the eyes of another person, and when those
ayes are no longer at his disposal he can see nothing.

I leave modern history on one side, not only because it has no character
and all our people are alike, but because our historians, wholly taken
up with effect, think of nothing but highly coloured portraits, which
often represent nothing. {[}Footnote: Take, for instance, Guicciardini,
Streda, Solis, Machiavelli, and sometimes even De Thou himself. Vertot
is almost the only one who knows how to describe without giving fancy
portraits.{]} The old historians generally give fewer portraits and
bring more intelligence and common-sense to their judgments; but even
among them there is plenty of scope for choice, and you must not begin
with the wisest but with the simplest. I would not put Polybius or
Sallust into the hands of a youth; Tacitus is the author of the old,
young men cannot understand him; you must learn to see in human actions
the simplest features of the heart of man before you try to sound its
depths. You must be able to read facts clearly before you begin to study
maxims. Philosophy in the form of maxims is only fit for the
experienced. Youth should never deal with the general, all its teaching
should deal with individual instances.

To my mind Thucydides is the true model of historians. He relates facts
without giving his opinion; but he omits no circumstance adapted to make
us judge for ourselves. He puts everything that he relates before his
reader; far from interposing between the facts and the readers, he
conceals himself; we seem not to read but to see. Unfortunately he
speaks of nothing but war, and in his stories we only see the least
instructive part of the world, that is to say the battles. The virtues
and defects of the Retreat of the Ten Thousand and the Commentaries of
Caesar are almost the same. The kindly Herodotus, without portraits,
without maxims, yet flowing, simple, full of details calculated to
delight and interest in the highest degree, would be perhaps the best
historian if these very details did not often degenerate into childish
folly, better adapted to spoil the taste of youth than to form it; we
need discretion before we can read him. I say nothing of Livy, his turn
will come; but he is a statesman, a rhetorician, he is everything which
is unsuitable for a youth.

History in general is lacking because it only takes note of striking and
clearly marked facts which may be fixed by names, places, and dates; but
the slow evolution of these facts, which cannot be definitely noted in
this way, still remains unknown. We often find in some battle, lost or
won, the ostensible cause of a revolution which was inevitable before
this battle took place. War only makes manifest events already
determined by moral causes, which few historians can perceive.

The philosophic spirit has turned the thoughts of many of the historians
of our times in this direction; but I doubt whether truth has profited
by their labours. The rage for systems has got possession of all alike,
no one seeks to see things as they are, but only as they agree with his
system.

Add to all these considerations the fact that history shows us actions
rather than men, because she only seizes men at certain chosen times in
full dress; she only portrays the statesman when he is prepared to be
seen; she does not follow him to his home, to his study, among his
family and his friends; she only shows him in state; it is his clothes
rather than himself that she describes.

I would prefer to begin the study of the human heart with reading the
lives of individuals; for then the man hides himself in vain, the
historian follows him everywhere; he never gives him a moment's grace
nor any corner where he can escape the piercing eye of the spectator;
and when he thinks he is concealing himself, then it is that the writer
shows him up most plainly.

``Those who write lives,'' says Montaigne, ``in so far as they delight
more in ideas than in events, more in that which comes from within than
in that which comes from without, these are the writers I prefer; for
this reason Plutarch is in every way the man for me.''

It is true that the genius of men in groups or nations is very different
from the character of the individual man, and that we have a very
imperfect knowledge of the human heart if we do not also examine it in
crowds; but it is none the less true that to judge of men we must study
the individual man, and that he who had a perfect knowledge of the
inclinations of each individual might foresee all their combined effects
in the body of the nation.

We must go back again to the ancients, for the reasons already stated,
and also because all the details common and familiar, but true and
characteristic, are banished by modern stylists, so that men are as much
tricked out by our modern authors in their private life as in public.
Propriety, no less strict in literature than in life, no longer permits
us to say anything in public which we might not do in public; and as we
may only show the man dressed up for his part, we never see a man in our
books any more than we do on the stage. The lives of kings may be
written a hundred times, but to no purpose; we shall never have another
Suetonius.

The excellence of Plutarch consists in these very details which we are
no longer permitted to describe. With inimitable grace he paints the
great man in little things; and he is so happy in the choice of his
instances that a word, a smile, a gesture, will often suffice to
indicate the nature of his hero. With a jest Hannibal cheers his
frightened soldiers, and leads them laughing to the battle which will
lay Italy at his feet; Agesilaus riding on a stick makes me love the
conqueror of the great king; Caesar passing through a poor village and
chatting with his friends unconsciously betrays the traitor who
professed that he only wished to be Pompey's equal. Alexander swallows a
draught without a word---it is the finest moment in his life; Aristides
writes his own name on the shell and so justifies his title;
Philopoemen, his mantle laid aside, chops firewood in the kitchen of his
host. This is the true art of portraiture. Our disposition does not show
itself in our features, nor our character in our great deeds; it is
trifles that show what we really are. What is done in public is either
too commonplace or too artificial, and our modern authors are almost too
grand to tell us anything else.

M. de Turenne was undoubtedly one of the greatest men of the last
century. They have had the courage to make his life interesting by the
little details which make us know and love him; but how many details
have they felt obliged to omit which might have made us know and love
him better still? I will only quote one which I have on good authority,
one which Plutarch would never have omitted, and one which Ramsai would
never have inserted had he been acquainted with it.

On a hot summer's day Viscount Turenne in a little white vest and
nightcap was standing at the window of his antechamber; one of his men
came up and, misled by the dress, took him for one of the kitchen lads
whom he knew. He crept up behind him and smacked him with no light hand.
The man he struck turned round hastily. The valet saw it was his master
and trembled at the sight of his face. He fell on his knees in
desperation. ``Sir, I thought it was George.'' ``Well, even if it was
George,'' exclaimed Turenne rubbing the injured part, ``you need not
have struck so hard.'' You do not dare to say this, you miserable
writers! Remain for ever without humanity and without feeling; steel
your hard hearts in your vile propriety, make yourselves contemptible
through your high-mightiness. But as for you, dear youth, when you read
this anecdote, when you are touched by all the kindliness displayed even
on the impulse of the moment, read also the littleness of this great man
when it was a question of his name and birth. Remember it was this very
Turenne who always professed to yield precedence to his nephew, so that
all men might see that this child was the head of a royal house. Look on
this picture and on that, love nature, despise popular prejudice, and
know the man as he was.

There are few people able to realise what an effect such reading,
carefully directed, will have upon the unspoilt mind of a youth. Weighed
down by books from our earliest childhood, accustomed to read without
thinking, what we read strikes us even less, because we already bear in
ourselves the passions and prejudices with which history and the lives
of men are filled; all that they do strikes us as only natural, for we
ourselves are unnatural and we judge others by ourselves. But imagine my
Emile, who has been carefully guarded for eighteen years with the sole
object of preserving a right judgment and a healthy heart, imagine him
when the curtain goes up casting his eyes for the first time upon the
world's stage; or rather picture him behind the scenes watching the
actors don their costumes, and counting the cords and pulleys which
deceive with their feigned shows the eyes of the spectators. His first
surprise will soon give place to feelings of shame and scorn of his
fellow-man; he will be indignant at the sight of the whole human race
deceiving itself and stooping to this childish folly; he will grieve to
see his brothers tearing each other limb from limb for a mere dream, and
transforming themselves into wild beasts because they could not be
content to be men.

Given the natural disposition of the pupil, there is no doubt that if
the master exercises any sort of prudence or discretion in his choice of
reading, however little he may put him in the way of reflecting on the
subject-matter, this exercise will serve as a course in practical
philosophy, a philosophy better understood and more thoroughly mastered
than all the empty speculations with which the brains of lads are
muddled in our schools. After following the romantic schemes of Pyrrhus,
Cineas asks him what real good he would gain by the conquest of the
world, which he can never enjoy without such great sufferings; this only
arouses in us a passing interest as a smart saying; but Emile will think
it a very wise thought, one which had already occurred to himself, and
one which he will never forget, because there is no hostile prejudice in
his mind to prevent it sinking in. When he reads more of the life of
this madman, he will find that all his great plans resulted in his death
at the hands of a woman, and instead of admiring this pinchbeck heroism,
what will he see in the exploits of this great captain and the schemes
of this great statesman but so many steps towards that unlucky tile
which was to bring life and schemes alike to a shameful death?

All conquerors have not been killed; all usurpers have not failed in
their plans; to minds imbued with vulgar prejudices many of them will
seem happy, but he who looks below the surface and reckons men's
happiness by the condition of their hearts will perceive their
wretchedness even in the midst of their successes; he will see them
panting after advancement and never attaining their prize, he will find
them like those inexperienced travellers among the Alps, who think that
every height they see is the last, who reach its summit only to find to
their disappointment there are loftier peaks beyond.

Augustus, when he had subdued his fellow-citizens and destroyed his
rivals, reigned for forty years over the greatest empire that ever
existed; but all this vast power could not hinder him from beating his
head against the walls, and filling his palace with his groans as he
cried to Varus to restore his slaughtered legions. If he had conquered
all his foes what good would his empty triumphs have done him, when
troubles of every kind beset his path, when his life was threatened by
his dearest friends, and when he had to mourn the disgrace or death of
all near and dear to him? The wretched man desired to rule the world and
failed to rule his own household. What was the result of this neglect?
He beheld his nephew, his adopted child, his son-in-law, perish in the
flower of youth, his grandson reduced to eat the stuffing of his
mattress to prolong his wretched existence for a few hours; his daughter
and his granddaughter, after they had covered him with infamy, died, the
one of hunger and want on a desert island, the other in prison by the
hand of a common archer. He himself, the last survivor of his unhappy
house, found himself compelled by his own wife to acknowledge a monster
as his heir. Such was the fate of the master of the world, so famous for
his glory and his good fortune. I cannot believe that any one of those
who admire his glory and fortune would accept them at the same price.

I have taken ambition as my example, but the play of every human passion
offers similar lessons to any one who will study history to make himself
wise and good at the expense of those who went before. The time is
drawing near when the teaching of the life of Anthony will appeal more
forcibly to the youth than the life of Augustus. Emile will scarcely
know where he is among the many strange sights in his new studies; but
he will know beforehand how to avoid the illusion of passions before
they arise, and seeing how in all ages they have blinded men's eyes, he
will be forewarned of the way in which they may one day blind his own
should he abandon himself to them. {[}Footnote: It is always prejudice
which stirs up passion in our heart. He who only sees what really exists
and only values what he knows, rarely becomes angry. The errors of our
judgment produce the warmth of our desires.{]} These lessons, I know,
are unsuited to him, perhaps at need they may prove scanty and
ill-timed; but remember they are not the lessons I wished to draw from
this study. To begin with, I had quite another end in view; and indeed,
if this purpose is unfulfilled, the teacher will be to blame.

Remember that, as soon as selfishness has developed, the self in its
relations to others is always with us, and the youth never observes
others without coming back to himself and comparing himself with them.
From the way young men are taught to study history I see that they are
transformed, so to speak, into the people they behold, that you strive
to make a Cicero, a Trajan, or an Alexander of them, to discourage them
when they are themselves again, to make every one regret that he is
merely himself. There are certain advantages in this plan which I do not
deny; but, so far as Emile is concerned, should it happen at any time
when he is making these comparisons that he wishes to be any one but
himself---were it Socrates or Cato---I have failed entirely; he who
begins to regard himself as a stranger will soon forget himself
altogether.

It is not philosophers who know most about men; they only view them
through the preconceived ideas of philosophy, and I know no one so
prejudiced as philosophers. A savage would judge us more sanely. The
philosopher is aware of his own vices, he is indignant at ours, and he
says to himself, ``We are all bad alike;'' the savage beholds us unmoved
and says, ``You are mad.'' He is right, for no one does evil for evil's
sake. My pupil is that savage, with this difference: Emile has thought
more, he has compared ideas, seen our errors at close quarters, he is
more on his guard against himself, and only judges of what he knows.

It is our own passions that excite us against the passions of others; it
is our self-interest which makes us hate the wicked; if they did us no
harm we should pity rather than hate them. We should readily forgive
their vices if we could perceive how their own heart punishes those
vices. We are aware of the offence, but we do not see the punishment;
the advantages are plain, the penalty is hidden. The man who thinks he
is enjoying the fruits of his vices is no less tormented by them than if
they had not been successful; the object is different, the anxiety is
the same; in vain he displays his good fortune and hides his heart; in
spite of himself his conduct betrays him; but to discern this, our own
heart must be utterly unlike his.

We are led astray by those passions which we share; we are disgusted by
those that militate against our own interests; and with a want of logic
due to these very passions, we blame in others what we fain would
imitate. Aversion and self-deception are inevitable when we are forced
to endure at another's hands what we ourselves would do in his place.

What then is required for the proper study of men? A great wish to know
men, great impartiality of judgment, a heart sufficiently sensitive to
understand every human passion, and calm enough to be free from passion.
If there is any time in our life when this study is likely to be
appreciated, it is this that I have chosen for Emile; before this time
men would have been strangers to him; later on he would have been like
them. Convention, the effects of which he already perceives, has not yet
made him its slave, the passions, whose consequences he realises, have
not yet stirred his heart. He is a man; he takes an interest in his
brethren; he is a just man and he judges his peers. Now it is certain
that if he judges them rightly he will not want to change places with
any one of them, for the goal of all their anxious efforts is the result
of prejudices which he does not share, and that goal seems to him a mere
dream. For his own part, he has all he wants within his reach. How
should he be dependent on any one when he is self-sufficing and free
from prejudice? Strong arms, good health, {[}Footnote: I think I may
fairly reckon health and strength among the advantages he has obtained
by his education, or rather among the gifts of nature which his
education has preserved for him.{]} moderation, few needs, together with
the means to satisfy those needs, are his. He has been brought up in
complete liberty and servitude is the greatest ill he understands. He
pities these miserable kings, the slaves of all who obey them; he pities
these false prophets fettered by their empty fame; he pities these rich
fools, martyrs to their own pomp; he pities these ostentatious
voluptuaries, who spend their life in deadly dullness that they may seem
to enjoy its pleasures. He would pity the very foe who harmed him, for
he would discern his wretchedness beneath his cloak of spite. He would
say to himself, ``This man has yielded to his desire to hurt me, and
this need of his places him at my mercy.''

One step more and our goal is attained. Selfishness is a dangerous tool
though a useful one; it often wounds the hand that uses it, and it
rarely does good unmixed with evil. When Emile considers his place among
men, when he finds himself so fortunately situated, he will be tempted
to give credit to his own reason for the work of yours, and to attribute
to his own deserts what is really the result of his good fortune. He
will say to himself, ``I am wise and other men are fools.'' He will pity
and despise them and will congratulate himself all the more heartily;
and as he knows he is happier than they, he will think his deserts are
greater. This is the fault we have most to fear, for it is the most
difficult to eradicate. If he remained in this state of mind, he would
have profited little by all our care; and if I had to choose, I hardly
know whether I would not rather choose the illusions of prejudice than
those of pride.

Great men are under no illusion with respect to their superiority; they
see it and know it, but they are none the less modest. The more they
have, the better they know what they lack. They are less vain of their
superiority over us than ashamed by the consciousness of their weakness,
and among the good things they really possess, they are too wise to
pride themselves on a gift which is none of their getting. The good man
may be proud of his virtue for it is his own, but what cause for pride
has the man of intellect? What has Racine done that he is not Pradon,
and Boileau that he is not Cotin?

The circumstances with which we are concerned are quite different. Let
us keep to the common level. I assumed that my pupil had neither
surpassing genius nor a defective understanding. I chose him of an
ordinary mind to show what education could do for man. Exceptions defy
all rules. If, therefore, as a result of my care, Emile prefers his way
of living, seeing, and feeling to that of others, he is right; but if he
thinks because of this that he is nobler and better born than they, he
is wrong; he is deceiving himself; he must be undeceived, or rather let
us prevent the mistake, lest it be too late to correct it.

Provided a man is not mad, he can be cured of any folly but vanity;
there is no cure for this but experience, if indeed there is any cure
for it at all; when it first appears we can at least prevent its further
growth. But do not on this account waste your breath on empty arguments
to prove to the youth that he is like other men and subject to the same
weaknesses. Make him feel it or he will never know it. This is another
instance of an exception to my own rules; I must voluntarily expose my
pupil to every accident which may convince him that he is no wiser than
we. The adventure with the conjurer will be repeated again and again in
different ways; I shall let flatterers take advantage of him; if rash
comrades draw him into some perilous adventure, I will let him run the
risk; if he falls into the hands of sharpers at the card-table, I will
abandon him to them as their dupe.{[}Footnote: Moreover our pupil will
be little tempted by this snare; he has so many amusements about him, he
has never been bored in his life, and he scarcely knows the use of
money. As children have been led by these two motives, self-interest and
vanity, rogues and courtesans use the same means to get hold of them
later. When you see their greediness encouraged by prizes and rewards,
when you find their public performances at ten years old applauded at
school or college, you see too how at twenty they will be induced to
leave their purse in a gambling hell and their health in a worse place.
You may safely wager that the sharpest boy in the class will become the
greatest gambler and debauchee. Now the means which have not been
employed in childhood have not the same effect in youth. But we must
bear in mind my constant plan and take the thing at its worst. First I
try to prevent the vice; then I assume its existence in order to correct
it.{]} I will let them flatter him, pluck him, and rob him; and when
having sucked him dry they turn and mock him, I will even thank them to
his face for the lessons they have been good enough to give him. The
only snares from which I will guard him with my utmost care are the
wiles of wanton women. The only precaution I shall take will be to share
all the dangers I let him run, and all the insults I let him receive. I
will bear everything in silence, without a murmur or reproach, without a
word to him, and be sure that if this wise conduct is faithfully adhered
to, what he sees me endure on his account will make more impression on
his heart than what he himself suffers.

I cannot refrain at this point from drawing attention to the sham
dignity of tutors, who foolishly pretend to be wise, who discourage
their pupils by always professing to treat them as children, and by
emphasising the difference between themselves and their scholars in
everything they do. Far from damping their youthful spirits in this
fashion, spare no effort to stimulate their courage; that they may
become your equals, treat them as such already, and if they cannot rise
to your level, do not scruple to come down to theirs without being
ashamed of it. Remember that your honour is no longer in your own
keeping but in your pupil's. Share his faults that you may correct them,
bear his disgrace that you may wipe it out; follow the example of that
brave Roman who, unable to rally his fleeing soldiers, placed himself at
their head, exclaiming, ``They do not flee, they follow their captain!''
Did this dishonour him? Not so; by sacrificing his glory he increased
it. The power of duty, the beauty of virtue, compel our respect in spite
of all our foolish prejudices. If I received a blow in the course of my
duties to Emile, far from avenging it I would boast of it; and I doubt
whether there is in the whole world a man so vile as to respect me any
the less on this account.

I do not intend the pupil to suppose his master to be as ignorant, or as
liable to be led astray, as he is himself. This idea is all very well
for a child who can neither see nor compare things, who thinks
everything is within his reach, and only bestows his confidence on those
who know how to come down to his level. But a youth of Emile's age and
sense is no longer so foolish as to make this mistake, and it would not
be desirable that he should. The confidence he ought to have in his
tutor is of another kind; it should rest on the authority of reason, and
on superior knowledge, advantages which the young man is capable of
appreciating while he perceives how useful they are to himself. Long
experience has convinced him that his tutor loves him, that he is a wise
and good man who desires his happiness and knows how to procure it. He
ought to know that it is to his own advantage to listen to his advice.
But if the master lets himself be taken in like the disciple, he will
lose his right to expect deference from him, and to give him
instruction. Still less should the pupil suppose that his master is
purposely letting him fall into snares or preparing pitfalls for his
inexperience. How can we avoid these two difficulties? Choose the best
and most natural means; be frank and straightforward like himself; warn
him of the dangers to which he is exposed, point them out plainly and
sensibly, without exaggeration, without temper, without pedantic
display, and above all without giving your opinions in the form of
orders, until they have become such, and until this imperious tone is
absolutely necessary. Should he still be obstinate as he often will be,
leave him free to follow his own choice, follow him, copy his example,
and that cheerfully and frankly; if possible fling yourself into things,
amuse yourself as much as he does. If the consequences become too
serious, you are at hand to prevent them; and yet when this young man
has beheld your foresight and your kindliness, will he not be at once
struck by the one and touched by the other? All his faults are but so
many hands with which he himself provides you to restrain him at need.
Now under these circumstances the great art of the master consists in
controlling events and directing his exhortations so that he may know
beforehand when the youth will give in, and when he will refuse to do
so, so that all around him he may encompass him with the lessons of
experience, and yet never let him run too great a risk.

Warn him of his faults before he commits them; do not blame him when
once they are committed; you would only stir his self-love to mutiny. We
learn nothing from a lesson we detest. I know nothing more foolish than
the phrase, ``I told you so.'' The best way to make him remember what
you told him is to seem to have forgotten it. Go further than this, and
when you find him ashamed of having refused to believe you, gently
smooth away the shame with kindly words. He will indeed hold you dear
when he sees how you forget yourself on his account, and how you console
him instead of reproaching him. But if you increase his annoyance by
your reproaches he will hate you, and will make it a rule never to heed
you, as if to show you that he does not agree with you as to the value
of your opinion.

The turn you give to your consolation may itself be a lesson to him, and
all the more because he does not suspect it. When you tell him, for
example, that many other people have made the same mistakes, this is not
what he was expecting; you are administering correction under the guise
of pity; for when one thinks oneself better than other people it is a
very mortifying excuse to console oneself by their example; it means
that we must realise that the most we can say is that they are no better
than we.

The time of faults is the time for fables. When we blame the guilty
under the cover of a story we instruct without offending him; and he
then understands that the story is not untrue by means of the truth he
finds in its application to himself. The child who has never been
deceived by flattery understands nothing of the fable I recently
examined; but the rash youth who has just become the dupe of a flatterer
perceives only too readily that the crow was a fool. Thus he acquires a
maxim from the fact, and the experience he would soon have forgotten is
engraved on his mind by means of the fable. There is no knowledge of
morals which cannot be acquired through our own experience or that of
others. When there is danger, instead of letting him try the experiment
himself, we have recourse to history. When the risk is comparatively
slight, it is just as well that the youth should be exposed to it; then
by means of the apologue the special cases with which the young man is
now acquainted are transformed into maxims.

It is not, however, my intention that these maxims should be explained,
nor even formulated. Nothing is so foolish and unwise as the moral at
the end of most of the fables; as if the moral was not, or ought not to
be so clear in the fable itself that the reader cannot fail to perceive
it. Why then add the moral at the end, and go deprive him of the
pleasure of discovering it for himself. The art of teaching consists in
making the pupil wish to learn. But if the pupil is to wish to learn,
his mind must not remain in such a passive state with regard to what you
tell him that there is really nothing for him to do but listen to you.
The master's vanity must always give way to the scholars; he must be
able to say, I understand, I see it, I am getting at it, I am learning
something. One of the things which makes the Pantaloon in the Italian
comedies so wearisome is the pains taken by him to explain to the
audience the platitudes they understand only too well already. We must
always be intelligible, but we need not say all there is to be said. If
you talk much you will say little, for at last no one will listen to
you. What is the sense of the four lines at the end of La Fontaine's
fable of the frog who puffed herself up. Is he afraid we should not
understand it? Does this great painter need to write the names beneath
the things he has painted? His morals, far from generalising, restrict
the lesson to some extent to the examples given, and prevent our
applying them to others. Before I put the fables of this inimitable
author into the hands of a youth, I should like to cut out all the
conclusions with which he strives to explain what he has just said so
clearly and pleasantly. If your pupil does not understand the fable
without the explanation, he will not understand it with it.

Moreover, the fables would require to be arranged in a more didactic
order, one more in agreement with the feelings and knowledge of the
young adolescent. Can you imagine anything so foolish as to follow the
mere numerical order of the book without regard to our requirements or
our opportunities. First the grasshopper, then the crow, then the frog,
then the two mules, etc. I am sick of these two mules; I remember seeing
a child who was being educated for finance; they never let him alone,
but were always insisting on the profession he was to follow; they made
him read this fable, learn it, say it, repeat it again and again without
finding in it the slightest argument against his future calling. Not
only have I never found children make any real use of the fables they
learn, but I have never found anybody who took the trouble to see that
they made such a use of them. The study claims to be instruction in
morals; but the real aim of mother and child is nothing but to set a
whole party watching the child while he recites his fables; when he is
too old to recite them and old enough to make use of them, they are
altogether forgotten. Only men, I repeat, can learn from fables, and
Emile is now old enough to begin.

I do not mean to tell you everything, so I only indicate the paths which
diverge from the right way, so that you may know how to avoid them. If
you follow the road I have marked out for you, I think your pupil will
buy his knowledge of mankind and his knowledge of himself in the
cheapest market; you will enable him to behold the tricks of fortune
without envying the lot of her favourites, and to be content with
himself without thinking himself better than others. You have begun by
making him an actor that he may learn to be one of the audience; you
must continue your task, for from the theatre things are what they seem,
from the stage they seem what they are. For the general effect we must
get a distant view, for the details we must observe more closely. But
how can a young man take part in the business of life? What right has he
to be initiated into its dark secrets? His interests are confined within
the limits of his own pleasures, he has no power over others, it is much
the same as if he had no power at all. Man is the cheapest commodity on
the market, and among all our important rights of property, the rights
of the individual are always considered last of all.

When I see the studies of young men at the period of their greatest
activity confined to purely speculative matters, while later on they are
suddenly plunged, without any sort of experience, into the world of men
and affairs, it strikes me as contrary alike to reason and to nature,
and I cease to be surprised that so few men know what to do. How strange
a choice to teach us so many useless things, while the art of doing is
never touched upon! They profess to fit us for society, and we are
taught as if each of us were to live a life of contemplation in a
solitary cell, or to discuss theories with persons whom they did not
concern. You think you are teaching your scholars how to live, and you
teach them certain bodily contortions and certain forms of words without
meaning. I, too, have taught Emile how to live; for I have taught him to
enjoy his own society and, more than that, to earn his own bread. But
this is not enough. To live in the world he must know how to get on with
other people, he must know what forces move them, he must calculate the
action and re-action of self-interest in civil society, he must estimate
the results so accurately that he will rarely fail in his undertakings,
or he will at least have tried in the best possible way. The law does
not allow young people to manage their own affairs nor to dispose of
their own property; but what would be the use of these precautions if
they never gained any experience until they were of age. They would have
gained nothing by the delay, and would have no more experience at
five-and-twenty than at fifteen. No doubt we must take precautions, so
that a youth, blinded by ignorance or misled by passion, may not hurt
himself; but at any age there are opportunities when deeds of kindness
and of care for the weak may be performed under the direction of a wise
man, on behalf of the unfortunate who need help.

Mothers and nurses grow fond of children because of the care they lavish
on them; the practice of social virtues touches the very heart with the
love of humanity; by doing good we become good; and I know no surer way
to this end. Keep your pupil busy with the good deeds that are within
his power, let the cause of the poor be his own, let him help them not
merely with his money, but with his service; let him work for them,
protect them, let his person and his time be at their disposal; let him
be their agent; he will never all his life long have a more honourable
office. How many of the oppressed, who have never got a hearing, will
obtain justice when he demands it for them with that courage and
firmness which the practice of virtue inspires; when he makes his way
into the presence of the rich and great, when he goes, if need be, to
the footstool of the king himself, to plead the cause of the wretched,
the cause of those who find all doors closed to them by their poverty,
those who are so afraid of being punished for their misfortunes that
they do not dare to complain?

But shall we make of Emile a knight-errant, a redresser of wrongs, a
paladin? Shall he thrust himself into public life, play the sage and the
defender of the laws before the great, before the magistrates, before
the king? Shall he lay petitions before the judges and plead in the law
courts? That I cannot say. The nature of things is not changed by terms
of mockery and scorn. He will do all that he knows to be useful and
good. He will do nothing more, and he knows that nothing is useful and
good for him which is unbefitting his age. He knows that his first duty
is to himself; that young men should distrust themselves; that they
should act circumspectly; that they should show respect to those older
than themselves, reticence and discretion in talking without cause,
modesty in things indifferent, but courage in well doing, and boldness
to speak the truth. Such were those illustrious Romans who, having been
admitted into public life, spent their days in bringing criminals to
justice and in protecting the innocent, without any motives beyond those
of learning, and of the furtherance of justice and of the protection of
right conduct.

Emile is not fond of noise or quarrelling, not only among men, but among
animals. {[}Footnote: ``But what will he do if any one seeks a quarrel
with him?'' My answer is that no one will ever quarrel with him, he will
never lend himself to such a thing. But, indeed, you continue, who can
be safe from a blow, or an insult from a bully, a drunkard, a bravo, who
for the joy of killing his man begins by dishonouring him? That is
another matter. The life and honour of the citizens should not be at the
mercy of a bully, a drunkard, or a bravo, and one can no more insure
oneself against such an accident than against a falling tile. A blow
given, or a lie in the teeth, if he submit to them, have social
consequences which no wisdom can prevent and no tribunal can avenge. The
weakness of the laws, therefore, so far restores a man's independence;
he is the sole magistrate and judge between the offender and himself,
the sole interpreter and administrator of natural law. Justice is his
due, and he alone can obtain it, and in such a case there is no
government on earth so foolish as to punish him for so doing. I do not
say he must fight; that is absurd; I say justice is his due, and he
alone can dispense it. If I were king, I promise you that in my kingdom
no one would ever strike a man or call him a liar, and yet I would do
without all those useless laws against duels; the means are simple and
require no law courts. However that may be, Emile knows what is due to
himself in such a case, and the example due from him to the safety of
men of honour. The strongest of men cannot prevent insult, but he can
take good care that his adversary has no opportunity to boast of that
insult.{]} He will never set two dogs to fight, he will never set a dog
to chase a cat. This peaceful spirit is one of the results of his
education, which has never stimulated self-love or a high opinion of
himself, and so has not encouraged him to seek his pleasure in
domination and in the sufferings of others. The sight of suffering makes
him suffer too; this is a natural feeling. It is one of the after
effects of vanity that hardens a young man and makes him take a delight
in seeing the torments of a living and feeling creature; it makes him
consider himself beyond the reach of similar sufferings through his
superior wisdom or virtue. He who is beyond the reach of vanity cannot
fall into the vice which results from vanity. So Emile loves peace. He
is delighted at the sight of happiness, and if he can help to bring it
about, this is an additional reason for sharing it. I do not assume that
when he sees the unhappy he will merely feel for them that barren and
cruel pity which is content to pity the ills it can heal. His kindness
is active and teaches him much he would have learnt far more slowly, or
he would never have learnt at all, if his heart had been harder. If he
finds his comrades at strife, he tries to reconcile them; if he sees the
afflicted, he inquires as to the cause of their sufferings; if he meets
two men who hate each other, he wants to know the reason of their
enmity; if he finds one who is down-trodden groaning under the
oppression of the rich and powerful, he tries to discover by what means
he can counteract this oppression, and in the interest he takes with
regard to all these unhappy persons, the means of removing their
sufferings are never out of his sight. What use shall we make of this
disposition so that it may re-act in a way suited to his age? Let us
direct his efforts and his knowledge, and use his zeal to increase them.

I am never weary of repeating: let all the lessons of young people take
the form of doing rather than talking; let them learn nothing from books
which they can learn from experience. How absurd to attempt to give them
practice in speaking when they have nothing to say, to expect to make
them feel, at their school desks, the vigour of the language of passion
and all the force of the arts of persuasion when they have nothing and
nobody to persuade! All the rules of rhetoric are mere waste of words to
those who do not know how to use them for their own purposes. How does
it concern a schoolboy to know how Hannibal encouraged his soldiers to
cross the Alps? If instead of these grand speeches you showed him how to
induce his prefect to give him a holiday, you may be sure he would pay
more attention to your rules.

If I wanted to teach rhetoric to a youth whose passions were as yet
undeveloped, I would draw his attention continually to things that would
stir his passions, and I would discuss with him how he should talk to
people so as to get them to regard his wishes favourably. But Emile is
not in a condition so favourable to the art of oratory. Concerned mainly
with his physical well-being, he has less need of others than they of
him; and having nothing to ask of others on his own account, what he
wants to persuade them to do does not affect him sufficiently to awake
any very strong feeling. From this it follows that his language will be
on the whole simple and literal. He usually speaks to the point and only
to make himself understood. He is not sententious, for he has not learnt
to generalise; he does not speak in figures, for he is rarely
impassioned.

Yet this is not because he is altogether cold and phlegmatic, neither
his age, his character, nor his tastes permit of this. In the fire of
adolescence the life-giving spirits, retained in the blood and distilled
again and again, inspire his young heart with a warmth which glows in
his eye, a warmth which is felt in his words and perceived in his
actions. The lofty feeling with which he is inspired gives him strength
and nobility; imbued with tender love for mankind his words betray the
thoughts of his heart; I know not how it is, but there is more charm in
his open-hearted generosity than in the artificial eloquence of others;
or rather this eloquence of his is the only true eloquence, for he has
only to show what he feels to make others share his feelings.

The more I think of it the more convinced I am that by thus translating
our kindly impulses into action, by drawing from our good or ill success
conclusions as to their cause, we shall find that there is little useful
knowledge that cannot be imparted to a youth; and that together with
such true learning as may be got at college he will learn a science of
more importance than all the rest together, the application of what he
has learned to the purposes of life. Taking such an interest in his
fellow-creatures, it is impossible that he should fail to learn very
quickly how to note and weigh their actions, their tastes, their
pleasures, and to estimate generally at their true value what may
increase or diminish the happiness of men; he should do this better than
those who care for nobody and never do anything for any one. The
feelings of those who are always occupied with their own concerns are
too keenly affected for them to judge wisely of things. They consider
everything as it affects themselves, they form their ideas of good and
ill solely on their own experience, their minds are filled with all
sorts of absurd prejudices, and anything which affects their own
advantage ever so little, seems an upheaval of the universe.

Extend self-love to others and it is transformed into virtue, a virtue
which has its root in the heart of every one of us. The less the object
of our care is directly dependent on ourselves, the less we have to fear
from the illusion of self-interest; the more general this interest
becomes, the juster it is; and the love of the human race is nothing but
the love of justice within us. If therefore we desire Emile to be a
lover of truth, if we desire that he should indeed perceive it, let us
keep him far from self-interest in all his business. The more care he
bestows upon the happiness of others the wiser and better he is, and the
fewer mistakes he will make between good and evil; but never allow him
any blind preference founded merely on personal predilection or unfair
prejudice. Why should he harm one person to serve another? What does it
matter to him who has the greater share of happiness, providing he
promotes the happiness of all? Apart from self-interest this care for
the general well-being is the first concern of the wise man, for each of
us forms part of the human race and not part of any individual member of
that race.

To prevent pity degenerating into weakness we must generalise it and
extend it to mankind. Then we only yield to it when it is in accordance
with justice, since justice is of all the virtues that which contributes
most to the common good. Reason and self-love compel us to love mankind
even more than our neighbour, and to pity the wicked is to be very cruel
to other men.

Moreover, you must bear in mind that all these means employed to project
my pupil beyond himself have also a distinct relation to himself; since
they not only cause him inward delight, but I am also endeavouring to
instruct him, while I am making him kindly disposed towards others.

First I showed the means employed, now I will show the result. What wide
prospects do I perceive unfolding themselves before his mind! What noble
feelings stifle the lesser passions in his heart! What clearness of
judgment, what accuracy in reasoning, do I see developing from the
inclinations we have cultivated, from the experience which concentrates
the desires of a great heart within the narrow bounds of possibility, so
that a man superior to others can come down to their level if he cannot
raise them to his own! True principles of justice, true types of beauty,
all moral relations between man and man, all ideas of order, these are
engraved on his understanding; he sees the right place for everything
and the causes which drive it from that place; he sees what may do good,
and what hinders it. Without having felt the passions of mankind, he
knows the illusions they produce and their mode of action.

I proceed along the path which the force of circumstances compels me to
tread, but I do not insist that my readers shall follow me. Long ago
they have made up their minds that I am wandering in the land of
chimeras, while for my part I think they are dwelling in the country of
prejudice. When I wander so far from popular beliefs I do not cease to
bear them in mind; I examine them, I consider them, not that I may
follow them or shun them, but that I may weigh them in the balance of
reason. Whenever reason compels me to abandon these popular beliefs, I
know by experience that my readers will not follow my example; I know
that they will persist in refusing to go beyond what they can see, and
that they will take the youth I am describing for the creation of my
fanciful imagination, merely because he is unlike the youths with whom
they compare him; they forget that he must needs be different, because
he has been brought up in a totally different fashion; he has been
influenced by wholly different feelings, instructed in a wholly
different manner, so that it would be far stranger if he were like your
pupils than if he were what I have supposed. He is a man of nature's
making, not man's. No wonder men find him strange.

When I began this work I took for granted nothing but what could be
observed as readily by others as by myself; for our starting-point, the
birth of man, is the same for all; but the further we go, while I am
seeking to cultivate nature and you are seeking to deprave it, the
further apart we find ourselves. At six years old my pupil was not so
very unlike yours, whom you had not yet had time to disfigure; now there
is nothing in common between them; and when they reach the age of
manhood, which is now approaching, they will show themselves utterly
different from each other, unless all my pains have been thrown away.
There may not be so very great a difference in the amount of knowledge
they possess, but there is all the difference in the world in the kind
of knowledge. You are amazed to find that the one has noble sentiments
of which the others have not the smallest germ, but remember that the
latter are already philosophers and theologians while Emile does not
even know what is meant by a philosopher and has scarcely heard the name
of God.

But if you come and tell me, ``There are no such young men, young people
are not made that way; they have this passion or that, they do this or
that,'' it is as if you denied that a pear tree could ever be a tall
tree because the pear trees in our gardens are all dwarfs.

I beg these critics who are so ready with their blame to consider that I
am as well acquainted as they are with everything they say, that I have
probably given more thought to it, and that, as I have no private end to
serve in getting them to agree with me, I have a right to demand that
they should at least take time to find out where I am mistaken. Let them
thoroughly examine the nature of man, let them follow the earliest
growth of the heart in any given circumstances, so as to see what a
difference education may make in the individual; then let them compare
my method of education with the results I ascribe to it; and let them
tell me where my reasoning is unsound, and I shall have no answer to
give them.

It is this that makes me speak so strongly, and as I think with good
excuse: I have not pledged myself to any system, I depend as little as
possible on arguments, and I trust to what I myself have observed. I do
not base my ideas on what I have imagined, but on what I have seen. It
is true that I have not confined my observations within the walls of any
one town, nor to a single class of people; but having compared men of
every class and every nation which I have been able to observe in the
course of a life spent in this pursuit, I have discarded as artificial
what belonged to one nation and not to another, to one rank and not to
another; and I have regarded as proper to mankind what was common to
all, at any age, in any station, and in any nation whatsoever.

Now if in accordance with this method you follow from infancy the course
of a youth who has not been shaped to any special mould, one who depends
as little as possible on authority and the opinions of others, which
will he most resemble, my pupil or yours? It seems to me that this is
the question you must answer if you would know if I am mistaken.

It is not easy for a man to begin to think; but when once he has begun
he will never leave off. Once a thinker, always a thinker, and the
understanding once practised in reflection will never rest. You may
therefore think that I do too much or too little; that the human mind is
not by nature so quick to unfold; and that after having given it
opportunities it has not got, I keep it too long confined within a
circle of ideas which it ought to have outgrown.

But remember, in the first place, that when I want to train a natural
man, I do not want to make him a savage and to send him back to the
woods, but that living in the whirl of social life it is enough that he
should not let himself be carried away by the passions and prejudices of
men; let him see with his eyes and feel with his heart, let him own no
sway but that of reason. Under these conditions it is plain that many
things will strike him; the oft-recurring feelings which affect him, the
different ways of satisfying his real needs, must give him many ideas he
would not otherwise have acquired or would only have acquired much
later. The natural progress of the mind is quickened but not reversed.
The same man who would remain stupid in the forests should become wise
and reasonable in towns, if he were merely a spectator in them. Nothing
is better fitted to make one wise than the sight of follies we do not
share, and even if we share them, we still learn, provided we are not
the dupe of our follies and provided we do not bring to them the same
mistakes as the others.

Consider also that while our faculties are confined to the things of
sense, we offer scarcely any hold to the abstractions of philosophy or
to purely intellectual ideas. To attain to these we require either to
free ourselves from the body to which we are so strongly bound, or to
proceed step by step in a slow and gradual course, or else to leap
across the intervening space with a gigantic bound of which no child is
capable, one for which grown men even require many steps hewn on purpose
for them; but I find it very difficult to see how you propose to
construct such steps.

The Incomprehensible embraces all, he gives its motion to the earth, and
shapes the system of all creatures, but our eyes cannot see him nor can
our hands search him out, he evades the efforts of our senses; we behold
the work, but the workman is hidden from our eyes. It is no small matter
to know that he exists, and when we have got so far, and when we ask.
What is he? Where is he? our mind is overwhelmed, we lose ourselves, we
know not what to think.

Locke would have us begin with the study of spirits and go on to that of
bodies. This is the method of superstition, prejudice, and error; it is
not the method of nature, nor even that of well-ordered reason; it is to
learn to see by shutting our eyes. We must have studied bodies long
enough before we can form any true idea of spirits, or even suspect that
there are such beings. The contrary practice merely puts materialism on
a firmer footing.

Since our senses are the first instruments to our learning, corporeal
and sensible bodies are the only bodies we directly apprehend. The word
``spirit'' has no meaning for any one who has not philosophised. To the
unlearned and to the child a spirit is merely a body. Do they not fancy
that spirits groan, speak, fight, and make noises? Now you must own that
spirits with arms and voices are very like bodies. This is why every
nation on the face of the earth, not even excepting the Jews, have made
to themselves idols. We, ourselves, with our words, Spirit, Trinity,
Persons, are for the most part quite anthropomorphic. I admit that we
are taught that God is everywhere; but we also believe that there is air
everywhere, at least in our atmosphere; and the word Spirit meant
originally nothing more than breath and wind. Once you teach people to
say what they do not understand, it is easy enough to get them to say
anything you like.

The perception of our action upon other bodies must have first induced
us to suppose that their action upon us was effected in like manner.
Thus man began by thinking that all things whose action affected him
were alive. He did not recognise the limits of their powers, and he
therefore supposed that they were boundless; as soon as he had supplied
them with bodies they became his gods. In the earliest times men went in
terror of everything and everything in nature seemed alive. The idea of
matter was developed as slowly as that of spirit, for the former is
itself an abstraction.

Thus the universe was peopled with gods like themselves. The stars, the
winds and the mountains, rivers, trees, and towns, their very dwellings,
each had its soul, its god, its life. The teraphim of Laban, the manitos
of savages, the fetishes of the negroes, every work of nature and of
man, were the first gods of mortals; polytheism was their first religion
and idolatry their earliest form of worship. The idea of one God was
beyond their grasp, till little by little they formed general ideas, and
they rose to the idea of a first cause and gave meaning to the word
``substance,'' which is at bottom the greatest of abstractions. So every
child who believes in God is of necessity an idolater or at least he
regards the Deity as a man, and when once the imagination has perceived
God, it is very seldom that the understanding conceives him. Locke's
order leads us into this same mistake.

Having arrived, I know not how, at the idea of substance, it is clear
that to allow of a single substance it must be assumed that this
substance is endowed with incompatible and mutually exclusive
properties, such as thought and size, one of which is by its nature
divisible and the other wholly incapable of division. Moreover it is
assumed that thought or, if you prefer it, feeling is a primitive
quality inseparable from the substance to which it belongs, that its
relation to the substance is like the relation between substance and
size. Hence it is inferred that beings who lose one of these attributes
lose the substance to which it belongs, and that death is, therefore,
but a separation of substances, and that those beings in whom the two
attributes are found are composed of the two substances to which those
two qualities belong.

But consider what a gulf there still is between the idea of two
substances and that of the divine nature, between the incomprehensible
idea of the influence of our soul upon our body and the idea of the
influence of God upon every living creature. The ideas of creation,
destruction, ubiquity, eternity, almighty power, those of the divine
attributes---these are all ideas so confused and obscure that few men
succeed in grasping them; yet there is nothing obscure about them to the
common people, because they do not understand them in the least; how
then should they present themselves in full force, that is to say in all
their obscurity, to the young mind which is still occupied with the
first working of the senses, and fails to realise anything but what it
handles? In vain do the abysses of the Infinite open around us, a child
does not know the meaning of fear; his weak eyes cannot gauge their
depths. To children everything is infinite, they cannot assign limits to
anything; not that their measure is so large, but because their
understanding is so small. I have even noticed that they place the
infinite rather below than above the dimensions known to them. They
judge a distance to be immense rather by their feet than by their eyes;
infinity is bounded for them, not so much by what they can see, but how
far they can go. If you talk to them of the power of God, they will
think he is nearly as strong as their father. As their own knowledge is
in everything the standard by which they judge of what is possible, they
always picture what is described to them as rather smaller than what
they know. Such are the natural reasonings of an ignorant and feeble
mind. Ajax was afraid to measure his strength against Achilles, yet he
challenged Jupiter to combat, for he knew Achilles and did not know
Jupiter. A Swiss peasant thought himself the richest man alive; when
they tried to explain to him what a king was, he asked with pride, ``Has
the king got a hundred cows on the high pastures?''

I am aware that many of my readers will be surprised to find me tracing
the course of my scholar through his early years without speaking to him
of religion. At fifteen he will not even know that he has a soul, at
eighteen even he may not be ready to learn about it. For if he learns
about it too soon, there is the risk of his never really knowing
anything about it.

If I had to depict the most heart-breaking stupidity, I would paint a
pedant teaching children the catechism; if I wanted to drive a child
crazy I would set him to explain what he learned in his catechism. You
will reply that as most of the Christian doctrines are mysteries, you
must wait, not merely till the child is a man, but till the man is dead,
before the human mind will understand those doctrines. To that I reply,
that there are mysteries which the heart of man can neither conceive nor
believe, and I see no use in teaching them to children, unless you want
to make liars of them. Moreover, I assert that to admit that there are
mysteries, you must at least realise that they are incomprehensible, and
children are not even capable of this conception! At an age when
everything is mysterious, there are no mysteries properly so-called.

``We must believe in God if we would be saved.'' This doctrine wrongly
understood is the root of bloodthirsty intolerance and the cause of all
the futile teaching which strikes a deadly blow at human reason by
training it to cheat itself with mere words. No doubt there is not a
moment to be lost if we would deserve eternal salvation; but if the
repetition of certain words suffices to obtain it, I do not see why we
should not people heaven with starlings and magpies as well as with
children.

The obligation of faith assumes the possibility of belief. The
philosopher who does not believe is wrong, for he misuses the reason he
has cultivated, and he is able to understand the truths he rejects. But
the child who professes the Christian faith---what does he believe? Just
what he understands; and he understands so little of what he is made to
repeat that if you tell him to say just the opposite he will be quite
ready to do it. The faith of children and the faith of many men is a
matter of geography. Will they be rewarded for having been born in Rome
rather than in Mecca? One is told that Mahomet is the prophet of God and
he says, ``Mahomet is the prophet of God.'' The other is told that
Mahomet is a rogue and he says, ``Mahomet is a rogue.'' Either of them
would have said just the opposite had he stood in the other's shoes.
When they are so much alike to begin with, can the one be consigned to
Paradise and the other to Hell? When a child says he believes in God, it
is not God he believes in, but Peter or James who told him that there is
something called God, and he believes it after the fashion of
Euripides---

``O Jupiter, of whom I know nothing but thy name.''

{[}Footnote: Plutarch. It is thus that the tragedy of Menalippus
originally began, but the clamour of the Athenians compelled Euripides
to change these opening lines.{]}

We hold that no child who dies before the age of reason will be deprived
of everlasting happiness; the Catholics believe the same of all children
who have been baptised, even though they have never heard of God. There
are, therefore, circumstances in which one can be saved without belief
in God, and these circumstances occur in the case of children or madmen
when the human mind is incapable of the operations necessary to perceive
the Godhead. The only difference I see between you and me is that you
profess that children of seven years old are able to do this and I do
not think them ready for it at fifteen. Whether I am right or wrong
depends, not on an article of the creed, but on a simple observation in
natural history.

From the same principle it is plain that any man having reached old age
without faith in God will not, therefore, be deprived of God's presence
in another life if his blindness was not wilful; and I maintain that it
is not always wilful. You admit that it is so in the case of lunatics
deprived by disease of their spiritual faculties, but not of their
manhood, and therefore still entitled to the goodness of their Creator.
Why then should we not admit it in the case of those brought up from
infancy in seclusion, those who have led the life of a savage and are
without the knowledge that comes from intercourse with other men.
{[}Footnote: For the natural condition of the human mind and its slow
development, cf. the first part of the Discours sur Inegalite.{]} For it
is clearly impossible that such a savage could ever raise his thoughts
to the knowledge of the true God. Reason tells that man should only be
punished for his wilful faults, and that invincible ignorance can never
be imputed to him as a crime. Hence it follows that in the sight of the
Eternal Justice every man who would believe if he had the necessary
knowledge is counted a believer, and that there will be no unbelievers
to be punished except those who have closed their hearts against the
truth.

Let us beware of proclaiming the truth to those who cannot as yet
comprehend it, for to do so is to try to inculcate error. It would be
better to have no idea at all of the Divinity than to have mean,
grotesque, harmful, and unworthy ideas; to fail to perceive the Divine
is a lesser evil than to insult it. The worthy Plutarch says, ``I would
rather men said, `There is no such person as Plutarch,' than that they
should say, `Plutarch is unjust, envious, jealous, and such a tyrant
that he demands more than can be performed.'\,''

The chief harm which results from the monstrous ideas of God which are
instilled into the minds of children is that they last all their life
long, and as men they understand no more of God than they did as
children. In Switzerland I once saw a good and pious mother who was so
convinced of the truth of this maxim that she refused to teach her son
religion when he was a little child for fear lest he should be satisfied
with this crude teaching and neglect a better teaching when he reached
the age of reason. This child never heard the name of God pronounced
except with reverence and devotion, and as soon as he attempted to say
the word he was told to hold his tongue, as if the subject were too
sublime and great for him. This reticence aroused his curiosity and his
self-love; he looked forward to the time when he would know this mystery
so carefully hidden from him. The less they spoke of God to him, the
less he was himself permitted to speak of God, the more he thought about
Him; this child beheld God everywhere. What I should most dread as the
result of this unwise affectation of mystery is this: by
over-stimulating the youth's imagination you may turn his head, and make
him at the best a fanatic rather than a believer.

But we need fear nothing of the sort for Emile, who always declines to
pay attention to what is beyond his reach, and listens with profound
indifference to things he does not understand. There are so many things
of which he is accustomed to say, ``That is no concern of mine,'' that
one more or less makes little difference to him; and when he does begin
to perplex himself with these great matters, it is because the natural
growth of his knowledge is turning his thoughts that way.

We have seen the road by which the cultivated human mind approaches
these mysteries, and I am ready to admit that it would not attain to
them naturally, even in the bosom of society, till a much later age. But
as there are in this same society inevitable causes which hasten the
development of the passions, if we did not also hasten the development
of the knowledge which controls these passions we should indeed depart
from the path of nature and disturb her equilibrium. When we can no
longer restrain a precocious development in one direction we must
promote a corresponding development in another direction, so that the
order of nature may not be inverted, and so that things should progress
together, not separately, so that the man, complete at every moment of
his life, may never find himself at one stage in one of his faculties
and at another stage in another faculty.

What a difficulty do I see before me! A difficulty all the greater
because it depends less on actual facts than on the cowardice of those
who dare not look the difficulty in the face. Let us at least venture to
state our problem. A child should always be brought up in his father's
religion; he is always given plain proofs that this religion, whatever
it may be, is the only true religion, that all others are ridiculous and
absurd. The force of the argument depends entirely on the country in
which it is put forward. Let a Turk, who thinks Christianity so absurd
at Constantinople, come to Paris and see what they think of Mahomet. It
is in matters of religion more than in anything else that prejudice is
triumphant. But when we who profess to shake off its yoke entirely, we
who refuse to yield any homage to authority, decline to teach Emile
anything which he could not learn for himself in any country, what
religion shall we give him, to what sect shall this child of nature
belong? The answer strikes me as quite easy. We will not attach him to
any sect, but we will give him the means to choose for himself according
to the right use of his own reason.
\aquote{Incedo per ignes \\
Suppositos cineri doloso.}{Horace, lib. ii. ode I.}
No matter! Thus far zeal and prudence have taken the place of caution. I
hope that these guardians will not fail me now. Reader, do not fear lest
I should take precautions unworthy of a lover of truth; I shall never
forget my motto, but I distrust my own judgment all too easily. Instead
of telling you what I think myself, I will tell you the thoughts of one
whose opinions carry more weight than mine. I guarantee the truth of the
facts I am about to relate; they actually happened to the author whose
writings I am about to transcribe; it is for you to judge whether we can
draw from them any considerations bearing on the matter in hand. I do
not offer you my own idea or another's as your rule; I merely present
them for your examination.

Thirty years ago there was a young man in an Italian town; he was an
exile from his native land and found himself reduced to the depths of
poverty. He had been born a Calvinist, but the consequences of his own
folly had made him a fugitive in a strange land; he had no money and he
changed his religion for a morsel of bread. There was a hostel for
proselytes in that town to which he gained admission. The study of
controversy inspired doubts he had never felt before, and he made
acquaintance with evil hitherto unsuspected by him; he heard strange
doctrines and he met with morals still stranger to him; he beheld this
evil conduct and nearly fell a victim to it. He longed to escape, but he
was locked up; he complained, but his complaints were unheeded; at the
mercy of his tyrants, he found himself treated as a criminal because he
would not share their crimes. The anger kindled in a young and untried
heart by the first experience of violence and injustice may be realised
by those who have themselves experienced it. Tears of anger flowed from
his eyes, he was wild with rage; he prayed to heaven and to man, and his
prayers were unheard; he spoke to every one and no one listened to him.
He saw no one but the vilest servants under the control of the wretch
who insulted him, or accomplices in the same crime who laughed at his
resistance and encouraged him to follow their example. He would have
been ruined had not a worthy priest visited the hostel on some matter of
business. He found an opportunity of consulting him secretly. The priest
was poor and in need of help himself, but the victim had more need of
his assistance, and he did not hesitate to help him to escape at the
risk of making a dangerous enemy.

Having escaped from vice to return to poverty, the young man struggled
vainly against fate: for a moment he thought he had gained the victory.
At the first gleam of good fortune his woes and his protector were alike
forgotten. He was soon punished for this ingratitude; all his hopes
vanished; youth indeed was on his side, but his romantic ideas spoiled
everything. He had neither talent nor skill to make his way easily, he
could neither be commonplace nor wicked, he expected so much that he got
nothing. When he had sunk to his former poverty, when he was without
food or shelter and ready to die of hunger, he remembered his
benefactor.

He went back to him, found him, and was kindly welcomed; the sight of
him reminded the priest of a good deed he had done; such a memory always
rejoices the heart. This man was by nature humane and pitiful; he felt
the sufferings of others through his own, and his heart had not been
hardened by prosperity; in a word, the lessons of wisdom and an
enlightened virtue had reinforced his natural kindness of heart. He
welcomed the young man, found him a lodging, and recommended him; he
shared with him his living which was barely enough for two. He did more,
he instructed him, consoled him, and taught him the difficult art of
bearing adversity in patience. You prejudiced people, would you have
expected to find all this in a priest and in Italy?

This worthy priest was a poor Savoyard clergyman who had offended his
bishop by some youthful fault; he had crossed the Alps to find a
position which he could not obtain in his own country. He lacked neither
wit nor learning, and with his interesting countenance he had met with
patrons who found him a place in the household of one of the ministers,
as tutor to his son. He preferred poverty to dependence, and he did not
know how to get on with the great. He did not stay long with this
minister, and when he departed he took with him his good opinion; and as
he lived a good life and gained the hearts of everybody, he was glad to
be forgiven by his bishop and to obtain from him a small parish among
the mountains, where he might pass the rest of his life. This was the
limit of his ambition.

He was attracted by the young fugitive and he questioned him closely. He
saw that ill-fortune had already seared his heart, that scorn and
disgrace had overthrown his courage, and that his pride, transformed
into bitterness and spite, led him to see nothing in the harshness and
injustice of men but their evil disposition and the vanity of all
virtue. He had seen that religion was but a mask for selfishness, and
its holy services but a screen for hypocrisy; he had found in the
subtleties of empty disputations heaven and hell awarded as prizes for
mere words; he had seen the sublime and primitive idea of Divinity
disfigured by the vain fancies of men; and when, as he thought, faith in
God required him to renounce the reason God himself had given him, he
held in equal scorn our foolish imaginings and the object with which
they are concerned. With no knowledge of things as they are, without any
idea of their origins, he was immersed in his stubborn ignorance and
utterly despised those who thought they knew more than himself.

The neglect of all religion soon leads to the neglect of a man's duties.
The heart of this young libertine was already far on this road. Yet his
was not a bad nature, though incredulity and misery were gradually
stifling his natural disposition and dragging him down to ruin; they
were leading him into the conduct of a rascal and the morals of an
atheist.

The almost inevitable evil was not actually consummated. The young man
was not ignorant, his education had not been neglected. He was at that
happy age when the pulse beats strongly and the heart is warm, but is
not yet enslaved by the madness of the senses. His heart had not lost
its elasticity. A native modesty, a timid disposition restrained him,
and prolonged for him that period during which you watch your pupil so
carefully. The hateful example of brutal depravity, of vice without any
charm, had not merely failed to quicken his imagination, it had deadened
it. For a long time disgust rather than virtue preserved his innocence,
which would only succumb to more seductive charms.

The priest saw the danger and the way of escape. He was not discouraged
by difficulties, he took a pleasure in his task; he determined to
complete it and to restore to virtue the victim he had snatched from
vice. He set about it cautiously; the beauty of the motive gave him
courage and inspired him with means worthy of his zeal. Whatever might
be the result, his pains would not be wasted. We are always successful
when our sole aim is to do good.

He began to win the confidence of the proselyte by not asking any price
for his kindness, by not intruding himself upon him, by not preaching at
him, by always coming down to his level, and treating him as an equal.
It was, so I think, a touching sight to see a serious person becoming
the comrade of a young scamp, and virtue putting up with the speech of
licence in order to triumph over it more completely. When the young fool
came to him with his silly confidences and opened his heart to him, the
priest listened and set him at his ease; without giving his approval to
what was bad, he took an interest in everything; no tactless reproof
checked his chatter or closed his heart; the pleasure which he thought
was given by his conversation increased his pleasure in telling
everything; thus he made his general confession without knowing he was
confessing anything.

After he had made a thorough study of his feelings and disposition, the
priest saw plainly that, although he was not ignorant for his age, he
had forgotten everything that he most needed to know, and that the
disgrace which fortune had brought upon him had stifled in him all real
sense of good and evil. There is a stage of degradation which robs the
soul of its life; and the inner voice cannot be heard by one whose whole
mind is bent on getting food. To protect the unlucky youth from the
moral death which threatened him, he began to revive his self-love and
his good opinion of himself. He showed him a happier future in the right
use of his talents; he revived the generous warmth of his heart by
stories of the noble deeds of others; by rousing his admiration for the
doers of these deeds he revived his desire to do like deeds himself. To
draw him gradually from his idle and wandering life, he made him copy
out extracts from well-chosen books; he pretended to want these
extracts, and so nourished in him the noble feeling of gratitude. He
taught him indirectly through these books, and thus he made him
sufficiently regain his good opinion of himself so that he would no
longer think himself good for nothing, and would not make himself
despicable in his own eyes.

A trifling incident will show how this kindly man tried, unknown to him,
to raise the heart of his disciple out of its degradation, without
seeming to think of teaching. The priest was so well known for his
uprightness and his discretion, that many people preferred to entrust
their alms to him, rather than to the wealthy clergy of the town. One
day some one had given him some money to distribute among the poor, and
the young man was mean enough to ask for some of it on the score of
poverty. ``No,'' said he, ``we are brothers, you belong to me and I must
not touch the money entrusted to me.'' Then he gave him the sum he had
asked for out of his own pocket. Lessons of this sort seldom fail to
make an impression on the heart of young people who are not wholly
corrupt.

I am weary of speaking in the third person, and the precaution is
unnecessary; for you are well aware, my dear friend, that I myself was
this unhappy fugitive; I think I am so far removed from the disorders of
my youth that I may venture to confess them, and the hand which rescued
me well deserves that I should at least do honour to its goodness at the
cost of some slight shame.

What struck me most was to see in the private life of my worthy master,
virtue without hypocrisy, humanity without weakness, speech always plain
and straightforward, and conduct in accordance with this speech. I never
saw him trouble himself whether those whom he assisted went to vespers
or confession, whether they fasted at the appointed seasons and went
without meat; nor did he impose upon them any other like conditions,
without which you might die of hunger before you could hope for any help
from the devout.

Far from displaying before him the zeal of a new convert, I was
encouraged by these observations and I made no secret of my way of
thinking, nor did he seem to be shocked by it. Sometimes I would say to
myself, he overlooks my indifference to the religion I have adopted
because he sees I am equally indifferent to the religion in which I was
brought up; he knows that my scorn for religion is not confined to one
sect. But what could I think when I sometimes heard him give his
approval to doctrines contrary to those of the Roman Catholic Church,
and apparently having but a poor opinion of its ceremonies. I should
have thought him a Protestant in disguise if I had not beheld him so
faithful to those very customs which he seemed to value so lightly; but
I knew he fulfilled his priestly duties as carefully in private as in
public, and I knew not what to think of these apparent contradictions.
Except for the fault which had formerly brought about his disgrace, a
fault which he had only partially overcome, his life was exemplary, his
conduct beyond reproach, his conversation honest and discreet. While I
lived on very friendly terms with him, I learnt day by day to respect
him more; and when he had completely won my heart by such great
kindness, I awaited with eager curiosity the time when I should learn
what was the principle on which the uniformity of this strange life was
based.

This opportunity was a long time coming. Before taking his disciple into
his confidence, he tried to get the seeds of reason and kindness which
he had sown in my heart to germinate. The most difficult fault to
overcome in me was a certain haughty misanthropy, a certain bitterness
against the rich and successful, as if their wealth and happiness had
been gained at my own expense, and as if their supposed happiness had
been unjustly taken from my own. The foolish vanity of youth, which
kicks against the pricks of humiliation, made me only too much inclined
to this angry temper; and the self-respect, which my mentor strove to
revive, led to pride, which made men still more vile in my eyes, and
only added scorn to my hatred.

Without directly attacking this pride, he prevented it from developing
into hardness of heart; and without depriving me of my self-esteem, he
made me less scornful of my neighbours. By continually drawing my
attention from the empty show, and directing it to the genuine
sufferings concealed by it, he taught me to deplore the faults of my
fellows and feel for their sufferings, to pity rather than envy them.
Touched with compassion towards human weaknesses through the profound
conviction of his own failings, he viewed all men as the victims of
their own vices and those of others; he beheld the poor groaning under
the tyranny of the rich, and the rich under the tyranny of their own
prejudices. ``Believe me,'' said he, ``our illusions, far from
concealing our woes, only increase them by giving value to what is in
itself valueless, in making us aware of all sorts of fancied privations
which we should not otherwise feel. Peace of heart consists in despising
everything that might disturb that peace; the man who clings most
closely to life is the man who can least enjoy it; and the man who most
eagerly desires happiness is always most miserable.''

``What gloomy ideas!'' I exclaimed bitterly. ``If we must deny ourselves
everything, we might as well never have been born; and if we must
despise even happiness itself who can be happy?'' ``I am,'' replied the
priest one day, in a tone which made a great impression on me. ``You
happy! So little favoured by fortune, so poor, an exile and persecuted,
you are happy! How have you contrived to be happy?'' ``My child,'' he
answered, ``I will gladly tell you.''

Thereupon he explained that, having heard my confessions, he would
confess to me. ``I will open my whole heart to yours,'' he said,
embracing me. ``You will see me, if not as I am, at least as I seem to
myself. When you have heard my whole confession of faith, when you
really know the condition of my heart, you will know why I think myself
happy, and if you think as I do, you will know how to be happy too. But
these explanations are not the affair of a moment, it will take time to
show you all my ideas about the lot of man and the true value of life;
let us choose a fitting time and a place where we may continue this
conversation without interruption.''

I showed him how eager I was to hear him. The meeting was fixed for the
very next morning. It was summer time; we rose at daybreak. He took me
out of the town on to a high hill above the river Po, whose course we
beheld as it flowed between its fertile banks; in the distance the
landscape was crowned by the vast chain of the Alps; the beams of the
rising sun already touched the plains and cast across the fields long
shadows of trees, hillocks, and houses, and enriched with a thousand
gleams of light the fairest picture which the human eye can see. You
would have thought that nature was displaying all her splendour before
our eyes to furnish a text for our conversation. After contemplating
this scene for a space in silence, the man of peace spoke to me.

\subparagraph{THE CREED OF A SAVOYARD PRIEST}\label{id01003}

My child, do not look to me for learned speeches or profound arguments.
I am no great philosopher, nor do I desire to be one. I have, however, a
certain amount of common-sense and a constant devotion to truth. I have
no wish to argue with you nor even to convince you; it is enough for me
to show you, in all simplicity of heart, what I really think. Consult
your own heart while I speak; that is all I ask. If I am mistaken, I am
honestly mistaken, and therefore my error will not be counted to me as a
crime; if you, too, are honestly mistaken, there is no great harm done.
If I am right, we are both endowed with reason, we have both the same
motive for listening to the voice of reason. Why should not you think as
I do?

By birth I was a peasant and poor; to till the ground was my portion;
but my parents thought it a finer thing that I should learn to get my
living as a priest and they found means to send me to college. I am
quite sure that neither my parents nor I had any idea of seeking after
what was good, useful, or true; we only sought what was wanted to get me
ordained. I learned what was taught me, I said what I was told to say, I
promised all that was required, and I became a priest. But I soon
discovered that when I promised not to be a man, I had promised more
than I could perform.

Conscience, they tell us, is the creature of prejudice, but I know from
experience that conscience persists in following the order of nature in
spite of all the laws of man. In vain is this or that forbidden; remorse
makes her voice heard but feebly when what we do is permitted by
well-ordered nature, and still more when we are doing her bidding. My
good youth, nature has not yet appealed to your senses; may you long
remain in this happy state when her voice is the voice of innocence.
Remember that to anticipate her teaching is to offend more deeply
against her than to resist her teaching; you must first learn to resist,
that you may know when to yield without wrong-doing.

From my youth up I had reverenced the married state as the first and
most sacred institution of nature. Having renounced the right to marry,
I was resolved not to profane the sanctity of marriage; for in spite of
my education and reading I had always led a simple and regular life, and
my mind had preserved the innocence of its natural instincts; these
instincts had not been obscured by worldly wisdom, while my poverty kept
me remote from the temptations dictated by the sophistry of vice.

This very resolution proved my ruin. My respect for marriage led to the
discovery of my misconduct. The scandal must be expiated; I was
arrested, suspended, and dismissed; I was the victim of my scruples
rather than of my incontinence, and I had reason to believe, from the
reproaches which accompanied my disgrace, that one can often escape
punishment by being guilty of a worse fault.

A thoughtful mind soon learns from such experiences. I found my former
ideas of justice, honesty, and every duty of man overturned by these
painful events, and day by day I was losing my hold on one or another of
the opinions I had accepted. What was left was not enough to form a body
of ideas which could stand alone, and I felt that the evidence on which
my principles rested was being weakened; at last I knew not what to
think, and I came to the same conclusion as yourself, but with this
difference: My lack of faith was the slow growth of manhood, attained
with great difficulty, and all the harder to uproot.

I was in that state of doubt and uncertainty which Descartes considers
essential to the search for truth. It is a state which cannot continue,
it is disquieting and painful; only vicious tendencies and an idle heart
can keep us in that state. My heart was not so corrupt as to delight in
it, and there is nothing which so maintains the habit of thinking as
being better pleased with oneself than with one's lot.

I pondered, therefore, on the sad fate of mortals, adrift upon this sea
of human opinions, without compass or rudder, and abandoned to their
stormy passions with no guide but an inexperienced pilot who does not
know whence he comes or whither he is going. I said to myself, ``I love
truth, I seek her, and cannot find her. Show me truth and I will hold
her fast; why does she hide her face from the eager heart that would
fain worship her?''

Although I have often experienced worse sufferings, I have never led a
life so uniformly distressing as this period of unrest and anxiety, when
I wandered incessantly from one doubt to another, gaining nothing from
my prolonged meditations but uncertainty, darkness, and contradiction
with regard to the source of my being and the rule of my duties.

I cannot understand how any one can be a sceptic sincerely and on
principle. Either such philosophers do not exist or they are the most
miserable of men. Doubt with regard to what we ought to know is a
condition too violent for the human mind; it cannot long be endured; in
spite of itself the mind decides one way or another, and it prefers to
be deceived rather than to believe nothing.

My perplexity was increased by the fact that I had been brought up in a
church which decides everything and permits no doubts, so that having
rejected one article of faith I was forced to reject the rest; as I
could not accept absurd decisions, I was deprived of those which were
not absurd. When I was told to believe everything, I could believe
nothing, and I knew not where to stop.

I consulted the philosophers, I searched their books and examined their
various theories; I found them all alike proud, assertive, dogmatic,
professing, even in their so-called scepticism, to know everything,
proving nothing, scoffing at each other. This last trait, which was
common to all of them, struck me as the only point in which they were
right. Braggarts in attack, they are weaklings in defence. Weigh their
arguments, they are all destructive; count their voices, every one
speaks for himself; they are only agreed in arguing with each other. I
could find no way out of my uncertainty by listening to them.

I suppose this prodigious diversity of opinion is caused, in the first
place, by the weakness of the human intellect; and, in the second, by
pride. We have no means of measuring this vast machine, we are unable to
calculate its workings; we know neither its guiding principles nor its
final purpose; we do not know ourselves, we know neither our nature nor
the spirit that moves us; we scarcely know whether man is one or many;
we are surrounded by impenetrable mysteries. These mysteries are beyond
the region of sense, we think we can penetrate them by the light of
reason, but we fall back on our imagination. Through this imagined world
each forces a way for himself which he holds to be right; none can tell
whether his path will lead him to the goal. Yet we long to know and
understand it all. The one thing we do not know is the limit of the
knowable. We prefer to trust to chance and to believe what is not true,
rather than to own that not one of us can see what really is. A fragment
of some vast whole whose bounds are beyond our gaze, a fragment
abandoned by its Creator to our foolish quarrels, we are vain enough to
want to determine the nature of that whole and our own relations with
regard to it.

If the philosophers were in a position to declare the truth, which of
them would care to do so? Every one of them knows that his own system
rests on no surer foundations than the rest, but he maintains it because
it is his own. There is not one of them who, if he chanced to discover
the difference between truth and falsehood, would not prefer his own lie
to the truth which another had discovered. Where is the philosopher who
would not deceive the whole world for his own glory? If he can rise
above the crowd, if he can excel his rivals, what more does he want?
Among believers he is an atheist; among atheists he would be a believer.

The first thing I learned from these considerations was to restrict my
inquiries to what directly concerned myself, to rest in profound
ignorance of everything else, and not even to trouble myself to doubt
anything beyond what I required to know.

I also realised that the philosophers, far from ridding me of my vain
doubts, only multiplied the doubts that tormented me and failed to
remove any one of them. So I chose another guide and said, ``Let me
follow the Inner Light; it will not lead me so far astray as others have
done, or if it does it will be my own fault, and I shall not go so far
wrong if I follow my own illusions as if I trusted to their deceits.''

I then went over in my mind the various opinions which I had held in the
course of my life, and I saw that although no one of them was plain
enough to gain immediate belief, some were more probable than others,
and my inward consent was given or withheld in proportion to this
improbability. Having discovered this, I made an unprejudiced comparison
of all these different ideas, and I perceived that the first and most
general of them was also the simplest and the most reasonable, and that
it would have been accepted by every one if only it had been last
instead of first. Imagine all your philosophers, ancient and modern,
having exhausted their strange systems of force, chance, fate,
necessity, atoms, a living world, animated matter, and every variety of
materialism. Then comes the illustrious Clarke who gives light to the
world and proclaims the Being of beings and the Giver of things. What
universal admiration, what unanimous applause would have greeted this
new system---a system so great, so illuminating, and so simple. Other
systems are full of absurdities; this system seems to me to contain
fewer things which are beyond the understanding of the human mind. I
said to myself, ``Every system has its insoluble problems, for the
finite mind of man is too small to deal with them; these difficulties
are therefore no final arguments, against any system. But what a
difference there is between the direct evidence on which these systems
are based! Should we not prefer that theory which alone explains all the
facts, when it is no more difficult than the rest?''

Bearing thus within my heart the love of truth as my only philosophy,
and as my only method a clear and simple rule which dispensed with the
need for vain and subtle arguments, I returned with the help of this
rule to the examination of such knowledge as concerned myself; I was
resolved to admit as self-evident all that I could not honestly refuse
to believe, and to admit as true all that seemed to follow directly from
this; all the rest I determined to leave undecided, neither accepting
nor rejecting it, nor yet troubling myself to clear up difficulties
which did not lead to any practical ends.

But who am I? What right have I to decide? What is it that determines my
judgments? If they are inevitable, if they are the results of the
impressions I receive, I am wasting my strength in such inquiries; they
would be made or not without any interference of mine. I must therefore
first turn my eyes upon myself to acquaint myself with the instrument I
desire to use, and to discover how far it is reliable.

I exist, and I have senses through which I receive impressions. This is
the first truth that strikes me and I am forced to accept it. Have I any
independent knowledge of my existence, or am I only aware of it through
my sensations? This is my first difficulty, and so far I cannot solve
it. For I continually experience sensations, either directly or
indirectly through memory, so how can I know if the feeling of self is
something beyond these sensations or if it can exist independently of
them?

My sensations take place in myself, for they make me aware of my own
existence; but their cause is outside me, for they affect me whether I
have any reason for them or not, and they are produced or destroyed
independently of me. So I clearly perceive that my sensation, which is
within me, and its cause or its object, which is outside me, are
different things.

Thus, not only do I exist, but other entities exist also, that is to
say, the objects of my sensations; and even if these objects are merely
ideas, still these ideas are not me.

But everything outside myself, everything which acts upon my senses, I
call matter, and all the particles of matter which I suppose to be
united into separate entities I call bodies. Thus all the disputes of
the idealists and the realists have no meaning for me; their
distinctions between the appearance and the reality of bodies are wholly
fanciful.

I am now as convinced of the existence of the universe as of my own. I
next consider the objects of my sensations, and I find that I have the
power of comparing them, so I perceive that I am endowed with an active
force of which I was not previously aware.

To perceive is to feel; to compare is to judge; to judge and to feel are
not the same. Through sensation objects present themselves to me
separately and singly as they are in nature; by comparing them I
rearrange them, I shift them so to speak, I place one upon another to
decide whether they are alike or different, or more generally to find
out their relations. To my mind, the distinctive faculty of an active or
intelligent being is the power of understanding this word ``is.'' I seek
in vain in the merely sensitive entity that intelligent force which
compares and judges; I can find no trace of it in its nature. This
passive entity will be aware of each object separately, it will even be
aware of the whole formed by the two together, but having no power to
place them side by side it can never compare them, it can never form a
judgment with regard to them.

To see two things at once is not to see their relations nor to judge of
their differences; to perceive several objects, one beyond the other, is
not to relate them. I may have at the same moment an idea of a big stick
and a little stick without comparing them, without judging that one is
less than the other, just as I can see my whole hand without counting my
fingers. {[}Footnote: M. de le Cordamines' narratives tell of a people
who only know how to count up to three. Yet the men of this nation,
having hands, have often seen their fingers without learning to count up
to five.{]} These comparative ideas, `greater', `smaller', together with
number ideas of `one', `two', etc. are certainly not sensations,
although my mind only produces them when my sensations occur.

We are told that a sensitive being distinguishes sensations from each
other by the inherent differences in the sensations; this requires
explanation. When the sensations are different, the sensitive being
distinguishes them by their differences; when they are alike, he
distinguishes them because he is aware of them one beyond the other.
Otherwise, how could he distinguish between two equal objects
simultaneously experienced? He would necessarily confound the two
objects and take them for one object, especially under a system which
professed that the representative sensations of space have no extension.

When we become aware of the two sensations to be compared, their
impression is made, each object is perceived, both are perceived, but
for all that their relation is not perceived. If the judgment of this
relation were merely a sensation, and came to me solely from the object
itself, my judgments would never be mistaken, for it is never untrue
that I feel what I feel.

Why then am I mistaken as to the relation between these two sticks,
especially when they are not parallel? Why, for example, do I say the
small stick is a third of the large, when it is only a quarter? Why is
the picture, which is the sensation, unlike its model which is the
object? It is because I am active when I judge, because the operation of
comparison is at fault; because my understanding, which judges of
relations, mingles its errors with the truth of sensations, which only
reveal to me things.

Add to this a consideration which will, I feel sure, appeal to you when
you have thought about it: it is this---If we were purely passive in the
use of our senses, there would be no communication between them; it
would be impossible to know that the body we are touching and the thing
we are looking at is the same. Either we should never perceive anything
outside ourselves, or there would be for us five substances perceptible
by the senses, whose identity we should have no means of perceiving.

This power of my mind which brings my sensations together and compares
them may be called by any name; let it be called attention, meditation,
reflection, or what you will; it is still true that it is in me and not
in things, that it is I alone who produce it, though I only produce it
when I receive an impression from things. Though I am compelled to feel
or not to feel, I am free to examine more or less what I feel.

Being now, so to speak, sure of myself, I begin to look at things
outside myself, and I behold myself with a sort of shudder flung at
random into this vast universe, plunged as it were into the vast number
of entities, knowing nothing of what they are in themselves or in
relation to me. I study them, I observe them; and the first object which
suggests itself for comparison with them is myself.

All that I perceive through the senses is matter, and I deduce all the
essential properties of matter from the sensible qualities which make me
perceive it, qualities which are inseparable from it. I see it sometimes
in motion, sometimes at rest, {[}Footnote: This repose is, if you prefer
it, merely relative; but as we perceive more or less of motion, we may
plainly conceive one of two extremes, which is rest; and we conceive it
so clearly that we are even disposed to take for absolute rest what is
only relative. But it is not true that motion is of the essence of
matter, if matter may be conceived of as at rest.{]} hence I infer that
neither motion nor rest is essential to it, but motion, being an action,
is the result of a cause of which rest is only the absence. When,
therefore, there is nothing acting upon matter it does not move, and for
the very reason that rest and motion are indifferent to it, its natural
state is a state of rest.

I perceive two sorts of motions of bodies, acquired motion and
spontaneous or voluntary motion. In the first the cause is external to
the body moved, in the second it is within. I shall not conclude from
that that the motion, say of a watch, is spontaneous, for if no external
cause operated upon the spring it would run down and the watch would
cease to go. For the same reason I should not admit that the movements
of fluids are spontaneous, neither should I attribute spontaneous motion
to fire which causes their fluidity. {[}Footnote: Chemists regard
phlogiston or the element of fire as diffused, motionless, and stagnant
in the compounds of which it forms part, until external forces set it
free, collect it and set it in motion, and change it into fire.{]}

You ask me if the movements of animals are spontaneous; my answer is,
``I cannot tell,'' but analogy points that way. You ask me again, how do
I know that there are spontaneous movements? I tell you, ``I know it
because I feel them.'' I want to move my arm and I move it without any
other immediate cause of the movement but my own will. In vain would any
one try to argue me out of this feeling, it is stronger than any proofs;
you might as well try to convince me that I do not exist.

If there were no spontaneity in men's actions, nor in anything that
happens on this earth, it would be all the more difficult to imagine a
first cause for all motion. For my own part, I feel myself so thoroughly
convinced that the natural state of matter is a state of rest, and that
it has no power of action in itself, that when I see a body in motion I
at once assume that it is either a living body or that this motion has
been imparted to it. My mind declines to accept in any way the idea of
inorganic matter moving of its own accord, or giving rise to any action.

Yet this visible universe consists of matter, matter diffused and dead,
{[}Footnote: I have tried hard to grasp the idea of a living molecule,
but in vain. The idea of matter feeling without any senses seems to me
unintelligible and self-contradictory. To accept or reject this idea one
must first understand it, and I confess that so far I have not
succeeded.{]} matter which has none of the cohesion, the organisation,
the common feeling of the parts of a living body, for it is certain that
we who are parts have no consciousness of the whole. This same universe
is in motion, and in its movements, ordered, uniform, and subject to
fixed laws, it has none of that freedom which appears in the spontaneous
movements of men and animals. So the world is not some huge animal which
moves of its own accord; its movements are therefore due to some
external cause, a cause which I cannot perceive, but the inner voice
makes this cause so apparent to me that I cannot watch the course of the
sun without imagining a force which drives it, and when the earth
revolves I think I see the hand that sets it in motion.

If I must accept general laws whose essential relation to matter is
unperceived by me, how much further have I got? These laws, not being
real things, not being substances, have therefore some other basis
unknown to me. Experiment and observation have acquainted us with the
laws of motion; these laws determine the results without showing their
causes; they are quite inadequate to explain the system of the world and
the course of the universe. With the help of dice Descartes made heaven
and earth; but he could not set his dice in motion, nor start the action
of his centrifugal force without the help of rotation. Newton discovered
the law of gravitation; but gravitation alone would soon reduce the
universe to a motionless mass; he was compelled to add a projectile
force to account for the elliptical course of the celestial bodies; let
Newton show us the hand that launched the planets in the tangent of
their orbits.

The first causes of motion are not to be found in matter; matter
receives and transmits motion, but does not produce it. The more I
observe the action and reaction of the forces of nature playing on one
another, the more I see that we must always go back from one effect to
another, till we arrive at a first cause in some will; for to assume an
infinite succession of causes is to assume that there is no first cause.
In a word, no motion which is not caused by another motion can take
place, except by a spontaneous, voluntary action; inanimate bodies have
no action but motion, and there is no real action without will. This is
my first principle. I believe, therefore, that there is a will which
sets the universe in motion and gives life to nature. This is my first
dogma, or the first article of my creed.

How does a will produce a physical and corporeal action? I cannot tell,
but I perceive that it does so in myself; I will to do something and I
do it; I will to move my body and it moves, but if an inanimate body,
when at rest, should begin to move itself, the thing is incomprehensible
and without precedent. The will is known to me in its action, not in its
nature. I know this will as a cause of motion, but to conceive of matter
as producing motion is clearly to conceive of an effect without a cause,
which is not to conceive at all.

It is no more possible for me to conceive how my will moves my body than
to conceive how my sensations affect my mind. I do not even know why one
of these mysteries has seemed less inexplicable than the other. For my
own part, whether I am active or passive, the means of union of the two
substances seem to me absolutely incomprehensible. It is very strange
that people make this very incomprehensibility a step towards the
compounding of the two substances, as if operations so different in kind
were more easily explained in one case than in two.

The doctrine I have just laid down is indeed obscure; but at least it
suggests a meaning and there is nothing in it repugnant to reason or
experience; can we say as much of materialism? Is it not plain that if
motion is essential to matter it would be inseparable from it, it would
always be present in it in the same degree, always present in every
particle of matter, always the same in each particle of matter, it would
not be capable of transmission, it could neither increase nor diminish,
nor could we ever conceive of matter at rest. When you tell me that
motion is not essential to matter but necessary to it, you try to cheat
me with words which would be easier to refute if there was a little more
sense in them. For either the motion of matter arises from the matter
itself and is therefore essential to it; or it arises from an external
cause and is not necessary to the matter, because the motive cause acts
upon it; we have got back to our original difficulty.

The chief source of human error is to be found in general and abstract
ideas; the jargon of metaphysics has never led to the discovery of any
single truth, and it has filled philosophy with absurdities of which we
are ashamed as soon as we strip them of their long words. Tell me, my
friend, when they talk to you of a blind force diffused throughout
nature, do they present any real idea to your mind? They think they are
saying something by these vague expressions---universal force, essential
motion---but they are saying nothing at all. The idea of motion is
nothing more than the idea of transference from place to place; there is
no motion without direction; for no individual can move all ways at
once. In what direction then does matter move of necessity? Has the
whole body of matter a uniform motion, or has each atom its own motion?
According to the first idea the whole universe must form a solid and
indivisible mass; according to the second it can only form a diffused
and incoherent fluid, which would make the union of any two atoms
impossible. What direction shall be taken by this motion common to all
matter? Shall it be in a straight line, in a circle, or from above
downwards, to the right or to the left? If each molecule has its own
direction, what are the causes of all these directions and all these
differences? If every molecule or atom only revolved on its own axis,
nothing would ever leave its place and there would be no transmitted
motion, and even then this circular movement would require to follow
some direction. To set matter in motion by an abstraction is to utter
words without meaning, and to attribute to matter a given direction is
to assume a determining cause. The more examples I take, the more causes
I have to explain, without ever finding a common agent which controls
them. Far from being able to picture to myself an entire absence of
order in the fortuitous concurrence of elements, I cannot even imagine
such a strife, and the chaos of the universe is less conceivable to me
than its harmony. I can understand that the mechanism of the universe
may not be intelligible to the human mind, but when a man sets to work
to explain it, he must say what men can understand.

If matter in motion points me to a will, matter in motion according to
fixed laws points me to an intelligence; that is the second article of
my creed. To act, to compare, to choose, are the operations of an
active, thinking being; so this being exists. Where do you find him
existing, you will say? Not merely in the revolving heavens, nor in the
sun which gives us light, not in myself alone, but in the sheep that
grazes, the bird that flies, the stone that falls, and the leaf blown by
the wind.

I judge of the order of the world, although I know nothing of its
purpose, for to judge of this order it is enough for me to compare the
parts one with another, to study their co-operation, their relations,
and to observe their united action. I know not why the universe exists,
but I see continually how it is changed; I never fail to perceive the
close connection by which the entities of which it consists lend their
aid one to another. I am like a man who sees the works of a watch for
the first time; he is never weary of admiring the mechanism, though he
does not know the use of the instrument and has never seen its face. I
do not know what this is for, says he, but I see that each part of it is
fitted to the rest, I admire the workman in the details of his work, and
I am quite certain that all these wheels only work together in this
fashion for some common end which I cannot perceive.

Let us compare the special ends, the means, the ordered relations of
every kind, then let us listen to the inner voice of feeling; what
healthy mind can reject its evidence? Unless the eyes are blinded by
prejudices, can they fail to see that the visible order of the universe
proclaims a supreme intelligence? What sophisms must be brought together
before we fail to understand the harmony of existence and the wonderful
co-operation of every part for the maintenance of the rest? Say what you
will of combinations and probabilities; what do you gain by reducing me
to silence if you cannot gain my consent? And how can you rob me of the
spontaneous feeling which, in spite of myself, continually gives you the
lie? If organised bodies had come together fortuitously in all sorts of
ways before assuming settled forms, if stomachs are made without mouths,
feet without heads, hands without arms, imperfect organs of every kind
which died because they could not preserve their life, why do none of
these imperfect attempts now meet our eyes; why has nature at length
prescribed laws to herself which she did not at first recognise? I must
not be surprised if that which is possible should happen, and if the
improbability of the event is compensated for by the number of the
attempts. I grant this; yet if any one told me that printed characters
scattered broadcast had produced the Aeneid all complete, I would not
condescend to take a single step to verify this falsehood. You will tell
me I am forgetting the multitude of attempts. But how many such attempts
must I assume to bring the combination within the bounds of probability?
For my own part the only possible assumption is that the chances are
infinity to one that the product is not the work of chance. In addition
to this, chance combinations yield nothing but products of the same
nature as the elements combined, so that life and organisation will not
be produced by a flow of atoms, and a chemist when making his compounds
will never give them thought and feeling in his crucible. {[}Footnote:
Could one believe, if one had not seen it, that human absurdity could go
so far? Amatus Lusitanus asserts that he saw a little man an inch long
enclosed in a glass, which Julius Camillus, like a second Prometheus,
had made by alchemy. Paracelsis (De natura rerum) teaches the method of
making these tiny men, and he maintains that the pygmies, fauns, satyrs,
and nymphs have been made by chemistry. Indeed I cannot see that there
is anything more to be done, to establish the possibility of these
facts, unless it is to assert that organic matter resists the heat of
fire and that its molecules can preserve their life in the hottest
furnace.{]}

I was surprised and almost shocked when I read Neuwentit. How could this
man desire to make a book out of the wonders of nature, wonders which
show the wisdom of the author of nature? His book would have been as
large as the world itself before he had exhausted his subject, and as
soon as we attempt to give details, that greatest wonder of all, the
concord and harmony of the whole, escapes us. The mere generation of
living organic bodies is the despair of the human mind; the
insurmountable barrier raised by nature between the various species, so
that they should not mix with one another, is the clearest proof of her
intention. She is not content to have established order, she has taken
adequate measures to prevent the disturbance of that order.

There is not a being in the universe which may not be regarded as in
some respects the common centre of all, around which they are grouped,
so that they are all reciprocally end and means in relation to each
other. The mind is confused and lost amid these innumerable relations,
not one of which is itself confused or lost in the crowd. What absurd
assumptions are required to deduce all this harmony from the blind
mechanism of matter set in motion by chance! In vain do those who deny
the unity of intention manifested in the relations of all the parts of
this great whole, in vain do they conceal their nonsense under
abstractions, co-ordinations, general principles, symbolic expressions;
whatever they do I find it impossible to conceive of a system of
entities so firmly ordered unless I believe in an intelligence that
orders them. It is not in my power to believe that passive and dead
matter can have brought forth living and feeling beings, that blind
chance has brought forth intelligent beings, that that which does not
think has brought forth thinking beings.

I believe, therefore, that the world is governed by a wise and powerful
will; I see it or rather I feel it, and it is a great thing to know
this. But has this same world always existed, or has it been created? Is
there one source of all things? Are there two or many? What is their
nature? I know not; and what concern is it of mine? When these things
become of importance to me I will try to learn them; till then I abjure
these idle speculations, which may trouble my peace, but cannot affect
my conduct nor be comprehended by my reason.

Recollect that I am not preaching my own opinion but explaining it.
Whether matter is eternal or created, whether its origin is passive or
not, it is still certain that the whole is one, and that it proclaims a
single intelligence; for I see nothing that is not part of the same
ordered system, nothing which does not co-operate to the same end,
namely, the conservation of all within the established order. This being
who wills and can perform his will, this being active through his own
power, this being, whoever he may be, who moves the universe and orders
all things, is what I call God. To this name I add the ideas of
intelligence, power, will, which I have brought together, and that of
kindness which is their necessary consequence; but for all this I know
no more of the being to which I ascribe them. He hides himself alike
from my senses and my understanding; the more I think of him, the more
perplexed I am; I know full well that he exists, and that he exists of
himself alone; I know that my existence depends on his, and that
everything I know depends upon him also. I see God everywhere in his
works; I feel him within myself; I behold him all around me; but if I
try to ponder him himself, if I try to find out where he is, what he is,
what is his substance, he escapes me and my troubled spirit finds
nothing.

Convinced of my unfitness, I shall never argue about the nature of God
unless I am driven to it by the feeling of his relations with myself.
Such reasonings are always rash; a wise man should venture on them with
trembling, he should be certain that he can never sound their abysses;
for the most insolent attitude towards God is not to abstain from
thinking of him, but to think evil of him.

After the discovery of such of his attributes as enable me to conceive
of his existence, I return to myself, and I try to discover what is my
place in the order of things which he governs, and I can myself examine.
At once, and beyond possibility of doubt, I discover my species; for by
my own will and the instruments I can control to carry out my will, I
have more power to act upon all bodies about me, either to make use of
or to avoid their action at my pleasure, than any of them has power to
act upon me against my will by mere physical impulsion; and through my
intelligence I am the only one who can examine all the rest. What being
here below, except man, can observe others, measure, calculate, forecast
their motions, their effects, and unite, so to speak, the feeling of a
common existence with that of his individual existence? What is there so
absurd in the thought that all things are made for me, when I alone can
relate all things to myself?

It is true, therefore, that man is lord of the earth on which he dwells;
for not only does he tame all the beasts, not only does he control its
elements through his industry; but he alone knows how to control it; by
contemplation he takes possession of the stars which he cannot approach.
Show me any other creature on earth who can make a fire and who can
behold with admiration the sun. What! can I observe and know all
creatures and their relations; can I feel what is meant by order,
beauty, and virtue; can I consider the universe and raise myself towards
the hand that guides it; can I love good and perform it; and should I
then liken myself to the beasts? Wretched soul, it is your gloomy
philosophy which makes you like the beasts; or rather in vain do you
seek to degrade yourself; your genius belies your principles, your
kindly heart belies your doctrines, and even the abuse of your powers
proves their excellence in your own despite.

For myself, I am not pledged to the support of any system. I am a plain
and honest man, one who is not carried away by party spirit, one who has
no ambition to be head of a sect; I am content with the place where God
has set me; I see nothing, next to God himself, which is better than my
species; and if I had to choose my place in the order of creation, what
more could I choose than to be a man!

I am not puffed up by this thought, I am deeply moved by it; for this
state was no choice of mine, it was not due to the deserts of a creature
who as yet did not exist. Can I behold myself thus distinguished without
congratulating myself on this post of honour, without blessing the hand
which bestowed it? The first return to self has given birth to a feeling
of gratitude and thankfulness to the author of my species, and this
feeling calls forth my first homage to the beneficent Godhead. I worship
his Almighty power and my heart acknowledges his mercies. Is it not a
natural consequence of our self-love to honour our protector and to love
our benefactor?

But when, in my desire to discover my own place within my species, I
consider its different ranks and the men who fill them, where am I now?
What a sight meets my eyes! Where is now the order I perceived? Nature
showed me a scene of harmony and proportion; the human race shows me
nothing but confusion and disorder. The elements agree together; men are
in a state of chaos. The beasts are happy; their king alone is wretched.
O Wisdom, where are thy laws? O Providence, is this thy rule over the
world? Merciful God, where is thy Power? I behold the earth, and there
is evil upon it.

Would you believe it, dear friend, from these gloomy thoughts and
apparent contradictions, there was shaped in my mind the sublime idea of
the soul, which all my seeking had hitherto failed to discover? While I
meditated upon man's nature, I seemed to discover two distinct
principles in it; one of them raised him to the study of the eternal
truths, to the love of justice, and of true morality, to the regions of
the world of thought, which the wise delight to contemplate; the other
led him downwards to himself, made him the slave of his senses, of the
passions which are their instruments, and thus opposed everything
suggested to him by the former principle. When I felt myself carried
away, distracted by these conflicting motives, I said, No; man is not
one; I will and I will not; I feel myself at once a slave and a free
man; I perceive what is right, I love it, and I do what is wrong; I am
active when I listen to the voice of reason; I am passive when I am
carried away by my passions; and when I yield, my worst suffering is the
knowledge that I might have resisted.

Young man, hear me with confidence. I will always be honest with you. If
conscience is the creature of prejudice, I am certainly wrong, and there
is no such thing as a proof of morality; but if to put oneself first is
an inclination natural to man, and if the first sentiment of justice is
moreover inborn in the human heart, let those who say man is a simple
creature remove these contradictions and I will grant that there is but
one substance.

You will note that by this term `substance' I understand generally the
being endowed with some primitive quality, apart from all special and
secondary modifications. If then all the primitive qualities which are
known to us can be united in one and the same being, we should only
acknowledge one substance; but if there are qualities which are mutually
exclusive, there are as many different substances as there are such
exclusions. You will think this over; for my own part, whatever Locke
may say, it is enough for me to recognise matter as having merely
extension and divisibility to convince myself that it cannot think, and
if a philosopher tells me that trees feel and rocks think {[}Footnote:
It seems to me that modern philosophy, far from saying that rocks think,
has discovered that men do not think. It perceives nothing more in
nature than sensitive beings; and the only difference it finds between a
man and a stone is that a man is a sensitive being which experiences
sensations, and a stone is a sensitive being which does not experience
sensations. But if it is true that all matter feels, where shall I find
the sensitive unit, the individual ego? Shall it be in each molecule of
matter or in bodies as aggregates of molecules? Shall I place this unity
in fluids and solids alike, in compounds and in elements? You tell me
nature consists of individuals. But what are these individuals? Is that
stone an individual or an aggregate of individuals? Is it a single
sensitive being, or are there as many beings in it as there are grains
of sand? If every elementary atom is a sensitive being, how shall I
conceive of that intimate communication by which one feels within the
other, so that their two egos are blended in one? Attraction may be a
law of nature whose mystery is unknown to us; but at least we conceive
that there is nothing in attraction acting in proportion to mass which
is contrary to extension and divisibility. Can you conceive of sensation
in the same way? The sensitive parts have extension, but the sensitive
being is one and indivisible; he cannot be cut in two, he is a whole or
he is nothing; therefore the sensitive being is not a material body. I
know not how our materialists understand it, but it seems to me that the
same difficulties which have led them to reject thought, should have
made them also reject feeling; and I see no reason why, when the first
step has been taken, they should not take the second too; what more
would it cost them? Since they are certain they do not think, why do
they dare to affirm that they feel?{]} in vain will he perplex me with
his cunning arguments; I merely regard him as a dishonest sophist, who
prefers to say that stones have feeling rather than that men have souls.

Suppose a deaf man denies the existence of sounds because he has never
heard them. I put before his eyes a stringed instrument and cause it to
sound in unison by means of another instrument concealed from him; the
deaf man sees the chord vibrate. I tell him, ``The sound makes it do
that.'' ``Not at all,'' says he, ``the string itself is the cause of the
vibration; to vibrate in that way is a quality common to all bodies.''
``Then show me this vibration in other bodies,'' I answer, ``or at least
show me its cause in this string.'' ``I cannot,'' replies the deaf man;
``but because I do not understand how that string vibrates why should I
try to explain it by means of your sounds, of which I have not the least
idea? It is explaining one obscure fact by means of a cause still more
obscure. Make me perceive your sounds; or I say there are no such
things.''

The more I consider thought and the nature of the human mind, the more
likeness I find between the arguments of the materialists and those of
the deaf man. Indeed, they are deaf to the inner voice which cries aloud
to them, in a tone which can hardly be mistaken. A machine does not
think, there is neither movement nor form which can produce reflection;
something within thee tries to break the bands which confine it; space
is not thy measure, the whole universe does not suffice to contain thee;
thy sentiments, thy desires, thy anxiety, thy pride itself, have another
origin than this small body in which thou art imprisoned.

No material creature is in itself active, and I am active. In vain do
you argue this point with me; I feel it, and it is this feeling which
speaks to me more forcibly than the reason which disputes it. I have a
body which is acted upon by other bodies, and it acts in turn upon them;
there is no doubt about this reciprocal action; but my will is
independent of my senses; I consent or I resist; I yield or I win the
victory, and I know very well in myself when I have done what I wanted
and when I have merely given way to my passions. I have always the power
to will, but not always the strength to do what I will. When I yield to
temptation I surrender myself to the action of external objects. When I
blame myself for this weakness, I listen to my own will alone; I am a
slave in my vices, a free man in my remorse; the feeling of freedom is
never effaced in me but when I myself do wrong, and when I at length
prevent the voice of the soul from protesting against the authority of
the body.

I am only aware of will through the consciousness of my own will, and
intelligence is no better known to me. When you ask me what is the cause
which determines my will, it is my turn to ask what cause determines my
judgment; for it is plain that these two causes are but one; and if you
understand clearly that man is active in his judgments, that his
intelligence is only the power to compare and judge, you will see that
his freedom is only a similar power or one derived from this; he chooses
between good and evil as he judges between truth and falsehood; if his
judgment is at fault, he chooses amiss. What then is the cause that
determines his will? It is his judgment. And what is the cause that
determines his judgment? It is his intelligence, his power of judging;
the determining cause is in himself. Beyond that, I understand nothing.

No doubt I am not free not to desire my own welfare, I am not free to
desire my own hurt; but my freedom consists in this very thing, that I
can will what is for my own good, or what I esteem as such, without any
external compulsion. Does it follow that I am not my own master because
I cannot be other than myself?

The motive power of all action is in the will of a free creature; we can
go no farther. It is not the word freedom that is meaningless, but the
word necessity. To suppose some action which is not the effect of an
active motive power is indeed to suppose effects without cause, to
reason in a vicious circle. Either there is no original impulse, or
every original impulse has no antecedent cause, and there is no will
properly so-called without freedom. Man is therefore free to act, and as
such he is animated by an immaterial substance; that is the third
article of my creed. From these three you will easily deduce the rest,
so that I need not enumerate them.

If man is at once active and free, he acts of his own accord; what he
does freely is no part of the system marked out by Providence and it
cannot be imputed to Providence. Providence does not will the evil that
man does when he misuses the freedom given to him; neither does
Providence prevent him doing it, either because the wrong done by so
feeble a creature is as nothing in its eyes, or because it could not
prevent it without doing a greater wrong and degrading his nature.
Providence has made him free that he may choose the good and refuse the
evil. It has made him capable of this choice if he uses rightly the
faculties bestowed upon him, but it has so strictly limited his powers
that the misuse of his freedom cannot disturb the general order. The
evil that man does reacts upon himself without affecting the system of
the world, without preventing the preservation of the human species in
spite of itself. To complain that God does not prevent us from doing
wrong is to complain because he has made man of so excellent a nature,
that he has endowed his actions with that morality by which they are
ennobled, that he has made virtue man's birthright. Supreme happiness
consists in self-content; that we may gain this self-content we are
placed upon this earth and endowed with freedom, we are tempted by our
passions and restrained by conscience. What more could divine power
itself have done on our behalf? Could it have made our nature a
contradiction, and have given the prize of well-doing to one who was
incapable of evil? To prevent a man from wickedness, should Providence
have restricted him to instinct and made him a fool? Not so, O God of my
soul, I will never reproach thee that thou hast created me in thine own
image, that I may be free and good and happy like my Maker!

It is the abuse of our powers that makes us unhappy and wicked. Our
cares, our sorrows, our sufferings are of our own making. Moral ills are
undoubtedly the work of man, and physical ills would be nothing but for
our vices which have made us liable to them. Has not nature made us feel
our needs as a means to our preservation! Is not bodily suffering a sign
that the machine is out of order and needs attention? Death\ldots{}. Do
not the wicked poison their own life and ours? Who would wish to live
for ever? Death is the cure for the evils you bring upon yourself;
nature would not have you suffer perpetually. How few sufferings are
felt by man living in a state of primitive simplicity! His life is
almost entirely free from suffering and from passion; he neither fears
nor feels death; if he feels it, his sufferings make him desire it;
henceforth it is no evil in his eyes. If we were but content to be
ourselves we should have no cause to complain of our lot; but in the
search for an imaginary good we find a thousand real ills. He who cannot
bear a little pain must expect to suffer greatly. If a man injures his
constitution by dissipation, you try to cure him with medicine; the ill
he fears is added to the ill he feels; the thought of death makes it
horrible and hastens its approach; the more we seek to escape from it,
the more we are aware of it; and we go through life in the fear of
death, blaming nature for the evils we have inflicted on ourselves by
our neglect of her laws.

O Man! seek no further for the author of evil; thou art he. There is no
evil but the evil you do or the evil you suffer, and both come from
yourself. Evil in general can only spring from disorder, and in the
order of the world I find a never failing system. Evil in particular
cases exists only in the mind of those who experience it; and this
feeling is not the gift of nature, but the work of man himself. Pain has
little power over those who, having thought little, look neither before
nor after. Take away our fatal progress, take away our faults and our
vices, take away man's handiwork, and all is well.

Where all is well, there is no such thing as injustice. Justice and
goodness are inseparable; now goodness is the necessary result of
boundless power and of that self-love which is innate in all sentient
beings. The omnipotent projects himself, so to speak, into the being of
his creatures. Creation and preservation are the everlasting work of
power; it does not act on that which has no existence; God is not the
God of the dead; he could not harm and destroy without injury to
himself. The omnipotent can only will what is good. {[}Footnote: The
ancients were right when they called the supreme God Optimus Maximus,
but it would have been better to say Maximus Optimus, for his goodness
springs from his power, he is good because he is great.{]} Therefore he
who is supremely good, because he is supremely powerful, must also be
supremely just, otherwise he would contradict himself; for that love of
order which creates order we call goodness and that love of order which
preserves order we call justice.

Men say God owes nothing to his creatures. I think he owes them all he
promised when he gave them their being. Now to give them the idea of
something good and to make them feel the need of it, is to promise it to
them. The more closely I study myself, the more carefully I consider,
the more plainly do I read these words, ``Be just and you will be
happy.'' It is not so, however, in the present condition of things, the
wicked prospers and the oppression of the righteous continues. Observe
how angry we are when this expectation is disappointed. Conscience
revolts and murmurs against her Creator; she exclaims with cries and
groans, ``Thou hast deceived me.''

``I have deceived thee, rash soul! Who told thee this? Is thy soul
destroyed? Hast thou ceased to exist? O Brutus! O my son! let there be
no stain upon the close of thy noble life; do not abandon thy hope and
thy glory with thy corpse upon the plains of Philippi. Why dost thou
say, `Virtue is naught,' when thou art about to enjoy the reward of
virtue? Thou art about to die! Nay, thou shalt live, and thus my promise
is fulfilled.''

One might judge from the complaints of impatient men that God owes them
the reward before they have deserved it, that he is bound to pay for
virtue in advance. Oh! let us first be good and then we shall be happy.
Let us not claim the prize before we have won it, nor demand our wages
before we have finished our work. ``It is not in the lists that we crown
the victors in the sacred games,'' says Plutarch, ``it is when they have
finished their course.''

If the soul is immaterial, it may survive the body; and if it so
survives, Providence is justified. Had I no other proof of the
immaterial nature of the soul, the triumph of the wicked and the
oppression of the righteous in this world would be enough to convince
me. I should seek to resolve so appalling a discord in the universal
harmony. I should say to myself, ``All is not over with life, everything
finds its place at death.'' I should still have to answer the question,
``What becomes of man when all we know of him through our senses has
vanished?'' This question no longer presents any difficulty to me when I
admit the two substances. It is easy to understand that what is
imperceptible to those senses escapes me, during my bodily life, when I
perceive through my senses only. When the union of soul and body is
destroyed, I think one may be dissolved and the other may be preserved.
Why should the destruction of the one imply the destruction of the
other? On the contrary, so unlike in their nature, they were during
their union in a highly unstable condition, and when this union comes to
an end they both return to their natural state; the active vital
substance regains all the force which it expended to set in motion the
passive dead substance. Alas! my vices make me only too well aware that
man is but half alive during this life; the life of the soul only begins
with the death of the body.

But what is that life? Is the soul of man in its nature immortal? I know
not. My finite understanding cannot hold the infinite; what is called
eternity eludes my grasp. What can I assert or deny, how can I reason
with regard to what I cannot conceive? I believe that the soul survives
the body for the maintenance of order; who knows if this is enough to
make it eternal? However, I know that the body is worn out and destroyed
by the division of its parts, but I cannot conceive a similar
destruction of the conscious nature, and as I cannot imagine how it can
die, I presume that it does not die. As this assumption is consoling and
in itself not unreasonable, why should I fear to accept it?

I am aware of my soul; it is known to me in feeling and in thought; I
know what it is without knowing its essence; I cannot reason about ideas
which are unknown to me. What I do know is this, that my personal
identity depends upon memory, and that to be indeed the same self I must
remember that I have existed. Now after death I could not recall what I
was when alive unless I also remembered what I felt and therefore what I
did; and I have no doubt that this remembrance will one day form the
happiness of the good and the torment of the bad. In this world our
inner consciousness is absorbed by the crowd of eager passions which
cheat remorse. The humiliation and disgrace involved in the practice of
virtue do not permit us to realise its charm. But when, freed from the
illusions of the bodily senses, we behold with joy the supreme Being and
the eternal truths which flow from him; when all the powers of our soul
are alive to the beauty of order and we are wholly occupied in comparing
what we have done with what we ought to have done, then it is that the
voice of conscience will regain its strength and sway; then it is that
the pure delight which springs from self-content, and the sharp regret
for our own degradation of that self, will decide by means of
overpowering feeling what shall be the fate which each has prepared for
himself. My good friend, do not ask me whether there are other sources
of happiness or suffering; I cannot tell; that which my fancy pictures
is enough to console me in this life and to bid me look for a life to
come. I do not say the good will be rewarded, for what greater good can
a truly good being expect than to exist in accordance with his nature?
But I do assert that the good will be happy, because their maker, the
author of all justice, who has made them capable of feeling, has not
made them that they may suffer; moreover, they have not abused their
freedom upon earth and they have not changed their fate through any
fault of their own; yet they have suffered in this life and it will be
made up to them in the life to come. This feeling relies not so much on
man's deserts as on the idea of good which seems to me inseparable from
the divine essence. I only assume that the laws of order are constant
and that God is true to himself.

Do not ask me whether the torments of the wicked will endure for ever,
whether the goodness of their creator can condemn them to the eternal
suffering; again, I cannot tell, and I have no empty curiosity for the
investigation of useless problems. How does the fate of the wicked
concern me? I take little interest in it. All the same I find it hard to
believe that they will be condemned to everlasting torments. If the
supreme justice calls for vengeance, it claims it in this life. The
nations of the world with their errors are its ministers. Justice uses
self-inflicted ills to punish the crimes which have deserved them. It is
in your own insatiable souls, devoured by envy, greed, and ambition, it
is in the midst of your false prosperity, that the avenging passions
find the due reward of your crimes. What need to seek a hell in the
future life? It is here in the breast of the wicked.

When our fleeting needs are over, and our mad desires are at rest, there
should also be an end of our passions and our crimes. Can pure spirits
be capable of any perversity? Having need of nothing, why should they be
wicked? If they are free from our gross senses, if their happiness
consists in the contemplation of other beings, they can only desire what
is good; and he who ceases to be bad can never be miserable. This is
what I am inclined to think though I have not been at the pains to come
to any decision. O God, merciful and good, whatever thy decrees may be I
adore them; if thou shouldst commit the wicked to everlasting
punishment, I abandon my feeble reason to thy justice; but if the
remorse of these wretched beings should in the course of time be
extinguished, if their sufferings should come to an end, and if the same
peace shall one day be the lot of all mankind, I give thanks to thee for
this. Is not the wicked my brother? How often have I been tempted to be
like him? Let him be delivered from his misery and freed from the spirit
of hatred that accompanied it; let him be as happy as I myself; his
happiness, far from arousing my jealousy, will only increase my own.

Thus it is that, in the contemplation of God in his works, and in the
study of such of his attributes as it concerned me to know, I have
slowly grasped and developed the idea, at first partial and imperfect,
which I have formed of this Infinite Being. But if this idea has become
nobler and greater it is also more suited to the human reason. As I
approach in spirit the eternal light, I am confused and dazzled by its
glory, and compelled to abandon all the earthly notions which helped me
to picture it to myself. God is no longer corporeal and sensible; the
supreme mind which rules the world is no longer the world itself; in
vain do I strive to grasp his inconceivable essence. When I think that
it is he that gives life and movement to the living and moving substance
which controls all living bodies; when I hear it said that my soul is
spiritual and that God is a spirit, I revolt against this abasement of
the divine essence; as if God and my soul were of one and the same
nature! As if God were not the one and only absolute being, the only
really active, feeling, thinking, willing being, from whom we derive our
thought, feeling, motion, will, our freedom and our very existence! We
are free because he wills our freedom, and his inexplicable substance is
to our souls what our souls are to our bodies. I know not whether he has
created matter, body, soul, the world itself. The idea of creation
confounds me and eludes my grasp; so far as I can conceive of it I
believe it; but I know that he has formed the universe and all that is,
that he has made and ordered all things. No doubt God is eternal; but
can my mind grasp the idea of eternity? Why should I cheat myself with
meaningless words? This is what I do understand; before things
were---God was; he will be when they are no more, and if all things come
to an end he will still endure. That a being beyond my comprehension
should give life to other beings, this is merely difficult and beyond my
understanding; but that Being and Nothing should be convertible terms,
this is indeed a palpable contradiction, an evident absurdity.

God is intelligent, but how? Man is intelligent when he reasons, but the
Supreme Intelligence does not need to reason; there is neither premise
nor conclusion for him, there is not even a proposition. The Supreme
Intelligence is wholly intuitive, it sees what is and what shall be; all
truths are one for it, as all places are but one point and all time but
one moment. Man's power makes use of means, the divine power is
self-active. God can because he wills; his will is his power. God is
good; this is certain; but man finds his happiness in the welfare of his
kind. God's happiness consists in the love of order; for it is through
order that he maintains what is, and unites each part in the whole. God
is just; of this I am sure, it is a consequence of his goodness; man's
injustice is not God's work, but his own; that moral justice which seems
to the philosophers a presumption against Providence, is to me a proof
of its existence. But man's justice consists in giving to each his due;
God's justice consists in demanding from each of us an account of that
which he has given us.

If I have succeeded in discerning these attributes of which I have no
absolute idea, it is in the form of unavoidable deductions, and by the
right use of my reason; but I affirm them without understanding them,
and at bottom that is no affirmation at all. In vain do I say, God is
thus, I feel it, I experience it, none the more do I understand how God
can be thus.

In a word: the more I strive to envisage his infinite essence the less
do I comprehend it; but it is, and that is enough for me; the less I
understand, the more I adore. I abase myself, saying, ``Being of beings,
I am because thou art; to fix my thoughts on thee is to ascend to the
source of my being. The best use I can make of my reason is to resign it
before thee; my mind delights, my weakness rejoices, to feel myself
overwhelmed by thy greatness.''

Having thus deduced from the perception of objects of sense and from my
inner consciousness, which leads me to judge of causes by my native
reason, the principal truths which I require to know, I must now seek
such principles of conduct as I can draw from them, and such rules as I
must lay down for my guidance in the fulfilment of my destiny in this
world, according to the purpose of my Maker. Still following the same
method, I do not derive these rules from the principles of the higher
philosophy, I find them in the depths of my heart, traced by nature in
characters which nothing can efface. I need only consult myself with
regard to what I wish to do; what I feel to be right is right, what I
feel to be wrong is wrong; conscience is the best casuist; and it is
only when we haggle with conscience that we have recourse to the
subtleties of argument. Our first duty is towards ourself; yet how often
does the voice of others tell us that in seeking our good at the expense
of others we are doing ill? We think we are following the guidance of
nature, and we are resisting it; we listen to what she says to our
senses, and we neglect what she says to our heart; the active being
obeys, the passive commands. Conscience is the voice of the soul, the
passions are the voice of the body. It is strange that these voices
often contradict each other? And then to which should we give heed? Too
often does reason deceive us; we have only too good a right to doubt
her; but conscience never deceives us; she is the true guide of man; it
is to the soul what instinct is to the body, {[}Footnote: Modern
philosophy, which only admits what it can understand, is careful not to
admit this obscure power called instinct which seems to guide the
animals to some end without any acquired experience. Instinct, according
to some of our wise philosophers, is only a secret habit of reflection,
acquired by reflection; and from the way in which they explain this
development one ought to suppose that children reflect more than
grown-up people: a paradox strange enough to be worth examining. Without
entering upon this discussion I must ask what name I shall give to the
eagerness with which my dog makes war on the moles he does not eat, or
to the patience with which he sometimes watches them for hours and the
skill with which he seizes them, throws them to a distance from their
earth as soon as they emerge, and then kills them and leaves them. Yet
no one has trained him to this sport, nor even told him there were such
things as moles. Again, I ask, and this is a more important question,
why, when I threatened this same dog for the first time, why did he
throw himself on the ground with his paws folded, in such a suppliant
attitude \ldots{}..calculated to touch me, a position which he would
have maintained if, without being touched by it, I had continued to beat
him in that position? What! Had my dog, little more than a puppy,
acquired moral ideas? Did he know the meaning of mercy and generosity?
By what acquired knowledge did he seek to appease my wrath by yielding
to my discretion? Every dog in the world does almost the same thing in
similar circumstances, and I am asserting nothing but what any one can
verify for himself. Will the philosophers, who so scornfully reject
instinct, kindly explain this fact by the mere play of sensations and
experience which they assume we have acquired? Let them give an account
of it which will satisfy any sensible man; in that case I have nothing
further to urge, and I will say no more of instinct.{]} he who obeys his
conscience is following nature and he need not fear that he will go
astray. This is a matter of great importance, continued my benefactor,
seeing that I was about to interrupt him; let me stop awhile to explain
it more fully.

The morality of our actions consists entirely in the judgments we
ourselves form with regard to them. If good is good, it must be good in
the depth of our heart as well as in our actions; and the first reward
of justice is the consciousness that we are acting justly. If moral
goodness is in accordance with our nature, man can only be healthy in
mind and body when he is good. If it is not so, and if man is by nature
evil, he cannot cease to be evil without corrupting his nature, and
goodness in him is a crime against nature. If he is made to do harm to
his fellow-creatures, as the wolf is made to devour his prey, a humane
man would be as depraved a creature as a pitiful wolf; and virtue alone
would cause remorse.

My young friend, let us look within, let us set aside all personal
prejudices and see whither our inclinations lead us. Do we take more
pleasure in the sight of the sufferings of others or their joys? Is it
pleasanter to do a kind action or an unkind action, and which leaves the
more delightful memory behind it? Why do you enjoy the theatre? Do you
delight in the crimes you behold? Do you weep over the punishment which
overtakes the criminal? They say we are indifferent to everything but
self-interest; yet we find our consolation in our sufferings in the
charms of friendship and humanity, and even in our pleasures we should
be too lonely and miserable if we had no one to share them with us. If
there is no such thing as morality in man's heart, what is the source of
his rapturous admiration of noble deeds, his passionate devotion to
great men? What connection is there between self-interest and this
enthusiasm for virtue? Why should I choose to be Cato dying by his own
hand, rather than Caesar in his triumphs? Take from our hearts this love
of what is noble and you rob us of the joy of life. The mean-spirited
man in whom these delicious feelings have been stifled among vile
passions, who by thinking of no one but himself comes at last to love no
one but himself, this man feels no raptures, his cold heart no longer
throbs with joy, and his eyes no longer fill with the sweet tears of
sympathy, he delights in nothing; the wretch has neither life nor
feeling, he is already dead.

There are many bad men in this world, but there are few of these dead
souls, alive only to self-interest, and insensible to all that is right
and good. We only delight in injustice so long as it is to our own
advantage; in every other case we wish the innocent to be protected. If
we see some act of violence or injustice in town or country, our hearts
are at once stirred to their depths by an instinctive anger and wrath,
which bids us go to the help of the oppressed; but we are restrained by
a stronger duty, and the law deprives us of our right to protect the
innocent. On the other hand, if some deed of mercy or generosity meets
our eye, what reverence and love does it inspire! Do we not say to
ourselves, ``I should like to have done that myself''? What does it
matter to us that two thousand years ago a man was just or unjust? and
yet we take the same interest in ancient history as if it happened
yesterday. What are the crimes of Cataline to me? I shall not be his
victim. Why then have I the same horror of his crimes as if he were
living now? We do not hate the wicked merely because of the harm they do
to ourselves, but because they are wicked. Not only do we wish to be
happy ourselves, we wish others to be happy too, and if this happiness
does not interfere with our own happiness, it increases it. In
conclusion, whether we will or not, we pity the unfortunate; when we see
their suffering we suffer too. Even the most depraved are not wholly
without this instinct, and it often leads them to self-contradiction.
The highwayman who robs the traveller, clothes the nakedness of the
poor; the fiercest murderer supports a fainting man.

Men speak of the voice of remorse, the secret punishment of hidden
crimes, by which such are often brought to light. Alas! who does not
know its unwelcome voice? We speak from experience, and we would gladly
stifle this imperious feeling which causes us such agony. Let us obey
the call of nature; we shall see that her yoke is easy and that when we
give heed to her voice we find a joy in the answer of a good conscience.
The wicked fears and flees from her; he delights to escape from himself;
his anxious eyes look around him for some object of diversion; without
bitter satire and rude mockery he would always be sorrowful; the
scornful laugh is his one pleasure. Not so the just man, who finds his
peace within himself; there is joy not malice in his laughter, a joy
which springs from his own heart; he is as cheerful alone as in company,
his satisfaction does not depend on those who approach him; it includes
them.

Cast your eyes over every nation of the world; peruse every volume of
its history; in the midst of all these strange and cruel forms of
worship, among this amazing variety of manners and customs, you will
everywhere find the same ideas of right and justice; everywhere the same
principles of morality, the same ideas of good and evil. The old
paganism gave birth to abominable gods who would have been punished as
scoundrels here below, gods who merely offered, as a picture of supreme
happiness, crimes to be committed and lust to be gratified. But in vain
did vice descend from the abode of the gods armed with their sacred
authority; the moral instinct refused to admit it into the heart of man.
While the debaucheries of Jupiter were celebrated, the continence of
Xenocrates was revered; the chaste Lucrece adored the shameless Venus;
the bold Roman offered sacrifices to Fear; he invoked the god who
mutilated his father, and he died without a murmur at the hand of his
own father. The most unworthy gods were worshipped by the noblest men.
The sacred voice of nature was stronger than the voice of the gods, and
won reverence upon earth; it seemed to relegate guilt and the guilty
alike to heaven.

There is therefore at the bottom of our hearts an innate principle of
justice and virtue, by which, in spite of our maxims, we judge our own
actions or those of others to be good or evil; and it is this principle
that I call conscience.

But at this word I hear the murmurs of all the wise men so-called.
Childish errors, prejudices of our upbringing, they exclaim in concert!
There is nothing in the human mind but what it has gained by experience;
and we judge everything solely by means of the ideas we have acquired.
They go further; they even venture to reject the clear and universal
agreement of all peoples, and to set against this striking unanimity in
the judgment of mankind, they seek out some obscure exception known to
themselves alone; as if the whole trend of nature were rendered null by
the depravity of a single nation, and as if the existence of
monstrosities made an end of species. But to what purpose does the
sceptic Montaigne strive himself to unearth in some obscure corner of
the world a custom which is contrary to the ideas of justice? To what
purpose does he credit the most untrustworthy travellers, while he
refuses to believe the greatest writers? A few strange and doubtful
customs, based on local causes, unknown to us; shall these destroy a
general inference based on the agreement of all the nations of the
earth, differing from each other in all else, but agreed in this? O
Montaigne, you pride yourself on your truth and honesty; be sincere and
truthful, if a philosopher can be so, and tell me if there is any
country upon earth where it is a crime to keep one's plighted word, to
be merciful, helpful, and generous, where the good man is scorned, and
the traitor is held in honour.

Self-interest, so they say, induces each of us to agree for the common
good. But how is it that the good man consents to this to his own hurt?
Does a man go to death from self-interest? No doubt each man acts for
his own good, but if there is no such thing as moral good to be taken
into consideration, self-interest will only enable you to account for
the deeds of the wicked; possibly you will not attempt to do more. A
philosophy which could find no place for good deeds would be too
detestable; you would find yourself compelled either to find some mean
purpose, some wicked motive, or to abuse Socrates and slander Regulus.
If such doctrines ever took root among us, the voice of nature, together
with the voice of reason, would constantly protest against them, till no
adherent of such teaching could plead an honest excuse for his
partisanship.

It is no part of my scheme to enter at present into metaphysical
discussions which neither you nor I can understand, discussions which
really lead nowhere. I have told you already that I do not wish to
philosophise with you, but to help you to consult your own heart. If all
the philosophers in the world should prove that I am wrong, and you feel
that I am right, that is all I ask.

For this purpose it is enough to lead you to distinguish between our
acquired ideas and our natural feelings; for feeling precedes knowledge;
and since we do not learn to seek what is good for us and avoid what is
bad for us, but get this desire from nature, in the same way the love of
good and the hatred of evil are as natural to us as our self-love. The
decrees of conscience are not judgments but feelings. Although all our
ideas come from without, the feelings by which they are weighed are
within us, and it is by these feelings alone that we perceive fitness or
unfitness of things in relation to ourselves, which leads us to seek or
shun these things.

To exist is to feel; our feeling is undoubtedly earlier than our
intelligence, and we had feelings before we had ideas.{[}Footnote: In
some respects ideas are feelings and feelings are ideas. Both terms are
appropriate to any perception with which we are concerned, appropriate
both to the object of that perception and to ourselves who are affected
by it; it is merely the order in which we are affected which decides the
appropriate term. When we are chiefly concerned with the object and only
think of ourselves as it were by reflection, that is an idea; when, on
the other hand, the impression received excites our chief attention and
we only think in the second place of the object which caused it, it is a
feeling.{]} Whatever may be the cause of our being, it has provided for
our preservation by giving us feelings suited to our nature; and no one
can deny that these at least are innate. These feelings, so far as the
individual is concerned, are self-love, fear, pain, the dread of death,
the desire for comfort. Again, if, as it is impossible to doubt, man is
by nature sociable, or at least fitted to become sociable, he can only
be so by means of other innate feelings, relative to his kind; for if
only physical well-being were considered, men would certainly be
scattered rather than brought together. But the motive power of
conscience is derived from the moral system formed through this twofold
relation to himself and to his fellow-men. To know good is not to love
it; this knowledge is not innate in man; but as soon as his reason leads
him to perceive it, his conscience impels him to love it; it is this
feeling which is innate.

So I do not think, my young friend, that it is impossible to explain the
immediate force of conscience as a result of our own nature, independent
of reason itself. And even should it be impossible, it is unnecessary;
for those who deny this principle, admitted and received by everybody
else in the world, do not prove that there is no such thing; they are
content to affirm, and when we affirm its existence we have quite as
good grounds as they, while we have moreover the witness within us, the
voice of conscience, which speaks on its own behalf. If the first beams
of judgment dazzle us and confuse the objects we behold, let us wait
till our feeble sight grows clear and strong, and in the light of reason
we shall soon behold these very objects as nature has already showed
them to us. Or rather let us be simpler and less pretentious; let us be
content with the first feelings we experience in ourselves, since
science always brings us back to these, unless it has led us astray.

Conscience! Conscience! Divine instinct, immortal voice from heaven;
sure guide for a creature ignorant and finite indeed, yet intelligent
and free; infallible judge of good and evil, making man like to God! In
thee consists the excellence of man's nature and the morality of his
actions; apart from thee, I find nothing in myself to raise me above the
beasts---nothing but the sad privilege of wandering from one error to
another, by the help of an unbridled understanding and a reason which
knows no principle.

Thank heaven we have now got rid of all that alarming show of
philosophy; we may be men without being scholars; now that we need not
spend our life in the study of morality, we have found a less costly and
surer guide through this vast labyrinth of human thought. But it is not
enough to be aware that there is such a guide; we must know her and
follow her. If she speaks to all hearts, how is it that so few give heed
to her voice? She speaks to us in the language of nature, and everything
leads us to forget that tongue. Conscience is timid, she loves peace and
retirement; she is startled by noise and numbers; the prejudices from
which she is said to arise are her worst enemies. She flees before them
or she is silent; their noisy voices drown her words, so that she cannot
get a hearing; fanaticism dares to counterfeit her voice and to inspire
crimes in her name. She is discouraged by ill-treatment; she no longer
speaks to us, no longer answers to our call; when she has been scorned
so long, it is as hard to recall her as it was to banish her.

How often in the course of my inquiries have I grown weary of my own
coldness of heart! How often have grief and weariness poured their
poison into my first meditations and made them hateful to me! My barren
heart yielded nothing but a feeble zeal and a lukewarm love of truth. I
said to myself: Why should I strive to find what does not exist? Moral
good is a dream, the pleasures of sense are the only real good. When
once we have lost the taste for the pleasures of the soul, how hard it
is to recover it! How much more difficult to acquire it if we have never
possessed it! If there were any man so wretched as never to have done
anything all his life long which he could remember with pleasure, and
which would make him glad to have lived, that man would be incapable of
self-knowledge, and for want of knowledge of goodness, of which his
nature is capable, he would be constrained to remain in his wickedness
and would be for ever miserable. But do you think there is any one man
upon earth so depraved that he has never yielded to the temptation of
well-doing? This temptation is so natural, so pleasant, that it is
impossible always to resist it; and the thought of the pleasure it has
once afforded is enough to recall it constantly to our memory. Unluckily
it is hard at first to find satisfaction for it; we have any number of
reasons for refusing to follow the inclinations of our heart; prudence,
so called, restricts the heart within the limits of the self; a thousand
efforts are needed to break these bonds. The joy of well-doing is the
prize of having done well, and we must deserve the prize before we win
it. There is nothing sweeter than virtue; but we do not know this till
we have tried it. Like Proteus in the fable, she first assumes a
thousand terrible shapes when we would embrace her, and only shows her
true self to those who refuse to let her go.

Ever at strife between my natural feelings, which spoke of the common
weal, and my reason, which spoke of self, I should have drifted through
life in perpetual uncertainty, hating evil, loving good, and always at
war with myself, if my heart had not received further light, if that
truth which determined my opinions had not also settled my conduct, and
set me at peace with myself. Reason alone is not a sufficient foundation
for virtue; what solid ground can be found? Virtue we are told is love
of order. But can this love prevail over my love for my own well-being,
and ought it so to prevail? Let them give me clear and sufficient reason
for this preference. Their so-called principle is in truth a mere
playing with words; for I also say that vice is love of order,
differently understood. Wherever there is feeling and intelligence,
there is some sort of moral order. The difference is this: the good man
orders his life with regard to all men; the wicked orders it for self
alone. The latter centres all things round himself; the other measures
his radius and remains on the circumference. Thus his place depends on
the common centre, which is God, and on all the concentric circles which
are His creatures. If there is no God, the wicked is right and the good
man is nothing but a fool.

My child! May you one day feel what a burden is removed when, having
fathomed the vanity of human thoughts and tasted the bitterness of
passion, you find at length near at hand the path of wisdom, the prize
of this life's labours, the source of that happiness which you despaired
of. Every duty of natural law, which man's injustice had almost effaced
from my heart, is engraven there, for the second time in the name of
that eternal justice which lays these duties upon me and beholds my
fulfilment of them. I feel myself merely the instrument of the
Omnipotent, who wills what is good, who performs it, who will bring
about my own good through the co-operation of my will with his own, and
by the right use of my liberty. I acquiesce in the order he establishes,
certain that one day I shall enjoy that order and find my happiness in
it; for what sweeter joy is there than this, to feel oneself a part of a
system where all is good? A prey to pain, I bear it in patience,
remembering that it will soon be over, and that it results from a body
which is not mine. If I do a good deed in secret, I know that it is
seen, and my conduct in this life is a pledge of the life to come. When
I suffer injustice, I say to myself, the Almighty who does all things
well will reward me: my bodily needs, my poverty, make the idea of death
less intolerable. There will be all the fewer bonds to be broken when my
hour comes.

Why is my soul subjected to my senses, and imprisoned in this body by
which it is enslaved and thwarted? I know not; have I entered into the
counsels of the Almighty? But I may, without rashness, venture on a
modest conjecture. I say to myself: If man's soul had remained in a
state of freedom and innocence, what merit would there have been in
loving and obeying the order he found established, an order which it
would not have been to his advantage to disturb? He would be happy, no
doubt, but his happiness would not attain to the highest point, the
pride of virtue, and the witness of a good conscience within him; he
would be but as the angels are, and no doubt the good man will be more
than they. Bound to a mortal body, by bonds as strange as they are
powerful, his care for the preservation of this body tempts the soul to
think only of self, and gives it an interest opposed to the general
order of things, which it is still capable of knowing and loving; then
it is that the right use of his freedom becomes at once the merit and
the reward; then it is that it prepares for itself unending happiness,
by resisting its earthly passions and following its original direction.

If even in the lowly position in which we are placed during our present
life our first impulses are always good, if all our vices are of our own
making, why should we complain that they are our masters? Why should we
blame the Creator for the ills we have ourselves created, and the
enemies we ourselves have armed against us? Oh, let us leave man
unspoilt; he will always find it easy to be good and he will always be
happy without remorse. The guilty, who assert that they are driven to
crime, are liars as well as evil-doers; how is it that they fail to
perceive that the weakness they bewail is of their own making; that
their earliest depravity was the result of their own will; that by dint
of wishing to yield to temptations, they at length yield to them whether
they will or no and make them irresistible? No doubt they can no longer
avoid being weak and wicked, but they need not have become weak and
wicked. Oh, how easy would it be to preserve control of ourselves and of
our passions, even in this life, if with habits still unformed, with a
mind beginning to expand, we were able to keep to such things as we
ought to know, in order to value rightly what is unknown; if we really
wished to learn, not that we might shine before the eyes of others, but
that we might be wise and good in accordance with our nature, that we
might be happy in the performance of our duty. This study seems tedious
and painful to us, for we do not attempt it till we are already
corrupted by vice and enslaved by our passions. Our judgments and our
standards of worth are determined before we have the knowledge of good
and evil; and then we measure all things by this false standard, and
give nothing its true worth.

There is an age when the heart is still free, but eager, unquiet, greedy
of a happiness which is still unknown, a happiness which it seeks in
curiosity and doubt; deceived by the senses it settles at length upon
the empty show of happiness and thinks it has found it where it is not.
In my own case these illusions endured for a long time. Alas! too late
did I become aware of them, and I have not succeeded in overcoming them
altogether; they will last as long as this mortal body from which they
arise. If they lead me astray, I am at least no longer deceived by them;
I know them for what they are, and even when I give way to them, I
despise myself; far from regarding them as the goal of my happiness, I
behold in them an obstacle to it. I long for the time when, freed from
the fetters of the body, I shall be myself, at one with myself, no
longer torn in two, when I myself shall suffice for my own happiness.
Meanwhile I am happy even in this life, for I make small account of all
its evils, in which I regard myself as having little or no part, while
all the real good that I can get out of this life depends on myself
alone.

To raise myself so far as may be even now to this state of happiness,
strength, and freedom, I exercise myself in lofty contemplation. I
consider the order of the universe, not to explain it by any futile
system, but to revere it without ceasing, to adore the wise Author who
reveals himself in it. I hold intercourse with him; I immerse all my
powers in his divine essence; I am overwhelmed by his kindness, I bless
him and his gifts, but I do not pray to him. What should I ask of
him---to change the order of nature, to work miracles on my behalf?
Should I, who am bound to love above all things the order which he has
established in his wisdom and maintained by his providence, should I
desire the disturbance of that order on my own account? No, that rash
prayer would deserve to be punished rather than to be granted. Neither
do I ask of him the power to do right; why should I ask what he has
given me already? Has he not given me conscience that I may love the
right, reason that I may perceive it, and freedom that I may choose it?
If I do evil, I have no excuse; I do it of my own free will; to ask him
to change my will is to ask him to do what he asks of me; it is to want
him to do the work while I get the wages; to be dissatisfied with my lot
is to wish to be no longer a man, to wish to be other than what I am, to
wish for disorder and evil. Thou source of justice and truth, merciful
and gracious God, in thee do I trust, and the desire of my heart
is---Thy will be done. When I unite my will with thine, I do what thou
doest; I have a share in thy goodness; I believe that I enjoy beforehand
the supreme happiness which is the reward of goodness.

In my well-founded self-distrust the only thing that I ask of God, or
rather expect from his justice, is to correct my error if I go astray,
if that error is dangerous to me. To be honest I need not think myself
infallible; my opinions, which seem to me true, may be so many lies; for
what man is there who does not cling to his own beliefs; and how many
men are agreed in everything? The illusion which deceives me may indeed
have its source in myself, but it is God alone who can remove it. I have
done all I can to attain to truth; but its source is beyond my reach; is
it my fault if my strength fails me and I can go no further; it is for
Truth to draw near to me.

The good priest had spoken with passion; he and I were overcome with
emotion. It seemed to me as if I were listening to the divine Orpheus
when he sang the earliest hymns and taught men the worship of the gods.
I saw any number of objections which might be raised; yet I raised none,
for I perceived that they were more perplexing than serious, and that my
inclination took his part. When he spoke to me according to his
conscience, my own seemed to confirm what he said.

``The novelty of the sentiments you have made known to me,'' said I,
``strikes me all the more because of what you confess you do not know,
than because of what you say you believe. They seem to be very like that
theism or natural religion, which Christians profess to confound with
atheism or irreligion which is their exact opposite. But in the present
state of my faith I should have to ascend rather than descend to accept
your views, and I find it difficult to remain just where you are unless
I were as wise as you. That I may be at least as honest, I want time to
take counsel with myself. By your own showing, the inner voice must be
my guide, and you have yourself told me that when it has long been
silenced it cannot be recalled in a moment. I take what you have said to
heart, and I must consider it. If after I have thought things out, I am
as convinced as you are, you will be my final teacher, and I will be
your disciple till death. Continue your teaching however; you have only
told me half what I must know. Speak to me of revelation, of the
Scriptures, of those difficult doctrines among which I have strayed ever
since I was a child, incapable either of understanding or believing
them, unable to adopt or reject them.''

``Yes, my child,'' said he, embracing me, ``I will tell you all I think;
I will not open my heart to you by halves; but the desire you express
was necessary before I could cast aside all reserve. So far I have told
you nothing but what I thought would be of service to you, nothing but
what I was quite convinced of. The inquiry which remains to be made is
very difficult. It seems to me full of perplexity, mystery, and
darkness; I bring to it only doubt and distrust. I make up my mind with
trembling, and I tell you my doubts rather than my convictions. If your
own opinions were more settled I should hesitate to show you mine; but
in your present condition, to think like me would be gain. {[}Footnote:
I think the worthy clergyman might say this at the present time to the
general public.{]} Moreover, give to my words only the authority of
reason; I know not whether I am mistaken. It is difficult in discussion
to avoid assuming sometimes a dogmatic tone; but remember in this
respect that all my assertions are but reasons to doubt me. Seek truth
for yourself, for my own part I only promise you sincerity.

``In my exposition you find nothing but natural religion; strange that we
should need more! How shall I become aware of this need? What guilt can
be mine so long as I serve God according to the knowledge he has given
to my mind, and the feelings he has put into my heart? What purity of
morals, what dogma useful to man and worthy of its author, can I derive
from a positive doctrine which cannot be derived without the aid of this
doctrine by the right use of my faculties? Show me what you can add to
the duties of the natural law, for the glory of God, for the good of
mankind, and for my own welfare; and what virtue you will get from the
new form of religion which does not result from mine. The grandest ideas
of the Divine nature come to us from reason only. Behold the spectacle
of nature; listen to the inner voice. Has not God spoken it all to our
eyes, to our conscience, to our reason? What more can man tell us? Their
revelations do but degrade God, by investing him with passions like our
own. Far from throwing light upon the ideas of the Supreme Being,
special doctrines seem to me to confuse these ideas; far from ennobling
them, they degrade them; to the inconceivable mysteries which surround
the Almighty, they add absurd contradictions, they make man proud,
intolerant, and cruel; instead of bringing peace upon earth, they bring
fire and sword. I ask myself what is the use of it all, and I find no
answer. I see nothing but the crimes of men and the misery of mankind.

``They tell me a revelation was required to teach men how God would be
served; as a proof of this they point to the many strange rites which
men have instituted, and they do not perceive that this very diversity
springs from the fanciful nature of the revelations. As soon as the
nations took to making God speak, every one made him speak in his own
fashion, and made him say what he himself wanted. Had they listened only
to what God says in the heart of man, there would have been but one
religion upon earth.

``One form of worship was required; just so, but was this a matter of
such importance as to require all the power of the Godhead to establish
it? Do not let us confuse the outward forms of religion with religion
itself. The service God requires is of the heart; and when the heart is
sincere that is ever the same. It is a strange sort of conceit which
fancies that God takes such an interest in the shape of the priest's
vestments, the form of words he utters, the gestures he makes before the
altar and all his genuflections. Oh, my friend, stand upright, you will
still be too near the earth. God desires to be worshipped in spirit and
in truth; this duty belongs to every religion, every country, every
individual. As to the form of worship, if order demands uniformity, that
is only a matter of discipline and needs no revelation.

``These thoughts did not come to me to begin with. Carried away by the
prejudices of my education, and by that dangerous vanity which always
strives to lift man out of his proper sphere, when I could not raise my
feeble thoughts up to the great Being, I tried to bring him down to my
own level. I tried to reduce the distance he has placed between his
nature and mine. I desired more immediate relations, more individual
instruction; not content to make God in the image of man that I might be
favoured above my fellows, I desired supernatural knowledge; I required
a special form of worship; I wanted God to tell me what he had not told
others, or what others had not understood like myself.

``Considering the point I had now reached as the common centre from
which all believers set out on the quest for a more enlightened form of
religion, I merely found in natural religion the elements of all
religion. I beheld the multitude of diverse sects which hold sway upon
earth, each of which accuses the other of falsehood and error; which of
these, I asked, is the right? Every one replied, `My own;' every one
said, `I alone and those who agree with me think rightly, all the others
are mistaken.' And how do you know that your sect is in the right?
Because God said so. And how do you know God said so? \footnote{``All
men," said a wise and good priest, ``maintain that they hold and believe
their religion (and all use the same jargon), not of man, nor of any
creature, but of God. But to speak truly, without pretence or flattery,
none of them do so; whatever they may say, religions are taught by human
hands and means; take, for example, the way in which religions have been
received by the world, the way in which they are still received every
day by individuals; the nation, the country, the locality gives the
religion; we belong to the religion of the place where we are born and
brought up; we are baptised or circumcised, we are Christians, Jews,
Mohametans before we know that we are men; we do not pick and choose our
religion for see how ill the life and conduct agree with the religion,
see for what slight and human causes men go against the teaching of
their religion.''---Charron, De la Sagesse.---It seems clear that the
honest creed of the holy theologian of Condom would not have differed
greatly from that of the Savoyard priest.} And who told you that God
said it? My pastor, who knows all about it. My pastor tells me what to
believe and I believe it; he assures me that any one who says anything
else is mistaken, and I give not heed to them.

``What! thought I, is not truth one; can that which is true for me be
false for you? If those who follow the right path and those who go
astray have the same method, what merit or what blame can be assigned to
one more than to the other? Their choice is the result of chance; it is
unjust to hold them responsible for it, to reward or punish them for
being born in one country or another. To dare to say that God judges us
in this manner is an outrage on his justice.

``Either all religions are good and pleasing to God, or if there is one
which he prescribes for men, if they will be punished for despising it,
he will have distinguished it by plain and certain signs by which it can
be known as the only true religion; these signs are alike in every time
and place, equally plain to all men, great or small, learned or
unlearned, Europeans, Indians, Africans, savages. If there were but one
religion upon earth, and if all beyond its pale were condemned to
eternal punishment, and if there were in any corner of the world one
single honest man who was not convinced by this evidence, the God of
that religion would be the most unjust and cruel of tyrants.

``Let us therefore seek honestly after truth; let us yield nothing to the
claims of birth, to the authority of parents and pastors, but let us
summon to the bar of conscience and of reason all that they have taught
us from our childhood. In vain do they exclaim, `Submit your reason;' a
deceiver might say as much; I must have reasons for submitting my
reason.

``All the theology I can get for myself by observation of the universe
and by the use of my faculties is contained in what I have already told
you. To know more one must have recourse to strange means. These means
cannot be the authority of men, for every man is of the same species as
myself, and all that a man knows by nature I am capable of knowing, and
another may be deceived as much as I; when I believe what he says, it is
not because he says it but because he proves its truth. The witness of
man is therefore nothing more than the witness of my own reason, and it
adds nothing to the natural means which God has given me for the
knowledge of truth.

``Apostle of truth, what have you to tell me of which I am not the sole
judge? God himself has spoken; give heed to his revelation. That is
another matter. God has spoken, these are indeed words which demand
attention. To whom has he spoken? He has spoken to men. Why then have I
heard nothing? He has instructed others to make known his words to you.
I understand; it is men who come and tell me what God has said. I would
rather have heard the words of God himself; it would have been as easy
for him and I should have been secure from fraud. He protects you from
fraud by showing that his envoys come from him. How does he show this?
By miracles. Where are these miracles? In the books. And who wrote the
books? Men. And who saw the miracles? The men who bear witness to them.
What! Nothing but human testimony! Nothing but men who tell me what
others told them! How many men between God and me! Let us see, however,
let us examine, compare, and verify. Oh! if God had but deigned to free
me from all this labour, I would have served him with all my heart.

``Consider, my friend, the terrible controversy in which I am now
engaged; what vast learning is required to go back to the remotest
antiquity, to examine, weigh, confront prophecies, revelations, facts,
all the monuments of faith set forth throughout the world, to assign
their date, place, authorship, and occasion. What exactness of critical
judgment is needed to distinguish genuine documents from forgeries, to
compare objections with their answers, translations with their
originals; to decide as to the impartiality of witnesses, their
common-sense, their knowledge; to make sure that nothing has been
omitted, nothing added, nothing transposed, altered, or falsified; to
point out any remaining contradictions, to determine what weight should
be given to the silence of our adversaries with regard to the charges
brought against them; how far were they aware of those charges; did they
think them sufficiently serious to require an answer; were books
sufficiently well known for our books to reach them; have we been honest
enough to allow their books to circulate among ourselves and to leave
their strongest objections unaltered?

``When the authenticity of all these documents is accepted, we must now
pass to the evidence of their authors' mission; we must know the laws of
chance, and probability, to decide which prophecy cannot be fulfilled
without a miracle; we must know the spirit of the original languages, to
distinguish between prophecy and figures of speech; we must know what
facts are in accordance with nature and what facts are not, so that we
may say how far a clever man may deceive the eyes of the simple and may
even astonish the learned; we must discover what are the characteristics
of a prodigy and how its authenticity may be established, not only so
far as to gain credence, but so that doubt may be deserving of
punishment; we must compare the evidence for true and false miracles,
and find sure tests to distinguish between them; lastly we must say why
God chose as a witness to his words means which themselves require so
much evidence on their behalf, as if he were playing with human
credulity, and avoiding of set purpose the true means of persuasion.

``Assuming that the divine majesty condescends so far as to make a man
the channel of his sacred will, is it reasonable, is it fair, to demand
that the whole of mankind should obey the voice of this minister without
making him known as such? Is it just to give him as his sole credentials
certain private signs, performed in the presence of a few obscure
persons, signs which everybody else can only know by hearsay? If one
were to believe all the miracles that the uneducated and credulous
profess to have seen in every country upon earth, every sect would be in
the right; there would be more miracles than ordinary events; and it
would be the greatest miracle if there were no miracles wherever there
were persecuted fanatics. The unchangeable order of nature is the chief
witness to the wise hand that guides it; if there were many exceptions,
I should hardly know what to think; for my own part I have too great a
faith in God to believe in so many miracles which are so little worthy
of him.

``Let a man come and say to us: Mortals, I proclaim to you the will of
the Most Highest; accept my words as those of him who has sent me; I bid
the sun to change his course, the stars to range themselves in a fresh
order, the high places to become smooth, the floods to rise up, the
earth to change her face. By these miracles who will not recognise the
master of nature? She does not obey impostors, their miracles are
wrought in holes and corners, in deserts, within closed doors, where
they find easy dupes among a small company of spectators already
disposed to believe them. Who will venture to tell me how many
eye-witnesses are required to make a miracle credible! What use are your
miracles, performed if proof of your doctrine, if they themselves
require so much proof! You might as well have let them alone.

``There still remains the most important inquiry of all with regard to
the doctrine proclaimed; for since those who tell us God works miracles
in this world, profess that the devil sometimes imitates them, when we
have found the best attested miracles we have got very little further;
and since the magicians of Pharaoh dared in the presence of Moses to
counterfeit the very signs he wrought at God's command, why should they
not, behind his back, claim a like authority? So when we have proved our
doctrine by means of miracles, we must prove our miracles by means of
doctrine, {[}Footnote: This is expressly stated in many passages of
Scripture, among others in Deuteronomy xiii., where it is said that when
a prophet preaching strange gods confirms his words by means of miracles
and what he foretells comes to pass, far from giving heed to him, this
prophet must be put to death. If then the heathen put the apostles to
death when they preached a strange god and confirmed their words by
miracles which came to pass I cannot see what grounds we have for
complaint which they could not at once turn against us. Now, what should
be done in such a case? There is only one course; to return to argument
and let the miracles alone. It would have been better not to have had
recourse to them at all. That is plain common-sense which can only be
obscured by great subtlety of distinction. Subtleties in Christianity!
So Jesus Christ was mistaken when he promised the kingdom of heaven to
the simple, he was mistaken when he began his finest discourse with the
praise of the poor in spirit, if so much wit is needed to understand his
teaching and to get others to believe in him. When you have convinced me
that submission is my duty, all will be well; but to convince me of
this, come down to my level; adapt your arguments to a lowly mind, or I
shall not recognise you as a true disciple of your master, and it is not
his doctrine that you are teaching me.{]} for fear lest we should take
the devil's doings for the handiwork of God. What think you of this
dilemma?

``This doctrine, if it comes from God, should bear the sacred stamp of
the godhead; not only should it illumine the troubled thoughts which
reason imprints on our minds, but it should also offer us a form of
worship, a morality, and rules of conduct in accordance with the
attributes by means of which we alone conceive of God's essence. If then
it teaches us what is absurd and unreasonable, if it inspires us with
feelings of aversion for our fellows and terror for ourselves, if it
paints us a God, angry, jealous, revengeful, partial, hating men, a God
of war and battles, ever ready to strike and to destroy, ever speaking
of punishment and torment, boasting even of the punishment of the
innocent, my heart would not be drawn towards this terrible God, I would
take good care not to quit the realm of natural religion to embrace such
a religion as that; for you see plainly I must choose between them. Your
God is not ours. He who begins by selecting a chosen people, and
proscribing the rest of mankind, is not our common father; he who
consigns to eternal punishment the greater part of his creatures, is not
the merciful and gracious God revealed to me by my reason.

``Reason tells me that dogmas should be plain, clear, and striking in
their simplicity. If there is something lacking in natural religion, it
is with respect to the obscurity in which it leaves the great truths it
teaches; revelation should teach us these truths in a way which the mind
of man can understand; it should bring them within his reach, make him
comprehend them, so that he may believe them. Faith is confirmed and
strengthened by understanding; the best religion is of necessity the
simplest. He who hides beneath mysteries and contradictions the religion
that he preaches to me, teaches me at the same time to distrust that
religion. The God whom I adore is not the God of darkness, he has not
given me understanding in order to forbid me to use it; to tell me to
submit my reason is to insult the giver of reason. The minister of truth
does not tyrannise over my reason, he enlightens it.

``We have set aside all human authority, and without it I do not see how
any man can convince another by preaching a doctrine contrary to reason.
Let them fight it out, and let us see what they have to say with that
harshness of speech which is common to both.

``INSPIRATION: Reason tells you that the whole is greater than the part;
but I tell you, in God's name, that the part is greater than the whole.

``REASON: And who are you to dare to tell me that God contradicts
himself? And which shall I choose to believe. God who teaches me,
through my reason, the eternal truth, or you who, in his name, proclaim
an absurdity?

``INSPIRATION: Believe me, for my teaching is more positive; and I will
prove to you beyond all manner of doubt that he has sent me.

``REASON: What! you will convince me that God has sent you to bear
witness against himself? What sort of proofs will you adduce to convince
me that God speaks more surely by your mouth than through the
understanding he has given me?

``INSPIRATION: The understanding he has given you! Petty, conceited
creature! As if you were the first impious person who had been led
astray through his reason corrupted by sin.

``REASON: Man of God, you would not be the first scoundrel who asserts
his arrogance as a proof of his mission.

``INSPIRATION: What! do even philosophers call names?

``REASON: Sometimes, when the saints set them the example.

``INSPIRATION: Oh, but I have a right to do it, for I am speaking on
God's behalf.

``REASON: You would do well to show your credentials before you make use
of your privileges.

``INSPIRATION: My credentials are authentic, earth and heaven will bear
witness on my behalf. Follow my arguments carefully, if you please.

``REASON: Your arguments! You forget what you are saying. When you teach
me that my reason misleads me, do you not refute what it might have said
on your behalf? He who denies the right of reason, must convince me
without recourse to her aid. For suppose you have convinced me by
reason, how am I to know that it is not my reason, corrupted by sin,
which makes me accept what you say? besides, what proof, what
demonstration, can you advance, more self-evident than the axiom it is
to destroy? It is more credible that a good syllogism is a lie, than
that the part is greater than the whole.

``INSPIRATION: What a difference! There is no answer to my evidence; it
is of a supernatural kind.

``REASON: Supernatural! What do you mean by the word? I do not understand
it.

``INSPIRATION: I mean changes in the order of nature, prophecies, signs,
and wonders of every kind.

``REASON: Signs and wonders! I have never seen anything of the kind.

``INSPIRATION: Others have seen them for you. Clouds of witnesses---the
witness of whole nations\ldots{}.

``REASON: Is the witness of nations supernatural?

"INSPIRATION: No; but when it is unanimous, it is incontestable.

"REASON: There is nothing so incontestable as the principles of reason,
and one cannot accept an absurdity on human evidence. Once more, let us
see your supernatural evidence, for the consent of mankind is not
supernatural.

"INSPIRATION: Oh, hardened heart, grace does not speak to you.

"REASON: That is not my fault; for by your own showing, one must have
already received grace before one is able to ask for it. Begin by
speaking to me in its stead.

"INSPIRATION: But that is just what I am doing, and you will not listen.
But what do you say to prophecy?

"REASON: In the first place, I say I have no more heard a prophet than I
have seen a miracle. In the next, I say that no prophet could claim
authority over me.

"INSPIRATION: Follower of the devil! Why should not the words of the
prophets have authority over you?

"REASON: Because three things are required, three things which will
never happen: firstly, I must have heard the prophecy; secondly, I must
have seen its fulfilment; and thirdly, it must be clearly proved that
the fulfilment of the prophecy could not by any possibility have been a
mere coincidence; for even if it was as precise, as plain, and clear as
an axiom of geometry, since the clearness of a chance prediction does
not make its fulfilment impossible, this fulfilment when it does take
place does not, strictly speaking, prove what was foretold.

``See what your so-called supernatural proofs, your miracles, your
prophecies come to: believe all this upon the word of another. Submit to
the authority of men the authority of God which speaks to my reason. If
the eternal truths which my mind conceives of could suffer any shock,
there would be no sort of certainty for me; and far from being sure that
you speak to me on God's behalf, I should not even be sure that there is
a God.

``My child, here are difficulties enough, but these are not all. Among so
many religions, mutually excluding and proscribing each other, one only
is true, if indeed any one of them is true. To recognise the true
religion we must inquire into, not one, but all; and in any question
whatsoever we have no right to condemn unheard. {[}Footnote: On the
other hand, Plutarch relates that the Stoics maintained, among other
strange paradoxes, that it was no use hearing both sides; for, said
they, the first either proves his point or he does not prove it; if he
has proved it, there is an end of it, and the other should be condemned:
if he has not proved it, he himself is in the wrong and judgment should
be given against him. I consider the method of those who accept an
exclusive revelation very much like that of these Stoics. When each of
them claims to be the sole guardian of truth, we must hear them all
before we can choose between them without injustice.{]} The objections
must be compared with the evidence; we must know what accusation each
brings against the other, and what answers they receive. The plainer any
feeling appears to us, the more we must try to discover why so many
other people refuse to accept it. We should be simple, indeed, if we
thought it enough to hear the doctors on our own side, in order to
acquaint ourselves with the arguments of the other. Where can you find
theologians who pride themselves on their honesty? Where are those who,
to refute the arguments of their opponents, do not begin by making out
that they are of little importance? A man may make a good show among his
own friends, and be very proud of his arguments, who would cut a very
poor figure with those same arguments among those who are on the other
side. Would you find out for yourself from books? What learning you will
need! What languages you must learn; what libraries you must ransack;
what an amount of reading must be got through! Who will guide me in such
a choice? It will be hard to find the best books on the opposite side in
any one country, and all the harder to find those on all sides; when
found they would be easily answered. The absent are always in the wrong,
and bad arguments boldly asserted easily efface good arguments put
forward with scorn. Besides books are often very misleading, and
scarcely express the opinions of their authors. If you think you can
judge the Catholic faith from the writings of Bossuet, you will find
yourself greatly mistaken when you have lived among us. You will see
that the doctrines with which Protestants are answered are quite
different from those of the pulpit. To judge a religion rightly, you
must not study it in the books of its partisans, you must learn it in
their lives; this is quite another matter. Each religion has its own
traditions, meaning, customs, prejudices, which form the spirit of its
creed, and must be taken in connection with it.

``How many great nations neither print books of their own nor read ours!
How shall they judge of our opinions, or we of theirs? We laugh at them,
they despise us; and if our travellers turn them into ridicule, they
need only travel among us to pay us back in our own coin. Are there not,
in every country, men of common-sense, honesty, and good faith, lovers
of truth, who only seek to know what truth is that they may profess it?
Yet every one finds truth in his own religion, and thinks the religion
of other nations absurd; so all these foreign religions are not so
absurd as they seem to us, or else the reason we find for our own proves
nothing.

``We have three principal forms of religion in Europe. One accepts one
revelation, another two, and another three. Each hates the others,
showers curses on them, accuses them of blindness, obstinacy, hardness
of heart, and falsehood. What fair-minded man will dare to decide
between them without first carefully weighing their evidence, without
listening attentively to their arguments? That which accepts only one
revelation is the oldest and seems the best established; that which
accepts three is the newest and seems the most consistent; that which
accepts two revelations and rejects the third may perhaps be the best,
but prejudice is certainly against it; its inconsistency is glaring.

``In all three revelations the sacred books are written in languages
unknown to the people who believe in them. The Jews no longer understand
Hebrew, the Christians understand neither Hebrew nor Greek; the Turks
and Persians do not understand Arabic, and the Arabs of our time do not
speak the language of Mahomet. Is not it a very foolish way of teaching,
to teach people in an unknown tongue? These books are translated, you
say. What an answer! How am I to know that the translations are correct,
or how am I to make sure that such a thing as a correct translation is
possible? If God has gone so far as to speak to men, why should he
require an interpreter?

``I can never believe that every man is obliged to know what is contained
in books, and that he who is out of reach of these books, and of those
who understand them, will be punished for an ignorance which is no fault
of his. Books upon books! What madness! As all Europe is full of books,
Europeans regard them as necessary, forgetting that they are unknown
throughout three-quarters of the globe. Were not all these books written
by men? Why then should a man need them to teach him his duty, and how
did he learn his duty before these books were in existence? Either he
must have learnt his duties for himself, or his ignorance must have been
excused.

``Our Catholics talk loudly of the authority of the Church; but what is
the use of it all, if they also need just as great an array of proofs to
establish that authority as the other seeks to establish their doctrine?
The Church decides that the Church has a right to decide. What a
well-founded authority! Go beyond it, and you are back again in our
discussions.

``Do you know many Christians who have taken the trouble to inquire what
the Jews allege against them? If any one knows anything at all about it,
it is from the writings of Christians. What a way of ascertaining the
arguments of our adversaries! But what is to be done? If any one dared
to publish in our day books which were openly in favour of the Jewish
religion, we should punish the author, publisher, and bookseller. This
regulation is a sure and certain plan for always being in the right. It
is easy to refute those who dare not venture to speak.

``Those among us who have the opportunity of talking with Jews are little
better off. These unhappy people feel that they are in our power; the
tyranny they have suffered makes them timid; they know that Christian
charity thinks nothing of injustice and cruelty; will they dare to run
the risk of an outcry against blasphemy? Our greed inspires us with
zeal, and they are so rich that they must be in the wrong. The more
learned, the more enlightened they are, the more cautious. You may
convert some poor wretch whom you have paid to slander his religion; you
get some wretched old-clothes-man to speak, and he says what you want;
you may triumph over their ignorance and cowardice, while all the time
their men of learning are laughing at your stupidity. But do you think
you would get off so easily in any place where they knew they were safe!
At the Sorbonne it is plain that the Messianic prophecies refer to Jesus
Christ. Among the rabbis of Amsterdam it is just as clear that they have
nothing to do with him. I do not think I have ever heard the arguments
of the Jews as to why they should not have a free state, schools and
universities, where they can speak and argue without danger. Then alone
can we know what they have to say.

``At Constantinople the Turks state their arguments, but we dare not give
ours; then it is our turn to cringe. Can we blame the Turks if they
require us to show the same respect for Mahomet, in whom we do not
believe, as we demand from the Jews with regard to Jesus Christ in whom
they do not believe? Are we right? On what grounds of justice can we
answer this question?

``Two-thirds of mankind are neither Jews, Mahometans, nor Christians; and
how many millions of men have never heard the name of Moses, Jesus
Christ, or Mahomet? They deny it; they maintain that our missionaries go
everywhere. That is easily said. But do they go into the heart of
Africa, still undiscovered, where as yet no European has ever ventured?
Do they go to Eastern Tartary to follow on horseback the wandering
tribes, whom no stranger approaches, who not only know nothing of the
pope, but have scarcely heard tell of the Grand Lama! Do they penetrate
into the vast continents of America, where there are still whole nations
unaware that the people of another world have set foot on their shores?
Do they go to Japan, where their intrigues have led to their perpetual
banishment, where their predecessors are only known to the rising
generation as skilful plotters who came with feigned zeal to take
possession in secret of the empire? Do they reach the harems of the
Asiatic princes to preach the gospel to those thousands of poor slaves?
What have the women of those countries done that no missionary may
preach the faith to them? Will they all go to hell because of their
seclusion?

``If it were true that the gospel is preached throughout the world, what
advantage would there be? The day before the first missionary set foot
in any country, no doubt somebody died who could not hear him. Now tell
me what we shall do with him? If there were a single soul in the whole
world, to whom Jesus Christ had never been preached, this objection
would be as strong for that man as for a quarter of the human race.

``If the ministers of the gospel have made themselves heard among far-off
nations, what have they told them which might reasonably be accepted on
their word, without further and more exact verification? You preach to
me God, born and dying, two thousand years ago, at the other end of the
world, in some small town I know not where; and you tell me that all who
have not believed this mystery are damned. These are strange things to
be believed so quickly on the authority of an unknown person. Why did
your God make these things happen so far off, if he would compel me to
know about them? Is it a crime to be unaware of what is happening half a
world away? Could I guess that in another hemisphere there was a Hebrew
nation and a town called Jerusalem? You might as well expect me to know
what was happening in the moon. You say you have come to teach me; but
why did you not come and teach my father, or why do you consign that
good old man to damnation because he knew nothing of all this? Must he
be punished everlastingly for your laziness, he who was so kind and
helpful, he who sought only for truth? Be honest; put yourself in my
place; see if I ought to believe, on your word alone, all these
incredible things which you have told me, and reconcile all this
injustice with the just God you proclaim to me. At least allow me to go
and see this distant land where such wonders, unheard of in my own
country, took place; let me go and see why the inhabitants of Jerusalem
put their God to death as a robber. You tell me they did not know he was
God. What then shall I do, I who have only heard of him from you? You
say they have been punished, dispersed, oppressed, enslaved; that none
of them dare approach that town. Indeed they richly deserved it; but
what do its present inhabitants say of their crime in slaying their God!
They deny him; they too refuse to recognise God as God. They are no
better than the children of the original inhabitants.

``What! In the very town where God was put to death, neither the former
nor the latter inhabitants knew him, and you expect that I should know
him, I who was born two thousand years after his time, and two thousand
leagues away? Do you not see that before I can believe this book which
you call sacred, but which I do not in the least understand, I must know
from others than yourself when and by whom it was written, how it has
been preserved, how it came into your possession, what they say about it
in those lands where it is rejected, and what are their reasons for
rejecting it, though they know as well as you what you are telling me?
You perceive I must go to Europe, Asia, Palestine, to examine these
things for myself; it would be madness to listen to you before that.

``Not only does this seem reasonable to me, but I maintain that it is
what every wise man ought to say in similar circumstances; that he ought
to banish to a great distance the missionary who wants to instruct and
baptise him all of a sudden before the evidence is verified. Now I
maintain that there is no revelation against which these or similar
objections cannot be made, and with more force than against
Christianity. Hence it follows that if there is but one true religion
and if every man is bound to follow it under pain of damnation, he must
spend his whole life in studying, testing, comparing all these
religions, in travelling through the countries in which they are
established. No man is free from a man's first duty; no one has a right
to depend on another's judgment. The artisan who earns his bread by his
daily toil, the ploughboy who cannot read, the delicate and timid
maiden, the invalid who can scarcely leave his bed, all without
exception must study, consider, argue, travel over the whole world;
there will be no more fixed and settled nations; the whole earth will
swarm with pilgrims on their way, at great cost of time and trouble, to
verify, compare, and examine for themselves the various religions to be
found. Then farewell to the trades, the arts, the sciences of mankind,
farewell to all peaceful occupations; there can be no study but that of
religion, even the strongest, the most industrious, the most
intelligent, the oldest, will hardly be able in his last years to know
where he is; and it will be a wonder if he manages to find out what
religion he ought to live by, before the hour of his death.

``Hard pressed by these arguments, some prefer to make God unjust and to
punish the innocent for the sins of their fathers, rather than to
renounce their barbarous dogmas. Others get out of the difficulty by
kindly sending an angel to instruct all those who in invincible
ignorance have lived a righteous life. A good idea, that angel! Not
content to be the slaves of their own inventions they expect God to make
use of them also!

``Behold, my son, the absurdities to which pride and intolerance bring
us, when everybody wants others to think as he does, and everybody
fancies that he has an exclusive claim upon the rest of mankind. I call
to witness the God of Peace whom I adore, and whom I proclaim to you,
that my inquiries were honestly made; but when I discovered that they
were and always would be unsuccessful, and that I was embarked upon a
boundless ocean, I turned back, and restricted my faith within the
limits of my primitive ideas. I could never convince myself that God
would require such learning of me under pain of hell. So I closed all my
books. There is one book which is open to every one---the book of
nature. In this good and great volume I learn to serve and adore its
Author. There is no excuse for not reading this book, for it speaks to
all in a language they can understand. Suppose I had been born in a
desert island, suppose I had never seen any man but myself, suppose I
had never heard what took place in olden days in a remote corner of the
world; yet if I use my reason, if I cultivate it, if I employ rightly
the innate faculties which God bestows upon me, I shall learn by myself
to know and love him, to love his works, to will what he wills, and to
fulfil all my duties upon earth, that I may do his pleasure. What more
can all human learning teach me?

``With regard to revelation, if I were a more accomplished disputant, or
a more learned person, perhaps I should feel its truth, its usefulness
for those who are happy enough to perceive it; but if I find evidence
for it which I cannot combat, I also find objections against it which I
cannot overcome. There are so many weighty reasons for and against that
I do not know what to decide, so that I neither accept nor reject it. I
only reject all obligation to be convinced of its truth; for this
so-called obligation is incompatible with God's justice, and far from
removing objections in this way it would multiply them, and would make
them insurmountable for the greater part of mankind. In this respect I
maintain an attitude of reverent doubt. I do not presume to think myself
infallible; other men may have been able to make up their minds though
the matter seems doubtful to myself; I am speaking for myself, not for
them; I neither blame them nor follow in their steps; their judgment may
be superior to mine, but it is no fault of mine that my judgment does
not agree with it.

``I own also that the holiness of the gospel speaks to my heart, and that
this is an argument which I should be sorry to refute. Consider the
books of the philosophers with all their outward show; how petty they
are in comparison! Can a book at once so grand and so simple be the work
of men? Is it possible that he whose history is contained in this book
is no more than man? Is the tone of this book, the tone of the
enthusiast or the ambitious sectary? What gentleness and purity in his
actions, what a touching grace in his teaching, how lofty are his
sayings, how profoundly wise are his sermons, how ready, how
discriminating, and how just are his answers! What man, what sage, can
live, suffer, and die without weakness or ostentation? When Plato
describes his imaginary good man, overwhelmed with the disgrace of
crime, and deserving of all the rewards of virtue, every feature of the
portrait is that of Christ; the resemblance is so striking that it has
been noticed by all the Fathers, and there can be no doubt about it.
What prejudices and blindness must there be before we dare to compare
the son of Sophronisca with the son of Mary. How far apart they are!
Socrates dies a painless death, he is not put to open shame, and he
plays his part easily to the last; and if this easy death had not done
honour to his life, we might have doubted whether Socrates, with all his
intellect, was more than a mere sophist. He invented morality, so they
say; others before him had practised it; he only said what they had
done, and made use of their example in his teaching. Aristides was just
before Socrates defined justice; Leonidas died for his country before
Socrates declared that patriotism was a virtue; Sparta was sober before
Socrates extolled sobriety; there were plenty of virtuous men in Greece
before he defined virtue. But among the men of his own time where did
Jesus find that pure and lofty morality of which he is both the teacher
and pattern? {[}Footnote: Cf. in the Sermon on the Mount the parallel he
himself draws between the teaching of Moses and his own.---Matt. v.{]}
The voice of loftiest wisdom arose among the fiercest fanaticism, the
simplicity of the most heroic virtues did honour to the most degraded of
nations. One could wish no easier death than that of Socrates, calmly
discussing philosophy with his friends; one could fear nothing worse
than that of Jesus, dying in torment, among the insults, the mockery,
the curses of the whole nation. In the midst of these terrible
sufferings, Jesus prays for his cruel murderers. Yes, if the life and
death of Socrates are those of a philosopher, the life and death of
Christ are those of a God. Shall we say that the gospel story is the
work of the imagination? My friend, such things are not imagined; and
the doings of Socrates, which no one doubts, are less well attested than
those of Jesus Christ. At best, you only put the difficulty from you; it
would be still more incredible that several persons should have agreed
together to invent such a book, than that there was one man who supplied
its subject matter. The tone and morality of this story are not those of
any Jewish authors, and the gospel indeed contains characters so great,
so striking, so entirely inimitable, that their invention would be more
astonishing than their hero. With all this the same gospel is full of
incredible things, things repugnant to reason, things which no natural
man can understand or accept. What can you do among so many
contradictions? You can be modest and wary, my child; respect in silence
what you can neither reject nor understand, and humble yourself in the
sight of the Divine Being who alone knows the truth.

``This is the unwilling scepticism in which I rest; but this scepticism
is in no way painful to me, for it does not extend to matters of
practice, and I am well assured as to the principles underlying all my
duties. I serve God in the simplicity of my heart; I only seek to know
what affects my conduct. As to those dogmas which have no effect upon
action or morality, dogmas about which so many men torment themselves, I
give no heed to them. I regard all individual religions as so many
wholesome institutions which prescribe a uniform method by which each
country may do honour to God in public worship; institutions which may
each have its reason in the country, the government, the genius of the
people, or in other local causes which make one preferable to another in
a given time or place. I think them all good alike, when God is served
in a fitting manner. True worship is of the heart. God rejects no
homage, however offered, provided it is sincere. Called to the service
of the Church in my own religion, I fulfil as scrupulously as I can all
the duties prescribed to me, and my conscience would reproach me if I
were knowingly wanting with regard to any point. You are aware that
after being suspended for a long time, I have, through the influence of
M. Mellarede, obtained permission to resume my priestly duties, as a
means of livelihood. I used to say Mass with the levity that comes from
long experience even of the most serious matters when they are too
familiar to us; with my new principles I now celebrate it with more
reverence; I dwell upon the majesty of the Supreme Being, his presence,
the insufficiency of the human mind, which so little realises what
concerns its Creator. When I consider how I present before him the
prayers of all the people in a form laid down for me, I carry out the
whole ritual exactly; I give heed to what I say, I am careful not to
omit the least word, the least ceremony; when the moment of the
consecration approaches, I collect my powers, that I may do all things
as required by the Church and by the greatness of this sacrament; I
strive to annihilate my own reason before the Supreme Mind; I say to
myself, Who art thou to measure infinite power? I reverently pronounce
the sacramental words, and I give to their effect all the faith I can
bestow. Whatever may be this mystery which passes understanding, I am
not afraid that at the day of judgment I shall be punished for having
profaned it in my heart.''

Honoured with the sacred ministry, though in its lowest ranks, I will
never do or say anything which may make me unworthy to fulfil these
sublime duties. I will always preach virtue and exhort men to
well-doing; and so far as I can I will set them a good example. It will
be my business to make religion attractive; it will be my business to
strengthen their faith in those doctrines which are really useful, those
which every man must believe; but, please God, I shall never teach them
to hate their neighbour, to say to other men, You will be damned; to
say, No salvation outside the Church. {[}Footnote: The duty of following
and loving the religion of our country does not go so far as to require
us to accept doctrines contrary to good morals, such as intolerance.
This horrible doctrine sets men in arms against their fellow-men, and
makes them all enemies of mankind. The distinction between civil
toleration and theological toleration is vain and childish. These two
kinds of toleration are inseparable, and we cannot accept one without
the other. Even the angels could not live at peace with men whom they
regarded as the enemies of God.{]} If I were in a more conspicuous
position, this reticence might get me into trouble; but I am too obscure
to have much to fear, and I could hardly sink lower than I am. Come what
may, I will never blaspheme the justice of God, nor lie against the Holy
Ghost.

``I have long desired to have a parish of my own; it is still my
ambition, but I no longer hope to attain it. My dear friend, I think
there is nothing so delightful as to be a parish priest. A good
clergyman is a minister of mercy, as a good magistrate is a minister of
justice. A clergyman is never called upon to do evil; if he cannot
always do good himself, it is never out of place for him to beg for
others, and he often gets what he asks if he knows how to gain respect.
Oh! if I should ever have some poor mountain parish where I might
minister to kindly folk, I should be happy indeed; for it seems to me
that I should make my parishioners happy. I should not bring them
riches, but I should share their poverty; I should remove from them the
scorn and opprobrium which are harder to bear than poverty. I should
make them love peace and equality, which often remove poverty, and
always make it tolerable. When they saw that I was in no way better off
than themselves, and that yet I was content with my lot, they would
learn to put up with their fate and to be content like me. In my sermons
I would lay more stress on the spirit of the gospel than on the spirit
of the church; its teaching is simple, its morality sublime; there is
little in it about the practices of religion, but much about works of
charity. Before I teach them what they ought to do, I would try to
practise it myself, that they might see that at least I think what I
say. If there were Protestants in the neighbourhood or in my parish, I
would make no difference between them and my own congregation so far as
concerns Christian charity; I would get them to love one another, to
consider themselves brethren, to respect all religions, and each to live
peaceably in his own religion. To ask any one to abandon the religion in
which he was born is, I consider, to ask him to do wrong, and therefore
to do wrong oneself. While we await further knowledge, let us respect
public order; in every country let us respect the laws, let us not
disturb the form of worship prescribed by law; let us not lead its
citizens into disobedience; for we have no certain knowledge that it is
good for them to abandon their own opinions for others, and on the other
hand we are quite certain that it is a bad thing to disobey the law.

``My young friend, I have now repeated to you my creed as God reads it in
my heart; you are the first to whom I have told it; perhaps you will be
the last. As long as there is any true faith left among men, we must not
trouble quiet souls, nor scare the faith of the ignorant with problems
they cannot solve, with difficulties which cause them uneasiness, but do
not give them any guidance. But when once everything is shaken, the
trunk must be preserved at the cost of the branches. Consciences,
restless, uncertain, and almost quenched like yours, require to be
strengthened and aroused; to set the feet again upon the foundation of
eternal truth, we must remove the trembling supports on which they think
they rest.

``You are at that critical age when the mind is open to conviction, when
the heart receives its form and character, when we decide our own fate
for life, either for good or evil. At a later date, the material has
hardened and fresh impressions leave no trace. Young man, take the stamp
of truth upon your heart which is not yet hardened, if I were more
certain of myself, I should have adopted a more decided and dogmatic
tone; but I am a man ignorant and liable to error; what could I do? I
have opened my heart fully to you; and I have told what I myself hold
for certain and sure; I have told you my doubts as doubts, my opinions
as opinions; I have given you my reasons both for faith and doubt. It is
now your turn to judge; you have asked for time; that is a wise
precaution and it makes me think well of you. Begin by bringing your
conscience into that state in which it desires to see clearly; be honest
with yourself. Take to yourself such of my opinions as convince you,
reject the rest. You are not yet so depraved by vice as to run the risk
of choosing amiss. I would offer to argue with you, but as soon as men
dispute they lose their temper; pride and obstinacy come in, and there
is an end of honesty. My friend, never argue; for by arguing we gain no
light for ourselves or for others. So far as I myself am concerned, I
have only made up my mind after many years of meditation; here I rest,
my conscience is at peace, my heart is satisfied. If I wanted to begin
afresh the examination of my feelings, I should not bring to the task a
purer love of truth; and my mind, which is already less active, would be
less able to perceive the truth. Here I shall rest, lest the love of
contemplation, developing step by step into an idle passion, should make
me lukewarm in the performance of my duties, lest I should fall into my
former scepticism without strength to struggle out of it. More than half
my life is spent; I have barely time to make good use of what is left,
to blot out my faults by my virtues. If I am mistaken, it is against my
will. He who reads my inmost heart knows that I have no love for my
blindness. As my own knowledge is powerless to free me from this
blindness, my only way out of it is by a good life; and if God from the
very stones can raise up children to Abraham, every man has a right to
hope that he may be taught the truth, if he makes himself worthy of it.

``If my reflections lead you to think as I do, if you share my feelings,
if we have the same creed, I give you this advice: Do not continue to
expose your life to the temptations of poverty and despair, nor waste it
in degradation and at the mercy of strangers; no longer eat the shameful
bread of charity. Return to your own country, go back to the religion of
your fathers, and follow it in sincerity of heart, and never forsake it;
it is very simple and very holy; I think there is no other religion upon
earth whose morality is purer, no other more satisfying to the reason.
Do not trouble about the cost of the journey, that will be provided for
you. Neither do you fear the false shame of a humiliating return; we
should blush to commit a fault, not to repair it. You are still at an
age when all is forgiven, but when we cannot go on sinning with
impunity. If you desire to listen to your conscience, a thousand empty
objections will disappear at her voice. You will feel that, in our
present state of uncertainty, it is an inexcusable presumption to
profess any faith but that we were born into, while it is treachery not
to practise honestly the faith we profess. If we go astray, we deprive
ourselves of a great excuse before the tribunal of the sovereign judge.
Will he not pardon the errors in which we were brought up, rather than
those of our own choosing?

``My son, keep your soul in such a state that you always desire that
there should be a God and you will never doubt it. Moreover, whatever
decision you come to, remember that the real duties of religion are
independent of human institutions; that a righteous heart is the true
temple of the Godhead; that in every land, in every sect, to love God
above all things and to love our neighbour as ourself is the whole law;
remember there is no religion which absolves us from our moral duties;
that these alone are really essential, that the service of the heart is
the first of these duties, and that without faith there is no such thing
as true virtue.

``Shun those who, under the pretence of explaining nature, sow
destructive doctrines in the heart of men, those whose apparent
scepticism is a hundredfold more self-assertive and dogmatic than the
firm tone of their opponents. Under the arrogant claim, that they alone
are enlightened, true, honest, they subject us imperiously to their
far-reaching decisions, and profess to give us, as the true principles
of all things, the unintelligible systems framed by their imagination.
Moreover, they overthrow, destroy, and trample under foot all that men
reverence; they rob the afflicted of their last consolation in their
misery; they deprive the rich and powerful of the sole bridle of their
passions; they tear from the very depths of man's heart all remorse for
crime, and all hope of virtue; and they boast, moreover, that they are
the benefactors of the human race. Truth, they say, can never do a man
harm. I think so too, and to my mind that is strong evidence that what
they teach is not true. {[}Footnote: The rival parties attack each other
with so many sophistries that it would be a rash and overwhelming
enterprise to attempt to deal with all of them; it is difficult enough
to note some of them as they occur. One of the commonest errors among
the partisans of philosophy is to contrast a nation of good philosophers
with a nation of bad Christians; as if it were easier to make a nation
of good philosophers than a nation of good Christians. I know not
whether in individual cases it is easier to discover one rather than the
other; but I am quite certain that, as far as nations are concerned, we
must assume that there will be those who misuse their philosophy without
religion, just as our people misuse their religion without philosophy,
and that seems to put quite a different face upon the matter.{]}---Bayle
has proved very satisfactorily that fanaticism is more harmful than
atheism, and that cannot be denied; but what he has not taken the
trouble to say, though it is none the less true, is this: Fanaticism,
though cruel and bloodthirsty, is still a great and powerful passion,
which stirs the heart of man, teaching him to despise death, and giving
him an enormous motive power, which only needs to be guided rightly to
produce the noblest virtues; while irreligion, and the argumentative
philosophic spirit generally, on the other hand, assaults the life and
enfeebles it, degrades the soul, concentrates all the passions in the
basest self-interest, in the meanness of the human self; thus it saps
unnoticed the very foundations of all society, for what is common to all
these private interests is so small that it will never outweigh their
opposing interests.---If atheism does not lead to bloodshed, it is less
from love of peace than from indifference to what is good; as if it
mattered little what happened to others, provided the sage remained
undisturbed in his study. His principles do not kill men, but they
prevent their birth, by destroying the morals by which they were
multiplied, by detaching them from their fellows, by reducing all their
affections to a secret selfishness, as fatal to population as to virtue.
The indifference of the philosopher is like the peace in a despotic
state; it is the repose of death; war itself is not more
destructive.---Thus fanaticism though its immediate results are more
fatal than those of what is now called the philosophic mind, is much
less fatal in its after effects. Moreover, it is an easy matter to
exhibit fine maxims in books; but the real question is---Are they really
in accordance with your teaching, are they the necessary consequences of
it? and this has not been clearly proved so far. It remains to be seen
whether philosophy, safely enthroned, could control successfully man's
petty vanity, his self-interest, his ambition, all the lesser passions
of mankind, and whether it would practise that sweet humanity which it
boasts of, pen in hand.---In theory, there is no good which philosophy
can bring about which is not equally secured by religion, while religion
secures much that philosophy cannot secure.---In practice, it is another
matter; but still we must put it to the proof. No man follows his
religion in all things, even if his religion is true; most people have
hardly any religion, and they do not in the least follow what they have;
that is still more true; but still there are some people who have a
religion and follow it, at least to some extent; and beyond doubt
religious motives do prevent them from wrong-doing, and win from them
virtues, praiseworthy actions, which would not have existed but for
these motives.---A monk denies that money was entrusted to him; what of
that? It only proves that the man who entrusted the money to him was a
fool. If Pascal had done the same, that would have proved that Pascal
was a hypocrite. But a monk! Are those who make a trade of religion
religious people? All the crimes committed by the clergy, as by other
men, do not prove that religion is useless, but that very few people are
religious.---Most certainly our modern governments owe to Christianity
their more stable authority, their less frequent revolutions; it has
made those governments less bloodthirsty; this can be shown by comparing
them with the governments of former times. Apart from fanaticism, the
best known religion has given greater gentleness to Christian conduct.
This change is not the result of learning; for wherever learning has
been most illustrious humanity has been no more respected on that
account; the cruelties of the Athenians, the Egyptians, the Roman
emperors, the Chinese bear witness to this. What works of mercy spring
from the gospel! How many acts of restitution, reparation, confession
does the gospel lead to among Catholics! Among ourselves, as the times
of communion draw near, do they not lead us to reconciliation and to
alms-giving? Did not the Hebrew Jubilee make the grasping less greedy,
did it not prevent much poverty? The brotherhood of the Law made the
nation one; no beggar was found among them. Neither are there beggars
among the Turks, where there are countless pious institutions; from
motives of religion they even show hospitality to the foes of their
religion.---''The Mahometans say, according to Chardin, that after the
interrogation which will follow the general resurrection, all bodies
will traverse a bridge called Poul-Serrho, which is thrown across the
eternal fires, a bridge which may be called the third and last test of
the great Judgment, because it is there that the good and bad will be
separated, etc.---``The Persians, continues Chardin, make a great point
of this bridge; and when any one suffers a wrong which he can never hope
to wipe out by any means or at any time, he finds his last consolation
in these words: `By the living God, you will pay me double at the last
day; you will never get across the Poul-Serrho if you do not first do me
justice; I will hold the hem of your garment, I will cling about your
knees.' I have seen many eminent men, of every profession, who for fear
lest this hue and cry should be raised against them as they cross that
fearful bridge, beg pardon of those who complained against them; it has
happened to me myself on many occasions. Men of rank, who had compelled
me by their importunity to do what I did not wish to do, have come to me
when they thought my anger had had time to cool, and have said to me; I
pray you ''Halal becon antchisra," that is, ``Make this matter lawful
and right.'' Some of them have even sent gifts and done me service, so
that I might forgive them and say I did it willingly; the cause of this
is nothing else but this belief that they will not be able to get across
the bridge of hell until they have paid the uttermost farthing to the
oppressed."---Must I think that the idea of this bridge where so many
iniquities are made good is of no avail? If the Persians were deprived
of this idea, if they were persuaded that there was no Poul-Serrho, nor
anything of the kind, where the oppressed were avenged of their tyrants
after death, is it not clear that they would be very much at their ease,
and they would be freed from the care of appeasing the wretched? But it
is false to say that this doctrine is hurtful; yet it would not be
true.---O Philosopher, your moral laws are all very fine; but kindly
show me their sanction. Cease to shirk the question, and tell me plainly
what you would put in the place of Poul-Serrho.

``My good youth, be honest and humble; learn how to be ignorant, then
you will never deceive yourself or others. If ever your talents are so
far cultivated as to enable you to speak to other men, always speak
according to your conscience, without caring for their applause. The
abuse of knowledge causes incredulity. The learned always despise the
opinions of the crowd; each of them must have his own opinion. A haughty
philosophy leads to atheism just as blind devotion leads to fanaticism.
Avoid these extremes; keep steadfastly to the path of truth, or what
seems to you truth, in simplicity of heart, and never let yourself be
turned aside by pride or weakness. Dare to confess God before the
philosophers; dare to preach humanity to the intolerant. It may be you
will stand alone, but you will bear within you a witness which will make
the witness of men of no account with you. Let them love or hate, let
them read your writings or despise them; no matter. Speak the truth and
do the right; the one thing that really matters is to do one's duty in
this world; and when we forget ourselves we are really working for
ourselves. My child, self-interest misleads us; the hope of the just is
the only sure guide.''

I have transcribed this document not as a rule for the sentiments we
should adopt in matters of religion, but as an example of the way in
which we may reason with our pupil without forsaking the method I have
tried to establish. So long as we yield nothing to human authority, nor
to the prejudices of our native land, the light of reason alone, in a
state of nature, can lead us no further than to natural religion; and
this is as far as I should go with Emile. If he must have any other
religion, I have no right to be his guide; he must choose for himself.

We are working in agreement with nature, and while she is shaping the
physical man, we are striving to shape his moral being, but we do not
make the same progress. The body is already strong and vigorous, the
soul is still frail and delicate, and whatever can be done by human art,
the body is always ahead of the mind. Hitherto all our care has been
devoted to restrain the one and stimulate the other, so that the man
might be as far as possible at one with himself. By developing his
individuality, we have kept his growing susceptibilities in check; we
have controlled it by cultivating his reason. Objects of thought
moderate the influence of objects of sense. By going back to the causes
of things, we have withdrawn him from the sway of the senses; it is an
easy thing to raise him from the study of nature to the search for the
author of nature.

When we have reached this point, what a fresh hold we have got over our
pupil; what fresh ways of speaking to his heart! Then alone does he find
a real motive for being good, for doing right when he is far from every
human eye, and when he is not driven to it by law. To be just in his own
eyes and in the sight of God, to do his duty, even at the cost of life
itself, and to bear in his heart virtue, not only for the love of order
which we all subordinate to the love of self, but for the love of the
Author of his being, a love which mingles with that self-love, so that
he may at length enjoy the lasting happiness which the peace of a good
conscience and the contemplation of that supreme being promise him in
another life, after he has used this life aright. Go beyond this, and I
see nothing but injustice, hypocrisy, and falsehood among men; private
interest, which in competition necessarily prevails over everything
else, teaches all things to adorn vice with the outward show of virtue.
Let all men do what is good for me at the cost of what is good for
themselves; let everything depend on me alone; let the whole human race
perish, if needs be, in suffering and want, to spare me a moment's pain
or hunger. Yes, I shall always maintain that whoso says in his heart,
``There is no God,'' while he takes the name of God upon his lips, is
either a liar or a madman.

Reader, it is all in vain; I perceive that you and I shall never see
Emile with the same eyes; you will always fancy him like your own young
people, hasty, impetuous, flighty, wandering from fete to fete, from
amusement to amusement, never able to settle to anything. You smile when
I expect to make a thinker, a philosopher, a young theologian, of an
ardent, lively, eager, and fiery young man, at the most impulsive period
of youth. This dreamer, you say, is always in pursuit of his fancy; when
he gives us a pupil of his own making, he does not merely form him, he
creates him, he makes him up out of his own head; and while he thinks he
is treading in the steps of nature, he is getting further and further
from her. As for me, when I compare my pupil with yours, I can scarcely
find anything in common between them. So differently brought up, it is
almost a miracle if they are alike in any respect. As his childhood was
passed in the freedom they assume in youth, in his youth he begins to
bear the yoke they bore as children; this yoke becomes hateful to them,
they are sick of it, and they see in it nothing but their masters'
tyranny; when they escape from childhood, they think they must shake off
all control, they make up for the prolonged restraint imposed upon them,
as a prisoner, freed from his fetters, moves and stretches and shakes
his limbs. {[}Footnote: There is no one who looks down upon childhood
with such lofty scorn as those who are barely grown-up; just as there is
no country where rank is more strictly regarded than that where there is
little real inequality; everybody is afraid of being confounded with his
inferiors.{]} Emile, however, is proud to be a man, and to submit to the
yoke of his growing reason; his body, already well grown, no longer
needs so much action, and begins to control itself, while his
half-fledged mind tries its wings on every occasion. Thus the age of
reason becomes for the one the age of licence; for the other, the age of
reasoning.

Would you know which of the two is nearer to the order of nature!
Consider the differences between those who are more or less removed from
a state of nature. Observe young villagers and see if they are as
undisciplined as your scholars. The Sieur de Beau says that savages in
childhood are always active, and ever busy with sports that keep the
body in motion; but scarcely do they reach adolescence than they become
quiet and dreamy; they no longer devote themselves to games of skill or
chance. Emile, who has been brought up in full freedom like young
peasants and savages, should behave like them and change as he grows up.
The whole difference is in this, that instead of merely being active in
sport or for food, he has, in the course of his sports, learned to
think. Having reached this stage, and by this road, he is quite ready to
enter upon the next stage to which I introduce him; the subjects I
suggest for his consideration rouse his curiosity, because they are fine
in themselves, because they are quite new to him, and because he is able
to understand them. Your young people, on the other hand, are weary and
overdone with your stupid lessons, your long sermons, and your tedious
catechisms; why should they not refuse to devote their minds to what has
made them sad, to the burdensome precepts which have been continually
piled upon them, to the thought of the Author of their being, who has
been represented as the enemy of their pleasures? All this has only
inspired in them aversion, disgust, and weariness; constraint has set
them against it; why then should they devote themselves to it when they
are beginning to choose for themselves? They require novelty, you must
not repeat what they learned as children. Just so with my own pupil,
when he is a man I speak to him as a man, and only tell him what is new
to him; it is just because they are tedious to your pupils that he will
find them to his taste.

This is how I doubly gain time for him by retarding nature to the
advantage of reason. But have I indeed retarded the progress of nature?
No, I have only prevented the imagination from hastening it; I have
employed another sort of teaching to counterbalance the precocious
instruction which the young man receives from other sources. When he is
carried away by the flood of existing customs and I draw him in the
opposite direction by means of other customs, this is not to remove him
from his place, but to keep him in it.

Nature's due time comes at length, as come it must. Since man must die,
he must reproduce himself, so that the species may endure and the order
of the world continue. When by the signs I have spoken of you perceive
that the critical moment is at hand, at once abandon for ever your
former tone. He is still your disciple, but not your scholar. He is a
man and your friend; henceforth you must treat him as such.

What! Must I abdicate my authority when most I need it? Must I abandon
the adult to himself just when he least knows how to control himself,
when he may fall into the gravest errors! Must I renounce my rights when
it matters most that I should use them on his behalf? Who bids you
renounce them; he is only just becoming conscious of them. Hitherto all
you have gained has been won by force or guile; authority, the law of
duty, were unknown to him, you had to constrain or deceive him to gain
his obedience. But see what fresh chains you have bound about his heart.
Reason, friendship, affection, gratitude, a thousand bonds of affection,
speak to him in a voice he cannot fail to hear. His ears are not yet
dulled by vice, he is still sensitive only to the passions of nature.
Self-love, the first of these, delivers him into your hands; habit
confirms this. If a passing transport tears him from you, regret
restores him to you without delay; the sentiment which attaches him to
you is the only lasting sentiment, all the rest are fleeting and
self-effacing. Do not let him become corrupt, and he will always be
docile; he will not begin to rebel till he is already perverted.

I grant you, indeed, that if you directly oppose his growing desires and
foolishly treat as crimes the fresh needs which are beginning to make
themselves felt in him, he will not listen to you for long; but as soon
as you abandon my method I cannot be answerable for the consequences.
Remember that you are nature's minister; you will never be her foe.

But what shall we decide to do? You see no alternative but either to
favour his inclinations or to resist them; to tyrannise or to wink at
his misconduct; and both of these may lead to such dangerous results
that one must indeed hesitate between them.

The first way out of the difficulty is a very early marriage; this is
undoubtedly the safest and most natural plan. I doubt, however, whether
it is the best or the most useful. I will give my reasons later;
meanwhile I admit that young men should marry when they reach a
marriageable age. But this age comes too soon; we have made them
precocious; marriage should be postponed to maturity.

If it were merely a case of listening to their wishes and following
their lead it would be an easy matter; but there are so many
contradictions between the rights of nature and the laws of society that
to conciliate them we must continually contradict ourselves. Much art is
required to prevent man in society from being altogether artificial.

For the reasons just stated, I consider that by the means I have
indicated and others like them the young man's desires may be kept in
ignorance and his senses pure up to the age of twenty. This is so true
that among the Germans a young man who lost his virginity before that
age was considered dishonoured; and the writers justly attribute the
vigour of constitution and the number of children among the Germans to
the continence of these nations during youth.

This period may be prolonged still further, and a few centuries ago
nothing was more common even in France. Among other well-known examples,
Montaigne's father, a man no less scrupulously truthful than strong and
healthy, swore that his was a virgin marriage at three and thirty, and
he had served for a long time in the Italian wars. We may see in the
writings of his son what strength and spirit were shown by the father
when he was over sixty. Certainly the contrary opinion depends rather on
our own morals and our own prejudices than on the experience of the race
as a whole.

I may, therefore, leave on one side the experience of our young people;
it proves nothing for those who have been educated in another fashion.
Considering that nature has fixed no exact limits which cannot be
advanced or postponed, I think I may, without going beyond the law of
nature, assume that under my care Emil has so far remained in his first
innocence, but I see that this happy period is drawing to a close.
Surrounded by ever-increasing perils, he will escape me at the first
opportunity in spite of all my efforts, and this opportunity will not
long be delayed; he will follow the blind instinct of his senses; the
chances are a thousand to one on his ruin. I have considered the morals
of mankind too profoundly not to be aware of the irrevocable influence
of this first moment on all the rest of his life. If I dissimulate and
pretend to see nothing, he will take advantage of my weakness; if he
thinks he can deceive me, he will despise me, and I become an accomplice
in his destruction. If I try to recall him, the time is past, he no
longer heeds me, he finds me tiresome, hateful, intolerable; it will not
be long before he is rid of me. There is therefore only one reasonable
course open to me; I must make him accountable for his own actions, I
must at least preserve him from being taken unawares, and I must show
him plainly the dangers which beset his path. I have restrained him so
far through his ignorance; henceforward his restraint must be his own
knowledge.

This new teaching is of great importance, and we will take up our story
where we left it. This is the time to present my accounts, to show him
how his time and mine have been spent, to make known to him what he is
and what I am; what I have done, and what he has done; what we owe to
each other; all his moral relations, all the undertakings to which he is
pledged, all those to which others have pledged themselves in respect to
him; the stage he has reached in the development of his faculties, the
road that remains to be travelled, the difficulties he will meet, and
the way to overcome them; how I can still help him and how he must
henceforward help himself; in a word, the critical time which he has
reached, the new dangers round about him, and all the valid reasons
which should induce him to keep a close watch upon himself before giving
heed to his growing desires.

Remember that to guide a grown man you must reverse all that you did to
guide the child. Do not hesitate to speak to him of those dangerous
mysteries which you have so carefully concealed from him hitherto. Since
he must become aware of them, let him not learn them from another, nor
from himself, but from you alone; since he must henceforth fight against
them, let him know his enemy, that he may not be taken unawares.

Young people who are found to be aware of these matters, without our
knowing how they obtained their knowledge, have not obtained it with
impunity. This unwise teaching, which can have no honourable object,
stains the imagination of those who receive it if it does nothing worse,
and it inclines them to the vices of their instructors. This is not all;
servants, by this means, ingratiate themselves with a child, gain his
confidence, make him regard his tutor as a gloomy and tiresome person;
and one of the favourite subjects of their secret colloquies is to
slander him. When the pupil has got so far, the master may abandon his
task; he can do no good.

But why does the child choose special confidants? Because of the tyranny
of those who control him. Why should he hide himself from them if he
were not driven to it? Why should he complain if he had nothing to
complain of? Naturally those who control him are his first confidants;
you can see from his eagerness to tell them what he thinks that he feels
he has only half thought till he has told his thoughts to them. You may
be sure that when the child knows you will neither preach nor scold, he
will always tell you everything, and that no one will dare to tell him
anything he must conceal from you, for they will know very well that he
will tell you everything.

What makes me most confident in my method is this: when I follow it out
as closely as possible, I find no situation in the life of my scholar
which does not leave me some pleasing memory of him. Even when he is
carried away by his ardent temperament or when he revolts against the
hand that guides him, when he struggles and is on the point of escaping
from me, I still find his first simplicity in his agitation and his
anger; his heart as pure as his body, he has no more knowledge of
pretence than of vice; reproach and scorn have not made a coward of him;
base fears have never taught him the art of concealment. He has all the
indiscretion of innocence; he is absolutely out-spoken; he does not even
know the use of deceit. Every impulse of his heart is betrayed either by
word or look, and I often know what he is feeling before he is aware of
it himself.

So long as his heart is thus freely opened to me, so long as he delights
to tell me what he feels, I have nothing to fear; the danger is not yet
at hand; but if he becomes more timid, more reserved, if I perceive in
his conversation the first signs of confusion and shame, his instincts
are beginning to develop, he is beginning to connect the idea of evil
with these instincts, there is not a moment to lose, and if I do not
hasten to instruct him, he will learn in spite of me.

Some of my readers, even of those who agree with me, will think that it
is only a question of a conversation with the young man at any time. Oh,
this is not the way to control the human heart. What we say has no
meaning unless the opportunity has been carefully chosen. Before we sow
we must till the ground; the seed of virtue is hard to grow; and a long
period of preparation is required before it will take root. One reason
why sermons have so little effect is that they are offered to everybody
alike, without discrimination or choice. How can any one imagine that
the same sermon could be suitable for so many hearers, with their
different dispositions, so unlike in mind, temper, age, sex, station,
and opinion. Perhaps there are not two among those to whom what is
addressed to all is really suitable; and all our affections are so
transitory that perhaps there are not even two occasions in the life of
any man when the same speech would have the same effect on him. Judge
for yourself whether the time when the eager senses disturb the
understanding and tyrannise over the will, is the time to listen to the
solemn lessons of wisdom. Therefore never reason with young men, even
when they have reached the age of reason, unless you have first prepared
the way. Most lectures miss their mark more through the master's fault
than the disciple's. The pedant and the teacher say much the same; but
the former says it at random, and the latter only when he is sure of its
effect.

As a somnambulist, wandering in his sleep, walks along the edge of a
precipice, over which he would fall if he were awake, so my Emile, in
the sleep of ignorance, escapes the perils which he does not see; were I
to wake him with a start, he might fall. Let us first try to withdraw
him from the edge of the precipice, and then we will awake him to show
him it from a distance.

Reading, solitude, idleness, a soft and sedentary life, intercourse with
women and young people, these are perilous paths for a young man, and
these lead him constantly into danger. I divert his senses by other
objects of sense; I trace another course for his spirits by which I
distract them from the course they would have taken; it is by bodily
exercise and hard work that I check the activity of the imagination,
which was leading him astray. When the arms are hard at work, the
imagination is quiet; when the body is very weary, the passions are not
easily inflamed. The quickest and easiest precaution is to remove him
from immediate danger. At once I take him away from towns, away from
things which might lead him into temptation. But that is not enough; in
what desert, in what wilds, shall he escape from the thoughts which
pursue him? It is not enough to remove dangerous objects; if I fail to
remove the memory of them, if I fail to find a way to detach him from
everything, if I fail to distract him from himself, I might as well have
left him where he was.

Emile has learned a trade, but we do not have recourse to it; he is fond
of farming and understands it, but farming is not enough; the
occupations he is acquainted with degenerate into routine; when he is
engaged in them he is not really occupied; he is thinking of other
things; head and hand are at work on different subjects. He must have
some fresh occupation which has the interest of novelty---an occupation
which keeps him busy, diligent, and hard at work, an occupation which he
may become passionately fond of, one to which he will devote himself
entirely. Now the only one which seems to possess all these
characteristics is the chase. If hunting is ever an innocent pleasure,
if it is ever worthy of a man, now is the time to betake ourselves to
it. Emile is well-fitted to succeed in it. He is strong, skilful,
patient, unwearied. He is sure to take a fancy to this sport; he will
bring to it all the ardour of youth; in it he will lose, at least for a
time, the dangerous inclinations which spring from softness. The chase
hardens the heart a well as the body; we get used to the sight of blood
and cruelty. Diana is represented as the enemy of love; and the allegory
is true to life; the languors of love are born of soft repose, and
tender feelings are stifled by violent exercise. In the woods and
fields, the lover and the sportsman are so diversely affected that they
receive very different impressions. The fresh shade, the arbours, the
pleasant resting-places of the one, to the other are but feeding
grounds, or places where the quarry will hide or turn to bay. Where the
lover hears the flute and the nightingale, the hunter hears the horn and
the hounds; one pictures to himself the nymphs and dryads, the other
sees the horses, the huntsman, and the pack. Take a country walk with
one or other of these men; their different conversation will soon show
you that they behold the earth with other eyes, and that the direction
of their thoughts is as different as their favourite pursuit.

I understand how these tastes may be combined, and that at last men find
time for both. But the passions of youth cannot be divided in this way.
Give the youth a single occupation which he loves, and the rest will
soon be forgotten. Varied desires come with varied knowledge, and the
first pleasures we know are the only ones we desire for long enough. I
would not have the whole of Emile's youth spent in killing creatures,
and I do not even profess to justify this cruel passion; it is enough
for me that it serves to delay a more dangerous passion, so that he may
listen to me calmly when I speak of it, and give me time to describe it
without stimulating it.

There are moments in human life which can never be forgotten. Such is
the time when Emile receives the instruction of which I have spoken; its
influence should endure all his life through. Let us try to engrave it
on his memory so that it may never fade away. It is one of the faults of
our age to rely too much on cold reason, as if men were all mind. By
neglecting the language of expression we have lost the most forcible
mode of speech. The spoken word is always weak, and we speak to the
heart rather through the eyes than the ears. In our attempt to appeal to
reason only, we have reduced our precepts to words, we have not embodied
them in deed. Mere reason is not active; occasionally she restrains,
more rarely she stimulates, but she never does any great thing. Small
minds have a mania for reasoning. Strong souls speak a very different
language, and it is by this language that men are persuaded and driven
to action.

I observe that in modern times men only get a hold over others by force
or self-interest, while the ancients did more by persuasion, by the
affections of the heart; because they did not neglect the language; of
symbolic expression. All agreements were drawn up solemnly, so that they
might be more inviolable; before the reign of force, the gods were the
judges of mankind; in their presence, individuals made their treaties
and alliances, and pledged themselves to perform their promises; the
face of the earth was the book in which the archives were preserved. The
leaves of this book were rocks, trees, piles of stones, made sacred by
these transactions, and regarded with reverence by barbarous men; and
these pages were always open before their eyes. The well of the oath,
the well of the living and seeing one; the ancient oak of Mamre, the
stones of witness, such were the simple but stately monuments of the
sanctity of contracts; none dared to lay a sacrilegious hand on these
monuments, and man's faith was more secure under the warrant of these
dumb witnesses than it is to-day upon all the rigour of the law.

In government the people were over-awed by the pomp and splendour of
royal power. The symbols of greatness, a throne, a sceptre, a purple
robe, a crown, a fillet, these were sacred in their sight. These
symbols, and the respect which they inspired, led them to reverence the
venerable man whom they beheld adorned with them; without soldiers and
without threats, he spoke and was obeyed. {[}Footnote: The Roman
Catholic clergy have very wisely retained these symbols, and certain
republics, such as Venice, have followed their example. Thus the
Venetian government, despite the fallen condition of the state, still
enjoys, under the trappings of its former greatness, all the affection,
all the reverence of the people; and next to the pope in his triple
crown, there is perhaps no king, no potentate, no person in the world so
much respected as the Doge of Venice; he has no power, no authority, but
he is rendered sacred by his pomp, and he wears beneath his ducal
coronet a woman's flowing locks. That ceremony of the Bucentaurius,
which stirs the laughter of fools, stirs the Venetian populace to shed
its life-blood for the maintenance of this tyrannical government.{]} In
our own day men profess to do away with these symbols. What are the
consequences of this contempt? The kingly majesty makes no impression on
all hearts, kings can only gain obedience by the help of troops, and the
respect of their subjects is based only on the fear of punishment. Kings
are spared the trouble of wearing their crowns, and our nobles escape
from the outward signs of their station, but they must have a hundred
thousand men at their command if their orders are to be obeyed. Though
this may seem a finer thing, it is easy to see that in the long run they
will gain nothing.

It is amazing what the ancients accomplished with the aid of eloquence;
but this eloquence did not merely consist in fine speeches carefully
prepared; and it was most effective when the orator said least. The most
startling speeches were expressed not in words but in signs; they were
not uttered but shown. A thing beheld by the eyes kindles the
imagination, stirs the curiosity, and keeps the mind on the alert for
what we are about to say, and often enough the thing tells the whole
story. Thrasybulus and Tarquin cutting off the heads of the poppies,
Alexander placing his seal on the lips of his favourite, Diogenes
marching before Zeno, do not these speak more plainly than if they had
uttered long orations? What flow of words could have expressed the ideas
as clearly? Darius, in the course of the Scythian war, received from the
king of the Scythians a bird, a frog, a mouse, and five arrows. The
ambassador deposited this gift and retired without a word. In our days
he would have been taken for a madman. This terrible speech was
understood, and Darius withdrew to his own country with what speed he
could. Substitute a letter for these symbols and the more threatening it
was the less terror it would inspire; it would have been merely a piece
of bluff, to which Darius would have paid no attention.

What heed the Romans gave to the language of signs! Different ages and
different ranks had their appropriate garments, toga, tunic, patrician
robes, fringes and borders, seats of honour, lictors, rods and axes,
crowns of gold, crowns of leaves, crowns of flowers, ovations, triumphs,
everything had its pomp, its observances, its ceremonial, and all these
spoke to the heart of the citizens. The state regarded it as a matter of
importance that the populace should assemble in one place rather than
another, that they should or should not behold the Capitol, that they
should or should not turn towards the Senate, that this day or that
should be chosen for their deliberations. The accused wore a special
dress, so did the candidates for election; warriors did not boast of
their exploits, they showed their scars. I can fancy one of our orators
at the death of Caesar exhausting all the commonplaces of rhetoric to
give a pathetic description of his wounds, his blood, his dead body;
Anthony was an orator, but he said none of this; he showed the murdered
Caesar. What rhetoric was this!

But this digression, like many others, is drawing me unawares away from
my subject; and my digressions are too frequent to be borne with
patience. I therefore return to the point.

Do not reason coldly with youth. Clothe your reason with a body, if you
would make it felt. Let the mind speak the language of the heart, that
it may be understood. I say again our opinions, not our actions, may be
influenced by cold argument; they set us thinking, not doing; they show
us what we ought to think, not what we ought to do. If this is true of
men, it is all the truer of young people who are still enwrapped in
their senses and cannot think otherwise than they imagine.

Even after the preparations of which I have spoken, I shall take good
care not to go all of a sudden to Emile's room and preach a long and
heavy sermon on the subject in which he is to be instructed. I shall
begin by rousing his imagination; I shall choose the time, place, and
surroundings most favourable to the impression I wish to make; I shall,
so to speak, summon all nature as witness to our conversations; I shall
call upon the eternal God, the Creator of nature, to bear witness to the
truth of what I say. He shall judge between Emile and myself; I will
make the rocks, the woods, the mountains round about us, the monuments
of his promises and mine; eyes, voice, and gesture shall show the
enthusiasm I desire to inspire. Then I will speak and he will listen,
and his emotion will be stirred by my own. The more impressed I am by
the sanctity of my duties, the more sacred he will regard his own. I
will enforce the voice of reason with images and figures, I will not
give him long-winded speeches or cold precepts, but my overflowing
feelings will break their bounds; my reason shall be grave and serious,
but my heart cannot speak too warmly. Then when I have shown him all
that I have done for him, I will show him how he is made for me; he will
see in my tender affection the cause of all my care. How greatly shall I
surprise and disturb him when I change my tone. Instead of shrivelling
up his soul by always talking of his own interests, I shall henceforth
speak of my own; he will be more deeply touched by this. I will kindle
in his young heart all the sentiments of affection, generosity, and
gratitude which I have already called into being, and it will indeed be
sweet to watch their growth. I will press him to my bosom, and weep over
him in my emotion; I will say to him: ``You are my wealth, my child, my
handiwork; my happiness is bound up in yours; if you frustrate my hopes,
you rob me of twenty years of my life, and you bring my grey hairs with
sorrow to the grave.'' This is the way to gain a hearing and to impress
what is said upon the heart and memory of the young man.

Hitherto I have tried to give examples of the way in which a tutor
should instruct his pupil in cases of difficulty. I have tried to do so
in this instance; but after many attempts I have abandoned the task,
convinced that the French language is too artificial to permit in print
the plainness of speech required for the first lessons in certain
subjects.

They say French is more chaste than other languages; for my own part I
think it more obscene; for it seems to me that the purity of a language
does not consist in avoiding coarse expressions but in having none.
Indeed, if we are to avoid them, they must be in our thoughts, and there
is no language in which it is so difficult to speak with purity on every
subject than French. The reader is always quicker to detect than the
author to avoid a gross meaning, and he is shocked and startled by
everything. How can what is heard by impure ears avoid coarseness? On
the other hand, a nation whose morals are pure has fit terms for
everything, and these terms are always right because they are rightly
used. One could not imagine more modest language than that of the Bible,
just because of its plainness of speech. The same things translated into
French would become immodest. What I ought to say to Emile will sound
pure and honourable to him; but to make the same impression in print
would demand a like purity of heart in the reader.

I should even think that reflections on true purity of speech and the
sham delicacy of vice might find a useful place in the conversations as
to morality to which this subject brings us; for when he learns the
language of plain-spoken goodness, he must also learn the language of
decency, and he must know why the two are so different. However this may
be, I maintain that if instead of the empty precepts which are
prematurely dinned into the ears of children, only to be scoffed at when
the time comes when they might prove useful, if instead of this we bide
our time, if we prepare the way for a hearing, if we then show him the
laws of nature in all their truth, if we show him the sanction of these
laws in the physical and moral evils which overtake those who neglect
them, if while we speak to him of this great mystery of generation, we
join to the idea of the pleasure which the Author of nature has given to
this act the idea of the exclusive affection which makes it delightful,
the idea of the duties of faithfulness and modesty which surround it,
and redouble its charm while fulfilling its purpose; if we paint to him
marriage, not only as the sweetest form of society, but also as the most
sacred and inviolable of contracts, if we tell him plainly all the
reasons which lead men to respect this sacred bond, and to pour hatred
and curses upon him who dares to dishonour it; if we give him a true and
terrible picture of the horrors of debauch, of its stupid brutality, of
the downward road by which a first act of misconduct leads from bad to
worse, and at last drags the sinner to his ruin; if, I say, we give him
proofs that on a desire for chastity depends health, strength, courage,
virtue, love itself, and all that is truly good for man---I maintain
that this chastity will be so dear and so desirable in his eyes, that
his mind will be ready to receive our teaching as to the way to preserve
it; for so long as we are chaste we respect chastity; it is only when we
have lost this virtue that we scorn it.

It is not true that the inclination to evil is beyond our control, and
that we cannot overcome it until we have acquired the habit of yielding
to it. Aurelius Victor says that many men were mad enough to purchase a
night with Cleopatra at the price of their life, and this is not
incredible in the madness of passion. But let us suppose the maddest of
men, the man who has his senses least under control; let him see the
preparations for his death, let him realise that he will certainly die
in torment a quarter of an hour later; not only would that man, from
that time forward, become able to resist temptation, he would even find
it easy to do so; the terrible picture with which they are associated
will soon distract his attention from these temptations, and when they
are continually put aside they will cease to recur. The sole cause of
our weakness is the feebleness of our will, and we have always strength
to perform what we strongly desire. ``Volenti nihil difficile!'' Oh! if
only we hated vice as much as we love life, we should abstain as easily
from a pleasant sin as from a deadly poison in a delicious dish.

How is it that you fail to perceive that if all the lessons given to a
young man on this subject have no effect, it is because they are not
adapted to his age, and that at every age reason must be presented in a
shape which will win his affection? Speak seriously to him if required,
but let what you say to him always have a charm which will compel him to
listen. Do not coldly oppose his wishes; do not stifle his imagination,
but direct it lest it should bring forth monsters. Speak to him of love,
of women, of pleasure; let him find in your conversation a charm which
delights his youthful heart; spare no pains to make yourself his
confidant; under this name alone will you really be his master. Then you
need not fear he will find your conversation tedious; he will make you
talk more than you desire.

If I have managed to take all the requisite precautions in accordance
with these maxims, and have said the right things to Emile at the age he
has now reached, I am quite convinced that he will come of his own
accord to the point to which I would lead him, and will eagerly confide
himself to my care. When he sees the dangers by which he is surrounded,
he will say to me with all the warmth of youth, ``Oh, my friend, my
protector, my master! resume the authority you desire to lay aside at
the very time when I most need it; hitherto my weakness has given you
this power. I now place it in your hands of my own free-will, and it
will be all the more sacred in my eyes. Protect me from all the foes
which are attacking me, and above all from the traitors within the
citadel; watch over your work, that it may still be worthy of you. I
mean to obey your laws, I shall ever do so, that is my steadfast
purpose; if I ever disobey you, it will be against my will; make me free
by guarding me against the passions which do me violence; do not let me
become their slave; compel me to be my own master and to obey, not my
senses, but my reason.''

When you have led your pupil so far (and it will be your own fault if
you fail to do so), beware of taking him too readily at his word, lest
your rule should seem too strict to him, and lest he should think he has
a right to escape from it, by accusing you of taking him by surprise.
This is the time for reserve and seriousness; and this attitude will
have all the more effect upon him seeing that it is the first time you
have adopted it towards him.

You will say to him therefore: ``Young man, you readily make promises
which are hard to keep; you must understand what they mean before you
have a right to make them; you do not know how your fellows are drawn by
their passions into the whirlpool of vice masquerading as pleasure. You
are honourable, I know; you will never break your word, but how often
will you repent of having given it? How often will you curse your
friend, when, in order to guard you from the ills which threaten you, he
finds himself compelled to do violence to your heart. Like Ulysses who,
hearing the song of the Sirens, cried aloud to his rowers to unbind him,
you will break your chains at the call of pleasure; you will importune
me with your lamentations, you will reproach me as a tyrant when I have
your welfare most at heart; when I am trying to make you happy, I shall
incur your hatred. Oh, Emile, I can never bear to be hateful in your
eyes; this is too heavy a price to pay even for your happiness. My dear
young man, do you not see that when you undertake to obey me, you compel
me to promise to be your guide, to forget myself in my devotion to you,
to refuse to listen to your murmurs and complaints, to wage unceasing
war against your wishes and my own. Before we either of us undertake
such a task, let us count our resources; take your time, give me time to
consider, and be sure that the slower we are to promise, the more
faithfully will our promises be kept.''

You may be sure that the more difficulty he finds in getting your
promise, the easier you will find it to carry it out. The young man must
learn that he is promising a great deal, and that you are promising
still more. When the time is come, when he has, so to say, signed the
contract, then change your tone, and make your rule as gentle as you
said it would be severe. Say to him, ``My young friend, it is experience
that you lack; but I have taken care that you do not lack reason. You
are ready to see the motives of my conduct in every respect; to do this
you need only wait till you are free from excitement. Always obey me
first, and then ask the reasons for my commands; I am always ready to
give my reasons so soon as you are ready to listen to them, and I shall
never be afraid to make you the judge between us. You promise to follow
my teaching, and I promise only to use your obedience to make you the
happiest of men. For proof of this I have the life you have lived
hitherto. Show me any one of your age who has led as happy a life as
yours, and I promise you nothing more.''

When my authority is firmly established, my first care will be to avoid
the necessity of using it. I shall spare no pains to become more and
more firmly established in his confidence, to make myself the confidant
of his heart and the arbiter of his pleasures. Far from combating his
youthful tastes, I shall consult them that I may be their master; I will
look at things from his point of view that I may be his guide; I will
not seek a remote distant good at the cost of his present happiness. I
would always have him happy always if that may be.

Those who desire to guide young people rightly and to preserve them from
the snares of sense give them a disgust for love, and would willingly
make the very thought of it a crime, as if love were for the old. All
these mistaken lessons have no effect; the heart gives the lie to them.
The young man, guided by a surer instinct, laughs to himself over the
gloomy maxims which he pretends to accept, and only awaits the chance of
disregarding them. All that is contrary to nature. By following the
opposite course I reach the same end more safely. I am not afraid to
encourage in him the tender feeling for which he is so eager, I shall
paint it as the supreme joy of life, as indeed it is; when I picture it
to him, I desire that he shall give himself up to it; by making him feel
the charm which the union of hearts adds to the delights of sense, I
shall inspire him with a disgust for debauchery; I shall make him a
lover and a good man.

How narrow-minded to see nothing in the rising desires of a young heart
but obstacles to the teaching of reason. In my eyes, these are the right
means to make him obedient to that very teaching. Only through passion
can we gain the mastery over passions; their tyranny must be controlled
by their legitimate power, and nature herself must furnish us with the
means to control her.

Emile is not made to live alone, he is a member of society, and must
fulfil his duties as such. He is made to live among his fellow-men and
he must get to know them. He knows mankind in general; he has still to
learn to know individual men. He knows what goes on in the world; he has
now to learn how men live in the world. It is time to show him the front
of that vast stage, of which he already knows the hidden workings. It
will not arouse in him the foolish admiration of a giddy youth, but the
discrimination of an exact and upright spirit. He may no doubt be
deceived by his passions; who is there who yields to his passions
without being led astray by them? At least he will not be deceived by
the passions of other people. If he sees them, he will regard them with
the eye of the wise, and will neither be led away by their example nor
seduced by their prejudices.

As there is a fitting age for the study of the sciences, so there is a
fitting age for the study of the ways of the world. Those who learn
these too soon, follow them throughout life, without choice or
consideration, and although they follow them fairly well they never
really know what they are about. But he who studies the ways of the
world and sees the reason for them, follows them with more insight, and
therefore more exactly and gracefully. Give me a child of twelve who
knows nothing at all; at fifteen I will restore him to you knowing as
much as those who have been under instruction from infancy; with this
difference, that your scholars only know things by heart, while mine
knows how to use his knowledge. In the same way plunge a young man of
twenty into society; under good guidance, in a year's time, he will be
more charming and more truly polite than one brought up in society from
childhood. For the former is able to perceive the reasons for all the
proceedings relating to age, position, and sex, on which the customs of
society depend, and can reduce them to general principles, and apply
them to unforeseen emergencies; while the latter, who is guided solely
by habit, is at a loss when habit fails him.

Young French ladies are all brought up in convents till they are
married. Do they seem to find any difficulty in acquiring the ways which
are so new to them, and is it possible to accuse the ladies of Paris of
awkward and embarrassed manners or of ignorance of the ways of society,
because they have not acquired them in infancy! This is the prejudice of
men of the world, who know nothing of more importance than this trifling
science, and wrongly imagine that you cannot begin to acquire it too
soon.

On the other hand, it is quite true that we must not wait too long. Any
one who has spent the whole of his youth far from the great world is all
his life long awkward, constrained, out of place; his manners will be
heavy and clumsy, no amount of practice will get rid of this, and he
will only make himself more ridiculous by trying to do so. There is a
time for every kind of teaching and we ought to recognise it, and each
has its own dangers to be avoided. At this age there are more dangers
than at any other; but I do not expose my pupil to them without
safeguards.

When my method succeeds completely in attaining one object, and when in
avoiding one difficulty it also provides against another, I then
consider that it is a good method, and that I am on the right track.
This seems to be the case with regard to the expedient suggested by me
in the present case. If I desire to be stern and cold towards my pupil,
I shall lose his confidence, and he will soon conceal himself from me.
If I wish to be easy and complaisant, to shut my eyes, what good does it
do him to be under my care? I only give my authority to his excesses,
and relieve his conscience at the expense of my own. If I introduce him
into society with no object but to teach him, he will learn more than I
want. If I keep him apart from society, what will he have learnt from
me? Everything perhaps, except the one art absolutely necessary to a
civilised man, the art of living among his fellow-men. If I try to
attend to this at a distance, it will be of no avail; he is only
concerned with the present. If I am content to supply him with
amusement, he will acquire habits of luxury and will learn nothing.

We will have none of this. My plan provides for everything. Your heart,
I say to the young man, requires a companion; let us go in search of a
fitting one; perhaps we shall not easily find such a one, true worth is
always rare, but we will be in no hurry, nor will we be easily
discouraged. No doubt there is such a one, and we shall find her at
last, or at least we shall find some one like her. With an end so
attractive to himself, I introduce him into society. What more need I
say? Have I not achieved my purpose?

By describing to him his future mistress, you may imagine whether I
shall gain a hearing, whether I shall succeed in making the qualities he
ought to love pleasing and dear to him, whether I shall sway his
feelings to seek or shun what is good or bad for him. I shall be the
stupidest of men if I fail to make him in love with he knows not whom.
No matter that the person I describe is imaginary, it is enough to
disgust him with those who might have attracted him; it is enough if it
is continually suggesting comparisons which make him prefer his fancy to
the real people he sees; and is not love itself a fancy, a falsehood, an
illusion? We are far more in love with our own fancy than with the
object of it. If we saw the object of our affections as it is, there
would be no such thing as love. When we cease to love, the person we
used to love remains unchanged, but we no longer see with the same eyes;
the magic veil is drawn aside, and love disappears. But when I supply
the object of imagination, I have control over comparisons, and I am
able easily to prevent illusion with regard to realities.

For all that I would not mislead a young man by describing a model of
perfection which could never exist; but I would so choose the faults of
his mistress that they will suit him, that he will be pleased by them,
and they may serve to correct his own. Neither would I lie to him and
affirm that there really is such a person; let him delight in the
portrait, he will soon desire to find the original. From desire to
belief the transition is easy; it is a matter of a little skilful
description, which under more perceptible features will give to this
imaginary object an air of greater reality. I would go so far as to give
her a name; I would say, smiling. Let us call your future mistress
Sophy; Sophy is a name of good omen; if it is not the name of the lady
of your choice at least she will be worthy of the name; we may honour
her with it meanwhile. If after all these details, without affirming or
denying, we excuse ourselves from giving an answer, his suspicions will
become certainty; he will think that his destined bride is purposely
concealed from him, and that he will see her in good time. If once he
has arrived at this conclusion and if the characteristics to be shown to
him have been well chosen, the rest is easy; there will be little risk
in exposing him to the world; protect him from his senses, and his heart
is safe.

But whether or no he personifies the model I have contrived to make so
attractive to him, this model, if well done, will attach him none the
less to everything that resembles itself, and will give him as great a
distaste for all that is unlike it as if Sophy really existed. What a
means to preserve his heart from the dangers to which his appearance
would expose him, to repress his senses by means of his imagination, to
rescue him from the hands of those women who profess to educate young
men, and make them pay so dear for their teaching, and only teach a
young man manners by making him utterly shameless. Sophy is so modest?
What would she think of their advances! Sophy is so simple! How would
she like their airs? They are too far from his thoughts and his
observations to be dangerous.

Every one who deals with the control of children follows the same
prejudices and the same maxima, for their observation is at fault, and
their reflection still more so. A young man is led astray in the first
place neither by temperament nor by the senses, but by popular opinion.
If we were concerned with boys brought up in boarding schools or girls
in convents, I would show that this applies even to them; for the first
lessons they learn from each other, the only lessons that bear fruit,
are those of vice; and it is not nature that corrupts them but example.
But let us leave the boarders in schools and convents to their bad
morals; there is no cure for them. I am dealing only with home training.
Take a young man carefully educated in his father's country house, and
examine him when he reaches Paris and makes his entrance into society;
you will find him thinking clearly about honest matters, and you will
find his will as wholesome as his reason. You will find scorn of vice
and disgust for debauchery; his face will betray his innocent horror at
the very mention of a prostitute. I maintain that no young man could
make up his mind to enter the gloomy abodes of these unfortunates by
himself, if indeed he were aware of their purpose and felt their
necessity.

See the same young man six months later, you will not know him; from his
bold conversation, his fashionable maxims, his easy air, you would take
him for another man, if his jests over his former simplicity and his
shame when any one recalls it did not show that it is he indeed and that
he is ashamed of himself. How greatly has he changed in so short a time!
What has brought about so sudden and complete a change? His physical
development? Would not that have taken place in his father's house, and
certainly he would not have acquired these maxims and this tone at home?
The first charms of sense? On the contrary; those who are beginning to
abandon themselves to these pleasures are timid and anxious, they shun
the light and noise. The first pleasures are always mysterious, modesty
gives them their savour, and modesty conceals them; the first mistress
does not make a man bold but timid. Wholly absorbed in a situation so
novel to him, the young man retires into himself to enjoy it, and
trembles for fear it should escape him. If he is noisy he knows neither
passion nor love; however he may boast, he has not enjoyed.

These changes are merely the result of changed ideas. His heart is the
same, but his opinions have altered. His feelings, which change more
slowly, will at length yield to his opinions and it is then that he is
indeed corrupted. He has scarcely made his entrance into society before
he receives a second education quite unlike the first, which teaches him
to despise what he esteemed, and esteem what he despised; he learns to
consider the teaching of his parents and masters as the jargon of
pedants, and the duties they have instilled into him as a childish
morality, to be scorned now that he is grown up. He thinks he is bound
in honour to change his conduct; he becomes forward without desire, and
he talks foolishly from false shame. He rails against morality before he
has any taste for vice, and prides himself on debauchery without knowing
how to set about it. I shall never forget the confession of a young
officer in the Swiss Guards, who was utterly sick of the noisy pleasures
of his comrades, but dared not refuse to take part in them lest he
should be laughed at. ``I am getting used to it,'' he said, ``as I am
getting used to taking snuff; the taste will come with practice; it will
not do to be a child for ever.''

So a young man when he enters society must be preserved from vanity
rather than from sensibility; he succumbs rather to the tastes of others
than to his own, and self-love is responsible for more libertines than
love.

This being granted, I ask you. Is there any one on earth better armed
than my pupil against all that may attack his morals, his sentiments,
his principles; is there any one more able to resist the flood? What
seduction is there against which he is not forearmed? If his desires
attract him towards women, he fails to find what he seeks, and his
heart, already occupied, holds him back. If he is disturbed and urged
onward by his senses, where will he find satisfaction? His horror of
adultery and debauch keeps him at a distance from prostitutes and
married women, and the disorders of youth may always be traced to one or
other of these. A maiden may be a coquette, but she will not be
shameless, she will not fling herself at the head of a young man who may
marry her if he believes in her virtue; besides she is always under
supervision. Emile, too, will not be left entirely to himself; both of
them will be under the guardianship of fear and shame, the constant
companions of a first passion; they will not proceed at once to
misconduct, and they will not have time to come to it gradually without
hindrance. If he behaves otherwise, he must have taken lessons from his
comrades, he must have learned from them to despise his self-control,
and to imitate their boldness. But there is no one in the whole world so
little given to imitation as Emile. What man is there who is so little
influenced by mockery as one who has no prejudices himself and yields
nothing to the prejudices of others. I have laboured twenty years to arm
him against mockery; they will not make him their dupe in a day; for in
his eyes ridicule is the argument of fools, and nothing makes one less
susceptible to raillery than to be beyond the influence of prejudice.
Instead of jests he must have arguments, and while he is in this frame
of mind, I am not afraid that he will be carried away by young fools;
conscience and truth are on my side. If prejudice is to enter into the
matter at all, an affection of twenty years' standing counts for
something; no one will ever convince him that I have wearied him with
vain lessons; and in a heart so upright and so sensitive the voice of a
tried and trusted friend will soon efface the shouts of twenty
libertines. As it is therefore merely a question of showing him that he
is deceived, that while they pretend to treat him as a man they are
really treating him as a child, I shall choose to be always simple but
serious and plain in my arguments, so that he may feel that I do indeed
treat him as a man. I will say to him, You will see that your welfare,
in which my own is bound up, compels me to speak; I can do nothing else.
But why do these young men want to persuade you? Because they desire to
seduce you; they do not care for you, they take no real interest in you;
their only motive is a secret spite because they see you are better than
they; they want to drag you down to their own level, and they only
reproach you with submitting to control that they may themselves control
you. Do you think you have anything to gain by this? Are they so much
wiser than I, is the affection of a day stronger than mine? To give any
weight to their jests they must give weight to their authority; and by
what experience do they support their maxima above ours? They have only
followed the example of other giddy youths, as they would have you
follow theirs. To escape from the so-called prejudices of their fathers,
they yield to those of their comrades. I cannot see that they are any
the better off; but I see that they lose two things of value---the
affection of their parents, whose advice is that of tenderness and
truth, and the wisdom of experience which teaches us to judge by what we
know; for their fathers have once been young, but the young men have
never been fathers.

But you think they are at least sincere in their foolish precepts. Not
so, dear Emile; they deceive themselves in order to deceive you; they
are not in agreement with themselves; their heart continually revolts,
and their very words often contradict themselves. This man who mocks at
everything good would be in despair if his wife held the same views.
Another extends his indifference to good morals even to his future wife,
or he sinks to such depths of infamy as to be indifferent to his wife's
conduct; but go a step further; speak to him of his mother; is he
willing to be treated as the child of an adulteress and the son of a
woman of bad character, is he ready to assume the name of a family, to
steal the patrimony of the true heir, in a word will he bear being
treated as a bastard? Which of them will permit his daughter to be
dishonoured as he dishonours the daughter of another? There is not one
of them who would not kill you if you adopted in your conduct towards
him all the principles he tries to teach you. Thus they prove their
inconsistency, and we know they do not believe what they say. Here are
reasons, dear Emile; weigh their arguments if they have any, and compare
them with mine. If I wished to have recourse like them to scorn and
mockery, you would see that they lend themselves to ridicule as much or
more than myself. But I am not afraid of serious inquiry. The triumph of
mockers is soon over; truth endures, and their foolish laughter dies
away.

You do not think that Emile, at twenty, can possibly be docile. How
differently we think! I cannot understand how he could be docile at ten,
for what hold have I on him at that age? It took me fifteen years of
careful preparation to secure that hold. I was not educating him, but
preparing him for education. He is now sufficiently educated to be
docile; he recognises the voice of friendship and he knows how to obey
reason. It is true I allow him a show of freedom, but he was never more
completely under control, because he obeys of his own free will. So long
as I could not get the mastery over his will, I retained my control over
his person; I never left him for a moment. Now I sometimes leave him to
himself because I control him continually. When I leave him I embrace
him and I say with confidence: Emile, I trust you to my friend, I leave
you to his honour; he will answer for you.

To corrupt healthy affections which have not been previously depraved,
to efface principles which are directly derived from our own reasoning,
is not the work of a moment. If any change takes place during my
absence, that absence will not be long, he will never be able to conceal
himself from me, so that I shall perceive the danger before any harm
comes of it, and I shall be in time to provide a remedy. As we do not
become depraved all at once, neither do we learn to deceive all at once;
and if ever there was a man unskilled in the art of deception it is
Emile, who has never had any occasion for deceit.

By means of these precautions and others like them, I expect to guard
him so completely against strange sights and vulgar precepts that I
would rather see him in the worst company in Paris than alone in his
room or in a park left to all the restlessness of his age. Whatever we
may do, a young man's worst enemy is himself, and this is an enemy we
cannot avoid. Yet this is an enemy of our own making, for, as I have
said again and again, it is the imagination which stirs the senses.
Desire is not a physical need; it is not true that it is a need at all.
If no lascivious object had met our eye, if no unclean thought had
entered our mind, this so-called need might never have made itself felt,
and we should have remained chaste, without temptation, effort, or
merit. We do not know how the blood of youth is stirred by certain
situations and certain sights, while the youth himself does not
understand the cause of his uneasiness-an uneasiness difficult to subdue
and certain to recur. For my own part, the more I consider this serious
crisis and its causes, immediate and remote, the more convinced I am
that a solitary brought up in some desert, apart from books, teaching,
and women, would die a virgin, however long he lived.

But we are not concerned with a savage of this sort. When we educate a
man among his fellow-men and for social life, we cannot, and indeed we
ought not to, bring him up in this wholesome ignorance, and half
knowledge is worse than none. The memory of things we have observed, the
ideas we have acquired, follow us into retirement and people it, against
our will, with images more seductive than the things themselves, and
these make solitude as fatal to those who bring such ideas with them as
it is wholesome for those who have never left it.

Therefore, watch carefully over the young man; he can protect himself
from all other foes, but it is for you to protect him against himself.
Never leave him night or day, or at least share his room; never let him
go to bed till he is sleepy, and let him rise as soon as he wakes.
Distrust instinct as soon as you cease to rely altogether upon it.
Instinct was good while he acted under its guidance only; now that he is
in the midst of human institutions, instinct is not to be trusted; it
must not be destroyed, it must be controlled, which is perhaps a more
difficult matter. It would be a dangerous matter if instinct taught your
pupil to abuse his senses; if once he acquires this dangerous habit he
is ruined. From that time forward, body and soul will be enervated; he
will carry to the grave the sad effects of this habit, the most fatal
habit which a young man can acquire. If you cannot attain to the mastery
of your passions, dear Emile, I pity you; but I shall not hesitate for a
moment, I will not permit the purposes of nature to be evaded. If you
must be a slave, I prefer to surrender you to a tyrant from whom I may
deliver you; whatever happens, I can free you more easily from the
slavery of women than from yourself.

Up to the age of twenty, the body is still growing and requires all its
strength; till that age continence is the law of nature, and this law is
rarely violated without injury to the constitution. After twenty,
continence is a moral duty; it is an important duty, for it teaches us
to control ourselves, to be masters of our own appetites. But moral
duties have their modifications, their exceptions, their rules. When
human weakness makes an alternative inevitable, of two evils choose the
least; in any case it is better to commit a misdeed than to contract a
vicious habit.

Remember, I am not talking of my pupil now, but of yours. His passions,
to which you have given way, are your master; yield to them openly and
without concealing his victory. If you are able to show him it in its
true light, he will be ashamed rather than proud of it, and you will
secure the right to guide him in his wanderings, at least so as to avoid
precipices. The disciple must do nothing, not even evil, without the
knowledge and consent of his master; it is a hundredfold better that the
tutor should approve of a misdeed than that he should deceive himself or
be deceived by his pupil, and the wrong should be done without his
knowledge. He who thinks he must shut his eyes to one thing, must soon
shut them altogether; the first abuse which is permitted leads to
others, and this chain of consequences only ends in the complete
overthrow of all order and contempt for every law.

There is another mistake which I have already dealt with, a mistake
continually made by narrow-minded persons; they constantly affect the
dignity of a master, and wish to be regarded by their disciples as
perfect. This method is just the contrary of what should be done. How is
it that they fail to perceive that when they try to strengthen their
authority they are really destroying it; that to gain a hearing one must
put oneself in the place of our hearers, and that to speak to the human
heart, one must be a man. All these perfect people neither touch nor
persuade; people always say, ``It is easy for them to fight against
passions they do not feel.'' Show your pupil your own weaknesses if you
want to cure his; let him see in you struggles like his own; let him
learn by your example to master himself and let him not say like other
young men, ``These old people, who are vexed because they are no longer
young, want to treat all young people as if they were old; and they make
a crime of our passions because their own passions are dead.''

Montaigne tells us that he once asked Seigneur de Langey how often, in
his negotiations with Germany, he had got drunk in his king's service. I
would willingly ask the tutor of a certain young man how often he has
entered a house of ill-fame for his pupil's sake. How often? I am wrong.
If the first time has not cured the young libertine of all desire to go
there again, if he does not return penitent and ashamed, if he does not
shed torrents of tears upon your bosom, leave him on the spot; either he
is a monster or you are a fool; you will never do him any good. But let
us have done with these last expedients, which are as distressing as
they are dangerous; our kind of education has no need of them.

What precautions we must take with a young man of good birth before
exposing him to the scandalous manners of our age! These precautions are
painful but necessary; negligence in this matter is the ruin of all our
young men; degeneracy is the result of youthful excesses, and it is
these excesses which make men what they are. Old and base in their
vices, their hearts are shrivelled, because their worn-out bodies were
corrupted at an early age; they have scarcely strength to stir. The
subtlety of their thoughts betrays a mind lacking in substance; they are
incapable of any great or noble feeling, they have neither simplicity
nor vigour; altogether abject and meanly wicked, they are merely
frivolous, deceitful, and false; they have not even courage enough to be
distinguished criminals. Such are the despicable men produced by early
debauchery; if there were but one among them who knew how to be sober
and temperate, to guard his heart, his body, his morals from the
contagion of bad example, at the age of thirty he would crush all these
insects, and would become their master with far less trouble than it
cost him to become master of himself.

However little Emile owes to birth and fortune, he might be this man if
he chose; but he despises such people too much to condescend to make
them his slaves. Let us now watch him in their midst, as he enters into
society, not to claim the first place, but to acquaint himself with it
and to seek a helpmeet worthy of himself.

Whatever his rank or birth, whatever the society into which he is
introduced, his entrance into that society will be simple and
unaffected; God grant he may not be unlucky enough to shine in society;
the qualities which make a good impression at the first glance are not
his, he neither possesses them, nor desires to possess them. He cares
too little for the opinions of other people to value their prejudices,
and he is indifferent whether people esteem him or not until they know
him. His address is neither shy nor conceited, but natural and sincere,
he knows nothing of constraint or concealment, and he is just the same
among a group of people as he is when he is alone. Will this make him
rude, scornful, and careless of others? On the contrary; if he were not
heedless of others when he lived alone, why should he be heedless of
them now that he is living among them? He does not prefer them to
himself in his manners, because he does not prefer them to himself in
his heart, but neither does he show them an indifference which he is far
from feeling; if he is unacquainted with the forms of politeness, he is
not unacquainted with the attentions dictated by humanity. He cannot
bear to see any one suffer; he will not give up his place to another
from mere external politeness, but he will willingly yield it to him out
of kindness if he sees that he is being neglected and that this neglect
hurts him; for it will be less disagreeable to Emile to remain standing
of his own accord than to see another compelled to stand.

Although Emile has no very high opinion of people in general, he does
not show any scorn of them, because he pities them and is sorry for
them. As he cannot give them a taste for what is truly good, he leaves
them the imaginary good with which they are satisfied, lest by robbing
them of this he should leave them worse off than before. So he neither
argues nor contradicts; neither does he flatter nor agree; he states his
opinion without arguing with others, because he loves liberty above all
things, and freedom is one of the fairest gifts of liberty.

He says little, for he is not anxious to attract attention; for the same
reason he only says what is to the point; who could induce him to speak
otherwise? Emile is too well informed to be a chatter-box. A great flow
of words comes either from a pretentious spirit, of which I shall speak
presently, or from the value laid upon trifles which we foolishly think
to be as important in the eyes of others as in our own. He who knows
enough of things to value them at their true worth never says too much;
for he can also judge of the attention bestowed on him and the interest
aroused by what he says. People who know little are usually great
talkers, while men who know much say little. It is plain that an
ignorant person thinks everything he does know important, and he tells
it to everybody. But a well-educated man is not so ready to display his
learning; he would have too much to say, and he sees that there is much
more to be said, so he holds his peace.

Far from disregarding the ways of other people, Emile conforms to them
readily enough; not that he may appear to know all about them, nor yet
to affect the airs of a man of fashion, but on the contrary for fear
lest he should attract attention, and in order to pass unnoticed; he is
most at his ease when no one pays any attention to him.

Although when he makes his entrance into society he knows nothing of its
customs, this does not make him shy or timid; if he keeps in the
background, it is not because he is embarrassed, but because, if you
want to see, you must not be seen; for he scarcely troubles himself at
all about what people think of him, and he is not the least afraid of
ridicule. Hence he is always quiet and self-possessed and is not
troubled with shyness. All he has to do is done as well as he knows how
to do it, whether people are looking at him or not; and as he is always
on the alert to observe other people, he acquires their ways with an
ease impossible to the slaves of other people's opinions. We might say
that he acquires the ways of society just because he cares so little
about them.

But do not make any mistake as to his bearing; it is not to be compared
with that of your young dandies. It is self-possessed, not conceited;
his manners are easy, not haughty; an insolent look is the mark of a
slave, there is nothing affected in independence. I never saw a man of
lofty soul who showed it in his bearing; this affectation is more suited
to vile and frivolous souls, who have no other means of asserting
themselves. I read somewhere that a foreigner appeared one day in the
presence of the famous Marcel, who asked him what country he came from.
``I am an Englishman,'' replied the stranger. ``You are an Englishman!''
replied the dancer, ``You come from that island where the citizens have
a share in the government, and form part of the sovereign power?
{[}Footnote: As if there were citizens who were not part of the city and
had not, as such, a share in sovereign power! But the French, who have
thought fit to usurp the honourable name of citizen which was formerly
the right of the members of the Gallic cities, have degraded the idea
till it has no longer any sort of meaning. A man who recently wrote a
number of silly criticisms on the ''Nouvelle Heloise" added to his
signature the title ``Citizen of Paimboeuf,'' and he thought it a
capital joke.{]} No, sir, that modest bearing, that timid glance, that
hesitating manner, proclaim only a slave adorned with the title of an
elector."

I cannot say whether this saying shows much knowledge of the true
relation between a man's character and his appearance. I have not the
honour of being a dancing master, and I should have thought just the
opposite. I should have said, ``This Englishman is no courtier; I never
heard that courtiers have a timid bearing and a hesitating manner. A man
whose appearance is timid in the presence of a dancer might not be timid
in the House of Commons.'' Surely this M. Marcel must take his
fellow-countrymen for so many Romans.

He who loves desires to be loved, Emile loves his fellows and desires to
please them. Even more does he wish to please the women; his age, his
character, the object he has in view, all increase this desire. I say
his character, for this has a great effect; men of good character are
those who really adore women. They have not the mocking jargon of
gallantry like the rest, but their eagerness is more genuinely tender,
because it comes from the heart. In the presence of a young woman, I
could pick out a young man of character and self-control from among a
hundred thousand libertines. Consider what Emile must be, with all the
eagerness of early youth and so many reasons for resistance! For in the
presence of women I think he will sometimes be shy and timid; but this
shyness will certainly not be displeasing, and the least foolish of them
will only too often find a way to enjoy it and augment it. Moreover, his
eagerness will take a different shape according to those he has to do
with. He will be more modest and respectful to married women, more eager
and tender towards young girls. He never loses sight of his purpose, and
it is always those who most recall it to him who receive the greater
share of his attentions.

No one could be more attentive to every consideration based upon the
laws of nature, and even on the laws of good society; but the former are
always preferred before the latter, and Emile will show more respect to
an elderly person in private life than to a young magistrate of his own
age. As he is generally one of the youngest in the company, he will
always be one of the most modest, not from the vanity which apes
humility, but from a natural feeling founded upon reason. He will not
have the effrontery of the young fop, who speaks louder than the wise
and interrupts the old in order to amuse the company. He will never give
any cause for the reply given to Louis XV by an old gentleman who was
asked whether he preferred this century or the last: ``Sire, I spent my
youth in reverence towards the old; I find myself compelled to spend my
old age in reverence towards the young.''

His heart is tender and sensitive, but he cares nothing for the weight
of popular opinion, though he loves to give pleasure to others; so he
will care little to be thought a person of importance. Hence he will be
affectionate rather than polite, he will never be pompous or affected,
and he will be always more touched by a caress than by much praise. For
the same reasons he will never be careless of his manners or his
clothes; perhaps he will be rather particular about his dress, not that
he may show himself a man of taste, but to make his appearance more
pleasing; he will never require a gilt frame, and he will never spoil
his style by a display of wealth.

All this demands, as you see, no stock of precepts from me; it is all
the result of his early education. People make a great mystery of the
ways of society, as if, at the age when these ways are acquired, we did
not take to them quite naturally, and as if the first laws of politeness
were not to be found in a kindly heart. True politeness consists in
showing our goodwill towards men; it shows its presence without any
difficulty; those only who lack this goodwill are compelled to reduce
the outward signs of it to an art.

"The worst effect of artificial politeness is that it teaches us how to
dispense with the virtues it imitates. If our education teaches us
kindness and humanity, we shall be polite, or we shall have no need of
politeness.

"If we have not those qualities which display themselves gracefully we
shall have those which proclaim the honest man and the citizen; we shall
have no need for falsehood.

"Instead of seeking to please by artificiality, it will suffice that we
are kindly; instead of flattering the weaknesses of others by falsehood,
it will suffice to tolerate them.

``Those with whom we have to do will neither be puffed up nor corrupted
by such intercourse; they will only be grateful and will be informed by
it.'' {[}Footnote: Considerations sur les moeurs de ce siecle, par M.
Duclos.{]}

It seems to me that if any education is calculated to produce the sort
of politeness required by M. Duclos in this passage, it is the education
I have already described.

Yet I admit that with such different teaching Emile will not be just
like everybody else, and heaven preserve him from such a fate! But where
he is unlike other people, he will neither cause annoyance nor will he
be absurd; the difference will be perceptible but not unpleasant. Emile
will be, if you like, an agreeable foreigner. At first his peculiarities
will be excused with the phrase, ``He will learn.'' After a time people
will get used to his ways, and seeing that he does not change they will
still make excuses for him and say, ``He is made that way.''

He will not be feted as a charming man, but every one will like him
without knowing why; no one will praise his intellect, but every one
will be ready to make him the judge between men of intellect; his own
intelligence will be clear and limited, his mind will be accurate, and
his judgment sane. As he never runs after new ideas, he cannot pride
himself on his wit. I have convinced him that all wholesome ideas, ideas
which are really useful to mankind, were among the earliest known, that
in all times they have formed the true bonds of society, and that there
is nothing left for ambitious minds but to seek distinction for
themselves by means of ideas which are injurious and fatal to mankind.
This way of winning admiration scarcely appeals to him; he knows how he
ought to seek his own happiness in life, and how he can contribute to
the happiness of others. The sphere of his knowledge is restricted to
what is profitable. His path is narrow and clearly defined; as he has no
temptation to leave it, he is lost in the crowd; he will neither
distinguish himself nor will he lose his way. Emile is a man of common
sense and he has no desire to be anything more; you may try in vain to
insult him by applying this phrase to him; he will always consider it a
title of honour.

Although from his wish to please he is no longer wholly indifferent to
the opinion of others, he only considers that opinion so far as he
himself is directly concerned, without troubling himself about arbitrary
values, which are subject to no law but that of fashion or
conventionality. He will have pride enough to wish to do well in
everything that he undertakes, and even to wish to do it better than
others; he will want to be the swiftest runner, the strongest wrestler,
the cleverest workman, the readiest in games of skill; but he will not
seek advantages which are not in themselves clear gain, but need to be
supported by the opinion of others, such as to be thought wittier than
another, a better speaker, more learned, etc.; still less will he
trouble himself with those which have nothing to do with the man
himself, such as higher birth, a greater reputation for wealth, credit,
or public estimation, or the impression created by a showy exterior.

As he loves his fellows because they are like himself, he will prefer
him who is most like himself, because he will feel that he is good; and
as he will judge of this resemblance by similarity of taste in morals,
in all that belongs to a good character, he will be delighted to win
approval. He will not say to himself in so many words, ``I am delighted
to gain approval,'' but ``I am delighted because they say I have done
right; I am delighted because the men who honour me are worthy of
honour; while they judge so wisely, it is a fine thing to win their
respect.''

As he studies men in their conduct in society, just as he formerly
studied them through their passions in history, he will often have
occasion to consider what it is that pleases or offends the human heart.
He is now busy with the philosophy of the principles of taste, and this
is the most suitable subject for his present study.

The further we seek our definitions of taste, the further we go astray;
taste is merely the power of judging what is pleasing or displeasing to
most people. Go beyond this, and you cannot say what taste is. It does
not follow that the men of taste are in the majority; for though the
majority judges wisely with regard to each individual thing, there are
few men who follow the judgment of the majority in everything; and
though the most general agreement in taste constitutes good taste, there
are few men of good taste just as there are few beautiful people,
although beauty consists in the sum of the most usual features.

It must be observed that we are not here concerned with what we like
because it is serviceable, or hate because it is harmful to us. Taste
deals only with things that are indifferent to us, or which affect at
most our amusements, not those which relate to our needs; taste is not
required to judge of these, appetite only is sufficient. It is this
which makes mere decisions of taste so difficult and as it seems so
arbitrary; for beyond the instinct they follow there appears to be no
reason whatever for them. We must also make a distinction between the
laws of good taste in morals and its laws in physical matters. In the
latter the laws of taste appear to be absolutely inexplicable. But it
must be observed that there is a moral element in everything which
involves imitation.{[}Footnote: This is demonstrated in an ``Essay on
the Origin of Languages'' which will be found in my collected works.{]}
This is the explanation of beauties which seem to be physical, but are
not so in reality. I may add that taste has local rules which make it
dependent in many respects on the country we are in, its manners,
government, institutions; it has other rules which depend upon age, sex,
and character, and it is in this sense that we must not dispute over
matters of taste.

Taste is natural to men; but all do not possess it in the same degree,
it is not developed to the same extent in every one; and in every one it
is liable to be modified by a variety of causes. Such taste as we may
possess depends on our native sensibility; its cultivation and its form
depend upon the society in which we have lived. In the first place we
must live in societies of many different kinds, so as to compare much.
In the next place, there must be societies for amusement and idleness,
for in business relations, interest, not pleasure, is our rule. Lastly,
there must be societies in which people are fairly equal, where the
tyranny of public opinion may be moderate, where pleasure rather than
vanity is queen; where this is not so, fashion stifles taste, and we
seek what gives distinction rather than delight.

In the latter case it is no longer true that good taste is the taste of
the majority. Why is this? Because the purpose is different. Then the
crowd has no longer any opinion of its own, it only follows the judgment
of those who are supposed to know more about it; its approval is
bestowed not on what is good, but on what they have already approved. At
any time let every man have his own opinion, and what is most pleasing
in itself will always secure most votes.

Every beauty that is to be found in the works of man is imitated. All
the true models of taste are to be found in nature. The further we get
from the master, the worse are our pictures. Then it is that we find our
models in what we ourselves like, and the beauty of fancy, subject to
caprice and to authority, is nothing but what is pleasing to our
leaders.

Those leaders are the artists, the wealthy, and the great, and they
themselves follow the lead of self-interest or pride. Some to display
their wealth, others to profit by it, they seek eagerly for new ways of
spending it. This is how luxury acquires its power and makes us love
what is rare and costly; this so-called beauty consists, not in
following nature, but in disobeying her. Hence luxury and bad taste are
inseparable. Wherever taste is lavish, it is bad.

Taste, good or bad, takes its shape especially in the intercourse
between the two sexes; the cultivation of taste is a necessary
consequence of this form of society. But when enjoyment is easily
obtained, and the desire to please becomes lukewarm, taste must
degenerate; and this is, in my opinion, one of the best reasons why good
taste implies good morals.

Consult the women's opinions in bodily matters, in all that concerns the
senses; consult the men in matters of morality and all that concerns the
understanding. When women are what they ought to be, they will keep to
what they can understand, and their judgment will be right; but since
they have set themselves up as judges of literature, since they have
begun to criticise books and to make them with might and main, they are
altogether astray. Authors who take the advice of blue-stockings will
always be ill-advised; gallants who consult them about their clothes
will always be absurdly dressed. I shall presently have an opportunity
of speaking of the real talents of the female sex, the way to cultivate
these talents, and the matters in regard to which their decisions should
receive attention.

These are the elementary considerations which I shall lay down as
principles when I discuss with Emile this matter which is by no means
indifferent to him in his present inquiries. And to whom should it be a
matter of indifference? To know what people may find pleasant or
unpleasant is not only necessary to any one who requires their help, it
is still more necessary to any one who would help them; you must please
them if you would do them service; and the art of writing is no idle
pursuit if it is used to make men hear the truth.

If in order to cultivate my pupil's taste, I were compelled to choose
between a country where this form of culture has not yet arisen and
those in which it has already degenerated, I would progress backwards; I
would begin his survey with the latter and end with the former. My
reason for this choice is, that taste becomes corrupted through
excessive delicacy, which makes it sensitive to things which most men do
not perceive; this delicacy leads to a spirit of discussion, for the
more subtle is our discrimination of things the more things there are
for us. This subtlety increases the delicacy and decreases the
uniformity of our touch. So there are as many tastes as there are
people. In disputes as to our preferences, philosophy and knowledge are
enlarged, and thus we learn to think. It is only men accustomed to
plenty of society who are capable of very delicate observations, for
these observations do not occur to us till the last, and people who are
unused to all sorts of society exhaust their attention in the
consideration of the more conspicuous features. There is perhaps no
civilised place upon earth where the common taste is so bad as in Paris.
Yet it is in this capital that good taste is cultivated, and it seems
that few books make any impression in Europe whose authors have not
studied in Paris. Those who think it is enough to read our books are
mistaken; there is more to be learnt from the conversation of authors
than from their books; and it is not from the authors that we learn
most. It is the spirit of social life which develops a thinking mind,
and carries the eye as far as it can reach. If you have a spark of
genius, go and spend a year in Paris; you will soon be all that you are
capable of becoming, or you will never be good for anything at all.

One may learn to think in places where bad taste rules supreme; but we
must not think like those whose taste is bad, and it is very difficult
to avoid this if we spend much time among them. We must use their
efforts to perfect the machinery of judgment, but we must be careful not
to make the same use of it. I shall take care not to polish Emile's
judgment so far as to transform it, and when he has acquired discernment
enough to feel and compare the varied tastes of men, I shall lead him to
fix his own taste upon simpler matters.

I will go still further in order to keep his taste pure and wholesome.
In the tumult of dissipation I shall find opportunities for useful
conversation with him; and while these conversations are always about
things in which he takes a delight, I shall take care to make them as
amusing as they are instructive. Now is the time to read pleasant books;
now is the time to teach him to analyse speech and to appreciate all the
beauties of eloquence and diction. It is a small matter to learn
languages, they are less useful than people think; but the study of
languages leads us on to that of grammar in general. We must learn Latin
if we would have a thorough knowledge of French; these two languages
must be studied and compared if we would understand the rules of the art
of speaking.

There is, moreover, a certain simplicity of taste which goes straight to
the heart; and this is only to be found in the classics. In oratory,
poetry, and every kind of literature, Emile will find the classical
authors as he found them in history, full of matter and sober in their
judgment. The authors of our own time, on the contrary, say little and
talk much. To take their judgment as our constant law is not the way to
form our own judgment. These differences of taste make themselves felt
in all that is left of classical times and even on their tombs. Our
monuments are covered with praises, theirs recorded facts.

``Sta, viator; heroem calcas.''

If I had found this epitaph on an ancient monument, I should at once
have guessed it was modern; for there is nothing so common among us as
heroes, but among the ancients they were rare. Instead of saying a man
was a hero, they would have said what he had done to gain that name.
With the epitaph of this hero compare that of the effeminate
Sardanapalus---

``Tarsus and Anchiales I built in a day, and now I am dead.''

Which do you think says most? Our inflated monumental style is only fit
to trumpet forth the praises of pygmies. The ancients showed men as they
were, and it was plain that they were men indeed. Xenophon did honour to
the memory of some warriors who were slain by treason during the retreat
of the Ten Thousand. ``They died,'' said he, ``without stain in war and
in love.'' That is all, but think how full was the heart of the author
of this short and simple elegy. Woe to him who fails to perceive its
charm. The following words were engraved on a tomb at Thermopylae---

``Go, Traveller, tell Sparta that here we fell in obedience to her
laws.''

It is pretty clear that this was not the work of the Academy of\\
Inscriptions.\\

If I am not mistaken, the attention of my pupil, who sets so small value
upon words, will be directed in the first place to these differences,
and they will affect his choice in his reading. He will be carried away
by the manly eloquence of Demosthenes, and will say, ``This is an
orator;'' but when he reads Cicero, he will say, ``This is a lawyer.''

Speaking generally Emile will have more taste for the books of the
ancients than for our own, just because they were the first, and
therefore the ancients are nearer to nature and their genius is more
distinct. Whatever La Motte and the Abbe Terrasson may say, there is no
real advance in human reason, for what we gain in one direction we lose
in another; for all minds start from the same point, and as the time
spent in learning what others have thought is so much time lost in
learning to think for ourselves, we have more acquired knowledge and
less vigour of mind. Our minds like our arms are accustomed to use tools
for everything, and to do nothing for themselves. Fontenelle used to say
that all these disputes as to the ancients and the moderns came to
this---Were the trees in former times taller than they are now. If
agriculture had changed, it would be worth our while to ask this
question.

After I have led Emile to the sources of pure literature, I will also
show him the channels into the reservoirs of modern compilers; journals,
translations, dictionaries, he shall cast a glance at them all, and then
leave them for ever. To amuse him he shall hear the chatter of the
academies; I will draw his attention to the fast that every member of
them is worth more by himself than he is as a member of the society; he
will then draw his own conclusions as to the utility of these fine
institutions.

I take him to the theatre to study taste, not morals; for in the theatre
above all taste is revealed to those who can think. Lay aside precepts
and morality, I should say; this is not the place to study them. The
stage is not made for truth; its object is to flatter and amuse: there
is no place where one can learn so completely the art of pleasing and of
interesting the human heart. The study of plays leads to the study of
poetry; both have the same end in view. If he has the least glimmering
of taste for poetry, how eagerly will he study the languages of the
poets, Greek, Latin, and Italian! These studies will afford him
unlimited amusement and will be none the less valuable; they will be a
delight to him at an age and in circumstances when the heart finds so
great a charm in every kind of beauty which affects it. Picture to
yourself on the one hand Emile, on the other some young rascal from
college, reading the fourth book of the Aeneid, or Tibollus, or the
Banquet of Plato: what a difference between them! What stirs the heart
of Emile to its depths, makes not the least impression on the other! Oh,
good youth, stay, make a pause in your reading, you are too deeply
moved; I would have you find pleasure in the language of love, but I
would not have you carried away by it; be a wise man, but be a good man
too. If you are only one of these, you are nothing. After this let him
win fame or not in dead languages, in literature, in poetry, I care
little. He will be none the worse if he knows nothing of them, and his
education is not concerned with these mere words.

My main object in teaching him to feel and love beauty of every kind is
to fix his affections and his taste on these, to prevent the corruption
of his natural appetites, lest he should have to seek some day in the
midst of his wealth for the means of happiness which should be found
close at hand. I have said elsewhere that taste is only the art of being
a connoisseur in matters of little importance, and this is quite true;
but since the charm of life depends on a tissue of these matters of
little importance, such efforts are no small thing; through their means
we learn how to fill our life with the good things within our reach,
with as much truth as they may hold for us. I do not refer to the
morally good which depends on a good disposition of the heart, but only
to that which depends on the body, on real delight, apart from the
prejudices of public opinion.

The better to unfold my idea, allow me for a moment to leave Emile,
whose pure and wholesome heart cannot be taken as a rule for others, and
to seek in my own memory for an illustration better suited to the reader
and more in accordance with his own manners.

There are professions which seem to change a man's nature, to recast,
either for better or worse, the men who adopt them. A coward becomes a
brave man in the regiment of Navarre. It is not only in the army that
esprit de corps is acquired, and its effects are not always for good. I
have thought again and again with terror that if I had the misfortune to
fill a certain post I am thinking of in a certain country, before
to-morrow I should certainly be a tyrant, an extortioner, a destroyer of
the people, harmful to my king, and a professed enemy of mankind, a foe
to justice and every kind of virtue.

In the same way, if I were rich, I should have done all that is required
to gain riches; I should therefore be insolent and degraded, sensitive
and feeling only on my own behalf, harsh and pitiless to all besides, a
scornful spectator of the sufferings of the lower classes; for that is
what I should call the poor, to make people forget that I was once poor
myself. Lastly I should make my fortune a means to my own pleasures with
which I should be wholly occupied; and so far I should be just like
other people.

But in one respect I should be very unlike them; I should be sensual and
voluptuous rather than proud and vain, and I should give myself up to
the luxury of comfort rather than to that of ostentation. I should even
be somewhat ashamed to make too great a show of my wealth, and if I
overwhelmed the envious with my pomp I should always fancy I heard him
saying, ``Here is a rascal who is greatly afraid lest we should take him
for anything but what he is.''

In the vast profusion of good things upon this earth I should seek what
I like best, and what I can best appropriate to myself.

To this end, the first use I should make of my wealth would be to
purchase leisure and freedom, to which I would add health, if it were to
be purchased; but health can only be bought by temperance, and as there
is no real pleasure without health, I should be temperate from sensual
motives.

I should also keep as close as possible to nature, to gratify the senses
given me by nature, being quite convinced that, the greater her share in
my pleasures, the more real I shall find them. In the choice of models
for imitation I shall always choose nature as my pattern; in my
appetites I will give her the preference; in my tastes she shall always
be consulted; in my food I will always choose what most owes its charm
to her, and what has passed through the fewest possible hands on its way
to table. I will be on my guard against fraudulent shams; I will go out
to meet pleasure. No cook shall grow rich on my gross and foolish
greediness; he shall not poison me with fish which cost its weight in
gold, my table shall not be decked with fetid splendour or putrid flesh
from far-off lands. I will take any amount of trouble to gratify my
sensibility, since this trouble has a pleasure of its own, a pleasure
more than we expect. If I wished to taste a food from the ends of the
earth, I would go, like Apicius, in search of it, rather than send for
it; for the daintiest dishes always lack a charm which cannot be brought
along with them, a flavour which no cook can give them---the air of the
country where they are produced.

For the same reason I would not follow the example of those who are
never well off where they are, but are always setting the seasons at
nought, and confusing countries and their seasons; those who seek winter
in summer and summer in winter, and go to Italy to be cold and to the
north to be warm, do not consider that when they think they are escaping
from the severity of the seasons, they are going to meet that severity
in places where people are not prepared for it. I shall stay in one
place, or I shall adopt just the opposite course; I should like to get
all possible enjoyment out of one season to discover what is peculiar to
any given country. I would have a variety of pleasures, and habits quite
unlike one another, but each according to nature; I would spend the
summer at Naples and the winter in St. Petersburg; sometimes I would
breathe the soft zephyr lying in the cool grottoes of Tarentum, and
again I would enjoy the illuminations of an ice palace, breathless and
wearied with the pleasures of the dance.

In the service of my table and the adornment of my dwelling I would
imitate in the simplest ornaments the variety of the seasons, and draw
from each its charm without anticipating its successor. There is no
taste but only difficulty to be found in thus disturbing the order of
nature; to snatch from her unwilling gifts, which she yields
regretfully, with her curse upon them; gifts which have neither strength
nor flavour, which can neither nourish the body nor tickle the palate.
Nothing is more insipid than forced fruits. A wealthy man in Paris, with
all his stoves and hot-houses, only succeeds in getting all the year
round poor fruit and poor vegetables for his table at a very high price.
If I had cherries in frost, and golden melons in the depths of winter,
what pleasure should I find in them when my palate did not need moisture
or refreshment. Would the heavy chestnut be very pleasant in the heat of
the dog-days; should I prefer to have it hot from the stove, rather than
the gooseberry, the strawberry, the refreshing fruits which the earth
takes care to provide for me. A mantelpiece covered in January with
forced vegetation, with pale and scentless flowers, is not winter
adorned, but spring robbed of its beauty; we deprive ourselves of the
pleasure of seeking the first violet in the woods, of noting the
earliest buds, and exclaiming in a rapture of delight, ``Mortals, you
are not forsaken, nature is living still.''

To be well served I would have few servants; this has been said before,
but it is worth saying again. A tradesman gets more real service from
his one man than a duke from the ten gentlemen round about him. It has
often struck me when I am sitting at table with my glass beside me that
I can drink whenever I please; whereas, if I were dining in state,
twenty men would have to call for ``Wine'' before I could quench my
thirst. You may be sure that whatever is done for you by other people is
ill done. I would not send to the shops, I would go myself; I would go
so that my servants should not make their own terms with the
shopkeepers, and to get a better choice and cheaper prices; I would go
for the sake of pleasant exercise and to get a glimpse of what was going
on out of doors; this is amusing and sometimes instructive; lastly I
would go for the sake of the walk; there is always something in that. A
sedentary life is the source of tedium; when we walk a good deal we are
never dull. A porter and footmen are poor interpreters, I should never
wish to have such people between the world and myself, nor would I
travel with all the fuss of a coach, as if I were afraid people would
speak to me. Shanks' mare is always ready; if she is tired or ill, her
owner is the first to know it; he need not be afraid of being kept at
home while his coachman is on the spree; on the road he will not have to
submit to all sorts of delays, nor will he be consumed with impatience,
nor compelled to stay in one place a moment longer than he chooses.
Lastly, since no one serves us so well as we serve ourselves, had we the
power of Alexander and the wealth of Croesus we should accept no
services from others, except those we cannot perform for ourselves.

I would not live in a palace; for even in a palace I should only occupy
one room; every room which is common property belongs to nobody, and the
rooms of each of my servants would be as strange to me as my
neighbour's. The Orientals, although very voluptuous, are lodged in
plain and simply furnished dwellings. They consider life as a journey,
and their house as an inn. This reason scarcely appeals to us rich
people who propose to live for ever; but I should find another reason
which would have the same effect. It would seem to me that if I settled
myself in one place in the midst of such splendour, I should banish
myself from every other place, and imprison myself, so to speak, in my
palace. The world is a palace fair enough for any one; and is not
everything at the disposal of the rich man when he seeks enjoyment?
``Ubi bene, ibi patria,'' that is his motto; his home is anywhere where
money will carry him, his country is anywhere where there is room for
his strong-box, as Philip considered as his own any place where a mule
laden with silver could enter. {[}Footnote: A stranger, splendidly clad,
was asked in Athens what country he belonged to. ``I am one of the
rich,'' was his answer; and a very good answer in my opinion.{]} Why
then should we shut ourselves up within walls and gates as if we never
meant to leave them? If pestilence, war, or rebellion drive me from one
place, I go to another, and I find my hotel there before me. Why should
I build a mansion for myself when the world is already at my disposal?
Why should I be in such a hurry to live, to bring from afar delights
which I can find on the spot? It is impossible to make a pleasant life
for oneself when one is always at war with oneself. Thus Empedocles
reproached the men of Agrigentum with heaping up pleasures as if they
had but one day to live, and building as if they would live for ever.

And what use have I for so large a dwelling, as I have so few people to
live in it, and still fewer goods to fill it? My furniture would be as
simple as my tastes; I would have neither picture-gallery nor library,
especially if I was fond of reading and knew something about pictures. I
should then know that such collections are never complete, and that the
lack of that which is wanting causes more annoyance than if one had
nothing at all. In this respect abundance is the cause of want, as every
collector knows to his cost. If you are an expert, do not make a
collection; if you know how to use your cabinets, you will not have any
to show.

Gambling is no sport for the rich, it is the resource of those who have
nothing to do; I shall be so busy with my pleasures that I shall have no
time to waste. I am poor and lonely and I never play, unless it is a
game of chess now and then, and that is more than enough. If I were rich
I would play even less, and for very low stakes, so that I should not be
disappointed myself, nor see the disappointment of others. The wealthy
man has no motive for play, and the love of play will not degenerate
into the passion for gambling unless the disposition is evil. The rich
man is always more keenly aware of his losses than his gains, and as in
games where the stakes are not high the winnings are generally exhausted
in the long run, he will usually lose more than he gains, so that if we
reason rightly we shall scarcely take a great fancy to games where the
odds are against us. He who flatters his vanity so far as to believe
that Fortune favours him can seek her favour in more exciting ways; and
her favours are just as clearly shown when the stakes are low as when
they are high. The taste for play, the result of greed and dullness,
only lays hold of empty hearts and heads; and I think I should have
enough feeling and knowledge to dispense with its help. Thinkers are
seldom gamblers; gambling interrupts the habit of thought and turns it
towards barren combinations; thus one good result, perhaps the only good
result of the taste for science, is that it deadens to some extent this
vulgar passion; people will prefer to try to discover the uses of play
rather than to devote themselves to it. I should argue with the gamblers
against gambling, and I should find more delight in scoffing at their
losses than in winning their money.

I should be the same in private life as in my social intercourse. I
should wish my fortune to bring comfort in its train, and never to make
people conscious of inequalities of wealth. Showy dress is inconvenient
in many ways. To preserve as much freedom as possible among other men, I
should like to be dressed in such a way that I should not seem out of
place among all classes, and should not attract attention in any; so
that without affectation or change I might mingle with the crowd at the
inn or with the nobility at the Palais Royal. In this way I should be
more than ever my own master, and should be free to enjoy the pleasures
of all sorts and conditions of men. There are women, so they say, whose
doors are closed to embroidered cuffs, women who will only receive
guests who wear lace ruffles; I should spend my days elsewhere; though
if these women were young and pretty I might sometimes put on lace
ruffles to spend an evening or so in their company.

Mutual affection, similarity of tastes, suitability of character; these
are the only bonds between my companions and myself; among them I would
be a man, not a person of wealth; the charm of their society should
never be embittered by self-seeking. If my wealth had not robbed me of
all humanity, I would scatter my benefits and my services broadcast, but
I should want companions about me, not courtiers, friends, not proteges;
I should wish my friends to regard me as their host, not their patron.
Independence and equality would leave to my relations with my friends
the sincerity of goodwill; while duty and self-seeking would have no
place among us, and we should know no law but that of pleasure and
friendship.

Neither a friend nor a mistress can be bought. Women may be got for
money, but that road will never lead to love. Love is not only not for
sale; money strikes it dead. If a man pays, were he indeed the most
lovable of men, the mere fact of payment would prevent any lasting
affection. He will soon be paying for some one else, or rather some one
else will get his money; and in this double connection based on
self-seeking and debauchery, without love, honour, or true pleasure, the
woman is grasping, faithless, and unhappy, and she is treated by the
wretch to whom she gives her money as she treats the fool who gives his
money to her; she has no love for either. It would be sweet to lie
generous towards one we love, if that did not make a bargain of love. I
know only one way of gratifying this desire with the woman one loves
without embittering love; it is to bestow our all upon her and to live
at her expense. It remains to be seen whether there is any woman with
regard to whom such conduct would not be unwise.

He who said, ``Lais is mine, but I am not hers,'' was talking nonsense.
Possession which is not mutual is nothing at all; at most it is the
possession of the sex not of the individual. But where there is no
morality in love, why make such ado about the rest? Nothing is so easy
to find. A muleteer is in this respect as near to happiness as a
millionaire.

Oh, if we could thus trace out the unreasonableness of vice, how often
should we find that, when it has attained its object, it discovers it is
not what it seemed! Why is there this cruel haste to corrupt innocence,
to make, a victim of a young creature whom we ought to protect, one who
is dragged by this first false step into a gulf of misery from which
only death can release her? Brutality, vanity, folly, error, and nothing
more. This pleasure itself is unnatural; it rests on popular opinion,
and popular opinion at its worst, since it depends on scorn of self. He
who knows he is the basest of men fears comparison with others, and
would be the first that he may be less hateful. See if those who are
most greedy in pursuit of such fancied pleasures are ever attractive
young men---men worthy of pleasing, men who might have some excuse if
they were hard to please. Not so; any one with good looks, merit, and
feeling has little fear of his mistress' experience; with well-placed
confidence he says to her, ``You know what pleasure is, what is that to
me? my heart assures me that this is not so.''

But an aged satyr, worn out with debauchery, with no charm, no
consideration, no thought for any but himself, with no shred of honour,
incapable and unworthy of finding favour in the eyes of any woman who
knows anything of men deserving of love, expects to make up for all this
with an innocent girl by trading on her inexperience and stirring her
emotions for the first time. His last hope is to find favour as a
novelty; no doubt this is the secret motive of this desire; but he is
mistaken, the horror he excites is just as natural as the desires he
wishes to arouse. He is also mistaken in his foolish attempt; that very
nature takes care to assert her rights; every girl who sells herself is
no longer a maid; she has given herself to the man of her choice, and
she is making the very comparison he dreads. The pleasure purchased is
imaginary, but none the less hateful.

For my own part, however riches may change me, there is one matter in
which I shall never change. If I have neither morals nor virtue, I shall
not be wholly without taste, without sense, without delicacy; and this
will prevent me from spending my fortune in the pursuit of empty dreams,
from wasting my money and my strength in teaching children to betray me
and mock at me. If I were young, I would seek the pleasures of youth;
and as I would have them at their best I would not seek them in the
guise of a rich man. If I were at my present age, it would be another
matter; I would wisely confine myself to the pleasures of my age; I
would form tastes which I could enjoy, and I would stifle those which
could only cause suffering. I would not go and offer my grey beard to
the scornful jests of young girls; I could never bear to sicken them
with my disgusting caresses, to furnish them at my expense with the most
absurd stories, to imagine them describing the vile pleasures of the old
ape, so as to avenge themselves for what they had endured. But if habits
unresisted had changed my former desires into needs, I would perhaps
satisfy those needs, but with shame and blushes. I would distinguish
between passion and necessity, I would find a suitable mistress and
would keep to her. I would not make a business of my weakness, and above
all I would only have one person aware of it. Life has other pleasures
when these fail us; by hastening in vain after those that fly us, we
deprive ourselves of those that remain. Let our tastes change with our
years, let us no more meddle with age than with the seasons. We should
be ourselves at all times, instead of struggling against nature; such
vain attempts exhaust our strength and prevent the right use of life.

The lower classes are seldom dull, their life is full of activity; if
there is little variety in their amusements they do not recur
frequently; many days of labour teach them to enjoy their rare holidays.
Short intervals of leisure between long periods of labour give a spice
to the pleasures of their station. The chief curse of the rich is
dullness; in the midst of costly amusements, among so many men striving
to give them pleasure, they are devoured and slain by dullness; their
life is spent in fleeing from it and in being overtaken by it; they are
overwhelmed by the intolerable burden; women more especially, who do not
know how to work or play, are a prey to tedium under the name of the
vapours; with them it takes the shape of a dreadful disease, which robs
them of their reason and even of their life. For my own part I know no
more terrible fate than that of a pretty woman in Paris, unless it is
that of the pretty manikin who devotes himself to her, who becomes idle
and effeminate like her, and so deprives himself twice over of his
manhood, while he prides himself on his successes and for their sake
endures the longest and dullest days which human being ever put up with.

Proprieties, fashions, customs which depend on luxury and breeding,
confine the course of life within the limits of the most miserable
uniformity. The pleasure we desire to display to others is a pleasure
lost; we neither enjoy it ourselves, nor do others enjoy it.
{[}Footnote: Two ladies of fashion, who wished to seem to be enjoying
themselves greatly, decided never to go to bed before five o'clock in
the morning. In the depths of winter their servants spent the night in
the street waiting for them, and with great difficulty kept themselves
from freezing. One night, or rather one morning, some one entered the
room where these merry people spent their hours without knowing how time
passed. He found them quite alone; each of them was asleep in her
arm-chair.{]} Ridicule, which public opinion dreads more than anything,
is ever at hand to tyrannise, and punish. It is only ceremony that makes
us ridiculous; if we can vary our place and our pleasures, to-day's
impressions can efface those of yesterday; in the mind of men they are
as if they had never been; but we enjoy ourselves for we throw ourselves
into every hour and everything. My only set rule would be this: wherever
I was I would pay no heed to anything else. I would take each day as it
came, as if there were neither yesterday nor to-morrow. As I should be a
man of the people, with the populace, I should be a countryman in the
fields; and if I spoke of farming, the peasant should not laugh at my
expense. I would not go and build a town in the country nor erect the
Tuileries at the door of my lodgings. On some pleasant shady hill-side I
would have a little cottage, a white house with green shutters, and
though a thatched roof is the best all the year round, I would be grand
enough to have, not those gloomy slates, but tiles, because they look
brighter and more cheerful than thatch, and the houses in my own country
are always roofed with them, and so they would recall to me something of
the happy days of my youth. For my courtyard I would have a
poultry-yard, and for my stables a cowshed for the sake of the milk
which I love. My garden should be a kitchen-garden, and my park an
orchard, like the one described further on. The fruit would be free to
those who walked in the orchard, my gardener should neither count it nor
gather it; I would not, with greedy show, display before your eyes
superb espaliers which one scarcely dare touch. But this small
extravagance would not be costly, for I would choose my abode in some
remote province where silver is scarce and food plentiful, where plenty
and poverty have their seat.

There I would gather round me a company, select rather than numerous, a
band of friends who know what pleasure is, and how to enjoy it, women
who can leave their arm-chairs and betake themselves to outdoor sports,
women who can exchange the shuttle or the cards for the fishing line or
the bird-trap, the gleaner's rake or grape-gatherer's basket. There all
the pretensions of the town will be forgotten, and we shall be villagers
in a village; we shall find all sorts of different sports and we shall
hardly know how to choose the morrow's occupation. Exercise and an
active life will improve our digestion and modify our tastes. Every meal
will be a feast, where plenty will be more pleasing than any delicacies.
There are no such cooks in the world as mirth, rural pursuits, and merry
games; and the finest made dishes are quite ridiculous in the eyes of
people who have been on foot since early dawn. Our meals will be served
without regard to order or elegance; we shall make our dining-room
anywhere, in the garden, on a boat, beneath a tree; sometimes at a
distance from the house on the banks of a running stream, on the fresh
green grass, among the clumps of willow and hazel; a long procession of
guests will carry the material for the feast with laughter and singing;
the turf will be our chairs and table, the banks of the stream our
side-board, and our dessert is hanging on the trees; the dishes will be
served in any order, appetite needs no ceremony; each one of us, openly
putting himself first, would gladly see every one else do the same; from
this warm-hearted and temperate familiarity there would arise, without
coarseness, pretence, or constraint, a laughing conflict a hundredfold
more delightful than politeness, and more likely to cement our
friendship. No tedious flunkeys to listen to our words, to whisper
criticisms on our behaviour, to count every mouthful with greedy eyes,
to amuse themselves by keeping us waiting for our wine, to complain of
the length of our dinner. We will be our own servants, in order to be
our own masters. Time will fly unheeded, our meal will be an interval of
rest during the heat of the day. If some peasant comes our way,
returning from his work with his tools over his shoulder, I will cheer
his heart with kindly words, and a glass or two of good wine, which will
help him to bear his poverty more cheerfully; and I too shall have the
joy of feeling my heart stirred within me, and I should say to
myself---I too am a man.

If the inhabitants of the district assembled for some rustic feast, I
and my friends would be there among the first; if there were marriages,
more blessed than those of towns, celebrated near my home, every one
would know how I love to see people happy, and I should be invited. I
would take these good folks some gift as simple as themselves, a gift
which would be my share of the feast; and in exchange I should obtain
gifts beyond price, gifts so little known among my equals, the gifts of
freedom and true pleasure. I should sup gaily at the head of their long
table; I should join in the chorus of some rustic song and I should
dance in the barn more merrily than at a ball in the Opera House.

``This is all very well so far,'' you will say, ``but what about the
shooting! One must have some sport in the country.'' Just so; I only
wanted a farm, but I was wrong. I assume I am rich, I must keep my
pleasures to myself, I must be free to kill something; this is quite
another matter. I must have estates, woods, keepers, rents, seignorial
rights, particularly incense and holy water.

Well and good. But I shall have neighbours about my estate who are
jealous of their rights and anxious to encroach on those of others; our
keepers will quarrel, and possibly their masters will quarrel too; this
means altercations, disputes, ill-will, or law-suits at the least; this
in itself is not very pleasant. My tenants will not enjoy finding my
hares at work upon their corn, or my wild boars among their beans. As
they dare not kill the enemy, every one of them will try to drive him
from their fields; when the day has been spent in cultivating the
ground, they will be compelled to sit up at night to watch it; they will
have watch-dogs, drums, horns, and bells; my sleep will be disturbed by
their racket. Do what I will, I cannot help thinking of the misery of
these poor people, and I cannot help blaming myself for it. If I had the
honour of being a prince, this would make little impression on me; but
as I am a self-made man who has only just come into his property, I am
still rather vulgar at heart.

That is not all; abundance of game attracts trespassers; I shall soon
have poachers to punish; I shall require prisons, gaolers, guards, and
galleys; all this strikes me as cruel. The wives of those miserable
creatures will besiege my door and disturb me with their crying; they
must either be driven away or roughly handled. The poor people who are
not poachers, whose harvest has been destroyed by my game, will come
next with their complaints. Some people will be put to death for killing
the game, the rest will be punished for having spared it; what a choice
of evils! On every side I shall find nothing but misery and hear nothing
but groans. So far as I can see this must greatly disturb the pleasure
of slaying at one's ease heaps of partridges and hares which are tame
enough to run about one's feet.

If you would have pleasure without pain let there be no monopoly; the
more you leave it free to everybody, the purer will be your own
enjoyment. Therefore I should not do what I have just described, but
without change of tastes I would follow those which seem likely to cause
me least pain. I would fix my rustic abode in a district where game is
not preserved, and where I can have my sport without hindrance. Game
will be less plentiful, but there will be more skill in finding it, and
more pleasure in securing it. I remember the start of delight with which
my father watched the rise of his first partridge and the rapture with
which he found the hare he had sought all day long. Yes, I declare, that
alone with his dog, carrying his own gun, cartridges, and game bag
together with his hare, he came home at nightfall, worn out with fatigue
and torn to pieces by brambles, but better pleased with his day's sport
than all your ordinary sportsmen, who on a good horse, with twenty guns
ready for them, merely take one gun after another, and shoot and kill
everything that comes their way, without skill, without glory, and
almost without exercise. The pleasure is none the less, and the
difficulties are removed; there is no estate to be preserved, no poacher
to be punished, and no wretches to be tormented; here are solid grounds
for preference. Whatever you do, you cannot torment men for ever without
experiencing some amount of discomfort; and sooner or later the muttered
curses of the people will spoil the flavour of your game.

Again, monopoly destroys pleasure. Real pleasures are those which we
share with the crowd; we lose what we try to keep to ourselves alone. If
the walls I build round my park transform it into a gloomy prison, I
have only deprived myself, at great expense, of the pleasure of a walk;
I must now seek that pleasure at a distance. The demon of property
spoils everything he lays hands upon. A rich man wants to be master
everywhere, and he is never happy where he is; he is continually driven
to flee from himself. I shall therefore continue to do in my prosperity
what I did in my poverty. Henceforward, richer in the wealth of others
than I ever shall be in my own wealth, I will take possession of
everything in my neighbourhood that takes my fancy; no conqueror is so
determined as I; I even usurp the rights of princes; I take possession
of every open place that pleases me, I give them names; this is my park,
chat is my terrace, and I am their owner; henceforward I wander among
them at will; I often return to maintain my proprietary rights; I make
what use I choose of the ground to walk upon, and you will never
convince me that the nominal owner of the property which I have
appropriated gets better value out of the money it yields him than I do
out of his land. No matter if I am interrupted by hedges and ditches, I
take my park on my back, and I carry it elsewhere; there will be space
enough for it near at hand, and I may plunder my neighbours long enough
before I outstay my welcome.

This is an attempt to show what is meant by good taste in the choice of
pleasant occupations for our leisure hours; this is the spirit of
enjoyment; all else is illusion, fancy, and foolish pride. He who
disobeys these rules, however rich he may be, will devour his gold on a
dung-hill, and will never know what it is to live.

You will say, no doubt, that such amusements lie within the reach of
all, that we need not be rich to enjoy them. That is the very point I
was coming to. Pleasure is ours when we want it; it is only social
prejudice which makes everything hard to obtain, and drives pleasure
before us. To be happy is a hundredfold easier than it seems. If he
really desires to enjoy himself the man of taste has no need of riches;
all he wants is to be free and to be his own master. With health and
daily bread we are rich enough, if we will but get rid of our
prejudices; this is the ``Golden Mean'' of Horace. You folks with your
strong-boxes may find some other use for your wealth, for it cannot buy
you pleasure. Emile knows this as well as I, but his heart is purer and
more healthy, so he will feel it more strongly, and all that he has
beheld in society will only serve to confirm him in this opinion.

While our time is thus employed, we are ever on the look-out for Sophy,
and we have not yet found her. It was not desirable that she should be
found too easily, and I have taken care to look for her where I knew we
should not find her.

The time is come; we must now seek her in earnest, lest Emile should
mistake some one else for Sophy, and only discover his error when it is
too late. Then farewell Paris, far-famed Paris, with all your noise and
smoke and dirt, where the women have ceased to believe in honour and the
men in virtue. We are in search of love, happiness, innocence; the
further we go from Paris the better.

\mychapter{6}{BOOK V}

We have reached the last act of youth's drams; we are approaching its
closing scene.

It is not good that man should be alone. Emile is now a man, and we must
give him his promised helpmeet. That helpmeet is Sophy. Where is her
dwelling-place, where shall she be found? We must know beforehand what
she is, and then we can decide where to look for her. And when she is
found, our task is not ended. ``Since our young gentleman,'' says Locke,
``is about to marry, it is time to leave him with his mistress.'' And
with these words he ends his book. As I have not the honour of educating
``A young gentleman,'' I shall take care not to follow his example.

\subparagraph{SOPHY, OR WOMAN}\label{id01336}

Sophy should be as truly a woman as Emile is a man, i.e., she must
possess all those characters of her sex which are required to enable her
to play her part in the physical and moral order. Let us inquire to
begin with in what respects her sex differs from our own.

But for her sex, a woman is a man; she has the same organs, the same
needs, the same faculties. The machine is the same in its construction;
its parts, its working, and its appearance are similar. Regard it as you
will the difference is only in degree.

Yet where sex is concerned man and woman are unlike; each is the
complement of the other; the difficulty in comparing them lies in our
inability to decide, in either case, what is a matter of sex, and what
is not. General differences present themselves to the comparative
anatomist and even to the superficial observer; they seem not to be a
matter of sex; yet they are really sex differences, though the
connection eludes our observation. How far such differences may extend
we cannot tell; all we know for certain is that where man and woman are
alike we have to do with the characteristics of the species; where they
are unlike, we have to do with the characteristics of sex. Considered
from these two standpoints, we find so many instances of likeness and
unlikeness that it is perhaps one of the greatest of marvels how nature
has contrived to make two beings so like and yet so different.

These resemblances and differences must have an influence on the moral
nature; this inference is obvious, and it is confirmed by experience; it
shows the vanity of the disputes as to the superiority or the equality
of the sexes; as if each sex, pursuing the path marked out for it by
nature, were not more perfect in that very divergence than if it more
closely resembled the other. A perfect man and a perfect woman should no
more be alike in mind than in face, and perfection admits of neither
less nor more.

In the union of the sexes each alike contributes to the common end, but
in different ways. From this diversity springs the first difference
which may be observed between man and woman in their moral relations.
The man should be strong and active; the woman should be weak and
passive; the one must have both the power and the will; it is enough
that the other should offer little resistance.

When this principle is admitted, it follows that woman is specially made
for man's delight. If man in his turn ought to be pleasing in her eyes,
the necessity is less urgent, his virtue is in his strength, he pleases
because he is strong. I grant you this is not the law of love, but it is
the law of nature, which is older than love itself.

If woman is made to please and to be in subjection to man, she ought to
make herself pleasing in his eyes and not provoke him to anger; her
strength is in her charms, by their means she should compel him to
discover and use his strength. The surest way of arousing this strength
is to make it necessary by resistance. Thus pride comes to the help of
desire and each exults in the other's victory. This is the origin of
attack and defence, of the boldness of one sex and the timidity of the
other, and even of the shame and modesty with which nature has armed the
weak for the conquest of the strong.

Who can possibly suppose that nature has prescribed the same advances to
the one sex as to the other, or that the first to feel desire should be
the first to show it? What strange depravity of judgment! The
consequences of the act being so different for the two sexes, is it
natural that they should enter upon it with equal boldness? How can any
one fail to see that when the share of each is so unequal, if the one
were not controlled by modesty as the other is controlled by nature, the
result would be the destruction of both, and the human race would perish
through the very means ordained for its continuance?

Women so easily stir a man's senses and fan the ashes of a dying
passion, that if philosophy ever succeeded in introducing this custom
into any unlucky country, especially if it were a warm country where
more women are born than men, the men, tyrannised over by the women,
would at last become their victims, and would be dragged to their death
without the least chance of escape.

Female animals are without this sense of shame, but what of that? Are
their desires as boundless as those of women, which are curbed by this
shame? The desires of the animals are the result of necessity, and when
the need is satisfied, the desire ceases; they no longer make a feint of
repulsing the male, they do it in earnest. Their seasons of complaisance
are short and soon over. Impulse and restraint are alike the work of
nature. But what would take the place of this negative instinct in women
if you rob them of their modesty?

The Most High has deigned to do honour to mankind; he has endowed man
with boundless passions, together with a law to guide them, so that man
may be alike free and self-controlled; though swayed by these passions
man is endowed with reason by which to control them. Woman is also
endowed with boundless passions; God has given her modesty to restrain
them. Moreover, he has given to both a present reward for the right use
of their powers, in the delight which springs from that right use of
them, i.e., the taste for right conduct established as the law of our
behaviour. To my mind this is far higher than the instinct of the
beasts.

Whether the woman shares the man's passion or not, whether she is
willing or unwilling to satisfy it, she always repulses him and defends
herself, though not always with the same vigour, and therefore not
always with the same success. If the siege is to be successful, the
besieged must permit or direct the attack. How skilfully can she
stimulate the efforts of the aggressor. The freest and most delightful
of activities does not permit of any real violence; reason and nature
are alike against it; nature, in that she has given the weaker party
strength enough to resist if she chooses; reason, in that actual
violence is not only most brutal in itself, but it defeats its own ends,
not only because the man thus declares war against his companion and
thus gives her a right to defend her person and her liberty even at the
cost of the enemy's life, but also because the woman alone is the judge
of her condition, and a child would have no father if any man might
usurp a father's rights.

Thus the different constitution of the two sexes leads us to a third
conclusion, that the stronger party seems to be master, but is as a
matter of fact dependent on the weaker, and that, not by any foolish
custom of gallantry, nor yet by the magnanimity of the protector, but by
an inexorable law of nature. For nature has endowed woman with a power
of stimulating man's passions in excess of man's power of satisfying
those passions, and has thus made him dependent on her goodwill, and
compelled him in his turn to endeavour to please her, so that she may be
willing to yield to his superior strength. Is it weakness which yields
to force, or is it voluntary self-surrender? This uncertainty
constitutes the chief charm of the man's victory, and the woman is
usually cunning enough to leave him in doubt. In this respect the
woman's mind exactly resembles her body; far from being ashamed of her
weakness, she is proud of it; her soft muscles offer no resistance, she
professes that she cannot lift the lightest weight; she would be ashamed
to be strong. And why? Not only to gain an appearance of refinement; she
is too clever for that; she is providing herself beforehand with
excuses, with the right to be weak if she chooses.

The experience we have gained through our vices has considerably
modified the views held in older times; we rarely hear of violence for
which there is so little occasion that it would hardly be credited. Yet
such stories are common enough among the Jews and ancient Greeks; for
such views belong to the simplicity of nature, and have only been
uprooted by our profligacy. If fewer deeds of violence are quoted in our
days, it is not that men are more temperate, but because they are less
credulous, and a complaint which would have been believed among a simple
people would only excite laughter among ourselves; therefore silence is
the better course. There is a law in Deuteronomy, under which the
outraged maiden was punished, along with her assailant, if the crime
were committed in a town; but if in the country or in a lonely place,
the latter alone was punished. ``For,'' says the law, ``the maiden cried
for help, and there was none to hear.'' From this merciful
interpretation of the law, girls learnt not to let themselves be
surprised in lonely places.

This change in public opinion has had a perceptible effect on our
morals. It has produced our modern gallantry. Men have found that their
pleasures depend, more than they expected, on the goodwill of the fair
sex, and have secured this goodwill by attentions which have had their
reward.

See how we find ourselves led unconsciously from the physical to the
moral constitution, how from the grosser union of the sexes spring the
sweet laws of love. Woman reigns, not by the will of man, but by the
decrees of nature herself; she had the power long before she showed it.
That same Hercules who proposed to violate all the fifty daughters of
Thespis was compelled to spin at the feet of Omphale, and Samson, the
strong man, was less strong than Delilah. This power cannot be taken
from woman; it is hers by right; she would have lost it long ago, were
it possible.

The consequences of sex are wholly unlike for man and woman. The male is
only a male now and again, the female is always a female, or at least
all her youth; everything reminds her of her sex; the performance of her
functions requires a special constitution. She needs care during
pregnancy and freedom from work when her child is born; she must have a
quiet, easy life while she nurses her children; their education calls
for patience and gentleness, for a zeal and love which nothing can
dismay; she forms a bond between father and child, she alone can win the
father's love for his children and convince him that they are indeed his
own. What loving care is required to preserve a united family! And there
should be no question of virtue in all this, it must be a labour of
love, without which the human race would be doomed to extinction.

The mutual duties of the two sexes are not, and cannot be, equally
binding on both. Women do wrong to complain of the inequality of
man-made laws; this inequality is not of man's making, or at any rate it
is not the result of mere prejudice, but of reason. She to whom nature
has entrusted the care of the children must hold herself responsible for
them to their father. No doubt every breach of faith is wrong, and every
faithless husband, who robs his wife of the sole reward of the stern
duties of her sex, is cruel and unjust; but the faithless wife is worse;
she destroys the family and breaks the bonds of nature; when she gives
her husband children who are not his own, she is false both to him and
them, her crime is not infidelity but treason. To my mind, it is the
source of dissension and of crime of every kind. Can any position be
more wretched than that of the unhappy father who, when he clasps his
child to his breast, is haunted by the suspicion that this is the child
of another, the badge of his own dishonour, a thief who is robbing his
own children of their inheritance. Under such circumstances the family
is little more than a group of secret enemies, armed against each other
by a guilty woman, who compels them to pretend to love one another.

Thus it is not enough that a wife should be faithful; her husband, along
with his friends and neighbours, must believe in her fidelity; she must
be modest, devoted, retiring; she should have the witness not only of a
good conscience, but of a good reputation. In a word, if a father must
love his children, he must be able to respect their mother. For these
reasons it is not enough that the woman should be chaste, she must
preserve her reputation and her good name. From these principles there
arises not only a moral difference between the sexes, but also a fresh
motive for duty and propriety, which prescribes to women in particular
the most scrupulous attention to their conduct, their manners, their
behaviour. Vague assertions as to the equality of the sexes and the
similarity of their duties are only empty words; they are no answer to
my argument.

It is a poor sort of logic to quote isolated exceptions against laws so
firmly established. Women, you say, are not always bearing children.
Granted; yet that is their proper business. Because there are a hundred
or so of large towns in the world where women live licentiously and have
few children, will you maintain that it is their business to have few
children? And what would become of your towns if the remote country
districts, with their simpler and purer women, did not make up for the
barrenness of your fine ladies? There are plenty of country places where
women with only four or five children are reckoned unfruitful. In
conclusion, although here and there a woman may have few children, what
difference does it make? {[}Footnote: Without this the race would
necessarily diminish; all things considered, for its preservation each
woman ought to have about four children, for about half the children
born die before they can become parents, and two must survive to replace
the father and mother. See whether the towns will supply them?{]} Is it
any the less a woman's business to be a mother? And to not the general
laws of nature and morality make provision for this state of things?

Even if there were these long intervals, which you assume, between the
periods of pregnancy, can a woman suddenly change her way of life
without danger? Can she be a nursing mother to-day and a soldier
to-morrow? Will she change her tastes and her feelings as a chameleon
changes his colour? Will she pass at once from the privacy of household
duties and indoor occupations to the buffeting of the winds, the toils,
the labours, the perils of war? Will she be now timid, {[}Footnote:
Women's timidity is yet another instinct of nature against the double
risk she runs during pregnancy.{]} now brave, now fragile, now robust?
If the young men of Paris find a soldier's life too hard for them, how
would a woman put up with it, a woman who has hardly ventured out of
doors without a parasol and who has scarcely put a foot to the ground?
Will she make a good soldier at an age when even men are retiring from
this arduous business?

There are countries, I grant you, where women bear and rear children
with little or no difficulty, but in those lands the men go half-naked
in all weathers, they strike down the wild beasts, they carry a canoe as
easily as a knapsack, they pursue the chase for 700 or 800 leagues, they
sleep in the open on the bare ground, they bear incredible fatigues and
go many days without food. When women become strong, men become still
stronger; when men become soft, women become softer; change both the
terms and the ratio remains unaltered.

I am quite aware that Plato, in the Republic, assigns the same
gymnastics to women and men. Having got rid of the family there is no
place for women in his system of government, so he is forced to turn
them into men. That great genius has worked out his plans in detail and
has provided for every contingency; he has even provided against a
difficulty which in all likelihood no one would ever have raised; but he
has not succeeded in meeting the real difficulty. I am not speaking of
the alleged community of wives which has often been laid to his charge;
this assertion only shows that his detractors have never read his works.
I refer to that political promiscuity under which the same occupations
are assigned to both sexes alike, a scheme which could only lead to
intolerable evils; I refer to that subversion of all the tenderest of
our natural feelings, which he sacrificed to an artificial sentiment
which can only exist by their aid. Will the bonds of convention hold
firm without some foundation in nature? Can devotion to the state exist
apart from the love of those near and dear to us? Can patriotism thrive
except in the soil of that miniature fatherland, the home? Is it not the
good son, the good husband, the good father, who makes the good citizen?

When once it is proved that men and women are and ought to be unlike in
constitution and in temperament, it follows that their education must be
different. Nature teaches us that they should work together, but that
each has its own share of the work; the end is the same, but the means
are different, as are also the feelings which direct them. We have
attempted to paint a natural man, let us try to paint a helpmeet for
him.

You must follow nature's guidance if you would walk aright. The native
characters of sex should be respected as nature's handiwork. You are
always saying, ``Women have such and such faults, from which we are
free.'' You are misled by your vanity; what would be faults in you are
virtues in them; and things would go worse, if they were without these
so-called faults. Take care that they do not degenerate into evil, but
beware of destroying them.

On the other hand, women are always exclaiming that we educate them for
nothing but vanity and coquetry, that we keep them amused with trifles
that we may be their masters; we are responsible, so they say, for the
faults we attribute to them. How silly! What have men to do with the
education of girls? What is there to hinder their mothers educating them
as they please? There are no colleges for girls; so much the better for
them! Would God there were none for the boys, their education would be
more sensible and more wholesome. Who is it that compels a girl to waste
her time on foolish trifles? Are they forced, against their will, to
spend half their time over their toilet, following the example set them
by you? Who prevents you teaching them, or having them taught, whatever
seems good in your eyes? Is it our fault that we are charmed by their
beauty and delighted by their airs and graces, if we are attracted and
flattered by the arts they learn from you, if we love to see them
prettily dressed, if we let them display at leisure the weapons by which
we are subjugated? Well then, educate them like men. The more women are
like men, the less influence they will have over men, and then men will
be masters indeed.

All the faculties common to both sexes are not equally shared between
them, but taken as a whole they are fairly divided. Woman is worth more
as a woman and less as a man; when she makes a good use of her own
rights, she has the best of it; when she tries to usurp our rights, she
is our inferior. It is impossible to controvert this, except by quoting
exceptions after the usual fashion of the partisans of the fair sex.

To cultivate the masculine virtues in women and to neglect their own is
evidently to do them an injury. Women are too clear-sighted to be thus
deceived; when they try to usurp our privileges they do not abandon
their own; with this result: they are unable to make use of two
incompatible things, so they fall below their own level as women,
instead of rising to the level of men. If you are a sensible mother you
will take my advice. Do not try to make your daughter a good man in
defiance of nature. Make her a good woman, and be sure it will be better
both for her and us.

Does this mean that she must be brought up in ignorance and kept to
housework only? Is she to be man's handmaid or his help-meet? Will he
dispense with her greatest charm, her companionship? To keep her a slave
will he prevent her knowing and feeling? Will he make an automaton of
her? No, indeed, that is not the teaching of nature, who has given women
such a pleasant easy wit. On the contrary, nature means them to think,
to will, to love, to cultivate their minds as well as their persons; she
puts these weapons in their hands to make up for their lack of strength
and to enable them to direct the strength of men. They should learn many
things, but only such things as are suitable.

When I consider the special purpose of woman, when I observe her
inclinations or reckon up her duties, everything combines to indicate
the mode of education she requires. Men and women are made for each
other, but their mutual dependence differs in degree; man is dependent
on woman through his desires; woman is dependent on man through her
desires and also through her needs; he could do without her better than
she can do without him. She cannot fulfil her purpose in life without
his aid, without his goodwill, without his respect; she is dependent on
our feelings, on the price we put upon her virtue, and the opinion we
have of her charms and her deserts. Nature herself has decreed that
woman, both for herself and her children, should be at the mercy of
man's judgment.

Worth alone will not suffice, a woman must be thought worthy; nor
beauty, she must be admired; nor virtue, she must be respected. A
woman's honour does not depend on her conduct alone, but on her
reputation, and no woman who permits herself to be considered vile is
really virtuous. A man has no one but himself to consider, and so long
as he does right he may defy public opinion; but when a woman does right
her task is only half finished, and what people think of her matters as
much as what she really is. Hence her education must, in this respect,
be different from man's education. ``What will people think'' is the
grave of a man's virtue and the throne of a woman's.

The children's health depends in the first place on the mother's, and
the early education of man is also in a woman's hands; his morals, his
passions, his tastes, his pleasures, his happiness itself, depend on
her. A woman's education must therefore be planned in relation to man.
To be pleasing in his sight, to win his respect and love, to train him
in childhood, to tend him in manhood, to counsel and console, to make
his life pleasant and happy, these are the duties of woman for all time,
and this is what she should be taught while she is young. The further we
depart from this principle, the further we shall be from our goal, and
all our precepts will fail to secure her happiness or our own.

Every woman desires to be pleasing in men's eyes, and this is right; but
there is a great difference between wishing to please a man of worth, a
really lovable man, and seeking to please those foppish manikins who are
a disgrace to their own sex and to the sex which they imitate. Neither
nature nor reason can induce a woman to love an effeminate person, nor
will she win love by imitating such a person.

If a woman discards the quiet modest bearing of her sex, and adopts the
airs of such foolish creatures, she is not following her vocation, she
is forsaking it; she is robbing herself of the rights to which she lays
claim. ``If we were different,'' she says, ``the men would not like
us.'' She is mistaken. Only a fool likes folly; to wish to attract such
men only shows her own foolishness. If there were no frivolous men,
women would soon make them, and women are more responsible for men's
follies than men are for theirs. The woman who loves true manhood and
seeks to find favour in its sight will adopt means adapted to her ends.
Woman is a coquette by profession, but her coquetry varies with her
aims; let these aims be in accordance with those of nature, and a woman
will receive a fitting education.

Even the tiniest little girls love finery; they are not content to be
pretty, they must be admired; their little airs and graces show that
their heads are full of this idea, and as soon as they can understand
they are controlled by ``What will people think of you?'' If you are
foolish enough to try this way with little boys, it will not have the
same effect; give them their freedom and their sports, and they care
very little what people think; it is a work of time to bring them under
the control of this law.

However acquired, this early education of little girls is an excellent
thing in itself. As the birth of the body must precede the birth of the
mind, so the training of the body must precede the cultivation of the
mind. This is true of both sexes; but the aim of physical training for
boys and girls is not the same; in the one case it is the development of
strength, in the other of grace; not that these qualities should be
peculiar to either sex, but that their relative values should be
different. Women should be strong enough to do anything gracefully; men
should be skilful enough to do anything easily.

The exaggeration of feminine delicacy leads to effeminacy in men. Women
should not be strong like men but for them, so that their sons may be
strong. Convents and boarding-schools, with their plain food and ample
opportunities for amusements, races, and games in the open air and in
the garden, are better in this respect than the home, where the little
girl is fed on delicacies, continually encouraged or reproved, where she
is kept sitting in a stuffy room, always under her mother's eye, afraid
to stand or walk or speak or breathe, without a moment's freedom to play
or jump or run or shout, or to be her natural, lively, little self;
there is either harmful indulgence or misguided severity, and no trace
of reason. In this fashion heart and body are alike destroyed.

In Sparta the girls used to take part in military sports just like the
boys, not that they might go to war, but that they might bear sons who
could endure hardship. That is not what I desire. To provide the state
with soldiers it is not necessary that the mother should carry a musket
and master the Prussian drill. Yet, on the whole, I think the Greeks
were very wise in this matter of physical training. Young girls
frequently appeared in public, not with the boys, but in groups apart.
There was scarcely a festival, a sacrifice, or a procession without its
bands of maidens, the daughters of the chief citizens. Crowned with
flowers, chanting hymns, forming the chorus of the dance, bearing
baskets, vases, offerings, they presented a charming spectacle to the
depraved senses of the Greeks, a spectacle well fitted to efface the
evil effects of their unseemly gymnastics. Whatever this custom may have
done for the Greek men, it was well fitted to develop in the Greek women
a sound constitution by means of pleasant, moderate, and healthy
exercise; while the desire to please would develop a keen and cultivated
taste without risk to character.

When the Greek women married, they disappeared from public life; within
the four walls of their home they devoted themselves to the care of
their household and family. This is the mode of life prescribed for
women alike by nature and reason. These women gave birth to the
healthiest, strongest, and best proportioned men who ever lived, and
except in certain islands of ill repute, no women in the whole world,
not even the Roman matrons, were ever at once so wise and so charming,
so beautiful and so virtuous, as the women of ancient Greece.

It is admitted that their flowing garments, which did not cramp the
figure, preserved in men and women alike the fine proportions which are
seen in their statues. These are still the models of art, although
nature is so disfigured that they are no longer to be found among us.
The Gothic trammels, the innumerable bands which confine our limbs as in
a press, were quite unknown. The Greek women were wholly unacquainted
with those frames of whalebone in which our women distort rather than
display their figures. It seems to me that this abuse, which is carried
to an incredible degree of folly in England, must sooner or later lead
to the production of a degenerate race. Moreover, I maintain that the
charm which these corsets are supposed to produce is in the worst
possible taste; it is not a pleasant thing to see a woman cut in two
like a wasp---it offends both the eye and the imagination. A slender
waist has its limits, like everything else, in proportion and
suitability, and beyond these limits it becomes a defect. This defect
would be a glaring one in the nude; why should it be beautiful under the
costume?

I will not venture upon the reasons which induce women to incase
themselves in these coats of mail. A clumsy figure, a large waist, are
no doubt very ugly at twenty, but at thirty they cease to offend the
eye, and as we are bound to be what nature has made us at any given age,
and as there is no deceiving the eye of man, such defects are less
offensive at any age than the foolish affectations of a young thing of
forty.

Everything which cramps and confines nature is in bad taste; this is as
true of the adornments of the person as of the ornaments of the mind.
Life, health, common-sense, and comfort must come first; there is no
grace in discomfort, languor is not refinement, there is no charm in
ill-health; suffering may excite pity, but pleasure and delight demand
the freshness of health.

Boys and girls have many games in common, and this is as it should be;
do they not play together when they are grown up? They have also special
tastes of their own. Boys want movement and noise, drums, tops,
toy-carts; girls prefer things which appeal to the eye, and can be used
for dressing-up---mirrors, jewellery, finery, and specially dolls. The
doll is the girl's special plaything; this shows her instinctive bent
towards her life's work. The art of pleasing finds its physical basis in
personal adornment, and this physical side of the art is the only one
which the child can cultivate.

Here is a little girl busy all day with her doll; she is always changing
its clothes, dressing and undressing it, trying new combinations of
trimmings well or ill matched; her fingers are clumsy, her taste is
crude, but there is no mistaking her bent; in this endless occupation
time flies unheeded, the hours slip away unnoticed, even meals are
forgotten. She is more eager for adornment than for food. ``But she is
dressing her doll, not herself,'' you will say. Just so; she sees her
doll, she cannot see herself; she cannot do anything for herself, she
has neither the training, nor the talent, nor the strength; as yet she
herself is nothing, she is engrossed in her doll and all her coquetry is
devoted to it. This will not always be so; in due time she will be her
own doll.

We have here a very early and clearly-marked bent; you have only to
follow it and train it. What the little girl most clearly desires is to
dress her doll, to make its bows, its tippets, its sashes, and its
tuckers; she is dependent on other people's kindness in all this, and it
would be much pleasanter to be able to do it herself. Here is a motive
for her earliest lessons, they are not tasks prescribed, but favours
bestowed. Little girls always dislike learning to read and write, but
they are always ready to learn to sew. They think they are grown up, and
in imagination they are using their knowledge for their own adornment.

The way is open and it is easy to follow it; cutting out, embroidery,
lace-making follow naturally. Tapestry is not popular; furniture is too
remote from the child's interests, it has nothing to do with the person,
it depends on conventional tastes. Tapestry is a woman's amusement;
young girls never care for it.

This voluntary course is easily extended to include drawing, an art
which is closely connected with taste in dress; but I would not have
them taught landscape and still less figure painting. Leaves, fruit,
flowers, draperies, anything that will make an elegant trimming for the
accessories of the toilet, and enable the girl to design her own
embroidery if she cannot find a pattern to her taste; that will be quite
enough. Speaking generally, if it is desirable to restrict a man's
studies to what is useful, this is even more necessary for women, whose
life, though less laborious, should be even more industrious and more
uniformly employed in a variety of duties, so that one talent should not
be encouraged at the expense of others.

Whatever may be said by the scornful, good sense belongs to both sexes
alike. Girls are usually more docile than boys, and they should be
subjected to more authority, as I shall show later on, but that is no
reason why they should be required to do things in which they can see
neither rhyme nor reason. The mother's art consists in showing the use
of everything they are set to do, and this is all the easier as the
girl's intelligence is more precocious than the boy's. This principle
banishes, both for boys and girls, not only those pursuits which never
lead to any appreciable results, not even increasing the charms of those
who have pursued them, but also those studies whose utility is beyond
the scholar's present age and can only be appreciated in later years. If
I object to little boys being made to learn to read, still more do I
object to it for little girls until they are able to see the use of
reading; we generally think more of our own ideas than theirs in our
attempts to convince them of the utility of this art. After all, why
should a little girl know how to read and write! Has she a house to
manage? Most of them make a bad use of this fatal knowledge, and girls
are so full of curiosity that few of them will fail to learn without
compulsion. Possibly cyphering should come first; there is nothing so
obviously useful, nothing which needs so much practice or gives so much
opportunity for error as reckoning. If the little girl does not get the
cherries for her lunch without an arithmetical exercise, she will soon
learn to count.

I once knew a little girl who learnt to write before she could read, and
she began to write with her needle. To begin with, she would write
nothing but O's; she was always making O's, large and small, of all
kinds and one within another, but always drawn backwards. Unluckily one
day she caught a glimpse of herself in the glass while she was at this
useful work, and thinking that the cramped attitude was not pretty, like
another Minerva she flung away her pen and declined to make any more
O's. Her brother was no fonder of writing, but what he disliked was the
constraint, not the look of the thing. She was induced to go on with her
writing in this way. The child was fastidious and vain; she could not
bear her sisters to wear her clothes. Her things had been marked, they
declined to mark them any more, she must learn to mark them herself;
there is no need to continue the story.

Show the sense of the tasks you set your little girls, but keep them
busy. Idleness and insubordination are two very dangerous faults, and
very hard to cure when once established. Girls should be attentive and
industrious, but this is not enough by itself; they should early be
accustomed to restraint. This misfortune, if such it be, is inherent in
their sex, and they will never escape from it, unless to endure more
cruel sufferings. All their life long, they will have to submit to the
strictest and most enduring restraints, those of propriety. They must be
trained to bear the yoke from the first, so that they may not feel it,
to master their own caprices and to submit themselves to the will of
others. If they were always eager to be at work, they should sometimes
be compelled to do nothing. Their childish faults, unchecked and
unheeded, may easily lead to dissipation, frivolity, and inconstancy. To
guard against this, teach them above all things self-control. Under our
senseless conditions, the life of a good woman is a perpetual struggle
against self; it is only fair that woman should bear her share of the
ills she has brought upon man.

Beware lest your girls become weary of their tasks and infatuated with
their amusements; this often happens under our ordinary methods of
education, where, as Fenelon says, all the tedium is on one side and all
the pleasure on the other. If the rules already laid down are followed,
the first of these dangers will be avoided, unless the child dislikes
those about her. A little girl who is fond of her mother or her friend
will work by her side all day without getting tired; the chatter alone
will make up for any loss of liberty. But if her companion is
distasteful to her, everything done under her direction will be
distasteful too. Children who take no delight in their mother's company
are not likely to turn out well; but to judge of their real feelings you
must watch them and not trust to their words alone, for they are
flatterers and deceitful and soon learn to conceal their thoughts.
Neither should they be told that they ought to love their mother.
Affection is not the result of duty, and in this respect constraint is
out of place. Continual intercourse, constant care, habit itself, all
these will lead a child to love her mother, if the mother does nothing
to deserve the child's ill-will. The very control she exercises over the
child, if well directed, will increase rather than diminish the
affection, for women being made for dependence, girls feel themselves
made to obey.

Just because they have, or ought to have, little freedom, they are apt
to indulge themselves too fully with regard to such freedom as they
have; they carry everything to extremes, and they devote themselves to
their games with an enthusiasm even greater than that of boys. This is
the second difficulty to which I referred. This enthusiasm must be kept
in check, for it is the source of several vices commonly found among
women, caprice and that extravagant admiration which leads a woman to
regard a thing with rapture to-day and to be quite indifferent to it
to-morrow. This fickleness of taste is as dangerous as exaggeration; and
both spring from the same cause. Do not deprive them of mirth, laughter,
noise, and romping games, but do not let them tire of one game and go
off to another; do not leave them for a moment without restraint. Train
them to break off their games and return to their other occupations
without a murmur. Habit is all that is needed, as you have nature on
your side.

This habitual restraint produces a docility which woman requires all her
life long, for she will always be in subjection to a man, or to man's
judgment, and she will never be free to set her own opinion above his.
What is most wanted in a woman is gentleness; formed to obey a creature
so imperfect as man, a creature often vicious and always faulty, she
should early learn to submit to injustice and to suffer the wrongs
inflicted on her by her husband without complaint; she must be gentle
for her own sake, not his. Bitterness and obstinacy only multiply the
sufferings of the wife and the misdeeds of the husband; the man feels
that these are not the weapons to be used against him. Heaven did not
make women attractive and persuasive that they might degenerate into
bitterness, or meek that they should desire the mastery; their soft
voice was not meant for hard words, nor their delicate features for the
frowns of anger. When they lose their temper they forget themselves;
often enough they have just cause of complaint; but when they scold they
always put themselves in the wrong. We should each adopt the tone which
befits our sex; a soft-hearted husband may make an overbearing wife, but
a man, unless he is a perfect monster, will sooner or later yield to his
wife's gentleness, and the victory will be hers.

Daughters must always be obedient, but mothers need not always be harsh.
To make a girl docile you need not make her miserable; to make her
modest you need not terrify her; on the contrary, I should not be sorry
to see her allowed occasionally to exercise a little ingenuity, not to
escape punishment for her disobedience, but to evade the necessity for
obedience. Her dependence need not be made unpleasant, it is enough that
she should realise that she is dependent. Cunning is a natural gift of
woman, and so convinced am I that all our natural inclinations are
right, that I would cultivate this among others, only guarding against
its abuse.

For the truth of this I appeal to every honest observer. I do not ask
you to question women themselves, our cramping institutions may compel
them to sharpen their wits; I would have you examine girls, little
girls, newly-born so to speak; compare them with boys of the same age,
and I am greatly mistaken if you do not find the little boys heavy,
silly, and foolish, in comparison. Let me give one illustration in all
its childish simplicity.

Children are commonly forbidden to ask for anything at table, for people
think they can do nothing better in the way of education than to burden
them with useless precepts; as if a little bit of this or that were not
readily given or refused without leaving a poor child dying of
greediness intensified by hope. Every one knows how cunningly a little
boy brought up in this way asked for salt when he had been overlooked at
table. I do not suppose any one will blame him for asking directly for
salt and indirectly for meat; the neglect was so cruel that I hardly
think he would have been punished had he broken the rule and said
plainly that he was hungry. But this is what I saw done by a little girl
of six; the circumstances were much more difficult, for not only was she
strictly forbidden to ask for anything directly or indirectly, but
disobedience would have been unpardonable, for she had eaten of every
dish; one only had been overlooked, and on this she had set her heart.
This is what she did to repair the omission without laying herself open
to the charge of disobedience; she pointed to every dish in turn,
saying, ``I've had some of this; I've had some of this;'' however she
omitted the one dish so markedly that some one noticed it and said,
``Have not you had some of this?'' ``Oh, no,'' replied the greedy little
girl with soft voice and downcast eyes. These instances are typical of
the cunning of the little boy and girl.

What is, is good, and no general law can be bad. This special skill with
which the female sex is endowed is a fair equivalent for its lack of
strength; without it woman would be man's slave, not his helpmeet. By
her superiority in this respect she maintains her equality with man, and
rules in obedience. She has everything against her, our faults and her
own weakness and timidity; her beauty and her wiles are all that she
has. Should she not cultivate both? Yet beauty is not universal; it may
be destroyed by all sorts of accidents, it will disappear with years,
and habit will destroy its influence. A woman's real resource is her
wit; not that foolish wit which is so greatly admired in society, a wit
which does nothing to make life happier; but that wit which is adapted
to her condition, the art of taking advantage of our position and
controlling us through our own strength. Words cannot tell how
beneficial this is to man, what a charm it gives to the society of men
and women, how it checks the petulant child and restrains the brutal
husband; without it the home would be a scene of strife; with it, it is
the abode of happiness. I know that this power is abused by the sly and
the spiteful; but what is there that is not liable to abuse? Do not
destroy the means of happiness because the wicked use them to our hurt.

The toilet may attract notice, but it is the person that wins our
hearts. Our finery is not us; its very artificiality often offends, and
that which is least noticeable in itself often wins the most attention.
The education of our girls is, in this respect, absolutely topsy-turvy.
Ornaments are promised them as rewards, and they are taught to delight
in elaborate finery. ``How lovely she is!'' people say when she is most
dressed up. On the contrary, they should be taught that so much finery
is only required to hide their defects, and that beauty's real triumph
is to shine alone. The love of fashion is contrary to good taste, for
faces do not change with the fashion, and while the person remains
unchanged, what suits it at one time will suit it always.

If I saw a young girl decked out like a little peacock, I should show
myself anxious about her figure so disguised, and anxious what people
would think of her; I should say, ``She is over-dressed with all those
ornaments; what a pity! Do you think she could do with something
simpler? Is she pretty enough to do without this or that?'' Possibly she
herself would be the first to ask that her finery might be taken off and
that we should see how she looked without it. In that case her beauty
should receive such praise as it deserves. I should never praise her
unless simply dressed. If she only regards fine clothes as an aid to
personal beauty, and as a tacit confession that she needs their aid, she
will not be proud of her finery, she will be humbled by it; and if she
hears some one say, ``How pretty she is,'' when she is smarter than
usual, she will blush for shame.

Moreover, though there are figures that require adornment there are none
that require expensive clothes. Extravagance in dress is the folly of
the class rather than the individual, it is merely conventional. Genuine
coquetry is sometimes carefully thought out, but never sumptuous, and
Juno dressed herself more magnificently than Venus. ``As you cannot make
her beautiful you are making her fine,'' said Apelles to an unskilful
artist who was painting Helen loaded with jewellery. I have also noticed
that the smartest clothes proclaim the plainest women; no folly could be
more misguided. If a young girl has good taste and a contempt for
fashion, give her a few yards of ribbon, muslin, and gauze, and a
handful of flowers, without any diamonds, fringes, or lace, and she will
make herself a dress a hundredfold more becoming than all the smart
clothes of La Duchapt.

Good is always good, and as you should always look your best, the women
who know what they are about select a good style and keep to it, and as
they are not always changing their style they think less about dress
than those who can never settle to any one style. A genuine desire to
dress becomingly does not require an elaborate toilet. Young girls
rarely give much time to dress; needlework and lessons are the business
of the day; yet, except for the rouge, they are generally as carefully
dressed as older women and often in better taste. Contrary to the usual
opinion, the real cause of the abuse of the toilet is not vanity but
lack of occupation. The woman who devotes six hours to her toilet is
well aware that she is no better dressed than the woman who took half an
hour, but she has got rid of so many of the tedious hours and it is
better to amuse oneself with one's clothes than to be sick of
everything. Without the toilet how would she spend the time between
dinner and supper. With a crowd of women about her, she can at least
cause them annoyance, which is amusement of a kind; better still she
avoids a tete-a-tete with the husband whom she never sees at any other
time; then there are the tradespeople, the dealers in bric-a-brac, the
fine gentlemen, the minor poets with their songs, their verses, and
their pamphlets; how could you get them together but for the toilet. Its
only real advantage is the chance of a little more display than is
permitted by full dress, and perhaps this is less than it seems and a
woman gains less than she thinks. Do not be afraid to educate your women
as women; teach them a woman's business, that they be modest, that they
may know how to manage their house and look after their family; the
grand toilet will soon disappear, and they will be more tastefully
dressed.

Growing girls perceive at once that all this outside adornment is not
enough unless they have charms of their own. They cannot make themselves
beautiful, they are too young for coquetry, but they are not too young
to acquire graceful gestures, a pleasing voice, a self-possessed manner,
a light step, a graceful bearing, to choose whatever advantages are
within their reach. The voice extends its range, it grows stronger and
more resonant, the arms become plumper, the bearing more assured, and
they perceive that it is easy to attract attention however dressed.
Needlework and industry suffice no longer, fresh gifts are developing
and their usefulness is already recognised.

I know that stern teachers would have us refuse to teach little girls to
sing or dance, or to acquire any of the pleasing arts. This strikes me
as absurd. Who should learn these arts---our boys? Are these to be the
favourite accomplishments of men or women? Of neither, say they; profane
songs are simply so many crimes, dancing is an invention of the Evil
One; her tasks and her prayers we all the amusement a young girl should
have. What strange amusements for a child of ten! I fear that these
little saints who have been forced to spend their childhood in prayers
to God will pass their youth in another fashion; when they are married
they will try to make up for lost time. I think we must consider age as
well as sex; a young girl should not live like her grandmother; she
should be lively, merry, and eager; she should sing and dance to her
heart's content, and enjoy all the innocent pleasures of youth; the time
will come, all too soon, when she must settle down and adopt a more
serious tone.

But is this change in itself really necessary? Is it not merely another
result of our own prejudices? By making good women the slaves of dismal
duties, we have deprived marriage of its charm for men. Can we wonder
that the gloomy silence they find at home drives them elsewhere, or
inspires little desire to enter a state which offers so few attractions?
Christianity, by exaggerating every duty, has made our duties
impracticable and useless; by forbidding singing, dancing, and
amusements of every kind, it renders women sulky, fault-finding, and
intolerable at home. There is no religion which imposes such strict
duties upon married life, and none in which such a sacred engagement is
so often profaned. Such pains has been taken to prevent wives being
amiable, that their husbands have become indifferent to them. This
should not be, I grant you, but it will be, since husbands are but men.
I would have an English maiden cultivate the talents which will delight
her husband as zealously as the Circassian cultivates the
accomplishments of an Eastern harem. Husbands, you say, care little for
such accomplishments. So I should suppose, when they are employed, not
for the husband, but to attract the young rakes who dishonour the home.
But imagine a virtuous and charming wife, adorned with such
accomplishments and devoting them to her husband's amusement; will she
not add to his happiness? When he leaves his office worn out with the
day's work, will she not prevent him seeking recreation elsewhere? Have
we not all beheld happy families gathered together, each contributing to
the general amusement? Are not the confidence and familiarity thus
established, the innocence and the charm of the pleasures thus enjoyed,
more than enough to make up for the more riotous pleasures of public
entertainments?

Pleasant accomplishments have been made too formal an affair of rules
and precepts, so that young people find them very tedious instead of a
mere amusement or a merry game as they ought to be. Nothing can be more
absurd than an elderly singing or dancing master frowning upon young
people, whose one desire is to laugh, and adopting a more pedantic and
magisterial manner in teaching his frivolous art than if he were
teaching the catechism. Take the case of singing; does this art depend
on reading music; cannot the voice be made true and flexible, can we not
learn to sing with taste and even to play an accompaniment without
knowing a note? Does the same kind of singing suit all voices alike? Is
the same method adapted to every mind? You will never persuade me that
the same attitudes, the same steps, the same movements, the same
gestures, the same dances will suit a lively little brunette and a tall
fair maiden with languishing eyes. So when I find a master giving the
same lessons to all his pupils I say, ``He has his own routine, but he
knows nothing of his art!''

Should young girls have masters or mistresses? I cannot say; I wish they
could dispense with both; I wish they could learn of their own accord
what they are already so willing to learn. I wish there were fewer of
these dressed-up old ballet masters promenading our streets. I fear our
young people will get more harm from intercourse with such people than
profit from their instruction, and that their jargon, their tone, their
airs and graces, will instil a precocious taste for the frivolities
which the teacher thinks so important, and to which the scholars are
only too likely to devote themselves.

Where pleasure is the only end in view, any one may serve as
teacher---father, mother, brother, sister, friend, governess, the girl's
mirror, and above all her own taste. Do not offer to teach, let her ask;
do not make a task of what should be a reward, and in these studies
above all remember that the wish to succeed is the first step. If formal
instruction is required I leave it to you to choose between a master and
a mistress. How can I tell whether a dancing master should take a young
pupil by her soft white hand, make her lift her skirt and raise her
eyes, open her arms and advance her throbbing bosom? but this I know,
nothing on earth would induce me to be that master.

Taste is formed partly by industry and partly by talent, and by its
means the mind is unconsciously opened to the idea of beauty of every
kind, till at length it attains to those moral ideas which are so
closely related to beauty. Perhaps this is one reason why ideas of
propriety and modesty are acquired earlier by girls than by boys, for to
suppose that this early feeling is due to the teaching of the
governesses would show little knowledge of their style of teaching and
of the natural development of the human mind. The art of speaking stands
first among the pleasing arts; it alone can add fresh charms to those
which have been blunted by habit. It is the mind which not only gives
life to the body, but renews, so to speak, its youth; the flow of
feelings and ideas give life and variety to the countenance, and the
conversation to which it gives rise arouses and sustains attention, and
fixes it continuously on one object. I suppose this is why little girls
so soon learn to prattle prettily, and why men enjoy listening to them
even before the child can understand them; they are watching for the
first gleam of intelligence and sentiment.

Women have ready tongues; they talk earlier, more easily, and more
pleasantly than men. They are also said to talk more; this may be true,
but I am prepared to reckon it to their credit; eyes and mouth are
equally busy and for the same cause. A man says what he knows, a woman
says what will please; the one needs knowledge, the other taste; utility
should be the man's object; the woman speaks to give pleasure. There
should be nothing in common but truth.

You should not check a girl's prattle like a boy's by the harsh
question, ``What is the use of that?'' but by another question at least
as difficult to answer, ``What effect will that have?'' At this early
age when they know neither good nor evil, and are incapable of judging
others, they should make this their rule and never say anything which is
unpleasant to those about them; this rule is all the more difficult to
apply because it must always be subordinated to our first rule, ``Never
tell a lie.''

I can see many other difficulties, but they belong to a later stage. For
the present it is enough for your little girls to speak the truth
without grossness, and as they are naturally averse to what is gross,
education easily teaches them to avoid it. In social intercourse I
observe that a man's politeness is usually more helpful and a woman's
more caressing. This distinction is natural, not artificial. A man seeks
to serve, a woman seeks to please. Hence a woman's politeness is less
insincere than ours, whatever we may think of her character; for she is
only acting upon a fundamental instinct; but when a man professes to put
my interests before his own, I detect the falsehood, however disguised.
Hence it is easy for women to be polite, and easy to teach little girls
politeness. The first lessons come by nature; art only supplements them
and determines the conventional form which politeness shall take. The
courtesy of woman to woman is another matter; their manner is so
constrained, their attentions so chilly, they find each other so
wearisome, that they take little pains to conceal the fact, and seem
sincere even in their falsehood, since they take so little pains to
conceal it. Still young girls do sometimes become sincerely attached to
one another. At their age good spirits take the place of a good
disposition, and they are so pleased with themselves that they are
pleased with every one else. Moreover, it is certain that they kiss each
other more affectionately and caress each other more gracefully in the
presence of men, for they are proud to be able to arouse their envy
without danger to themselves by the sight of favours which they know
will arouse that envy.

If young boys must not be allowed to ask unsuitable questions, much more
must they be forbidden to little girls; if their curiosity is satisfied
or unskilfully evaded it is a much more serious matter, for they are so
keen to guess the mysteries concealed from them and so skilful to
discover them. But while I would not permit them to ask questions, I
would have them questioned frequently, and pains should be taken to make
them talk; let them be teased to make them speak freely, to make them
answer readily, to loosen mind and tongue while it can be done without
danger. Such conversation always leading to merriment, yet skilfully
controlled and directed, would form a delightful amusement at this age
and might instil into these youthful hearts the first and perhaps the
most helpful lessons in morals which they will ever receive, by teaching
them in the guise of pleasure and fun what qualities are esteemed by men
and what is the true glory and happiness of a good woman.

If boys are incapable of forming any true idea of religion, much more is
it beyond the grasp of girls; and for this reason I would speak of it
all the sooner to little girls, for if we wait till they are ready for a
serious discussion of these deep subjects we should be in danger of
never speaking of religion at all. A woman's reason is practical, and
therefore she soon arrives at a given conclusion, but she fails to
discover it for herself. The social relation of the sexes is a wonderful
thing. This relation produces a moral person of which woman is the eye
and man the hand, but the two are so dependent on one another that the
man teaches the woman what to see, while she teaches him what to do. If
women could discover principles and if men had as good heads for detail,
they would be mutually independent, they would live in perpetual strife,
and there would be an end to all society. But in their mutual harmony
each contributes to a common purpose; each follows the other's lead,
each commands and each obeys.

As a woman's conduct is controlled by public opinion, so is her religion
ruled by authority. The daughter should follow her mother's religion,
the wife her husband's. Were that religion false, the docility which
leads mother and daughter to submit to nature's laws would blot out the
sin of error in the sight of God. Unable to judge for themselves they
should accept the judgment of father and husband as that of the church.

While women unaided cannot deduce the rules of their faith, neither can
they assign limits to that faith by the evidence of reason; they allow
themselves to be driven hither and thither by all sorts of external
influences, they are ever above or below the truth. Extreme in
everything, they are either altogether reckless or altogether pious; you
never find them able to combine virtue and piety. Their natural
exaggeration is not wholly to blame; the ill-regulated control exercised
over them by men is partly responsible. Loose morals bring religion into
contempt; the terrors of remorse make it a tyrant; this is why women
have always too much or too little religion.

As a woman's religion is controlled by authority it is more important to
show her plainly what to believe than to explain the reasons for belief;
for faith attached to ideas half-understood is the main source of
fanaticism, and faith demanded on behalf of what is absurd leads to
madness or unbelief. Whether our catechisms tend to produce impiety
rather than fanaticism I cannot say, but I do know that they lead to one
or other.

In the first place, when you teach religion to little girls never make
it gloomy or tiresome, never make it a task or a duty, and therefore
never give them anything to learn by heart, not even their prayers. Be
content to say your own prayers regularly in their presence, but do not
compel them to join you. Let their prayers be short, as Christ himself
has taught us. Let them always be said with becoming reverence and
respect; remember that if we ask the Almighty to give heed to our words,
we should at least give heed to what we mean to say.

It does not much matter that a girl should learn her religion young, but
it does matter that she should learn it thoroughly, and still more that
she should learn to love it. If you make religion a burden to her, if
you always speak of God's anger, if in the name of religion you impose
all sorts of disagreeable duties, duties which she never sees you
perform, what can she suppose but that to learn one's catechism and to
say one's prayers is only the duty of a little girl, and she will long
to be grown-up to escape, like you, from these duties. Example! Example!
Without it you will never succeed in teaching children anything.

When you explain the Articles of Faith let it be by direct teaching, not
by question and answer. Children should only answer what they think, not
what has been drilled into them. All the answers in the catechism are
the wrong way about; it is the scholar who instructs the teacher; in the
child's mouth they are a downright lie, since they explain what he does
not understand, and affirm what he cannot believe. Find me, if you can,
an intelligent man who could honestly say his catechism. The first
question I find in our catechism is as follows: ``Who created you and
brought you into the world?'' To which the girl, who thinks it was her
mother, replies without hesitation, ``It was God.'' All she knows is
that she is asked a question which she only half understands and she
gives an answer she does not understand at all.

I wish some one who really understands the development of children's
minds would write a catechism for them. It might be the most useful book
ever written, and, in my opinion, it would do its author no little
honour. This at least is certain---if it were a good book it would be
very unlike our catechisms.

Such a catechism will not be satisfactory unless the child can answer
the questions of its own accord without having to learn the answers;
indeed the child will often ask the questions itself. An example is
required to make my meaning plain and I feel how ill equipped I am to
furnish such an example. I will try to give some sort of outline of my
meaning.

To get to the first question in our catechism I suppose we must begin
somewhat after the following fashion.

NURSE: Do you remember when your mother was a little girl?

CHILD: No, nurse.

NURSE: Why not, when you have such a good memory?

CHILD: I was not alive.

NURSE: Then you were not always alive!

CHILD: No.

NURSE: Will you live for ever!

CHILD: Yes.

NURSE: Are you young or old?

CHILD: I am young.

NURSE: Is your grandmamma old or young?

CHILD: She is old.

NURSE: Was she ever young?

CHILD: Yes.

NURSE: Why is she not young now?

CHILD: She has grown old.

NURSE: Will you grow old too?

CHILD: I don't know.

NURSE: Where are your last year's frocks?

CHILD: They have been unpicked.

NURSE: Why!

CHILD: Because they were too small for me.

NURSE: Why were they too small?

CHILD: I have grown bigger.

NURSE: Will you grow any more!

CHILD: Oh, yes.

NURSE: And what becomes of big girls?

CHILD: They grow into women.

NURSE: And what becomes of women!

CHILD: They are mothers.

NURSE: And what becomes of mothers?

CHILD: They grow old.

NURSE: Will you grow old?

CHILD: When I am a mother.

NURSE: And what becomes of old people?

CHILD: I don't know.

NURSE: What became of your grandfather?

CHILD: He died. {[}Footnote: The child will say this because she has
heard it said; but you must make sure she knows what death is, for the
idea is not so simple and within the child's grasp as people think. In
that little poem ``Abel'' you will find an example of the way to teach
them. This charming work breathes a delightful simplicity with which one
should feed one's own mind so as to talk with children.{]}

NURSE: Why did he die?

CHILD: Because he was so old.

NURSE: What becomes of old people!

CHILD: They die.

NURSE: And when you are old------?

CHILD: Oh nurse! I don't want to die!

NURSE: My dear, no one wants to die, and everybody dies.

CHILD: Why, will mamma die too!

NURSE: Yes, like everybody else. Women grow old as well as men, and old
age ends in death.

CHILD: What must I do to grow old very, very slowly?

NURSE: Be good while you are little.

CHILD: I will always be good, nurse.

NURSE: So much the better. But do you suppose you will live for ever?

CHILD: When I am very, very old------

NURSE: Well?

CHILD: When we are so very old you say we must die?

NURSE: You must die some day.

CHILD: Oh dear! I suppose I must.

NURSE: Who lived before you?

CHILD: My father and mother.

NURSE: And before them?

CHILD: Their father and mother.

NURSE: Who will live after you?

CHILD: My children.

NURSE: Who will live after them?

CHILD: Their children.

In this way, by concrete examples, you will find a beginning and end for
the human race like everything else---that is to say, a father and
mother who never had a father and mother, and children who will never
have children of their own.

It is only after a long course of similar questions that we are ready
for the first question in the catechism; then alone can we put the
question and the child may be able to understand it. But what a gap
there is between the first and the second question which is concerned
with the definitions of the divine nature. When will this chasm be
bridged? ``God is a spirit.'' ``And what is a spirit?'' Shall I start
the child upon this difficult question of metaphysics which grown men
find so hard to understand? These are no questions for a little girl to
answer; if she asks them, it is as much or more than we can expect. In
that case I should tell her quite simply, ``You ask me what God is; it
is not easy to say; we can neither hear nor see nor handle God; we can
only know Him by His works. To learn what He is, you must wait till you
know what He has done.''

If our dogmas are all equally true, they are not equally important. It
makes little difference to the glory of God that we should perceive it
everywhere, but it does make a difference to human society, and to every
member of that society, that a man should know and do the duties which
are laid upon him by the law of God, his duty to his neighbour and to
himself. This is what we should always be teaching one another, and it
is this which fathers and mothers are specially bound to teach their
little ones. Whether a virgin became the mother of her Creator, whether
she gave birth to God, or merely to a man into whom God has entered,
whether the Father and the Son are of the same substance or of like
substance only, whether the Spirit proceeded from one or both of these
who are but one, or from both together, however important these
questions may seem, I cannot see that it is any more necessary for the
human race to come to a decision with regard to them than to know what
day to keep Easter, or whether we should tell our beads, fast, and
refuse to eat meat, speak Latin or French in church, adorn the walls
with statues, hear or say mass, and have no wife of our own. Let each
think as he pleases; I cannot see that it matters to any one but
himself; for my own part it is no concern of mine. But what does concern
my fellow-creatures and myself alike is to know that there is indeed a
judge of human fate, that we are all His children, that He bids us all
be just, He bids us love one another, He bids us be kindly and merciful,
He bids us keep our word with all men, even with our own enemies and
His; we must know that the apparent happiness of this world is naught;
that there is another life to come, in which this Supreme Being will be
the rewarder of the just and the judge of the unjust. Children need to
be taught these doctrines and others like them and all citizens require
to be persuaded of their truth. Whoever sets his face against these
doctrines is indeed guilty; he is the disturber of the peace, the enemy
of society. Whoever goes beyond these doctrines and seeks to make us the
slaves of his private opinions, reaches the same goal by another way; to
establish his own kind of order he disturbs the peace; in his rash pride
he makes himself the interpreter of the Divine, and in His name demands
the homage and the reverence of mankind; so far as may be, he sets
himself in God's place; he should receive the punishment of sacrilege if
he is not punished for his intolerance.

Give no heed, therefore, to all those mysterious doctrines which are
words without ideas for us, all those strange teachings, the study of
which is too often offered as a substitute for virtue, a study which
more often makes men mad rather than good. Keep your children ever
within the little circle of dogmas which are related to morality.
Convince them that the only useful learning is that which teaches us to
act rightly. Do not make your daughters theologians and casuists; only
teach them such things of heaven as conduce to human goodness; train
them to feel that they are always in the presence of God, who sees their
thoughts and deeds, their virtue and their pleasures; teach them to do
good without ostentation and because they love it, to suffer evil
without a murmur, because God will reward them; in a word to be all
their life long what they will be glad to have been when they appear in
His presence. This is true religion; this alone is incapable of abuse,
impiety, or fanaticism. Let those who will, teach a religion more
sublime, but this is the only religion I know.

Moreover, it is as well to observe that, until the age when the reason
becomes enlightened, when growing emotion gives a voice to conscience,
what is wrong for young people is what those about have decided to be
wrong. What they are told to do is good; what they are forbidden to do
is bad; that is all they ought to know: this shows how important it is
for girls, even more than for boys, that the right people should be
chosen to be with them and to have authority over them. At last there
comes a time when they begin to judge things for themselves, and that is
the time to change your method of education.

Perhaps I have said too much already. To what shall we reduce the
education of our women if we give them no law but that of conventional
prejudice? Let us not degrade so far the set which rules over us, and
which does us honour when we have not made it vile. For all mankind
there is a law anterior to that of public opinion. All other laws should
bend before the inflexible control of this law; it is the judge of
public opinion, and only in so far as the esteem of men is in accordance
with this law has it any claim on our obedience.

This law is our individual conscience. I will not repeat what has been
said already; it is enough to point out that if these two laws clash,
the education of women will always be imperfect. Right feeling without
respect for public opinion will not give them that delicacy of soul
which lends to right conduct the charm of social approval; while respect
for public opinion without right feeling will only make false and wicked
women who put appearances in the place of virtue.

It is, therefore, important to cultivate a faculty which serves as judge
between the two guides, which does not permit conscience to go astray
and corrects the errors of prejudice. That faculty is reason. But what a
crowd of questions arise at this word. Are women capable of solid
reason; should they cultivate it, can they cultivate it successfully? Is
this culture useful in relation to the functions laid upon them? Is it
compatible with becoming simplicity?

The different ways of envisaging and answering these questions lead to
two extremes; some would have us keep women indoors sewing and spinning
with their maids; thus they make them nothing more than the chief
servant of their master. Others, not content to secure their rights,
lead them to usurp ours; for to make woman our superior in all the
qualities proper to her sex, and to make her our equal in all the rest,
what is this but to transfer to the woman the superiority which nature
has given to her husband? The reason which teaches a man his duties is
not very complex; the reason which teaches a woman hers is even simpler.
The obedience and fidelity which she owes to her husband, the tenderness
and care due to her children, are such natural and self-evident
consequences of her position that she cannot honestly refuse her consent
to the inner voice which is her guide, nor fail to discern her duty in
her natural inclination.

I would not altogether blame those who would restrict a woman to the
labours of her sex and would leave her in profound ignorance of
everything else; but that would require a standard of morality at once
very simple and very healthy, or a life withdrawn from the world. In
great towns, among immoral men, such a woman would be too easily led
astray; her virtue would too often be at the mercy of circumstances; in
this age of philosophy, virtue must be able to resist temptation; she
must know beforehand what she may hear and what she should think of it.

Moreover, in submission to man's judgment she should deserve his esteem;
above all she should obtain the esteem of her husband; she should not
only make him love her person, she should make him approve her conduct;
she should justify his choice before the world, and do honour to her
husband through the honour given to the wife. But how can she set about
this task if she is ignorant of our institutions, our customs, our
notions of propriety, if she knows nothing of the source of man's
judgment, nor the passions by which it is swayed! Since she depends both
on her own conscience and on public opinion, she must learn to know and
reconcile these two laws, and to put her own conscience first only when
the two are opposed to each other. She becomes the judge of her own
judges, she decides when she should obey and when she should refuse her
obedience. She weighs their prejudices before she accepts or rejects
them; she learns to trace them to their source, to foresee what they
will be, and to turn them in her own favour; she is careful never to
give cause for blame if duty allows her to avoid it. This cannot be
properly done without cultivating her mind and reason.

I always come back to my first principle and it supplies the solution of
all my difficulties. I study what is, I seek its cause, and I discover
in the end that what is, is good. I go to houses where the master and
mistress do the honours together. They are equally well educated,
equally polite, equally well equipped with wit and good taste, both of
them are inspired with the same desire to give their guests a good
reception and to send every one away satisfied. The husband omits no
pains to be attentive to every one; he comes and goes and sees to every
one and takes all sorts of trouble; he is attention itself. The wife
remains in her place; a little circle gathers round her and apparently
conceals the rest of the company from her; yet she sees everything that
goes on, no one goes without a word with her; she has omitted nothing
which might interest anybody, she has said nothing unpleasant to any
one, and without any fuss the least is no more overlooked than the
greatest. Dinner is announced, they take their places; the man knowing
the assembled guests will place them according to his knowledge; the
wife, without previous acquaintance, never makes a mistake; their looks
and bearing have already shown her what is wanted and every one will
find himself where he wishes to be. I do not assert that the servants
forget no one. The master of the house may have omitted no one, but the
mistress perceives what you like and sees that you get it; while she is
talking to her neighbour she has one eye on the other end of the table;
she sees who is not eating because he is not hungry and who is afraid to
help himself because he is clumsy and timid. When the guests leave the
table every one thinks she has had no thought but for him, everybody
thinks she has had no time to eat anything, but she has really eaten
more than anybody.

When the guests are gone, husband and wife tails over the events of the
evening. He relates what was said to him, what was said and done by
those with whom he conversed. If the lady is not always quite exact in
this respect, yet on the other hand she perceived what was whispered at
the other end of the room; she knows what so-and-so thought, and what
was the meaning of this speech or that gesture; there is scarcely a
change of expression for which she has not an explanation in readiness,
and she is almost always right.

The same turn of mind which makes a woman of the world such an excellent
hostess, enables a flirt to excel in the art of amusing a number of
suitors. Coquetry, cleverly carried out, demands an even finer
discernment than courtesy; provided a polite lady is civil to everybody,
she has done fairly well in any case; but the flirt would soon lose her
hold by such clumsy uniformity; if she tries to be pleasant to all her
lovers alike, she will disgust them all. In ordinary social intercourse
the manners adopted towards everybody are good enough for all; no
question is asked as to private likes or dislikes provided all are alike
well received. But in love, a favour shared with others is an insult. A
man of feeling would rather be singled out for ill-treatment than be
caressed with the crowd, and the worst that can befall him is to be
treated like every one else. So a woman who wants to keep several lovers
at her feet must persuade every one of them that she prefers him, and
she must contrive to do this in the sight of all the rest, each of whom
is equally convinced that he is her favourite.

If you want to see a man in a quandary, place him between two women with
each of whom he has a secret understanding, and see what a fool he
looks. But put a woman in similar circumstances between two men, and the
results will be even more remarkable; you will be astonished at the
skill with which she cheats them both, and makes them laugh at each
other. Now if that woman were to show the same confidence in both, if
she were to be equally familiar with both, how could they be deceived
for a moment? If she treated them alike, would she not show that they
both had the same claims upon her? Oh, she is far too clever for that;
so far from treating them just alike, she makes a marked difference
between them, and she does it so skilfully that the man she flatters
thinks it is affection, and the man she ill uses think it is spite. So
that each of them believes she is thinking of him, when she is thinking
of no one but herself.

A general desire to please suggests similar measures; people would be
disgusted with a woman's whims if they were not skilfully managed, and
when they are artistically distributed her servants are more than ever
enslaved.
\aquote{``Usa ogn'arte la donna, onde sia colto \\
     Nella sua rete alcun novello amante; \\
     Ne con tutti, ne sempre un stesso volto \\
     Serba; ma cangia a tempo atto e sembiante."}{Tasso, Jerus. Del., c. iv., v. 87.}
What is the secret of this art? Is it not the result of a delicate and
continuous observation which shows her what is taking place in a man's
heart, so that she is able to encourage or to check every hidden
impulse? Can this art be acquired? No; it is born with women; it is
common to them all, and men never show it to the same degree. It is one
of the distinctive characters of the sex. Self-possession, penetration,
delicate observation, this is a woman's science; the skill to make use
of it is her chief accomplishment.

This is what is, and we have seen why it is so. It is said that women
are false. They become false. They are really endowed with skill not
duplicity; in the genuine inclinations of their sex they are not false
even when they tell a lie. Why do you consult their words when it is not
their mouths that speak? Consult their eyes, their colour, their
breathing, their timid manner, their slight resistance, that is the
language nature gave them for your answer. The lips always say ``No,''
and rightly so; but the tone is not always the same, and that cannot
lie. Has not a woman the same needs as a man, but without the same right
to make them known? Her fate would be too cruel if she had no language
in which to express her legitimate desires except the words which she
dare not utter. Must her modesty condemn her to misery? Does she not
require a means of indicating her inclinations without open expression?
What skill is needed to hide from her lover what she would fain reveal!
Is it not of vital importance that she should learn to touch his heart
without showing that she cares for him? It is a pretty story that tale
of Galatea with her apple and her clumsy flight. What more is needed?
Will she tell the shepherd who pursues her among the willows that she
only flees that he may follow? If she did, it would be a lie; for she
would no longer attract him. The more modest a woman is, the more art
she needs, even with her husband. Yes, I maintain that coquetry, kept
within bounds, becomes modest and true, and out of it springs a law of
right conduct.

One of my opponents has very truly asserted that virtue is one; you
cannot disintegrate it and choose this and reject the other. If you love
virtue, you love it in its entirety, and you close your heart when you
can, and you always close your lips to the feelings which you ought not
to allow. Moral truth is not only what is, but what is good; what is bad
ought not to be, and ought not to be confessed, especially when that
confession produces results which might have been avoided. If I were
tempted to steal, and in confessing it I tempted another to become my
accomplice, the very confession of my temptation would amount to a
yielding to that temptation. Why do you say that modesty makes women
false? Are those who lose their modesty more sincere than the rest? Not
so, they are a thousandfold more deceitful. This degree of depravity is
due to many vices, none of which is rejected, vices which owe their
power to intrigue and falsehood. {[}Footnote: I know that women who have
openly decided on a certain course of conduct profess that their lack of
concealment is a virtue in itself, and swear that, with one exception,
they are possessed of all the virtues; but I am sure they never
persuaded any but fools to believe them. When the natural curb is
removed from their sex, what is there left to restrain them? What honour
will they prize when they have rejected the honour of their sex? Having
once given the rein to passion they have no longer any reason for
self-control. ``Nec femina, amissa pudicitia, alia abnuerit.'' No author
ever understood more thoroughly the heart of both sexes than Tacitus
when he wrote those words.{]}

On the other hand, those who are not utterly shameless, who take no
pride in their faults, who are able to conceal their desires even from
those who inspire them, those who confess their passion most
reluctantly, these are the truest and most sincere, these are they on
whose fidelity you may generally rely.

The only example I know which might be quoted as a recognised exception
to these remarks is Mlle. de L'Enclos; and she was considered a prodigy.
In her scorn for the virtues of women, she practised, so they say, the
virtues of a man. She is praised for her frankness and uprightness; she
was a trustworthy acquaintance and a faithful friend. To complete the
picture of her glory it is said that she became a man. That may be, but
in spite of her high reputation I should no more desire that man as my
friend than as my mistress.

This is not so irrelevant as it seems. I am aware of the tendencies of
our modern philosophy which make a jest of female modesty and its
so-called insincerity; I also perceive that the most certain result of
this philosophy will be to deprive the women of this century of such
shreds of honour as they still possess.

On these grounds I think we may decide in general terms what sort of
education is suited to the female mind, and the objects to which we
should turn its attention in early youth.

As I have already said, the duties of their sex are more easily
recognised than performed. They must learn in the first place to love
those duties by considering the advantages to be derived from
them---that is the only way to make duty easy. Every age and condition
has its own duties. We are quick to see our duty if we love it. Honour
your position as a woman, and in whatever station of life to which it
shall please heaven to call you, you will be well off. The essential
thing is to be what nature has made you; women are only too ready to be
what men would have them.

The search for abstract and speculative truths, for principles and
axioms in science, for all that tends to wide generalisation, is beyond
a woman's grasp; their studies should be thoroughly practical. It is
their business to apply the principles discovered by men, it is their
place to make the observations which lead men to discover those
principles. A woman's thoughts, beyond the range of her immediate
duties, should be directed to the study of men, or the acquirement of
that agreeable learning whose sole end is the formation of taste; for
the works of genius are beyond her reach, and she has neither the
accuracy nor the attention for success in the exact sciences; as for the
physical sciences, to decide the relations between living creatures and
the laws of nature is the task of that sex which is more active and
enterprising, which sees more things, that sex which is possessed of
greater strength and is more accustomed to the exercise of that
strength. Woman, weak as she is and limited in her range of observation,
perceives and judges the forces at her disposal to supplement her
weakness, and those forces are the passions of man. Her own mechanism is
more powerful than ours; she has many levers which may set the human
heart in motion. She must find a way to make us desire what she cannot
achieve unaided and what she considers necessary or pleasing; therefore
she must have a thorough knowledge of man's mind; not an abstract
knowledge of the mind of man in general, but the mind of those men who
are about her, the mind of those men who have authority over her, either
by law or custom. She must learn to divine their feelings from speech
and action, look and gesture. By her own speech and action, look and
gesture, she must be able to inspire them with the feelings she desires,
without seeming to have any such purpose. The men will have a better
philosophy of the human heart, but she will read more accurately in the
heart of men. Woman should discover, so to speak, an experimental
morality, man should reduce it to a system. Woman has more wit, man more
genius; woman observes, man reasons; together they provide the clearest
light and the profoundest knowledge which is possible to the unaided
human mind; in a word, the surest knowledge of self and of others of
which the human race is capable. In this way art may constantly tend to
the perfection of the instrument which nature has given us.

The world is woman's book; if she reads it ill, it is either her own
fault or she is blinded by passion. Yet the genuine mother of a family
is no woman of the world, she is almost as much of a recluse as the nun
in her convent. Those who have marriageable daughters should do what is
or ought to be done for those who are entering the cloisters: they
should show them the pleasures they forsake before they are allowed to
renounce them, lest the deceitful picture of unknown pleasures should
creep in to disturb the happiness of their retreat. In France it is the
girls who live in convents and the wives who flaunt in society. Among
the ancients it was quite otherwise; girls enjoyed, as I have said
already, many games and public festivals; the married women lived in
retirement. This was a more reasonable custom and more conducive to
morality. A girl may be allowed a certain amount of coquetry, and she
may be mainly occupied at amusement. A wife has other responsibilities
at home, and she is no longer on the look-out for a husband; but women
would not appreciate the change, and unluckily it is they who set the
fashion. Mothers, let your daughters be your companions. Give them good
sense and an honest heart, and then conceal from them nothing that a
pure eye may behold. Balls, assemblies, sports, the theatre itself;
everything which viewed amiss delights imprudent youth may be safely
displayed to a healthy mind. The more they know of these noisy
pleasures, the sooner they will cease to desire them.

I can fancy the outcry with which this will be received. What girl will
resist such an example? Their heads are turned by the first glimpse of
the world; not one of them is ready to give it up. That may be; but
before you showed them this deceitful prospect, did you prepare them to
behold it without emotion? Did you tell them plainly what it was they
would see? Did you show it in its true light? Did you arm them against
the illusions of vanity? Did you inspire their young hearts with a taste
for the true pleasures which are not to be met with in this tumult? What
precautions, what steps, did you take to preserve them from the false
taste which leads them astray? Not only have you done nothing to
preserve their minds from the tyranny of prejudice, you have fostered
that prejudice; you have taught them to desire every foolish amusement
they can get. Your own example is their teacher. Young people on their
entrance into society have no guide but their mother, who is often just
as silly as they are themselves, and quite unable to show them things
except as she sees them herself. Her example is stronger than reason; it
justifies them in their own eyes, and the mother's authority is an
unanswerable excuse for the daughter. If I ask a mother to bring her
daughter into society, I assume that she will show it in its true light.

The evil begins still earlier; the convents are regular schools of
coquetry; not that honest coquetry which I have described, but a
coquetry the source of every kind of misconduct, a coquetry which turns
out girls who are the most ridiculous little madams. When they leave the
convent to take their place in smart society, young women find
themselves quite at home. They have been educated for such a life; is it
strange that they like it? I am afraid what I am going to say may be
based on prejudice rather than observation, but so far as I can see, one
finds more family affection, more good wives and loving mothers in
Protestant than in Catholic countries; if that is so, we cannot fail to
suspect that the difference is partly due to the convent schools.

The charms of a peaceful family life must be known to be enjoyed; their
delights should be tasted in childhood. It is only in our father's home
that we learn to love our own, and a woman whose mother did not educate
her herself will not be willing to educate her own children.
Unfortunately, there is no such thing as home education in our large
towns. Society is so general and so mixed there is no place left for
retirement, and even in the home we live in public. We live in company
till we have no family, and we scarcely know our own relations, we see
them as strangers; and the simplicity of home life disappears together
with the sweet familiarity which was its charm. In this wise do we draw
with our mother's milk a taste for the pleasures of the age and the
maxims by which it is controlled.

Girls are compelled to assume an air of propriety so that men may be
deceived into marrying them by their appearance. But watch these young
people for a moment; under a pretence of coyness they barely conceal the
passion which devours them, and already you may read in their eager eyes
their desire to imitate their mothers. It is not a husband they want,
but the licence of a married woman. What need of a husband when there
are so many other resources; but a husband there must be to act as a
screen. {[}Footnote: The way of a man in his youth was one of the four
things that the sage could not understand; the fifth was the
shamelessness of an adulteress. ``Quae comedit, et tergens os suum
dicit; non sum operata malum.'' Prov. xxx. 20.{]} There is modesty on
the brow, but vice in the heart; this sham modesty is one of its outward
signs; they affect it that they may be rid of it once for all. Women of
Paris and London, forgive me! There may be miracles everywhere, but I am
not aware of them; and if there is even one among you who is really pure
in heart, I know nothing of our institutions.

All these different methods of education lead alike to a taste for the
pleasures of the great world, and to the passions which this taste so
soon kindles. In our great towns depravity begins at birth; in the
smaller towns it begins with reason. Young women brought up in the
country are soon taught to despise the happy simplicity of their lives,
and hasten to Paris to share the corruption of ours. Vices, cloaked
under the fair name of accomplishments, are the sole object of their
journey; ashamed to find themselves so much behind the noble licence of
the Parisian ladies, they hasten to become worthy of the name of
Parisian. Which is responsible for the evil---the place where it begins,
or the place where it is accomplished?

I would not have a sensible mother bring her girl to Paris to show her
these sights so harmful to others; but I assert that if she did so,
either the girl has been badly brought up, or such sights have little
danger for her. With good taste, good sense, and a love of what is
right, these things are less attractive than to those who abandon
themselves to their charm. In Paris you may see giddy young things
hastening to adopt the tone and fashions of the town for some six
months, so that they may spend the rest of their life in disgrace; but
who gives any heed to those who, disgusted with the rout, return to
their distant home and are contented with their lot when they have
compared it with that which others desire. How many young wives have I
seen whose good-natured husbands have taken them to Paris where they
might live if they pleased; but they have shrunk from it and returned
home more willingly than they went, saying tenderly, ``Ah, let us go
back to our cottage, life is happier there than in these palaces.'' We
do not know how many there are who have not bowed the knee to Baal, who
scorn his senseless worship. Fools make a stir; good women pass
unnoticed.

If so many women preserve a judgment which is proof against temptation,
in spite of universal prejudice, in spite of the bad education of girls,
what would their judgment have been, had it been strengthened by
suitable instruction, or rather left unaffected by evil teaching, for to
preserve or restore the natural feelings is our main business? You can
do this without preaching endless sermons to your daughters, without
crediting them with your harsh morality. The only effect of such
teaching is to inspire a dislike for the teacher and the lessons. In
talking to a young girl you need not make her afraid of her duties, nor
need you increase the burden laid upon her by nature. When you explain
her duties speak plainly and pleasantly; do not let her suppose that the
performance of these duties is a dismal thing---away with every
affectation of disgust or pride. Every thought which we desire to arouse
should find its expression in our pupils, their catechism of conduct
should be as brief and plain as their catechism of religion, but it need
not be so serious. Show them that these same duties are the source of
their pleasures and the basis of their rights. Is it so hard to win love
by love, happiness by an amiable disposition, obedience by worth, and
honour by self-respect? How fair are these woman's rights, how worthy of
reverence, how dear to the heart of man when a woman is able to show
their worth! These rights are no privilege of years; a woman's empire
begins with her virtues; her charms are only in the bud, yet she reigns
already by the gentleness of her character and the dignity of her
modesty. Is there any man so hard-hearted and uncivilised that he does
not abate his pride and take heed to his manners with a sweet and
virtuous girl of sixteen, who listens but says little; her bearing is
modest, her conversation honest, her beauty does not lead her to forget
her sex and her youth, her very timidity arouses interest, while she
wins for herself the respect which she shows to others?

These external signs are not devoid of meaning; they do not rest
entirely upon the charms of sense; they arise from that conviction that
we all feel that women are the natural judges of a man's worth. Who
would be scorned by women? not even he who has ceased to desire their
love. And do you suppose that I, who tell them such harsh truths, am
indifferent to their verdict? Reader, I care more for their approval
than for yours; you are often more effeminate than they. While I scorn
their morals, I will revere their justice; I care not though they hate
me, if I can compel their esteem.

What great things might be accomplished by their influence if only we
could bring it to bear! Alas for the age whose women lose their
ascendancy, and fail to make men respect their judgment! This is the
last stage of degradation. Every virtuous nation has shown respect to
women. Consider Sparta, Germany, and Rome; Rome the throne of glory and
virtue, if ever they were enthroned on earth. The Roman women awarded
honour to the deeds of great generals, they mourned in public for the
fathers of the country, their awards and their tears were alike held
sacred as the most solemn utterance of the Republic. Every great
revolution began with the women. Through a woman Rome gained her
liberty, through a woman the plebeians won the consulate, through a
woman the tyranny of the decemvirs was overthrown; it was the women who
saved Rome when besieged by Coriolanus. What would you have said at the
sight of this procession, you Frenchmen who pride yourselves on your
gallantry, would you not have followed it with shouts of laughter? You
and I see things with such different eyes, and perhaps we are both
right. Such a procession formed of the fairest beauties of France would
be an indecent spectacle; but let it consist of Roman ladies, you will
all gaze with the eyes of the Volscians and feel with the heart of
Coriolanus.

I will go further and maintain that virtue is no less favourable to love
than to other rights of nature, and that it adds as much to the power of
the beloved as to that of the wife or mother. There is no real love
without enthusiasm, and no enthusiasm without an object of perfection
real or supposed, but always present in the imagination. What is there
to kindle the hearts of lovers for whom this perfection is nothing, for
whom the loved one is merely the means to sensual pleasure? Nay, not
thus is the heart kindled, not thus does it abandon itself to those
sublime transports which form the rapture of lovers and the charm of
love. Love is an illusion, I grant you, but its reality consists in the
feelings it awakes, in the love of true beauty which it inspires. That
beauty is not to be found in the object of our affections, it is the
creation of our illusions. What matter! do we not still sacrifice all
those baser feelings to the imaginary model? and we still feed our
hearts on the virtues we attribute to the beloved, we still withdraw
ourselves from the baseness of human nature. What lover is there who
would not give his life for his mistress? What gross and sensual passion
is there in a man who is willing to die? We scoff at the knights of old;
they knew the meaning of love; we know nothing but debauchery. When the
teachings of romance began to seem ridiculous, it was not so much the
work of reason as of immorality.

Natural relations remain the same throughout the centuries, their good
or evil effects are unchanged; prejudices, masquerading as reason, can
but change their outward seeming; self-mastery, even at the behest of
fantastic opinions, will not cease to be great and good. And the true
motives of honour will not fail to appeal to the heart of every woman
who is able to seek happiness in life in her woman's duties. To a
high-souled woman chastity above all must be a delightful virtue. She
sees all the kingdoms of the world before her and she triumphs over
herself and them; she sits enthroned in her own soul and all men do her
homage; a few passing struggles are crowned with perpetual glory; she
secures the affection, or it may be the envy, she secures in any case
the esteem of both sexes and the universal respect of her own. The loss
is fleeting, the gain is permanent. What a joy for a noble heart---the
pride of virtue combined with beauty. Let her be a heroine of romance;
she will taste delights more exquisite than those of Lais and Cleopatra;
and when her beauty is fled, her glory and her joys remain; she alone
can enjoy the past.

The harder and more important the duties, the stronger and clearer must
be the reasons on which they are based. There is a sort of pious talk
about the most serious subjects which is dinned in vain into the ears of
young people. This talk, quite unsuited to their ideas and the small
importance they attach to it in secret, inclines them to yield readily
to their inclinations, for lack of any reasons for resistance drawn from
the facts themselves. No doubt a girl brought up to goodness and piety
has strong weapons against temptation; but one whose heart, or rather
her ears, are merely filled with the jargon of piety, will certainly
fall a prey to the first skilful seducer who attacks her. A young and
beautiful girl will never despise her body, she will never really
deplore sins which her beauty leads men to commit, she will never lament
earnestly in the sight of God that she is an object of desire, she will
never be convinced that the tenderest feeling is an invention of the
Evil One. Give her other and more pertinent reasons for her own sake,
for these will have no effect. It will be worse to instil, as is often
done, ideas which contradict each other, and after having humbled and
degraded her person and her charms as the stain of sin, to bid her
reverence that same vile body as the temple of Jesus Christ. Ideas too
sublime and too humble are equally ineffective and they cannot both be
true. A reason adapted to her age and sex is what is needed.
Considerations of duty are of no effect unless they are combined with
some motive for the performance of our duty.
\aquote{``Quae quia non liceat non facit, illa facit.''}{OVID, Amor. I. iii. eleg. iv.}
One would not suspect Ovid of such a harsh judgment.

If you would inspire young people with a love of good conduct avoid
saying, ``Be good;'' make it their interest to be good; make them feel
the value of goodness and they will love it. It is not enough to show
this effect in the distant future, show it now, in the relations of the
present, in the character of their lovers. Describe a good man, a man of
worth, teach them to recognise him when they see him, to love him for
their own sake; convince them that such a man alone can make them happy
as friend, wife, or mistress. Let reason lead the way to virtue; make
them feel that the empire of their sex and all the advantages derived
from it depend not merely on the right conduct, the morality, of women,
but also on that of men; that they have little hold over the vile and
base, and that the lover is incapable of serving his mistress unless he
can do homage to virtue. You may then be sure that when you describe the
manners of our age you will inspire them with a genuine disgust; when
you show them men of fashion they will despise them; you will give them
a distaste for their maxims, an aversion to their sentiments, and a
scorn for their empty gallantry; you will arouse a nobler ambition, to
reign over great and strong souls, the ambition of the Spartan women to
rule over men. A bold, shameless, intriguing woman, who can only attract
her lovers by coquetry and retain them by her favours, wins a servile
obedience in common things; in weighty and important matters she has no
influence over them. But the woman who is both virtuous, wise, and
charming, she who, in a word, combines love and esteem, can send them at
her bidding to the end of the world, to war, to glory, and to death at
her behest. This is a fine kingdom and worth the winning.

This is the spirit in which Sophy has been educated, she has been
trained carefully rather than strictly, and her taste has been followed
rather than thwarted. Let us say just a word about her person, according
to the description I have given to Emile and the picture he himself has
formed of the wife in whom he hopes to find happiness.

I cannot repeat too often that I am not dealing with prodigies. Emile is
no prodigy, neither is Sophy. He is a man and she is a woman; this is
all they have to boast of. In the present confusion between the sexes it
is almost a miracle to belong to one's own sex. Sophy is well born and
she has a good disposition; she is very warm-hearted, and this warmth of
heart sometimes makes her imagination run away with her. Her mind is
keen rather than accurate, her temper is pleasant but variable, her
person pleasing though nothing out of the common, her countenance
bespeaks a soul and it speaks true; you may meet her with indifference,
but you will not leave her without emotion. Others possess good
qualities which she lacks; others possess her good qualities in a higher
degree, but in no one are these qualities better blended to form a happy
disposition. She knows how to make the best of her very faults, and if
she were more perfect she would be less pleasing.

Sophy is not beautiful; but in her presence men forget the fairer women,
and the latter are dissatisfied with themselves. At first sight she is
hardly pretty; but the more we see her the prettier she is; she wins
where so many lose, and what she wins she keeps. Her eyes might be
finer, her mouth more beautiful, her stature more imposing; but no one
could have a more graceful figure, a finer complexion, a whiter hand, a
daintier foot, a sweeter look, and a more expressive countenance. She
does not dazzle; she arouses interest; she delights us, we know not why.

Sophy is fond of dress, and she knows how to dress; her mother has no
other maid; she has taste enough to dress herself well; but she hates
rich clothes; her own are always simple but elegant. She does not like
showy but becoming things. She does not know what colours are
fashionable, but she makes no mistake about those that suit her. No girl
seems more simply dressed, but no one could take more pains over her
toilet; no article is selected at random, and yet there is no trace of
artificiality. Her dress is very modest in appearance and very
coquettish in reality; she does not display her charms, she conceals
them, but in such a way as to enhance them. When you see her you say,
``That is a good modest girl,'' but while you are with her, you cannot
take your eyes or your thoughts off her and one might say that this very
simple adornment is only put on to be removed bit by bit by the
imagination.

Sophy has natural gifts; she is aware of them, and they have not been
neglected; but never having had a chance of much training she is content
to use her pretty voice to sing tastefully and truly; her little feet
step lightly, easily, and gracefully, she can always make an easy
graceful courtesy. She has had no singing master but her father, no
dancing mistress but her mother; a neighbouring organist has given her a
few lessons in playing accompaniments on the spinet, and she has
improved herself by practice. At first she only wished to show off her
hand on the dark keys; then she discovered that the thin clear tone of
the spinet made her voice sound sweeter; little by little she recognised
the charms of harmony; as she grew older she at last began to enjoy the
charms of expression, to love music for its own sake. But she has taste
rather than talent; she cannot read a simple air from notes.

Needlework is what Sophy likes best; and the feminine arts have been
taught her most carefully, even those you would not expect, such as
cutting out and dressmaking. There is nothing she cannot do with her
needle, and nothing that she does not take a delight in doing; but
lace-making is her favourite occupation, because there is nothing which
requires such a pleasing attitude, nothing which calls for such grace
and dexterity of finger. She has also studied all the details of
housekeeping; she understands cooking and cleaning; she knows the prices
of food, and also how to choose it; she can keep accounts accurately,
she is her mother's housekeeper. Some day she will be the mother of a
family; by managing her father's house she is preparing to manage her
own; she can take the place of any of the servants and she is always
ready to do so. You cannot give orders unless you can do the work
yourself; that is why her mother sets her to do it. Sophy does not think
of that; her first duty is to be a good daughter, and that is all she
thinks about for the present. Her one idea is to help her mother and
relieve her of some of her anxieties. However, she does not like them
all equally well. For instance, she likes dainty food, but she does not
like cooking; the details of cookery offend her, and things are never
clean enough for her. She is extremely sensitive in this respect and
carries her sensitiveness to a fault; she would let the whole dinner
boil over into the fire rather than soil her cuffs. She has always
disliked inspecting the kitchen-garden for the same reason. The soil is
dirty, and as soon as she sees the manure heap she fancies there is a
disagreeable smell.

This defect is the result of her mother's teaching. According to her,
cleanliness is one of the most necessary of a woman's duties, a special
duty, of the highest importance and a duty imposed by nature. Nothing
could be more revolting than a dirty woman, and a husband who tires of
her is not to blame. She insisted so strongly on this duty when Sophy
was little, she required such absolute cleanliness in her person,
clothing, room, work, and toilet, that use has become habit, till it
absorbs one half of her time and controls the other; so that she thinks
less of how to do a thing than of how to do it without getting dirty.

Yet this has not degenerated into mere affectation and softness; there
is none of the over refinement of luxury. Nothing but clean water enters
her room; she knows no perfumes but the scent of flowers, and her
husband will never find anything sweeter than her breath. In conclusion,
the attention she pays to the outside does not blind her to the fact
that time and strength are meant for greater tasks; either she does not
know or she despises that exaggerated cleanliness of body which degrades
the soul. Sophy is more than clean, she is pure.

I said that Sophy was fond of good things. She was so by nature; but she
became temperate by habit and now she is temperate by virtue. Little
girls are not to be controlled, as little boys are, to some extent,
through their greediness. This tendency may have ill effects on women
and it is too dangerous to be left unchecked. When Sophy was little, she
did not always return empty handed if she was sent to her mother's
cupboard, and she was not quite to be trusted with sweets and
sugar-almonds. Her mother caught her, took them from her, punished her,
and made her go without her dinner. At last she managed to persuade her
that sweets were bad for the teeth, and that over-eating spoiled the
figure. Thus Sophy overcame her faults; and when she grew older other
tastes distracted her from this low kind of self-indulgence. With
awakening feeling greediness ceases to be the ruling passion, both with
men and women. Sophy has preserved her feminine tastes; she likes milk
and sweets; she likes pastry and made-dishes, but not much meat. She has
never tasted wine or spirits; moreover, she eats sparingly; women, who
do not work so hard as men, have less waste to repair. In all things she
likes what is good, and knows how to appreciate it; but she can also put
up with what is not so good, or can go without it.

Sophy's mind is pleasing but not brilliant, and thorough but not deep;
it is the sort of mind which calls for no remark, as she never seems
cleverer or stupider than oneself. When people talk to her they always
find what she says attractive, though it may not be highly ornamental
according to modern ideas of an educated woman; her mind has been formed
not only by reading, but by conversation with her father and mother, by
her own reflections, and by her own observations in the little world in
which she has lived. Sophy is naturally merry; as a child she was even
giddy; but her mother cured her of her silly ways, little by little,
lest too sudden a change should make her self-conscious. Thus she became
modest and retiring while still a child, and now that she is a child no
longer, she finds it easier to continue this conduct than it would have
been to acquire it without knowing why. It is amusing to see her
occasionally return to her old ways and indulge in childish mirth and
then suddenly check herself, with silent lips, downcast eyes, and rosy
blushes; neither child nor woman, she may well partake of both.

Sophy is too sensitive to be always good humoured, but too gentle to let
this be really disagreeable to other people; it is only herself who
suffers. If you say anything that hurts her she does not sulk, but her
heart swells; she tries to run away and cry. In the midst of her tears,
at a word from her father or mother she returns at once laughing and
playing, secretly wiping her eyes and trying to stifle her sobs.

Yet she has her whims; if her temper is too much indulged it degenerates
into rebellion, and then she forgets herself. But give her time to come
round and her way of making you forget her wrong-doing is almost a
virtue. If you punish her she is gentle and submissive, and you see that
she is more ashamed of the fault than the punishment. If you say
nothing, she never fails to make amends, and she does it so frankly and
so readily that you cannot be angry with her. She would kiss the ground
before the lowest servant and would make no fuss about it; and as soon
as she is forgiven, you can see by her delight and her caresses that a
load is taken off her heart. In a word, she endures patiently the
wrong-doing of others, and she is eager to atone for her own. This
amiability is natural to her sex when unspoiled. Woman is made to submit
to man and to endure even injustice at his hands. You will never bring
young lads to this; their feelings rise in revolt against injustice;
nature has not fitted them to put up with it.
\aquote{``Gravem Pelidae stomachum cedere nescii.''}{HORACE, lib. i. ode vi.}
Sophy's religion is reasonable and simple, with few doctrines and fewer
observances; or rather as she knows no course of conduct but the right
her whole life is devoted to the service of God and to doing good. In
all her parents' teaching of religion she has been trained to a reverent
submission; they have often said, ``My little girl, this is too hard for
you; your husband will teach you when you are grown up.'' Instead of
long sermons about piety, they have been content to preach by their
example, and this example is engraved on her heart.

Sophy loves virtue; this love has come to be her ruling passion; she
loves virtue because there is nothing fairer in itself, she loves it
because it is a woman's glory and because a virtuous woman is little
lower than the angels; she loves virtue as the only road to real
happiness, because she sees nothing but poverty, neglect, unhappiness,
shame, and disgrace in the life of a bad woman; she loves virtue because
it is dear to her revered father and to her tender and worthy mother;
they are not content to be happy in their own virtue, they desire hers;
and she finds her chief happiness in the hope of making them happy. All
these feelings inspire an enthusiasm which stirs her heart and keeps all
its budding passions in subjection to this noble enthusiasm. Sophy will
be chaste and good till her dying day; she has vowed it in her secret
heart, and not before she knew how hard it would be to keep her vow; she
made this vow at a time when she would have revoked it had she been the
slave of her senses.

Sophy is not so fortunate as to be a charming French woman, cold-hearted
and vain, who would rather attract attention than give pleasure, who
seeks amusement rather than delight. She suffers from a consuming desire
for love; it even disturbs and troubles her heart in the midst of
festivities; she has lost her former liveliness, and her taste for merry
games; far from being afraid of the tedium of solitude she desires it.
Her thoughts go out to him who will make solitude sweet to her. She
finds strangers tedious, she wants a lover, not a circle of admirers.
She would rather give pleasure to one good man than be a general
favourite, or win that applause of society which lasts but a day and
to-morrow is turned to scorn.

A woman's judgment develops sooner than a man's; being on the defensive
from her childhood up, and intrusted with a treasure so hard to keep,
she is earlier acquainted with good and evil. Sophy is precocious by
temperament in everything, and her judgment is more formed than that of
most girls of her age. There is nothing strange in that, maturity is not
always reached at the same age.

Sophy has been taught the duties and rights of her own sex and of ours.
She knows men's faults and women's vices; she also knows their
corresponding good qualities and virtues, and has them by heart. No one
can have a higher ideal of a virtuous woman, but she would rather think
of a virtuous man, a man of true worth; she knows that she is made for
such a man, that she is worthy of him, that she can make him as happy as
he will make her; she is sure she will know him when she sees him; the
difficulty is to find him.

Women are by nature judges of a man's worth, as he is of theirs; this
right is reciprocal, and it is recognised as such both by men and women.
Sophy recognises this right and exercises it, but with the modesty
becoming her youth, her inexperience, and her position; she confines her
judgment to what she knows, and she only forms an opinion when it may
help to illustrate some useful precept. She is extremely careful what
she says about those who are absent, particularly if they are women. She
thinks that talking about each other makes women spiteful and satirical;
so long as they only talk about men they are merely just. So Sophy stops
there. As to women she never says anything at all about them, except to
tell the good she knows; she thinks this is only fair to her sex; and if
she knows no good of any woman, she says nothing, and that is enough.

Sophy has little knowledge of society, but she is observant and
obliging, and all that she does is full of grace. A happy disposition
does more for her than much art. She has a certain courtesy of her own,
which is not dependent on fashion, and does not change with its changes;
it is not a matter of custom, but it arises from a feminine desire to
please. She is unacquainted with the language of empty compliment, nor
does she invent more elaborate compliments of her own; she does not say
that she is greatly obliged, that you do her too much honour, that you
should not take so much trouble, etc. Still less does she try to make
phrases of her own. She responds to an attention or a customary piece of
politeness by a courtesy or a mere ``Thank you;'' but this phrase in her
mouth is quite enough. If you do her a real service, she lets her heart
speak, and its words are no empty compliment. She has never allowed
French manners to make her a slave to appearances; when she goes from
one room to another she does not take the arm of an old gentleman, whom
she would much rather help. When a scented fop offers her this empty
attention, she leaves him on the staircase and rushes into the room
saying that she is not lame. Indeed, she will never wear high heels
though she is not tall; her feet are small enough to dispense with them.

Not only does she adopt a silent and respectful attitude towards women,
but also towards married men, or those who are much older than herself;
she will never take her place above them, unless compelled to do so; and
she will return to her own lower place as soon as she can; for she knows
that the rights of age take precedence of those of sex, as age is
presumably wiser than youth, and wisdom should be held in the greatest
honour.

With young folks of her own age it is another matter; she requires a
different manner to gain their respect, and she knows how to adopt it
without dropping the modest ways which become her. If they themselves
are shy and modest, she will gladly preserve the friendly familiarity of
youth; their innocent conversation will be merry but suitable; if they
become serious they must say something useful; if they become silly, she
soon puts a stop to it, for she has an utter contempt for the jargon of
gallantry, which she considers an insult to her sex. She feels sure that
the man she seeks does not speak that jargon, and she will never permit
in another what would be displeasing to her in him whose character is
engraved on her heart. Her high opinion of the rights of women, her
pride in the purity of her feelings, that active virtue which is the
basis of her self-respect, make her indignant at the sentimental
speeches intended for her amusement. She does not receive them with open
anger, but with a disconcerting irony or an unexpected iciness. If a
fair Apollo displays his charms, and makes use of his wit in the praise
of her wit, her beauty, and her grace; at the risk of offending him she
is quite capable of saying politely, ``Sir, I am afraid I know that
better than you; if we have nothing more interesting to talk about, I
think we may put an end to this conversation.'' To say this with a deep
courtesy, and then to withdraw to a considerable distance, is the work
of a moment. Ask your lady-killers if it is easy to continue to babble
to such, an unsympathetic ear.

It is not that she is not fond of praise if it is really sincere, and if
she thinks you believe what you say. You must show that you appreciate
her merit if you would have her believe you. Her proud spirit may take
pleasure in homage which is based upon esteem, but empty compliments are
always rejected; Sophy was not meant to practise the small arts of the
dancing-girl.

With a judgment so mature, and a mind like that of a woman of twenty,
Sophy, at fifteen, is no longer treated as a child by her parents. No
sooner do they perceive the first signs of youthful disquiet than they
hasten to anticipate its development, their conversations with her are
wise and tender. These wise and tender conversations are in keeping with
her age and disposition. If her disposition is what I fancy why should
not her father speak to her somewhat after this fashion?

"You are a big girl now, Sophy, you will soon be a woman. We want you to
be happy, for our own sakes as well as yours, for our happiness depends
on yours. A good girl finds her own happiness in the happiness of a good
man, so we must consider your marriage; we must think of it in good
time, for marriage makes or mars our whole life, and we cannot have too
much time to consider it.

"There is nothing so hard to choose as a good husband, unless it is a
good wife. You will be that rare creature, Sophy, you will be the crown
of our life and the blessing of our declining years; but however worthy
you are, there are worthier people upon earth. There is no one who would
not do himself honour by marriage with you; there are many who would do
you even greater honour than themselves. Among these we must try to find
one who suits you, we must get to know him and introduce you to him.

"The greatest possible happiness in marriage depends on so many points
of agreement that it is folly to expect to secure them all. We must
first consider the more important matters; if others are to be found
along with them, so much the better; if not we must do without them.
Perfect happiness is not to be found in this world, but we can, at
least, avoid the worst form of unhappiness, that for which ourselves are
to blame.

"There is a natural suitability, there is a suitability of established
usage, and a suitability which is merely conventional. Parents should
decide as to the two latters, and the children themselves should decide
as to the former. Marriages arranged by parents only depend on a
suitability of custom and convention; it is not two people who are
united, but two positions and two properties; but these things may
change, the people remain, they are always there; and in spite of
fortune it is the personal relation that makes a happy or an unhappy
marriage.

"Your mother had rank, I had wealth; this was all that our parents
considered in arranging our marriage. I lost my money, she lost her
position; forgotten by her family, what good did it do her to be a lady
born? In the midst of our misfortunes, the union of our hearts has
outweighed them all; the similarity of our tastes led us to choose this
retreat; we live happily in our poverty, we are all in all to each
other. Sophy is a treasure we hold in common, and we thank Heaven which
has bestowed this treasure and deprived us of all others. You see, my
child, whither we have been led by Providence; the conventional motives
which brought about our marriage no longer exist, our happiness consists
in that natural suitability which was held of no account.

"Husband and wife should choose each other. A mutual liking should be
the first bond between them. They should follow the guidance of their
own eyes and hearts; when they are married their first duty will be to
love one another, and as love and hatred do not depend on ourselves,
this duty brings another with it, and they must begin to love each other
before marriage. That is the law of nature, and no power can abrogate
it; those who have fettered it by so many legal restrictions have given
heed rather to the outward show of order than to the happiness of
marriage or the morals of the citizen. You see, my dear Sophy, we do not
preach a harsh morality. It tends to make you your own mistress and to
make us leave the choice of your husband to yourself.

"When we have told you our reasons for giving you full liberty, it is
only fair to speak of your reasons for making a wise use of that
liberty. My child, you are good and sensible, upright and pious, you
have the accomplishments of a good woman and you are not altogether
without charms; but you are poor; you have the gifts most worthy of
esteem, but not those which are most esteemed. Do not seek what is
beyond your reach, and let your ambition be controlled, not by your
ideas or ours, but by the opinion of others. If it were merely a
question of equal merits, I know not what limits to impose on your
hopes; but do not let your ambitions outrun your fortune, and remember
it is very small. Although a man worthy of you would not consider this
inequality an obstacle, you must do what he would not do; Sophy must
follow her mother's example and only enter a family which counts it an
honour to receive her. You never saw our wealth, you were born in our
poverty; you make it sweet for us, and you share it without hardship.
Believe me, Sophy, do not seek those good things we indeed thank heaven
for having taken from us; we did not know what happiness was till we
lost our money.

"You are so amiable that you will win affection, and you are not go poor
as to be a burden. You will be sought in marriage, it may be by those
who are unworthy of you. If they showed themselves in their true
colours, you would rate them at their real value; all their outward show
would not long deceive you; but though your judgment is good and you
know what merit is when you see it, you are inexperienced and you do not
know how people can conceal their real selves. A skilful knave might
study your tastes in order to seduce you, and make a pretence of those
virtues which he does not possess. You would be ruined, Sophy, before
you knew what you were doing, and you would only perceive your error
when you had cause to lament it. The most dangerous snare, the only
snare which reason cannot avoid, is that of the senses; if ever you have
the misfortune to fall into its toils, you will perceive nothing but
fancies and illusions; your eyes will be fascinated, your judgment
troubled, your will corrupted, your very error will be dear to you, and
even if you were able to perceive it you would not be willing to escape
from it. My child, I trust you to Sophy's own reason; I do not trust you
to the fancies of your own heart. Judge for yourself so long as your
heart is untouched, but when you love betake yourself to your mother's
care.

``I propose a treaty between us which shows our esteem for you, and
restores the order of nature between us. Parents choose a husband for
their daughter and she is only consulted as a matter of form; that is
the custom. We shall do just the opposite; you will choose, and we shall
be consulted. Use your right, Sophy, use it freely and wisely. The
husband suitable for you should be chosen by you not us. But it is for
us to judge whether he is really suitable, or whether, without knowing
it, you are only following your own wishes. Birth, wealth, position,
conventional opinions will count for nothing with us. Choose a good man
whose person and character suit you; whatever he may be in other
respects, we will accept him as our son-in-law. He will be rich enough
if he has bodily strength, a good character, and family affection. His
position will be good enough if it is ennobled by virtue. If everybody
blames us, we do not care. We do not seek the approbation of men, but
your happiness.''

I cannot tell my readers what effect such words would have upon girls
brought up in their fashion. As for Sophy, she will have no words to
reply; shame and emotion will not permit her to express herself easily;
but I am sure that what was said will remain engraved upon her heart as
long as she lives, and that if any human resolution may be trusted, we
may rely on her determination to deserve her parent's esteem.

At worst let us suppose her endowed with an ardent disposition which
will make her impatient of long delays; I maintain that her judgment,
her knowledge, her taste, her refinement, and, above all, the sentiments
in which she has been brought up from childhood, will outweigh the
impetuosity of the senses, and enable her to offer a prolonged
resistance, if not to overcome them altogether. She would rather die a
virgin martyr than distress her parents by marrying a worthless man and
exposing herself to the unhappiness of an ill-assorted marriage. Ardent
as an Italian and sentimental as an Englishwoman, she has a curb upon
heart and sense in the pride of a Spaniard, who even when she seeks a
lover does not easily discover one worthy of her.

Not every one can realise the motive power to be found in a love of what
is right, nor the inner strength which results from a genuine love of
virtue. There are men who think that all greatness is a figment of the
brain, men who with their vile and degraded reason will never recognise
the power over human passions which is wielded by the very madness of
virtue. You can only teach such men by examples; if they persist in
denying their existence, so much the worse for them. If I told them that
Sophy is no imaginary person, that her name alone is my invention, that
her education, her conduct, her character, her very features, really
existed, and that her loss is still mourned by a very worthy family,
they would, no doubt, refuse to believe me; but indeed why should I not
venture to relate word for word the story of a girl so like Sophy that
this story might be hers without surprising any one. Believe it or no,
it is all the same to me; call my history fiction if you will; in any
case I have explained my method and furthered my purpose.

This young girl with the temperament which I have attributed to Sophy
was so like her in other respects that she was worthy of the name, and
so we will continue to use it. After the conversation related above, her
father and mother thought that suitable husbands would not be likely to
offer themselves in the hamlet where they lived; so they decided to send
her to spend the winter in town, under the care of an aunt who was
privately acquainted with the object of the journey; for Sophy's heart
throbbed with noble pride at the thought of her self-control; and
however much she might want to marry, she would rather have died a maid
than have brought herself to go in search of a husband.

In response to her parents' wishes her aunt introduced her to her
friends, took her into company, both private and public, showed her
society, or rather showed her in society, for Sophy paid little heed to
its bustle. Yet it was plain that she did not shrink from young men of
pleasing appearance and modest seemly behaviour. Her very shyness had a
charm of its own, which was very much like coquetry; but after talking
to them once or twice she repulsed them. She soon exchanged that air of
authority which seems to accept men's homage for a humbler bearing and a
still more chilling politeness. Always watchful over her conduct, she
gave them no chance of doing her the least service; it was perfectly
plain that she was determined not to accept any one of them.

Never did sensitive heart take pleasure in noisy amusements, the empty
and barren delights of those who have no feelings, those who think that
a merry life is a happy life. Sophy did not find what she sought, and
she felt sure she never would, so she got tired of the town. She loved
her parents dearly and nothing made up for their absence, nothing could
make her forget them; she went home long before the time fixed for the
end of her visit.

Scarcely had she resumed her home duties when they perceived that her
temper had changed though her conduct was unaltered, she was forgetful,
impatient, sad, and dreamy; she wept in secret. At first they thought
she was in love and was ashamed to own it; they spoke to her, but she
repudiated the idea. She protested she had seen no one who could touch
her heart, and Sophy always spoke the truth.

Yet her languor steadily increased, and her health began to give way.
Her mother was anxious about her, and determined to know the reason for
this change. She took her aside, and with the winning speech and the
irresistible caresses which only a mother can employ, she said, ``My
child, whom I have borne beneath my heart, whom I bear ever in my
affection, confide your secret to your mother's bosom. What secrets are
these which a mother may not know? Who pities your sufferings, who
shares them, who would gladly relieve them, if not your father and
myself? Ah, my child! would you have me die of grief for your sorrow
without letting me share it?''

Far from hiding her griefs from her mother, the young girl asked nothing
better than to have her as friend and comforter; but she could not speak
for shame, her modesty could find no words to describe a condition so
unworthy of her, as the emotion which disturbed her senses in spite of
all her efforts. At length her very shame gave her mother a clue to her
difficulty, and she drew from her the humiliating confession. Far from
distressing her with reproaches or unjust blame, she consoled her,
pitied her, wept over her; she was too wise to make a crime of an evil
which virtue alone made so cruel. But why put up with such an evil when
there was no necessity to do so, when the remedy was so easy and so
legitimate? Why did she not use the freedom they had granted her? Why
did she not take a husband? Why did she not make her choice? Did she not
know that she was perfectly independent in this matter, that whatever
her choice, it would be approved, for it was sure to be good? They had
sent her to town, but she would not stay; many suitors had offered
themselves, but she would have none of them. What did she expect? What
did she want? What an inexplicable contradiction?

The reply was simple. If it were only a question of the partner of her
youth, her choice would soon be made; but a master for life is not so
easily chosen; and since the two cannot be separated, people must often
wait and sacrifice their youth before they find the man with whom they
could spend their life. Such was Sophy's case; she wanted a lover, but
this lover must be her husband; and to discover a heart such as she
required, a lover and husband were equally difficult to find. All these
dashing young men were only her equals in age, in everything else they
were found lacking; their empty wit, their vanity, their affectations of
speech, their ill-regulated conduct, their frivolous imitations alike
disgusted her. She sought a man and she found monkeys; she sought a soul
and there was none to be found.

``How unhappy I am!'' said she to her mother; ``I am compelled to love
and yet I am dissatisfied with every one. My heart rejects every one who
appeals to my senses. Every one of them stirs my passions and all alike
revolt them; a liking unaccompanied by respect cannot last. That is not
the sort of man for your Sophy; the delightful image of her ideal is too
deeply graven in her heart. She can love no other; she can make no one
happy but him, and she cannot be happy without him. She would rather
consume herself in ceaseless conflicts, she would rather die free and
wretched, than driven desperate by the company of a man she did not
love, a man she would make as unhappy as herself; she would rather die
than live to suffer.''

Amazed at these strange ideas, her mother found them so peculiar that
she could not fail to suspect some mystery. Sophy was neither affected
nor absurd. How could such exaggerated delicacy exist in one who had
been so carefully taught from her childhood to adapt herself to those
with whom she must live, and to make a virtue of necessity? This ideal
of the delightful man with which she was so enchanted, who appeared so
often in her conversation, made her mother suspect that there was some
foundation for her caprices which was still unknown to her, and that
Sophy had not told her all. The unhappy girl, overwhelmed with her
secret grief, was only too eager to confide it to another. Her mother
urged her to speak; she hesitated, she yielded, and leaving the room
without a word, she presently returned with a book in her hand. ``Have
pity on your unhappy daughter, there is no remedy for her grief, her
tears cannot be dried. You would know the cause: well, here it is,''
said she, flinging the book on the table. Her mother took the book and
opened it; it was The Adventures of Telemachus. At first she could make
nothing of this riddle; by dint of questions and vague replies, she
discovered to her great surprise that her daughter was the rival of
Eucharis.

Sophy was in love with Telemachus, and loved him with a passion which
nothing could cure. When her father and mother became aware of her
infatuation, they laughed at it and tried to cure her by reasoning with
her. They were mistaken, reason was not altogether on their side; Sophy
had her own reason and knew how to use it. Many a time did she reduce
them to silence by turning their own arguments against them, by showing
them that it was all their own fault for not having trained her to suit
the men of that century; that she would be compelled to adopt her
husband's way of thinking or he must adopt hers, that they had made the
former course impossible by the way she had been brought up, and that
the latter was just what she wanted. ``Give me,'' said she, ``a man who
holds the same opinions as I do, or one who will be willing to learn
them from me, and I will marry him; but until then, why do you scold me?
Pity me; I am miserable, but not mad. Is the heart controlled by the
will? Did my father not ask that very question? Is it my fault if I love
what has no existence? I am no visionary; I desire no prince, I seek no
Telemachus, I know he is only an imaginary person; I seek some one like
him. And why should there be no such person, since there is such a
person as I, I who feel that my heart is like his? No, let us not wrong
humanity so greatly, let us not think that an amiable and virtuous man
is a figment of the imagination. He exists, he lives, perhaps he is
seeking me; he is seeking a soul which is capable of love for him. But
who is he, where is he? I know not; he is not among those I have seen;
and no doubt I shall never see him. Oh! mother, why did you make virtue
too attractive? If I can love nothing less, you are more to blame than
I.''

Must I continue this sad story to its close? Must I describe the long
struggles which preceded it? Must I show an impatient mother exchanging
her former caresses for severity? Must I paint an angry father
forgetting his former promises, and treating the most virtuous of
daughters as a mad woman? Must I portray the unhappy girl, more than
ever devoted to her imaginary hero, because of the persecution brought
upon her by that devotion, drawing nearer step by step to her death, and
descending into the grave when they were about to force her to the
altar? No; I will not dwell upon these gloomy scenes; I have no need to
go so far to show, by what I consider a sufficiently striking example,
that in spite of the prejudices arising from the manners of our age, the
enthusiasm for the good and the beautiful is no more foreign to women
than to men, and that there is nothing which, under nature's guidance,
cannot be obtained from them as well as from us.

You stop me here to inquire whether it is nature which teaches us to
take such pains to repress our immoderate desires. No, I reply, but
neither is it nature who gives us these immoderate desires. Now, all
that is not from nature is contrary to nature, as I have proved again
and again.

Let us give Emile his Sophy; let us restore this sweet girl to life and
provide her with a less vivid imagination and a happier fate. I desired
to paint an ordinary woman, but by endowing her with a great soul, I
have disturbed her reason. I have gone astray. Let us retrace our steps.
Sophy has only a good disposition and an ordinary heart; her education
is responsible for everything in which she excels other women.

In this book I intended to describe all that might be done and to leave
every one free to choose what he could out of all the good things I
described. I meant to train a helpmeet for Emile, from the very first,
and to educate them for each other and with each other. But on
consideration I thought all these premature arrangements undesirable,
for it was absurd to plan the marriage of two children before I could
tell whether this union was in accordance with nature and whether they
were really suited to each other. We must not confuse what is suitable
in a state of savagery with what is suitable in civilised life. In the
former, any woman will suit any man, for both are still in their
primitive and undifferentiated condition; in the latter, all their
characteristics have been developed by social institutions, and each
mind, having taken its own settled form, not from education alone, but
by the co-operation, more or less well-regulated, of natural disposition
and education, we can only make a match by introducing them to each
other to see if they suit each other in every respect, or at least we
can let them make that choice which gives the most promise of mutual
suitability.

The difficulty is this: while social life develops character it
differentiates classes, and these two classifications do not correspond,
so that the greater the social distinctions, the greater the difficulty
of finding the corresponding character. Hence we have ill-assorted
marriages and all their accompanying evils; and we find that it follows
logically that the further we get from equality, the greater the change
in our natural feelings; the wider the distance between great and small,
the looser the marriage tie; the deeper the gulf between rich and poor
the fewer husbands and fathers. Neither master nor slave belongs to a
family, but only to a class.

If you would guard against these abuses, and secure happy marriages, you
must stifle your prejudices, forget human institutions, and consult
nature. Do not join together those who are only alike in one given
condition, those who will not suit one another if that condition is
changed; but those who are adapted to one another in every situation, in
every country, and in every rank in which they may be placed. I do not
say that conventional considerations are of no importance in marriage,
but I do say that the influence of natural relations is so much more
important, that our fate in life is decided by them alone, and that
there is such an agreement of taste, temper, feeling, and disposition as
should induce a wise father, though he were a prince, to marry his son,
without a moment's hesitation, to the woman so adapted to him, were she
born in a bad home, were she even the hangman's daughter. I maintain
indeed that every possible misfortune may overtake husband and wife if
they are thus united, yet they will enjoy more real happiness while they
mingle their tears, than if they possessed all the riches of the world,
poisoned by divided hearts.

Instead of providing a wife for Emile in childhood, I have waited till I
knew what would suit him. It is not for me to decide, but for nature; my
task is to discover the choice she has made. My business, mine I repeat,
not his father's; for when he entrusted his son to my care, he gave up
his place to me. He gave me his rights; it is I who am really Emile's
father; it is I who have made a man of him. I would have refused to
educate him if I were not free to marry him according to his own choice,
which is mine. Nothing but the pleasure of bestowing happiness on a man
can repay me for the cost of making him capable of happiness.

Do not suppose, however, that I have delayed to find a wife for Emile
till I sent him in search of her. This search is only a pretext for
acquainting him with women, so that he may perceive the value of a
suitable wife. Sophy was discovered long since; Emile may even have seen
her already, but he will not recognise her till the time is come.

Although equality of rank is not essential in marriage, yet this
equality along with other kinds of suitability increases their value; it
is not to be weighed against any one of them, but, other things being
equal, it turns the scale.

A man, unless he is a king, cannot seek a wife in any and every class;
if he himself is free from prejudices, he will find them in others; and
this girl or that might perhaps suit him and yet she would be beyond his
reach. A wise father will therefore restrict his inquiries within the
bounds of prudence. He should not wish to marry his pupil into a family
above his own, for that is not within his power. If he could do so he
ought not desire it; for what difference does rank make to a young man,
at least to my pupil? Yet, if he rises he is exposed to all sorts of
real evils which he will feel all his life long. I even say that he
should not try to adjust the balance between different gifts, such as
rank and money; for each of these adds less to the value of the other
than the amount deducted from its own value in the process of
adjustment; moreover, we can never agree as to a common denominator; and
finally the preference, which each feels for his own surroundings, paves
the way for discord between the two families and often to difficulties
between husband and wife.

It makes a considerable difference as to the suitability of a marriage
whether a man marries above or beneath him. The former case is quite
contrary to reason, the latter is more in conformity with reason. As the
family is only connected with society through its head, it is the rank
of that head which decides that of the family as a whole. When he
marries into a lower rank, a man does not lower himself, he raises his
wife; if, on the other hand, he marries above his position, he lowers
his wife and does not raise himself. Thus there is in the first case
good unmixed with evil, in the other evil unmixed with good. Moreover,
the law of nature bids the woman obey the man. If he takes a wife from a
lower class, natural and civil law are in accordance and all goes well.
When he marries a woman of higher rank it is just the opposite case; the
man must choose between diminished rights or imperfect gratitude; he
must be ungrateful or despised. Then the wife, laying claim to
authority, makes herself a tyrant over her lawful head; and the master,
who has become a slave, is the most ridiculous and miserable of
creatures. Such are the unhappy favourites whom the sovereigns of Asia
honour and torment with their alliance; people tell us that if they
desire to sleep with their wife they must enter by the foot of the bed.

I expect that many of my readers will remember that I think women have a
natural gift for managing men, and will accuse me of contradicting
myself; yet they are mistaken. There is a vast difference between
claiming the right to command, and managing him who commands. Woman's
reign is a reign of gentleness, tact, and kindness; her commands are
caresses, her threats are tears. She should reign in the home as a
minister reigns in the state, by contriving to be ordered to do what she
wants. In this sense, I grant you, that the best managed homes are those
where the wife has most power. But when she despises the voice of her
head, when she desires to usurp his rights and take the command upon
herself, this inversion of the proper order of things leads only to
misery, scandal, and dishonour.

There remains the choice between our equals and our inferiors; and I
think we ought also to make certain restrictions with regard to the
latter; for it is hard to find in the lowest stratum of society a woman
who is able to make a good man happy; not that the lower classes are
more vicious than the higher, but because they have so little idea of
what is good and beautiful, and because the injustice of other classes
makes its very vices seem right in the eyes of this class.

By nature man thinks but seldom. He learns to think as he acquires the
other arts, but with even greater difficulty. In both sexes alike I am
only aware of two really distinct classes, those who think and those who
do not; and this difference is almost entirely one of education. A man
who thinks should not ally himself with a woman who does not think, for
he loses the chief delight of social life if he has a wife who cannot
share his thoughts. People who spend their whole life in working for a
living have no ideas beyond their work and their own interests, and
their mind seems to reside in their arms. This ignorance is not
necessarily unfavourable either to their honesty or their morals; it is
often favourable; we often content ourselves with thinking about our
duties, and in the end we substitute words for things. Conscience is the
most enlightened philosopher; to be an honest man we need not read
Cicero's De Officiis, and the most virtuous woman in the world is
probably she who knows least about virtue. But it is none the less true
that a cultivated mind alone makes intercourse pleasant, and it is a sad
thing for a father of a family, who delights in his home, to be forced
to shut himself up in himself and to be unable to make himself
understood.

Moreover, if a woman is quite unaccustomed to think, how can she bring
up her children? How will she know what is good for them? How can she
incline them to virtues of which she is ignorant, to merit of which she
has no conception? She can only flatter or threaten, she can only make
them insolent or timid; she will make them performing monkeys or noisy
little rascals; she will never make them intelligent or pleasing
children.

Therefore it is not fitting that a man of education should choose a wife
who has none, or take her from a class where she cannot be expected to
have any education. But I would a thousand times rather have a homely
girl, simply brought up, than a learned lady and a wit who would make a
literary circle of my house and install herself as its president. A
female wit is a scourge to her husband, her children, her friends, her
servants, to everybody. From the lofty height of her genius she scorns
every womanly duty, and she is always trying to make a man of herself
after the fashion of Mlle. de L'Enclos. Outside her home she always
makes herself ridiculous and she is very rightly a butt for criticism,
as we always are when we try to escape from our own position into one
for which we are unfitted. These highly talented women only get a hold
over fools. We can always tell what artist or friend holds the pen or
pencil when they are at work; we know what discreet man of letters
dictates their oracles in private. This trickery is unworthy of a decent
woman. If she really had talents, her pretentiousness would degrade
them. Her honour is to be unknown; her glory is the respect of her
husband; her joys the happiness of her family. I appeal to my readers to
give me an honest answer; when you enter a woman's room what makes you
think more highly of her, what makes you address her with more
respect---to see her busy with feminine occupations, with her household
duties, with her children's clothes about her, or to find her writing
verses at her toilet table surrounded with pamphlets of every kind and
with notes on tinted paper? If there were none but wise men upon earth
such a woman would die an old maid.
\aquote{``Quaeris cur nolim te ducere, galla? diserta es.''}{Martial, xi. 20.}
Looks must next be considered; they are the first thing that strikes us
and they ought to be the last, still they should not count for nothing.
I think that great beauty is rather to be shunned than sought after in
marriage. Possession soon exhausts our appreciation of beauty; in six
weeks' time we think no more about it, but its dangers endure as long as
life itself. Unless a beautiful woman is an angel, her husband is the
most miserable of men; and even if she were an angel he would still be
the centre of a hostile crowd and she could not prevent it. If extreme
ugliness were not repulsive I should prefer it to extreme beauty; for
before very long the husband would cease to notice either, but beauty
would still have its disadvantages and ugliness its advantages. But
ugliness which is actually repulsive is the worst misfortune; repulsion
increases rather than diminishes, and it turns to hatred. Such a union
is a hell upon earth; better death than such a marriage.

Desire mediocrity in all things, even in beauty. A pleasant attractive
countenance, which inspires kindly feelings rather than love, is what we
should prefer; the husband runs no risk, and the advantages are common
to husband and wife; charm is less perishable than beauty; it is a
living thing, which constantly renews itself, and after thirty years of
married life, the charms of a good woman delight her husband even as
they did on the wedding-day.

Such are the considerations which decided my choice of Sophy. Brought
up, like Emile, by Nature, she is better suited to him than any other;
she will be his true mate. She is his equal in birth and character, his
inferior in fortune. She makes no great impression at first sight, but
day by day reveals fresh charms. Her chief influence only takes effect
gradually, it is only discovered in friendly intercourse; and her
husband will feel it more than any one. Her education is neither showy
nor neglected; she has taste without deep study, talent without art,
judgment without learning. Her mind knows little, but it is trained to
learn; it is well-tilled soil ready for the sower. She has read no book
but Bareme and Telemachus which happened to fall into her hands; but no
girl who can feel so passionately towards Telemachus can have a heart
without feeling or a mind without discernment. What charming ignorance!
Happy is he who is destined to be her tutor. She will not be her
husband's teacher but his scholar; far from seeking to control his
tastes, she will share them. She will suit him far better than a
blue-stocking and he will have the pleasure of teaching her everything.
It is time they made acquaintance; let us try to plan a meeting.

When we left Paris we were sorrowful and wrapped in thought. This Babel
is not our home. Emile casts a scornful glance towards the great city,
saying angrily, ``What a time we have wasted; the bride of my heart is
not there. My friend, you knew it, but you think nothing of my time, and
you pay no heed to my sufferings.'' With steady look and firm voice I
reply, ``Emile, do you mean what you say?'' At once he flings his arms
round my neck and clasps me to his breast without speaking. That is his
answer when he knows he is in the wrong.

And now we are wandering through the country like true knights-errant;
yet we are not seeking adventures when we leave Paris; we are escaping
from them; now fast now slow, we wander through the country like
knights-errants. By following my usual practice the taste for it has
become established; and I do not suppose any of my readers are such
slaves of custom as to picture us dozing in a post-chaise with closed
windows, travelling, yet seeing nothing, observing nothing, making the
time between our start and our arrival a mere blank, and losing in the
speed of our journey, the time we meant to save.

Men say life is short, and I see them doing their best to shorten it. As
they do not know how to spend their time they lament the swiftness of
its flight, and I perceive that for them it goes only too slowly. Intent
merely on the object of their pursuit, they behold unwillingly the space
between them and it; one desires to-morrow, another looks a month ahead,
another ten years beyond that. No one wants to live to-day, no one
contents himself with the present hour, all complain that it passes
slowly. When they complain that time flies, they lie; they would gladly
purchase the power to hasten it; they would gladly spend their fortune
to get rid of their whole life; and there is probably not a single one
who would not have reduced his life to a few hours if he had been free
to get rid of those hours he found tedious, and those which separated
him from the desired moment. A man spends his whole life rushing from
Paris to Versailles, from Versailles to Paris, from town to country,
from country to town, from one district of the town to another; but he
would not know what to do with his time if he had not discovered this
way of wasting it, by leaving his business on purpose to find something
to do in coming back to it; he thinks he is saving the time he spends,
which would otherwise be unoccupied; or maybe he rushes for the sake of
rushing, and travels post in order to return in the same fashion. When
will mankind cease to slander nature? Why do you complain that life is
short when it is never short enough for you? If there were but one of
you, able to moderate his desires, so that he did not desire the flight
of time, he would never find life too short; for him life and the joy of
life would be one and the same; should he die young, he would still die
full of days.

If this were the only advantage of my way of travelling it would be
enough. I have brought Emile up neither to desire nor to wait, but to
enjoy; and when his desires are bent upon the future, their ardour is
not so great as to make time seem tedious. He will not only enjoy the
delights of longing, but the delights of approaching the object of his
desires; and his passions are under such restraint that he lives to a
great extent in the present.

So we do not travel like couriers but like explorers. We do not merely
consider the beginning and the end, but the space between. The journey
itself is a delight. We do not travel sitting, dismally imprisoned, so
to speak, in a tightly closed cage. We do not travel with the ease and
comfort of ladies. We do not deprive ourselves of the fresh air, nor the
sight of the things about us, nor the opportunity of examining them at
our pleasure. Emile will never enter a post-chaise, nor will he ride
post unless in a great hurry. But what cause has Emile for haste? None
but the joy of life. Shall I add to this the desire to do good when he
can? No, for that is itself one of the joys of life.

I can only think of one way of travelling pleasanter than travelling on
horseback, and that is to travel on foot. You start at your own time,
you stop when you will, you do as much or as little as you choose. You
see the country, you turn off to the right or left; you examine anything
which interests you, you stop to admire every view. Do I see a stream, I
wander by its banks; a leafy wood, I seek its shade; a cave, I enter it;
a quarry, I study its geology. If I like a place, I stop there. As soon
as I am weary of it, I go on. I am independent of horses and
postillions; I need not stick to regular routes or good roads; I go
anywhere where a man can go; I see all that a man can see; and as I am
quite independent of everybody, I enjoy all the freedom man can enjoy.
If I am stopped by bad weather and I find myself getting bored, then I
take horses. If I am tired---but Emile is hardly ever tired; he is
strong; why should he get tired? There is no hurry? If he stops, why
should he be bored? He always finds some amusement. He works at a trade;
he uses his arms to rest his feet.

To travel on foot is to travel in the fashion of Thales, Plato, and
Pythagoras. I find it hard to understand how a philosopher can bring
himself to travel in any other way; how he can tear himself from the
study of the wealth which lies before his eyes and beneath his feet. Is
there any one with an interest in agriculture, who does not want to know
the special products of the district through which he is passing, and
their method of cultivation? Is there any one with a taste for natural
history, who can pass a piece of ground without examining it, a rock
without breaking off a piece of it, hills without looking for plants,
and stones without seeking for fossils?

Your town-bred scientists study natural history in cabinets; they have
small specimens; they know their names but nothing of their nature.
Emile's museum is richer than that of kings; it is the whole world.
Everything is in its right place; the Naturalist who is its curator has
taken care to arrange it in the fairest order; Dauberton could do no
better.

What varied pleasures we enjoy in this delightful way of travelling, not
to speak of increasing health and a cheerful spirit. I notice that those
who ride in nice, well-padded carriages are always wrapped in thought,
gloomy, fault-finding, or sick; while those who go on foot are always
merry, light-hearted, and delighted with everything. How cheerful we are
when we get near our lodging for the night! How savoury is the coarse
food! How we linger at table enjoying our rest! How soundly we sleep on
a hard bed! If you only want to get to a place you may ride in a
post-chaise; if you want to travel you must go on foot.

If Sophy is not forgotten before we have gone fifty leagues in the way I
propose, either I am a bungler or Emile lacks curiosity; for with an
elementary knowledge of so many things, it is hardly to be supposed that
he will not be tempted to extend his knowledge. It is knowledge that
makes us curious; and Emile knows just enough to want to know more.

One thing leads on to another, and we make our way forward. If I chose a
distant object for the end of our first journey, it is not difficult to
find an excuse for it; when we leave Paris we must seek a wife at a
distance.

A few days later we had wandered further than usual among hills and
valleys where no road was to be seen and we lost our way completely. No
matter, all roads are alike if they bring you to your journey's end, but
if you are hungry they must lead somewhere. Luckily we came across a
peasant who took up to his cottage; we enjoyed his poor dinner with a
hearty appetite. When he saw how hungry and tired we were he said, ``If
the Lord had led you to the other side of the hill you would have had a
better welcome, you would have found a good resting place, such good,
kindly people! They could not wish to do more for you than I, but they
are richer, though folks say they used to be much better off. Still they
are not reduced to poverty, and the whole country-side is the better for
what they have.''

When Emile heard of these good people his heart warmed to them. ``My
friend,'' said he, looking at me, ``let us visit this house, whose
owners are a blessing to the district; I shall be very glad to see them;
perhaps they will be pleased to see us too; I am sure we shall be
welcome; we shall just suit each other.''

Our host told us how to find our way to the house and we set off, but
lost our way in the woods. We were caught in a heavy rainstorm, which
delayed us further. At last we found the right path and in the evening
we reached the house, which had been described to us. It was the only
house among the cottages of the little hamlet, and though plain it had
an air of dignity. We went up to the door and asked for hospitality. We
were taken to the owner of the house, who questioned us courteously;
without telling him the object of our journey, we told him why we had
left our path. His former wealth enabled him to judge a man's position
by his manners; those who have lived in society are rarely mistaken;
with this passport we were admitted.

The room we were shown into was very small, but clean and comfortable; a
fire was lighted, and we found linen, clothes, and everything we needed.
``Why,'' said Emile, in astonishment, ``one would think they were
expecting us. The peasant was quite right; how kind and attentive, how
considerate, and for strangers too! I shall think I am living in the
times of Homer.'' ``I am glad you feel this,'' said I, ``but you need
not be surprised; where strangers are scarce, they are welcome; nothing
makes people more hospitable than the fact that calls upon their
hospitality are rare; when guests are frequent there is an end to
hospitality. In Homer's time, people rarely travelled, and travellers
were everywhere welcome. Very likely we are the only people who have
passed this way this year.'' ``Never mind,'' said he, ``to know how to
do without guests and yet to give them a kind welcome, is its own
praise.''

Having dried ourselves and changed our clothes, we rejoined the master
of the house, who introduced us to his wife; she received us not merely
with courtesy but with kindness. Her glance rested on Emile. A mother,
in her position, rarely receives a young man into her house without some
anxiety or some curiosity at least.

Supper was hurried forward on our account. When we went into the
dining-room there were five places laid; we took our seats and the fifth
chair remained empty. Presently a young girl entered, made a deep
courtesy, and modestly took her place without a word. Emile was busy
with his supper or considering how to reply to what was said to him; he
bowed to her and continued talking and eating. The main object of his
journey was as far from his thoughts as he believed himself to be from
the end of his journey. The conversation turned upon our losing our way.
``Sir,'' said the master of the house to Emile, ``you seem to be a
pleasant well-behaved young gentleman, and that reminds me that your
tutor and you arrived wet and weary like Telemachus and Mentor in the
island of Calypso.'' ``Indeed,'' said Emile, ``we have found the
hospitality of Calypso.'' His Mentor added, ``And the charms of
Eucharis.'' But Emile knew the Odyssey and he had not read Telemachus,
so he knew nothing of Eucharis. As for the young girl, I saw she blushed
up to her eyebrows, fixed her eyes on her plate, and hardly dared to
breathe. Her mother, noticing her confusion, made a sign to her father
to turn the conversation. When he talked of his lonely life, he
unconsciously began to relate the circumstances which brought him into
it; his misfortunes, his wife's fidelity, the consolations they found in
their marriage, their quiet, peaceful life in their retirement, and all
this without a word of the young girl; it is a pleasing and a touching
story, which cannot fail to interest. Emile, interested and sympathetic,
leaves off eating and listens. When finally this best of men discourses
with delight of the affection of the best of women, the young traveller,
carried away by his feelings, stretches one hand to the husband, and
taking the wife's hand with the other, he kisses it rapturously and
bathes it with his tears. Everybody is charmed with the simple
enthusiasm of the young man; but the daughter, more deeply touched than
the rest by this evidence of his kindly heart, is reminded of Telemachus
weeping for the woes of Philoctetus. She looks at him shyly, the better
to study his countenance; there is nothing in it to give the lie to her
comparison.

His easy bearing shows freedom without pride; his manners are lively but
not boisterous; sympathy makes his glance softer and his expression more
pleasing; the young girl, seeing him weep, is ready to mingle her tears
with his. With so good an excuse for tears, she is restrained by a
secret shame; she blames herself already for the tears which tremble on
her eyelids, as though it were wrong to weep for one's family.

Her mother, who has been watching her ever since she sat down to supper,
sees her distress, and to relieve it she sends her on some errand. The
daughter returns directly, but so little recovered that her distress is
apparent to all. Her mother says gently, ``Sophy, control yourself; will
you never cease to weep for the misfortunes of your parents? Why should
you, who are their chief comfort, be more sensitive than they are
themselves?''

At the name of Sophy you would have seen Emile give a start. His
attention is arrested by this dear name, and he awakes all at once and
looks eagerly at one who dares to bear it. Sophy! Are you the Sophy whom
my heart is seeking? Is it you that I love? He looks at her; he watches
her with a sort of fear and self-distrust. The face is not quite what he
pictured; he cannot tell whether he likes it more or less. He studies
every feature, he watches every movement, every gesture; he has a
hundred fleeting interpretations for them all; he would give half his
life if she would but speak. He looks at me anxiously and uneasily; his
eyes are full of questions and reproaches. His every glance seems to
say, ``Guide me while there is yet time; if my heart yields itself and
is deceived, I shall never get over it.''

There is no one in the world less able to conceal his feelings than
Emile. How should he conceal them, in the midst of the greatest
disturbance he has ever experienced, and under the eyes of four
spectators who are all watching him, while she who seems to heed him
least is really most occupied with him. His uneasiness does not escape
the keen eyes of Sophy; his own eyes tell her that she is its cause; she
sees that this uneasiness is not yet love; what matter? He is thinking
of her, and that is enough; she will be very unlucky if he thinks of her
with impunity.

Mothers, like daughters, have eyes; and they have experience too.
Sophy's mother smiles at the success of our schemes. She reads the
hearts of the young people; she sees that the time has come to secure
the heart of this new Telemachus; she makes her daughter speak. Her
daughter, with her native sweetness, replies in a timid tone which makes
all the more impression. At the first sound of her voice, Emile
surrenders; it is Sophy herself; there can be no doubt about it. If it
were not so, it would be too late to deny it.

The charms of this maiden enchantress rush like torrents through his
heart, and he begins to drain the draughts of poison with which he is
intoxicated. He says nothing; questions pass unheeded; he sees only
Sophy, he hears only Sophy; if she says a word, he opens his mouth; if
her eyes are cast down, so are his; if he sees her sigh, he sighs too;
it is Sophy's heart which seems to speak in his. What a change have
these few moments wrought in her heart! It is no longer her turn to
tremble, it is Emile's. Farewell liberty, simplicity, frankness.
Confused, embarrassed, fearful, he dare not look about him for fear he
should see that we are watching him. Ashamed that we should read his
secret, he would fain become invisible to every one, that he might feed
in secret on the sight of Sophy. Sophy, on the other hand, regains her
confidence at the sight of Emile's fear; she sees her triumph and
rejoices in it.
\aquote{``No'l mostra gia, ben che in suo cor ne rida.''}{Tasso, Jerus. Del., c. iv. v. 33.}
Her expression remains unchanged; but in spite of her modest look and
downcast eyes, her tender heart is throbbing with joy, and it tells her
that she has found Telemachus.

If I relate the plain and simple tale of their innocent affections you
will accuse me of frivolity, but you will be mistaken. Sufficient
attention is not given to the effect which the first connection between
man and woman is bound to produce on the future life of both. People do
not see that a first impression so vivid as that of love, or the liking
which takes the place of love, produces lasting effects whose influence
continues till death. Works on education are crammed with wordy and
unnecessary accounts of the imaginary duties of children; but there is
not a word about the most important and most difficult part of their
education, the crisis which forms the bridge between the child and the
man. If any part of this work is really useful, it will be because I
have dwelt at great length on this matter, so essential in itself and so
neglected by other authors, and because I have not allowed myself to be
discouraged either by false delicacy or by the difficulties of
expression. The story of human nature is a fair romance. Am I to blame
if it is not found elsewhere? I am trying to write the history of
mankind. If my book is a romance, the fault lies with those who deprave
mankind.

This is supported by another reason; we are not dealing with a youth
given over from childhood to fear, greed, envy, pride, and all those
passions which are the common tools of the schoolmaster; we have to do
with a youth who is not only in love for the first time, but with one
who is also experiencing his first passion of any kind; very likely it
will be the only strong passion he will ever know, and upon it depends
the final formation of his character. His mode of thought, his feelings,
his tastes, determined by a lasting passion, are about to become so
fixed that they will be incapable of further change.

You will easily understand that Emile and I do not spend the whole of
the night which follows after such an evening in sleep. Why! Do you mean
to tell me that a wise man should be so much affected by a mere
coincidence of name! Is there only one Sophy in the world? Are they all
alike in heart and in name? Is every Sophy he meets his Sophy? Is he mad
to fall in love with a person of whom he knows so little, with whom he
has scarcely exchanged a couple of words? Wait, young man; examine,
observe. You do not even know who our hosts may be, and to hear you talk
one would think the house was your own.

This is no time for teaching, and what I say will receive scant
attention. It only serves to stimulate Emile to further interest in
Sophy, through his desire to find reasons for his fancy. The unexpected
coincidence in the name, the meeting which, so far as he knows, was
quite accidental, my very caution itself, only serve as fuel to the
fire. He is so convinced already of Sophy's excellence, that he feels
sure he can make me fond of her.

Next morning I have no doubt Emile will make himself as smart as his old
travelling suit permits. I am not mistaken; but I am amused to see how
eager he is to wear the clean linen put out for us. I know his thoughts,
and I am delighted to see that he is trying to establish a means of
intercourse, through the return and exchange of the linen; so that he
may have a right to return it and so pay another visit to the house.

I expected to find Sophy rather more carefully dressed too; but I was
mistaken. Such common coquetry is all very well for those who merely
desire to please. The coquetry of true love is a more delicate matter;
it has quite another end in view. Sophy is dressed, if possible, more
simply than last night, though as usual her frock is exquisitely clean.
The only sign of coquetry is her self-consciousness. She knows that an
elaborate toilet is a sign of love, but she does not know that a
careless toilet is another of its signs; it shows a desire to be like
not merely for one's clothes but for oneself. What does a lover care for
her clothes if he knows she is thinking of him? Sophy is already sure of
her power over Emile, and she is not content to delight his eyes if his
heart is not hers also; he must not only perceive her charms, he must
divine them; has he not seen enough to guess the rest?

We may take it for granted that while Emile and I were talking last
night, Sophy and her mother were not silent; a confession was made and
instructions given. The morning's meeting is not unprepared. Twelve
hours ago our young people had never met; they have never said a word to
each other; but it is clear that there is already an understanding
between them. Their greeting is formal, confused, timid; they say
nothing, their downcast eyes seem to avoid each other, but that is in
itself a sign that they understand, they avoid each other with one
consent; they already feel the need of concealment, though not a word
has been uttered. When we depart we ask leave to come again to return
the borrowed clothes in person, Emile's words are addressed to the
father and mother, but his eyes seek Sophy's, and his looks are more
eloquent than his words. Sophy says nothing by word or gesture; she
seems deaf and blind, but she blushes, and that blush is an answer even
plainer than that of her parents.

We receive permission to come again, though we are not invited to stay.
This is only fitting; you offer shelter to benighted travellers, but a
lover does not sleep in the house of his mistress.

We have hardly left the beloved abode before Emile is thinking of taking
rooms in the neighbourhood; the nearest cottage seems too far; he would
like to sleep in the next ditch. ``You young fool!'' I said in a tone of
pity, ``are you already blinded by passion? Have you no regard for
manners or for reason? Wretched youth, you call yourself a lover and you
would bring disgrace upon her you love! What would people say of her if
they knew that a young man who has been staying at her house was
sleeping close by? You say you love her! Would you ruin her reputation?
Is that the price you offer for her parents' hospitality? Would you
bring disgrace on her who will one day make you the happiest of men?''
``Why should we trouble ourselves about the empty words and unjust
suspicions of other people?'' said he eagerly. ``Have you not taught me
yourself to make light of them? Who knows better than I how greatly I
honour Sophy, what respect I desire to show her? My attachment will not
cause her shame, it will be her glory, it shall be worthy of her. If my
heart and my actions continually give her the homage she deserves, what
harm can I do her?'' ``Dear Emile,'' I said, as I clasped him to my
heart, ``you are thinking of yourself alone; learn to think for her too.
Do not compare the honour of one sex with that of the other, they rest
on different foundations. These foundations are equally firm and right,
because they are both laid by nature, and that same virtue which makes
you scorn what men say about yourself, binds you to respect what they
say of her you love. Your honour is in your own keeping, her honour
depends on others. To neglect it is to wound your own honour, and you
fail in what is due to yourself if you do not give her the respect she
deserves.''

Then while I explain the reasons for this difference, I make him realise
how wrong it would be to pay no attention to it. Who can say if he will
really be Sophy's husband? He does not know how she feels towards him;
her own heart or her parents' will may already have formed other
engagements; he knows nothing of her, perhaps there are none of those
grounds of suitability which make a happy marriage. Is he not aware that
the least breath of scandal with regard to a young girl is an indelible
stain, which not even marriage with him who has caused the scandal can
efface? What man of feeling would ruin the woman he loves? What man of
honour would desire that a miserable woman should for ever lament the
misfortune of having found favour in his eyes?

Always prone to extremes, the youth takes alarm at the consequences
which I have compelled him to consider, and now he thinks that he cannot
be too far from Sophy's home; he hastens his steps to get further from
it; he glances round to make sure that no one is listening; he would
sacrifice his own happiness a thousand times to the honour of her whom
he loves; he would rather never see her again than cause her the least
unpleasantness. This is the first result of the pains I have taken ever
since he was a child to make him capable of affection.

We must therefore seek a lodging at a distance, but not too far. We look
about us, we make inquiries; we find that there is a town at least two
leagues away. We try and find lodgings in this town, rather than in the
nearer villages, where our presence might give rise to suspicion. It is
there that the new lover takes up his abode, full of love, hope, joy,
above all full of right feeling. In this way, I guide his rising passion
towards all that is honourable and good, so that his inclinations
unconsciously follow the same bent.

My course is drawing to a close; the end is in view. All the chief
difficulties are vanquished, the chief obstacles overcome; the hardest
thing left to do is to refrain from spoiling my work by undue haste to
complete it. Amid the uncertainty of human life, let us shun that false
prudence which seeks to sacrifice the present to the future; what is, is
too often sacrificed to what will never be. Let us make man happy at
every age lest in spite of our care he should die without knowing the
meaning of happiness. Now if there is a time to enjoy life, it is
undoubtedly the close of adolescence, when the powers of mind and body
have reached their greatest strength, and when man in the midst of his
course is furthest from those two extremes which tell him ``Life is
short.'' If the imprudence of youth deceives itself it is not in its
desire for enjoyment, but because it seeks enjoyment where it is not to
be found, and lays up misery for the future, while unable to enjoy the
present.

Consider my Emile over twenty years of age, well formed, well developed
in mind and body, strong, healthy, active, skilful, robust, full of
sense, reason, kindness, humanity, possessed of good morals and good
taste, loving what is beautiful, doing what is good, free from the sway
of fierce passions, released from the tyranny of popular prejudices, but
subject to the law of wisdom, and easily guided by the voice of a
friend; gifted with so many useful and pleasant accomplishments, caring
little for wealth, able to earn a living with his own hands, and not
afraid of want, whatever may come. Behold him in the intoxication of a
growing passion; his heart opens to the first beams of love; its
pleasant fancies reveal to him a whole world of new delights and
enjoyments; he loves a sweet woman, whose character is even more
delightful than her person; he hopes, he expects the reward which he
deserves.

Their first attachment took its rise in mutual affection, in community
of honourable feelings; therefore this affection is lasting. It abandons
itself, with confidence, with reason, to the most delightful madness,
without fear, regret, remorse, or any other disturbing thought, but that
which is inseparable from all happiness. What lacks there yet? Behold,
inquire, imagine what still is lacking, that can be combined with
present joys. Every happiness which can exist in combination is already
present; nothing could be added without taking away from what there is;
he is as happy as man can be. Shall I choose this time to cut short so
sweet a period? Shall I disturb such pure enjoyment? The happiness he
enjoys is my life's reward. What could I give that could outweigh what I
should take away? Even if I set the crown to his happiness I should
destroy its greatest charm. That supreme joy is a hundredfold greater in
anticipation than in possession; its savour is greater while we wait for
it than when it is ours. O worthy Emile! love and be loved! prolong your
enjoyment before it is yours; rejoice in your love and in your
innocence, find your paradise upon earth, while you await your heaven. I
shall not cut short this happy period of life. I will draw out its
enchantments, I will prolong them as far as possible. Alas! it must come
to an end and that soon; but it shall at least linger in your memory,
and you will never repent of its joys.

Emile has not forgotten that we have something to return. As soon as the
things are ready, we take horse and set off at a great pace, for on this
occasion he is anxious to get there. When the heart opens the door to
passion, it becomes conscious of the slow flight of time. If my time has
not been wasted he will not spend his life like this.

Unluckily the road is intricate and the country difficult. We lose our
way; he is the first to notice it, and without losing his temper, and
without grumbling, he devotes his whole attention to discovering the
path; he wanders for a long time before he knows where he is and always
with the same self-control. You think nothing of that; but I think it a
matter of great importance, for I know how eager he is; I see the
results of the care I have taken from his infancy to harden him to
endure the blows of necessity.

We are there at last! Our reception is much simpler and more friendly
than on the previous occasion; we are already old acquaintances. Emile
and Sophy bow shyly and say nothing; what can they say in our presence?
What they wish to say requires no spectators. We walk in the garden; a
well-kept kitchen-garden takes the place of flower-beds, the park is an
orchard full of fine tall fruit trees of every kind, divided by pretty
streams and borders full of flowers. ``What a lovely place!'' exclaims
Emile, still thinking of his Homer, and still full of enthusiasm, ``I
could fancy myself in the garden of Alcinous.'' The daughter wishes she
knew who Alcinous was; her mother asks. ``Alcinous,'' I tell them, ``was
a king of Coreyra. Homer describes his garden and the critics think it
too simple and unadorned. {[}Footnote: ''`When you leave the palace you
enter a vast garden, four acres in extent, walled in on every side,
planted with tall trees in blossom, and yielding pears, pomegranates,
and other goodly fruits, fig-trees with their luscious burden and green
olives. All the year round these fair trees are heavy with fruit; summer
and winter the soft breath of the west wind sways the trees and ripens
the fruit. Pears and apples wither on the branches, the fig on the
fig-tree, and the clusters of grapes on the vine. The inexhaustible
stock bears fresh grapes, some are baked, some are spread out on the
threshing floor to dry, others are made into wine, while flowers, sour
grapes, and those which are beginning to wither are left upon the tree.
At either end is a square garden filled with flowers which bloom
throughout the year, these gardens are adorned by two fountains, one of
these streams waters the garden, the other passes through the palace and
is then taken to a lofty tower in the town to provide drinking water for
its citizens.' Such is the description of the royal garden of Alcinous
in the 7th book of the Odyssey, a garden in which, to the lasting
disgrace of that old dreamer Homer and the princes of his day, there
were neither trellises, statues, cascades, nor bowling-greens.``{]} This
Alcinous had a charming daughter who dreamed the night before her father
received a stranger at his board that she would soon have a husband.''
Sophy, taken unawares, blushed, hung her head, and bit her lips; no one
could be more confused. Her father, who was enjoying her confusion,
added that the young princess bent herself to wash the linen in the
river. ``Do you think,'' said he, ``she would have scorned to touch the
dirty clothes, saying, that they smelt of grease?'' Sophy, touched to
the quick, forgot her natural timidity and defended herself eagerly. Her
papa knew very well all the smaller things would have had no other
laundress if she had been allowed to wash them, and she would gladly
have done more had she been set to do it. {[}Footnote: I own I feel
grateful to Sophy's mother for not letting her spoil such pretty hands
with soap, hands which Emile will kiss so often.{]} Meanwhile she
watched me secretly with such anxiety that I could not suppress a smile,
while I read the terrors of her simple heart which urged her to speak.
Her father was cruel enough to continue this foolish sport, by asking
her, in jest, why she spoke on her own behalf and what had she in common
with the daughter of Alcinous. Trembling and ashamed she dared hardly
breathe or look at us. Charming girl! This is no time for feigning, you
have shown your true feelings in spite of yourself.

To all appearance this little scene is soon forgotten; luckily for
Sophy, Emile, at least, is unaware of it. We continue our walk, the
young people at first keeping close beside us; but they find it hard to
adapt themselves to our slower pace, and presently they are a little in
front of us, they are walking side by side, they begin to talk, and
before long they are a good way ahead. Sophy seems to be listening
quietly, Emile is talking and gesticulating vigorously; they seem to
find their conversation interesting. When we turn homewards a full hour
later, we call them to us and they return slowly enough now, and we can
see they are making good use of their time. Their conversation ceases
suddenly before they come within earshot, and they hurry up to us. Emile
meets us with a frank affectionate expression; his eyes are sparkling
with joy; yet he looks anxiously at Sophy's mother to see how she takes
it. Sophy is not nearly so much at her ease; as she approaches us she
seems covered with confusion at finding herself tete-a-tete with a young
man, though she has met so many other young men frankly enough, and
without being found fault with for it. She runs up to her mother,
somewhat out of breath, and makes some trivial remark, as if to pretend
she had been with her for some time.

From the happy expression of these dear children we see that this
conversation has taken a load off their hearts. They are no less
reticent in their intercourse, but their reticence is less embarrassing,
it is only due to Emile's reverence and Sophy's modesty, to the goodness
of both. Emile ventures to say a few words to her, she ventures to
reply, but she always looks at her mother before she dares to answer.
The most remarkable change is in her attitude towards me. She shows me
the greatest respect, she watches me with interest, she takes pains to
please me; I see that I am honoured with her esteem, and that she is not
indifferent to mine. I understand that Emile has been talking to her
about me; you might say they have been scheming to win me over to their
side; yet it is not so, and Sophy herself is not so easily won. Perhaps
Emile will have more need of my influence with her than of hers with me.
What a charming pair! When I consider that the tender love of my young
friend has brought my name so prominently into his first conversation
with his lady-love, I enjoy the reward of all my trouble; his affection
is a sufficient recompense.

Our visit is repeated. There are frequent conversations between the
young people. Emile is madly in love and thinks that his happiness is
within his grasp. Yet he does not succeed in winning any formal avowal
from Sophy; she listens to what he says and answers nothing. Emile knows
how modest she is, and is not surprised at her reticence; he feels sure
that she likes him; he knows that parents decide whom their daughters
shall marry; he supposes that Sophy is awaiting her parents' commands;
he asks her permission to speak to them, and she makes no objection. He
talks to me and I speak on his behalf and in his presence. He is
immensely surprised to hear that Sophy is her own mistress, that his
happiness depends on her alone. He begins to be puzzled by her conduct.
He is less self-confident, he takes alarm, he sees that he has not made
so much progress as he expected, and then it is that his love appeals to
her in the tenderest and most moving language.

Emile is not the sort of man to guess what is the matter; if no one told
him he would never discover it as long as he lived, and Sophy is too
proud to tell him. What she considers obstacles, others would call
advantages. She has not forgotten her parents' teaching. She is poor;
Emile is rich; so much she knows. He must win her esteem; his deserts
must be great indeed to remove this inequality. But how should he
perceive these obstacles? Is Emile aware that he is rich? Has he ever
condescended to inquire? Thank heaven, he has no need of riches, he can
do good without their aid. The good he does comes from his heart, not
his purse. He gives the wretched his time, his care, his affection,
himself; and when he reckons up what he has done, he hardly dares to
mention the money spent on the poor.

As he does not know what to make of his disgrace, he thinks it is his
own fault; for who would venture to accuse the adored one of caprice.
The shame of humiliation adds to the pangs of disappointed love. He no
longer approaches Sophy with that pleasant confidence of his own worth;
he is shy and timid in her presence. He no longer hopes to win her
affections, but to gain her pity. Sometimes he loses patience and is
almost angry with her. Sophy seems to guess his angry feelings and she
looks at him. Her glance is enough to disarm and terrify him; he is more
submissive than he used to be.

Disturbed by this stubborn resistance, this invincible silence, he pours
out his heart to his friend. He shares with him the pangs of a heart
devoured by sorrow; he implores his help and counsel. ``How mysterious
it is, how hard to understand! She takes an interest in me, that I am
sure; far from avoiding me she is pleased to see me; when I come she
shows signs of pleasure, when I go she shows regret; she receives my
attentions kindly, my services seem to give her pleasure, she
condescends to give me her advice and even her commands. Yet she rejects
my requests and my prayers. When I venture to speak of marriage, she
bids me be silent; if I say a word, she leaves me at once. Why on earth
should she wish me to be hers but refuse to be mine? She respects and
loves you, and she will not dare to refuse to listen to you. Speak to
her, make her answer. Come to your friend's help, and put the coping
stone to all you have done for him; do not let him fall a victim to your
care! If you fail to secure his happiness, your own teaching will have
been the cause of his misery.''

I speak to Sophy, and have no difficulty in getting her to confide her
secret to me, a secret which was known to me already. It is not so easy
to get permission to tell Emile; but at last she gives me leave and I
tell him what is the matter. He cannot get over his surprise at this
explanation. He cannot understand this delicacy; he cannot see how a few
pounds more or less can affect his character or his deserts. When I get
him to see their effect on people's prejudices he begins to laugh; he is
so wild with delight that he wants to be off at once to tear up his
title deeds and renounce his money, so as to have the honour of being as
poor as Sophy, and to return worthy to be her husband.

``Why,'' said I, trying to check him, and laughing in my turn at his
impetuosity, "will this young head never grow any older? Having dabbled
all your life in philosophy, will you never learn to reason? Do not you
see that your wild scheme would only make things worse, and Sophy more
obstinate? It is a small superiority to be rather richer than she, but
to give up all for her would be a very great superiority; if her pride
cannot bear to be under the small obligation, how will she make up her
mind to the greater? If she cannot bear to think that her husband might
taunt her with the fact that he has enriched her, would she permit him
to blame her for having brought him to poverty? Wretched boy, beware
lest she suspects you of such a plan! On the contrary, be careful and
economical for her sake, lest she should accuse you of trying to gain
her by cunning, by sacrificing of your own free will what you are really
wasting through carelessness.

``Do you really think that she is afraid of wealth, and that she is
opposed to great possessions in themselves? No, dear Emile; there are
more serious and substantial grounds for her opinion, in the effect
produced by wealth on its possessor. She knows that those who are
possessed of fortune's gifts are apt to place them first. The rich
always put wealth before merit. When services are reckoned against
silver, the latter always outweighs the former, and those who have spent
their life in their master's service are considered his debtors for the
very bread they eat. What must you do, Emile, to calm her fears? Let her
get to know you better; that is not done in a day. Show her the
treasures of your heart, to counterbalance the wealth which is
unfortunately yours. Time and constancy will overcome her resistance;
let your great and noble feelings make her forget your wealth. Love her,
serve her, serve her worthy parents. Convince her that these attentions
are not the result of a foolish fleeting passion, but of settled
principles engraved upon your heart. Show them the honour deserved by
worth when exposed to the buffets of Fortune; that is the only way to
reconcile it with that worth which basks in her smiles.''

The transports of joy experienced by the young man at these words may
easily be imagined; they restore confidence and hope, his good heart
rejoices to do something to please Sophy, which he would have done if
there had been no such person, or if he had not been in love with her.
However little his character has been understood, anybody can see how he
would behave under such circumstances.

Here am I, the confidant of these two young people and the mediator of
their affection. What a fine task for a tutor! So fine that never in all
my life have I stood so high in my own eyes, nor felt so pleased with
myself. Moreover, this duty is not without its charms. I am not
unwelcome in the home; it is my business to see that the lovers behave
themselves; Emile, ever afraid of offending me, was never so docile. The
little lady herself overwhelms me with a kindness which does not deceive
me, and of which I only take my proper share. This is her way of making
up for her severity towards Emile. For his sake she bestows on me a
hundred tender caresses, though she would die rather than bestow them on
him; and he, knowing that I would never stand in his way, is delighted
that I should get on so well with her. If she refuses his arm when we
are out walking, he consoles himself with the thought that she has taken
mine. He makes way for me without a murmur, he clasps my hand, and voice
and look alike whisper, ``My friend, plead for me!'' and his eyes follow
us with interest; he tries to read our feelings in our faces, and to
interpret our conversation by our gestures; he knows that everything we
are saying concerns him. Dear Sophy, how frank and easy you are when you
can talk to Mentor without being overheard by Telemachus. How freely and
delightfully you permit him to read what is passing in your tender
little heart! How delighted you are to show him how you esteem his
pupil! How cunningly and appealingly you allow him to divine still
tenderer sentiments. With what a pretence of anger you dismiss Emile
when his impatience leads him to interrupt you? With what pretty
vexation you reproach his indiscretion when he comes and prevents you
saying something to his credit, or listening to what I say about him, or
finding in my words some new excuse to love him!

Having got so far as to be tolerated as an acknowledged lover, Emile
takes full advantage of his position; he speaks, he urges, he implores,
he demands. Hard words or ill treatment make no difference, provided he
gets a hearing. At length Sophy is persuaded, though with some
difficulty, to assume the authority of a betrothed, to decide what he
shall do, to command instead of to ask, to accept instead of to thank,
to control the frequency and the hours of his visits, to forbid him to
come till such a day or to stay beyond such an hour. This is not done in
play, but in earnest, and if it was hard to induce her to accept these
rights, she uses them so sternly that Emile is often ready to regret
that he gave them to her. But whatever her commands, they are obeyed
without question, and often when at her bidding he is about to leave
her, he glances at me his eyes full of delight, as if to say, ``You see
she has taken possession of me.'' Yet unknown to him, Sophy, with all
her pride, is observing him closely, and she is smiling to herself at
the pride of her slave.

Oh that I had the brush of an Alban or a Raphael to paint their bliss,
or the pen of the divine Milton to describe the pleasures of love and
innocence! Not so; let such hollow arts shrink back before the sacred
truth of nature. In tenderness and pureness of heart let your
imagination freely trace the raptures of these young lovers, who under
the eyes of parents and tutor, abandon themselves to their blissful
illusions; in the intoxication of passion they are advancing step by
step to its consummation; with flowers and garlands they are weaving the
bonds which are to bind them till death do part. I am carried away by
this succession of pictures, I am so happy that I cannot group them in
any sort of order or scheme; any one with a heart in his breast can
paint the charming picture for himself and realise the different
experiences of father, mother, daughter, tutor, and pupil, and the part
played by each and all in the union of the most delightful couple whom
love and virtue have ever led to happiness.

Now that he is really eager to please, Emile begins to feel the value of
the accomplishments he has acquired. Sophy is fond of singing, he sings
with her; he does more, he teaches her music. She is lively and light of
foot, she loves skipping; he dances with her, he perfects and develops
her untrained movements into the steps of the dance. These lessons,
enlivened by the gayest mirth, are quite delightful, they melt the timid
respect of love; a lover may enjoy teaching his betrothed---he has a
right to be her teacher.

There is an old spinet quite out of order. Emile mends and tunes it; he
is a maker and mender of musical instruments as well as a carpenter; it
has always been his rule to learn to do everything he can for himself.
The house is picturesquely situated and he makes several sketches of it,
in some of which Sophy does her share, and she hangs them in her
father's study. The frames are not gilded, nor do they require gilding.
When she sees Emile drawing, she draws too, and improves her own
drawing; she cultivates all her talents, and her grace gives a charm to
all she does. Her father and mother recall the days of their wealth,
when they find themselves surrounded by the works of art which alone
gave value to wealth; the whole house is adorned by love; love alone has
enthroned among them, without cost or effort, the very same pleasures
which were gathered together in former days by dint of toil and money.

As the idolater gives what he loves best to the shrine of the object of
his worship, so the lover is not content to see perfection in his
mistress, he must be ever trying to add to her adornment. She does not
need it for his pleasure, it is he who needs the pleasure of giving, it
is a fresh homage to be rendered to her, a fresh pleasure in the joy of
beholding her. Everything of beauty seems to find its place only as an
accessory to the supreme beauty. It is both touching and amusing to see
Emile eager to teach Sophy everything he knows, without asking whether
she wants to learn it or whether it is suitable for her. He talks about
all sorts of things and explains them to her with boyish eagerness; he
thinks he has only to speak and she will understand; he looks forward to
arguing, and discussing philosophy with her; everything he cannot
display before her is so much useless learning; he is quite ashamed of
knowing more than she.

So he gives her lessons in philosophy, physics, mathematics, history,
and everything else. Sophy is delighted to share his enthusiasm and to
try and profit by it. How pleased Emile is when he can get leave to give
these lessons on his knees before her! He thinks the heavens are open.
Yet this position, more trying to pupil than to teacher, is hardly
favourable to study. It is not easy to know where to look, to avoid
meeting the eyes which follow our own, and if they meet so much the
worse for the lesson.

Women are no strangers to the art of thinking, but they should only skim
the surface of logic and metaphysics. Sophy understands readily, but she
soon forgets. She makes most progress in the moral sciences and
aesthetics; as to physical science she retains some vague idea of the
general laws and order of this world. Sometimes in the course of their
walks, the spectacle of the wonders of nature bids them not fear to
raise their pure and innocent hearts to nature's God; they are not
afraid of His presence, and they pour out their hearts before him.

What! Two young lovers spending their time together talking of religion!
Have they nothing better to do than to say their catechism! What profit
is there in the attempt to degrade what is noble? Yes, no doubt they are
saying their catechism in their delightful land of romance; they are
perfect in each other's eyes; they love one another, they talk eagerly
of all that makes virtue worth having. Their sacrifices to virtue make
her all the dearer to them. Their struggles after self-control draw from
them tears purer than the dew of heaven, and these sweet tears are the
joy of life; no human heart has ever experienced a sweeter intoxication.
Their very renunciation adds to their happiness, and their sacrifices
increase their self-respect. Sensual men, bodies without souls, some day
they will know your pleasures, and all their life long they will recall
with regret the happy days when they refused the cup of pleasure.

In spite of this good understanding, differences and even quarrels occur
from time to time; the lady has her whims, the lover has a hot temper;
but these passing showers are soon over and only serve to strengthen
their union. Emile learns by experience not to attach too much
importance to them, he always gains more by the reconciliation than he
lost by the quarrel. The results of the first difference made him expect
a like result from all; he was mistaken, but even if he does not make
any appreciable step forward, he has always the satisfaction of finding
Sophy's genuine concern for his affection more firmly established.
``What advantage is this to him?'' you would ask. I will gladly tell
you; all the more gladly because it will give me an opportunity to
establish clearly a very important principle, and to combat a very
deadly one.

Emile is in love, but he is not presuming; and you will easily
understand that the dignified Sophy is not the sort of girl to allow any
kind of familiarity. Yet virtue has its bounds like everything else, and
she is rather to be blamed for her severity than for indulgence; even
her father himself is sometimes afraid lest her lofty pride should
degenerate into a haughty spirit. When most alone, Emile dare not ask
for the slightest favour, he must not even seem to desire it; and if she
is gracious enough to take his arm when they are out walking, a favour
which she will never permit him to claim as a right, it is only
occasionally that he dare venture with a sigh to press her hand to his
heart. However, after a long period of self-restraint, he ventured
secretly to kiss the hem of her dress, and several times he was lucky
enough to find her willing at least to pretend she was not aware of it.
One day he attempts to take the same privilege rather more openly, and
Sophy takes it into her head to be greatly offended. He persists, she
gets angry and speaks sharply to him; Emile will not put up with this
without reply; the rest of the day is given over to sulks, and they part
in a very ill temper.

Sophy is ill at ease; her mother is her confidant in all things, how can
she keep this from her? It is their first misunderstanding, and the
misunderstanding of an hour is such a serious business. She is sorry for
what she has done, she has her mother's permission and her father's
commands to make reparation.

The next day Emile returns somewhat earlier than usual and in a state of
some anxiety. Sophy is in her mother's dressing-room and her father is
also present. Emile enters respectfully but gloomily. Scarcely have her
parents greeted him than Sophy turns round and holding out her hand asks
him in an affectionate tone how he is. That pretty hand is clearly held
out to be kissed; he takes it but does not kiss it. Sophy, rather
ashamed of herself, withdraws her hand as best she may. Emile, who is
not used to a woman's whims, and does not know how far caprice may be
carried, does not forget so easily or make friends again all at once.
Sophy's father, seeing her confusion, completes her discomfiture by his
jokes. The poor girl, confused and ashamed, does not know what to do
with herself and would gladly have a good cry. The more she tries to
control herself the worse she feels; at last a tear escapes in spite of
all she can do to prevent it. Emile, seeing this tear, rushes towards
her, falls on his knees, takes her hand and kisses it again and again
with the greatest devotion. ``My word, you are too kind to her,'' says
her father, laughing; ``if I were you, I should deal more severely with
these follies, I should punish the mouth that wronged me.'' Emboldened
by these words, Emile turns a suppliant eye towards her mother, and
thinking she is not unwilling, he tremblingly approaches Sophy's face;
she turns away her head, and to save her mouth she exposes a blushing
cheek. The daring young man is not content with this; there is no great
resistance. What a kiss, if it were not taken under her mother's eyes.
Have a care, Sophy, in your severity; he will be ready enough to try to
kiss your dress if only you will sometimes say ``No.''

After this exemplary punishment, Sophy's father goes about his business,
and her mother makes some excuse for sending her out of the room; then
she speaks to Emile very seriously. ``Sir,'' she says, ``I think a young
man so well born and well bred as yourself, a man of feeling and
character, would never reward with dishonour the confidence reposed in
him by the friendship of this family. I am neither prudish nor over
strict; I know how to make excuses for youthful folly, and what I have
permitted in my own presence is sufficient proof of this. Consult your
friend as to your own duty, he will tell you there is all the difference
in the world between the playful kisses sanctioned by the presence of
father and mother, and the same freedom taken in their absence and in
betrayal of their confidence, a freedom which makes a snare of the very
favours which in the parents' presence were wholly innocent. He will
tell you, sir, that my daughter is only to blame for not having
perceived from the first what she ought never to have permitted; he will
tell you that every favour, taken as such, is a favour, and that it is
unworthy of a man of honour to take advantage of a young girl's
innocence, to usurp in private the same freedom which she may permit in
the presence of others. For good manners teach us what is permitted in
public; but we do not know what a man will permit to himself in private,
if he makes himself the sole judge of his conduct.''

After this well-deserved rebuke, addressed rather to me than to my
pupil, the good mother leaves us, and I am amazed by her rare prudence,
in thinking it a little thing that Emile should kiss her daughter's lips
in her presence, while fearing lest he should venture to kiss her dress
when they are alone. When I consider the folly of worldly maxims,
whereby real purity is continually sacrificed to a show of propriety, I
understand why speech becomes more refined while the heart becomes more
corrupt, and why etiquette is stricter while those who conform to it are
most immoral.

While I am trying to convince Emile's heart with regard to these duties
which I ought to have instilled into him sooner, a new idea occurs to
me, an idea which perhaps does Sophy all the more credit, though I shall
take care not to tell her lover; this so-called pride, for which she has
been censured, is clearly only a very wise precaution to protect her
from herself. Being aware that, unfortunately, her own temperament is
inflammable, she dreads the least spark, and keeps out of reach so far
as she can. Her sternness is due not to pride but to humility. She
assumes a control over Emile because she doubts her control of herself;
she turns the one against the other. If she had more confidence in
herself she would be much less haughty. With this exception is there
anywhere on earth a gentler, sweeter girl? Is there any who endures an
affront with greater patience, any who is more afraid of annoying
others? Is there any with less pretension, except in the matter of
virtue? Moreover, she is not proud of her virtue, she is only proud in
order to preserve her virtue, and if she can follow the guidance of her
heart without danger, she caresses her lover himself. But her wise
mother does not confide all this even to her father; men should not hear
everything.

Far from seeming proud of her conquest, Sophy has grown more friendly
and less exacting towards everybody, except perhaps the one person who
has wrought this change. Her noble heart no longer swells with the
feeling of independence. She triumphs modestly over a victory gained at
the price of her freedom. Her bearing is more restrained, her speech
more timid, since she has begun to blush at the word ``lover''; but
contentment may be seen beneath her outward confusion and this very
shame is not painful. This change is most noticeable in her behaviour
towards the young men she meets. Now that she has ceased to be afraid of
them, much of her extreme reserve has disappeared. Now that her choice
is made, she does not hesitate to be gracious to those to whom she is
quite indifferent; taking no more interest in them, she is less
difficult to please, and she always finds them pleasant enough for
people who are of no importance to her.

If true love were capable of coquetry, I should fancy I saw traces of it
in the way Sophy behaves towards other young men in her lover's
presence. One would say that not content with the ardent passion she
inspires by a mixture of shyness and caresses, she is not sorry to rouse
this passion by a little anxiety; one would say that when she is
purposely amusing her young guests she means to torment Emile by the
charms of a freedom she will not allow herself with him; but Sophy is
too considerate, too kindly, too wise to really torment him. Love and
honour take the place of prudence and control the use of this dangerous
weapon. She can alarm and reassure him just as he needs it; and if she
sometimes makes him uneasy she never really gives him pain. The anxiety
she causes to her beloved may be forgiven because of her fear that he is
not sufficiently her own.

But what effect will this little performance have upon Emile? Will he be
jealous or not? That is what we must discover; for such digressions form
part of the purpose of my book, and they do not lead me far from my main
subject.

I have already shown how this passion of jealousy in matters of
convention finds its way into the heart of man. In love it is another
matter; then jealousy is so near akin to nature, that it is hard to
believe that it is not her work; and the example of the very beasts,
many of whom are madly jealous, seems to prove this point beyond reply.
Is it man's influence that has taught cooks to tear each other to pieces
or bulls to fight to the death?

No one can deny that the aversion to everything which may disturb or
interfere with our pleasures is a natural impulse. Up to a certain point
the desire for the exclusive possession of that which ministers to our
pleasure is in the same case. But when this desire has become a passion,
when it is transformed into madness, or into a bitter and suspicious
fancy known as jealousy, that is quite another matter; such a passion
may be natural or it may not; we must distinguish between these
different cases.

I have already analysed the example of the animal world in my Discourse
on Inequality, and on further consideration I think I may refer my
readers to that analysis as sufficiently thorough. I will only add this
further point to those already made in that work, that the jealousy
which springs from nature depends greatly on sexual power, and that when
sexual power is or appears to be boundless, that jealousy is at its
height; for then the male, measuring his rights by his needs, can never
see another male except as an unwelcome rival. In such species the
females always submit to the first comer, they only belong to the male
by right of conquest, and they are the cause of unending strife.

Among the monogamous species, where intercourse seems to give rise to
some sort of moral bond, a kind of marriage, the female who belongs by
choice to the male on whom she has bestowed herself usually denies
herself to all others; and the male, having this preference of affection
as a pledge of her fidelity, is less uneasy at the sight of other males
and lives more peaceably with them. Among these species the male shares
the care of the little ones; and by one of those touching laws of nature
it seems as if the female rewards the father for his love for his
children.

Now consider the human species in its primitive simplicity; it is easy
to see, from the limited powers of the male, and the moderation of his
desires, that nature meant him to be content with one female; this is
confirmed by the numerical equality of the two sexes, at any rate in our
part of the world; an equality which does not exist in anything like the
same degree among those species in which several females are collected
around one male. Though a man does not brood like a pigeon, and though
he has no milk to suckle the young, and must in this respect be classed
with the quadrupeds, his children are feeble and helpless for so long a
time, that mother and children could ill dispense with the father's
affection, and the care which results from it.

All these observations combine to prove that the jealous fury of the
males of certain animals proves nothing with regard to man; and the
exceptional case of those southern regions were polygamy is the
established custom, only confirms the rule, since it is the plurality of
wives that gives rise to the tyrannical precautions of the husband, and
the consciousness of his own weakness makes the man resort to constraint
to evade the laws of nature.

Among ourselves where these same laws are less frequently evaded in this
respect, but are more frequently evaded in another and even more
detestable manner, jealousy finds its motives in the passions of society
rather than in those of primitive instinct. In most irregular
connections the hatred of the lover for his rivals far exceeds his love
for his mistress; if he fears a rival in her affections it is the effect
of that self-love whose origin I have already traced out, and he is
moved by vanity rather than affection. Moreover, our clumsy systems of
education have made women so deceitful, {[}Footnote: The kind of deceit
referred to here is just the opposite of that deceit becoming in a
woman, and taught her by nature; the latter consists in concealing her
real feelings, the former in feigning what she does not feel. Every
society lady spends her life in boasting of her supposed sensibility,
when in reality she cares for no one but herself.{]} and have so
over-stimulated their appetites, that you cannot rely even on the most
clearly proved affection; they can no longer display a preference which
secures you against the fear of a rival.

True love is another matter. I have shown, in the work already referred
to, that this sentiment is not so natural as men think, and that there
is a great difference between the gentle habit which binds a man with
cords of love to his helpmeet, and the unbridled passion which is
intoxicated by the fancied charms of an object which he no longer sees
in its true light. This passion which is full of exclusions and
preferences, only differs from vanity in this respect, that vanity
demands all and gives nothing, so that it is always harmful, while love,
bestowing as much as it demands, is in itself a sentiment full of
equity. Moreover, the more exacting it is, the more credulous; that very
illusion which gave rise to it, makes it easy to persuade. If love is
suspicious, esteem is trustful; and love will never exist in an honest
heart without esteem, for every one loves in another the qualities which
he himself holds in honour.

When once this is clearly understood, we can predict with confidence the
kind of jealousy which Emile will be capable of experiencing; as there
is only the smallest germ of this passion in the human heart, the form
it takes must depend solely upon education: Emile, full of love and
jealousy, will not be angry, sullen, suspicious, but delicate,
sensitive, and timid; he will be more alarmed than vexed; he will think
more of securing his lady-love than of threatening his rival; he will
treat him as an obstacle to be removed if possible from his path, rather
than as a rival to be hated; if he hates him, it is not because he
presumes to compete with him for Sophy's affection, but because Emile
feels that there is a real danger of losing that affection; he will not
be so unjust and foolish as to take offence at the rivalry itself; he
understands that the law of preference rests upon merit only, and that
honour depends upon success; he will redouble his efforts to make
himself acceptable, and he will probably succeed. His generous Sophy,
though she has given alarm to his love, is well able to allay that fear,
to atone for it; and the rivals who were only suffered to put him to the
proof are speedily dismissed.

But whither am I going? O Emile! what art thou now? Is this my pupil?
How art thou fallen! Where is that young man so sternly fashioned, who
braved all weathers, who devoted his body to the hardest tasks and his
soul to the laws of wisdom; untouched by prejudice or passion, a lover
of truth, swayed by reason only, unheeding all that was not hers? Living
in softness and idleness he now lets himself be ruled by women; their
amusements are the business of his life, their wishes are his laws; a
young girl is the arbiter of his fate, he cringes and grovels before
her; the earnest Emile is the plaything of a child.

So shift the scenes of life; each age is swayed by its own motives, but
the man is the same. At ten his mind was set upon cakes, at twenty it is
set upon his mistress; at thirty it will be set upon pleasure; at forty
on ambition, at fifty on avarice; when will he seek after wisdom only?
Happy is he who is compelled to follow her against his will! What matter
who is the guide, if the end is attained. Heroes and sages have
themselves paid tribute to this human weakness; and those who handled
the distaff with clumsy fingers were none the less great men.

If you would prolong the influence of a good education through life
itself, the good habits acquired in childhood must be carried forward
into adolescence, and when your pupil is what he ought to be you must
manage to keep him what he ought to be. This is the coping-stone of your
work. This is why it is of the first importance that the tutor should
remain with young men; otherwise there is little doubt they will learn
to make love without him. The great mistake of tutors and still more of
fathers is to think that one way of living makes another impossible, and
that as soon as the child is grown up, you must abandon everything you
used to do when he was little. If that were so, why should we take such
pains in childhood, since the good or bad use we make of it will vanish
with childhood itself; if another way of life were necessarily
accompanied by other ways of thinking?

The stream of memory is only interrupted by great illnesses, and the
stream of conduct, by great passions. Our tastes and inclinations may
change, but this change, though it may be sudden enough, is rendered
less abrupt by our habits. The skilful artist, in a good colour scheme,
contrives so to mingle and blend his tints that the transitions are
imperceptible; and certain colour washes are spread over the whole
picture so that there may be no sudden breaks. So should it be with our
likings. Unbalanced characters are always changing their affections,
their tastes, their sentiments; the only constant factor is the habit of
change; but the man of settled character always returns to his former
habits and preserves to old age the tastes and the pleasures of his
childhood.

If you contrive that young people passing from one stage of life to
another do not despise what has gone before, that when they form new
habits, they do not forsake the old, and that they always love to do
what is right, in things new and old; then only are the fruits of your
toil secure, and you are sure of your scholars as long as they live; for
the revolution most to be dreaded is that of the age over which you are
now watching. As men always look back to this period with regret so the
tastes carried forward into it from childhood are not easily destroyed;
but if once interrupted they are never resumed.

Most of the habits you think you have instilled into children and young
people are not really habits at all; they have only been acquired under
compulsion, and being followed reluctantly they will be cast off at the
first opportunity. However long you remain in prison you never get a
taste for prison life; so aversion is increased rather than diminished
by habit. Not so with Emile; as a child he only did what he could do
willingly and with pleasure, and as a man he will do the same, and the
force of habit will only lend its help to the joys of freedom. An active
life, bodily labour, exercise, movement, have become so essential to him
that he could not relinquish them without suffering. Reduce him all at
once to a soft and sedentary life and you condemn him to chains and
imprisonment, you keep him in a condition of thraldom and constraint; he
would suffer, no doubt, both in health and temper. He can scarcely
breathe in a stuffy room, he requires open air, movement, fatigue. Even
at Sophy's feet he cannot help casting a glance at the country and
longing to explore it in her company. Yet he remains if he must; but he
is anxious and ill at ease; he seems to be struggling with himself; he
remains because he is a captive. ``Yes,'' you will say, ``these are
necessities to which you have subjected him, a yoke which you have laid
upon him.'' You speak truly, I have subjected him to the yoke of
manhood.

Emile loves Sophy; but what were the charms by which he was first
attracted? Sensibility, virtue, and love for things pure and honest.
When he loves this love in Sophy, will he cease to feel it himself? And
what price did she put upon herself? She required all her lover's
natural feelings---esteem of what is really good, frugality, simplicity,
generous unselfishness, a scorn of pomp and riches. These virtues were
Emile's before love claimed them of him. Is he really changed? He has
all the more reason to be himself; that is the only difference. The
careful reader will not suppose that all the circumstances in which he
is placed are the work of chance. There were many charming girls in the
town; is it chance that his choice is discovered in a distant retreat?
Is their meeting the work of chance? Is it chance that makes them so
suited to each other? Is it chance that they cannot live in the same
place, that he is compelled to find a lodging so far from her? Is it
chance that he can see her so seldom and must purchase the pleasure of
seeing her at the price of such fatigue? You say he is becoming
effeminate. Not so, he is growing stronger; he must be fairly robust to
stand the fatigue he endures on Sophy's account.

He lives more than two leagues away. That distance serves to temper the
shafts of love. If they lived next door to each other, or if he could
drive to see her in a comfortable carriage, he would love at his ease in
the Paris fashion. Would Leander have braved death for the sake of Hero
if the sea had not lain between them? Need I say more; if my reader is
able to take my meaning, he will be able to follow out my principles in
detail.

The first time we went to see Sophy, we went on horseback, so as to get
there more quickly. We continue this convenient plan until our fifth
visit. We were expected; and more than half a league from the house we
see people on the road. Emile watches them, his pulse quickens as he
gets nearer, he recognises Sophy and dismounts quickly; he hastens to
join the charming family. Emile is fond of good horses; his horse is
fresh, he feels he is free, and gallops off across the fields; I follow
and with some difficulty I succeed in catching him and bringing him
back. Unluckily Sophy is afraid of horses, and I dare not approach her.
Emile has not seen what happened, but Sophy whispers to him that he is
giving his friend a great deal of trouble. He hurries up quite ashamed
of himself, takes the horses, and follows after the party. It is only
fair that each should take his turn and he rides on to get rid of our
mounts. He has to leave Sophy behind him, and he no longer thinks riding
a convenient mode of travelling. He returns out of breath and meets us
half-way.

The next time, Emile will not hear of horses. ``Why,'' say I, ``we need
only take a servant to look after them.'' ``Shall we put our worthy
friends to such expense?'' he replies. ``You see they would insist on
feeding man and horse.'' ``That is true,'' I reply; ``theirs is the
generous hospitality of the poor. The rich man in his niggardly pride
only welcomes his friends, but the poor find room for their friends'
horses.'' ``Let us go on foot,'' says he; ``won't you venture on the
walk, when you are always so ready to share the toilsome pleasures of
your child?'' ``I will gladly go with you,'' I reply at once, ``and it
seems to me that love does not desire so much show.''

As we draw near, we meet the mother and daughter even further from home
than on the last occasion. We have come at a great pace. Emile is very
warm; his beloved condescends to pass her handkerchief over his cheeks.
It would take a good many horses to make us ride there after this.

But it is rather hard never to be able to spend an evening together.
Midsummer is long past and the days are growing shorter. Whatever we
say, we are not allowed to return home in the dark, and unless we make a
very early start, we have to go back almost as soon as we get there. The
mother is sorry for us and uneasy on our account, and it occurs to her
that, though it would not be proper for us to stay in the house, beds
might be found for us in the village, if we liked to stay there
occasionally. Emile claps his hands at this idea and trembles with joy;
Sophy, unwittingly, kisses her mother rather oftener than usual on the
day this idea occurs to her.

Little by little the charm of friendship and the familiarity of
innocence take root and grow among us. I generally accompany my young
friend on the days appointed by Sophy or her mother, but sometimes I let
him go alone. The heart thrives in the sunshine of confidence, and a man
must not be treated as a child; and what have I accomplished so far, if
my pupil is unworthy of my esteem? Now and then I go without him; he is
sorry, but he does not complain; what use would it be? And then he knows
I shall not interfere with his interests. However, whether we go
together or separately you will understand that we are not stopped by
the weather; we are only too proud to arrive in a condition which calls
for pity. Unluckily Sophy deprives us of this honour and forbids us to
come in bad weather. This is the only occasion on which she rebels
against the rules which I laid down for her in private.

One day Emile had gone alone and I did not expect him back till the
following day, but he returned the same evening. ``My dear Emile,'' said
I, ``have you come back to your old friend already?'' But instead of
responding to my caresses he replied with some show of temper, ``You
need not suppose I came back so soon of my own accord; she insisted on
it; it is for her sake not yours that I am here.'' Touched by his
frankness I renewed my caresses, saying, ``Truthful heart and faithful
friend, do not conceal from me anything I ought to know. If you came
back for her sake, you told me so for my own; your return is her doing,
your frankness is mine. Continue to preserve the noble candour of great
souls; strangers may think what they will, but it is a crime to let our
friends think us better than we are.''

I take care not to let him underrate the cost of his confession by
assuming that there is more love than generosity in it, and by telling
him that he would rather deprive himself of the honour of this return,
than give it to Sophy. But this is how he revealed to me, all
unconsciously, what were his real feelings; if he had returned slowly
and comfortably, dreaming of his sweetheart, I should know he was merely
her lover; when he hurried back, even if he was a little out of temper,
he was the friend of his Mentor.

You see that the young man is very far from spending his days with
Sophy, and seeing as much of her as he wants. One or two visits a week
are all that is permitted, and these visits are often only for the
afternoon and are rarely extended to the next day. He spends much more
of his time in longing to see her, or in rejoicing that he has seen her,
than he actually spends in her presence. Even when he goes to see her,
more time is spent in going and returning than by her side. His
pleasures, genuine, pure, delicious, but more imaginary than real, serve
to kindle his love but not to make him effeminate.

On the days when he does not see Sophy he is not sitting idle at home.
He is Emile himself and quite unchanged. He usually scours the country
round in pursuit of its natural history; he observes and studies the
soil, its products, and their mode of cultivation; he compares the
methods he sees with those with which he is already familiar; he tries
to find the reasons for any differences; if he thinks other methods
better than those of the locality, he introduces them to the farmers'
notice; if he suggests a better kind of plough, he has one made from his
own drawings; if he finds a lime pit he teaches them how to use the lime
on the land, a process new to them; he often lends a hand himself; they
are surprised to find him handling all manner of tools more easily than
they can themselves; his furrows are deeper and straighter than theirs,
he is a more skilful sower, and his beds for early produce are more
cleverly planned. They do not scoff at him as a fine talker, they see he
knows what he is talking about. In a word, his zeal and attention are
bestowed on everything that is really useful to everybody; nor does he
stop there. He visits the peasants in their homes; inquires into their
circumstances, their families, the number of their children, the extent
of their holdings, the nature of their produce, their markets, their
rights, their burdens, their debts, etc. He gives away very little
money, for he knows it is usually ill spent; but he himself directs the
use of his money, and makes it helpful to them without distributing it
among them. He supplies them with labourers, and often pays them for
work done by themselves, on tasks for their own benefit. For one he has
the falling thatch repaired or renewed; for another he clears a piece of
land which had gone out of cultivation for lack of means; to another he
gives a cow, a horse, or stock of any kind to replace a loss; two
neighbours are ready to go to law, he wins them over, and makes them
friends again; a peasant falls ill, he has him cared for, he looks after
him himself; {[}Footnote: To look after a sick peasant is not merely to
give him a pill, or medicine, or to send a surgeon to him. That is not
what these poor folk require in sickness; what they want is more and
better food. When you have fever, you will do well to fast, but when
your peasants have it, give them meat and wine; illness, in their case,
is nearly always due to poverty and exhaustion; your cellar will supply
the best draught, your butchers will be the best apothecary.{]} another
is harassed by a rich and powerful neighbor, he protects him and speaks
on his behalf; young people are fond of one another, he helps forward
their marriage; a good woman has lost her beloved child, he goes to see
her, he speaks words of comfort and sits a while with her; he does not
despise the poor, he is in no hurry to avoid the unfortunate; he often
takes his dinner with some peasant he is helping, and he will even
accept a meal from those who have no need of his help; though he is the
benefactor of some and the friend of all, he is none the less their
equal. In conclusion, he always does as much good by his personal
efforts as by his money.

Sometimes his steps are turned in the direction of the happy abode; he
may hope to see Sophy without her knowing, to see her out walking
without being seen. But Emile is always quite open in everything he
does; he neither can nor would deceive. His delicacy is of that pleasing
type in which pride rests on the foundation of a good conscience. He
keeps strictly within bounds, and never comes near enough to gain from
chance what he only desires to win from Sophy herself. On the other
hand, he delights to roam about the neighbourhood, looking for the trace
of Sophy's steps, feeling what pains she has taken and what a distance
she has walked to please him.

The day before his visit, he will go to some neighbouring farm and order
a little feast for the morrow. We shall take our walk in that direction
without any special object, we shall turn in apparently by chance;
fruit, cakes, and cream are waiting for us. Sophy likes sweets, so is
not insensible to these attentions, and she is quite ready to do honour
to what we have provided; for I always have my share of the credit even
if I have had no part in the trouble; it is a girl's way of returning
thanks more easily. Her father and I have cakes and wine; Emile keeps
the ladies company and is always on the look-out to secure a dish of
cream in which Sophy has dipped her spoon.

The cakes lead me to talk of the races Emile used to run. Every one
wants to hear about them; I explain amid much laughter; they ask him if
he can run as well as ever. ``Better,'' says he; ``I should be sorry to
forget how to run.'' One member of the company is dying to see him run,
but she dare not say so; some one else undertakes to suggest it; he
agrees and we send for two or three young men of the neighbourhood; a
prize is offered, and in imitation of our earlier games a cake is placed
on the goal. Every one is ready, Sophy's father gives the signal by
clapping his hands. The nimble Emile flies like lightning and reaches
the goal almost before the others have started. He receives his prize at
Sophy's hands, and no less generous than Aeneas, he gives gifts to all
the vanquished.

In the midst of his triumph, Sophy dares to challenge the victor, and to
assert that she can run as fast as he. He does not refuse to enter the
lists with her, and while she is getting ready to start, while she is
tucking up her skirt at each side, more eager to show Emile a pretty
ankle than to vanquish him in the race, while she is seeing if her
petticoats are short enough, he whispers a word to her mother who smiles
and nods approval. Then he takes his place by his competitor; no sooner
is the signal given than she is off like a bird.

Women were not meant to run; they flee that they may be overtaken.
Running is not the only thing they do ill, but it is the only thing they
do awkwardly; their elbows glued to their sides and pointed backwards
look ridiculous, and the high heels on which they are perched make them
look like so many grasshoppers trying to run instead of to jump.

Emile, supposing that Sophy runs no better than other women, does not
deign to stir from his place and watches her start with a smile of
mockery. But Sophy is light of foot and she wears low heels; she needs
no pretence to make her foot look smaller; she runs so quickly that he
has only just time to overtake this new Atalanta when he sees her so far
ahead. Then he starts like an eagle dashing upon its prey; he pursues
her, clutches her, grasps her at last quite out of breath, and gently
placing his left arm about her, he lifts her like a feather, and
pressing his sweet burden to his heart, he finishes the race, makes her
touch the goal first, and then exclaiming, ``Sophy wins!'' he sinks on
one knee before her and owns himself beaten.

Along with such occupations there is also the trade we learnt. One day a
week at least, and every day when the weather is too bad for country
pursuits, Emile and I go to work under a master-joiner. We do not work
for show, like people above our trade; we work in earnest like regular
workmen. Once when Sophy's father came to see us, he found us at work,
and did not fail to report his wonder to his wife and daughter. ``Go and
see that young man in the workshop,'' said he, ``and you will soon see
if he despises the condition of the poor.'' You may fancy how pleased
Sophy was at this! They talk it over, and they decide to surprise him at
his work. They question me, apparently without any special object, and
having made sure of the time, mother and daughter take a little carriage
and come to town on that very day.

On her arrival, Sophy sees, at the other end of the shop, a young man in
his shirt sleeves, with his hair all untidy, so hard at work that he
does not see her; she makes a sign to her mother. Emile, a chisel in one
hand and a hammer in the other, is just finishing a mortise; then he
saws a piece of wood and places it in the vice in order to polish it.
The sight of this does not set Sophy laughing; it affects her greatly;
it wins her respect. Woman, honour your master; he it is who works for
you, he it is who gives you bread to eat; this is he!

While they are busy watching him, I perceive them and pull Emile by the
sleeve; he turns round, drops his tools, and hastens to them with an
exclamation of delight. After he has given way to his first raptures, he
makes them take a seat and he goes back to his work. But Sophy cannot
keep quiet; she gets up hastily, runs about the workshop, looks at the
tools, feels the polish of the boards, picks up shavings, looks at our
hands, and says she likes this trade, it is so clean. The merry girl
tries to copy Emile. With her delicate white hand she passes a plane
over a bit of wood; the plane slips and makes no impression. It seems to
me that Love himself is hovering over us and beating his wings; I think
I can hear his joyous cries, ``Hercules is avenged.''

Yet Sophy's mother questions the master. ``Sir, how much do you pay
these two men a day?'' ``I give them each tenpence a day and their food;
but if that young fellow wanted he could earn much more, for he is the
best workman in the country.'' ``Tenpence a day and their food,'' said
she looking at us tenderly. ``That is so, madam,'' replied the master.
At these words she hurries up to Emile, kisses him, and clasps him to
her breast with tears; unable to say more she repeats again and again,
``My son, my son!''

When they had spent some time chatting with us, but without interrupting
our work, ``We must be going now,'' said the mother to her daughter,
``it is getting late and we must not keep your father waiting.'' Then
approaching Emile she tapped him playfully on the cheek, saying, ``Well,
my good workman, won't you come with us?'' He replied sadly, ``I am at
work, ask the master.'' The master is asked if he can spare us. He
replies that he cannot. ``I have work on hand,'' said he, ``which is
wanted the day after to-morrow, so there is not much time. Counting on
these gentlemen I refused other workmen who came; if they fail me I
don't know how to replace them and I shall not be able to send the work
home at the time promised.'' The mother said nothing, she was waiting to
hear what Emile would say. Emile hung his head in silence. ``Sir,'' she
said, somewhat surprised at this, ``have you nothing to say to that?''
Emile looked tenderly at her daughter and merely said, ``You see I am
bound to stay.'' Then the ladies left us. Emile went with them to the
door, gazed after them as long as they were in sight, and returned to
his work without a word.

On the way home, the mother, somewhat vexed at his conduct, spoke to her
daughter of the strange way in which he had behaved. ``Why,'' said she,
``was it so difficult to arrange matters with the master without being
obliged to stay. The young man is generous enough and ready to spend
money when there is no need for it, could not he spend a little on such
a fitting occasion?'' ``Oh, mamma,'' replied Sophy, ``I trust Emile will
never rely so much on money as to use it to break an engagement, to fail
to keep his own word, and to make another break his! I know he could
easily give the master a trifle to make up for the slight inconvenience
caused by his absence; but his soul would become the slave of riches, he
would become accustomed to place wealth before duty, and he would think
that any duty might be neglected provided he was ready to pay. That is
not Emile's way of thinking, and I hope he will never change on my
account. Do you think it cost him nothing to stay? You are quite wrong,
mamma; it was for my sake that he stayed; I saw it in his eyes.''

It is not that Sophy is indifferent to genuine proofs of love; on the
contrary she is imperious and exacting; she would rather not be loved at
all than be loved half-heartedly. Hers is the noble pride of worth,
conscious of its own value, self-respecting and claiming a like honour
from others. She would scorn a heart that did not recognise the full
worth of her own; that did not love her for her virtues as much and more
than for her charms; a heart which did not put duty first, and prefer it
to everything. She did not desire a lover who knew no will but hers. She
wished to reign over a man whom she had not spoilt. Thus Circe, having
changed into swine the comrades of Ulysses, bestowed herself on him over
whom she had no power.

Except for this sacred and inviolable right, Sophy is very jealous of
her own rights; she observes how carefully Emile respects them, how
zealously he does her will; how cleverly he guesses her wishes, how
exactly he arrives at the appointed time; she will have him neither late
nor early; he must arrive to the moment. To come early is to think more
of himself than of her; to come late is to neglect her. To neglect
Sophy, that could not happen twice. An unfounded suspicion on her part
nearly ruined everything, but Sophy is really just and knows how to
atone for her faults.

They were expecting us one evening; Emile had received his orders. They
came to meet us, but we were not there. What has become of us? What
accident have we met with? No message from us! The evening is spent in
expectation of our arrival. Sophy thinks we are dead; she is miserable
and in an agony of distress; she cries all the night through. In the
course of the evening a messenger was despatched to inquire after us and
bring back news in the morning. The messenger returns together with
another messenger sent by us, who makes our excuses verbally and says we
are quite well. Then the scene is changed; Sophy dries her tears, or if
she still weeps it is for anger. It is small consolation to her proud
spirit to know that we are alive; Emile lives and he has kept her
waiting.

When we arrive she tries to escape to her own room; her parents desire
her to remain, so she is obliged to do so; but deciding at once what
course she will take she assumes a calm and contented expression which
would deceive most people. Her father comes forward to receive us
saying, ``You have made your friends very uneasy; there are people here
who will not forgive you very readily.'' ``Who are they, papa,'' said
Sophy with the most gracious smile she could assume. ``What business is
that of yours,'' said her father, ``if it is not you?'' Sophy bent over
her work without reply. Her mother received us coldly and formally.
Emile was so confused he dared not speak to Sophy. She spoke first,
inquired how he was, asked him to take a chair, and pretended so
cleverly that the poor young fellow, who as yet knew nothing of the
language of angry passions, was quite deceived by her apparent
indifference, and ready to take offence on his own account.

To undeceive him I was going to take Sophy's hand and raise it to my
lips as I sometimes did; she drew it back so hastily, with the word,
``Sir,'' uttered in such a strange manner that Emile's eyes were opened
at once by this involuntary movement.

Sophy herself, seeing that she had betrayed herself, exercised less
control over herself. Her apparent indifference was succeeded by
scornful irony. She replied to everything he said in monosyllables
uttered slowly and hesitatingly as if she were afraid her anger should
show itself too plainly. Emile half dead with terror stared at her full
of sorrow, and tried to get her to look at him so that his eyes might
read in hers her real feelings. Sophy, still more angry at his boldness,
gave him one look which removed all wish for another. Luckily for
himself, Emile, trembling and dumbfounded, dared neither look at her nor
speak to her again; for had he not been guilty, had he been able to
endure her wrath, she would never have forgiven him.

Seeing that it was my turn now, and that the time was ripe for
explanation, I returned to Sophy. I took her hand and this time she did
not snatch it away; she was ready to faint. I said gently, ``Dear Sophy,
we are the victims of misfortune; but you are just and reasonable; you
will not judge us unheard; listen to what we have to say.'' She said
nothing and I proceeded---

"We set out yesterday at four o'clock; we were told to be here at seven,
and we always allow ourselves rather more time than we need, so as to
rest a little before we get here. We were more than half way here when
we heard lamentable groans, which came from a little valley in the
hillside, some distance off. We hurried towards the place and found an
unlucky peasant who had taken rather more wine than was good for him; on
his way home he had fallen heavily from his horse and broken his leg. We
shouted and called for help; there was no answer; we tried to lift the
injured man on his horse, but without success; the least movement caused
intense agony. We decided to tie up the horse in a quiet part of the
wood; then we made a chair of our crossed arms and carried the man as
gently as possible, following his directions till we got him home. The
way was long, and we were constantly obliged to stop and rest. At last
we got there, but thoroughly exhausted. We were surprised and sorry to
find that it was a house we knew already and that the wretched creature
we had carried with such difficulty was the very man who received us so
kindly when first we came. We had all been so upset that until that
moment we had not recognised each other.

"There were only two little children. His wife was about to present him
with another, and she was so overwhelmed at the sight of him brought
home in such a condition, that she was taken ill and a few hours later
gave birth to another little one. What was to be done under such
circumstances in a lonely cottage far from any help? Emile decided to
fetch the horse we had left in the wood, to ride as fast as he could
into the town and fetch a surgeon. He let the surgeon have the horse,
and not succeeding in finding a nurse all at once, he returned on foot
with a servant, after having sent a messenger to you; meanwhile I hardly
knew what to do between a man with a broken leg and a woman in travail,
but I got ready as well as I could such things in the house as I thought
would be needed for the relief of both.

``I will pass over the rest of the details; they are not to the point.
It was two o'clock in the morning before we got a moment's rest. At last
we returned before daybreak to our lodging close at hand, where we
waited till you were up to let you know what had happened to us.''

That was all I said. But before any one could speak Emile, approaching
Sophy, raised his voice and said with greater firmness than I expected,
``Sophy, my fate is in your hands, as you very well know. You may
condemn me to die of grief; but do not hope to make me forget the rights
of humanity; they are even more sacred in my eyes than your own rights;
I will never renounce them for you.''

For all answer, Sophy rose, put her arm round his neck, and kissed him
on the cheek; then offering him her hand with inimitable grace she said
to him, ``Emile, take this hand; it is yours. When you will, you shall
be my husband and my master; I will try to be worthy of that honour.''

Scarcely had she kissed him, when her delighted father clapped his hands
calling, ``Encore, encore,'' and Sophy without further ado, kissed him
twice on the other cheek; but afraid of what she had done she took
refuge at once in her mother's arms and hid her blushing face on the
maternal bosom.

I will not describe our happiness; everybody will feel with us. After
dinner Sophy asked if it were too far to go and see the poor invalids.
It was her wish and it was a work of mercy. When we got there we found
them both in bed---Emile had sent for a second bedstead; there were
people there to look after them---Emile had seen to it. But in spite of
this everything was so untidy that they suffered almost as much from
discomfort as from their condition. Sophy asked for one of the good
wife's aprons and set to work to make her more comfortable in her bed;
then she did as much for the man; her soft and gentle hand seemed to
find out what was hurting them and how to settle them into less painful
positions. Her very presence seemed to make them more comfortable; she
seemed to guess what was the matter. This fastidious girl was not
disgusted by the dirt or smells, and she managed to get rid of both
without disturbing the sick people. She who had always appeared so
modest and sometimes so disdainful, she who would not for all the world
have touched a man's bed with her little finger, lifted the sick man and
changed his linen without any fuss, and placed him to rest in a more
comfortable position. The zeal of charity is of more value than modesty.
What she did was done so skilfully and with such a light touch that he
felt better almost without knowing she had touched him. Husband and wife
mingled their blessings upon the kindly girl who tended, pitied, and
consoled them. She was an angel from heaven come to visit them; she was
an angel in face and manner, in gentleness and goodness. Emile was
greatly touched by all this and he watched her without speaking. O man,
love thy helpmeet. God gave her to relieve thy sufferings, to comfort
thee in thy troubles. This is she!

The new-born baby was baptised. The two lovers were its god-parents, and
as they held it at the font they were longing, at the bottom of their
hearts, for the time when they should have a child of their own to be
baptised. They longed for their wedding day; they thought it was close
at hand; all Sophy's scruples had vanished, but mine remained. They had
not got so far as they expected; every one must have his turn.

One morning when they had not seen each other for two whole days, I
entered Emile's room with a letter in my hands, and looking fixedly at
him I said to him, ``What would you do if some one told you Sophy were
dead?'' He uttered a loud cry, got up and struck his hands together, and
without saying a single word, he looked at me with eyes of desperation.
``Answer me,'' I continued with the same calmness. Vexed at my
composure, he then approached me with eyes blazing with anger; and
checking himself in an almost threatening attitude, ``What would I do? I
know not; but this I do know, I would never set eyes again upon the
person who brought me such news.'' ``Comfort yourself,'' said I,
smiling, ``she lives, she is well, and they are expecting us this
evening. But let us go for a short walk and we can talk things over.''

The passion which engrosses him will no longer permit him to devote
himself as in former days to discussions of pure reason; this very
passion must be called to our aid if his attention is to be given to my
teaching. That is why I made use of this terrible preface; I am quite
sure he will listen to me now.

"We must be happy, dear Emile; it is the end of every feeling creature;
it is the first desire taught us by nature, and the only one which never
leaves us. But where is happiness? Who knows? Every one seeks it, and no
one finds it. We spend our lives in the search and we die before the end
is attained. My young friend, when I took you, a new-born infant, in my
arms, and called God himself to witness to the vow I dared to make that
I would devote my life to the happiness of your life, did I know myself
what I was undertaking? No; I only knew that in making you happy, I was
sure of my own happiness. By making this useful inquiry on your account,
I made it for us both.

"So long as we do not know what to do, wisdom consists in doing nothing.
Of all rules there is none so greatly needed by man, and none which he
is less able to obey. In seeking happiness when we know not where it is,
we are perhaps getting further and further from it, we are running as
many risks as there are roads to choose from. But it is not every one
that can keep still. Our passion for our own well-being makes us so
uneasy, that we would rather deceive ourselves in the search for
happiness than sit still and do nothing; and when once we have left the
place where we might have known happiness, we can never return.

"In ignorance like this I tried to avoid a similar fault. When I took
charge of you I decided to take no useless steps and to prevent you from
doing so too. I kept to the path of nature, until she should show me the
path of happiness. And lo! their paths were the same, and without
knowing it this was the path I trod.

"Be at once my witness and my judge; I will never refuse to accept your
decision. Your early years have not been sacrificed to those that were
to follow, you have enjoyed all the good gifts which nature bestowed
upon you. Of the ills to which you were by nature subject, and from
which I could shelter you, you have only experienced such as would
harden you to bear others. You have never suffered any evil, except to
escape a greater. You have known neither hatred nor servitude. Free and
happy, you have remained just and kindly; for suffering and vice are
inseparable, and no man ever became bad until he was unhappy. May the
memory of your childhood remain with you to old age! I am not afraid
that your kind heart will ever recall the hand that trained it without a
blessing upon it.

"When you reached the age of reason, I secured you from the influence of
human prejudice; when your heart awoke I preserved you from the sway of
passion. Had I been able to prolong this inner tranquillity till your
life's end, my work would have been secure, and you would have been as
happy as man can be; but, my dear Emile, in vain did I dip you in the
waters of Styx, I could not make you everywhere invulnerable; a fresh
enemy has appeared, whom you have not yet learnt to conquer, and from
whom I cannot save you. That enemy is yourself. Nature and fortune had
left you free. You could face poverty, you could bear bodily pain; the
sufferings of the heart were unknown to you; you were then dependent on
nothing but your position as a human being; now you depend on all the
ties you have formed for yourself; you have learnt to desire, and you
are now the slave of your desires. Without any change in yourself,
without any insult, any injury to yourself, what sorrows may attack your
soul, what pains may you suffer without sickness, how many deaths may
you die and yet live! A lie, an error, a suspicion, may plunge you in
despair.

"At the theatre you used to see heroes, abandoned to depths of woe,
making the stage re-echo with their wild cries, lamenting like women,
weeping like children, and thus securing the applause of the audience.
Do you remember how shocked you were by those lamentations, cries, and
groans, in men from whom one would only expect deeds of constancy and
heroism. `Why,' said you, `are those the patterns we are to follow, the
models set for our imitation! Are they afraid man will not be small
enough, unhappy enough, weak enough, if his weakness is not enshrined
under a false show of virtue.' My young friend, henceforward you must be
more merciful to the stage; you have become one of those heroes.

"You know how to suffer and to die; you know how to bear the heavy yoke
of necessity in ills of the body, but you have not yet learnt to give a
law to the desires of your heart; and the difficulties of life arise
rather from our affections than from our needs. Our desires are vast,
our strength is little better than nothing. In his wishes man is
dependent on many things; in himself he is dependent on nothing, not
even on his own life; the more his connections are multiplied, the
greater his sufferings. Everything upon earth has an end; sooner or
later all that we love escapes from our fingers, and we behave as if it
would last for ever. What was your terror at the mere suspicion of
Sophy's death? Do you suppose she will live for ever? Do not young
people of her age die? She must die, my son, and perhaps before you. Who
knows if she is alive at this moment? Nature meant you to die but once;
you have prepared a second death for yourself.

"A slave to your unbridled passions, how greatly are you to be pitied!
Ever privations, losses, alarms; you will not even enjoy what is left.
You will possess nothing because of the fear of losing it; you will
never be able to satisfy your passions, because you desired to follow
them continually. You will ever be seeking that which will fly before
you; you will be miserable and you will become wicked. How can you be
otherwise, having no care but your unbridled passions! If you cannot put
up with involuntary privations how will you voluntarily deprive
yourself? How can you sacrifice desire to duty, and resist your heart in
order to listen to your reason? You would never see that man again who
dared to bring you word of the death of your mistress; how would you
behold him who would deprive you of her living self, him who would dare
to tell you, `She is dead to you, virtue puts a gulf between you'? If
you must live with her whatever happens, whether Sophy is married or
single, whether you are free or not, whether she loves or hates you,
whether she is given or refused to you, no matter, it is your will and
you must have her at any price. Tell me then what crime will stop a man
who has no law but his heart's desires, who knows not how to resist his
own passions.

"My son, there is no happiness without courage, nor virtue without a
struggle. The word virtue is derived from a word signifying strength,
and strength is the foundation of all virtue. Virtue is the heritage of
a creature weak by nature but strong by will; that is the whole merit of
the righteous man; and though we call God good we do not call Him
virtuous, because He does good without effort. I waited to explain the
meaning of this word, so often profaned, until you were ready to
understand me. As long as virtue is quite easy to practise, there is
little need to know it. This need arises with the awakening of the
passions; your time has come.

"When I brought you up in all the simplicity of nature, instead of
preaching disagreeable duties, I secured for you immunity from the vices
which make such duties disagreeable; I made lying not so much hateful as
unnecessary in your sight; I taught you not so much to give others their
due, as to care little about your own rights; I made you kindly rather
than virtuous. But the kindly man is only kind so long as he finds it
pleasant; kindness falls to pieces at the shook of human passions; the
kindly man is only kind to himself.

"What is meant by a virtuous man? He who can conquer his affections; for
then he follows his reason, his conscience; he does his duty; he is his
own master and nothing can turn him from the right way. So far you have
had only the semblance of liberty, the precarious liberty of the slave
who has not received his orders. Now is the time for real freedom; learn
to be your own master; control your heart, my Emile, and you will be
virtuous.

"There is another apprenticeship before you, an apprenticeship more
difficult than the former; for nature delivers us from the evils she
lays upon us, or else she teaches us to submit to them; but she has no
message for us with regard to our self-imposed evils; she leaves us to
ourselves; she leaves us, victims of our own passions, to succumb to our
vain sorrows, to pride ourselves on the tears of which we should be
ashamed.

"This is your first passion. Perhaps it is the only passion worthy of
you. If you can control it like a man, it will be the last; you will be
master of all the rest, and you will obey nothing but the passion for
virtue.

"There is nothing criminal in this passion; I know it; it is as pure as
the hearts which experience it. It was born of honour and nursed by
innocence. Happy lovers! for you the charms of virtue do but add to
those of love; and the blessed union to which you are looking forward is
less the reward of your goodness than of your affection. But tell me, O
truthful man, though this passion is pure, is it any the less your
master? Are you the less its slave? And if to-morrow it should cease to
be innocent, would you strangle it on the spot? Now is the time to try
your strength; there is no time for that in hours of danger. These
perilous efforts should be made when danger is still afar. We do not
practise the use of our weapons when we are face to face with the enemy,
we do that before the war; we come to the battle-field ready prepared.

"It is a mistake to classify the passions as lawful and unlawful, so as
to yield to the one and refuse the other. All alike are good if we are
their masters; all alike are bad if we abandon ourselves to them. Nature
forbids us to extend our relations beyond the limits of our strength;
reason forbids us to want what we cannot get, conscience forbids us, not
to be tempted, but to yield to temptation. To feel or not to feel a
passion is beyond our control, but we can control ourselves. Every
sentiment under our own control is lawful; those which control us are
criminal. A man is not guilty if he loves his neighbour's wife, provided
he keeps this unhappy passion under the control of the law of duty; he
is guilty if he loves his own wife so greatly as to sacrifice everything
to that love.

"Do not expect me to supply you with lengthy precepts of morality, I
have only one rule to give you which sums up all the rest. Be a man;
restrain your heart within the limits of your manhood. Study and know
these limits; however narrow they may be, we are not unhappy within
them; it is only when we wish to go beyond them that we are unhappy,
only when, in our mad passions, we try to attain the impossible; we are
unhappy when we forget our manhood to make an imaginary world for
ourselves, from which we are always slipping back into our own. The only
good things, whose loss really affects us, are those which we claim as
our rights. If it is clear that we cannot obtain what we want, our mind
turns away from it; wishes without hope cease to torture us. A beggar is
not tormented by a desire to be a king; a king only wishes to be a god
when he thinks himself more than man.

"The illusions of pride are the source of our greatest ills; but the
contemplation of human suffering keeps the wise humble. He keeps to his
proper place and makes no attempt to depart from it; he does not waste
his strength in getting what he cannot keep; and his whole strength
being devoted to the right employment of what he has, he is in reality
richer and more powerful in proportion as he desires less than we. A
man, subject to death and change, shall I forge for myself lasting
chains upon this earth, where everything changes and disappears, whence
I myself shall shortly vanish! Oh, Emile! my son! if I were to lose you,
what would be left of myself? And yet I must learn to lose you, for who
knows when you may be taken from me?

``Would you live in wisdom and happiness, fix your heart on the beauty
that is eternal; let your desires be limited by your position, let your
duties take precedence of your wishes; extend the law of necessity into
the region of morals; learn to lose what may be taken from you; learn to
forsake all things at the command of virtue, to set yourself above the
chances of life, to detach your heart before it is torn in pieces, to be
brave in adversity so that you may never be wretched, to be steadfast in
duty that you may never be guilty of a crime. Then you will be happy in
spite of fortune, and good in spite of your passions. You will find a
pleasure that cannot be destroyed, even in the possession of the most
fragile things; you will possess them, they will not possess you, and
you will realise that the man who loses everything, only enjoys what he
knows how to resign. It is true you will not enjoy the illusions of
imaginary pleasures, neither will you feel the sufferings which are
their result. You will profit greatly by this exchange, for the
sufferings are real and frequent, the pleasures are rare and empty.
Victor over so many deceitful ideas, you will also vanquish the idea
that attaches such an excessive value to life. You will spend your life
in peace, and you will leave it without terror; you will detach yourself
from life as from other things. Let others, horror-struck, believe that
when this life is ended they cease to be; conscious of the nothingness
of life, you will think that you are but entering upon the true life. To
the wicked, death is the close of life; to the just it is its dawn.''

Emile heard me with attention not unmixed with anxiety. After such a
startling preface he feared some gloomy conclusion. He foresaw that when
I showed him how necessary it is to practise the strength of the soul, I
desired to subject him to this stern discipline; he was like a wounded
man who shrinks from the surgeon, and fancies he already feels the
painful but healing touch which will cure the deadly wound.

Uncertain, anxious, eager to know what I am driving at, he does not
answer, he questions me but timidly. ``What must I do?'' says he almost
trembling, not daring to raise his eyes. ``What must you do?'' I reply
firmly. ``You must leave Sophy.'' ``What are you saying?'' he exclaimed
angrily. ``Leave Sophy, leave Sophy, deceive her, become a traitor, a
villain, a perjurer!'' ``Why!'' I continue, interrupting him; ``does
Emile suppose I shall teach him to deserve such titles?'' ``No,'' he
continued with the same vigour. ``Neither you nor any one else; I am
capable of preserving your work; I shall not deserve such reproaches.''

I was prepared for this first outburst; I let it pass unheeded. If I had
not the moderation I preach it would not be much use preaching it! Emile
knows me too well to believe me capable of demanding any wrong action
from him, and he knows that it would be wrong to leave Sophy, in the
sense he attaches to the phrase. So he waits for an explanation. Then I
resume my speech.

"My dear Emile, do you think any man whatsoever can be happier than you
have been for the last three months? If you think so, undeceive
yourself. Before tasting the pleasures of life you have plumbed the
depths of its happiness. There is nothing more than you have already
experienced. The joys of sense are soon over; habit invariably destroys
them. You have tasted greater joys through hope than you will ever enjoy
in reality. The imagination which adorns what we long for, deserts its
possession. With the exception of the one self-existing Being, there is
nothing beautiful except that which is not. If that state could have
lasted for ever, you would have found perfect happiness. But all that is
related to man shares his decline; all is finite, all is fleeting in
human life, and even if the conditions which make us happy could be
prolonged for ever, habit would deprive us of all taste for that
happiness. If external circumstances remain unchanged, the heart
changes; either happiness forsakes us, or we forsake her.

"During your infatuation time has passed unheeded. Summer is over,
winter is at hand. Even if our expeditions were possible, at such a time
of year they would not be permitted. Whether we wish it or no, we shall
have to change our way of life; it cannot continue. I read in your eager
eyes that this does not disturb you greatly; Sophy's confession and your
own wishes suggest a simple plan for avoiding the snow and escaping the
journey. The plan has its advantages, no doubt; but when spring returns,
the snow will melt and the marriage will remain; you must reckon for all
seasons.

"You wish to marry Sophy and you have only known her five months! You
wish to marry her, not because she is a fit wife for you, but because
she pleases you; as if love were never mistaken as to fitness, as if
those, who begin with love, never ended with hatred! I know she is
virtuous; but is that enough? Is fitness merely a matter of honour? It
is not her virtue I misdoubt, it is her disposition. Does a woman show
her real character in a day? Do you know how often you must have seen
her and under what varying conditions to really know her temper? Is four
months of liking a sufficient pledge for the rest of your life? A couple
of months hence you may have forgotten her; as soon as you are gone
another may efface your image in her heart; on your return you may find
her as indifferent as you have hitherto found her affectionate.
Sentiments are not a matter of principle; she may be perfectly virtuous
and yet cease to love you. I am inclined to think she will be faithful
and true; but who will answer for her, and who will answer for you if
you are not put to the proof? Will you postpone this trial till it is
too late, will you wait to know your true selves till parting is no
longer possible?

"Sophy is not eighteen, and you are barely twenty-two; this is the age
for love, but not for marriage. What a father and mother for a family!
If you want to know how to bring up children, you should at least wait
till you yourselves are children no longer. Do you not know that too
early motherhood has weakened the constitution, destroyed the health,
and shortened the life of many young women? Do you not know that many
children have always been weak and sickly because their mother was
little more than a child herself? When mother and child are both
growing, the strength required for their growth is divided, and neither
gets all that nature intended; are not both sure to suffer? Either I
know very little of Emile, or he would rather wait and have a healthy
wife and children, than satisfy his impatience at the price of their
life and health.

"Let us speak of yourself. You hope to be a husband and a father; have
you seriously considered your duties? When you become the head of a
family you will become a citizen of your country. And what is a citizen
of the state? What do you know about it? You have studied your duties as
a man, but what do you know of the duties of a citizen? Do you know the
meaning of such terms as government, laws, country? Do you know the
price you must pay for life, and for what you must be prepared to die?
You think you know everything, when you really know nothing at all.
Before you take your place in the civil order, learn to perceive and
know what is your proper place.

``Emile, you must leave Sophy; I do not bid you forsake her; if you were
capable of such conduct, she would be only too happy not to have married
you; you must leave her in order to return worthy of her. Do not be vain
enough to think yourself already worthy. How much remains to be done!
Come and fulfil this splendid task; come and learn to submit to absence;
come and earn the prize of fidelity, so that when you return you may
indeed deserve some honour, and may ask her hand not as a favour but as
a reward.''

Unaccustomed to struggle with himself, untrained to desire one thing and
to will another, the young man will not give way; he resists, he argues.
Why should he refuse the happiness which awaits him? Would he not
despise the hand which is offered him if he hesitated to accept it? Why
need he leave her to learn what he ought to know? And if it were
necessary to leave her why not leave her as his wife with a certain
pledge of his return? Let him be her husband, and he is ready to follow
me; let them be married and he will leave her without fear. ``Marry her
in order to leave her, dear Emile! what a contradiction! A lover who can
leave his mistress shows himself capable of great things; a husband
should never leave his wife unless through necessity. To cure your
scruples, I see the delay must be involuntary on your part; you must be
able to tell Sophy you leave her against your will. Very well, be
content, and since you will not follow the commands of reason, you must
submit to another master. You have not forgotten your promise. Emile,
you must leave Sophy; I will have it.''

For a moment or two he was downcast, silent, and thoughtful, then
looking me full in the face he said, ``When do we start?'' ``In a week's
time,'' I replied; ``Sophy must be prepared for our going. Women are
weaker than we are, and we must show consideration for them; and this
parting is not a duty for her as it is for you, so she may be allowed to
bear it less bravely.''

The temptation to continue the daily history of their love up to the
time of their separation is very great; but I have already presumed too
much upon the good nature of my readers; let us abridge the story so as
to bring it to an end. Will Emile face the situation as bravely at his
mistress' feet as he has done in conversation with his friend? I think
he will; his confidence is rooted in the sincerity of his love. He would
be more at a loss with her, if it cost him less to leave her; he would
leave her feeling himself to blame, and that is a difficult part for a
man of honour to play; but the greater the sacrifice, the more credit he
demands for it in the sight of her who makes it so difficult. He has no
fear that she will misunderstand his motives. Every look seems to say,
``Oh, Sophy, read my heart and be faithful to me; your lover is not
without virtue.''

Sophy tries to bear the unforeseen blow with her usual pride and
dignity. She tries to seem as if she did not care, but as the honours of
war are not hers, but Emile's, her strength is less equal to the task.
She weeps, she sighs against her will, and the fear of being forgotten
embitters the pain of parting. She does not weep in her lover's sight,
she does not let him see her terror; she would die rather than utter a
sigh in his presence. I am the recipient of her lamentations, I behold
her tears, it is I who am supposed to be her confidant. Women are very
clever and know how to conceal their cleverness; the more she frets in
private, the more pains she takes to please me; she feels that her fate
is in my hands.

I console and comfort her; I make myself answerable for her lover, or
rather for her husband; let her be as true to him as he to her and I
promise they shall be married in two years' time. She respects me enough
to believe that I do not want to deceive her. I am guarantor to each for
the other. Their hearts, their virtue, my honesty, the confidence of
their parents, all combine to reassure them. But what can reason avail
against weakness? They part as if they were never to meet again.

Then it is that Sophy recalls the regrets of Eucharis, and fancies
herself in her place. Do not let us revive that fantastic affection
during his absence ``Sophy,'' say I one day, ``exchange books with
Emile; let him have your Telemachus that he may learn to be like him,
and let him give you his Spectator which you enjoy reading. Study the
duties of good wives in it, and remember that in two years' time you
will undertake those duties.'' The exchange gave pleasure to both and
inspired them with confidence. At last the sad day arrived and they must
part.

Sophy's worthy father, with whom I had arranged the whole business, took
affectionate leave of me, and taking me aside, he spoke seriously and
somewhat emphatically, saying, ``I have done everything to please you; I
knew I had to do with a man of honour; I have only one word to say.
Remembering your pupil has signed his contract of marriage on my
daughter's lips.''

What a difference in the behaviour of the two lovers! Emile, impetuous,
eager, excited, almost beside himself, cries aloud and sheds torrents of
tears upon the hands of father, mother, and daughter; with sobs he
embraces every one in the house and repeats the same thing over and over
again in a way that would be ludicrous at any other time. Sophy, pale,
sorrowful, doleful, and heavy-eyed, remains quiet without a word or a
tear, she sees no one, not even Emile. In vain he takes her hand, and
clasps her in his arms; she remains motionless, unheeding his tears, his
caresses, and everything he does; so far as she is concerned, he is gone
already. A sight more moving than the prolonged lamentations and noisy
regrets of her lover! He sees, he feels, he is heartbroken. I drag him
reluctantly away; if I left him another minute, he would never go. I am
delighted that he should carry this touching picture with him. If he
should ever be tempted to forget what is due to Sophy, his heart must
have strayed very far indeed if I cannot bring it back to her by
recalling her as he saw her last.

\subparagraph{OF TRAVEL}\label{id01753}

Is it good for young people to travel? The question is often asked and
as often hotly disputed. If it were stated otherwise---Are men the
better for having travelled?---perhaps there would be less difference of
opinion.

The misuse of books is the death of sound learning. People think they
know what they have read, and take no pains to learn. Too much reading
only produces a pretentious ignoramus. There was never so much reading
in any age as the present, and never was there less learning; in no
country of Europe are so many histories and books of travel printed as
in France, and nowhere is there less knowledge of the mind and manners
of other nations. So many books lead us to neglect the book of the
world; if we read it at all, we keep each to our own page. If the
phrase, ``Can one become a Persian,'' were unknown to me, I should
suspect on hearing it that it came from the country where national
prejudice is most prevalent and from the sex which does most to increase
it.

A Parisian thinks he has a knowledge of men and he knows only Frenchmen;
his town is always full of foreigners, but he considers every foreigner
as a strange phenomenon which has no equal in the universe. You must
have a close acquaintance with the middle classes of that great city,
you must have lived among them, before you can believe that people could
be at once so witty and so stupid. The strangest thing about it is that
probably every one of them has read a dozen times a description of the
country whose inhabitants inspire him with such wonder.

To discover the truth amidst our own prejudices and those of the authors
is too hard a task. I have been reading books of travels all my life,
but I never found two that gave me the same idea of the same nation. On
comparing my own scanty observations with what I have read, I have
decided to abandon the travellers and I regret the time wasted in trying
to learn from their books; for I am quite convinced that for that sort
of study, seeing not reading is required. That would be true enough if
every traveller were honest, if he only said what he saw and believed,
and if truth were not tinged with false colours from his own eyes. What
must it be when we have to disentangle the truth from the web of lies
and ill-faith?

Let us leave the boasted resources of books to those who are content to
use them. Like the art of Raymond Lully they are able to set people
chattering about things they do not know. They are able to set
fifteen-year-old Platos discussing philosophy in the clubs, and teaching
people the customs of Egypt and the Indies on the word of Paul Lucas or
Tavernier.

I maintain that it is beyond dispute that any one who has only seen one
nation does not know men; he only knows those men among whom he has
lived. Hence there is another way of stating the question about travel:
``Is it enough for a well-educated man to know his fellow-countrymen, or
ought he to know mankind in general?'' Then there is no place for
argument or uncertainty. See how greatly the solution of a difficult
problem may depend on the way in which it is stated.

But is it necessary to travel the whole globe to study mankind? Need we
go to Japan to study Europeans? Need we know every individual before we
know the species? No, there are men so much alike that it is not worth
while to study them individually. When you have seen a dozen Frenchmen
you have seen them all. Though one cannot say as much of the English and
other nations, it is, however, certain that every nation has its own
specific character, which is derived by induction from the study, not of
one, but many of its members. He who has compared a dozen nations knows
men, just he who has compared a dozen Frenchmen knows the French.

To acquire knowledge it is not enough to travel hastily through a
country. Observation demands eyes, and the power of directing them
towards the object we desire to know. There are plenty of people who
learn no more from their travels than from their books, because they do
not know how to think; because in reading their mind is at least under
the guidance of the author, and in their travels they do not know how to
see for themselves. Others learn nothing, because they have no desire to
learn. Their object is so entirely different, that this never occurs to
them; it is very unlikely that you will see clearly what you take no
trouble to look for. The French travel more than any other nation, but
they are so taken up with their own customs, that everything else is
confused together. There are Frenchmen in every corner of the globe. In
no country of the world do you find more people who have travelled than
in France. And yet of all the nations of Europe, that which has seen
most, knows least. The English are also travellers, but they travel in
another fashion; these two nations must always be at opposite extremes.
The English nobility travels, the French stays at home; the French
people travel, the English stay at home. This difference does credit, I
think, to the English. The French almost always travel for their own
ends; the English do not seek their fortune in other lands, unless in
the way of commerce and with their hands full; when they travel it is to
spend their money, not to live by their wits; they are too proud to
cringe before strangers. This is why they learn more abroad than the
French who have other fish to fry. Yet the English have their national
prejudices; but these prejudices are not so much the result of ignorance
as of feeling. The Englishman's prejudices are the result of pride, the
Frenchman's are due to vanity.

Just as the least cultivated nations are usually the best, so those
travel best who travel least; they have made less progress than we in
our frivolous pursuits, they are less concerned with the objects of our
empty curiosity, so that they give their attention to what is really
useful. I hardly know any but the Spaniards who travel in this fashion.
While the Frenchman is running after all the artists of the country,
while the Englishman is getting a copy of some antique, while the German
is taking his album to every man of science, the Spaniard is silently
studying the government, the manners of the country, its police, and he
is the only one of the four who from all that he has seen will carry
home any observation useful to his own country.

The ancients travelled little, read little, and wrote few books; yet we
see in those books that remain to us, that they observed each other more
thoroughly than we observe our contemporaries. Without going back to the
days of Homer, the only poet who transports us to the country he
describes, we cannot deny to Herodotus the glory of having painted
manners in his history, though he does it rather by narrative than by
comment; still he does it better than all our historians whose books are
overladen with portraits and characters. Tacitus has described the
Germans of his time better than any author has described the Germans of
to-day. There can be no doubt that those who have devoted themselves to
ancient history know more about the Greeks, Carthaginians, Romans,
Gauls, and Persians than any nation of to-day knows about its
neighbours.

It must also be admitted that the original characteristics of different
nations are changing day by day, and are therefore more difficult to
grasp. As races blend and nations intermingle, those national
differences which formerly struck the observer at first sight gradually
disappear. Before our time every nation remained more or less cut off
from the rest; the means of communication were fewer; there was less
travelling, less of mutual or conflicting interests, less political and
civil intercourse between nation and nation; those intricate schemes of
royalty, miscalled diplomacy, were less frequent; there were no
permanent ambassadors resident at foreign courts; long voyages were
rare, there was little foreign trade, and what little there was, was
either the work of princes, who employed foreigners, or of people of no
account who had no influence on others and did nothing to bring the
nations together. The relations between Europe and Asia in the present
century are a hundredfold more numerous than those between Gaul and
Spain in the past; Europe alone was less accessible than the whole world
is now.

Moreover, the peoples of antiquity usually considered themselves as the
original inhabitants of their country; they had dwelt there so long that
all record was lost of the far-off times when their ancestors settled
there; they had been there so long that the place had made a lasting
impression on them; but in modern Europe the invasions of the
barbarians, following upon the Roman conquests, have caused an
extraordinary confusion. The Frenchmen of to-day are no longer the big
fair men of old; the Greeks are no longer beautiful enough to serve as a
sculptor's model; the very face of the Romans has changed as well as
their character; the Persians, originally from Tartary, are daily losing
their native ugliness through the intermixture of Circassian blood.
Europeans are no longer Gauls, Germans, Iberians, Allobroges; they are
all Scythians, more or less degenerate in countenance, and still more so
in conduct.

This is why the ancient distinctions of race, the effect of soil and
climate, made a greater difference between nation and nation in respect
of temperament, looks, manners, and character than can be distinguished
in our own time, when the fickleness of Europe leaves no time for
natural causes to work, when the forests are cut down and the marshes
drained, when the earth is more generally, though less thoroughly,
tilled, so that the same differences between country and country can no
longer be detected even in purely physical features.

If they considered these facts perhaps people would not be in such a
hurry to ridicule Herodotus, Ctesias, Pliny for having described the
inhabitants of different countries each with its own peculiarities and
with striking differences which we no longer see. To recognise such
types of face we should need to see the men themselves; no change must
have passed over them, if they are to remain the same. If we could
behold all the people who have ever lived, who can doubt that we should
find greater variations between one century and another, than are now
found between nation and nation.

At the same time, while observation becomes more difficult, it is more
carelessly and badly done; this is another reason for the small success
of our researches into the natural history of the human race. The
information acquired by travel depends upon the object of the journey.
If this object is a system of philosophy, the traveller only sees what
he desires to see; if it is self-interest, it engrosses the whole
attention of those concerned. Commerce and the arts which blend and
mingle the nations at the same time prevent them from studying each
other. If they know how to make a profit out of their neighbours, what
more do they need to know?

It is a good thing to know all the places where we might live, so as to
choose those where we can live most comfortably. If every one lived by
his own efforts, all he would need to know would be how much land would
keep him in food. The savage, who has need of no one, and envies no one,
neither knows nor seeks to know any other country but his own. If he
requires more land for his subsistence he shuns inhabited places; he
makes war upon the wild beasts and feeds on them. But for us, to whom
civilised life has become a necessity, for us who must needs devour our
fellow-creatures, self-interest prompts each one of us to frequent those
districts where there are most people to be devoured. This is why we all
flock to Rome, Paris, and London. Human flesh and blood are always
cheapest in the capital cities. Thus we only know the great nations,
which are just like one another.

They say that men of learning travel to obtain information; not so, they
travel like other people from interested motives. Philosophers like
Plato and Pythagoras are no longer to be found, or if they are, it must
be in far-off lands. Our men of learning only travel at the king's
command; they are sent out, their expenses are paid, they receive a
salary for seeing such and such things, and the object of that journey
is certainly not the study of any question of morals. Their whole time
is required for the object of their journey, and they are too honest not
to earn their pay. If in any country whatsoever there are people
travelling at their own expense, you may be sure it is not to study men
but to teach them. It is not knowledge they desire but ostentation. How
should their travels teach them to shake off the yoke of prejudice? It
is prejudice that sends them on their travels.

To travel to see foreign lands or to see foreign nations are two very
different things. The former is the usual aim of the curious, the latter
is merely subordinate to it. If you wish to travel as a philosopher you
should reverse this order. The child observes things till he is old
enough to study men. Man should begin by studying his fellows; he can
study things later if time permits.

It is therefore illogical to conclude that travel is useless because we
travel ill. But granting the usefulness of travel, does it follow that
it is good for all of us? Far from it; there are very few people who are
really fit to travel; it is only good for those who are strong enough in
themselves to listen to the voice of error without being deceived,
strong enough to see the example of vice without being led away by it.
Travelling accelerates the progress of nature, and completes the man for
good or evil. When a man returns from travelling about the world, he is
what he will be all his life; there are more who return bad than good,
because there are more who start with an inclination towards evil. In
the course of their travels, young people, ill-educated and ill-behaved,
pick up all the vices of the nations among whom they have sojourned, and
none of the virtues with which those vices are associated; but those
who, happily for themselves, are well-born, those whose good disposition
has been well cultivated, those who travel with a real desire to learn,
all such return better and wiser than they went. Emile will travel in
this fashion; in this fashion there travelled another young man, worthy
of a nobler age; one whose worth was the admiration of Europe, one who
died for his country in the flower of his manhood; he deserved to live,
and his tomb, ennobled by his virtues only, received no honour till a
stranger's hand adorned it with flowers.

Everything that is done in reason should have its rules. Travel,
undertaken as a part of education, should therefore have its rules. To
travel for travelling's sake is to wander, to be a vagabond; to travel
to learn is still too vague; learning without some definite aim is
worthless. I would give a young man a personal interest in learning, and
that interest, well-chosen, will also decide the nature of the
instruction. This is merely the continuation of the method I have
hitherto practised.

Now after he has considered himself in his physical relations to other
creatures, in his moral relations with other men, there remains to be
considered his civil relations with his fellow-citizens. To do this he
must first study the nature of government in general, then the different
forms of government, and lastly the particular government under which he
was born, to know if it suits him to live under it; for by a right which
nothing can abrogate, every man, when he comes of age, becomes his own
master, free to renounce the contract by which he forms part of the
community, by leaving the country in which that contract holds good. It
is only by sojourning in that country, after he has come to years of
discretion, that he is supposed to have tacitly confirmed the pledge
given by his ancestors. He acquires the right to renounce his country,
just as he has the right to renounce all claim to his father's lands;
yet his place of birth was a gift of nature, and in renouncing it, he
renounces what is his own. Strictly speaking, every man remains in the
land of his birth at his own risk unless he voluntarily submits to its
laws in order to acquire a right to their protection.

For example, I should say to Emile, ``Hitherto you have lived under my
guidance, you were unable to rule yourself. But now you are approaching
the age when the law, giving you the control over your property, makes
you master of your person. You are about to find yourself alone in
society, dependent on everything, even on your patrimony. You mean to
marry; that is a praiseworthy intention, it is one of the duties of man;
but before you marry you must know what sort of man you want to be, how
you wish to spend your life, what steps you mean to take to secure a
living for your family and for yourself; for although we should not make
this our main business, it must be definitely considered. Do you wish to
be dependent on men whom you despise? Do you wish to establish your
fortune and determine your position by means of civil relations which
will make you always dependent on the choice of others, which will
compel you, if you would escape from knaves, to become a knave
yourself?''

In the next place I would show him every possible way of using his money
in trade, in the civil service, in finance, and I shall show him that in
every one of these there are risks to be taken, every one of them places
him in a precarious and dependent position, and compels him to adapt his
morals, his sentiments, his conduct to the example and the prejudices of
others.

"There is yet another way of spending your time and money; you may join
the army; that is to say, you may hire yourself out at very high wages
to go and kill men who never did you any harm. This trade is held in
great honour among men, and they cannot think too highly of those who
are fit for nothing better. Moreover, this profession, far from making
you independent of other resources, makes them all the more necessary;
for it is a point of honour in this profession to ruin those who have
adopted it. It is true they are not all ruined; it is even becoming
fashionable to grow rich in this as in other professions; but if I told
you how people manage to do it, I doubt whether you would desire to
follow their example.

``Moreover, you must know that, even in this trade, it is no longer a
question of courage or valour, unless with regard to the ladies; on the
contrary, the more cringing, mean, and degraded you are, the more honour
you obtain; if you have decided to take your profession seriously, you
will be despised, you will be hated, you will very possibly be driven
out of the service, or at least you will fall a victim to favouritism
and be supplanted by your comrades, because you have been doing your
duty in the trenches, while they have been attending to their toilet.''

We can hardly suppose that any of these occupations will be much to
Emile's taste. ``Why,'' he will exclaim, ``have I forgotten the
amusements of my childhood? Have I lost the use of my arms? Is my
strength failing me? Do I not know how to work? What do I care about all
your fine professions and all the silly prejudices of others? I know no
other pride than to be kindly and just; no other happiness than to live
in independence with her I love, gaining health and a good appetite by
the day's work. All these difficulties you speak of do not concern me.
The only property I desire is a little farm in some quiet corner. I will
devote all my efforts after wealth to making it pay, and I will live
without a care. Give me Sophy and my land, and I shall be rich.''

"Yes, my dear friend, that is all a wise man requires, a wife and land
of his own; but these treasures are scarcer than you think. The rarest
you have found already; let us discuss the other.

"A field of your own, dear Emile! Where will you find it, in what remote
corner of the earth can you say, `Here am I master of myself and of this
estate which belongs to me?' We know where a man may grow rich; who
knows where he can do without riches? Who knows where to live free and
independent, without ill-treating others and without fear of being
ill-treated himself! Do you think it is so easy to find a place where
you can always live like an honest man? If there is any safe and lawful
way of living without intrigues, without lawsuits, without dependence on
others, it is, I admit, to live by the labour of our hands, by the
cultivation of our own land; but where is the state in which a man can
say, `The earth which I dig is my own?' Before choosing this happy spot,
be sure that you will find the peace you desire; beware lest an unjust
government, a persecuting religion, and evil habits should disturb you
in your home. Secure yourself against the excessive taxes which devour
the fruits of your labours, and the endless lawsuits which consume your
capital. Take care that you can live rightly without having to pay court
to intendents, to their deputies, to judges, to priests, to powerful
neighbours, and to knaves of every kind, who are always ready to annoy
you if you neglect them. Above all, secure yourself from annoyance on
the part of the rich and great; remember that their estates may anywhere
adjoin your Naboth's vineyard. If unluckily for you some great man buys
or builds a house near your cottage, make sure that he will not find a
way, under some pretence or other, to encroach on your lands to round
off his estate, or that you do not find him at once absorbing all your
resources to make a wide highroad. If you keep sufficient credit to ward
off all these disagreeables, you might as well keep your money, for it
will cost you no more to keep it. Riches and credit lean upon each
other, the one can hardly stand without the other.

``I have more experience than you, dear Emile; I see more clearly the
difficulties in the way of your scheme. Yet it is a fine scheme and
honourable; it would make you happy indeed. Let us try to carry it out.
I have a suggestion to make; let us devote the two years from now till
the time of your return to choosing a place in Europe where you could
live happily with your family, secure from all the dangers I have just
described. If we succeed, you will have discovered that true happiness,
so often sought for in vain; and you will not have to regret the time
spent in its search. If we fail, you will be cured of a mistaken idea;
you will console yourself for an inevitable ill, and you will bow to the
law of necessity.''

I do not know whether all my readers will see whither this suggested
inquiry will lead us; but this I do know, if Emile returns from his
travels, begun and continued with this end in view, without a full
knowledge of questions of government, public morality, and political
philosophy of every kind, we are greatly lacking, he in intelligence and
I in judgment.

The science of politics is and probably always will be unknown. Grotius,
our leader in this branch of learning, is only a child, and what is
worse an untruthful child. When I hear Grotius praised to the skies and
Hobbes overwhelmed with abuse, I perceive how little sensible men have
read or understood these authors. As a matter of fact, their principles
are exactly alike, they only differ in their mode of expression. Their
methods are also different: Hobbes relies on sophism; Grotius relies on
the poets; they are agreed in everything else. In modern times the only
man who could have created this vast and useless science was the
illustrious Montesquieu. But he was not concerned with the principles of
political law; he was content to deal with the positive laws of settled
governments; and nothing could be more different than these two branches
of study.

Yet he who would judge wisely in matters of actual government is forced
to combine the two; he must know what ought to be in order to judge what
is. The chief difficulty in the way of throwing light upon this
important matter is to induce an individual to discuss and to answer
these two questions. ``How does it concern me; and what can I do?''
Emile is in a position to answer both.

The next difficulty is due to the prejudices of childhood, the
principles in which we were brought up; it is due above all to the
partiality of authors, who are always talking about truth, though they
care very little about it; it is only their own interests that they care
for, and of these they say nothing. Now the nation has neither
professorships, nor pensions, nor membership of the academies to bestow.
How then shall its rights be established by men of that type? The
education I have given him has removed this difficulty also from Emile's
path. He scarcely knows what is meant by government; his business is to
find the best; he does not want to write books; if ever he did so, it
would not be to pay court to those in authority, but to establish the
rights of humanity.

There is a third difficulty, more specious than real; a difficulty which
I neither desire to solve nor even to state; enough that I am not afraid
of it; sure I am that in inquiries of this kind, great talents are less
necessary than a genuine love of justice and a sincere reverence for
truth. If matters of government can ever be fairly discussed, now or
never is our chance.

Before beginning our observations we must lay down rules of procedure;
we must find a scale with which to compare our measurements. Our
principles of political law are our scale. Our actual measurements are
the civil law of each country.

Our elementary notions are plain and simple, being taken directly from
the nature of things. They will take the form of problems discussed
between us, and they will not be formulated into principles, until we
have found a satisfactory solution of our problems.

For example, we shall begin with the state of nature, we shall see
whether men are born slaves or free, in a community or independent; is
their association the result of free will or of force? Can the force
which compels them to united action ever form a permanent law, by which
this original force becomes binding, even when another has been imposed
upon it, so that since the power of King Nimrod, who is said to have
been the first conqueror, every other power which has overthrown the
original power is unjust and usurping, so that there are no lawful kings
but the descendants of Nimrod or their representatives; or if this
original power has ceased, has the power which succeeded it any right
over us, and does it destroy the binding force of the former power, so
that we are not bound to obey except under compulsion, and we are free
to rebel as soon as we are capable of resistance? Such a right is not
very different from might; it is little more than a play upon words.

We shall inquire whether man might not say that all sickness comes from
God, and that it is therefore a crime to send for the doctor.

Again, we shall inquire whether we are bound by our conscience to give
our purse to a highwayman when we might conceal it from him, for the
pistol in his hand is also a power.

Does this word power in this context mean something different from a
power which is lawful and therefore subject to the laws to which it owes
its being?

Suppose we reject this theory that might is right and admit the right of
nature, or the authority of the father, as the foundation of society; we
shall inquire into the extent of this authority; what is its foundation
in nature? Has it any other grounds but that of its usefulness to the
child, his weakness, and the natural love which his father feels towards
him? When the child is no longer feeble, when he is grown-up in mind as
well as in body, does not he become the sole judge of what is necessary
for his preservation? Is he not therefore his own master, independent of
all men, even of his father himself? For is it not still more certain
that the son loves himself, than that the father loves the son?

The father being dead, should the children obey the eldest brother, or
some other person who has not the natural affection of a father? Should
there always be, from family to family, one single head to whom all the
family owe obedience? If so, how has power ever come to be divided, and
how is it that there is more than one head to govern the human race
throughout the world?

Suppose the nations to have been formed each by its own choice; we shall
then distinguish between right and fact; being thus subjected to their
brothers, uncles, or other relations, not because they were obliged, but
because they choose, we shall inquire whether this kind of society is
not a sort of free and voluntary association?

Taking next the law of slavery, we shall inquire whether a man can make
over to another his right to himself, without restriction, without
reserve, without any kind of conditions; that is to say, can he renounce
his person, his life, his reason, his very self, can he renounce all
morality in his actions; in a word, can he cease to exist before his
death, in spite of nature who places him directly in charge of his own
preservation, in spite of reason and conscience which tell him what to
do and what to leave undone?

If there is any reservation or restriction in the deed of slavery, we
shall discuss whether this deed does not then become a true contract, in
which both the contracting powers, having in this respect no common
master, {[}Footnote: If they had such a common master, he would be no
other than the sovereign, and then the right of slavery resting on the
right of sovereignty would not be its origin.{]} remain their own judge
as to the conditions of the contract, and therefore free to this extent,
and able to break the contract as soon as it becomes hurtful.

If then a slave cannot convey himself altogether to his master, how can
a nation convey itself altogether to its head? If a slave is to judge
whether his master is fulfilling his contract, is not the nation to
judge whether its head is fulfilling his contract?

Thus we are compelled to retrace our steps, and when we consider the
meaning of this collective nation we shall inquire whether some
contract, a tacit contract at the least, is not required to make a
nation, a contract anterior to that which we are assuming.

Since the nation was a nation before it chose a king, what made it a
nation, except the social contract? Therefore the social contract is the
foundation of all civil society, and it is in the nature of this
contract that we must seek the nature of the society formed by it.

We will inquire into the meaning of this contract; may it not be fairly
well expressed in this formula? As an individual every one of us
contributes his goods, his person, his life, to the common stock, under
the supreme direction of the general will; while as a body we receive
each member as an indivisible part of the whole.

Assuming this, in order to define the terms we require, we shall observe
that, instead of the individual person of each contracting party, this
deed of association produces a moral and collective body, consisting of
as many members as there are votes in the Assembly. This public
personality is usually called the body politic, which is called by its
members the State when it is passive, and the Sovereign when it is
active, and a Power when compared with its equals. With regard to the
members themselves, collectively they are known as the nation, and
individually as citizens as members of the city or partakers in the
sovereign power, and subjects as obedient to the same authority.

We shall note that this contract of association includes a mutual pledge
on the part of the public and the individual; and that each individual,
entering, so to speak, into a contract with himself, finds himself in a
twofold capacity, i.e., as a member of the sovereign with regard to
others, as member of the state with regard to the sovereign.

We shall also note that while no one is bound by any engagement to which
he was not himself a party, the general deliberation which may be
binding on all the subjects with regard to the sovereign, because of the
two different relations under which each of them is envisaged, cannot be
binding on the state with regard to itself. Hence we see that there is
not, and cannot be, any other fundamental law, properly so called,
except the social contract only. This does not mean that the body
politic cannot, in certain respects, pledge itself to others; for in
regard to the foreigner, it then becomes a simple creature, an
individual.

Thus the two contracting parties, i.e., each individual and the public,
have no common superior to decide their differences; so we will inquire
if each of them remains free to break the contract at will, that is to
repudiate it on his side as soon as he considers it hurtful.

To clear up this difficulty, we shall observe that, according to the
social pact, the sovereign power is only able to act through the common,
general will; so its decrees can only have a general or common aim;
hence it follows that a private individual cannot be directly injured by
the sovereign, unless all are injured, which is impossible, for that
would be to want to harm oneself. Thus the social contract has no need
of any warrant but the general power, for it can only be broken by
individuals, and they are not therefore freed from their engagement, but
punished for having broken it.

To decide all such questions rightly, we must always bear in mind that
the nature of the social pact is private and peculiar to itself, in that
the nation only contracts with itself, i.e., the people as a whole as
sovereign, with the individuals as subjects; this condition is essential
to the construction and working of the political machine, it alone makes
pledges lawful, reasonable, and secure, without which it would be
absurd, tyrannical, and liable to the grossest abuse.

Individuals having only submitted themselves to the sovereign, and the
sovereign power being only the general will, we shall see that every man
in obeying the sovereign only obeys himself, and how much freer are we
under the social part than in the state of nature.

Having compared natural and civil liberty with regard to persons, we
will compare them as to property, the rights of ownership and the rights
of sovereignty, the private and the common domain. If the sovereign
power rests upon the right of ownership, there is no right more worthy
of respect; it is inviolable and sacred for the sovereign power, so long
as it remains a private individual right; as soon as it is viewed as
common to all the citizens, it is subject to the common will, and this
will may destroy it. Thus the sovereign has no right to touch the
property of one or many; but he may lawfully take possession of the
property of all, as was done in Sparta in the time of Lycurgus; while
the abolition of debts by Solon was an unlawful deed.

Since nothing is binding on the subjects except the general will, let us
inquire how this will is made manifest, by what signs we may recognise
it with certainty, what is a law, and what are the true characters of
the law? This is quite a fresh subject; we have still to define the term
law.

As soon as the nation considers one or more of its members, the nation
is divided. A relation is established between the whole and its part
which makes of them two separate entities, of which the part is one, and
the whole, minus that part, is the other. But the whole minus the part
is not the whole; as long as this relation exists, there is no longer a
whole, but two unequal parts.

On the other hand, if the whole nation makes a law for the whole nation,
it is only considering itself; and if a relation is set up, it is
between the whole community regarded from one point of view, and the
whole community regarded from another point of view, without any
division of that whole. Then the object of the statute is general, and
the will which makes that statute is general too. Let us see if there is
any other kind of decree which may bear the name of law.

If the sovereign can only speak through laws, and if the law can never
have any but a general purpose, concerning all the members of the state,
it follows that the sovereign never has the power to make any law with
regard to particular cases; and yet it is necessary for the preservation
of the state that particular oases should also be dealt with; let us see
how this can be done.

The decrees of the sovereign can only be decrees of the general will,
that is laws; there must also be determining decrees, decrees of power
or government, for the execution of those laws; and these, on the other
hand, can only have particular aims. Thus the decrees by which the
sovereign decides that a chief shall be elected is a law; the decree by
which that chief is elected, in pursuance of the law, is only a decree
of government.

This is a third relation in which the assembled people may be
considered, i.e., as magistrates or executors of the law which it has
passed in its capacity as sovereign. {[}Footnote: These problems and
theorems are mostly taken from the Treatise on the Social Contract,
itself a summary of a larger work, undertaken without due consideration
of my own powers, and long since abandoned.{]}

We will now inquire whether it is possible for the nation to deprive
itself of its right of sovereignty, to bestow it on one or more persons;
for the decree of election not being a law, and the people in this
decree not being themselves sovereign, we do not see how they can
transfer a right which they do not possess.

The essence of sovereignty consisting in the general will, it is equally
hard to see how we can be certain that an individual will shall always
be in agreement with the general will. We should rather assume that it
will often be opposed to it; for individual interest always tends to
privileges, while the common interest always tends to equality, and if
such an agreement were possible, no sovereign right could exist, unless
the agreement were either necessary or indestructible.

We will inquire if, without violating the social pact, the heads of the
nation, under whatever name they are chosen, can ever be more than the
officers of the people, entrusted by them with the duty of carrying the
law into execution. Are not these chiefs themselves accountable for
their administration, and are not they themselves subject to the laws
which it is their business to see carried out?

If the nation cannot alienate its supreme right, can it entrust it to
others for a time? Cannot it give itself a master, cannot it find
representatives? This is an important question and deserves discussion.

If the nation can have neither sovereign nor representatives we will
inquire how it can pass its own laws; must there be many laws; must they
be often altered; is it easy for a great nation to be its own lawgiver?

Was not the Roman people a great nation?

Is it a good thing that there should be great nations?

It follows from considerations already established that there is an
intermediate body in the state between subjects and sovereign; and this
intermediate body, consisting of one or more members, is entrusted with
the public administration, the carrying out of the laws, and the
maintenance of civil and political liberty.

The members of this body are called magistrates or kings, that is to
say, rulers. This body, as a whole, considered in relation to its
members, is called the prince, and considered in its actions it is
called the government.

If we consider the action of the whole body upon itself, that is to say,
the relation of the whole to the whole, of the sovereign to the state,
we can compare this relation to that of the extremes in a proportion of
which the government is the middle term. The magistrate receives from
the sovereign the commands which he gives to the nation, and when it is
reckoned up his product or his power is in the same degree as the
product or power of the citizens who are subjects on one side of the
proportion and sovereigns on the other. None of the three terms can be
varied without at once destroying this proportion. If the sovereign
tries to govern, and if the prince wants to make the laws, or if the
subject refuses to obey them, disorder takes the place of order, and the
state falls to pieces under despotism or anarchy.

Let us suppose that this state consists of ten thousand citizens. The
sovereign can only be considered collectively and as a body, but each
individual, as a subject, has his private and independent existence.
Thus the sovereign is as ten thousand to one; that is to say, every
member of the state has, as his own share, only one ten-thousandth part
of the sovereign power, although he is subject to the whole. Let the
nation be composed of one hundred thousand men, the position of the
subjects is unchanged, and each continues to bear the whole weight of
the laws, while his vote, reduced to the one hundred-thousandth part,
has ten times less influence in the making of the laws. Thus the subject
being always one, the sovereign is relatively greater as the number of
the citizens is increased. Hence it follows that the larger the state
the less liberty.

Now the greater the disproportion between private wishes and the general
will, i.e., between manners and laws, the greater must be the power of
repression. On the other side, the greatness of the state gives the
depositaries of public authority greater temptations and additional
means of abusing that authority, so that the more power is required by
the government to control the people, the more power should there be in
the sovereign to control the government.

From this twofold relation it follows that the continued proportion
between the sovereign, the prince, and the people is not an arbitrary
idea, but a consequence of the nature of the state. Moreover, it follows
that one of the extremes, i.e., the nation, being constant, every time
the double ratio increases or decreases, the simple ratio increases or
diminishes in its turn; which cannot be unless the middle term is as
often changed. From this we may conclude that there is no single
absolute form of government, but there must be as many different forms
of government as there are states of different size.

If the greater the numbers of the nation the less the ratio between its
manners and its laws, by a fairly clear analogy, we may also say, the
more numerous the magistrates, the weaker the government.

To make this principle clearer we will distinguish three essentially
different wills in the person of each magistrate; first, his own will as
an individual, which looks to his own advantage only; secondly, the
common will of the magistrates, which is concerned only with the
advantage of the prince, a will which may be called corporate, and one
which is general in relation to the government and particular in
relation to the state of which the government forms part; thirdly, the
will of the people, or the sovereign will, which is general, as much in
relation to the state viewed as the whole as in relation to the
government viewed as a part of the whole. In a perfect legislature the
private individual will should be almost nothing; the corporate will
belonging to the government should be quite subordinate, and therefore
the general and sovereign will is the master of all the others. On the
other hand, in the natural order, these different wills become more and
more active in proportion as they become centralised; the general will
is always weak, the corporate will takes the second place, the
individual will is preferred to all; so that every one is himself first,
then a magistrate, and then a citizen; a series just the opposite of
that required by the social order.

Having laid down this principle, let us assume that the government is in
the hands of one man. In this case the individual and the corporate will
are absolutely one, and therefore this will has reached the greatest
possible degree of intensity. Now the use of power depends on the degree
of this intensity, and as the absolute power of the government is always
that of the people, and therefore invariable, it follows that the rule
of one man is the most active form of government.

If, on the other hand, we unite the government with the supreme power,
and make the prince the sovereign and the citizens so many magistrates,
then the corporate will is completely lost in the general will, and will
have no more activity than the general will, and it will leave the
individual will in full vigour. Thus the government, though its absolute
force is constant, will have the minimum of activity.

These rules are incontestable in themselves, and other considerations
only serve to confirm them. For example, we see the magistrates as a
body far more active than the citizens as a body, so that the individual
will always counts for more. For each magistrate usually has charge of
some particular duty of government; while each citizen, in himself, has
no particular duty of sovereignty. Moreover, the greater the state the
greater its real power, although its power does not increase because of
the increase in territory; but the state remaining unchanged, the
magistrates are multiplied in vain, the government acquires no further
real strength, because it is the depositary of that of the state, which
I have assumed to be constant. Thus, this plurality of magistrates
decreases the activity of the government without increasing its power.

Having found that the power of the government is relaxed in proportion
as the number of magistrates is multiplied, and that the more numerous
the people, the more the controlling power must be increased, we shall
infer that the ratio between the magistrates and the government should
be inverse to that between subjects and sovereign, that is to say, that
the greater the state, the smaller the government, and that in like
manner the number of chiefs should be diminished because of the
increased numbers of the people.

In order to make this diversity of forms clearer, and to assign them
their different names, we shall observe in the first place that the
sovereign may entrust the care of the government to the whole nation or
to the greater part of the nation, so that there are more citizen
magistrates than private citizens. This form of government is called
Democracy.

Or the sovereign may restrict the government in the hands of a lesser
number, so that there are more plain citizens than magistrates; and this
form of government is called Aristocracy.

Finally, the sovereign may concentrate the whole government in the hands
of one man. This is the third and commonest form of government, and is
called Monarchy or royal government.

We shall observe that all these forms, or the first and second at least,
may be less or more, and that within tolerably wide limits. For the
democracy may include the whole nation, or may be confined to one half
of it. The aristocracy, in its turn, may shrink from the half of the
nation to the smallest number. Even royalty may be shared, either
between father and son, between two brothers, or in some other fashion.
There were always two kings in Sparta, and in the Roman empire there
were as many as eight emperors at once, and yet it cannot be said that
the empire was divided. There is a point where each form of government
blends with the next; and under the three specific forms there may be
really as many forms of government as there are citizens in the state.

Nor is this all. In certain respects each of these governments is
capable of subdivision into different parts, each administered in one of
these three ways. From these forms in combination there may arise a
multitude of mixed forms, since each may be multiplied by all the simple
forms.

In all ages there have been great disputes as to which is the best form
of government, and people have failed to consider that each is the best
in some cases and the worst in others. For ourselves, if the number of
magistrates {[}Footnote: You will remember that I mean, in this context,
the supreme magistrates or heads of the nation, the others being only
their deputies in this or that respect.{]} in the various states is to
be in inverse ratio to the number of the citizens, we infer that
generally a democratic government is adapted to small states, an
aristocratic government to those of moderate size, and a monarchy to
large states.

These inquiries furnish us with a clue by which we may discover what are
the duties and rights of citizens, and whether they can be separated one
from the other; what is our country, in what does it really consist, and
how can each of us ascertain whether he has a country or no?

Having thus considered every kind of civil society in itself, we shall
compare them, so as to note their relations one with another; great and
small, strong and weak, attacking one another, insulting one another,
destroying one another; and in this perpetual action and reaction
causing more misery and loss of life than if men had preserved their
original freedom. We shall inquire whether too much or too little has
not been accomplished in the matter of social institutions; whether
individuals who are subject to law and to men, while societies preserve
the independence of nature, are not exposed to the ills of both
conditions without the advantages of either, and whether it would not be
better to have no civil society in the world rather than to have many
such societies. Is it not that mixed condition which partakes of both
and secures neither?

``Per quem neutrum licet, nec tanquam in bello paratum esse, nec tanquam
in pace securum.''---Seneca De Trang: Animi, cap. I.

Is it not this partial and imperfect association which gives rise to
tyranny and war? And are not tyranny and war the worst scourges of
humanity?

Finally we will inquire how men seek to get rid of these difficulties by
means of leagues and confederations, which leave each state its own
master in internal affairs, while they arm it against any unjust
aggression. We will inquire how a good federal association may be
established, what can make it lasting, and how far the rights of the
federation may be stretched without destroying the right of sovereignty.

The Abbe de Saint-Pierre suggested an association of all the states of
Europe to maintain perpetual peace among themselves. Is this association
practicable, and supposing that it were established, would it be likely
to last? These inquiries lead us straight to all the questions of
international law which may clear up the remaining difficulties of
political law. Finally we shall lay down the real principles of the laws
of war, and we shall see why Grotius and others have only stated false
principles.

I should not be surprised if my pupil, who is a sensible young man,
should interrupt me saying, ``One would think we were building our
edifice of wood and not of men; we are putting everything so exactly in
its place!'' That is true; but remember that the law does not bow to the
passions of men, and that we have first to establish the true principles
of political law. Now that our foundations are laid, come and see what
men have built upon them; and you will see some strange sights!

Then I set him to read Telemachus, and we pursue our journey; we are
seeking that happy Salentum and the good Idomeneus made wise by
misfortunes. By the way we find many like Protesilas and no Philocles,
neither can Adrastes, King of the Daunians, be found. But let our
readers picture our travels for themselves, or take the same journeys
with Telemachus in their hand; and let us not suggest to them painful
applications which the author himself avoids or makes in spite of
himself.

Moreover, Emile is not a king, nor am I a god, so that we are not
distressed that we cannot imitate Telemachus and Mentor in the good they
did; none know better than we how to keep to our own place, none have
less desire to leave it. We know that the same task is allotted to all;
that whoever loves what is right with all his heart, and does the right
so far as it is in his power, has fulfilled that task. We know that
Telemachus and Mentor are creatures of the imagination. Emile does not
travel in idleness and he does more good than if he were a prince. If we
were kings we should be no greater benefactors. If we were kings and
benefactors we should cause any number of real evils for every apparent
good we supposed we were doing. If we were kings and sages, the first
good deed we should desire to perform, for ourselves and for others,
would be to abdicate our kingship and return to our present position.

I have said why travel does so little for every one. What makes it still
more barren for the young is the way in which they are sent on their
travels. Tutors, more concerned to amuse than to instruct, take them
from town to town, from palace to palace, where if they are men of
learning and letters, they make them spend their time in libraries, or
visiting antiquaries, or rummaging among old buildings transcribing
ancient inscriptions. In every country they are busy over some other
century, as if they were living in another country; so that after they
have travelled all over Europe at great expense, a prey to frivolity or
tedium, they return, having seen nothing to interest them, and having
learnt nothing that could be of any possible use to them.

All capitals are just alike, they are a mixture of all nations and all
ways of living; they are not the place in which to study the nations.
Paris and London seem to me the same town. Their inhabitants have a few
prejudices of their own, but each has as many as the other, and all
their rules of conduct are the same. We know the kind of people who will
throng the court. We know the way of living which the crowds of people
and the unequal distribution of wealth will produce. As soon as any one
tells me of a town with two hundred thousand people, I know its life
already. What I do not know about it is not worth going there to learn.

To study the genius and character of a nation you should go to the more
remote provinces, where there is less stir, less commerce, where
strangers seldom travel, where the inhabitants stay in one place, where
there are fewer changes of wealth and position. Take a look at the
capital on your way, but go and study the country far away from that
capital. The French are not in Paris, but in Touraine; the English are
more English in Mercia than in London, and the Spaniards more Spanish in
Galicia than in Madrid. In these remoter provinces a nation assumes its
true character and shows what it really is; there the good or ill
effects of the government are best perceived, just as you can measure
the arc more exactly at a greater radius.

The necessary relations between character and government have been so
clearly pointed out in the book of L'Esprit des Lois, that one cannot do
better than have recourse to that work for the study of those relations.
But speaking generally, there are two plain and simple standards by
which to decide whether governments are good or bad. One is the
population. Every country in which the population is decreasing is on
its way to ruin; and the countries in which the population increases
most rapidly, even were they the poorest countries in the world, are
certainly the best governed. {[}Footnote: I only know one exception to
this rule---it is China.{]} But this population must be the natural
result of the government and the national character, for if it is caused
by colonisation or any other temporary and accidental cause, then the
remedy itself is evidence of the disease. When Augustus passed laws
against celibacy, those laws showed that the Roman empire was already
beginning to decline. Citizens must be induced to marry by the goodness
of the government, not compelled to marry by law; you must not examine
the effects of force, for the law which strives against the constitution
has little or no effect; you should study what is done by the influence
of public morals and by the natural inclination of the government, for
these alone produce a lasting effect. It was the policy of the worthy
Abbe de Saint-Pierre always to look for a little remedy for every
individual ill, instead of tracing them to their common source and
seeing if they could not all be cured together. You do not need to treat
separately every sore on a rich man's body; you should purify the blood
which produces them. They say that in England there are prizes for
agriculture; that is enough for me; that is proof enough that
agriculture will not flourish there much longer.

The second sign of the goodness or badness of the government and the
laws is also to be found in the population, but it is to be found not in
its numbers but in its distribution. Two states equal in size and
population may be very unequal in strength; and the more powerful is
always that in which the people are more evenly distributed over its
territory; the country which has fewer large towns, and makes less show
on this account, will always defeat the other. It is the great towns
which exhaust the state and are the cause of its weakness; the wealth
which they produce is a sham wealth, there is much money and few goods.
They say the town of Paris is worth a whole province to the King of
France; for my own part I believe it costs him more than several
provinces. I believe that Paris is fed by the provinces in more senses
than one, and that the greater part of their revenues is poured into
that town and stays there, without ever returning to the people or to
the king. It is inconceivable that in this age of calculators there is
no one to see that France would be much more powerful if Paris were
destroyed. Not only is this ill-distributed population not advantageous
to the state, it is more ruinous than depopulation itself, because
depopulation only gives as produce nought, and the ill-regulated
addition of still more people gives a negative result. When I hear an
Englishman and a Frenchman so proud of the size of their capitals, and
disputing whether London or Paris has more inhabitants, it seems to me
that they are quarrelling as to which nation can claim the honour of
being the worst governed.

Study the nation outside its towns; thus only will you really get to
know it. It is nothing to see the apparent form of a government,
overladen with the machinery of administration and the jargon of the
administrators, if you have not also studied its nature as seen in the
effects it has upon the people, and in every degree of administration.
The difference of form is really shared by every degree of the
administration, and it is only by including every degree that you really
know the difference. In one country you begin to feel the spirit of the
minister in the manoeuvres of his underlings; in another you must see
the election of members of parliament to see if the nation is really
free; in each and every country, he who has only seen the towns cannot
possibly know what the government is like, as its spirit is never the
same in town and country. Now it is the agricultural districts which
form the country, and the country people who make the nation.

This study of different nations in their remoter provinces, and in the
simplicity of their native genius, gives a general result which is very
satisfactory, to my thinking, and very consoling to the human heart; it
is this: All the nations, if you observe them in this fashion, seem much
better worth observing; the nearer they are to nature, the more does
kindness hold sway in their character; it is only when they are cooped
up in towns, it is only when they are changed by cultivation, that they
become depraved, that certain faults which were rather coarse than
injurious are exchanged for pleasant but pernicious vices.

From this observation we see another advantage in the mode of travel I
suggest; for young men, sojourning less in the big towns which are
horribly corrupt, are less likely to catch the infection of vice; among
simpler people and less numerous company, they will preserve a surer
judgment, a healthier taste, and better morals. Besides this contagion
of vice is hardly to be feared for Emile; he has everything to protect
him from it. Among all the precautions I have taken, I reckon much on
the love he bears in his heart.

We do not know the power of true love over youthful desires, because we
are ourselves as ignorant of it as they are, and those who have control
over the young turn them from true love. Yet a young man must either
love or fall into bad ways. It is easy to be deceived by appearances.
You will quote any number of young men who are said to live very
chastely without love; but show me one grown man, a real man, who can
truly say that his youth was thus spent? In all our virtues, all our
duties, people are content with appearances; for my own part I want the
reality, and I am much mistaken if there is any other way of securing it
beyond the means I have suggested.

The idea of letting Emile fall in love before taking him on his travels
is not my own. It was suggested to me by the following incident.

I was in Venice calling on the tutor of a young Englishman. It was
winter and we were sitting round the fire. The tutor's letters were
brought from the post office. He glanced at them, and then read them
aloud to his pupil. They were in English; I understood not a word, but
while he was reading I saw the young man tear some fine point lace
ruffles which he was wearing, and throw them in the fire one after
another, as quietly as he could, so that no one should see it. Surprised
at this whim, I looked at his face and thought I perceived some emotion;
but the external signs of passion, though much alike in all men, have
national differences which may easily lead one astray. Nations have a
different language of facial expression as well as of speech. I waited
till the letters were finished and then showing the tutor the bare
wrists of his pupil, which he did his best to hide, I said, ``May I ask
the meaning of this?''

The tutor seeing what had happened began to laugh; he embraced his pupil
with an air of satisfaction and, with his consent, he gave me the
desired explanation.

``The ruffles,'' said he, "which Mr. John has just torn to pieces, were
a present from a lady in this town, who made them for him not long ago.
Now you must know that Mr. John is engaged to a young lady in his own
country, with whom he is greatly in love, and she well deserves it. This
letter is from the lady's mother, and I will translate the passage which
caused the destruction you beheld.

``\,`Lucy is always at work upon Mr. John's ruffles. Yesterday Miss
Betty Roldham came to spend the afternoon and insisted on doing some of
her work. I knew that Lucy was up very early this morning and I wanted
to see what she was doing; I found her busy unpicking what Miss Betty
had done. She would not have a single stitch in her present done by any
hand but her own.'\,''

Mr. John went to fetch another pair of ruffles, and I said to his tutor:
``Your pupil has a very good disposition; but tell me is not the letter
from Miss Lucy's mother a put up job? Is it not an expedient of your
designing against the lady of the ruffles?'' ``No,'' said he, ``it is
quite genuine; I am not so artful as that; I have made use of simplicity
and zeal, and God has blessed my efforts.''

This incident with regard to the young man stuck in my mind; it was sure
to set a dreamer like me thinking.

But it is time we finished. Let us take Mr. John back to Miss Lucy, or
rather Emile to Sophy. He brings her a heart as tender as ever, and a
more enlightened mind, and he returns to his native land all the bettor
for having made acquaintance with foreign governments through their
vices and foreign nations through their virtues. I have even taken care
that he should associate himself with some man of worth in every nation,
by means of a treaty of hospitality after the fashion of the ancients,
and I shall not be sorry if this acquaintance is kept up by means of
letters. Not only may this be useful, not only is it always pleasant to
have a correspondent in foreign lands, it is also an excellent antidote
against the sway of patriotic prejudices, to which we are liable all
through our life, and to which sooner or later we are more or less
enslaved. Nothing is better calculated to lessen the hold of such
prejudices than a friendly interchange of opinions with sensible people
whom we respect; they are free from our prejudices and we find ourselves
face to face with theirs, and so we can set the one set of prejudices
against the other and be safe from both. It is not the same thing to
have to do with strangers in our own country and in theirs. In the
former case there is always a certain amount of politeness which either
makes them conceal their real opinions, or makes them think more
favourably of our country while they are with us; when they get home
again this disappears, and they merely do us justice. I should be very
glad if the foreigner I consult has seen my country, but I shall not ask
what he thinks of it till he is at home again.

When we have spent nearly two years travelling in a few of the great
countries and many of the smaller countries of Europe, when we have
learnt two or three of the chief languages, when we have seen what is
really interesting in natural history, government, arts, or men, Emile,
devoured by impatience, reminds me that our time is almost up. Then I
say, ``Well, my friend, you remember the main object of our journey; you
have seen and observed; what is the final result of your observations?
What decision have you come to?'' Either my method is wrong, or he will
answer me somewhat after this fashion---

"What decision have I come to? I have decided to be what you made me; of
my own free will I will add no fetters to those imposed upon me by
nature and the laws. The more I study the works of men in their
institutions, the more clearly I see that, in their efforts after
independence, they become slaves, and that their very freedom is wasted
in vain attempts to assure its continuance. That they may not be carried
away by the flood of things, they form all sorts of attachments; then as
soon as they wish to move forward they are surprised to find that
everything drags them back. It seems to me that to set oneself free we
need do nothing, we need only continue to desire freedom. My master, you
have made me free by teaching me to yield to necessity. Let her come
when she will, I follow her without compulsion; I lay hold of nothing to
keep me back. In our travels I have sought for some corner of the earth
where I might be absolutely my own; but where can one dwell among men
without being dependent on their passions? On further consideration I
have discovered that my desire contradicted itself; for were I to hold
to nothing else, I should at least hold to the spot on which I had
settled; my life would be attached to that spot, as the dryads were
attached to their trees. I have discovered that the words liberty and
empire are incompatible; I can only be master of a cottage by ceasing to
be master of myself.
\aquote{``Hoc erat in votis, modus agri non ita magnus."}{Horace, lib. ii., sat. vi.}
"I remember that my property was the origin of our inquiries. You argued
very forcibly that I could not keep both my wealth and my liberty; but
when you wished me to be free and at the same time without needs, you
desired two incompatible things, for I could only be independent of men
by returning to dependence on nature. What then shall I do with the
fortune bequeathed to me by my parents? To begin with, I will not be
dependent on it; I will cut myself loose from all the ties which bind me
to it; if it is left in my hands, I shall keep it; if I am deprived of
it, I shall not be dragged away with it. I shall not trouble myself to
keep it, but I shall keep steadfastly to my own place. Rich or poor, I
shall be free. I shall be free not merely in this country or in that; I
shall be free in any part of the world. All the chains of prejudice are
broken; as far as I am concerned I know only the bonds of necessity. I
have been trained to endure them from my childhood, and I shall endure
them until death, for I am a man; and why should I not wear those chains
as a free man, for I should have to wear them even if I were a slave,
together with the additional fetters of slavery?

"What matters my place in the world? What matters it where I am?
Wherever there are men, I am among my brethren; wherever there are none,
I am in my own home. So long as I may be independent and rich, and have
wherewithal to live, and I shall live. If my wealth makes a slave of me,
I shall find it easy to renounce it. I have hands to work, and I shall
get a living. If my hands fail me, I shall live if others will support
me; if they forsake me I shall die; I shall die even if I am not
forsaken, for death is not the penalty of poverty, it is a law of
nature. Whensoever death comes I defy it; it shall never find me making
preparations for life; it shall never prevent me having lived.

``My father, this is my decision. But for my passions, I should be in my
manhood independent as God himself, for I only desire what is and I
should never fight against fate. At least, there is only one chain, a
chain which I shall ever wear, a chain of which I may be justly proud.
Come then, give me my Sophy, and I am free.''

"Dear Emile, I am glad indeed to hear you speak like a man, and to
behold the feelings of your heart. At your age this exaggerated
unselfishness is not unpleasing. It will decrease when you have children
of your own, and then you will be just what a good father and a wise man
ought to be. I knew what the result would be before our travels; I knew
that when you saw our institutions you would be far from reposing a
confidence in them which they do not deserve. In vain do we seek freedom
under the power of the laws. The laws! Where is there any law? Where is
there any respect for law? Under the name of law you have everywhere
seen the rule of self-interest and human passion. But the eternal laws
of nature and of order exist. For the wise man they take the place of
positive law; they are written in the depths of his heart by conscience
and reason; let him obey these laws and be free; for there is no slave
but the evil-doer, for he always does evil against his will. Liberty is
not to be found in any form of government, she is in the heart of the
free man, he bears her with him everywhere. The vile man bears his
slavery in himself; the one would be a slave in Geneva, the other free
in Paris.

"If I spoke to you of the duties of a citizen, you would perhaps ask me,
`Which is my country?' And you would think you had put me to confusion.
Yet you would be mistaken, dear Emile, for he who has no country has, at
least, the land in which he lives. There is always a government and
certain so-called laws under which he has lived in peace. What matter
though the social contract has not been observed, if he has been
protected by private interest against the general will, if he has been
secured by public violence against private aggressions, if the evil he
has beheld has taught him to love the good, and if our institutions
themselves have made him perceive and hate their own iniquities? Oh,
Emile, where is the man who owes nothing to the land in which he lives?
Whatever that land may be, he owes to it the most precious thing
possessed by man, the morality of his actions and the love of virtue.
Born in the depths of a forest he would have lived in greater happiness
and freedom; but being able to follow his inclinations without a
struggle there would have been no merit in his goodness, he would not
have been virtuous, as he may be now, in spite of his passions. The mere
sight of order teaches him to know and love it. The public good, which
to others is a mere pretext, is a real motive for him. He learns to
fight against himself and to prevail, to sacrifice his own interest to
the common weal. It is not true that he gains nothing from the laws;
they give him courage to be just, even in the midst of the wicked. It is
not true that they have failed to make him free; they have taught him to
rule himself.

"Do not say therefore, `What matter where I am?' It does matter that you
should be where you can best do your duty; and one of these duties is to
love your native land. Your fellow-countrymen protected you in
childhood; you should love them in your manhood. You should live among
them, or at least you should live where you can serve them to the best
of your power, and where they know where to find you if ever they are in
need of you. There are circumstances in which a man may be of more use
to his fellow-countrymen outside his country than within it. Then he
should listen only to his own zeal and should bear his exile without a
murmur; that exile is one of his duties. But you, dear Emile, you have
not undertaken the painful task of telling men the truth, you must live
in the midst of your fellow-creatures, cultivating their friendship in
pleasant intercourse; you must be their benefactor, their pattern; your
example will do more than all our books, and the good they see you do
will touch them more deeply than all our empty words.

"Yet I do not exhort you to live in a town; on the contrary, one of the
examples which the good should give to others is that of a patriarchal,
rural life, the earliest life of man, the most peaceful, the most
natural, and the most attractive to the uncorrupted heart. Happy is the
land, my young friend, where one need not seek peace in the wilderness!
But where is that country? A man of good will finds it hard to satisfy
his inclinations in the midst of towns, where he can find few but frauds
and rogues to work for. The welcome given by the towns to those idlers
who flock to them to seek their fortunes only completes the ruin of the
country, when the country ought really to be repopulated at the cost of
the towns. All the men who withdraw from high society are useful just
because of their withdrawal, since its vices are the result of its
numbers. They are also useful when they can bring with them into the
desert places life, culture, and the love of their first condition. I
like to think what benefits Emile and Sophy, in their simple home, may
spread about them, what a stimulus they may give to the country, how
they may revive the zeal of the unlucky villagers.

"In fancy I see the population increasing, the land coming under
cultivation, the earth clothed with fresh beauty. Many workers and
plenteous crops transform the labours of the fields into holidays; I see
the young couple in the midst of the rustic sports which they have
revived, and I hear the shouts of joy and the blessings of those about
them. Men say the golden age is a fable; it always will be for those
whose feelings and taste are depraved. People do not really regret the
golden age, for they do nothing to restore it. What is needed for its
restoration? One thing only, and that is an impossibility; we must love
the golden age.

``Already it seems to be reviving around Sophy's home; together you will
only complete what her worthy parents have begun. But, dear Emile, you
must not let so pleasant a life give you a distaste for sterner duties,
if every they are laid upon you; remember that the Romans sometimes left
the plough to become consul. If the prince or the state calls you to the
service of your country, leave all to fulfil the honourable duties of a
citizen in the post assigned to you. If you find that duty onerous,
there is a sure and honourable means of escaping from it; do your duty
so honestly that it will not long be left in your hands. Moreover, you
need not fear the difficulties of such a test; while there are men of
our own time, they will not summon you to serve the state.''

Why may I not paint the return of Emile to Sophy and the end of their
love, or rather the beginning of their wedded love! A love founded on
esteem which will last with life itself, on virtues which will not fade
with fading beauty, on fitness of character which gives a charm to
intercourse, and prolongs to old age the delights of early love. But all
such details would be pleasing but not useful, and so far I have not
permitted myself to give attractive details unless I thought they would
be useful. Shall I abandon this rule when my task is nearly ended? No, I
feel that my pen is weary. Too feeble for such prolonged labours, I
should abandon this if it were not so nearly completed; if it is not to
be left imperfect it is time it were finished.

At last I see the happy day approaching, the happiest day of Emile's
life and my own; I see the crown of my labours, I begin to appreciate
their results. The noble pair are united till death do part; heart and
lips confirm no empty vows; they are man and wife. When they return from
the church, they follow where they are led; they know not where they
are, whither they are going, or what is happening around them. They heed
nothing, they answer at random; their eyes are troubled and they see
nothing. Oh, rapture! Oh, human weakness! Man is overwhelmed by the
feeling of happiness, he is not strong enough to bear it.

There are few people who know how to talk to the newly-married couple.
The gloomy propriety of some and the light conversation of others seem
to me equally out of place. I would rather their young hearts were left
to themselves, to abandon themselves to an agitation which is not
without its charm, rather than that they should be so cruelly distressed
by a false modesty, or annoyed by coarse witticisms which, even if they
appealed to them at other times, are surely out of place on such a day.

I behold our young people, wrapped in a pleasant languor, giving no heed
to what is said. Shall I, who desire that they should enjoy all the days
of their life, shall I let them lose this precious day? No, I desire
that they shall taste its pleasures and enjoy them. I rescue them from
the foolish crowd, and walk with them in some quiet place; I recall them
to themselves by speaking of them I wish to speak, not merely to their
ears, but to their hearts, and I know that there is only one subject of
which they can think to-day.

``My children,'' say I, taking a hand of each, ``it is three years since
I beheld the birth of the pure and vigorous passion which is your
happiness to-day. It has gone on growing; your eyes tell me that it has
reached its highest point; it must inevitably decline.'' My readers can
fancy the raptures, the anger, the vows of Emile, and the scornful air
with which Sophy withdraws her hand from mine; how their eyes protest
that they will adore each other till their latest breath. I let them
have their way; then I continue:

``I have often thought that if the happiness of love could continue in
marriage, we should find a Paradise upon earth. So far this has never
been. But if it were not quite impossible, you two are quite worthy to
set an example you have not received, an example which few married
couples could follow. My children, shall I tell you what I think is the
way, and the only way, to do it?''

They look at one another and smile at my simplicity. Emile thanks me
curtly for my prescription, saying that he thinks Sophy has a better, at
any rate it is good enough for him. Sophy agrees with him and seems just
as certain. Yet in spite of her mockery, I think I see a trace of
curiosity. I study Emile; his eager eyes are fixed upon his wife's
beauty; he has no curiosity for anything else; and he pays little heed
to what I say. It is my turn to smile, and I say to myself, ``I will
soon get your attention.''

The almost imperceptible difference between these two hidden impulses is
characteristic of a real difference between the two sexes; it is that
men are generally less constant than women, and are sooner weary of
success in love. A woman foresees man's future inconstancy, and is
anxious; it is this which makes her more jealous. {[}Footnote: In France
it is the wives who first emancipate themselves; and necessarily so, for
having very little heart, and only desiring attention, when a husband
ceases to pay them attention they care very little for himself. In other
countries it is not so; it is the husband who first emancipates himself;
and necessarily so, for women, faithful, but foolish, importune men with
their desires and only disgust them. There may be plenty of exceptions
to these general truths; but I still think they are truths.{]} When his
passion begins to cool she is compelled to pay him the attentions he
used to bestow on her for her pleasure; she weeps, it is her turn to
humiliate herself, and she is rarely successful. Affection and kind
deeds rarely win hearts, and they hardly ever win them back. I return to
my prescription against the cooling of love in marriage.

``It is plain and simple,'' I continue. ``It consists in remaining
lovers when you are husband and wife.''

``Indeed,'' said Emile, laughing at my secret, ``we shall not find that
hard.''

"Perhaps you will find it harder than you think. Pray give me time to
explain.

"Cords too tightly stretched are soon broken. This is what happens when
the marriage bond is subjected to too great a strain. The fidelity
imposed by it upon husband and wife is the most sacred of all rights;
but it gives to each too great a power over the other. Constraint and
love do not agree together, and pleasure is not to be had for the
asking. Do not blush, Sophy, and do not try to run away. God forbid that
I should offend your modesty! But your fate for life is at stake. For so
great a cause, permit a conversation between your husband and your
father which you would not permit elsewhere.

"It is not so much possession as mastery of which people tire, and
affection is often more prolonged with regard to a mistress than a wife.
How can people make a duty of the tenderest caresses, and a right of the
sweetest pledges of love? It is mutual desire which gives the right, and
nature knows no other. The law may restrict this right, it cannot extend
it. The pleasure is so sweet in itself! Should it owe to sad constraint
the power which it cannot gain from its own charms? No, my children, in
marriage the hearts are bound, but the bodies are not enslaved. You owe
one another fidelity, but not complaisance. Neither of you may give
yourself to another, but neither of you belongs to the other except at
your own will.

``If it is true, dear Emile, that you would always be your wife's lover,
that she should always be your mistress and her own, be a happy but
respectful lover; obtain all from love and nothing from duty, and let
the slightest favours never be of right but of grace. I know that
modesty shuns formal confessions and requires to be overcome; but with
delicacy and true love, will the lover ever be mistaken as to the real
will? Will not he know when heart and eyes grant what the lips refuse?
Let both for ever be master of their person and their caresses, let them
have the right to bestow them only at their own will. Remember that even
in marriage this pleasure is only lawful when the desire is mutual. Do
not be afraid, my children, that this law will keep you apart; on the
contrary, it will make both more eager to please, and will prevent
satiety. True to one another, nature and love will draw you to each
other.''

Emile is angry and cries out against these and similar suggestions.
Sophy is ashamed, she hides her face behind her fan and says nothing.
Perhaps while she is saying nothing, she is the most annoyed. Yet I
insist, without mercy; I make Emile blush for his lack of delicacy; I
undertake to be surety for Sophy that she will undertake her share of
the treaty. I incite her to speak, you may guess she will not dare to
say I am mistaken. Emile anxiously consults the eyes of his young wife;
he beholds them, through all her confusion, filled with a, voluptuous
anxiety which reassures him against the dangers of trusting her. He
flings himself at her feet, kisses with rapture the hand extended to
him, and swears that beyond the fidelity he has already promised, he
will renounce all other rights over her. ``My dear wife,'' said he, ``be
the arbiter of my pleasures as you are already the arbiter of my life
and fate. Should your cruelty cost me life itself I would yield to you
my most cherished rights. I will owe nothing to your complaisance, but
all to your heart.''

Dear Emile, be comforted; Sophy herself is too generous to let you fall
a victim to your generosity.

In the evening, when I am about to leave them, I say in the most solemn
tone, ``Remember both of you, that you are free, that there is no
question of marital rights; believe me, no false deference. Emile will
you come home with me? Sophy permits it.'' Emile is ready to strike me
in his anger. ``And you, Sophy, what do you say? Shall I take him
away?'' The little liar, blushing, answers, ``Yes.'' A tender and
delightful falsehood, better than truth itself!

The next day. \ldots{} Men no longer delight in the picture of bliss;
their taste is as much depraved by the corruption of vice as their
hearts. They can no longer feel what is touching or perceive what is
truly delightful. You who, as a picture of voluptuous joys, see only the
happy lovers immersed in pleasure, your picture is very imperfect; you
have only its grosser part, the sweetest charms of pleasure are not
there. Which of you has seen a young couple, happily married, on the
morrow of their marriage? their chaste yet languid looks betray the
intoxication of the bliss they have enjoyed, the blessed security of
innocence, and the delightful certainty that they will spend the rest of
their life together. The heart of man can behold no more rapturous
sight; this is the real picture of happiness; you have beheld it a
hundred times without heeding it; your hearts are so hard that you
cannot love it. Sophy, peaceful and happy, spends the day in the arms of
her tender mother; a pleasant resting place, after a night spent in the
arms of her husband.

The day after I am aware of a slight change. Emile tries to look
somewhat vexed; but through this pretence I notice such a tender
eagerness, and indeed so much submission, that I do not think there is
much amiss. As for Sophy she is merrier than she was yesterday; her eyes
are sparkling and she looks very well pleased with herself; she is
charming to Emile; she ventures to tease him a little and vexes him
still more.

These changes are almost imperceptible, but they do not escape me; I am
anxious and I question Emile in private, and I learn that, to his great
regret, and in spite of all entreaties, he was not permitted last night
to share Sophy's bed. That haughty lady had made haste to assert her
right. An explanation takes place. Emile complains bitterly, Sophy
laughs; but at last, seeing that Emile is really getting angry, she
looks at him with eyes full of tenderness and love, and pressing my
hand, she only says these two words, but in a tone that goes to his
heart, ``Ungrateful man!'' Emile is too stupid to understand. But I
understand, and I send Emile away and speak to Sophy privately in her
turn.

``I see,'' said I, "the reason for this whim. No one could be more
delicate, and no one could use that delicacy so ill. Dear Sophy, do not
be anxious, I have given you a man; do not be afraid to treat him as
such. You have had the first fruits of his youth; he has not squandered
his manhood and it will endure for you. My dear child, I must explain to
you why I said what I did in our conversation of the day before
yesterday. Perhaps you only understood it as a way of restraining your
pleasures to secure their continuance. Oh, Sophy, there was another
object, more worthy of my care. When Emile became your husband, he
became your head, it is yours to obey; this is the will of nature. When
the wife is like Sophy, it is, however, good for the man to be led by
her; that is another of nature's laws, and it is to give you as much
authority over his heart, as his sex gives him over your person, that I
have made you the arbiter of his pleasures. It will be hard for you, but
you will control him if you can control yourself, and what has already
happened shows me that this difficult art is not beyond your courage.
You will long rule him by love if you make your favours scarce and
precious, if you know how to use them aright. If you want to have your
husband always in your power, keep him at a distance. But let your
sternness be the result of modesty not caprice; let him find you modest
not capricious; beware lest in controlling his love you make him doubt
your own. Be all the dearer for your favours and all the more respected
when you refuse them; let him honour his wife's chastity, without having
to complain of her coldness.

"Thus, my child, he will give you his confidence, he will listen to your
opinion, will consult you in his business, and will decide nothing
without you. Thus you may recall him to wisdom, if he strays, and bring
him back by a gentle persuasion, you may make yourself lovable in order
to be useful, you may employ coquetry on behalf of virtue, and love on
behalf of reason.

"Do not think that with all this, your art will always serve your
purpose. In spite of every precaution pleasures are destroyed by
possession, and love above all others. But when love has lasted long
enough, a gentle habit takes its place and the charm of confidence
succeeds the raptures of passion. Children form a bond between their
parents, a bond no less tender and a bond which is sometimes stronger
than love itself. When you cease to be Emile's mistress you will be his
friend and wife; you will be the mother of his children. Then instead of
your first reticence let there be the fullest intimacy between you; no
more separate beds, no more refusals, no more caprices. Become so truly
his better half that he can no longer do without you, and if he must
leave you, let him feel that he is far from himself. You have made the
charms of home life so powerful in your father's home, let them prevail
in your own. Every man who is happy at home loves his wife. Remember
that if your husband is happy in his home, you will be a happy wife.

``For the present, do not be too hard on your lover; he deserves more
consideration; he will be offended by your fears; do not care for his
health at the cost of his happiness, and enjoy your own happiness. You
must neither wait for disgust nor repulse desire; you must not refuse
for the sake of refusing, but only to add to the value of your
favours.''

Then, taking her back to Emile, I say to her young husband, ``One must
bear the yoke voluntarily imposed upon oneself. Let your deserts be such
that the yoke may be lightened. Above all, sacrifice to the graces, and
do not think that sulkiness will make you more amiable.'' Peace is soon
made, and everybody can guess its terms. The treaty is signed with a
kiss, after which I say to my pupil, ``Dear Emile, all his life through
a man needs a guide and counsellor. So far I have done my best to fulfil
that duty; my lengthy task is now ended, and another will undertake this
duty. To-day I abdicate the authority which you gave me; henceforward
Sophy is your guardian.''

Little by little the first raptures subside and they can peacefully
enjoy the delights of their new condition. Happy lovers, worthy husband
and wife! To do honour to their virtues, to paint their felicity, would
require the history of their lives. How often does my heart throb with
rapture when I behold in them the crown of my life's work! How often do
I take their hands in mine blessing God with all my heart! How often do
I kiss their clasped hands! How often do their tears of joy fall upon
mine! They are touched by my joy and they share my raptures. Their
worthy parents see their own youth renewed in that of their children;
they begin to live, as it were, afresh in them; or rather they perceive,
for the first time, the true value of life; they curse their former
wealth, which prevented them from enjoying so delightful a lot when they
were young. If there is such a thing as happiness upon earth, you must
seek it in our abode.

One morning a few months later Emile enters my room and embraces me,
saying, ``My master, congratulate your son; he hopes soon to have the
honour of being a father. What a responsibility will be ours, how much
we shall need you! Yet God forbid that I should let you educate the son
as you educated the father. God forbid that so sweet and holy a task
should be fulfilled by any but myself, even though I should make as good
a choice for my child as was made for me! But continue to be the teacher
of the young teachers. Advise and control us; we shall be easily led; as
long as I live I shall need you. I need you more than ever now that I am
taking up the duties of manhood. You have done your own duty; teach me
to follow your example, while you enjoy your well-earned leisure.''

\section*{THE END}



\end{document}